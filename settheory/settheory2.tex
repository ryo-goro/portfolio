\documentclass[leqno]{jsarticle}
\usepackage{amsmath,amssymb}
\usepackage{bm}
\usepackage{latexsym}
\usepackage{mathrsfs}
\usepackage{okumacro}
\usepackage{amsthm}
\usepackage{comment}
\newcommand{\halmos}{\rule[-0.5mm]{2mm}{3.5mm}}
\numberwithin{equation}{section}




\theoremstyle{definition}
\newtheorem{defi}{定義}
\newtheorem{defo}{変形法則}
\newtheorem{valu}{変数法則}
\newtheorem{gsub}{一般代入法則}
\newtheorem{subs}{代入法則}
\newtheorem{form}{構成法則}
\newtheorem{skel}{骨格式構成法則}
\newtheorem{dedu}{推論法則}
\newtheorem{theo}{Thm}
\newtheorem{thm}{定理}[section]
\newtheorem{lemma}{補題}[section]




\begin{document}




\part{記号論理}




\section{定義}




\mathstrut
\begin{defi}
\label{defprime}%定義1
$A$を$a_1a_2 \cdots a_n$という記号列とする.
但しここで$n$は自然数で, 各$a_i$ ($i$は$1, 2, \cdots, n$のいずれか) は一つの記号とする.
次の操作によって得られる記号$\alpha_1, \alpha_2, \cdots , \alpha_n$を
並べて得られる記号列$\alpha_1\alpha_2\cdots\alpha_n$を, 
$[A]'$と書き表す: 
各$i$ ($i$は$1, 2, \cdots , n$のいずれか) に対し, 

a)
$a_i$が文字でなく, prime文字でもなければ, $\alpha_i$は$a_i$とする.

b)
$a_i$が文字ならば, $\alpha_i$は$a_i'$ (即ち文字$a_i$の右上に$'$を一つつけて得られるprime文字) とする.

c)
$a_i$がprime文字ならば, 文字$x$があり, $a_i$は$x$の右上にいくつかの$'$ がついた記号である.
このとき$\alpha_i$はこのprime文字よりも$'$ の個数が一つだけ多い$x$のprime文字とする 
(例えば$a_i$が$x''$ならば, $\alpha_i$は$x'''$である).
\end{defi}




\mathstrut
\begin{defi}
\label{defsubsubst}%定義2
$A$を$a_1a_2 \cdots a_n$という記号列とする.
但しここで$n$は自然数で, 各$a_i$ ($i$は$1, 2, \cdots, n$のいずれか) は一つの記号とする.
また$x$を文字とし, $y$を文字あるいは$\Box$とする.
次の操作によって得られる記号$\alpha_1, \alpha_2, \cdots, \alpha_n$
を並べて得られる記号列$\alpha_1\alpha_2\cdots\alpha_n$を, $\{y|x\}(A)$と書き表す: 
各$i$ ($i$は$1, 2, \cdots, n$のいずれか) に対し, 

a)
$a_i$が$x$でなく, $x$のprime文字でもなければ, $\alpha_i$は$a_i$とする.

b)
$a_i$が$x$ならば, $\alpha_i$は$y$とする.

c)
$a_i$が$x$のprime文字ならば, 上記の略記法によれば, 
自然数$m$があり, $a_i$は$x^{(m)}$である.
このときは$\alpha_i$は$y^{(m)}$とする.
\end{defi}




\mathstrut
\begin{defi}
\label{defsubst}%定義3
$A$を$a_1a_2 \cdots a_n$という記号列とする.
但しここで$n$は自然数で, 各$a_i$ ($i$は$1, 2, \cdots, n$のいずれか) は一つの記号とする.
また$x_1, x_2, \cdots, x_m$を, どの二つも互いに異なる文字とする.
但しここで$m$は自然数である.
更に, $B_1, B_2, \cdots, B_m$を記号列とする.
次の操作によって得られる記号列$A_1, A_2, \cdots, A_n$を並べて得られる記号列
$A_1A_2 \cdots A_n$を, $(B_1|x_1, B_2|x_2, \cdots, B_m|x_m)(A)$と書き表す: 
各$i$ ($i$は$1, 2, \cdots, n$のいずれか) に対し, 

a)
$a_i$が$x_1, x_2, \cdots, x_m$のいずれでもなく, これらのprime文字でもなければ, $A_i$は$a_i$ 
(即ち$A_i$は唯一つの記号$a_i$から成る記号列) とする.

b)
$a_i$が$x_j$ ($j$は$1, 2, \cdots, m$のいずれか) ならば, $A_i$は$B_j$とする.

c)
$a_i$が$x_j$ ($j$は$1, 2, \cdots, m$のいずれか) のprime文字ならば, 上記の略記法によれば, 
自然数$l$があり, $a_i$は$x_j^{(l)}$である.
このときは$A_i$は$[B_j]^{(l)}$とする (これも上記の略記法).
\end{defi}




\mathstrut
\begin{defi}
\label{defappear}%定義4
$A$を$a_1a_2 \cdots a_n$という記号列とする.
但しここで$n$は自然数で, 
各$a_i$ ($i$は$1, 2, \cdots, n$のいずれか) は一つの記号とする.
また$a$を一つの記号とする.
$a$が$A$の{\bf 先頭の記号である}とは, 
$a$が$a_1$と同一の記号であることをいう.
このとき$A$は{\bf $\bm{a}$から始まる}ともいう.
また{\bf $\bm{a}$が$\bm{A}$の中に現れる}とは, 
$a$が$a_1, a_2, \cdots, a_n$のいずれかの記号と同一のものであることをいう.
\end{defi}




\mathstrut
\begin{defi}
\label{deffv}%定義5
$A$を記号列とし, $x$を文字とする.
{\bf $\bm{x}$が$\bm{A}$の中に自由変数として現れる}とは, 
$x$または$x$のprime文字が$A$の中に現れることをいう.
これを{\bf $\bm{x}$は$\bm{A}$の自由変数である}, 
{\bf $\bm{A}$は$\bm{x}$を自由変数として持つ}などともいう.
よって{\bf $\bm{x}$が$\bm{A}$の中に自由変数として現れない}とは, 
$x$も$x$のどのprime文字も$A$の中に現れないことである.
どの文字も$A$の中に自由変数として現れないとき, {\bf $\bm{A}$は自由変数を持たない}という.
これは$A$の中に文字及びprime文字が現れないことと同じことである.
\end{defi}




\mathstrut
\begin{defi}
\label{defvee}%定義6
記号列$\to\! \neg$を$\vee$とも書き表す.
$A$と$B$を記号列とするとき, 記号列$\vee AB$を
$A$と$B$の{\bf 論理和}という.
\end{defi}




\mathstrut
\begin{defi}
\label{deftau}%定義7
$A$を記号列とし, $x$を文字とするとき, 
記号列$\tau [\{\Box|x\}(A)]'$を$\tau_x(A)$と書き表す.
\end{defi}




\mathstrut
\begin{defi}
\label{def12}%定義8
$A$を記号列とする.
$A$が文字であるか, あるいは$\tau$から始まる記号列であるとき, 
$A$は{\bf 第一種}であるという.
$A$が第一種でないとき (即ち$A$が文字でなく, また$\tau$から始まってもいないとき), 
$A$は{\bf 第二種}であるという.
\end{defi}




\mathstrut
\begin{defi}
\label{defformproc}%定義9
理論$\mathscr{T}$の記号列の列が次の性質を持つとき, その列のことを
$\mathscr{T}$における{\bf 構成手続き}という: 

その列の中に現れる各記号列$A$に対して, 以下の条件a) - e)のいずれかが成立する: 

a)
$A$は文字である.

b)
その列の中に (その列の中でいま指定した位置の) $A$よりも前に
第二種の記号列$B$が現れており, $A$は$\neg B$である.

c)
その列の中に (その列の中でいま指定した位置の) $A$よりも前に
第二種の記号列$B$及び$C$ (これらは必ずしも相異ならない) が現れており, $A$は$\to\! BC$である.

d)
その列の中に (その列の中でいま指定した位置の) $A$よりも前に
文字$x$及び第二種の記号列$B$が現れており, $A$は$\tau_x(B)$である.

e)
$\mathscr{T}$の特殊記号$s$があり, またその列の中に (その列の中でいま指定した位置の) $A$よりも前に
第一種の記号列$B$及び$C$ (これらは必ずしも相異ならない) が現れており, $A$は$sBC$である.

\mathstrut
記号列$A$が理論$\mathscr{T}$における或る構成手続きに現れる最後の記号列となるとき, 
$A$を$\mathscr{T}$の{\bf 論理式}という.
またこのとき, その構成手続きを$\mathscr{T}$における{\bf $\bm{A}$の構成手続き}という.

理論$\mathscr{T}$の論理式のうち, 第一種のものを$\mathscr{T}$の{\bf 対象式}といい, 
第二種のものを$\mathscr{T}$の{\bf 関係式}という.
\end{defi}




\mathstrut
\begin{defi}
\label{defsk12}%定義10
$A$を記号列とする.
$A$が文字であるか, prime文字であるか, $\Box$型記号であるか, 
あるいは$\tau$から始まる記号列であるとき, 
$A$は{\bf sk-第一種}であるという.
$A$がsk-第一種でないとき (即ち$A$が文字でなく, prime文字でもなく, 
$\Box$型記号でもなく, また$\tau$から始まってもいないとき), 
$A$は{\bf sk-第二種}であるという.
\end{defi}




\mathstrut
\begin{defi}
\label{defskformproc}%定義11
理論$\mathscr{T}$の記号列の列が次の性質を持つとき, その列のことを
$\mathscr{T}$における{\bf sk-構成手続き}という: 

その列の中に現れる各記号列$A$に対して, 以下の条件a) - e)のいずれかが成立する: 

a)
$A$は文字であるか, prime文字であるか, $\Box$型記号である.

b)
その列の中に (その列の中でいま指定した位置の) $A$よりも前に
sk-第二種の記号列$B$が現れており, $A$は$\neg B$である.

c)
その列の中に (その列の中でいま指定した位置の) $A$よりも前に
sk-第二種の記号列$B$及び$C$ (これらは必ずしも相異ならない) が現れており, $A$は$\to\! BC$である.

d)
その列の中に (その列の中でいま指定した位置の) $A$よりも前に
sk-第二種の記号列$B$が現れており, $A$は$\tau B$である.

e)
$\mathscr{T}$の特殊記号$s$があり, またその列の中に (その列の中でいま指定した位置の) $A$よりも前に
sk-第一種の記号列$B$及び$C$ (これらは必ずしも相異ならない) が現れており, $A$は$sBC$である.

\mathstrut
記号列$A$が理論$\mathscr{T}$における或るsk-構成手続きに現れる最後の記号列となるとき, 
$A$を$\mathscr{T}$の{\bf 骨格式}という.
またこのとき, そのsk-構成手続きを$\mathscr{T}$における{\bf $\bm{A}$のsk-構成手続き}という.

理論$\mathscr{T}$の骨格式のうち, sk-第一種のものを$\mathscr{T}$の{\bf sk-対象式}といい, 
sk-第二種のものを$\mathscr{T}$の{\bf sk-関係式}という.
\end{defi}




\mathstrut
\begin{defi}
\label{defaxiom}%定義12
$\mathscr{T}$を理論とする.

$\mathscr{T}$の関係式をいくつか指定し, それらを$\mathscr{T}$の{\bf 明示的公理}という.
$\mathscr{T}$の明示的公理のいずれかの中に自由変数として現れる文字をすべて, 
$\mathscr{T}$の{\bf 定数}という.

$\mathscr{T}$のいくつかの対象式, 関係式や文字から一つの記号列を与える規則であって, 
次のような特性a), b)を持つものをいくつか指定し, それらを$\mathscr{T}$の{\bf schema}という: 

a)
そのような規則の任意の一つ$\mathscr{R}$を適用して得られる記号列は$\mathscr{T}$の関係式である.

b)
$T$が$\mathscr{T}$の対象式, $x$が文字, $R$がそのような規則の一つ$\mathscr{R}$を
適用して得られる記号列であれば, 記号列$(T|x)(R)$は
$\mathscr{R}$を直接適用することによっても得られる.

$\mathscr{T}$のschemaの適用によって得られる
すべての記号列 (それらは特性a)によって$\mathscr{T}$の関係式である) を, 
$\mathscr{T}$の{\bf 非明示的公理}という.
$\mathscr{T}$の明示的公理と非明示的公理とを合わせて, 
$\mathscr{T}$の{\bf 公理}という.
\end{defi}




\mathstrut
\begin{defi}
\label{defproof}%定義13
$\mathscr{T}$を理論とする.
次の性質を持つ記号列の列のことを, $\mathscr{T}$における{\bf 証明}という: 

その列に属する各記号列$A$に対して, 次の条件a), b)のどちらかが成立する: 

a)
$A$は$\mathscr{T}$の公理である (即ち$A$は$\mathscr{T}$の明示的公理であるか, 
$\mathscr{T}$のあるschemaの適用によって得られる$\mathscr{T}$の非明示的公理である).

b)
その列の中に (その列の中でいま指定した位置の) $A$よりも前に記号列$B$及び$C$が現れており, 
$C$は$B \to A$である (このとき$A$は$B$と$C$ (あるいは$C$と$B$) からの{\bf 直接の帰結}であるという).

\mathstrut
記号列$A$が理論$\mathscr{T}$における或る証明に現れる最後の記号列となるとき, 
$A$を$\mathscr{T}$の{\bf 定理}という.
またこのとき, その証明を$\mathscr{T}$における{\bf $\bm{A}$の証明}という.
\end{defi}




\mathstrut
\begin{defi}
\label{defcontra}%定義14
$\mathscr{T}$を理論とする.
$\mathscr{T}$において真であると同時に偽でもあるような記号列$A$が書き表せるとき, 
$\mathscr{T}$は{\bf 矛盾する}という.
\end{defi}




\mathstrut
\begin{defi}
\label{defgsubsttheory}%定義15
$\mathscr{T}$を一つの理論とし, $A_1, A_2, \cdots, A_m$をその明示的公理の全体とする ($m$は自然数).
また$n$を自然数とし, $T_1, T_2, \cdots, T_n$を$\mathscr{T}$の対象式, 
$x_1, x_2, \cdots, x_n$をどの二つも互いに異なる文字とする.
いま$\mathscr{T}$と同じ特殊記号及びschemaを持つ理論であって, その明示的公理の全体が
\[
  (T_1|x_1, T_2|x_2, \cdots, T_n|x_n)(A_1), 
  (T_1|x_1, T_2|x_2, \cdots, T_n|x_n)(A_2), 
  \cdots, 
  (T_1|x_1, T_2|x_2, \cdots, T_n|x_n)(A_m)
\]
であるようなものを, 
$(T_1|x_1, T_2|x_2, \cdots, T_n|x_n)(\mathscr{T})$と書き表す.
\end{defi}




\mathstrut
\begin{defi}
\label{deftheorystrength}%定義16
理論$\mathscr{T}$のすべての特殊記号が理論$\mathscr{T}'$の特殊記号であり, 
$\mathscr{T}$のすべての明示的公理が$\mathscr{T}'$の定理であり, 
$\mathscr{T}$のすべてのschemaが$\mathscr{T}'$のschemaであるとき, 
$\mathscr{T}'$は$\mathscr{T}${\bf より強い}という.
\end{defi}




\mathstrut
\begin{defi}
\label{deftheoryequiv}%定義17
$\mathscr{T}$と$\mathscr{T}'$を理論とする.
$\mathscr{T}$が$\mathscr{T}'$より強く, 同時に$\mathscr{T}'$が$\mathscr{T}$より強いとき, 
$\mathscr{T}$と$\mathscr{T}'$は{\bf 同値}であるという.
\end{defi}




\mathstrut
\begin{defi}
\label{deflogicaltheory}%定義18
S1, S2, S3の規則をschemaとして持つ理論を{\bf 論理的な理論}という.
\end{defi}




\mathstrut
\begin{defi}
\label{defwedge}%定義19
$A$と$B$を記号列とする.
$\neg (\neg A \vee \neg B)$, 即ち$\neg (\neg \neg A \to \neg B)$という
記号列 (記号列の本来の書き方では, $\neg\! \to\! \neg \neg A \neg B$) を
$(A) \wedge (B)$と書き表し, これを$A$と$B$の{\bf 論理積}という.
括弧は特に必要がなければ省略する.
\end{defi}




\mathstrut
\begin{defi}
\label{defgveegwedge}%定義20
$n$を自然数とし, $A_{1}, A_{2}, A_{3}, \cdots, A_{n}$を記号列とする.

$A_{1}$を一番目とし, そこから$A_{1} \vee A_{2}, (A_{1} \vee A_{2}) \vee A_{3}, \cdots$のようにして
次々に構成されていく記号列の$n$番目$( \cdots ((A_{1} \vee A_{2}) \vee A_{3}) \vee \cdots ) \vee A_{n}$を, 
\[
\tag{$*$}
  (A_{1}) \vee (A_{2}) \vee (A_{3}) \vee \cdots \vee (A_{n})
\]
と書き表す.
但し括弧は適宜省略する.
この定義によれば, $n$が$2$のときの($*$)はいままでの書き方と矛盾しない.
また$n$が$1$のときの($*$)は$A_{1}$である.
また$n$が$2$以上のとき, 
\[
  A_{1} \vee A_{2} \vee \cdots \vee A_{n} 
  \equiv (A_{1} \vee A_{2} \vee \cdots \vee A_{n - 1}) \vee A_{n}
\]
が成り立つ.

上と同様に, $A_{1}$を一番目とし, そこから$A_{1} \wedge A_{2}, (A_{1} \wedge A_{2}) \wedge A_{3}, \cdots$のようにして
次々に構成されていく記号列の$n$番目$( \cdots ((A_{1} \wedge A_{2}) \wedge A_{3}) \wedge \cdots ) \wedge A_{n}$を, 
\[
\tag{$**$}
  (A_{1}) \wedge (A_{2}) \wedge (A_{3}) \wedge \cdots \wedge (A_{n})
\]
と書き表す.
但し括弧は適宜省略する.
この定義によれば, $n$が$2$のときの($**$)はいままでの書き方と矛盾しない.
また$n$が$1$のときの($**$)は$A_{1}$である.
また$n$が$2$以上のとき, 
\[
  A_{1} \wedge A_{2} \wedge \cdots \wedge A_{n} 
  \equiv (A_{1} \wedge A_{2} \wedge \cdots \wedge A_{n - 1}) \wedge A_{n}
\]
が成り立つ.
\end{defi}




\mathstrut
\begin{defi}
\label{defequiv}%定義21
$A$と$B$を記号列とする.
$(A \to B) \wedge (B \to A)$, 即ち$\neg (\neg (A \to B) \vee \neg (B \to A))$, 
即ち$\neg (\neg \neg (A \to B) \to \neg (B \to A))$という記号列 (記号列の本来の書き方では, 
$\neg\! \to\! \neg \neg\! \to\! AB \neg\! \to\! BA$) を$(A) \leftrightarrow (B)$と書き表す.
括弧は特に必要がなければ省略する.
\end{defi}




\mathstrut
\begin{defi}
\label{defquan}%定義22
$R$を記号列とし, $x$を文字とする.
記号列$(\tau_{x}(R)|x)(R)$を$\exists x(R)$と書き表し, これを
{\bf $\bm{R}$である ($\bm{R}$となる, $\bm{R}$を満たす) $\bm{x}$が存在する}と読む.
また$\neg \exists x(\neg R)$, 即ち$\neg (\tau_{x}(\neg R)|x)(\neg R)$という記号列を
$\forall x(R)$と書き表し, これを
{\bf すべての$\bm{x}$に対して$\bm{R}$である}, {\bf 任意の$\bm{x}$に対して$\bm{R}$である}, 
{\bf どんな$\bm{x}$に対しても$\bm{R}$である}などと読む.
省略記法のための記号$\exists$及び$\forall$をそれぞれ, 
{\bf 存在作用素 (存在の限定作用素)}, {\bf 全称作用素 (全称の限定作用素)} という.
またこれらをまとめて{\bf 限定作用素}という.
\end{defi}




\mathstrut
\begin{defi}
\label{defquantheory}%定義23
S1, S2, S3に加え, 上記の規則S4をもschemaとして持つ理論のことを, 
{\bf 限定作用素を持つ理論}という.
\end{defi}




\mathstrut
\begin{defi}
\label{defspquan}%定義24
$A$と$R$を記号列とし, $x$を文字とする.
記号列$\exists x(A \wedge R)$を$\exists_{A}x(R)$と書き表す.
また記号列$\neg \exists_{A}x(\neg R)$, 即ち$\neg \exists x(A \wedge \neg R)$を
$\forall_{A}x(R)$と書き表す.
略記号$\exists_{A}$及び$\forall_{A}$を{\bf 特殊限定作用素}という.
\end{defi}




\mathstrut
\begin{defi}
\label{defequal}%定義25
特殊記号として$=$を持ち, S1からS6までのすべての規則をschemaとして持つ理論のことを, 
{\bf 等号を持つ理論}という.
\end{defi}




\mathstrut
\begin{defi}
\label{def!}%定義26
$\mathscr{T}$は特殊記号として$=$を持つ理論とする.
$R$を$\mathscr{T}$の記号列とし, $x$を文字とする.
また$y$と$z$を共に$x$と異なり, $R$の中に自由変数として現れない文字とする.
このとき
$\forall x(\forall y(R \wedge (y|x)(R) \to x = y))$と
$\forall x(\forall z(R \wedge (z|x)(R) \to x = z))$という二つの記号列は一致する.
実際$u$と$v$を, 互いに異なり, 共に$x$と異なり, $R$の中に自由変数として現れない文字とすれば, 
変形法則 \ref{singlevalued}により, この二つの記号列は共に
$\forall u(\forall v((u|x)(R) \wedge (v|x)(R) \to u = v))$と一致するからである.
また$s$と$t$を, 互いに異なり, 共に$x$と異なり, $R$の中に自由変数として現れない文字とすれば, 
やはり変形法則 \ref{singlevalued}により, 
$\forall x(\forall y(R \wedge (y|x)(R) \to x = y))$は
$\forall s(\forall t((s|x)(R) \wedge (t|x)(R) \to s = t))$とも一致する.
従って
\begin{align*}
  \forall x(\forall y(R \wedge (y|x)(R) \to x = y))&, ~~
  \forall x(\forall z(R \wedge (z|x)(R) \to x = z)), \\
  \mbox{} \\
  \forall u(\forall v((u|x)(R) \wedge (v|x)(R) \to u = v))&, ~~
  \forall s(\forall t((s|x)(R) \wedge (t|x)(R) \to s = t))
\end{align*}
という四つの記号列はすべて同一の記号列である.
$R$と$x$に対して定まるこの記号列を$!x(R)$と書き表し, 
これを{\bf $\bm{R}$となる ($\bm{R}$を満たす) $\bm{x}$が高々一つ存在する}と読む.
略記号$!$を{\bf 一価作用素}という.
\end{defi}




\mathstrut
\begin{defi}
\label{defsp!}%定義27
$\mathscr{T}$を特殊記号として$=$を持つ理論とし, 
$A$と$R$を$\mathscr{T}$の記号列, $x$を文字とする.
記号列$!x(A \wedge R)$を$!_{A}x(R)$と記す.
略記号$!_{A}$を{\bf 特殊一価作用素}という.
\end{defi}




\mathstrut
\begin{defi}
\label{defex!}%定義28
$\mathscr{T}$を特殊記号として$=$を持つ理論とし, 
$R$を$\mathscr{T}$の記号列, $x$を文字とする.
記号列$\exists x(R) \, \wedge \, !x(R)$を$\exists !x(R)$と書き表し, 
これを{\bf $\bm{R}$となる ($\bm{R}$を満たす) $\bm{x}$が唯一つ存在する}と読む.
略記号$\exists !$を{\bf 唯一存在作用素}という.
\end{defi}




\mathstrut
\begin{defi}
\label{defspex!}%定義29
$\mathscr{T}$を特殊記号として$=$を持つ理論とし, 
$A$と$R$を$\mathscr{T}$の記号列, $x$を文字とする.
記号列$\exists !x(A \wedge R)$を$\exists !_{A}x(R)$と記す.
これは$\exists x(A \wedge R) \, \wedge \, !x(A \wedge R)$, 
即ち$\exists_{A}x(R) \, \wedge \, !_{A}x(R)$と同じである.
略記号$\exists !_{A}$を{\bf 特殊唯一存在作用素}という.
\end{defi}




\newpage




\section{変形法則}




\mathstrut
\begin{defo}
\label{primesep}%変形1
$A$と$B$を記号列とするとき, $[AB]' \equiv [A]'[B]'$が成り立つ.
\end{defo}




\mathstrut
\begin{defo}
\label{primefree}%変形2
$A$を記号列とする.
$A$が自由変数を持たなければ, $[A]' \equiv A$が成り立つ.
\end{defo}




\mathstrut
\begin{defo}
\label{subsubstfree}%変形3
$A$を記号列, $x$を文字とし, $y$を文字または$\Box$とする.
$x$が$A$の中に自由変数として現れなければ, 
\[
  \{y|x\}(A) \equiv A
\]
が成り立つ.
\end{defo}




\mathstrut
\begin{defo}
\label{subsubstboth}%変形4
$A$と$B$を記号列, $x$を文字とし, $y$を文字または$\Box$とする.
このとき
\[
  \{y|x\}(AB) \equiv \{y|x\}(A)\{y|x\}(B)
\]
が成り立つ.
\end{defo}




\mathstrut
\begin{defo}
\label{subsubsttrans}%変形5
$A$を記号列, $x$と$y$を文字とし, $z$を文字または$\Box$とする.
$y$が$A$の中に自由変数として現れなければ, 
\[
  \{z|y\}(\{y|x\}(A)) \equiv \{z|x\}(A)
\]
が成り立つ.
\end{defo}




\mathstrut
\begin{defo}
\label{subsubstsubst}%変形6
$A$を記号列とし, $x$と$y$を文字とする.
このとき
\[
  \{y|x\}(A) \equiv (y|x)(A)
\]
が成り立つ.
\end{defo}




\mathstrut
\begin{defo}
\label{quanfree}%変形7
$R$を記号列とし, $x$を文字とする.
$x$が$R$の中に自由変数として現れなければ, 
\[
  \exists x(R) \equiv R, ~~
  \forall x(R) \equiv \neg \neg R
\]
が成り立つ.
\end{defo}




\mathstrut
\begin{defo}
\label{spquanfree}%変形8
$A$と$R$を記号列とし, $x$を文字とする.
$x$が$A$の中にも$R$の中にも自由変数として現れなければ, 
\[
  \exists_{A}x(R) \equiv A \wedge R, ~~
  \forall_{A}x(R) \equiv \neg (A \wedge \neg R)
\]
が成り立つ.
\end{defo}




\mathstrut
\begin{defo}
\label{singlevalued}%変形9
$\mathscr{T}$は特殊記号として$=$を持つ理論とする.
$R$を$\mathscr{T}$の記号列とし, $x$を文字とする.
また$y$を$x$と異なり, $R$の中に自由変数として現れない文字とする.
また$z$と$w$を, 互いに異なり, 共に$x$と異なり, $R$の中に自由変数として現れない文字とする.
このとき
\[
  \forall x(\forall y(R \wedge (y|x)(R) \to x = y)) \equiv \forall z(\forall w((z|x)(R) \wedge (w|x)(R) \to z = w))
\]
が成り立つ.
\end{defo}




\newpage




\section{変数法則}




\mathstrut
\begin{valu}
\label{valboth}%変数1
$A$と$B$を記号列とし, $x$を文字とする.
$x$が$A$の中にも$B$の中にも自由変数として現れなければ, 
$x$は記号列$AB$の中にも自由変数として現れない.
\end{valu}




\mathstrut
\begin{valu}
\label{valfund}%変数2
$A$と$B$を記号列とし, $x$を文字, $s$を特殊記号とする.

1)
$x$が$A$の中に自由変数として現れなければ, 
$x$は$\neg A$の中にも自由変数として現れない.

2)
$x$が$A$の中にも$B$の中にも自由変数として現れなければ, 
$x$は$\to\! AB$, $\vee AB$, $sAB$のいずれの記号列の中にも
自由変数として現れない.
\end{valu}




\mathstrut
\begin{valu}
\label{valprime}%変数3
$A$を記号列とし, $x$を文字とする.
$x$が$A$の中に自由変数として現れなければ, 
$x$は$[A]'$の中にも自由変数として現れない.
\end{valu}




\mathstrut
\begin{valu}
\label{valsubsubst}%変数4
$A$を記号列, $x$を文字とし, $y$を文字あるいは$\Box$とする.

1)
$x$と$y$が異なれば (即ち$y$が$x$と異なる文字であるか, $y$が$\Box$であれば), 
$x$は$\{y|x\}(A)$の中に自由変数として現れない.

2)
$z$を文字とする.
$z$が$y$と異なり, かつ$A$の中に自由変数として現れなければ, 
$z$は$\{y|x\}(A)$の中にも自由変数として現れない.
\end{valu}




\mathstrut
\begin{valu}
\label{valgsubst}%変数5
$A$を記号列とする.
また$n$を自然数とし, $B_1, B_2, \cdots, B_n$を記号列とする.
また$x_1, x_2, \cdots, x_n$を, どの二つも互いに異なる文字とする.

1)
$i$を$1, 2, \cdots, n$のいずれかとする.
$x_i$が$B_1, B_2, \cdots, B_n$のいずれの中にも自由変数として現れなければ, 
$x_i$は$(B_1|x_1, B_2|x_2, \cdots, B_n|x_n)(A)$の中にも自由変数として現れない.

2)
$y$を文字とする.
$y$が$A, B_1, B_2, \cdots, B_n$のいずれの中にも自由変数として現れなければ, 
$y$は$(B_1|x_1, B_2|x_2, \cdots, B_n|x_n)(A)$の中にも自由変数として現れない.
\end{valu}




\mathstrut
\begin{valu}
\label{valsubst}%変数6
$A$と$B$を記号列とし, $x$を文字とする.

1)
$x$が$B$の中に自由変数として現れなければ, 
$x$は$(B|x)(A)$の中にも自由変数として現れない.

2)
$y$を文字とする.
$y$が$A$の中にも$B$の中にも自由変数として現れなければ, 
$y$は$(B|x)(A)$の中にも自由変数として現れない.
\end{valu}




\mathstrut
\begin{valu}
\label{valtau}%変数7
$A$を記号列とし, $x$を文字とする.

1)
$x$は$\tau_x(A)$の中に自由変数として現れない.

2)
$y$を文字とする.
$y$が$A$の中に自由変数として現れなければ, 
$y$は$\tau_x(A)$の中にも自由変数として現れない.
\end{valu}




\mathstrut
\begin{valu}
\label{valwedge}%変数8
$A$と$B$を記号列とし, $x$を文字とする.
$x$が$A$の中にも$B$の中にも自由変数として現れなければ, 
$x$は$A \wedge B$の中にも自由変数として現れない.
\end{valu}




\mathstrut
\begin{valu}
\label{valgvee}%変数9
$n$を自然数とする.
また$A_{1}, A_{2}, \cdots, A_{n}$を記号列とし, 
$x$をこれらの中に自由変数として現れない文字とする.
このとき$x$は$A_{1} \vee A_{2} \vee \cdots \vee A_{n}$の中に自由変数として現れない.
\end{valu}




\mathstrut
\begin{valu}
\label{valgwedge}%変数10
$n$を自然数とする.
また$A_{1}, A_{2}, \cdots, A_{n}$を記号列とし, 
$x$をこれらの中に自由変数として現れない文字とする.
このとき$x$は$A_{1} \wedge A_{2} \wedge \cdots \wedge A_{n}$の中に自由変数として現れない.
\end{valu}




\mathstrut
\begin{valu}
\label{valequiv}%変数11
$A$と$B$を記号列とし, $x$を文字とする.
$x$が$A$の中にも$B$の中にも自由変数として現れなければ, 
$x$は$A \leftrightarrow B$の中にも自由変数として現れない.
\end{valu}




\mathstrut
\begin{valu}
\label{valquan}%変数12
$R$を記号列とし, $x$を文字とする.

1)
$x$は$\exists x(R)$の中にも$\forall x(R)$の中にも自由変数として現れない.

2)
$y$を文字とする.
$y$が$R$の中に自由変数として現れなければ, 
$y$は$\exists x(R)$の中にも$\forall x(R)$の中にも自由変数として現れない.
\end{valu}




\mathstrut
\begin{valu}
\label{valspquan}%変数13
$A$と$R$を記号列とし, $x$を文字とする.

1)
$x$は$\exists_{A}x(R)$の中にも$\forall_{A}x(R)$の中にも自由変数として現れない.

2)
$y$を文字とする.
$y$が$A$の中にも$R$の中にも自由変数として現れなければ, 
$y$は$\exists_{A}x(R)$の中にも$\forall_{A}x(R)$の中にも自由変数として現れない.
\end{valu}




\mathstrut
\begin{valu}
\label{val!}%変数14
$R$を記号列とし, $x$を文字とする.

1)
$x$は$!x(R)$の中に自由変数として現れない.

2)
$y$を文字とする.
$y$が$R$の中に自由変数として現れなければ, 
$y$は$!x(R)$の中にも自由変数として現れない.
\end{valu}




\mathstrut
\begin{valu}
\label{valsp!}%変数15
$A$と$R$を記号列とし, $x$を文字とする.

1)
$x$は$!_{A}x(R)$の中に自由変数として現れない.

2)
$y$を文字とする.
$y$が$A$の中にも$R$の中にも自由変数として現れなければ, 
$y$は$!_{A}x(R)$の中にも自由変数として現れない.
\end{valu}




\mathstrut
\begin{valu}
\label{valex!}%変数16
$R$を記号列とし, $x$を文字とする.

1)
$x$は$\exists !x(R)$の中に自由変数として現れない.

2)
$y$を文字とする.
$y$が$R$の中に自由変数として現れなければ, 
$y$は$\exists !x(R)$の中にも自由変数として現れない.
\end{valu}




\mathstrut
\begin{valu}
\label{valspex!}%変数17
$A$と$R$を記号列とし, $x$を文字とする.

1)
$x$は$\exists !_{A}x(R)$の中に自由変数として現れない.

2)
$y$を文字とする.
$y$が$A$の中にも$R$の中にも自由変数として現れなければ, 
$y$は$\exists !_{A}x(R)$の中にも自由変数として現れない.
\end{valu}




\newpage




\section{一般代入法則}




\mathstrut
\begin{gsub}
\label{gsubstprime}%一般代入1
$A$を記号列とする.
また$n$を自然数とし, $B_1, B_2, \cdots, B_n$を記号列とする.
また$x_1, x_2, \cdots, x_n$を, どの二つも互いに異なる文字とする.
このとき
\[
  (B_1|x_1, B_2|x_2, \cdots, B_n|x_n)([A]') \equiv [(B_1|x_1, B_2|x_2, \cdots, B_n|x_n)(A)]'
\]
が成り立つ.
\end{gsub}




\mathstrut
\begin{gsub}
\label{gsubstsubsubst}%一般代入2
$A$を記号列とする.
また$n$を自然数とし, $B_1, B_2, \cdots, B_n$を記号列とする.
また$x_1, x_2, \cdots, x_n$を, どの二つも互いに異なる文字とする.
更に$y$を$x_1, x_2, \cdots, x_n$のいずれとも異なり, 
$B_1, B_2, \cdots, B_n$のいずれの記号列の中にも自由変数として現れない文字とし, 
$z$を$x_1, x_2, \cdots, x_n$のいずれとも異なる文字または$\Box$とする.
このとき
\[
  (B_1|x_1, B_2|x_2, \cdots, B_n|x_n)(\{z|y\}(A)) \equiv \{z|y\}((B_1|x_1, B_2|x_2, \cdots, B_n|x_n)(A))
\]
が成り立つ.
\end{gsub}




\mathstrut
\begin{gsub}
\label{gsubstsame}%一般代入3
$A$を記号列とする.
また$n$を自然数とし, $x_1, x_2, \cdots, x_n$を, どの二つも互いに異なる文字とする.

1)
$k$を$n$以下の自然数とし, $i_1, i_2, \cdots, i_k$を
$i_1 < i_2 < \cdots < i_k \LEQQ n$なる自然数とする.
また$B_1, B_2, \cdots, B_n$を, 次の条件($*$)を満たす記号列とする: 

($*$) $i$が$n$以下の自然数で, かつ$i_1, i_2, \cdots, i_k$のいずれとも異なるならば, 
      $B_i$は$x_i$である.

このとき
\[
  (B_1|x_1, B_2|x_2, \cdots, B_n|x_n)(A) \equiv (B_{i_1}|x_{i_1}, B_{i_2}|x_{i_2}, \cdots, B_{i_k}|x_{i_k})(A)
\]
が成り立つ.

2)
特に, 
\[
  (x_1|x_1, x_2|x_2, \cdots, x_n|x_n)(A) \equiv A
\]
が成り立つ.
\end{gsub}




\mathstrut
\begin{gsub}
\label{gsubstfree}%一般代入4
$A$を記号列とする.
また$n$を自然数とし, $B_1, B_2, \cdots, B_n$を記号列とする.
また$x_1, x_2, \cdots, x_n$を, どの二つも互いに異なる文字とする.

1)
$k$を$n$以下の自然数とし, $i_1, i_2, \cdots, i_k$を
$i_1 < i_2 < \cdots < i_k \LEQQ n$なる自然数とする.
いま次の条件($*$)が満たされているとする: 

($*$) $i$が$n$以下の自然数で, かつ$i_1, i_2, \cdots, i_k$のいずれとも異なるならば, 
      $x_i$は$A$の中に自由変数として現れない.

このとき
\[
  (B_1|x_1, B_2|x_2, \cdots, B_n|x_n)(A) \equiv (B_{i_1}|x_{i_1}, B_{i_2}|x_{i_2}, \cdots, B_{i_k}|x_{i_k})(A)
\]
が成り立つ.

2)
特に$x_1, x_2, \cdots, x_n$がすべて$A$の中に自由変数として現れなければ, 
\[
  (B_1|x_1, B_2|x_2, \cdots, B_n|x_n)(A) \equiv A
\]
が成り立つ.
\end{gsub}




\mathstrut
\begin{gsub}
\label{gsubstshuffle}%一般代入5
$A$を記号列とする.
また$n$を自然数とし, $B_1, B_2, \cdots, B_n$を記号列とする.
また$x_1, x_2, \cdots, x_n$をどの二つも互いに異なる文字とする.
いま自然数$1, 2, \cdots, n$を任意の順に並べて書いたものを
$i_1, i_2, \cdots, i_n$とする (即ち, $i_1, i_2, \cdots, i_n$はいずれも$n$以下の自然数で, 
これらのうちのどの二つも互いに異なるものである).
このとき
\[
  (B_1|x_1, B_2|x_2, \cdots, B_n|x_n)(A) \equiv (B_{i_1}|x_{i_1}, B_{i_2}|x_{i_2}, \cdots, B_{i_n}|x_{i_n})(A)
\]
が成り立つ.
\end{gsub}




\mathstrut
\begin{gsub}
\label{gsubstboth}%一般代入6
$A$と$B$を記号列とする.
また$n$を自然数とし, $C_1, C_2, \cdots, C_n$を記号列とする.
また$x_1, x_2, \cdots, x_n$を, どの二つも互いに異なる文字とする.
このとき
\[
  (C_1|x_1, C_2|x_2, \cdots, C_n|x_n)(AB) \equiv (C_1|x_1, C_2|x_2, \cdots, C_n|x_n)(A)(C_1|x_1, C_2|x_2, \cdots, C_n|x_n)(B)
\]
が成り立つ.
\end{gsub}




\mathstrut
\begin{gsub}
\label{gsubstfund}%一般代入7
$A$と$B$を記号列とする.
また$n$を自然数とし, $C_1, C_2, \cdots, C_n$を記号列とする.
また$x_1, x_2, \cdots, x_n$を, どの二つも互いに異なる文字とする.
また$s$を特殊記号とする.
このとき
\begin{align*}
  &(C_1|x_1, C_2|x_2, \cdots, C_n|x_n)(\neg A) \equiv \neg (C_1|x_1, C_2|x_2, \cdots, C_n|x_n)(A), \\
  \mbox{} \\
  &(C_1|x_1, C_2|x_2, \cdots, C_n|x_n)(\to\! AB) \equiv \ \to\! (C_1|x_1, C_2|x_2, \cdots, C_n|x_n)(A)(C_1|x_1, C_2|x_2, \cdots, C_n|x_n)(B), \\
  \mbox{} \\
  &(C_1|x_1, C_2|x_2, \cdots, C_n|x_n)(\vee AB) \equiv \vee (C_1|x_1, C_2|x_2, \cdots, C_n|x_n)(A)(C_1|x_1, C_2|x_2, \cdots, C_n|x_n)(B), \\
  \mbox{} \\
  &(C_1|x_1, C_2|x_2, \cdots, C_n|x_n)(sAB) \equiv s(C_1|x_1, C_2|x_2, \cdots, C_n|x_n)(A)(C_1|x_1, C_2|x_2, \cdots, C_n|x_n)(B)
\end{align*}
が成り立つ.
\end{gsub}




\mathstrut
\begin{gsub}
\label{gsubsttrans}%一般代入8
$A$を記号列とする.
また$n$を自然数とし, $B_1, B_2, \cdots, B_n$を記号列とする.
また$x_1, x_2, \cdots, x_n$を, どの二つも互いに異なる文字とする.
また$k$を$n$以下の自然数とし, $i_1, i_2, \cdots, i_k$を
$i_1 < i_2 < \cdots < i_k \LEQQ n$なる自然数とする.
また$y_1, y_2, \cdots, y_k$はいずれも$A$の中に自由変数として現れない文字で, 
これらのうちのどの二つも互いに異なるとする.
いま$i_1, i_2, \cdots, i_k$のいずれとも異なるような$n$以下の任意の自然数$i$に対し, 
$x_i$は$y_1, y_2, \cdots, y_k$のいずれの文字とも異なるとする.
このとき, 記号列$(B_1|x_1, B_2|x_2, \cdots, B_n|x_n)(A)$は, 
\begin{multline*}
  (B_1|x_1, \cdots, B_{i_1-1}|x_{i_1-1}, B_{i_1}|y_1, B_{i_1+1}|x_{i_1+1}, \cdots, B_{i_2-1}|x_{i_2-1}, B_{i_2}|y_2, B_{i_2+1}|x_{i_2+1}, \\
  \cdots\cdots, B_{i_k-1}|x_{i_k-1}, B_{i_k}|y_k, B_{i_k+1}|x_{i_k+1}, \cdots, B_n|x_n)((y_1|x_{i_1}, y_2|x_{i_2}, \cdots, y_k|x_{i_k})(A))
\end{multline*}
と同じである.
\end{gsub}




\mathstrut
\begin{gsub}
\label{gsubsttrans2}%一般代入9
$A$を記号列とする.
また$n$を自然数とし, $B_1, B_2, \cdots, B_n$を記号列とする.
また$x_1, x_2, \cdots, x_n$を, どの二つも互いに異なる文字とする.
また$y_1, y_2, \cdots, y_n$を, どの二つも互いに異なり, 
かついずれも$A$の中に自由変数として現れない文字とする.
このとき
\[
  (B_1|x_1, B_2|x_2, \cdots, B_n|x_n)(A) \equiv (B_1|y_1, B_2|y_2, \cdots, B_n|y_n)((y_1|x_1, y_2|x_2, \cdots, y_n|x_n)(A))
\]
が成り立つ.
\end{gsub}




\mathstrut
\begin{gsub}
\label{gsubstsubst}%一般代入10
$A$を記号列とする.
また$m$と$n$を自然数とし, $B_1, \cdots, B_m$及び$C_1, \cdots, C_n$を記号列とする.
また$x_1, \cdots, x_m, y_1, \cdots, y_n$を, どの二つも互いに異なる文字とする.
$x_1, \cdots, x_m$がすべて$C_1, \cdots, C_n$のいずれの記号列の中にも自由変数として現れなければ, 
\begin{multline*}
  (C_1|y_1, \cdots, C_n|y_n)((B_1|x_1, \cdots, B_m|x_m)(A)) \\
  \equiv ((C_1|y_1, \cdots, C_n|y_n)(B_1)|x_1, \cdots, (C_1|y_1, \cdots, C_n|y_n)(B_m)|x_m)((C_1|y_1, \cdots, C_n|y_n)(A))
\end{multline*}
が成り立つ.
更にこのとき, $y_1, \cdots, y_n$がすべて$B_1, \cdots, B_m$のいずれの記号列の中にも自由変数として現れなければ, 
\[
  (C_1|y_1, \cdots, C_n|y_n)((B_1|x_1, \cdots, B_m|x_m)(A)) \equiv (B_1|x_1, \cdots, B_m|x_m)((C_1|y_1, \cdots, C_n|y_n)(A))
\]
が成り立つ.
\end{gsub}




\mathstrut
\begin{gsub}
\label{gsubstsynthesis}%一般代入11
$A$を記号列とする.
また$m$と$n$を自然数とし, $B_1, B_2, \cdots, B_m$及び$C_1, C_2, \cdots, C_n$を記号列とする.
また$x_1, x_2, \cdots, x_m, y_1, y_2, \cdots, y_n$を, どの二つも互いに異なる文字とする.
$y_1, y_2, \cdots, y_n$がすべて$B_1, B_2, \cdots, B_m$のいずれの記号列の中にも自由変数として現れなければ, 
\begin{multline*}
  (C_1|y_1, C_2|y_2, \cdots, C_n|y_n)((B_1|x_1, B_2|x_2, \cdots, B_m|x_m)(A)) \\
  \equiv (B_1|x_1, B_2|x_2, \cdots, B_m|x_m, C_1|y_1, C_2|y_2, \cdots, C_n|y_n)(A)
\end{multline*}
が成り立つ.
\end{gsub}




\mathstrut
\begin{gsub}
\label{gsubstarrange}%一般代入12
$A$を記号列とする.
また$n$を自然数とし, $B_1, B_2, \cdots, B_n$を記号列とする.
また$i_1, i_2, \cdots, i_n$を自然数とし, 
$x_{11}, x_{12}, \cdots, x_{1i_1}, x_{21}, x_{22}, \cdots, x_{2i_2}, \cdots\cdots, x_{n1}, x_{n2}, \cdots, x_{ni_n}$を, 
どの二つも互いに異なる文字とする.

1)
$j_1, j_2, \cdots, j_n$を, それぞれ$i_1, i_2, \cdots, i_n$以下の自然数とする.
このとき, 記号列
\[
  (B_1|x_{11}, B_1|x_{12}, \cdots, B_1|x_{1i_1}, B_2|x_{21}, B_2|x_{22}, \cdots, B_2|x_{2i_2}, \cdots\cdots, B_n|x_{n1}, B_n|x_{n2}, \cdots, B_n|x_{ni_n})(A)
\]
は, 
\begin{multline*}
  (B_1|x_{1j_1}, B_2|x_{2j_2}, \cdots, B_n|x_{nj_n})((x_{1j_1}|x_{11}, x_{1j_1}|x_{12}, \cdots, x_{1j_1}|x_{1i_1}, x_{2j_2}|x_{21}, x_{2j_2}|x_{22}, \cdots, x_{2j_2}|x_{2i_2}, \\
  \cdots\cdots, x_{nj_n}|x_{n1}, x_{nj_n}|x_{n2}, \cdots, x_{nj_n}|x_{ni_n})(A))
\end{multline*}
と同じである.

2)
$y_1, y_2, \cdots, y_n$を, どの二つも互いに異なり, 
いずれも$A$の中に自由変数として現れない文字とする.
このとき, 記号列
\[
  (B_1|x_{11}, B_1|x_{12}, \cdots, B_1|x_{1i_1}, B_2|x_{21}, B_2|x_{22}, \cdots, B_2|x_{2i_2}, \cdots\cdots, B_n|x_{n1}, B_n|x_{n2}, \cdots, B_n|x_{ni_n})(A)
\]
は, 
\begin{multline*}
  (B_1|y_1, B_2|y_2, \cdots, B_n|y_n)((y_1|x_{11}, y_1|x_{12}, \cdots, y_1|x_{1i_1}, y_2|x_{21}, y_2|x_{22}, \cdots, y_2|x_{2i_2}, \\
  \cdots\cdots, y_n|x_{n1}, y_n|x_{n2}, \cdots, y_n|x_{ni_n})(A))
\end{multline*}
と同じである.
\end{gsub}




\mathstrut
\begin{gsub}
\label{gsubsttau}%一般代入13
$A$を記号列とし, $x$を文字とする.
また$n$を自然数とし, $B_1, B_2, \cdots, B_n$を記号列とする.
また$y_1, y_2, \cdots, y_n$を, どの二つも互いに異なる文字とする.
$x$が$y_1, y_2, \cdots, y_n$のいずれとも異なり, かつ
$B_1, B_2, \cdots, B_n$のいずれの記号列の中にも自由変数として現れなければ, 
\[
  (B_1|y_1, B_2|y_2, \cdots, B_n|y_n)(\tau_x(A)) \equiv \tau_x((B_1|y_1, B_2|y_2, \cdots, B_n|y_n)(A))
\]
が成り立つ.
\end{gsub}




\mathstrut
\begin{gsub}
\label{gsubstwedge}%一般代入14
$A$と$B$を記号列とする.
また$n$を自然数とし, $C_{1}, C_{2}, \cdots, C_{n}$を記号列とする.
また$x_{1}, x_{2}, \cdots, x_{n}$を, どの二つも互いに異なる文字とする.
このとき
\[
  (C_{1}|x_{1}, C_{2}|x_{2}, \cdots, C_{n}|x_{n})(A \wedge B) 
  \equiv (C_{1}|x_{1}, C_{2}|x_{2}, \cdots, C_{n}|x_{n})(A) \wedge (C_{1}|x_{1}, C_{2}|x_{2}, \cdots, C_{n}|x_{n})(B)
\]
が成り立つ.
\end{gsub}




\mathstrut
\begin{gsub}
\label{gsubstgvee}%一般代入15
$m$を自然数とし, $A_{1}, \cdots, A_{m}$を記号列とする.
また$n$を自然数とし, $B_{1}, \cdots, B_{n}$を記号列とする.
また$x_{1}, \cdots, x_{n}$を, どの二つも互いに異なる文字とする.
このとき
\[
  (B_{1}|x_{1}, \cdots, B_{n}|x_{n})(A_{1} \vee \cdots \vee A_{m}) 
  \equiv (B_{1}|x_{1}, \cdots, B_{n}|x_{n})(A_{1}) \vee \cdots \vee (B_{1}|x_{1}, \cdots, B_{n}|x_{n})(A_{m})
\]
が成り立つ.
\end{gsub}




\mathstrut
\begin{gsub}
\label{gsubstgwedge}%一般代入16
$m$を自然数とし, $A_{1}, \cdots, A_{m}$を記号列とする.
また$n$を自然数とし, $B_{1}, \cdots, B_{n}$を記号列とする.
また$x_{1}, \cdots, x_{n}$を, どの二つも互いに異なる文字とする.
このとき
\[
  (B_{1}|x_{1}, \cdots, B_{n}|x_{n})(A_{1} \wedge \cdots \wedge A_{m}) 
  \equiv (B_{1}|x_{1}, \cdots, B_{n}|x_{n})(A_{1}) \wedge \cdots \wedge (B_{1}|x_{1}, \cdots, B_{n}|x_{n})(A_{m})
\]
が成り立つ.
\end{gsub}




\mathstrut
\begin{gsub}
\label{gsubstequiv}%一般代入17
$A$と$B$を記号列とする.
また$n$を自然数とし, $C_{1}, C_{2}, \cdots, C_{n}$を記号列とする.
また$x_{1}, x_{2}, \cdots, x_{n}$を, どの二つも互いに異なる文字とする.
このとき
\[
  (C_{1}|x_{1}, C_{2}|x_{2}, \cdots, C_{n}|x_{n})(A \leftrightarrow B) 
  \equiv (C_{1}|x_{1}, C_{2}|x_{2}, \cdots, C_{n}|x_{n})(A) \leftrightarrow (C_{1}|x_{1}, C_{2}|x_{2}, \cdots, C_{n}|x_{n})(B)
\]
が成り立つ.
\end{gsub}




\mathstrut
\begin{gsub}
\label{gsubstquan}%一般代入18
$R$を記号列とし, $x$を文字とする.
また$n$を自然数とし, $T_{1}, T_{2}, \cdots, T_{n}$を記号列とする.
また$y_{1}, y_{2}, \cdots, y_{n}$を, どの二つも互いに異なる文字とする.
$x$が$y_{1}, y_{2}, \cdots, y_{n}$のいずれとも異なり, かつ
$T_{1}, T_{2}, \cdots, T_{n}$のいずれの記号列の中にも自由変数として現れなければ, 
\begin{align*}
  (T_{1}|y_{1}, T_{2}|y_{2}, \cdots, T_{n}|y_{n})(\exists x(R)) 
  &\equiv \exists x((T_{1}|y_{1}, T_{2}|y_{2}, \cdots, T_{n}|y_{n})(R)), \\
  \mbox{} \\
  (T_{1}|y_{1}, T_{2}|y_{2}, \cdots, T_{n}|y_{n})(\forall x(R)) 
  &\equiv \forall x((T_{1}|y_{1}, T_{2}|y_{2}, \cdots, T_{n}|y_{n})(R))
\end{align*}
が成り立つ.
\end{gsub}




\mathstrut
\begin{gsub}
\label{gsubstspquan}%一般代入19
$A$と$R$を記号列とし, $x$を文字とする.
また$n$を自然数とし, $T_{1}, T_{2}, \cdots, T_{n}$を記号列とする.
また$y_{1}, y_{2}, \cdots, y_{n}$を, どの二つも互いに異なる文字とする.
$x$が$y_{1}, y_{2}, \cdots, y_{n}$のいずれとも異なり, かつ
$T_{1}, T_{2}, \cdots, T_{n}$のいずれの記号列の中にも自由変数として現れなければ, 
\begin{align*}
  (T_{1}|y_{1}, T_{2}|y_{2}, \cdots, T_{n}|y_{n})(\exists_{A}x(R)) 
  &\equiv \exists_{(T_{1}|y_{1}, T_{2}|y_{2}, \cdots, T_{n}|y_{n})(A)}x((T_{1}|y_{1}, T_{2}|y_{2}, \cdots, T_{n}|y_{n})(R)), \\
  \mbox{} \\
  (T_{1}|y_{1}, T_{2}|y_{2}, \cdots, T_{n}|y_{n})(\forall_{A}x(R)) 
  &\equiv \forall_{(T_{1}|y_{1}, T_{2}|y_{2}, \cdots, T_{n}|y_{n})(A)}x((T_{1}|y_{1}, T_{2}|y_{2}, \cdots, T_{n}|y_{n})(R))
\end{align*}
が成り立つ.
\end{gsub}




\mathstrut
\begin{gsub}
\label{gsubst!}%一般代入20
$R$を記号列とし, $x$を文字とする.
また$n$を自然数とし, $T_{1}, T_{2}, \cdots, T_{n}$を記号列とする.
また$y_{1}, y_{2}, \cdots, y_{n}$を, どの二つも互いに異なる文字とする.
$x$が$y_{1}, y_{2}, \cdots, y_{n}$のいずれとも異なり, かつ
$T_{1}, T_{2}, \cdots, T_{n}$のいずれの記号列の中にも自由変数として現れなければ, 
\[
  (T_{1}|y_{1}, T_{2}|y_{2}, \cdots, T_{n}|y_{n})(!x(R)) 
  \equiv \ !x((T_{1}|y_{1}, T_{2}|y_{2}, \cdots, T_{n}|y_{n})(R))
\]
が成り立つ.
\end{gsub}




\mathstrut
\begin{gsub}
\label{gsubstsp!}%一般代入21
$A$と$R$を記号列とし, $x$を文字とする.
また$n$を自然数とし, $T_{1}, T_{2}, \cdots, T_{n}$を記号列とする.
また$y_{1}, y_{2}, \cdots, y_{n}$を, どの二つも互いに異なる文字とする.
$x$が$y_{1}, y_{2}, \cdots, y_{n}$のいずれとも異なり, かつ
$T_{1}, T_{2}, \cdots, T_{n}$のいずれの記号列の中にも自由変数として現れなければ, 
\[
  (T_{1}|y_{1}, T_{2}|y_{2}, \cdots, T_{n}|y_{n})(!_{A}x(R)) 
  \equiv \ !_{(T_{1}|y_{1}, T_{2}|y_{2}, \cdots, T_{n}|y_{n})(A)}x((T_{1}|y_{1}, T_{2}|y_{2}, \cdots, T_{n}|y_{n})(R))
\]
が成り立つ.
\end{gsub}




\mathstrut
\begin{gsub}
\label{gsubstex!}%一般代入22
$R$を記号列とし, $x$を文字とする.
また$n$を自然数とし, $T_{1}, T_{2}, \cdots, T_{n}$を記号列とする.
また$y_{1}, y_{2}, \cdots, y_{n}$を, どの二つも互いに異なる文字とする.
$x$が$y_{1}, y_{2}, \cdots, y_{n}$のいずれとも異なり, かつ
$T_{1}, T_{2}, \cdots, T_{n}$のいずれの記号列の中にも自由変数として現れなければ, 
\[
  (T_{1}|y_{1}, T_{2}|y_{2}, \cdots, T_{n}|y_{n})(\exists !x(R)) 
  \equiv \exists !x((T_{1}|y_{1}, T_{2}|y_{2}, \cdots, T_{n}|y_{n})(R))
\]
が成り立つ.
\end{gsub}




\mathstrut
\begin{gsub}
\label{gsubstspex!}%一般代入23
$A$と$R$を記号列とし, $x$を文字とする.
また$n$を自然数とし, $T_{1}, T_{2}, \cdots, T_{n}$を記号列とする.
また$y_{1}, y_{2}, \cdots, y_{n}$を, どの二つも互いに異なる文字とする.
$x$が$y_{1}, y_{2}, \cdots, y_{n}$のいずれとも異なり, かつ
$T_{1}, T_{2}, \cdots, T_{n}$のいずれの記号列の中にも自由変数として現れなければ, 
\[
  (T_{1}|y_{1}, T_{2}|y_{2}, \cdots, T_{n}|y_{n})(\exists !_{A}x(R)) 
  \equiv \exists !_{(T_{1}|y_{1}, T_{2}|y_{2}, \cdots, T_{n}|y_{n})(A)}x((T_{1}|y_{1}, T_{2}|y_{2}, \cdots, T_{n}|y_{n})(R))
\]
が成り立つ.
\end{gsub}




\newpage




\section{代入法則}




\mathstrut
\begin{subs}
\label{substsame}%代入1
$A$を記号列とし, $x$を文字とする.
このとき
\[
  (x|x)(A) \equiv A
\]
が成り立つ.
\end{subs}




\mathstrut
\begin{subs}
\label{substfree}%代入2
$A$と$B$を記号列とし, $x$を文字とする.
$x$が$A$の中に自由変数として現れなければ, 
\[
  (B|x)(A) \equiv A
\]
が成り立つ.
\end{subs}




\mathstrut
\begin{subs}
\label{substboth}%代入3
$A$, $B$, $C$を記号列とし, $x$を文字とする.
このとき
\[
  (C|x)(AB) \equiv (C|x)(A)(C|x)(B)
\]
が成り立つ.
\end{subs}




\mathstrut
\begin{subs}
\label{substfund}%代入4
$A$, $B$, $C$を記号列, $x$を文字, $s$を特殊記号とする.
このとき
\begin{align*}
  &(C|x)(\neg A) \equiv \neg (C|x)(A), \\
  \mbox{} \\
  &(C|x)(\to\! AB) \equiv \ \to\! (C|x)(A)(C|x)(B), \\
  \mbox{} \\
  &(C|x)(\vee AB) \equiv \vee (C|x)(A)(C|x)(B), \\
  \mbox{} \\
  &(C|x)(sAB) \equiv s(C|x)(A)(C|x)(B)
\end{align*}
が成り立つ.
\end{subs}




\mathstrut
\begin{subs}
\label{substtrans}%代入5
$A$と$B$を記号列とし, $x$と$y$を文字とする.
$y$が$A$の中に自由変数として現れなければ, 
\[
  (B|x)(A) \equiv (B|y)((y|x)(A))
\]
が成り立つ.
\end{subs}




\mathstrut
\begin{subs}
\label{substsubst}%代入6
$A$, $B$, $C$を記号列とし, $x$と$y$を異なる文字とする.
$x$が$C$の中に自由変数として現れなければ, 
\[
  (C|y)((B|x)(A)) \equiv ((C|y)(B)|x)((C|y)(A))
\]
が成り立つ.
更にこのとき, $y$が$B$の中に自由変数として現れなければ, 
\[
  (C|y)((B|x)(A)) \equiv (B|x)((C|y)(A))
\]
が成り立つ.
\end{subs}




\mathstrut
\begin{subs}
\label{substtautrans}%代入7
$A$を記号列とし, $x$と$y$を文字とする.
$y$が$A$の中に自由変数として現れなければ, 
\[
  \tau_x(A) \equiv \tau_y((y|x)(A))
\]
が成り立つ.
\end{subs}




\mathstrut
\begin{subs}
\label{substtau}%代入8
$A$と$B$を記号列とし, $x$と$y$を異なる文字とする.
$x$が$B$の中に自由変数として現れなければ, 
\[
  (B|y)(\tau_x(A)) \equiv \tau_x((B|y)(A))
\]
が成り立つ.
\end{subs}




\mathstrut
\begin{subs}
\label{substwedge}%代入9
$A$, $B$, $C$を記号列とし, $x$を文字とする.
このとき
\[
  (C|x)(A \wedge B) \equiv (C|x)(A) \wedge (C|x)(B)
\]
が成り立つ.
\end{subs}




\mathstrut
\begin{subs}
\label{substgvee}%代入10
$n$を自然数とし, $A_{1}, A_{2}, \cdots, A_{n}$を記号列とする.
また$B$を記号列とし, $x$を文字とする.
このとき
\[
  (B|x)(A_{1} \vee A_{2} \vee \cdots \vee A_{n}) 
  \equiv (B|x)(A_{1}) \vee (B|x)(A_{2}) \vee \cdots \vee (B|x)(A_{n})
\]
が成り立つ.
\end{subs}




\mathstrut
\begin{subs}
\label{substgwedge}%代入11
$n$を自然数とし, $A_{1}, A_{2}, \cdots, A_{n}$を記号列とする.
また$B$を記号列とし, $x$を文字とする.
このとき
\[
  (B|x)(A_{1} \wedge A_{2} \wedge \cdots \wedge A_{n}) 
  \equiv (B|x)(A_{1}) \wedge (B|x)(A_{2}) \wedge \cdots \wedge (B|x)(A_{n})
\]
が成り立つ.
\end{subs}




\mathstrut
\begin{subs}
\label{substequiv}%代入12
$A$, $B$, $C$を記号列とし, $x$を文字とする.
このとき
\[
  (C|x)(A \leftrightarrow B) \equiv (C|x)(A) \leftrightarrow (C|x)(B)
\]
が成り立つ.
\end{subs}




\mathstrut
\begin{subs}
\label{substquantrans}%代入13
$R$を記号列とし, $x$と$y$を文字とする.
$y$が$R$の中に自由変数として現れなければ, 
\[
  \exists x(R) \equiv \exists y((y|x)(R)), ~~
  \forall x(R) \equiv \forall y((y|x)(R))
\]
が成り立つ.
\end{subs}




\mathstrut
\begin{subs}
\label{substquan}%代入14
$R$と$T$を記号列とし, $x$と$y$を異なる文字とする.
$x$が$T$の中に自由変数として現れなければ, 
\[
  (T|y)(\exists x(R)) \equiv \exists x((T|y)(R)), ~~
  (T|y)(\forall x(R)) \equiv \forall x((T|y)(R))
\]
が成り立つ.
\end{subs}




\mathstrut
\begin{subs}
\label{substspquantrans}%代入15
$A$と$R$を記号列とし, $x$と$y$を文字とする.
$y$が$A$の中にも$R$の中にも自由変数として現れなければ, 
\[
  \exists_{A}x(R) \equiv \exists_{(y|x)(A)}y((y|x)(R)), ~~
  \forall_{A}x(R) \equiv \forall_{(y|x)(A)}y((y|x)(R))
\]
が成り立つ.
\end{subs}




\mathstrut
\begin{subs}
\label{substspquan}%代入16
$A$, $R$, $T$を記号列とし, $x$と$y$を異なる文字とする.
$x$が$T$の中に自由変数として現れなければ, 
\[
  (T|y)(\exists_{A}x(R)) \equiv \exists_{(T|y)(A)}x((T|y)(R)), ~~
  (T|y)(\forall_{A}x(R)) \equiv \forall_{(T|y)(A)}x((T|y)(R))
\]
が成り立つ.
\end{subs}




\mathstrut
\begin{subs}
\label{subst!trans}%代入17
$R$を記号列とし, $x$と$y$を文字とする.
$y$が$R$の中に自由変数として現れなければ, 
\[
  !x(R) \equiv \ !y((y|x)(R))
\]
が成り立つ.
\end{subs}




\mathstrut
\begin{subs}
\label{subst!}%代入18
$R$と$T$を記号列とし, $x$と$y$を異なる文字とする.
$x$が$T$の中に自由変数として現れなければ, 
\[
  (T|y)(!x(R)) \equiv \ !x((T|y)(R))
\]
が成り立つ.
\end{subs}




\mathstrut
\begin{subs}
\label{substsp!trans}%代入19
$A$と$R$を記号列とし, $x$と$y$を文字とする.
$y$が$A$の中にも$R$の中にも自由変数として現れなければ, 
\[
  !_{A}x(R) \equiv \ !_{(y|x)(A)}y((y|x)(R))
\]
が成り立つ.
\end{subs}




\mathstrut
\begin{subs}
\label{substsp!}%代入20
$A$, $R$, $T$を記号列とし, $x$と$y$を異なる文字とする.
$x$が$T$の中に自由変数として現れなければ, 
\[
  (T|y)(!_{A}x(R)) \equiv \ !_{(T|y)(A)}x((T|y)(R))
\]
が成り立つ.
\end{subs}




\mathstrut
\begin{subs}
\label{substex!trans}%代入21
$R$を記号列とし, $x$と$y$を文字とする.
$y$が$R$の中に自由変数として現れなければ, 
\[
  \exists !x(R) \equiv \exists !y((y|x)(R))
\]
が成り立つ.
\end{subs}




\mathstrut
\begin{subs}
\label{substex!}%代入22
$R$と$T$を記号列とし, $x$と$y$を異なる文字とする.
$x$が$T$の中に自由変数として現れなければ, 
\[
  (T|y)(\exists !x(R)) \equiv \exists !x((T|y)(R))
\]
が成り立つ.
\end{subs}




\mathstrut
\begin{subs}
\label{substspex!trans}%代入23
$A$と$R$を記号列とし, $x$と$y$を文字とする.
$y$が$A$の中にも$R$の中にも自由変数として現れなければ, 
\[
  \exists !_{A}x(R) \equiv \exists !_{(y|x)(A)}y((y|x)(R))
\]
が成り立つ.
\end{subs}




\mathstrut
\begin{subs}
\label{substspex!}%代入24
$A$, $R$, $T$を記号列とし, $x$と$y$を異なる文字とする.
$x$が$T$の中に自由変数として現れなければ, 
\[
  (T|y)(\exists !_{A}x(R)) \equiv \exists !_{(T|y)(A)}x((T|y)(R))
\]
が成り立つ.
\end{subs}




\newpage




\section{構成法則}




\mathstrut
\begin{form}
\label{formprocprt}%構成1
$n$を自然数, $A_1, A_2, \cdots, A_n$を記号列とし, 
これらの列$\langle A_1, A_2, \cdots, A_n \rangle$が
理論$\mathscr{T}$における構成手続きであるとする.
このとき$n$以下の任意の自然数$i$に対して, 
列$\langle A_1, A_2, \cdots, A_i \rangle$は$\mathscr{T}$における構成手続きである.
故に$A_1, A_2, \cdots, A_n$はすべて$\mathscr{T}$の論理式である.
\end{form}




\mathstrut
\begin{form}
\label{formfund}%構成2
$A$と$B$を記号列とし, $x$を文字, $s$を理論$\mathscr{T}$の特殊記号とする.

1)
$A$が$\mathscr{T}$の関係式ならば, $\neg A$は$\mathscr{T}$の関係式である.

2)
$A$と$B$が共に$\mathscr{T}$の関係式ならば, $\to\! AB$は$\mathscr{T}$の関係式である.

3)
$A$と$B$が共に$\mathscr{T}$の関係式ならば, $\vee AB$は$\mathscr{T}$の関係式である.

4)
$x$は$\mathscr{T}$の対象式である.

5)
$A$が$\mathscr{T}$の関係式ならば, $\tau_x(A)$は$\mathscr{T}$の対象式である.

6)
$A$と$B$が共に$\mathscr{T}$の対象式ならば, $sAB$は$\mathscr{T}$の関係式である.
\end{form}




\mathstrut
\begin{form}
\label{formid}%構成3
$A$を記号列とする.

1)
$A$が理論$\mathscr{T}$の関係式ならば, 次のi) - iii)のいずれかが成立する: 

~~~i)
$\mathscr{T}$の関係式$B$があり, $A$は$\neg B$である.

~~~ii)
$\mathscr{T}$の関係式$B$, $C$があり, $A$は$\to\! BC$である.

~~~iii)
$\mathscr{T}$の対象式$B$, $C$及び$\mathscr{T}$の特殊記号$s$があり, $A$は$sBC$である.

2)
$A$が理論$\mathscr{T}$の対象式ならば, 次のiv), v)のどちらかが成立する: 

~~~iv)
$A$は文字である.

~~~v)
$\mathscr{T}$の関係式$B$及び文字$x$があり, $A$は$\tau_x(B)$である.
\end{form}




\mathstrut
\begin{form}
\label{formsubstlett12}%構成4
$A$を記号列とし, $x$と$y$を文字とする.

1)
$A$が第一種ならば, $(y|x)(A)$は第一種である.

2)
$A$が第二種ならば, $(y|x)(A)$は第二種である.
\end{form}




\mathstrut
\begin{form}
\label{formprocsubst}%構成5
$n$を自然数, $A_1, A_2, \cdots, A_n$を記号列とし, 
これらの記号列の列$\langle A_1, A_2, \cdots, A_n \rangle$が
理論$\mathscr{T}$における構成手続きであるとする.
また$x$と$y$を文字とし, $y$は$A_1, A_2, \cdots, A_n$のいずれの記号列の中にも自由変数として現れないとする.
このとき$\langle (y|x)(A_1), (y|x)(A_2), \cdots, (y|x)(A_n) \rangle$は$\mathscr{T}$における構成手続きである.
\end{form}




\mathstrut
\begin{form}
\label{formsubstlett}%構成6
$A$を記号列とし, $x$と$y$を文字とする.

1)
$A$が理論$\mathscr{T}$の対象式ならば, $(y|x)(A)$は$\mathscr{T}$の対象式である.

2)
$A$が$\mathscr{T}$の関係式ならば, $(y|x)(A)$は$\mathscr{T}$の関係式である.
\end{form}




\mathstrut
\begin{form}
\label{formgsubstlettcor}%構成7
$A$を記号列とする.
また$n$を自然数とし, $x_1, x_2, \cdots, x_n$をどの二つも互いに異なる文字とする.
また$y_1, y_2, \cdots, y_n$を, すべて$x_1, x_2, \cdots, x_n$のいずれとも異なる文字とする.

1)
$A$が理論$\mathscr{T}$の対象式ならば, 
$(y_1|x_1, y_2|x_2, \cdots, y_n|x_n)(A)$は$\mathscr{T}$の対象式である.

2)
$A$が$\mathscr{T}$の関係式ならば, 
$(y_1|x_1, y_2|x_2, \cdots, y_n|x_n)(A)$は$\mathscr{T}$の関係式である.
\end{form}




\mathstrut
\begin{form}
\label{formgsubstlett}%構成8
$A$を記号列とする.
また$n$を自然数とし, $x_1, x_2, \cdots, x_n$をどの二つも互いに異なる文字とする.
また$y_1, y_2, \cdots, y_n$を文字とする.

1)
$A$が理論$\mathscr{T}$の対象式ならば, 
$(y_1|x_1, y_2|x_2, \cdots, y_n|x_n)(A)$は$\mathscr{T}$の対象式である.

2)
$A$が$\mathscr{T}$の関係式ならば, 
$(y_1|x_1, y_2|x_2, \cdots, y_n|x_n)(A)$は$\mathscr{T}$の関係式である.
\end{form}




\mathstrut
\begin{form}
\label{formsubst12}%構成9
$A$を記号列とし, $x$を文字とする.
また$T$を理論$\mathscr{T}$の対象式とする.

1)
$A$が$\mathscr{T}$の対象式ならば, $(T|x)(A)$は第一種である.

2)
$A$が$\mathscr{T}$の関係式ならば, $(T|x)(A)$は第二種である.
\end{form}




\mathstrut
\begin{form}
\label{formsubst}%構成10
$A$を記号列とし, $x$を文字とする.
また$T$を理論$\mathscr{T}$の対象式とする.

1)
$A$が$\mathscr{T}$の対象式ならば, $(T|x)(A)$は$\mathscr{T}$の対象式である.

2)
$A$が$\mathscr{T}$の関係式ならば, $(T|x)(A)$は$\mathscr{T}$の関係式である.
\end{form}




\mathstrut
\begin{form}
\label{formgsubst}%構成11
$A$を記号列とする.
また$n$を自然数とし, $x_1, x_2, \cdots, x_n$をどの二つも互いに異なる文字とする.
また$T_1, T_2, \cdots, T_n$を理論$\mathscr{T}$の対象式とする.

1)
$A$が$\mathscr{T}$の対象式ならば, 
$(T_1|x_1, T_2|x_2, \cdots, T_n|x_n)(A)$は$\mathscr{T}$の対象式である.

2)
$A$が$\mathscr{T}$の関係式ならば, 
$(T_1|x_1, T_2|x_2, \cdots, T_n|x_n)(A)$は$\mathscr{T}$の関係式である.
\end{form}




\mathstrut
\begin{form}
\label{formsep}%構成12
$A$と$B$を理論$\mathscr{T}$の論理式, $C$と$D$を記号列とし, 
$AC \equiv BD$であるとする.
このとき$A \equiv B$かつ$C \equiv D$が成り立つ.
\end{form}




\mathstrut
\begin{form}
\label{formneg}%構成13
$A$を記号列とする.
$\neg A$が理論$\mathscr{T}$の関係式ならば, $A$は$\mathscr{T}$の関係式である.
\end{form}




\mathstrut
\begin{form}
\label{formtb}%構成14
$A$と$B$を記号列とする.
$A$と$\to\! AB$が共に理論$\mathscr{T}$の関係式ならば, $B$は$\mathscr{T}$の関係式である.
\end{form}




\mathstrut
\begin{form}
\label{formtsep}%構成15
$A$と$B$を理論$\mathscr{T}$の論理式, $C$と$D$を記号列とし, 
$\to\! AC \equiv \ \to\! BD$であるとする.
このとき$A \equiv B$かつ$C \equiv D$が成り立つ.
\end{form}




\mathstrut
\begin{form}
\label{formvb}%構成16
$A$と$B$を記号列とする.
$A$と$\vee AB$が共に理論$\mathscr{T}$の関係式ならば, $B$は$\mathscr{T}$の関係式である.
\end{form}




\mathstrut
\begin{form}
\label{formvsep}%構成17
$A$と$B$を理論$\mathscr{T}$の論理式, $C$と$D$を記号列とし, 
$\vee AC \equiv \vee BD$であるとする.
このとき$A \equiv B$かつ$C \equiv D$が成り立つ.
\end{form}




\mathstrut
\begin{form}
\label{formsb}%構成18
$A$と$B$を記号列とし, $s$を理論$\mathscr{T}$の特殊記号とする.
$A$が$\mathscr{T}$の対象式で, $sAB$が$\mathscr{T}$の関係式ならば, 
$B$は$\mathscr{T}$の対象式である.
\end{form}




\mathstrut
\begin{form}
\label{formssep}%構成19
$A$と$B$を理論$\mathscr{T}$の論理式, $C$と$D$を記号列, 
$s$を$\mathscr{T}$の特殊記号とし, $sAC \equiv sBD$であるとする.
このとき$A \equiv B$かつ$C \equiv D$が成り立つ.
\end{form}




\mathstrut
\begin{form}
\label{formproof}%構成20
$n$を自然数, $A_1, A_2, \cdots, A_n$を記号列とし, 
これらの列$\langle A_1, A_2, \cdots, A_n \rangle$が
理論$\mathscr{T}$における証明であるとする.
このとき$A_1, A_2, \cdots, A_n$はすべて$\mathscr{T}$の関係式である.
\end{form}




\mathstrut
\begin{form}
\label{formthm}%構成21
記号列$A$が理論$\mathscr{T}$の定理ならば, $A$は$\mathscr{T}$の関係式である.
\end{form}




\mathstrut
\begin{form}
\label{formwedge}%構成22
$A$と$B$が共に理論$\mathscr{T}$の関係式ならば, 
$A \wedge B$は$\mathscr{T}$の関係式である.
\end{form}




\mathstrut
\begin{form}
\label{formwb}%構成23
$A$と$B$を記号列とする.
$A$と$A \wedge B$が共に理論$\mathscr{T}$の関係式ならば, $B$は$\mathscr{T}$の関係式である.
\end{form}




\mathstrut
\begin{form}
\label{formwsep}%構成24
$A$と$B$を理論$\mathscr{T}$の論理式, $C$と$D$を記号列とし, 
$A \wedge C \equiv B \wedge D$であるとする.
このとき$A \equiv B$かつ$C \equiv D$が成り立つ.
\end{form}




\mathstrut
\begin{form}
\label{formgvee}%構成25
$n$を自然数とし, $A_{1}, A_{2}, \cdots, A_{n}$を理論$\mathscr{T}$の関係式とする.
このとき$A_{1} \vee A_{2} \vee \cdots \vee A_{n}$は$\mathscr{T}$の関係式である.
\end{form}




\mathstrut
\begin{form}
\label{formgwedge}%構成26
$n$を自然数とし, $A_{1}, A_{2}, \cdots, A_{n}$を理論$\mathscr{T}$の関係式とする.
このとき$A_{1} \wedge A_{2} \wedge \cdots \wedge A_{n}$は$\mathscr{T}$の関係式である.
\end{form}




\mathstrut
\begin{form}
\label{formequiv}%構成27
$A$と$B$が共に理論$\mathscr{T}$の関係式ならば, 
$A \leftrightarrow B$は$\mathscr{T}$の関係式である.
\end{form}




\mathstrut
\begin{form}
\label{formeqsep}%構成28
$A$と$B$を理論$\mathscr{T}$の論理式, $C$と$D$を記号列とし, 
$A \leftrightarrow C \equiv B \leftrightarrow D$であるとする.
このとき$A \equiv B$かつ$C \equiv D$が成り立つ.
\end{form}




\mathstrut
\begin{form}
\label{formquan}%構成29
$R$が理論$\mathscr{T}$の関係式, $x$が文字ならば, 
$\exists x(R)$と$\forall x(R)$は共に$\mathscr{T}$の関係式である.
\end{form}




\mathstrut
\begin{form}
\label{formspquan}%構成30
$A$と$R$が理論$\mathscr{T}$ (限定作用素を持つ理論とは仮定しない) の関係式, $x$が文字ならば, 
$\exists_{A}x(R)$と$\forall_{A}x(R)$は共に$\mathscr{T}$の関係式である.
\end{form}




\mathstrut
\begin{form}
\label{form!}%構成31
$R$が$\mathscr{T}$の関係式, $x$が文字ならば, 
$!x(R)$は$\mathscr{T}$の関係式である.
\end{form}




\mathstrut
\begin{form}
\label{formsp!}%構成32
$A$と$R$が$\mathscr{T}$の関係式, $x$が文字ならば, 
$!_{A}x(R)$は$\mathscr{T}$の関係式である.
\end{form}




\mathstrut
\begin{form}
\label{formex!}%構成33
$R$が$\mathscr{T}$の関係式, $x$が文字ならば, 
$\exists !x(R)$は$\mathscr{T}$の関係式である.
\end{form}




\mathstrut
\begin{form}
\label{formspex!}%構成34
$A$と$R$が$\mathscr{T}$の関係式, $x$が文字ならば, 
$\exists !_{A}x(R)$は$\mathscr{T}$の関係式である.
\end{form}




\newpage




\section{骨格式構成法則}




\mathstrut
\begin{skel}
\label{skprocprt}%骨格1
$n$を自然数, $A_1, A_2, \cdots, A_n$を記号列とし, 
これらの列$\langle A_1, A_2, \cdots, A_n \rangle$が
理論$\mathscr{T}$におけるsk-構成手続きであるとする.
このとき$n$以下の任意の自然数$i$に対して, 
列$\langle A_1, A_2, \cdots, A_i \rangle$は$\mathscr{T}$におけるsk-構成手続きである.
故に$A_1, A_2, \cdots, A_n$はすべて$\mathscr{T}$の骨格式である.
\end{skel}




\mathstrut
\begin{skel}
\label{skfund}%骨格2
$A$と$B$を記号列とし, $x$を文字またはprime文字または$\Box$型記号, 
$s$を理論$\mathscr{T}$の特殊記号とする.

1)
$A$が$\mathscr{T}$のsk-関係式ならば, $\neg A$は$\mathscr{T}$のsk-関係式である.

2)
$A$と$B$が共に$\mathscr{T}$のsk-関係式ならば, $\to\! AB$は$\mathscr{T}$のsk-関係式である.

3)
$x$は$\mathscr{T}$のsk-対象式である.

4)
$A$が$\mathscr{T}$のsk-関係式ならば, $\tau A$は$\mathscr{T}$のsk-対象式である.

5)
$A$と$B$が共に$\mathscr{T}$のsk-対象式ならば, $sAB$は$\mathscr{T}$のsk-関係式である.
\end{skel}




\mathstrut
\begin{skel}
\label{skid}%骨格3
$A$を記号列とする.

1)
$A$が理論$\mathscr{T}$のsk-関係式ならば, 次のi) - iii)のいずれかが成立する: 

~~~i)
$\mathscr{T}$のsk-関係式$B$があり, $A$は$\neg B$である.

~~~ii)
$\mathscr{T}$のsk-関係式$B$, $C$があり, $A$は$\to\! BC$である.

~~~iii)
$\mathscr{T}$のsk-対象式$B$, $C$及び$\mathscr{T}$の特殊記号$s$があり, $A$は$sBC$である.

2)
$A$が理論$\mathscr{T}$のsk-対象式ならば, 次のiv), v)のどちらかが成立する: 

~~~iv)
$A$は文字であるか, prime文字であるか, $\Box$型記号である.

~~~v)
$\mathscr{T}$のsk-関係式$B$があり, $A$は$\tau B$である.
\end{skel}




\mathstrut
\begin{skel}
\label{skprime12}%骨格4
$A$を記号列とする.

1)
$A$がsk-第一種ならば, $[A]'$はsk-第一種である.

2)
$A$がsk-第二種ならば, $[A]'$はsk-第二種である.
\end{skel}




\mathstrut
\begin{skel}
\label{skprime}%骨格5
$A$を記号列とする.

1)
$A$が理論$\mathscr{T}$のsk-対象式ならば, $[A]'$は$\mathscr{T}$のsk-対象式である.

2)
$A$が$\mathscr{T}$のsk-関係式ならば, $[A]'$は$\mathscr{T}$のsk-関係式である.
\end{skel}




\mathstrut
\begin{skel}
\label{sksubsubst12}%骨格6
$A$を記号列とし, $x$を文字とする.

1)
$A$がsk-第一種ならば, $\{\Box|x\}(A)$はsk-第一種である.

2)
$A$がsk-第二種ならば, $\{\Box|x\}(A)$はsk-第二種である.
\end{skel}




\mathstrut
\begin{skel}
\label{sksubsubst}%骨格7
$A$を記号列とし, $x$を文字とする.

1)
$A$が理論$\mathscr{T}$のsk-対象式ならば, 
$\{\Box|x\}(A)$は$\mathscr{T}$のsk-対象式である.

2)
$A$が$\mathscr{T}$のsk-関係式ならば, 
$\{\Box|x\}(A)$は$\mathscr{T}$のsk-関係式である.
\end{skel}




\mathstrut
\begin{skel}
\label{skform12}%骨格8
$A$を記号列とする.

1)
$A$が理論$\mathscr{T}$の対象式ならば, $A$はsk-第一種である.

2)
$A$が$\mathscr{T}$の関係式ならば, $A$はsk-第二種である.
\end{skel}




\mathstrut
\begin{skel}
\label{skform}%骨格9
$A$を記号列とする.

1)
$A$が理論$\mathscr{T}$の対象式ならば, $A$は$\mathscr{T}$のsk-対象式である.

2)
$A$が$\mathscr{T}$の関係式ならば, $A$は$\mathscr{T}$のsk-関係式である.
\end{skel}




\mathstrut
\begin{skel}
\label{sksep}%骨格10
$A$と$B$を理論$\mathscr{T}$の骨格式, $C$と$D$を記号列とし, 
$AC \equiv BD$であるとする.
このとき$A \equiv B$かつ$C \equiv D$が成り立つ.
\end{skel}




\newpage




\section{Schema}




\mathstrut
S1. $\mathscr{T}$の関係式$A$, $B$から記号列$A \to (B \to A)$を得る.




\mathstrut
S2. $\mathscr{T}$の関係式$A$, $B$, $C$から記号列$(A \to (B \to C)) \to ((A \to B) \to (A \to C))$を得る.




\mathstrut
S3. $\mathscr{T}$の関係式$A$, $B$から記号列$(\neg B \to \neg A) \to (A \to B)$を得る.




\mathstrut
S4. $\mathscr{T}$の関係式$R$, 対象式$T$と文字$x$から, 記号列$(T|x)(R) \to \exists x(R)$を得る.




\mathstrut
S5. $\mathscr{T}$の関係式$R$, 対象式$T$, $U$と文字$x$から, 記号列
    $T = U \to ((T|x)(R) \to (U|x)(R))$を得る.




\mathstrut
S6. $\mathscr{T}$の関係式$R$, $S$と文字$x$から, 記号列
    $\forall x(R \leftrightarrow S) \to \tau_{x}(R) = \tau_{x}(S)$を得る.




\newpage




\section{推論法則}




\mathstrut
\begin{dedu}
\label{dedprfprt}%推論1
$n$を自然数, $A_1, A_2, \cdots, A_n$を記号列とし, 
これらの列$\langle A_1, A_2, \cdots, A_n \rangle$が
理論$\mathscr{T}$における証明であるとする.
このとき$n$以下の任意の自然数$i$に対して, 
列$\langle A_1, A_2, \cdots, A_i \rangle$は$\mathscr{T}$における証明である.
故に$A_1, A_2, \cdots, A_n$はすべて$\mathscr{T}$の定理である.
\end{dedu}




\mathstrut
\begin{dedu}
\label{dedaxmthm}%推論2
$A$を記号列とする.
$A$が理論$\mathscr{T}$の公理ならば, $A$は$\mathscr{T}$の定理である.
\end{dedu}




\mathstrut
\begin{dedu}
\label{dedmp}%推論3
{\bf (三段論法)}~
$A$と$B$を記号列とする.
$A$と$A \to B$が共に理論$\mathscr{T}$の定理ならば, $B$は$\mathscr{T}$の定理である.
\end{dedu}




\mathstrut
\begin{dedu}
\label{dedgsubsttheory}%推論4
$A$を理論$\mathscr{T}$の定理とする.
また$n$を自然数とし, 
$T_1, T_2, \cdots, T_n$を$\mathscr{T}$の対象式とする.
また$x_1, x_2, \cdots, x_n$を, どの二つも互いに異なる文字とする.
このとき$(T_1|x_1, T_2|x_2, \cdots, T_n|x_n)(A)$は
$(T_1|x_1, T_2|x_2, \cdots, T_n|x_n)(\mathscr{T})$の定理である.
\end{dedu}




\mathstrut
\begin{dedu}
\label{dedgsubst}%推論5
$A$を理論$\mathscr{T}$の定理とする.
また$n$を自然数とし, $T_1, T_2, \cdots, T_n$を$\mathscr{T}$の対象式とする.
また$x_1, x_2, \cdots, x_n$を, どの二つも互いに異なる文字とする.
$x_1, x_2, \cdots, x_n$がいずれも$\mathscr{T}$の定数でなければ, 
$(T_1|x_1, T_2|x_2, \cdots, T_n|x_n)(A)$は$\mathscr{T}$の定理である.
\end{dedu}




\mathstrut
\begin{dedu}
\label{dedsubst}%推論6
$A$を理論$\mathscr{T}$の定理とし, $T$を$\mathscr{T}$の対象式, $x$を文字とする.
$x$が$\mathscr{T}$の定数でなければ, $(T|x)(A)$は$\mathscr{T}$の定理である.
\end{dedu}




\mathstrut
\begin{dedu}
\label{dedtheorystrength}%推論7
理論$\mathscr{T}'$が理論$\mathscr{T}$より強ければ, 
$\mathscr{T}$のすべての定理は$\mathscr{T}'$の定理である.
\end{dedu}




\mathstrut
\begin{dedu}
\label{dedmodel}%推論8
$\mathscr{T}$を理論とし, $A_1, A_2, \cdots, A_n$をその明示的公理の全体, 
$a_1, a_2, \cdots, a_h$をその定数の全体とする ($n$, $h$は自然数).
また$T_1, T_2, \cdots, T_h$を$\mathscr{T}$の対象式とする.
いま理論$\mathscr{T}'$があり, $\mathscr{T}$の特殊記号はすべて$\mathscr{T}'$の特殊記号, 
$\mathscr{T}$のschemaはすべて$\mathscr{T}'$のschemaで, 
各$i$ ($i$は$1, 2, \cdots, n$のいずれか) に対して
$(T_1|a_1, T_2|a_2, \cdots, T_h|a_h)(A_i)$が$\mathscr{T}'$の定理であるとする.
このとき$\mathscr{T}$の任意の定理$A$に対して, 
$(T_1|a_1, T_2|a_2, \cdots, T_h|a_h)(A)$は$\mathscr{T}'$の定理である.
\end{dedu}




\mathstrut
\begin{dedu}
\label{deds1}%推論9
$A$を$\mathscr{T}$の定理, $B$を$\mathscr{T}$の関係式とするとき, 
$B \to A$は$\mathscr{T}$の定理である.
\end{dedu}




\mathstrut
\begin{dedu}
\label{deds2}%推論10
$A$, $B$, $C$を$\mathscr{T}$の関係式とする.
$A \to (B \to C)$が$\mathscr{T}$の定理ならば, 
$(A \to B) \to (A \to C)$は$\mathscr{T}$の定理である.
\end{dedu}




\mathstrut
\begin{dedu}
\label{deds3}%推論11
$A$と$B$を$\mathscr{T}$の関係式とする.
$\neg B \to \neg A$が$\mathscr{T}$の定理ならば, 
$A \to B$は$\mathscr{T}$の定理である.
\end{dedu}




\mathstrut
\begin{dedu}
\label{dedaddb}%推論12
$A$, $B$, $C$を$\mathscr{T}$の関係式とする.
$A \to B$が$\mathscr{T}$の定理ならば, 
$(C \to A) \to (C \to B)$は$\mathscr{T}$の定理である.
\end{dedu}




\mathstrut
\begin{dedu}
\label{dedaddf}%推論13
$A$, $B$, $C$を$\mathscr{T}$の関係式とする.
$A \to B$が$\mathscr{T}$の定理ならば, 
$(B \to C) \to (A \to C)$は$\mathscr{T}$の定理である.
\end{dedu}




\mathstrut
\begin{dedu}
\label{dedmmp}%推論14
$A$, $B$, $C$を$\mathscr{T}$の関係式とする.
$A \to B$と$B \to C$が共に$\mathscr{T}$の定理ならば, 
$A \to C$は$\mathscr{T}$の定理である.
\end{dedu}




\mathstrut
\begin{dedu}
\label{dedch}%推論15
$A$, $B$, $C$を$\mathscr{T}$の関係式とする.
$A \to (B \to C)$が$\mathscr{T}$の定理ならば, 
$B \to (A \to C)$は$\mathscr{T}$の定理である.
\end{dedu}




\mathstrut
\begin{dedu}
\label{dedaabab}%推論16
$A$と$B$を$\mathscr{T}$の関係式とする.
$A \to (A \to B)$が$\mathscr{T}$の定理ならば, 
$A \to B$は$\mathscr{T}$の定理である.
\end{dedu}




\mathstrut
\begin{dedu}
\label{dedpremp}%推論17
$A$, $B$, $C$を$\mathscr{T}$の関係式とする.
$A \to B$と$A \to (B \to C)$が共に$\mathscr{T}$の定理ならば, 
$A \to C$は$\mathscr{T}$の定理である.
\end{dedu}




\mathstrut
\begin{dedu}
\label{deds2opp}%推論18
$A$, $B$, $C$を$\mathscr{T}$の関係式とする.
$(A \to B) \to (A \to C)$が$\mathscr{T}$の定理ならば, 
$A \to (B \to C)$は$\mathscr{T}$の定理である.
\end{dedu}




\mathstrut
\begin{dedu}
\label{dednt}%推論19
$A$と$B$を$\mathscr{T}$の関係式とする.

1)
$\neg A$が$\mathscr{T}$の定理ならば, $A \to B$は$\mathscr{T}$の定理である.

2)
$A$が$\mathscr{T}$の定理ならば, $\neg A \to B$は$\mathscr{T}$の定理である.
\end{dedu}




\mathstrut
\begin{dedu}
\label{dedcontra}%推論20
論理的な理論$\mathscr{T}$が矛盾すれば, $\mathscr{T}$のすべての関係式は
$\mathscr{T}$の定理である.
\end{dedu}




\mathstrut
\begin{dedu}
\label{dednn}%推論21
$A$を$\mathscr{T}$の関係式とする.

1)
$\neg \neg A$が$\mathscr{T}$の定理ならば, $A$は$\mathscr{T}$の定理である.

2)
$A$が$\mathscr{T}$の定理ならば, $\neg \neg A$は$\mathscr{T}$の定理である.
\end{dedu}




\mathstrut
\begin{dedu}
\label{dedcp}%推論22
$A$と$B$を$\mathscr{T}$の関係式とする.

1)
$A \to B$が$\mathscr{T}$の定理ならば, $\neg B \to \neg A$は$\mathscr{T}$の定理である.

2)
$\neg A \to B$が$\mathscr{T}$の定理ならば, $\neg B \to A$は$\mathscr{T}$の定理である.

3)
$A \to \neg B$が$\mathscr{T}$の定理ならば, $B \to \neg A$は$\mathscr{T}$の定理である.
\end{dedu}




\mathstrut
\begin{dedu}
\label{dedaddf2}%推論23
$A$, $B$, $C$を$\mathscr{T}$の関係式とする.

1)
$\neg A \to (B \to C)$が$\mathscr{T}$の定理ならば, 
$(C \to A) \to (B \to A)$は$\mathscr{T}$の定理である.

2)
$(C \to A) \to (B \to A)$が$\mathscr{T}$の定理ならば, 
$\neg A \to (B \to C)$は$\mathscr{T}$の定理である.
\end{dedu}




\mathstrut
\begin{dedu}
\label{dedatna}%推論24
$A$を$\mathscr{T}$の関係式とする.

1)
$A \to \neg A$が$\mathscr{T}$の定理ならば, $\neg A$は$\mathscr{T}$の定理である.

2)
$\neg A \to A$が$\mathscr{T}$の定理ならば, $A$は$\mathscr{T}$の定理である.
\end{dedu}




\mathstrut
\begin{dedu}
\label{dedpeirce}%推論25
$A$と$B$を$\mathscr{T}$の関係式とする.
$(A \to B) \to A$が$\mathscr{T}$の定理ならば, $A$は$\mathscr{T}$の定理である.
\end{dedu}




\mathstrut
\begin{dedu}
\label{dedcontrue}%推論26
$A$と$B$を$\mathscr{T}$の関係式とする.
$A \to B$と$\neg A \to B$が共に$\mathscr{T}$の定理ならば, 
$B$は$\mathscr{T}$の定理である.
\end{dedu}




\mathstrut
\begin{dedu}
\label{ded1atb1tbtrue1}%推論27
$A$と$B$を$\mathscr{T}$の関係式とする.
$\neg A \to B$が$\mathscr{T}$の定理ならば, 
$(A \to B) \to B$は$\mathscr{T}$の定理である.
\end{dedu}




\mathstrut
\begin{dedu}
\label{ded1atb1tbtrue2}%推論28
$A$と$B$を$\mathscr{T}$の関係式とする.
$A$が$\mathscr{T}$の定理ならば, 
$(A \to B) \to B$は$\mathscr{T}$の定理である.
\end{dedu}




\mathstrut
\begin{dedu}
\label{dedprefalse}%推論29
$A$と$B$を$\mathscr{T}$の関係式とする.

1)
$A \to B$と$A \to \neg B$が共に$\mathscr{T}$の定理ならば, 
$\neg A$は$\mathscr{T}$の定理である.

2)
$\neg A \to B$と$\neg A \to \neg B$が共に$\mathscr{T}$の定理ならば, 
$A$は$\mathscr{T}$の定理である.
\end{dedu}




\mathstrut
\begin{dedu}
\label{dedatbfalse}%推論30
$A$と$B$を$\mathscr{T}$の関係式とする.

1)
$A$と$\neg B$が共に$\mathscr{T}$の定理ならば, 
$\neg (A \to B)$は$\mathscr{T}$の定理である.

2)
$A$と$B$が共に$\mathscr{T}$の定理ならば, 
$\neg (A \to \neg B)$は$\mathscr{T}$の定理である.
\end{dedu}




\mathstrut
\begin{dedu}
\label{deddeduction}%推論31
{\bf (演繹法則)}~
$A$を$\mathscr{T}$の関係式とし, $\mathscr{T}$の明示的公理に
$A$を追加して得られる理論を$\mathscr{T}'$とする.
$B$が$\mathscr{T}'$の定理ならば, 
$A \to B$は$\mathscr{T}$の定理である.
\end{dedu}




\mathstrut
\begin{dedu}
\label{dedaddconst}%推論32
$A$と$B$を$\mathscr{T}$の関係式, $T$を$\mathscr{T}$の対象式, $x$を文字とし, 
これらが次のa), b)を満たすとする: 

a)
$x$は$\mathscr{T}$の定数ではなく, $B$の中に自由変数として現れない.

b)
$(T|x)(A)$は$\mathscr{T}$の定理である.

$\mathscr{T}$の明示的公理に$A$を追加して得られる理論を$\mathscr{T}'$とする.
$B$が$\mathscr{T}'$の定理ならば, $B$は$\mathscr{T}$の定理である.
\end{dedu}




\mathstrut
\begin{dedu}
\label{dedraa}%推論33
{\bf (帰謬法)}~
$A$を$\mathscr{T}$の関係式とする.
$\mathscr{T}$の明示的公理に$\neg A$を追加して得られる理論を$\mathscr{T}'$とし, 
$\mathscr{T}$の明示的公理に$A$を追加して得られる理論を$\mathscr{T}''$とする.

1)
$\mathscr{T}'$が矛盾すれば, $A$は$\mathscr{T}$の定理である.

2)
$\mathscr{T}''$が矛盾すれば, $\neg A$は$\mathscr{T}$の定理である.
\end{dedu}




\mathstrut
\begin{dedu}
\label{dedvee}%推論34
$A$と$B$を$\mathscr{T}$の関係式とする.

1)
$A$が$\mathscr{T}$の定理ならば, 
$A \vee B$は$\mathscr{T}$の定理である.

2)
$B$が$\mathscr{T}$の定理ならば, 
$A \vee B$は$\mathscr{T}$の定理である.
\end{dedu}




\mathstrut
\begin{dedu}
\label{deddil}%推論35
$A$, $B$, $C$を$\mathscr{T}$の関係式とする.

1)
$A \to C$と$B \to C$が共に$\mathscr{T}$の定理ならば, 
$A \vee B \to C$は$\mathscr{T}$の定理である.

2)
$A \vee B \to C$が$\mathscr{T}$の定理ならば, 
$A \to C$と$B \to C$は共に$\mathscr{T}$の定理である.
\end{dedu}




\mathstrut
\begin{dedu}
\label{dedfromdil}%推論36
$A$, $B$, $C$を$\mathscr{T}$の関係式とする.
$A \to C$, $B \to C$, $A \vee B$がいずれも$\mathscr{T}$の定理ならば, 
$C$は$\mathscr{T}$の定理である.
\end{dedu}




\mathstrut
\begin{dedu}
\label{dedavbtbtrue1}%推論37
$A$と$B$を$\mathscr{T}$の関係式とする.

1)
$A \to B$が$\mathscr{T}$の定理ならば, 
$A \vee B \to B$は$\mathscr{T}$の定理である.

2)
$B \to A$が$\mathscr{T}$の定理ならば, 
$A \vee B \to A$は$\mathscr{T}$の定理である.
\end{dedu}




\mathstrut
\begin{dedu}
\label{dedavbtbtrue2}%推論38
$A$と$B$を$\mathscr{T}$の関係式とする.

1)
$\neg A$が$\mathscr{T}$の定理ならば, 
$A \vee B \to B$は$\mathscr{T}$の定理である.

2)
$\neg B$が$\mathscr{T}$の定理ならば, 
$A \vee B \to A$は$\mathscr{T}$の定理である.
\end{dedu}




\mathstrut
\begin{dedu}
\label{dedvjudge}%推論39
$A$と$B$を$\mathscr{T}$の関係式とし, 
$A \vee B$が$\mathscr{T}$の定理であるとする.

1)
$A \to B$が$\mathscr{T}$の定理ならば, $B$は$\mathscr{T}$の定理である.

2)
$B \to A$が$\mathscr{T}$の定理ならば, $A$は$\mathscr{T}$の定理である.

3)
$\neg A$が$\mathscr{T}$の定理ならば, $B$は$\mathscr{T}$の定理である.

4)
$\neg B$が$\mathscr{T}$の定理ならば, $A$は$\mathscr{T}$の定理である.
\end{dedu}




\mathstrut
\begin{dedu}
\label{dednv}%推論40
$A$と$B$を$\mathscr{T}$の関係式とする.

1)
$\neg A$と$\neg B$が共に$\mathscr{T}$の定理ならば, 
$\neg (A \vee B)$は$\mathscr{T}$の定理である.

2)
$\neg (A \vee B)$が$\mathscr{T}$の定理ならば, 
$\neg A$と$\neg B$は共に$\mathscr{T}$の定理である.
\end{dedu}




\mathstrut
\begin{dedu}
\label{dedt&v}%推論41
$A$, $B$, $C$を$\mathscr{T}$の関係式とする.

1)
$(A \to B) \vee C$が$\mathscr{T}$の定理ならば, 
$A \vee C \to B \vee C$と$C \vee A \to C \vee B$は共に$\mathscr{T}$の定理である.

2)
$C \vee (A \to B)$が$\mathscr{T}$の定理ならば, 
$A \vee C \to B \vee C$と$C \vee A \to C \vee B$は共に$\mathscr{T}$の定理である.

3)
$A \vee C \to B \vee C$が$\mathscr{T}$の定理ならば, 
$(A \to B) \vee C$と$C \vee (A \to B)$は共に$\mathscr{T}$の定理である.

4)
$C \vee A \to C \vee B$が$\mathscr{T}$の定理ならば, 
$(A \to B) \vee C$と$C \vee (A \to B)$は共に$\mathscr{T}$の定理である.
\end{dedu}




\mathstrut
\begin{dedu}
\label{dedaddv}%推論42
$A$, $B$, $C$を$\mathscr{T}$の関係式とする.
$A \to B$が$\mathscr{T}$の定理ならば, 
$A \vee C \to B \vee C$と$C \vee A \to C \vee B$は
共に$\mathscr{T}$の定理である.
\end{dedu}




\mathstrut
\begin{dedu}
\label{dedfromaddv}%推論43
$A$, $B$, $C$, $D$を$\mathscr{T}$の関係式とする.
$A \to B$と$C \to D$が共に$\mathscr{T}$の定理ならば, 
$A \vee C \to B \vee D$は$\mathscr{T}$の定理である.
\end{dedu}




\mathstrut
\begin{dedu}
\label{dedvidempotent}%推論44
$A$を$\mathscr{T}$の関係式とする.
$A \vee A$が$\mathscr{T}$の定理ならば, 
$A$は$\mathscr{T}$の定理である.
\end{dedu}




\mathstrut
\begin{dedu}
\label{dedvch}%推論45
$A$と$B$を$\mathscr{T}$の関係式とする.
$A \vee B$が$\mathscr{T}$の定理ならば, 
$B \vee A$は$\mathscr{T}$の定理である.
\end{dedu}




\mathstrut
\begin{dedu}
\label{dedvcomb}%推論46
$A$, $B$, $C$を$\mathscr{T}$の関係式とする.

1)
$(A \vee B) \vee C$が$\mathscr{T}$の定理ならば, 
$A \vee (B \vee C)$は$\mathscr{T}$の定理である.

2)
$A \vee (B \vee C)$が$\mathscr{T}$の定理ならば, 
$(A \vee B) \vee C$は$\mathscr{T}$の定理である.
\end{dedu}




\mathstrut
\begin{dedu}
\label{dedvdist}%推論47
$A$, $B$, $C$を$\mathscr{T}$の関係式とする.

1)
$A \vee (B \vee C)$が$\mathscr{T}$の定理ならば, 
$(A \vee B) \vee (A \vee C)$は$\mathscr{T}$の定理である.

2)
$(A \vee B) \vee C$が$\mathscr{T}$の定理ならば, 
$(A \vee C) \vee (B \vee C)$は$\mathscr{T}$の定理である.

3)
$(A \vee B) \vee (A \vee C)$が$\mathscr{T}$の定理ならば, 
$A \vee (B \vee C)$は$\mathscr{T}$の定理である.

4)
$(A \vee C) \vee (B \vee C)$が$\mathscr{T}$の定理ならば, 
$(A \vee B) \vee C$は$\mathscr{T}$の定理である.
\end{dedu}




\mathstrut
\begin{dedu}
\label{dedtveq}%推論48
$A$と$B$を$\mathscr{T}$の関係式とする.

1)
$A \to B$が$\mathscr{T}$の定理ならば, 
$\neg A \vee B$は$\mathscr{T}$の定理である.

2)
$\neg A \vee B$が$\mathscr{T}$の定理ならば, 
$A \to B$は$\mathscr{T}$の定理である.
\end{dedu}




\mathstrut
\begin{dedu}
\label{dedt&v2}%推論49
$A$, $B$, $C$を$\mathscr{T}$の関係式とする.

1)
$(A \to B) \vee C$が$\mathscr{T}$の定理ならば, 
$A \to B \vee C$は$\mathscr{T}$の定理である.

2)
$C \vee (A \to B)$が$\mathscr{T}$の定理ならば, 
$A \to C \vee B$は$\mathscr{T}$の定理である.

3)
$A \to B \vee C$が$\mathscr{T}$の定理ならば, 
$(A \to B) \vee C$は$\mathscr{T}$の定理である.

4)
$A \to C \vee B$が$\mathscr{T}$の定理ならば, 
$C \vee (A \to B)$は$\mathscr{T}$の定理である.
\end{dedu}




\mathstrut
\begin{dedu}
\label{dedt&v3}%推論50
$A$, $B$, $C$を$\mathscr{T}$の関係式とする.

1)
$A \to B \vee C$が$\mathscr{T}$の定理ならば, 
$A \vee C \to B \vee C$は$\mathscr{T}$の定理である.

2)
$A \to C \vee B$が$\mathscr{T}$の定理ならば, 
$C \vee A \to C \vee B$は$\mathscr{T}$の定理である.

3)
$A \vee C \to B \vee C$が$\mathscr{T}$の定理ならば, 
$A \to B \vee C$は$\mathscr{T}$の定理である.

4)
$C \vee A \to C \vee B$が$\mathscr{T}$の定理ならば, 
$A \to C \vee B$は$\mathscr{T}$の定理である.
\end{dedu}




\mathstrut
\begin{dedu}
\label{dedtvsep}%推論51
$A$, $B$, $C$を$\mathscr{T}$の関係式とする.

1)
$A \to B \vee C$が$\mathscr{T}$の定理ならば, 
$(A \to B) \vee (A \to C)$は$\mathscr{T}$の定理である.

2)
$(A \to B) \vee (A \to C)$が$\mathscr{T}$の定理ならば, 
$A \to B \vee C$は$\mathscr{T}$の定理である.
\end{dedu}




\mathstrut
\begin{dedu}
\label{dedttv}%推論52
$A$と$B$を$\mathscr{T}$の関係式とする.

1)
$(A \to B) \to B$が$\mathscr{T}$の定理ならば, 
$A \vee B$は$\mathscr{T}$の定理である.

2)
$A \vee B$が$\mathscr{T}$の定理ならば, 
$(A \to B) \to B$は$\mathscr{T}$の定理である.
\end{dedu}




\mathstrut
\begin{dedu}
\label{dedwedge}%推論53
$A$と$B$を$\mathscr{T}$の関係式とする.

1)
$A$と$B$が共に$\mathscr{T}$の定理ならば, 
$A \wedge B$は$\mathscr{T}$の定理である.

2)
$A \wedge B$が$\mathscr{T}$の定理ならば, 
$A$と$B$は共に$\mathscr{T}$の定理である.
\end{dedu}




\mathstrut
\begin{dedu}
\label{dedprewedge}%推論54
$A$, $B$, $C$を$\mathscr{T}$の関係式とする.

1)
$C \to A$と$C \to B$が共に$\mathscr{T}$の定理ならば, 
$C \to A \wedge B$は$\mathscr{T}$の定理である.

2)
$C \to A \wedge B$が$\mathscr{T}$の定理ならば, 
$C \to A$と$C \to B$は共に$\mathscr{T}$の定理である.
\end{dedu}




\mathstrut
\begin{dedu}
\label{dedatawbtrue1}%推論55
$A$と$B$を$\mathscr{T}$の関係式とする.

1)
$A \to B$が$\mathscr{T}$の定理ならば, 
$A \to A \wedge B$は$\mathscr{T}$の定理である.

2)
$B \to A$が$\mathscr{T}$の定理ならば, 
$B \to A \wedge B$は$\mathscr{T}$の定理である.
\end{dedu}




\mathstrut
\begin{dedu}
\label{dedatawbtrue2}%推論56
$A$と$B$を$\mathscr{T}$の関係式とする.

1)
$B$が$\mathscr{T}$の定理ならば, 
$A \to A \wedge B$は$\mathscr{T}$の定理である.

2)
$A$が$\mathscr{T}$の定理ならば, 
$B \to A \wedge B$は$\mathscr{T}$の定理である.
\end{dedu}




\mathstrut
\begin{dedu}
\label{dednw}%推論57
$A$と$B$を$\mathscr{T}$の関係式とする.

1)
$\neg A$が$\mathscr{T}$の定理ならば, $\neg (A \wedge B)$は$\mathscr{T}$の定理である.

2)
$\neg B$が$\mathscr{T}$の定理ならば, $\neg (A \wedge B)$は$\mathscr{T}$の定理である.
\end{dedu}




\mathstrut
\begin{dedu}
\label{dedt&w}%推論58
$A$, $B$, $C$を$\mathscr{T}$の関係式とする.

1)
$C \to (A \to B)$が$\mathscr{T}$の定理ならば, 
$A \wedge C \to B \wedge C$と
$C \wedge A \to C \wedge B$は共に$\mathscr{T}$の定理である.

2)
$A \wedge C \to B \wedge C$が$\mathscr{T}$の定理ならば, 
$C \to (A \to B)$は$\mathscr{T}$の定理である.

3)
$C \wedge A \to C \wedge B$が$\mathscr{T}$の定理ならば, 
$C \to (A \to B)$は$\mathscr{T}$の定理である.
\end{dedu}




\mathstrut
\begin{dedu}
\label{dedaddw}%推論59
$A$, $B$, $C$を$\mathscr{T}$の関係式とする.
$A \to B$が$\mathscr{T}$の定理ならば, 
$A \wedge C \to B \wedge C$と$C \wedge A \to C \wedge B$は
共に$\mathscr{T}$の定理である.
\end{dedu}




\mathstrut
\begin{dedu}
\label{dedfromaddw}%推論60
$A$, $B$, $C$, $D$を$\mathscr{T}$の関係式とする.
$A \to B$と$C \to D$が共に$\mathscr{T}$の定理ならば, 
$A \wedge C \to B \wedge D$は$\mathscr{T}$の定理である.
\end{dedu}




\mathstrut
\begin{dedu}
\label{dedwidempotent}%推論61
$A$が$\mathscr{T}$の定理ならば, $A \wedge A$は$\mathscr{T}$の定理である.
\end{dedu}




\mathstrut
\begin{dedu}
\label{dedwch}%推論62
$A$と$B$を$\mathscr{T}$の関係式とする.
$A \wedge B$が$\mathscr{T}$の定理ならば, 
$B \wedge A$は$\mathscr{T}$の定理である.
\end{dedu}




\mathstrut
\begin{dedu}
\label{dedwcomb}%推論63
$A$, $B$, $C$を$\mathscr{T}$の関係式とする.

1)
$(A \wedge B) \wedge C$が$\mathscr{T}$の定理ならば, 
$A \wedge (B \wedge C)$は$\mathscr{T}$の定理である.

2)
$A \wedge (B \wedge C)$が$\mathscr{T}$の定理ならば, 
$(A \wedge B) \wedge C$は$\mathscr{T}$の定理である.
\end{dedu}




\mathstrut
\begin{dedu}
\label{dedwdist}%推論64
$A$, $B$, $C$を$\mathscr{T}$の関係式とする.

1)
$A \wedge (B \wedge C)$が$\mathscr{T}$の定理ならば, 
$(A \wedge B) \wedge (A \wedge C)$は$\mathscr{T}$の定理である.

2)
$(A \wedge B) \wedge C$が$\mathscr{T}$の定理ならば, 
$(A \wedge C) \wedge (B \wedge C)$は$\mathscr{T}$の定理である.

3)
$(A \wedge B) \wedge (A \wedge C)$が$\mathscr{T}$の定理ならば, 
$A \wedge (B \wedge C)$は$\mathscr{T}$の定理である.

4)
$(A \wedge C) \wedge (B \wedge C)$が$\mathscr{T}$の定理ならば, 
$(A \wedge B) \wedge C$は$\mathscr{T}$の定理である.
\end{dedu}




\mathstrut
\begin{dedu}
\label{dedtweq}%推論65
$A$と$B$を$\mathscr{T}$の関係式とする.

1)
$\neg (A \to B)$が$\mathscr{T}$の定理ならば, 
$A \wedge \neg B$は$\mathscr{T}$の定理である.

2)
$A \wedge \neg B$が$\mathscr{T}$の定理ならば, 
$\neg (A \to B)$は$\mathscr{T}$の定理である.
\end{dedu}




\mathstrut
\begin{dedu}
\label{dedtwch}%推論66
$A$, $B$, $C$を$\mathscr{T}$の関係式とする.

1)
$A \to (B \to C)$が$\mathscr{T}$の定理ならば, 
$A \wedge B \to C$は$\mathscr{T}$の定理である.

2)
$A \wedge B \to C$が$\mathscr{T}$の定理ならば, 
$A \to (B \to C)$は$\mathscr{T}$の定理である.
\end{dedu}




\mathstrut
\begin{dedu}
\label{dedt&w2}%推論67
$A$, $B$, $C$を$\mathscr{T}$の関係式とする.

1)
$A \wedge C \to B$が$\mathscr{T}$の定理ならば, 
$A \wedge C \to B \wedge C$は$\mathscr{T}$の定理である.

2)
$C \wedge A \to B$が$\mathscr{T}$の定理ならば, 
$C \wedge A \to C \wedge B$は$\mathscr{T}$の定理である.
\end{dedu}




\mathstrut
\begin{dedu}
\label{dedtwsep}%推論68
$A$, $B$, $C$を$\mathscr{T}$の関係式とする.

1)
$A \to B \wedge C$が$\mathscr{T}$の定理ならば, 
$(A \to B) \wedge (A \to C)$は$\mathscr{T}$の定理である.

2)
$(A \to B) \wedge (A \to C)$が$\mathscr{T}$の定理ならば, 
$A \to B \wedge C$は$\mathscr{T}$の定理である.
\end{dedu}




\mathstrut
\begin{dedu}
\label{dedwlemma}%推論69
$A$, $B$, $C$を$\mathscr{T}$の関係式とする.
$A \wedge (B \to C)$が$\mathscr{T}$の定理ならば, 
$B \to A \wedge C$は$\mathscr{T}$の定理である.
\end{dedu}




\mathstrut
\begin{dedu}
\label{deddemorgan}%推論70
$A$と$B$を$\mathscr{T}$の関係式とする.

1)
$\neg (A \wedge B)$が$\mathscr{T}$の定理ならば, 
$\neg A \vee \neg B$は$\mathscr{T}$の定理である.

2)
$\neg A \vee \neg B$が$\mathscr{T}$の定理ならば, 
$\neg (A \wedge B)$は$\mathscr{T}$の定理である.

3)
$\neg (A \vee B)$が$\mathscr{T}$の定理ならば, 
$\neg A \wedge \neg B$は$\mathscr{T}$の定理である.

4)
$\neg A \wedge \neg B$が$\mathscr{T}$の定理ならば, 
$\neg (A \vee B)$は$\mathscr{T}$の定理である.
\end{dedu}




\mathstrut
\begin{dedu}
\label{dedabsvw}%推論71
$A$と$B$を$\mathscr{T}$の関係式とする.
$A$が$\mathscr{T}$の定理ならば, 
$(A \vee B) \wedge A$は$\mathscr{T}$の定理である.
\end{dedu}




\mathstrut
\begin{dedu}
\label{dedabswv}%推論72
$A$と$B$を$\mathscr{T}$の関係式とする.
$(A \wedge B) \vee A$が$\mathscr{T}$の定理ならば, 
$A$は$\mathscr{T}$の定理である.
\end{dedu}




\mathstrut
\begin{dedu}
\label{dedtvwsep}%推論73
$A$, $B$, $C$を$\mathscr{T}$の関係式とする.

1)
$A \vee B \to C$が$\mathscr{T}$の定理ならば, 
$(A \to C) \wedge (B \to C)$は$\mathscr{T}$の定理である.

2)
$(A \to C) \wedge (B \to C)$が$\mathscr{T}$の定理ならば, 
$A \vee B \to C$は$\mathscr{T}$の定理である.

3)
$A \wedge B \to C$が$\mathscr{T}$の定理ならば, 
$(A \to C) \vee (B \to C)$は$\mathscr{T}$の定理である.

4)
$(A \to C) \vee (B \to C)$が$\mathscr{T}$の定理ならば, 
$A \wedge B \to C$は$\mathscr{T}$の定理である.
\end{dedu}




\mathstrut
\begin{dedu}
\label{deddist}%推論74
$A$, $B$, $C$を$\mathscr{T}$の関係式とする.

1)
$A \wedge (B \vee C)$が$\mathscr{T}$の定理ならば, 
$(A \wedge B) \vee (A \wedge C)$は$\mathscr{T}$の定理である.

2)
$(A \vee B) \wedge C$が$\mathscr{T}$の定理ならば, 
$(A \wedge C) \vee (B \wedge C)$は$\mathscr{T}$の定理である.

3)
$(A \wedge B) \vee (A \wedge C)$が$\mathscr{T}$の定理ならば, 
$A \wedge (B \vee C)$は$\mathscr{T}$の定理である.

4)
$(A \wedge C) \vee (B \wedge C)$が$\mathscr{T}$の定理ならば, 
$(A \vee B) \wedge C$は$\mathscr{T}$の定理である.

5)
$A \vee (B \wedge C)$が$\mathscr{T}$の定理ならば, 
$(A \vee B) \wedge (A \vee C)$は$\mathscr{T}$の定理である.

6)
$(A \wedge B) \vee C$が$\mathscr{T}$の定理ならば, 
$(A \vee C) \wedge (B \vee C)$は$\mathscr{T}$の定理である.

7)
$(A \vee B) \wedge (A \vee C)$が$\mathscr{T}$の定理ならば, 
$A \vee (B \wedge C)$は$\mathscr{T}$の定理である.

8)
$(A \vee C) \wedge (B \vee C)$が$\mathscr{T}$の定理ならば, 
$(A \wedge B) \vee C$は$\mathscr{T}$の定理である.
\end{dedu}




\mathstrut
\begin{dedu}
\label{dedgvee}%推論75
$n$を自然数とし, $A_{1}, A_{2}, \cdots, A_{n}$を$\mathscr{T}$の関係式とする.
また$i$を$n$以下の自然数とする.
$A_{i}$が$\mathscr{T}$の定理ならば, 
$A_{1} \vee A_{2} \vee \cdots \vee A_{n}$は$\mathscr{T}$の定理である.
\end{dedu}




\mathstrut
\begin{dedu}
\label{dedgdil}%推論76
$n$を自然数とし, $A_{1}, A_{2}, \cdots, A_{n}$を$\mathscr{T}$の関係式とする.
また$B$を$\mathscr{T}$の関係式とする.

1)
$A_{1} \to B, A_{2} \to B, \cdots, A_{n} \to B$がすべて$\mathscr{T}$の定理ならば, 
$A_{1} \vee A_{2} \vee \cdots \vee A_{n} \to B$は$\mathscr{T}$の定理である.

2)
$A_{1} \vee A_{2} \vee \cdots \vee A_{n} \to B$が$\mathscr{T}$の定理ならば, 
$A_{1} \to B, A_{2} \to B, \cdots, A_{n} \to B$はすべて$\mathscr{T}$の定理である.
\end{dedu}




\mathstrut
\begin{dedu}
\label{dedfromgdil}%推論77
$n$を自然数とし, $A_{1}, A_{2}, \cdots, A_{n}$を$\mathscr{T}$の関係式とする.
また$B$を$\mathscr{T}$の関係式とする.
$A_{1} \to B, A_{2} \to B, \cdots, A_{n} \to B$及び
$A_{1} \vee A_{2} \vee \cdots \vee A_{n}$がすべて$\mathscr{T}$の定理ならば, 
$B$は$\mathscr{T}$の定理である.
\end{dedu}




\mathstrut
\begin{dedu}
\label{dedgvee2}%推論78
$n$を自然数とし, $A_{1}, A_{2}, \cdots, A_{n}$を$\mathscr{T}$の関係式とする.
また$k$を自然数とし, $i_{1}, i_{2}, \cdots, i_{k}$を$n$以下の自然数とする.
$A_{i_{1}} \vee A_{i_{2}} \vee \cdots \vee A_{i_{k}}$が$\mathscr{T}$の定理ならば, 
$A_{1} \vee A_{2} \vee \cdots \vee A_{n}$は$\mathscr{T}$の定理である.
\end{dedu}




\mathstrut
\begin{dedu}
\label{dedgvgw}%推論79
$n$を自然数とし, $A_{1}, A_{2}, \cdots, A_{n}$を$\mathscr{T}$の関係式とする.

1)
$A_{1} \wedge A_{2} \wedge \cdots \wedge A_{n}$が$\mathscr{T}$の定理ならば, 
$\neg (\neg A_{1} \vee \neg A_{2} \vee \cdots \vee \neg A_{n})$は$\mathscr{T}$の定理である.

2)
$\neg (\neg A_{1} \vee \neg A_{2} \vee \cdots \vee \neg A_{n})$が$\mathscr{T}$の定理ならば, 
$A_{1} \wedge A_{2} \wedge \cdots \wedge A_{n}$は$\mathscr{T}$の定理である.
\end{dedu}




\mathstrut
\begin{dedu}
\label{dedgwedge}%推論80
$n$を自然数とし, $A_{1}, A_{2}, \cdots, A_{n}$を$\mathscr{T}$の関係式とする.

1)
$A_{1}, A_{2}, \cdots, A_{n}$がすべて$\mathscr{T}$の定理ならば, 
$A_{1} \wedge A_{2} \wedge \cdots \wedge A_{n}$は$\mathscr{T}$の定理である.

2)
$A_{1} \wedge A_{2} \wedge \cdots \wedge A_{n}$が$\mathscr{T}$の定理ならば, 
$A_{1}, A_{2}, \cdots, A_{n}$はすべて$\mathscr{T}$の定理である.
\end{dedu}




\mathstrut
\begin{dedu}
\label{dedpregwedge}%推論81
$n$を自然数とし, $A_{1}, A_{2}, \cdots, A_{n}$を$\mathscr{T}$の関係式とする.
また$B$を$\mathscr{T}$の関係式とする.

1)
$B \to A_{1}, B \to A_{2}, \cdots, B \to A_{n}$がすべて$\mathscr{T}$の定理ならば, 
$B \to A_{1} \wedge A_{2} \wedge \cdots \wedge A_{n}$は$\mathscr{T}$の定理である.

2)
$B \to A_{1} \wedge A_{2} \wedge \cdots \wedge A_{n}$が$\mathscr{T}$の定理ならば, 
$B \to A_{1}, B \to A_{2}, \cdots, B \to A_{n}$はすべて$\mathscr{T}$の定理である.
\end{dedu}




\mathstrut
\begin{dedu}
\label{dedgwedge2}%推論82
$n$を自然数とし, $A_{1}, A_{2}, \cdots, A_{n}$を$\mathscr{T}$の関係式とする.
また$k$を自然数とし, $i_{1}, i_{2}, \cdots, i_{k}$を$n$以下の自然数とする.
$A_{1} \wedge A_{2} \wedge \cdots \wedge A_{n}$が$\mathscr{T}$の定理ならば, 
$A_{i_{1}} \wedge A_{i_{2}} \wedge \cdots \wedge A_{i_{k}}$は$\mathscr{T}$の定理である.
\end{dedu}




\mathstrut
\begin{dedu}
\label{dedgvtrue}%推論83
$n$を自然数とし, $A_{1}, A_{2}, \cdots, A_{n}$を$\mathscr{T}$の関係式とする.
また$k$を自然数とし, $i_{1}, i_{2}, \cdots, i_{k}$を$n$以下の自然数とする.
いま$i_{1}, i_{2}, \cdots, i_{k}$のいずれとも異なるような$n$以下の各自然数$i$ごとに, 
次のa), b), c)のいずれかが成立するとする: 

\noindent
a)
$A_{i} \to A_{i_{1}} \vee A_{i_{2}} \vee \cdots \vee A_{i_{k}}$は$\mathscr{T}$の定理である.

\noindent
b)
$A_{i} \to A_{i_{1}}, A_{i} \to A_{i_{2}}, \cdots, A_{i} \to A_{i_{k}}$の
うち少なくとも一つは$\mathscr{T}$の定理である.

\noindent
c)
$\neg A_{i}$は$\mathscr{T}$の定理である.

\noindent
このとき, 
\[
  A_{1} \vee A_{2} \vee \cdots \vee A_{n} 
  \to A_{i_{1}} \vee A_{i_{2}} \vee \cdots \vee A_{i_{k}}
\]
は$\mathscr{T}$の定理である.
\end{dedu}




\mathstrut
\begin{dedu}
\label{dedfromgvtrue}%推論84
$n$を自然数, $A_{1}, A_{2}, \cdots, A_{n}$を$\mathscr{T}$の関係式とし, 
$A_{1} \vee A_{2} \vee \cdots \vee A_{n}$が$\mathscr{T}$の定理であるとする.
また$k$を自然数とし, $i_{1}, i_{2}, \cdots, i_{k}$を$n$以下の自然数とする.
いま$i_{1}, i_{2}, \cdots, i_{k}$のいずれとも異なるような$n$以下の各自然数$i$ごとに, 
次のa), b), c)のいずれかが成立するとする: 

\noindent
a)
$A_{i} \to A_{i_{1}} \vee A_{i_{2}} \vee \cdots \vee A_{i_{k}}$は$\mathscr{T}$の定理である.

\noindent
b)
$A_{i} \to A_{i_{1}}, A_{i} \to A_{i_{2}}, \cdots, A_{i} \to A_{i_{k}}$の
うち少なくとも一つは$\mathscr{T}$の定理である.

\noindent
c)
$\neg A_{i}$は$\mathscr{T}$の定理である.

\noindent
このとき, $A_{i_{1}} \vee A_{i_{2}} \vee \cdots \vee A_{i_{k}}$は$\mathscr{T}$の定理である.
\end{dedu}




\mathstrut
\begin{dedu}
\label{dedgwtrue}%推論85
$n$を自然数とし, $A_{1}, A_{2}, \cdots, A_{n}$を$\mathscr{T}$の関係式とする.
また$k$を自然数とし, $i_{1}, i_{2}, \cdots, i_{k}$を$n$以下の自然数とする.
いま$i_{1}, i_{2}, \cdots, i_{k}$のいずれとも異なるような$n$以下の各自然数$i$ごとに, 
次のa), b), c)のいずれかが成立するとする: 

\noindent
a)
$A_{i_{1}} \wedge A_{i_{2}} \wedge \cdots \wedge A_{i_{k}} \to A_{i}$は$\mathscr{T}$の定理である.

\noindent
b)
$A_{i_{1}} \to A_{i}, A_{i_{2}} \to A_{i}, \cdots, A_{i_{k}} \to A_{i}$の
うち少なくとも一つは$\mathscr{T}$の定理である.

\noindent
c)
$A_{i}$は$\mathscr{T}$の定理である.

\noindent
このとき, 
\[
  A_{i_{1}} \wedge A_{i_{2}} \wedge \cdots \wedge A_{i_{k}} 
  \to A_{1} \wedge A_{2} \wedge \cdots \wedge A_{n}
\]
は$\mathscr{T}$の定理である.
\end{dedu}




\mathstrut
\begin{dedu}
\label{dedfromgwtrue}%推論86
$n$を自然数とし, $A_{1}, A_{2}, \cdots, A_{n}$を$\mathscr{T}$の関係式とする.
また$k$を自然数, $i_{1}, i_{2}, \cdots, i_{k}$を$n$以下の自然数とし, 
$A_{i_{1}} \wedge A_{i_{2}} \wedge \cdots \wedge A_{i_{k}}$が$\mathscr{T}$の定理であるとする.
いま$i_{1}, i_{2}, \cdots, i_{k}$のいずれとも異なるような$n$以下の各自然数$i$ごとに, 
次のa), b), c)のいずれかが成立するとする: 

\noindent
a)
$A_{i_{1}} \wedge A_{i_{2}} \wedge \cdots \wedge A_{i_{k}} \to A_{i}$は$\mathscr{T}$の定理である.

\noindent
b)
$A_{i_{1}} \to A_{i}, A_{i_{2}} \to A_{i}, \cdots, A_{i_{k}} \to A_{i}$の
うち少なくとも一つは$\mathscr{T}$の定理である.

\noindent
c)
$A_{i}$は$\mathscr{T}$の定理である.

\noindent
このとき, $A_{1} \wedge A_{2} \wedge \cdots \wedge A_{n}$は$\mathscr{T}$の定理である.
\end{dedu}




\mathstrut
\begin{dedu}
\label{dedngv}%推論87
$n$を自然数とし, $A_{1}, A_{2}, \cdots, A_{n}$を$\mathscr{T}$の関係式とする.

1)
$\neg A_{1}, \neg A_{2}, \cdots, \neg A_{n}$がすべて$\mathscr{T}$の定理ならば, 
$\neg (A_{1} \vee A_{2} \vee \cdots \vee A_{n})$は$\mathscr{T}$の定理である.

2)
$\neg (A_{1} \vee A_{2} \vee \cdots \vee A_{n})$が$\mathscr{T}$の定理ならば, 
$\neg A_{1}, \neg A_{2}, \cdots, \neg A_{n}$はすべて$\mathscr{T}$の定理である.
\end{dedu}




\mathstrut
\begin{dedu}
\label{dedngw}%推論88
$n$を自然数とし, $A_{1}, A_{2}, \cdots, A_{n}$を$\mathscr{T}$の関係式とする.
また$i$を$n$以下の自然数とし, $\neg A_{i}$が$\mathscr{T}$の定理であるとする.
このとき$\neg (A_{1} \wedge A_{2} \wedge \cdots \wedge A_{n})$は$\mathscr{T}$の定理である.
\end{dedu}




\mathstrut
\begin{dedu}
\label{dedfromaddgv}%推論89
$n$を自然数とし, $A_{1}, A_{2}, \cdots, A_{n}, B_{1}, B_{2}, \cdots, B_{n}$を$\mathscr{T}$の関係式とする.
$A_{1} \to B_{1}, A_{2} \to B_{2}, \cdots, A_{n} \to B_{n}$がすべて$\mathscr{T}$の定理ならば, 
$A_{1} \vee A_{2} \vee \cdots \vee A_{n} \to B_{1} \vee B_{2} \vee \cdots \vee B_{n}$は
$\mathscr{T}$の定理である.
\end{dedu}




\mathstrut
\begin{dedu}
\label{dedaddgv}%推論90
$n$を自然数とし, $A_{1}, A_{2}, \cdots, A_{n}, B_{1}, B_{2}, \cdots, B_{n}$を$\mathscr{T}$の関係式とする.
いま$n$以下の各自然数$i$に対し, $B_{i}$は$A_{i}$であるか, 
または$A_{i} \to B_{i}$が$\mathscr{T}$の定理であるとする.
このとき
$A_{1} \vee A_{2} \vee \cdots \vee A_{n} \to B_{1} \vee B_{2} \vee \cdots \vee B_{n}$は
$\mathscr{T}$の定理である.
\end{dedu}




\mathstrut
\begin{dedu}
\label{dedfromaddgw}%推論91
$n$を自然数とし, $A_{1}, A_{2}, \cdots, A_{n}, B_{1}, B_{2}, \cdots, B_{n}$を$\mathscr{T}$の関係式とする.
$A_{1} \to B_{1}, A_{2} \to B_{2}, \cdots, A_{n} \to B_{n}$がすべて$\mathscr{T}$の定理ならば, 
$A_{1} \wedge A_{2} \wedge \cdots \wedge A_{n} \to B_{1} \wedge B_{2} \wedge \cdots \wedge B_{n}$は
$\mathscr{T}$の定理である.
\end{dedu}




\mathstrut
\begin{dedu}
\label{dedaddgw}%推論92
$n$を自然数とし, $A_{1}, A_{2}, \cdots, A_{n}, B_{1}, B_{2}, \cdots, B_{n}$を$\mathscr{T}$の関係式とする.
いま$n$以下の各自然数$i$に対し, $B_{i}$は$A_{i}$であるか, 
または$A_{i} \to B_{i}$が$\mathscr{T}$の定理であるとする.
このとき
$A_{1} \wedge A_{2} \wedge \cdots \wedge A_{n} \to B_{1} \wedge B_{2} \wedge \cdots \wedge B_{n}$は
$\mathscr{T}$の定理である.
\end{dedu}




\mathstrut
\begin{dedu}
\label{dedgvidempotent}%推論93
$A$を$\mathscr{T}$の関係式とする.
$A \vee A \vee \cdots \vee A$が$\mathscr{T}$の定理ならば, 
$A$は$\mathscr{T}$の定理である.
但しここで$A \vee A \vee \cdots \vee A$における$A$の個数は任意とする.
\end{dedu}




\mathstrut
\begin{dedu}
\label{dedgwidempotent}%推論94
$A$が$\mathscr{T}$の定理ならば, 
$A \wedge A \wedge \cdots \wedge A$は$\mathscr{T}$の定理である.
但しここで$A \wedge A \wedge \cdots \wedge A$における$A$の個数は任意とする.
\end{dedu}




\mathstrut
\begin{dedu}
\label{dedgvch}%推論95
$n$を自然数とし, $A_{1}, A_{2}, \cdots, A_{n}$を$\mathscr{T}$の関係式とする.
また自然数$1, 2, \cdots, n$の順序を任意に入れ替えたものを
$i_{1}, i_{2}, \cdots, i_{n}$とする.
$A_{1} \vee A_{2} \vee \cdots \vee A_{n}$が$\mathscr{T}$の定理ならば, 
$A_{i_{1}} \vee A_{i_{2}} \vee \cdots \vee A_{i_{n}}$は$\mathscr{T}$の定理である.
\end{dedu}




\mathstrut
\begin{dedu}
\label{dedgwch}%推論96
$n$を自然数とし, $A_{1}, A_{2}, \cdots, A_{n}$を$\mathscr{T}$の関係式とする.
また自然数$1, 2, \cdots, n$の順序を任意に入れ替えたものを
$i_{1}, i_{2}, \cdots, i_{n}$とする.
$A_{1} \wedge A_{2} \wedge \cdots \wedge A_{n}$が$\mathscr{T}$の定理ならば, 
$A_{i_{1}} \wedge A_{i_{2}} \wedge \cdots \wedge A_{i_{n}}$は$\mathscr{T}$の定理である.
\end{dedu}




\mathstrut
\begin{dedu}
\label{dedgvass}%推論97
$n$を自然数とし, $A_{1}, A_{2}, \cdots, A_{n}$を$\mathscr{T}$の関係式とする.
また$k$を$k < n$なる自然数とし, $i_{1}, i_{2}, \cdots, i_{k}$を
$i_{1} < i_{2} < \cdots < i_{k} < n$なる自然数とする.
同様に, $l$を$l < n$なる自然数とし, $j_{1}, j_{2}, \cdots, j_{l}$を
$j_{1} < j_{2} < \cdots < j_{l} < n$なる自然数とする.
このとき, 
\[
  (A_{1} \vee \cdots \vee A_{i_{1}}) \vee (A_{i_{1} + 1} \vee \cdots \vee A_{i_{2}}) \vee \cdots\cdots \vee (A_{i_{k} + 1} \vee \cdots \vee A_{n})
\]
が$\mathscr{T}$の定理ならば, 
\[
  (A_{1} \vee \cdots \vee A_{j_{1}}) \vee (A_{j_{1} + 1} \vee \cdots \vee A_{j_{2}}) \vee \cdots\cdots \vee (A_{j_{l} + 1} \vee \cdots \vee A_{n})
\]
は$\mathscr{T}$の定理である.
\end{dedu}




\mathstrut
\begin{dedu}
\label{dedgwass}%推論98
$n$を自然数とし, $A_{1}, A_{2}, \cdots, A_{n}$を$\mathscr{T}$の関係式とする.
また$k$を$k < n$なる自然数とし, $i_{1}, i_{2}, \cdots, i_{k}$を
$i_{1} < i_{2} < \cdots < i_{k} < n$なる自然数とする.
同様に, $l$を$l < n$なる自然数とし, $j_{1}, j_{2}, \cdots, j_{l}$を
$j_{1} < j_{2} < \cdots < j_{l} < n$なる自然数とする.
このとき, 
\[
  (A_{1} \wedge \cdots \wedge A_{i_{1}}) \wedge (A_{i_{1} + 1} \wedge \cdots \wedge A_{i_{2}}) \wedge \cdots\cdots \wedge (A_{i_{k} + 1} \wedge \cdots \wedge A_{n})
\]
が$\mathscr{T}$の定理ならば, 
\[
  (A_{1} \wedge \cdots \wedge A_{j_{1}}) \wedge (A_{j_{1} + 1} \wedge \cdots \wedge A_{j_{2}}) \wedge \cdots\cdots \wedge (A_{j_{l} + 1} \wedge \cdots \wedge A_{n})
\]
は$\mathscr{T}$の定理である.
\end{dedu}




\mathstrut
\begin{dedu}
\label{dedgvdist}%推論99
$n$を自然数とし, $A_{1}, A_{2}, \cdots, A_{n}$を$\mathscr{T}$の関係式とする.
また$B$を$\mathscr{T}$の関係式とする.

1)
$B \vee (A_{1} \vee A_{2} \vee \cdots \vee A_{n})$が$\mathscr{T}$の定理ならば, 
$(B \vee A_{1}) \vee (B \vee A_{2}) \vee \cdots \vee (B \vee A_{n})$は$\mathscr{T}$の定理である.

2)
$(A_{1} \vee A_{2} \vee \cdots \vee A_{n}) \vee B$が$\mathscr{T}$の定理ならば, 
$(A_{1} \vee B) \vee (A_{2} \vee B) \vee \cdots \vee (A_{n} \vee B)$は$\mathscr{T}$の定理である.

3)
$(B \vee A_{1}) \vee (B \vee A_{2}) \vee \cdots \vee (B \vee A_{n})$が$\mathscr{T}$の定理ならば, 
$B \vee (A_{1} \vee A_{2} \vee \cdots \vee A_{n})$は$\mathscr{T}$の定理である.

4)
$(A_{1} \vee B) \vee (A_{2} \vee B) \vee \cdots \vee (A_{n} \vee B)$が$\mathscr{T}$の定理ならば, 
$(A_{1} \vee A_{2} \vee \cdots \vee A_{n}) \vee B$は$\mathscr{T}$の定理である.
\end{dedu}




\mathstrut
\begin{dedu}
\label{dedgwdist}%推論100
$n$を自然数とし, $A_{1}, A_{2}, \cdots, A_{n}$を$\mathscr{T}$の関係式とする.
また$B$を$\mathscr{T}$の関係式とする.

1)
$B \wedge (A_{1} \wedge A_{2} \wedge \cdots \wedge A_{n})$が$\mathscr{T}$の定理ならば, 
$(B \wedge A_{1}) \wedge (B \wedge A_{2}) \wedge \cdots \wedge (B \wedge A_{n})$は$\mathscr{T}$の定理である.

2)
$(A_{1} \wedge A_{2} \wedge \cdots \wedge A_{n}) \wedge B$が$\mathscr{T}$の定理ならば, 
$(A_{1} \wedge B) \wedge (A_{2} \wedge B) \wedge \cdots \wedge (A_{n} \wedge B)$は$\mathscr{T}$の定理である.

3)
$(B \wedge A_{1}) \wedge (B \wedge A_{2}) \wedge \cdots \wedge (B \wedge A_{n})$が$\mathscr{T}$の定理ならば, 
$B \wedge (A_{1} \wedge A_{2} \wedge \cdots \wedge A_{n})$は$\mathscr{T}$の定理である.

4)
$(A_{1} \wedge B) \wedge (A_{2} \wedge B) \wedge \cdots \wedge (A_{n} \wedge B)$が$\mathscr{T}$の定理ならば, 
$(A_{1} \wedge A_{2} \wedge \cdots \wedge A_{n}) \wedge B$は$\mathscr{T}$の定理である.
\end{dedu}




\mathstrut
\begin{dedu}
\label{dedgdist}%推論101
$n$を自然数とし, $A_{1}, A_{2}, \cdots, A_{n}$を$\mathscr{T}$の関係式とする.
また$B$を$\mathscr{T}$の関係式とする.

1)
$B \wedge (A_{1} \vee A_{2} \vee \cdots \vee A_{n})$が$\mathscr{T}$の定理ならば, 
$(B \wedge A_{1}) \vee (B \wedge A_{2}) \vee \cdots \vee (B \wedge A_{n})$は$\mathscr{T}$の定理である.

2)
$(A_{1} \vee A_{2} \vee \cdots \vee A_{n}) \wedge B$が$\mathscr{T}$の定理ならば, 
$(A_{1} \wedge B) \vee (A_{2} \wedge B) \vee \cdots \vee (A_{n} \wedge B)$は$\mathscr{T}$の定理である.

3)
$(B \wedge A_{1}) \vee (B \wedge A_{2}) \vee \cdots \vee (B \wedge A_{n})$が$\mathscr{T}$の定理ならば, 
$B \wedge (A_{1} \vee A_{2} \vee \cdots \vee A_{n})$は$\mathscr{T}$の定理である.

4)
$(A_{1} \wedge B) \vee (A_{2} \wedge B) \vee \cdots \vee (A_{n} \wedge B)$が$\mathscr{T}$の定理ならば, 
$(A_{1} \vee A_{2} \vee \cdots \vee A_{n}) \wedge B$は$\mathscr{T}$の定理である.

5)
$B \vee (A_{1} \wedge A_{2} \wedge \cdots \wedge A_{n})$が$\mathscr{T}$の定理ならば, 
$(B \vee A_{1}) \wedge (B \vee A_{2}) \wedge \cdots \wedge (B \vee A_{n})$は$\mathscr{T}$の定理である.

6)
$(A_{1} \wedge A_{2} \wedge \cdots \wedge A_{n}) \vee B$が$\mathscr{T}$の定理ならば, 
$(A_{1} \vee B) \wedge (A_{2} \vee B) \wedge \cdots \wedge (A_{n} \vee B)$は$\mathscr{T}$の定理である.

7)
$(B \vee A_{1}) \wedge (B \vee A_{2}) \wedge \cdots \wedge (B \vee A_{n})$が$\mathscr{T}$の定理ならば, 
$B \vee (A_{1} \wedge A_{2} \wedge \cdots \wedge A_{n})$は$\mathscr{T}$の定理である.

8)
$(A_{1} \vee B) \wedge (A_{2} \vee B) \wedge \cdots \wedge (A_{n} \vee B)$が$\mathscr{T}$の定理ならば, 
$(A_{1} \wedge A_{2} \wedge \cdots \wedge A_{n}) \vee B$は$\mathscr{T}$の定理である.
\end{dedu}




\mathstrut
\begin{dedu}
\label{dedtgvdist}%推論102
$n$を自然数とし, $A_{1}, A_{2}, \cdots, A_{n}$を$\mathscr{T}$の関係式とする.
また$B$を$\mathscr{T}$の関係式とする.

1)
$B \to A_{1} \vee A_{2} \vee \cdots \vee A_{n}$が$\mathscr{T}$の定理ならば, 
$(B \to A_{1}) \vee (B \to A_{2}) \vee \cdots \vee (B \to A_{n})$は$\mathscr{T}$の定理である.

2)
$(B \to A_{1}) \vee (B \to A_{2}) \vee \cdots \vee (B \to A_{n})$が$\mathscr{T}$の定理ならば, 
$B \to A_{1} \vee A_{2} \vee \cdots \vee A_{n}$は$\mathscr{T}$の定理である.
\end{dedu}




\mathstrut
\begin{dedu}
\label{dedtgwdist}%推論103
$n$を自然数とし, $A_{1}, A_{2}, \cdots, A_{n}$を$\mathscr{T}$の関係式とする.
また$B$を$\mathscr{T}$の関係式とする.

1)
$B \to A_{1} \wedge A_{2} \wedge \cdots \wedge A_{n}$が$\mathscr{T}$の定理ならば, 
$(B \to A_{1}) \wedge (B \to A_{2}) \wedge \cdots \wedge (B \to A_{n})$は$\mathscr{T}$の定理である.

2)
$(B \to A_{1}) \wedge (B \to A_{2}) \wedge \cdots \wedge (B \to A_{n})$が$\mathscr{T}$の定理ならば, 
$B \to A_{1} \wedge A_{2} \wedge \cdots \wedge A_{n}$は$\mathscr{T}$の定理である.
\end{dedu}




\mathstrut
\begin{dedu}
\label{dedtgvgwdist}%推論104
$n$を自然数とし, $A_{1}, A_{2}, \cdots, A_{n}$を$\mathscr{T}$の関係式とする.
また$B$を$\mathscr{T}$の関係式とする.

1)
$A_{1} \vee A_{2} \vee \cdots \vee A_{n} \to B$が$\mathscr{T}$の定理ならば, 
$(A_{1} \to B) \wedge (A_{2} \to B) \wedge \cdots \wedge (A_{n} \to B)$は$\mathscr{T}$の定理である.

2)
$(A_{1} \to B) \wedge (A_{2} \to B) \wedge \cdots \wedge (A_{n} \to B)$が$\mathscr{T}$の定理ならば, 
$A_{1} \vee A_{2} \vee \cdots \vee A_{n} \to B$は$\mathscr{T}$の定理である.

3)
$A_{1} \wedge A_{2} \wedge \cdots \wedge A_{n} \to B$が$\mathscr{T}$の定理ならば, 
$(A_{1} \to B) \vee (A_{2} \to B) \vee \cdots \vee (A_{n} \to B)$は$\mathscr{T}$の定理である.

4)
$(A_{1} \to B) \vee (A_{2} \to B) \vee \cdots \vee (A_{n} \to B)$が$\mathscr{T}$の定理ならば, 
$A_{1} \wedge A_{2} \wedge \cdots \wedge A_{n} \to B$は$\mathscr{T}$の定理である.
\end{dedu}




\mathstrut
\begin{dedu}
\label{dedgdemorgan}%推論105
$n$を自然数とし, $A_{1}, A_{2}, \cdots, A_{n}$を$\mathscr{T}$の関係式とする.

1)
$\neg (A_{1} \vee A_{2} \vee \cdots \vee A_{n})$が$\mathscr{T}$の定理ならば, 
$\neg A_{1} \wedge \neg A_{2} \wedge \cdots \wedge \neg A_{n}$は$\mathscr{T}$の定理である.

2)
$\neg A_{1} \wedge \neg A_{2} \wedge \cdots \wedge \neg A_{n}$が$\mathscr{T}$の定理ならば, 
$\neg (A_{1} \vee A_{2} \vee \cdots \vee A_{n})$は$\mathscr{T}$の定理である.

3)
$\neg (A_{1} \wedge A_{2} \wedge \cdots \wedge A_{n})$が$\mathscr{T}$の定理ならば, 
$\neg A_{1} \vee \neg A_{2} \vee \cdots \vee \neg A_{n}$は$\mathscr{T}$の定理である.

4)
$\neg A_{1} \vee \neg A_{2} \vee \cdots \vee \neg A_{n}$が$\mathscr{T}$の定理ならば, 
$\neg (A_{1} \wedge A_{2} \wedge \cdots \wedge A_{n})$は$\mathscr{T}$の定理である.
\end{dedu}




\mathstrut
\begin{dedu}
\label{dedgabs}%推論106
$n$を自然数とし, $A_{1}, A_{2}, \cdots, A_{n}$を$\mathscr{T}$の関係式とする.
また$i$を$n$以下の自然数とする.

1)
$A_{i}$が$\mathscr{T}$の定理ならば, 
$(A_{1} \vee A_{2} \vee \cdots \vee A_{n}) \wedge A_{i}$は$\mathscr{T}$の定理である.

2)
$(A_{1} \wedge A_{2} \wedge \cdots \wedge A_{n}) \vee A_{i}$が$\mathscr{T}$の定理ならば, 
$A_{i}$は$\mathscr{T}$の定理である.
\end{dedu}




\mathstrut
\begin{dedu}
\label{dedequiv}%推論107
$A$と$B$を$\mathscr{T}$の関係式とする.

1)
$A \to B$と$B \to A$が共に$\mathscr{T}$の定理ならば, 
$A \leftrightarrow B$は$\mathscr{T}$の定理である.

2)
$A \leftrightarrow B$が$\mathscr{T}$の定理ならば, 
$A \to B$と$B \to A$は共に$\mathscr{T}$の定理である.
\end{dedu}




\mathstrut
\begin{dedu}
\label{dedpreequiv}%推論108
$A$, $B$, $C$を$\mathscr{T}$の関係式とする.

1)
$C \to (A \to B)$と$C \to (B \to A)$が共に$\mathscr{T}$の定理ならば, 
$C \to (A \leftrightarrow B)$は$\mathscr{T}$の定理である.

2)
$C \to (A \leftrightarrow B)$が$\mathscr{T}$の定理ならば, 
$C \to (A \to B)$と$C \to (B \to A)$は共に$\mathscr{T}$の定理である.
\end{dedu}




\mathstrut
\begin{dedu}
\label{dedeqch}%推論109
{\bf (対称律)}~
$A$と$B$を$\mathscr{T}$の関係式とする.
$A \leftrightarrow B$が$\mathscr{T}$の定理ならば, 
$B \leftrightarrow A$は$\mathscr{T}$の定理である.
\end{dedu}




\mathstrut
\begin{dedu}
\label{dedeqtrans}%推論110
{\bf (推移律)}~
$A$, $B$, $C$を$\mathscr{T}$の関係式とする.
$A \leftrightarrow B$と$B \leftrightarrow C$が共に$\mathscr{T}$の定理ならば, 
$A \leftrightarrow C$は$\mathscr{T}$の定理である.
\end{dedu}




\mathstrut
\begin{dedu}
\label{dedeqttrans}%推論111
$A$, $B$, $C$を$\mathscr{T}$の関係式とする.

1)
$A \leftrightarrow B$と$B \to C$が共に$\mathscr{T}$の定理ならば, 
$A \to C$は$\mathscr{T}$の定理である.

2)
$A \to B$と$B \leftrightarrow C$が共に$\mathscr{T}$の定理ならば, 
$A \to C$は$\mathscr{T}$の定理である.
\end{dedu}




\mathstrut
\begin{dedu}
\label{dedeqtorf}%推論112
$A$と$B$を$\mathscr{T}$の関係式とする.

1)
$A$と$B$が共に$\mathscr{T}$の定理ならば, 
$A \leftrightarrow B$は$\mathscr{T}$の定理である.

2)
$\neg A$と$\neg B$が共に$\mathscr{T}$の定理ならば, 
$A \leftrightarrow B$は$\mathscr{T}$の定理である.

3)
$\neg A$と$B$が共に$\mathscr{T}$の定理ならば, 
$\neg (A \leftrightarrow B)$は$\mathscr{T}$の定理である.

4)
$A$と$\neg B$が共に$\mathscr{T}$の定理ならば, 
$\neg (A \leftrightarrow B)$は$\mathscr{T}$の定理である.
\end{dedu}




\mathstrut
\begin{dedu}
\label{dedeqfund}%推論113
$A$と$B$を$\mathscr{T}$の関係式とする.
$A \leftrightarrow B$が$\mathscr{T}$の定理であるとき, 
次の1) - 4)が成り立つ.

1)
$A$が$\mathscr{T}$の定理ならば, $B$は$\mathscr{T}$の定理である.

2)
$B$が$\mathscr{T}$の定理ならば, $A$は$\mathscr{T}$の定理である.

3)
$\neg A$が$\mathscr{T}$の定理ならば, $\neg B$は$\mathscr{T}$の定理である.

4)
$\neg B$が$\mathscr{T}$の定理ならば, $\neg A$は$\mathscr{T}$の定理である.
\end{dedu}




\mathstrut
\begin{dedu}
\label{dedeqcontra}%推論114
$A$を$\mathscr{T}$の関係式とする.
$A \leftrightarrow \neg A$が$\mathscr{T}$の定理ならば, 
$\mathscr{T}$は矛盾する.
\end{dedu}




\mathstrut
\begin{dedu}
\label{dedavblbtrue1}%推論115
$A$と$B$を$\mathscr{T}$の関係式とする.

1)
$A \to B$が$\mathscr{T}$の定理ならば, 
$A \vee B \leftrightarrow B$は$\mathscr{T}$の定理である.

2)
$A \vee B \leftrightarrow B$が$\mathscr{T}$の定理ならば, 
$A \to B$は$\mathscr{T}$の定理である.

3)
$B \to A$が$\mathscr{T}$の定理ならば, 
$A \vee B \leftrightarrow A$は$\mathscr{T}$の定理である.

4)
$A \vee B \leftrightarrow A$が$\mathscr{T}$の定理ならば, 
$B \to A$は$\mathscr{T}$の定理である.
\end{dedu}




\mathstrut
\begin{dedu}
\label{dedavblbtrue2}%推論116
$A$と$B$を$\mathscr{T}$の関係式とする.

1)
$\neg A$が$\mathscr{T}$の定理ならば, 
$A \vee B \leftrightarrow B$は$\mathscr{T}$の定理である.

2)
$\neg B$が$\mathscr{T}$の定理ならば, 
$A \vee B \leftrightarrow A$は$\mathscr{T}$の定理である.
\end{dedu}




\mathstrut
\begin{dedu}
\label{dedtlv}%推論117
$A$と$B$を$\mathscr{T}$の関係式とする.

1)
$(A \to B) \leftrightarrow B$が$\mathscr{T}$の定理ならば, 
$A \vee B$は$\mathscr{T}$の定理である.

2)
$A \vee B$が$\mathscr{T}$の定理ならば, 
$(A \to B) \leftrightarrow B$は$\mathscr{T}$の定理である.
\end{dedu}




\mathstrut
\begin{dedu}
\label{ded1atb1lbtrue}%推論118
$A$と$B$を$\mathscr{T}$の関係式とする.

1)
$A$が$\mathscr{T}$の定理ならば, 
$(A \to B) \leftrightarrow B$は$\mathscr{T}$の定理である.

2)
$B$が$\mathscr{T}$の定理ならば, 
$(A \to B) \leftrightarrow B$は$\mathscr{T}$の定理である.
\end{dedu}




\mathstrut
\begin{dedu}
\label{dedawblatrue1}%推論119
$A$と$B$を$\mathscr{T}$の関係式とする.

1)
$A \to B$が$\mathscr{T}$の定理ならば, 
$A \wedge B \leftrightarrow A$は$\mathscr{T}$の定理である.

2)
$A \wedge B \leftrightarrow A$が$\mathscr{T}$の定理ならば, 
$A \to B$は$\mathscr{T}$の定理である.

3)
$B \to A$が$\mathscr{T}$の定理ならば, 
$A \wedge B \leftrightarrow B$は$\mathscr{T}$の定理である.

4)
$A \wedge B \leftrightarrow B$が$\mathscr{T}$の定理ならば, 
$B \to A$は$\mathscr{T}$の定理である.
\end{dedu}




\mathstrut
\begin{dedu}
\label{dedawblatrue2}%推論120
$A$と$B$を$\mathscr{T}$の関係式とする.

1)
$B$が$\mathscr{T}$の定理ならば, 
$A \wedge B \leftrightarrow A$は$\mathscr{T}$の定理である.

2)
$A$が$\mathscr{T}$の定理ならば, 
$A \wedge B \leftrightarrow B$は$\mathscr{T}$の定理である.
\end{dedu}




\mathstrut
\begin{dedu}
\label{dedgvtrueeq}%推論121
$n$を自然数とし, $A_{1}, A_{2}, \cdots, A_{n}$を$\mathscr{T}$の関係式とする.
また$k$を自然数とし, $i_{1}, i_{2}, \cdots, i_{k}$を$n$以下の自然数とする.
いま$i_{1}, i_{2}, \cdots, i_{k}$のいずれとも異なるような$n$以下の各自然数$i$ごとに, 
次のa), b), c)のいずれかが成立するとする: 

\noindent
a)
$A_{i} \to A_{i_{1}} \vee A_{i_{2}} \vee \cdots \vee A_{i_{k}}$は$\mathscr{T}$の定理である.

\noindent
b)
$A_{i} \to A_{i_{1}}, A_{i} \to A_{i_{2}}, \cdots, A_{i} \to A_{i_{k}}$の
うち少なくとも一つは$\mathscr{T}$の定理である.

\noindent
c)
$\neg A_{i}$は$\mathscr{T}$の定理である.

\noindent
このとき, 
\[
  A_{1} \vee A_{2} \vee \cdots \vee A_{n} 
  \leftrightarrow A_{i_{1}} \vee A_{i_{2}} \vee \cdots \vee A_{i_{k}}
\]
は$\mathscr{T}$の定理である.
\end{dedu}




\mathstrut
\begin{dedu}
\label{dedgwtrueeq}%推論122
$n$を自然数とし, $A_{1}, A_{2}, \cdots, A_{n}$を$\mathscr{T}$の関係式とする.
また$k$を自然数とし, $i_{1}, i_{2}, \cdots, i_{k}$を$n$以下の自然数とする.
いま$i_{1}, i_{2}, \cdots, i_{k}$のいずれとも異なるような$n$以下の各自然数$i$ごとに, 
次のa), b), c)のいずれかが成立するとする: 

\noindent
a)
$A_{i_{1}} \wedge A_{i_{2}} \wedge \cdots \wedge A_{i_{k}} \to A_{i}$は$\mathscr{T}$の定理である.

\noindent
b)
$A_{i_{1}} \to A_{i}, A_{i_{2}} \to A_{i}, \cdots, A_{i_{k}} \to A_{i}$の
うち少なくとも一つは$\mathscr{T}$の定理である.

\noindent
c)
$A_{i}$は$\mathscr{T}$の定理である.

\noindent
このとき, 
\[
  A_{1} \wedge A_{2} \wedge \cdots \wedge A_{n} 
  \leftrightarrow A_{i_{1}} \wedge A_{i_{2}} \wedge \cdots \wedge A_{i_{k}}
\]
は$\mathscr{T}$の定理である.
\end{dedu}




\mathstrut
\begin{dedu}
\label{dedeqcp}%推論123
$A$と$B$を$\mathscr{T}$の関係式とする.

1)
$A \leftrightarrow B$が$\mathscr{T}$の定理ならば, 
$\neg A \leftrightarrow \neg B$は$\mathscr{T}$の定理である.

2)
$\neg A \leftrightarrow \neg B$が$\mathscr{T}$の定理ならば, 
$A \leftrightarrow B$は$\mathscr{T}$の定理である.

3)
$\neg A \leftrightarrow B$が$\mathscr{T}$の定理ならば, 
$A \leftrightarrow \neg B$は$\mathscr{T}$の定理である.

4)
$A \leftrightarrow \neg B$が$\mathscr{T}$の定理ならば, 
$\neg A \leftrightarrow B$は$\mathscr{T}$の定理である.
\end{dedu}




\mathstrut
\begin{dedu}
\label{dedaddeqt}%推論124
\mbox{}

1)
$A$, $B$, $C$を$\mathscr{T}$の関係式とする.
$A \leftrightarrow B$が$\mathscr{T}$の定理ならば, 
$(A \to C) \leftrightarrow (B \to C)$と$(C \to A) \leftrightarrow (C \to B)$は
共に$\mathscr{T}$の定理である.

2)
$A$, $B$, $C$, $D$を$\mathscr{T}$の関係式とする.
$A \leftrightarrow B$と$C \leftrightarrow D$が共に$\mathscr{T}$の定理ならば, 
$(A \to C) \leftrightarrow (B \to D)$は$\mathscr{T}$の定理である.
\end{dedu}




\mathstrut
\begin{dedu}
\label{dedaddeqv}%推論125
\mbox{}

1)
$A$, $B$, $C$を$\mathscr{T}$の関係式とする.
$A \leftrightarrow B$が$\mathscr{T}$の定理ならば, 
$A \vee C \leftrightarrow B \vee C$と$C \vee A \leftrightarrow C \vee B$は
共に$\mathscr{T}$の定理である.

2)
$A$, $B$, $C$, $D$を$\mathscr{T}$の関係式とする.
$A \leftrightarrow B$と$C \leftrightarrow D$が共に$\mathscr{T}$の定理ならば, 
$A \vee C \leftrightarrow B \vee D$は$\mathscr{T}$の定理である.
\end{dedu}




\mathstrut
\begin{dedu}
\label{dedaddeqw}%推論126
\mbox{}

1)
$A$, $B$, $C$を$\mathscr{T}$の関係式とする.
$A \leftrightarrow B$が$\mathscr{T}$の定理ならば, 
$A \wedge C \leftrightarrow B \wedge C$と$C \wedge A \leftrightarrow C \wedge B$は
共に$\mathscr{T}$の定理である.

2)
$A$, $B$, $C$, $D$を$\mathscr{T}$の関係式とする.
$A \leftrightarrow B$と$C \leftrightarrow D$が共に$\mathscr{T}$の定理ならば, 
$A \wedge C \leftrightarrow B \wedge D$は$\mathscr{T}$の定理である.
\end{dedu}




\mathstrut
\begin{dedu}
\label{dedaddeqgv}%推論127
$n$を自然数とし, $A_{1}, A_{2}, \cdots, A_{n}, B_{1}, B_{2}, \cdots, B_{n}$を$\mathscr{T}$の関係式とする.
いま$n$以下の各自然数$i$に対し, $B_{i}$は$A_{i}$であるか, 
または$A_{i} \leftrightarrow B_{i}$が$\mathscr{T}$の定理であるとする.
このとき
$A_{1} \vee A_{2} \vee \cdots \vee A_{n} \leftrightarrow B_{1} \vee B_{2} \vee \cdots \vee B_{n}$は
$\mathscr{T}$の定理である.
\end{dedu}




\mathstrut
\begin{dedu}
\label{dedaddeqgw}%推論128
$n$を自然数とし, $A_{1}, A_{2}, \cdots, A_{n}, B_{1}, B_{2}, \cdots, B_{n}$を$\mathscr{T}$の関係式とする.
いま$n$以下の各自然数$i$に対し, $B_{i}$は$A_{i}$であるか, 
または$A_{i} \leftrightarrow B_{i}$が$\mathscr{T}$の定理であるとする.
このとき
$A_{1} \wedge A_{2} \wedge \cdots \wedge A_{n} \leftrightarrow B_{1} \wedge B_{2} \wedge \cdots \wedge B_{n}$は
$\mathscr{T}$の定理である.
\end{dedu}




\mathstrut
\begin{dedu}
\label{dedaddeqeq}%推論129
\mbox{}

1)
$A$, $B$, $C$を$\mathscr{T}$の関係式とする.
$A \leftrightarrow B$が$\mathscr{T}$の定理ならば, 
$(A \leftrightarrow C) \leftrightarrow (B \leftrightarrow C)$と
$(C \leftrightarrow A) \leftrightarrow (C \leftrightarrow B)$は
共に$\mathscr{T}$の定理である.

2)
$A$, $B$, $C$, $D$を$\mathscr{T}$の関係式とする.
$A \leftrightarrow B$と$C \leftrightarrow D$が共に$\mathscr{T}$の定理ならば, 
$(A \leftrightarrow C) \leftrightarrow (B \leftrightarrow D)$は$\mathscr{T}$の定理である.
\end{dedu}




\mathstrut
\begin{dedu}
\label{dedtbackeq}%推論130
$A$, $B$, $C$を$\mathscr{T}$の関係式とする.

1)
$C \to (A \leftrightarrow B)$が$\mathscr{T}$の定理ならば, 
$(C \to A) \leftrightarrow (C \to B)$は$\mathscr{T}$の定理である.

2)
$(C \to A) \leftrightarrow (C \to B)$が$\mathscr{T}$の定理ならば, 
$C \to (A \leftrightarrow B)$は$\mathscr{T}$の定理である.
\end{dedu}




\mathstrut
\begin{dedu}
\label{dedtfronteq}%推論131
$A$, $B$, $C$を$\mathscr{T}$の関係式とする.

1)
$\neg C \to (A \leftrightarrow B)$が$\mathscr{T}$の定理ならば, 
$(A \to C) \leftrightarrow (B \to C)$は$\mathscr{T}$の定理である.

2)
$(A \to C) \leftrightarrow (B \to C)$が$\mathscr{T}$の定理ならば, 
$\neg C \to (A \leftrightarrow B)$は$\mathscr{T}$の定理である.
\end{dedu}




\mathstrut
\begin{dedu}
\label{dedeq&v}%推論132
$A$, $B$, $C$を$\mathscr{T}$の関係式とする.

1)
$(A \leftrightarrow B) \vee C$が$\mathscr{T}$の定理ならば, 
$A \vee C \leftrightarrow B \vee C$と$C \vee A \leftrightarrow C \vee B$は共に
$\mathscr{T}$の定理である.

2)
$C \vee (A \leftrightarrow B)$が$\mathscr{T}$の定理ならば, 
$A \vee C \leftrightarrow B \vee C$と$C \vee A \leftrightarrow C \vee B$は共に
$\mathscr{T}$の定理である.

3)
$A \vee C \leftrightarrow B \vee C$が$\mathscr{T}$の定理ならば, 
$(A \leftrightarrow B) \vee C$と$C \vee (A \leftrightarrow B)$は共に
$\mathscr{T}$の定理である.

4)
$C \vee A \leftrightarrow C \vee B$が$\mathscr{T}$の定理ならば, 
$(A \leftrightarrow B) \vee C$と$C \vee (A \leftrightarrow B)$は共に
$\mathscr{T}$の定理である.
\end{dedu}




\mathstrut
\begin{dedu}
\label{dedeq&w}%推論133
$A$, $B$, $C$を$\mathscr{T}$の関係式とする.

1)
$C \to (A \leftrightarrow B)$が$\mathscr{T}$の定理ならば, 
$A \wedge C \leftrightarrow B \wedge C$と
$C \wedge A \leftrightarrow C \wedge B$は共に$\mathscr{T}$の定理である.

2)
$A \wedge C \leftrightarrow B \wedge C$が$\mathscr{T}$の定理ならば, 
$C \to (A \leftrightarrow B)$は$\mathscr{T}$の定理である.

3)
$C \wedge A \leftrightarrow C \wedge B$が$\mathscr{T}$の定理ならば, 
$C \to (A \leftrightarrow B)$は$\mathscr{T}$の定理である.
\end{dedu}




\mathstrut
\begin{dedu}
\label{dedeq&eq}%推論134
$A$, $B$, $C$を$\mathscr{T}$の関係式とする.

1)
$(A \leftrightarrow C) \leftrightarrow (B \leftrightarrow C)$が$\mathscr{T}$の定理ならば, 
$A \leftrightarrow B$は$\mathscr{T}$の定理である.

2)
$(C \leftrightarrow A) \leftrightarrow (C \leftrightarrow B)$が$\mathscr{T}$の定理ならば, 
$A \leftrightarrow B$は$\mathscr{T}$の定理である.
\end{dedu}




\mathstrut
\begin{dedu}
\label{dedat1alb1true1}%推論135
$A$と$B$を$\mathscr{T}$の関係式とする.

1)
$A \to B$が$\mathscr{T}$の定理ならば, 
$A \to (A \leftrightarrow B)$は$\mathscr{T}$の定理である.

2)
$A \to (A \leftrightarrow B)$が$\mathscr{T}$の定理ならば, 
$A \to B$は$\mathscr{T}$の定理である.

3)
$B \to A$が$\mathscr{T}$の定理ならば, 
$B \to (A \leftrightarrow B)$は$\mathscr{T}$の定理である.

4)
$B \to (A \leftrightarrow B)$が$\mathscr{T}$の定理ならば, 
$B \to A$は$\mathscr{T}$の定理である.
\end{dedu}




\mathstrut
\begin{dedu}
\label{dedat1alb1true2}%推論136
$A$と$B$を$\mathscr{T}$の関係式とする.

1)
$B$が$\mathscr{T}$の定理ならば, 
$A \to (A \leftrightarrow B)$は$\mathscr{T}$の定理である.

3)
$A$が$\mathscr{T}$の定理ならば, 
$B \to (A \leftrightarrow B)$は$\mathscr{T}$の定理である.
\end{dedu}




\mathstrut
\begin{dedu}
\label{ded1alb1tbtrue1}%推論137
$A$と$B$を$\mathscr{T}$の関係式とする.

1)
$A \vee B$が$\mathscr{T}$の定理ならば, 
$(A \leftrightarrow B) \to A$と$(A \leftrightarrow B) \to B$は共に$\mathscr{T}$の定理である.

2)
$(A \leftrightarrow B) \to A$が$\mathscr{T}$の定理ならば, 
$A \vee B$は$\mathscr{T}$の定理である.

3)
$(A \leftrightarrow B) \to B$が$\mathscr{T}$の定理ならば, 
$A \vee B$は$\mathscr{T}$の定理である.
\end{dedu}




\mathstrut
\begin{dedu}
\label{ded1alb1tbtrue2}%推論138
$A$と$B$を$\mathscr{T}$の関係式とする.

1)
$A$が$\mathscr{T}$の定理ならば, 
$(A \leftrightarrow B) \to B$は$\mathscr{T}$の定理である.

2)
$B$が$\mathscr{T}$の定理ならば, 
$(A \leftrightarrow B) \to A$は$\mathscr{T}$の定理である.
\end{dedu}




\mathstrut
\begin{dedu}
\label{ded1alb1lbtrue}%推論139
$A$と$B$を$\mathscr{T}$の関係式とする.

1)
$A$が$\mathscr{T}$の定理ならば, 
$(A \leftrightarrow B) \leftrightarrow B$は$\mathscr{T}$の定理である.

2)
$(A \leftrightarrow B) \leftrightarrow B$が$\mathscr{T}$の定理ならば, 
$A$は$\mathscr{T}$の定理である.

3)
$B$が$\mathscr{T}$の定理ならば, 
$(A \leftrightarrow B) \leftrightarrow A$は$\mathscr{T}$の定理である.

4)
$(A \leftrightarrow B) \leftrightarrow A$が$\mathscr{T}$の定理ならば, 
$B$は$\mathscr{T}$の定理である.
\end{dedu}




\mathstrut
\begin{dedu}
\label{dedeq&v&w}%推論140
$A$と$B$を$\mathscr{T}$の関係式とする.

1)
$A \leftrightarrow B$が$\mathscr{T}$の定理ならば, 
$(A \wedge B) \vee (\neg A \wedge \neg B)$は$\mathscr{T}$の定理である.

2)
$(A \wedge B) \vee (\neg A \wedge \neg B)$が$\mathscr{T}$の定理ならば, 
$A \leftrightarrow B$は$\mathscr{T}$の定理である.
\end{dedu}




\mathstrut
\begin{dedu}
\label{dedltthmquan}%推論141
$R$を論理的な理論$\mathscr{T}$の定理とし, $x$を文字とする.
$x$が$\mathscr{T}$の定数でなければ, 
$\exists x(R)$と$\forall x(R)$は共に$\mathscr{T}$の定理である.
\end{dedu}




\mathstrut
\begin{dedu}
\label{dedaddconstexist}%推論142
$R$と$S$を論理的な理論$\mathscr{T}$の関係式, $x$を文字とし, 
これらが次のa), b)を満たすとする: 

a)
$x$は$\mathscr{T}$の定数ではなく, $S$の中に自由変数として現れない.

b)
$\exists x(R)$は$\mathscr{T}$の定理である.

$\mathscr{T}$の明示的公理に$R$を追加して得られる理論を$\mathscr{T}'$とする.
$S$が$\mathscr{T}'$の定理ならば, $S$は$\mathscr{T}$の定理である.
\end{dedu}




\mathstrut
\begin{dedu}
\label{dedquanfree}%推論143
$\mathscr{T}$を論理的な理論とし, $R$を$\mathscr{T}$の関係式とする.
また$x$を$R$の中に自由変数として現れない文字とする.

1)
$\exists x(R)$が$\mathscr{T}$の定理ならば, $R$は$\mathscr{T}$の定理である.

2)
$\forall x(R)$が$\mathscr{T}$の定理ならば, $R$は$\mathscr{T}$の定理である.

3)
$R$が$\mathscr{T}$の定理ならば, $\exists x(R)$と$\forall x(R)$は共に$\mathscr{T}$の定理である.
\end{dedu}




\mathstrut
\begin{dedu}
\label{dedallfund}%推論144
$\mathscr{T}$を論理的な理論とし, $R$を$\mathscr{T}$の関係式, $x$を文字とする.

1)
$\forall x(R)$が$\mathscr{T}$の定理ならば, 
$(\tau_{x}(\neg R)|x)(R)$は$\mathscr{T}$の定理である.

2)
$(\tau_{x}(\neg R)|x)(R)$が$\mathscr{T}$の定理ならば, 
$\forall x(R)$は$\mathscr{T}$の定理である.
\end{dedu}




\mathstrut
\begin{dedu}
\label{dedaequandm}%推論145
$\mathscr{T}$を論理的な理論とし, $R$を$\mathscr{T}$の関係式, $x$を文字とする.

1)
$\neg \forall x(R)$が$\mathscr{T}$の定理ならば, 
$\exists x(\neg R)$は$\mathscr{T}$の定理である.

2)
$\exists x(\neg R)$が$\mathscr{T}$の定理ならば, 
$\neg \forall x(R)$は$\mathscr{T}$の定理である.
\end{dedu}




\mathstrut
\begin{dedu}
\label{deds4}%推論146
$R$を$\mathscr{T}$の関係式, $T$を$\mathscr{T}$の対象式とし, $x$を文字とする.
$(T|x)(R)$が$\mathscr{T}$の定理ならば, $\exists x(R)$は$\mathscr{T}$の定理である.
特に$R$が$\mathscr{T}$の定理ならば, $\exists x(R)$は$\mathscr{T}$の定理である.
\end{dedu}




\mathstrut
\begin{dedu}
\label{dedfromallthm}%推論147
$R$を$\mathscr{T}$の関係式とし, $x$を文字とする.
$\forall x(R)$が$\mathscr{T}$の定理ならば, $\mathscr{T}$の
任意の対象式$T$に対して$(T|x)(R)$は$\mathscr{T}$の定理である.
特に$R$及び$\exists x(R)$は共に$\mathscr{T}$の定理である.
\end{dedu}




\mathstrut
\begin{dedu}
\label{dedgs4}%推論148
$R$を$\mathscr{T}$の関係式とする.
また$n$を自然数とし, $T_{1}, T_{2}, \cdots, T_{n}$を$\mathscr{T}$の対象式とする.
また$x_{1}, x_{2}, \cdots, x_{n}$を, どの二つも互いに異なる文字とする.
$(T_{1}|x_{1}, T_{2}|x_{2}, \cdots, T_{n}|x_{n})(R)$が$\mathscr{T}$の定理ならば, 
$\exists x_{1}(\exists x_{2}( \cdots (\exists x_{n}(R)) \cdots ))$は$\mathscr{T}$の定理である.
\end{dedu}




\mathstrut
\begin{dedu}
\label{dedgallfund2}%推論149
$R$を$\mathscr{T}$の関係式とする.
また$n$を自然数とし, $T_{1}, T_{2}, \cdots, T_{n}$を$\mathscr{T}$の対象式とする.
また$x_{1}, x_{2}, \cdots, x_{n}$を, どの二つも互いに異なる文字とする.
$\forall x_{1}(\forall x_{2}( \cdots (\forall x_{n}(R)) \cdots ))$が$\mathscr{T}$の定理ならば, 
$(T_{1}|x_{1}, T_{2}|x_{2}, \cdots, T_{n}|x_{n})(R)$は$\mathscr{T}$の定理である.
\end{dedu}




\mathstrut
\begin{dedu}
\label{dedtquanfund}%推論150
$R$と$S$を$\mathscr{T}$の関係式とし, $x$を文字とする.

1)
$(\tau_{x}(R)|x)(R \to S)$が$\mathscr{T}$の定理ならば, 
$\exists x(R) \to \exists x(S)$は$\mathscr{T}$の定理である.

2)
$(\tau_{x}(\neg S)|x)(R \to S)$が$\mathscr{T}$の定理ならば, 
$\forall x(R) \to \forall x(S)$は$\mathscr{T}$の定理である.
\end{dedu}




\mathstrut
\begin{dedu}
\label{dedtquanfund2}%推論151
$R$と$S$を$\mathscr{T}$の関係式, $x$を文字とし, これらが次の性質($*$)を持つとする: 

($*$) ~~$\mathscr{T}$の任意の対象式$T$に対し, $(T|x)(R \to S)$は$\mathscr{T}$の定理となる.

このとき, $\exists x(R) \to \exists x(S)$と$\forall x(R) \to \forall x(S)$は共に
$\mathscr{T}$の定理である.
\end{dedu}




\mathstrut
\begin{dedu}
\label{dedeqquanfund}%推論152
$R$と$S$を$\mathscr{T}$の関係式とし, $x$を文字とする.

1)
$(\tau_{x}(R)|x)(R \to S)$と$(\tau_{x}(S)|x)(S \to R)$が
共に$\mathscr{T}$の定理ならば, 
$\exists x(R) \leftrightarrow \exists x(S)$は$\mathscr{T}$の定理である.

2)
$(\tau_{x}(\neg S)|x)(R \to S)$と$(\tau_{x}(\neg R)|x)(S \to R)$が
共に$\mathscr{T}$の定理ならば, 
$\forall x(R) \leftrightarrow \forall x(S)$は$\mathscr{T}$の定理である.
\end{dedu}




\mathstrut
\begin{dedu}
\label{dedeqquanfund2}%推論153
$R$と$S$を$\mathscr{T}$の関係式, $x$を文字とし, これらが次の性質($*$)を持つとする: 

($*$) ~~$\mathscr{T}$の任意の対象式$T$に対し, $(T|x)(R \leftrightarrow S)$は$\mathscr{T}$の定理となる.

このとき, $\exists x(R) \leftrightarrow \exists x(S)$と$\forall x(R) \leftrightarrow \forall x(S)$は共に
$\mathscr{T}$の定理である.
\end{dedu}




\mathstrut
\begin{dedu}
\label{dedeaquandm}%推論154
$R$を$\mathscr{T}$の関係式とし, $x$を文字とする.

1)
$\neg \exists x(R)$が$\mathscr{T}$の定理ならば, 
$\forall x(\neg R)$は$\mathscr{T}$の定理である.

2)
$\forall x(\neg R)$が$\mathscr{T}$の定理ならば, 
$\neg \exists x(R)$は$\mathscr{T}$の定理である.
\end{dedu}




\mathstrut
\begin{dedu}
\label{dedquangdm}%推論155
$R$を$\mathscr{T}$の関係式とする.
また$n$を自然数とし, $x_{1}, x_{2}, \cdots, x_{n}$を文字とする.
また$n$以下の各自然数$i$に対し, $p_{i}$を$\exists$, $\forall$のどちらかとし, 
$q_{i}$を, $p_{i}$が$\exists$ならば$\forall$, $p_{i}$が$\forall$ならば$\exists$とする.

1)
$\neg p_{1}x_{1}(p_{2}x_{2}( \cdots (p_{n}x_{n}(R)) \cdots ))$が$\mathscr{T}$の定理ならば, 
$q_{1}x_{1}(q_{2}x_{2}( \cdots (q_{n}x_{n}(\neg R)) \cdots ))$は$\mathscr{T}$の定理である.

2)
$q_{1}x_{1}(q_{2}x_{2}( \cdots (q_{n}x_{n}(\neg R)) \cdots ))$が$\mathscr{T}$の定理ならば, 
$\neg p_{1}x_{1}(p_{2}x_{2}( \cdots (p_{n}x_{n}(R)) \cdots ))$は$\mathscr{T}$の定理である.
\end{dedu}




\mathstrut
\begin{dedu}
\label{dedquanvee}%推論156
$R$と$S$を$\mathscr{T}$の関係式とし, $x$を文字とする.

1)
$\exists x(R)$が$\mathscr{T}$の定理ならば, 
$\exists x(R \vee S)$は$\mathscr{T}$の定理である.

2)
$\exists x(S)$が$\mathscr{T}$の定理ならば, 
$\exists x(R \vee S)$は$\mathscr{T}$の定理である.

3)
$\forall x(R)$が$\mathscr{T}$の定理ならば, 
$\forall x(R \vee S)$は$\mathscr{T}$の定理である.

4)
$\forall x(S)$が$\mathscr{T}$の定理ならば, 
$\forall x(R \vee S)$は$\mathscr{T}$の定理である.
\end{dedu}




\mathstrut
\begin{dedu}
\label{dedquanvch}%推論157
$R$と$S$を$\mathscr{T}$の関係式とし, $x$を文字とする.

1)
$\exists x(R \vee S)$が$\mathscr{T}$の定理ならば, 
$\exists x(S \vee R)$は$\mathscr{T}$の定理である.

2)
$\forall x(R \vee S)$が$\mathscr{T}$の定理ならば, 
$\forall x(S \vee R)$は$\mathscr{T}$の定理である.
\end{dedu}




\mathstrut
\begin{dedu}
\label{dedquantveq}%推論158
$R$と$S$を$\mathscr{T}$の関係式とし, $x$を文字とする.

1)
$\exists x(R \to S)$が$\mathscr{T}$の定理ならば, 
$\exists x(\neg R \vee S)$は$\mathscr{T}$の定理である.

2)
$\exists x(\neg R \vee S)$が$\mathscr{T}$の定理ならば, 
$\exists x(R \to S)$は$\mathscr{T}$の定理である.

3)
$\forall x(R \to S)$が$\mathscr{T}$の定理ならば, 
$\forall x(\neg R \vee S)$は$\mathscr{T}$の定理である.

4)
$\forall x(\neg R \vee S)$が$\mathscr{T}$の定理ならば, 
$\forall x(R \to S)$は$\mathscr{T}$の定理である.
\end{dedu}




\mathstrut
\begin{dedu}
\label{dedexv}%推論159
$R$と$S$を$\mathscr{T}$の関係式とし, $x$を文字とする.

1)
$\exists x(R \vee S)$が$\mathscr{T}$の定理ならば, 
$\exists x(R) \vee \exists x(S)$は$\mathscr{T}$の定理である.

2)
$\exists x(R) \vee \exists x(S)$が$\mathscr{T}$の定理ならば, 
$\exists x(R \vee S)$は$\mathscr{T}$の定理である.
\end{dedu}




\mathstrut
\begin{dedu}
\label{dedexvrfree}%推論160
$R$と$S$を$\mathscr{T}$の関係式とし, $x$を$R$の中に自由変数として現れない文字とする.

1)
$\exists x(R \vee S)$が$\mathscr{T}$の定理ならば, 
$R \vee \exists x(S)$は$\mathscr{T}$の定理である.

2)
$R \vee \exists x(S)$が$\mathscr{T}$の定理ならば, 
$\exists x(R \vee S)$は$\mathscr{T}$の定理である.

3)
$\exists x(S \vee R)$が$\mathscr{T}$の定理ならば, 
$\exists x(S) \vee R$は$\mathscr{T}$の定理である.

4)
$\exists x(S) \vee R$が$\mathscr{T}$の定理ならば, 
$\exists x(S \vee R)$は$\mathscr{T}$の定理である.
\end{dedu}




\mathstrut
\begin{dedu}
\label{dedallv}%推論161
$R$と$S$を$\mathscr{T}$の関係式とし, $x$を文字とする.

1)
$\forall x(R) \vee \forall x(S)$が$\mathscr{T}$の定理ならば, 
$\forall x(R \vee S)$は$\mathscr{T}$の定理である.

2)
$\forall x(R \vee S)$が$\mathscr{T}$の定理ならば, 
$\forall x(R) \vee \exists x(S)$と$\exists x(R) \vee \forall x(S)$は共に$\mathscr{T}$の定理である.
\end{dedu}




\mathstrut
\begin{dedu}
\label{dedallvrfree}%推論162
$R$と$S$を$\mathscr{T}$の関係式とし, $x$を$R$の中に自由変数として現れない文字とする.

1)
$\forall x(R \vee S)$が$\mathscr{T}$の定理ならば, 
$R \vee \forall x(S)$は$\mathscr{T}$の定理である.

2)
$R \vee \forall x(S)$が$\mathscr{T}$の定理ならば, 
$\forall x(R \vee S)$は$\mathscr{T}$の定理である.

3)
$\forall x(S \vee R)$が$\mathscr{T}$の定理ならば, 
$\forall x(S) \vee R$は$\mathscr{T}$の定理である.

4)
$\forall x(S) \vee R$が$\mathscr{T}$の定理ならば, 
$\forall x(S \vee R)$は$\mathscr{T}$の定理である.
\end{dedu}




\mathstrut
\begin{dedu}
\label{dedallvrfree2}%推論163
$R$と$S$を$\mathscr{T}$の関係式とし, $x$を$R$の中に自由変数として現れない文字とする.
$R$が$\mathscr{T}$の定理ならば, 
$\forall x(R \vee S)$と$\forall x(S \vee R)$は共に$\mathscr{T}$の定理である.
\end{dedu}




\mathstrut
\begin{dedu}
\label{dedquanwedge}%推論164
$R$と$S$を$\mathscr{T}$の関係式とし, $x$を文字とする.

1)
$\exists x(R \wedge S)$が$\mathscr{T}$の定理ならば, 
$\exists x(R)$と$\exists x(S)$は共に$\mathscr{T}$の定理である.

2)
$\forall x(R \wedge S)$が$\mathscr{T}$の定理ならば, 
$\forall x(R)$と$\forall x(S)$は共に$\mathscr{T}$の定理である.
\end{dedu}




\mathstrut
\begin{dedu}
\label{dedquanwch}%推論165
$R$と$S$を$\mathscr{T}$の関係式とし, $x$を文字とする.

1)
$\exists x(R \wedge S)$が$\mathscr{T}$の定理ならば, 
$\exists x(S \wedge R)$は$\mathscr{T}$の定理である.

2)
$\forall x(R \wedge S)$が$\mathscr{T}$の定理ならば, 
$\forall x(S \wedge R)$は$\mathscr{T}$の定理である.
\end{dedu}




\mathstrut
\begin{dedu}
\label{dedquantweq}%推論166
$R$と$S$を$\mathscr{T}$の関係式とし, $x$を文字とする.

1)
$\exists x(\neg (R \to S))$が$\mathscr{T}$の定理ならば, 
$\exists x(R \wedge \neg S)$は$\mathscr{T}$の定理である.

2)
$\exists x(R \wedge \neg S)$が$\mathscr{T}$の定理ならば, 
$\exists x(\neg (R \to S))$は$\mathscr{T}$の定理である.

3)
$\forall x(\neg (R \to S))$が$\mathscr{T}$の定理ならば, 
$\forall x(R \wedge \neg S)$は$\mathscr{T}$の定理である.

4)
$\forall x(R \wedge \neg S)$が$\mathscr{T}$の定理ならば, 
$\forall x(\neg (R \to S))$は$\mathscr{T}$の定理である.
\end{dedu}




\mathstrut
\begin{dedu}
\label{dedexw}%推論167
$R$と$S$を$\mathscr{T}$の関係式とし, $x$を文字とする.

1)
$\exists x(R \wedge S)$が$\mathscr{T}$の定理ならば, 
$\exists x(R) \wedge \exists x(S)$は$\mathscr{T}$の定理である.

2)
$\exists x(R) \wedge \forall x(S)$が$\mathscr{T}$の定理ならば, 
$\exists x(R \wedge S)$は$\mathscr{T}$の定理である.

3)
$\forall x(R) \wedge \exists x(S)$が$\mathscr{T}$の定理ならば, 
$\exists x(R \wedge S)$は$\mathscr{T}$の定理である.
\end{dedu}




\mathstrut
\begin{dedu}
\label{dedexw2}%推論168
$R$と$S$を$\mathscr{T}$の関係式とし, $x$を文字とする.

1)
$\exists x(R)$と$\forall x(S)$が共に$\mathscr{T}$の定理ならば, 
$\exists x(R \wedge S)$は$\mathscr{T}$の定理である.

2)
$\forall x(R)$と$\exists x(S)$が共に$\mathscr{T}$の定理ならば, 
$\exists x(R \wedge S)$は$\mathscr{T}$の定理である.
\end{dedu}




\mathstrut
\begin{dedu}
\label{dedexwrfree}%推論169
$R$と$S$を$\mathscr{T}$の関係式とし, $x$を$R$の中に自由変数として現れない文字とする.

1)
$\exists x(R \wedge S)$が$\mathscr{T}$の定理ならば, 
$R \wedge \exists x(S)$は$\mathscr{T}$の定理である.

2)
$R \wedge \exists x(S)$が$\mathscr{T}$の定理ならば, 
$\exists x(R \wedge S)$は$\mathscr{T}$の定理である.

3)
$\exists x(S \wedge R)$が$\mathscr{T}$の定理ならば, 
$\exists x(S) \wedge R$は$\mathscr{T}$の定理である.

4)
$\exists x(S) \wedge R$が$\mathscr{T}$の定理ならば, 
$\exists x(S \wedge R)$は$\mathscr{T}$の定理である.
\end{dedu}




\mathstrut
\begin{dedu}
\label{dedexwrfree2}%推論170
$R$と$S$を$\mathscr{T}$の関係式とし, $x$を$R$の中に自由変数として現れない文字とする.

1)
$\exists x(R \wedge S)$が$\mathscr{T}$の定理ならば, 
$R$は$\mathscr{T}$の定理である.

2)
$\exists x(S \wedge R)$が$\mathscr{T}$の定理ならば, 
$R$は$\mathscr{T}$の定理である.

3)
$R$と$\exists x(S)$が共に$\mathscr{T}$の定理ならば, 
$\exists x(R \wedge S)$と$\exists x(S \wedge R)$は共に$\mathscr{T}$の定理である.
\end{dedu}




\mathstrut
\begin{dedu}
\label{dedallw}%推論171
$R$と$S$を$\mathscr{T}$の関係式とし, $x$を文字とする.

1)
$\forall x(R \wedge S)$が$\mathscr{T}$の定理ならば, 
$\forall x(R) \wedge \forall x(S)$は$\mathscr{T}$の定理である.

2)
$\forall x(R) \wedge \forall x(S)$が$\mathscr{T}$の定理ならば, 
$\forall x(R \wedge S)$は$\mathscr{T}$の定理である.
\end{dedu}




\mathstrut
\begin{dedu}
\label{dedallw2}%推論172
$R$と$S$を$\mathscr{T}$の関係式とし, $x$を文字とする.
$\forall x(R)$と$\forall x(S)$が共に$\mathscr{T}$の定理ならば, 
$\forall x(R \wedge S)$は$\mathscr{T}$の定理である.
\end{dedu}




\mathstrut
\begin{dedu}
\label{dedallwrfree}%推論173
$R$と$S$を$\mathscr{T}$の関係式とし, $x$を$R$の中に自由変数として現れない文字とする.

1)
$\forall x(R \wedge S)$が$\mathscr{T}$の定理ならば, 
$R \wedge \forall x(S)$は$\mathscr{T}$の定理である.

2)
$R \wedge \forall x(S)$が$\mathscr{T}$の定理ならば, 
$\forall x(R \wedge S)$は$\mathscr{T}$の定理である.

3)
$\forall x(S \wedge R)$が$\mathscr{T}$の定理ならば, 
$\forall x(S) \wedge R$は$\mathscr{T}$の定理である.

4)
$\forall x(S) \wedge R$が$\mathscr{T}$の定理ならば, 
$\forall x(S \wedge R)$は$\mathscr{T}$の定理である.
\end{dedu}




\mathstrut
\begin{dedu}
\label{dedallwrfree2}%推論174
$R$と$S$を$\mathscr{T}$の関係式とし, $x$を$R$の中に自由変数として現れない文字とする.
$R$と$\forall x(S)$が共に$\mathscr{T}$の定理ならば, 
$\forall x(R \wedge S)$と$\forall x(S \wedge R)$は共に$\mathscr{T}$の定理である.
\end{dedu}




\mathstrut
\begin{dedu}
\label{dedquangvee}%推論175
$x$を文字とする.
また$n$を自然数とし, $R_{1}, R_{2}, \cdots, R_{n}$を$\mathscr{T}$の関係式とする.
また$i$を$n$以下の自然数とする.

1)
$\exists x(R_{i})$が$\mathscr{T}$の定理ならば, 
$\exists x(R_{1} \vee R_{2} \vee \cdots \vee R_{n})$は$\mathscr{T}$の定理である.

2)
$\forall x(R_{i})$が$\mathscr{T}$の定理ならば, 
$\forall x(R_{1} \vee R_{2} \vee \cdots \vee R_{n})$は$\mathscr{T}$の定理である.
\end{dedu}




\mathstrut
\begin{dedu}
\label{dedquangvee2}%推論176
$x$を文字とする.
また$n$を自然数とし, $R_{1}, R_{2}, \cdots, R_{n}$を$\mathscr{T}$の関係式とする.
また$k$を自然数とし, $i_{1}, i_{2}, \cdots, i_{k}$を$n$以下の自然数とする.

1)
$\exists x(R_{i_{1}} \vee R_{i_{2}} \vee \cdots \vee R_{i_{k}})$が$\mathscr{T}$の定理ならば, 
$\exists x(R_{1} \vee R_{2} \vee \cdots \vee R_{n})$は$\mathscr{T}$の定理である.

2)
$\forall x(R_{i_{1}} \vee R_{i_{2}} \vee \cdots \vee R_{i_{k}})$が$\mathscr{T}$の定理ならば, 
$\forall x(R_{1} \vee R_{2} \vee \cdots \vee R_{n})$は$\mathscr{T}$の定理である.
\end{dedu}




\mathstrut
\begin{dedu}
\label{dedexgv}%推論177
$x$を文字とする.
また$n$を自然数とし, $R_{1}, R_{2}, \cdots, R_{n}$を$\mathscr{T}$の関係式とする.

1)
$\exists x(R_{1} \vee R_{2} \vee \cdots \vee R_{n})$が$\mathscr{T}$の定理ならば, 
$\exists x(R_{1}) \vee \exists x(R_{2}) \vee \cdots \vee \exists x(R_{n})$は$\mathscr{T}$の定理である.

2)
$\exists x(R_{1}) \vee \exists x(R_{2}) \vee \cdots \vee \exists x(R_{n})$が$\mathscr{T}$の定理ならば, 
$\exists x(R_{1} \vee R_{2} \vee \cdots \vee R_{n})$は$\mathscr{T}$の定理である.
\end{dedu}




\mathstrut
\begin{dedu}
\label{dedexgvfree}%推論178
$n$を自然数とし, $R_{1}, R_{2}, \cdots, R_{n}$を$\mathscr{T}$の関係式とする.
また$k$を$n$以下の自然数とし, $i_{1}, i_{2}, \cdots, i_{k}$を
$i_{1} < i_{2} < \cdots < i_{k} \LEQQ n$なる自然数とする.
また$x$を$R_{i_{1}}, R_{i_{2}}, \cdots, R_{i_{k}}$の中に自由変数として現れない文字とする.

1)
$\exists x(R_{1} \vee R_{2} \vee \cdots \vee R_{n})$が$\mathscr{T}$の定理ならば, 
\[
\tag{$*$}
  \exists x(R_{1}) \vee \cdots \vee \exists x(R_{i_{1} - 1}) \vee R_{i_{1}} \vee \exists x(R_{i_{1} + 1}) \vee 
  \cdots\cdots \vee \exists x(R_{i_{k} - 1}) \vee R_{i_{k}} \vee \exists x(R_{i_{k} + 1}) \vee \cdots \vee \exists x(R_{n})
\]
は$\mathscr{T}$の定理である.

2)
($*$)が$\mathscr{T}$の定理ならば, 
$\exists x(R_{1} \vee R_{2} \vee \cdots \vee R_{n})$は$\mathscr{T}$の定理である.
\end{dedu}




\mathstrut
\begin{dedu}
\label{dedallgv}%推論179
$x$を文字とする.
また$n$を自然数とし, $R_{1}, R_{2}, \cdots, R_{n}$を$\mathscr{T}$の関係式とする.

1)
$\forall x(R_{1}) \vee \forall x(R_{2}) \vee \cdots \vee \forall x(R_{n})$が$\mathscr{T}$の定理ならば, 
$\forall x(R_{1} \vee R_{2} \vee \cdots \vee R_{n})$は$\mathscr{T}$の定理である.

2)
$\forall x(R_{1} \vee R_{2} \vee \cdots \vee R_{n})$が$\mathscr{T}$の定理ならば, 
$n$以下の任意の自然数$i$に対して
\[
  \exists x(R_{1}) \vee \cdots \vee \exists x(R_{i - 1}) \vee \forall x(R_{i}) \vee \exists x(R_{i + 1}) \vee \cdots \vee \exists x(R_{n})
\]
は$\mathscr{T}$の定理である.
\end{dedu}




\mathstrut
\begin{dedu}
\label{dedallgvfree}%推論180
$n$を自然数とし, $R_{1}, R_{2}, \cdots, R_{n}$を$\mathscr{T}$の関係式とする.
また$k$を$n$以下の自然数とし, $i_{1}, i_{2}, \cdots, i_{k}$を
$i_{1} < i_{2} < \cdots < i_{k} \LEQQ n$なる自然数とする.
また$x$を$R_{i_{1}}, R_{i_{2}}, \cdots, R_{i_{k}}$の中に自由変数として現れない文字とする.
このとき
\[
  \forall x(R_{1}) \vee \cdots \vee \forall x(R_{i_{1} - 1}) \vee R_{i_{1}} \vee \forall x(R_{i_{1} + 1}) \vee \cdots\cdots 
  \vee \forall x(R_{i_{k} - 1}) \vee R_{i_{k}} \vee \forall x(R_{i_{k} + 1}) \vee \cdots \vee \forall x(R_{n})
\]
が$\mathscr{T}$の定理ならば, 
$\forall x(R_{1} \vee R_{2} \vee \cdots \vee R_{n})$は$\mathscr{T}$の定理である.
\end{dedu}




\mathstrut
\begin{dedu}
\label{dedallgvfree2}%推論181
$n$を自然数とし, $R_{1}, R_{2}, \cdots, R_{n}$を$\mathscr{T}$の関係式とする.
また$i$を$n$以下の自然数とし, $x$を$R_{i}$の中に自由変数として現れない文字とする.
$R_{i}$が$\mathscr{T}$の定理ならば, 
$\forall x(R_{1} \vee R_{2} \vee \cdots \vee R_{n})$は$\mathscr{T}$の定理である.
\end{dedu}




\mathstrut
\begin{dedu}
\label{dedallgvfreeeq}%推論182
$x$を文字とする.
また$n$を自然数とし, $R_{1}, R_{2}, \cdots, R_{n}$を$\mathscr{T}$の関係式とする.
また$i$を$n$以下の自然数とする.
いま$i$と異なる$n$以下の任意の自然数$j$に対し, $x$は$R_{j}$の中に自由変数として現れないとする.

1)
$\forall x(R_{1} \vee R_{2} \vee \cdots \vee R_{n})$が$\mathscr{T}$の定理ならば, 
$R_{1} \vee \cdots \vee R_{i - 1} \vee \forall x(R_{i}) \vee R_{i + 1} \vee \cdots \vee R_{n}$は$\mathscr{T}$の定理である.

2)
$R_{1} \vee \cdots \vee R_{i - 1} \vee \forall x(R_{i}) \vee R_{i + 1} \vee \cdots \vee R_{n}$が$\mathscr{T}$の定理ならば, 
$\forall x(R_{1} \vee R_{2} \vee \cdots \vee R_{n})$は$\mathscr{T}$の定理である.
\end{dedu}




\mathstrut
\begin{dedu}
\label{dedquangwedge}%推論183
$x$を文字とする.
また$n$を自然数とし, $R_{1}, R_{2}, \cdots, R_{n}$を$\mathscr{T}$の関係式とする.
また$i$を$n$以下の自然数とする.

1)
$\exists x(R_{1} \wedge R_{2} \wedge \cdots \wedge R_{n})$が$\mathscr{T}$の定理ならば, 
$\exists x(R_{i})$は$\mathscr{T}$の定理である.

2)
$\forall x(R_{1} \wedge R_{2} \wedge \cdots \wedge R_{n})$が$\mathscr{T}$の定理ならば, 
$\forall x(R_{i})$は$\mathscr{T}$の定理である.
\end{dedu}




\mathstrut
\begin{dedu}
\label{dedquangwedge2}%推論184
$x$を文字とする.
また$n$を自然数とし, $R_{1}, R_{2}, \cdots, R_{n}$を$\mathscr{T}$の関係式とする.
また$k$を自然数とし, $i_{1}, i_{2}, \cdots, i_{k}$を$n$以下の自然数とする.

1)
$\exists x(R_{1} \wedge R_{2} \wedge \cdots \wedge R_{n})$が$\mathscr{T}$の定理ならば, 
$\exists x(R_{i_{1}} \wedge R_{i_{2}} \wedge \cdots \wedge R_{i_{k}})$は$\mathscr{T}$の定理である.

2)
$\forall x(R_{1} \wedge R_{2} \wedge \cdots \wedge R_{n})$が$\mathscr{T}$の定理ならば, 
$\forall x(R_{i_{1}} \wedge R_{i_{2}} \wedge \cdots \wedge R_{i_{k}})$は$\mathscr{T}$の定理である.
\end{dedu}




\mathstrut
\begin{dedu}
\label{dedexgw}%推論185
$x$を文字とする.
また$n$を自然数とし, $R_{1}, R_{2}, \cdots, R_{n}$を$\mathscr{T}$の関係式とする.

1)
$\exists x(R_{1} \wedge R_{2} \wedge \cdots \wedge R_{n})$が$\mathscr{T}$の定理ならば, 
$\exists x(R_{1}) \wedge \exists x(R_{2}) \wedge \cdots \wedge \exists x(R_{n})$は$\mathscr{T}$の定理である.

2)
$i$を$n$以下の自然数とする.
\[
  \forall x(R_{1}) \wedge \cdots \wedge \forall x(R_{i - 1}) \wedge \exists x(R_{i}) \wedge \forall x(R_{i + 1}) \wedge \cdots \wedge \forall x(R_{n})
\]
が$\mathscr{T}$の定理ならば, 
$\exists x(R_{1} \wedge R_{2} \wedge \cdots \wedge R_{n})$は$\mathscr{T}$の定理である.
\end{dedu}




\mathstrut
\begin{dedu}
\label{dedexgw2}%推論186
$x$を文字とする.
また$n$を自然数とし, $R_{1}, R_{2}, \cdots, R_{n}$を$\mathscr{T}$の関係式とする.
また$i$を$n$以下の自然数とする.
$\forall x(R_{1}), \cdots, \forall x(R_{i - 1}), \exists x(R_{i}), \forall x(R_{i + 1}), \cdots, \forall x(R_{n})$が
すべて$\mathscr{T}$の定理ならば, 
$\exists x(R_{1} \wedge R_{2} \wedge \cdots \wedge R_{n})$は$\mathscr{T}$の定理である.
\end{dedu}




\mathstrut
\begin{dedu}
\label{dedexgwfree}%推論187
$n$を自然数とし, $R_{1}, R_{2}, \cdots, R_{n}$を$\mathscr{T}$の関係式とする.
また$k$を$n$以下の自然数とし, $i_{1}, i_{2}, \cdots, i_{k}$を
$i_{1} < i_{2} < \cdots < i_{k} \LEQQ n$なる自然数とする.
また$x$を$R_{i_{1}}, R_{i_{2}}, \cdots, R_{i_{k}}$の中に自由変数として現れない文字とする.
このとき$\exists x(R_{1} \wedge R_{2} \wedge \cdots \wedge R_{n})$が$\mathscr{T}$の定理ならば, 
\[
  \exists x(R_{1}) \wedge \cdots \wedge \exists x(R_{i_{1} - 1}) \wedge R_{i_{1}} \wedge \exists x(R_{i_{1} + 1}) \wedge \cdots\cdots 
  \wedge \exists x(R_{i_{k} - 1}) \wedge R_{i_{k}} \wedge \exists x(R_{i_{k} + 1}) \wedge \cdots \wedge \exists x(R_{n})
\]
は$\mathscr{T}$の定理である.
\end{dedu}




\mathstrut
\begin{dedu}
\label{dedexgwfree2}%推論188
$n$を自然数とし, $R_{1}, R_{2}, \cdots, R_{n}$を$\mathscr{T}$の関係式とする.
また$i$を$n$以下の自然数とし, $x$を$R_{i}$の中に自由変数として現れない文字とする.
$\exists x(R_{1} \wedge R_{2} \wedge \cdots \wedge R_{n})$が$\mathscr{T}$の定理ならば, 
$R_{i}$は$\mathscr{T}$の定理である.
\end{dedu}




\mathstrut
\begin{dedu}
\label{dedexgwfreeeq}%推論189
$x$を文字とする.
また$n$を自然数とし, $R_{1}, R_{2}, \cdots, R_{n}$を$\mathscr{T}$の関係式とする.
また$i$を$n$以下の自然数とする.
いま$i$と異なる$n$以下の任意の自然数$j$に対し, $x$は$R_{j}$の中に自由変数として現れないとする.

1)
$\exists x(R_{1} \wedge R_{2} \wedge \cdots \wedge R_{n})$が$\mathscr{T}$の定理ならば, 
$R_{1} \wedge \cdots \wedge R_{i - 1} \wedge \exists x(R_{i}) \wedge R_{i + 1} \wedge \cdots \wedge R_{n}$は$\mathscr{T}$の定理である.

2)
$R_{1} \wedge \cdots \wedge R_{i - 1} \wedge \exists x(R_{i}) \wedge R_{i + 1} \wedge \cdots \wedge R_{n}$が$\mathscr{T}$の定理ならば, 
$\exists x(R_{1} \wedge R_{2} \wedge \cdots \wedge R_{n})$は$\mathscr{T}$の定理である.
\end{dedu}




\mathstrut
\begin{dedu}
\label{dedexgwfreeeq2}%推論190
$x$を文字とする.
また$n$を自然数とし, $R_{1}, R_{2}, \cdots, R_{n}$を$\mathscr{T}$の関係式とする.
また$i$を$n$以下の自然数とする.
いま$i$と異なる$n$以下の任意の自然数$j$に対し, $x$は$R_{j}$の中に自由変数として現れないとする.
このとき$R_{1}, \cdots, R_{i - 1}, \exists x(R_{i}), R_{i + 1}, \cdots, R_{n}$が
すべて$\mathscr{T}$の定理ならば, 
$\exists x(R_{1} \wedge R_{2} \wedge \cdots \wedge R_{n})$は$\mathscr{T}$の定理である.
\end{dedu}




\mathstrut
\begin{dedu}
\label{dedallgw}%推論191
$x$を文字とする.
また$n$を自然数とし, $R_{1}, R_{2}, \cdots, R_{n}$を$\mathscr{T}$の関係式とする.

1)
$\forall x(R_{1} \wedge R_{2} \wedge \cdots \wedge R_{n})$が$\mathscr{T}$の定理ならば, 
$\forall x(R_{1}) \wedge \forall x(R_{2}) \wedge \cdots \wedge \forall x(R_{n})$は$\mathscr{T}$の定理である.

2)
$\forall x(R_{1}) \wedge \forall x(R_{2}) \wedge \cdots \wedge \forall x(R_{n})$が$\mathscr{T}$の定理ならば, 
$\forall x(R_{1} \wedge R_{2} \wedge \cdots \wedge R_{n})$は$\mathscr{T}$の定理である.
\end{dedu}




\mathstrut
\begin{dedu}
\label{dedallgw2}%推論192
$x$を文字とする.
また$n$を自然数とし, $R_{1}, R_{2}, \cdots, R_{n}$を$\mathscr{T}$の関係式とする.
$\forall x(R_{1}), \forall x(R_{2}), \cdots, \forall x(R_{n})$がすべて$\mathscr{T}$の定理ならば, 
$\forall x(R_{1} \wedge R_{2} \wedge \cdots \wedge R_{n})$は$\mathscr{T}$の定理である.
\end{dedu}




\mathstrut
\begin{dedu}
\label{dedallgwfree}%推論193
$n$を自然数とし, $R_{1}, R_{2}, \cdots, R_{n}$を$\mathscr{T}$の関係式とする.
また$k$を$n$以下の自然数とし, $i_{1}, i_{2}, \cdots, i_{k}$を
$i_{1} < i_{2} < \cdots < i_{k} \LEQQ n$なる自然数とする.
また$x$を$R_{i_{1}}, R_{i_{2}}, \cdots, R_{i_{k}}$の中に自由変数として現れない文字とする.

1)
$\forall x(R_{1} \wedge R_{2} \wedge \cdots \wedge R_{n})$が$\mathscr{T}$の定理ならば, 
\[
\tag{$*$}
  \forall x(R_{1}) \wedge \cdots \wedge \forall x(R_{i_{1} - 1}) \wedge R_{i_{1}} \wedge \forall x(R_{i_{1} + 1}) \wedge \cdots\cdots 
  \wedge \forall x(R_{i_{k} - 1}) \wedge R_{i_{k}} \wedge \forall x(R_{i_{k} + 1}) \wedge \cdots \wedge \forall x(R_{n})
\]
は$\mathscr{T}$の定理である.

2)
($*$)が$\mathscr{T}$の定理ならば, 
$\forall x(R_{1} \wedge R_{2} \wedge \cdots \wedge R_{n})$は$\mathscr{T}$の定理である.
\end{dedu}




\mathstrut
\begin{dedu}
\label{dedallgwfree2}%推論194
$n$を自然数とし, $R_{1}, R_{2}, \cdots, R_{n}$を$\mathscr{T}$の関係式とする.
また$k$を$n$以下の自然数とし, $i_{1}, i_{2}, \cdots, i_{k}$を
$i_{1} < i_{2} < \cdots < i_{k} \LEQQ n$なる自然数とする.
また$x$を$R_{i_{1}}, R_{i_{2}}, \cdots, R_{i_{k}}$の中に自由変数として現れない文字とする.
このとき
\[
  \forall x(R_{1}), \cdots, \forall x(R_{i_{1} - 1}), R_{i_{1}}, \forall x(R_{i_{1} + 1}), \cdots\cdots, 
  \forall x(R_{i_{k} - 1}), R_{i_{k}}, \forall x(R_{i_{k} + 1}), \cdots, \forall x(R_{n})
\]
がすべて$\mathscr{T}$の定理ならば, 
$\forall x(R_{1} \wedge R_{2} \wedge \cdots \wedge R_{n})$は$\mathscr{T}$の定理である.
\end{dedu}




\mathstrut
\begin{dedu}
\label{dedextquansep}%推論195
$R$と$S$を$\mathscr{T}$の関係式とし, $x$を文字とする.

1)
$\exists x(R \to S)$が$\mathscr{T}$の定理ならば, 
$\forall x(R) \to \exists x(S)$は$\mathscr{T}$の定理である.

2)
$\forall x(R) \to \exists x(S)$が$\mathscr{T}$の定理ならば, 
$\exists x(R \to S)$は$\mathscr{T}$の定理である.
\end{dedu}




\mathstrut
\begin{dedu}
\label{dedextquansep2}%推論196
$R$と$S$を$\mathscr{T}$の関係式とし, $x$を文字とする.

1)
$\exists x(S)$が$\mathscr{T}$の定理ならば, 
$\exists x(R \to S)$は$\mathscr{T}$の定理である.

2)
$\neg \forall x(R)$が$\mathscr{T}$の定理ならば, 
$\exists x(R \to S)$は$\mathscr{T}$の定理である.

3)
$\exists x(\neg R)$が$\mathscr{T}$の定理ならば, 
$\exists x(R \to S)$は$\mathscr{T}$の定理である.
\end{dedu}




\mathstrut
\begin{dedu}
\label{dedextquansepfree}%推論197
$R$と$S$を$\mathscr{T}$の関係式とし, $x$を1), 2)では$R$の中に自由変数として現れない文字, 
3), 4)では$S$の中に自由変数として現れない文字とする.

1)
$\exists x(R \to S)$が$\mathscr{T}$の定理ならば, 
$R \to \exists x(S)$は$\mathscr{T}$の定理である.

2)
$R \to \exists x(S)$が$\mathscr{T}$の定理ならば, 
$\exists x(R \to S)$は$\mathscr{T}$の定理である.

3)
$\exists x(R \to S)$が$\mathscr{T}$の定理ならば, 
$\forall x(R) \to S$は$\mathscr{T}$の定理である.

4)
$\forall x(R) \to S$が$\mathscr{T}$の定理ならば, 
$\exists x(R \to S)$は$\mathscr{T}$の定理である.
\end{dedu}




\mathstrut
\begin{dedu}
\label{dedalltquansep}%推論198
$R$と$S$を$\mathscr{T}$の関係式とし, $x$を文字とする.

1)
$\forall x(R \to S)$が$\mathscr{T}$の定理ならば, 
$\exists x(R) \to \exists x(S)$と$\forall x(R) \to \forall x(S)$は共に$\mathscr{T}$の定理である.

2)
$\exists x(R) \to \forall x(S)$が$\mathscr{T}$の定理ならば, 
$\forall x(R \to S)$は$\mathscr{T}$の定理である.
\end{dedu}




\mathstrut
\begin{dedu}
\label{dedalltquansepconst}%推論199
$R$と$S$を$\mathscr{T}$の関係式とし, $x$を$\mathscr{T}$の定数でない文字とする.
$R \to S$が$\mathscr{T}$の定理ならば, 
$\exists x(R) \to \exists x(S)$と$\forall x(R) \to \forall x(S)$は共に$\mathscr{T}$の定理である.
\end{dedu}




\mathstrut
\begin{dedu}
\label{dedquanseptall2}%推論200
$R$と$S$を$\mathscr{T}$の関係式とし, $x$を文字とする.

1)
$\forall x(S)$が$\mathscr{T}$の定理ならば, 
$\forall x(R \to S)$は$\mathscr{T}$の定理である.

2)
$\neg \exists x(R)$が$\mathscr{T}$の定理ならば, 
$\forall x(R \to S)$は$\mathscr{T}$の定理である.

3)
$\forall x(\neg R)$が$\mathscr{T}$の定理ならば, 
$\forall x(R \to S)$は$\mathscr{T}$の定理である.
\end{dedu}




\mathstrut
\begin{dedu}
\label{dedquanseptall2const}%推論201
$R$と$S$を$\mathscr{T}$の関係式とし, $x$を$\mathscr{T}$の定数でない文字とする.

1)
$S$が$\mathscr{T}$の定理ならば, 
$\forall x(R \to S)$は$\mathscr{T}$の定理である.

2)
$\neg R$が$\mathscr{T}$の定理ならば, 
$\forall x(R \to S)$は$\mathscr{T}$の定理である.
\end{dedu}




\mathstrut
\begin{dedu}
\label{dedalltquansepfree}%推論202
$R$と$S$を$\mathscr{T}$の関係式とし, $x$を1), 2)では$R$の中に自由変数として現れない文字, 
3), 4)では$S$の中に自由変数として現れない文字とする.

1)
$\forall x(R \to S)$が$\mathscr{T}$の定理ならば, 
$R \to \forall x(S)$は$\mathscr{T}$の定理である.

2)
$R \to \forall x(S)$が$\mathscr{T}$の定理ならば, 
$\forall x(R \to S)$は$\mathscr{T}$の定理である.

3)
$\forall x(R \to S)$が$\mathscr{T}$の定理ならば, 
$\exists x(R) \to S$は$\mathscr{T}$の定理である.

4)
$\exists x(R) \to S$が$\mathscr{T}$の定理ならば, 
$\forall x(R \to S)$は$\mathscr{T}$の定理である.
\end{dedu}




\mathstrut
\begin{dedu}
\label{dedalltquansepfreeconst}%推論203
$R$と$S$を$\mathscr{T}$の関係式とし, $R \to S$が$\mathscr{T}$の定理であるとする.
また$x$を$\mathscr{T}$の定数でない文字とする.

1)
$x$が$R$の中に自由変数として現れなければ, 
$R \to \forall x(S)$は$\mathscr{T}$の定理である.

2)
$x$が$S$の中に自由変数として現れなければ, 
$\exists x(R) \to S$は$\mathscr{T}$の定理である.
\end{dedu}




\mathstrut
\begin{dedu}
\label{dedquansepeqex}%推論204
$R$と$S$を$\mathscr{T}$の関係式とし, $x$を文字とする.

1)
$\exists x(R) \leftrightarrow \forall x(S)$が$\mathscr{T}$の定理ならば, 
$\exists x(R \leftrightarrow S)$は$\mathscr{T}$の定理である.

2)
$\forall x(R) \leftrightarrow \exists x(S)$が$\mathscr{T}$の定理ならば, 
$\exists x(R \leftrightarrow S)$は$\mathscr{T}$の定理である.
\end{dedu}




\mathstrut
\begin{dedu}
\label{dedquansepeqexfree}%推論205
$R$と$S$を$\mathscr{T}$の関係式とし, $x$を1), 2)では$R$の中に自由変数として現れない文字, 
3), 4)では$S$の中に自由変数として現れない文字とする.

1)
$R \leftrightarrow \forall x(S)$が$\mathscr{T}$の定理ならば, 
$\exists x(R \leftrightarrow S)$は$\mathscr{T}$の定理である.

2)
$R \leftrightarrow \exists x(S)$が$\mathscr{T}$の定理ならば, 
$\exists x(R \leftrightarrow S)$は$\mathscr{T}$の定理である.

3)
$\exists x(R) \leftrightarrow S$が$\mathscr{T}$の定理ならば, 
$\exists x(R \leftrightarrow S)$は$\mathscr{T}$の定理である.

4)
$\forall x(R) \leftrightarrow S$が$\mathscr{T}$の定理ならば, 
$\exists x(R \leftrightarrow S)$は$\mathscr{T}$の定理である.
\end{dedu}




\mathstrut
\begin{dedu}
\label{dedalleqquansep}%推論206
$R$と$S$を$\mathscr{T}$の関係式とし, $x$を文字とする.
$\forall x(R \leftrightarrow S)$が$\mathscr{T}$の定理ならば, 
$\exists x(R) \leftrightarrow \exists x(S)$と
$\forall x(R) \leftrightarrow \forall x(S)$は共に$\mathscr{T}$の定理である.
\end{dedu}




\mathstrut
\begin{dedu}
\label{dedalleqquansepconst}%推論207
$R$と$S$を$\mathscr{T}$の関係式とし, $x$を$\mathscr{T}$の定数でない文字とする.
$R \leftrightarrow S$が$\mathscr{T}$の定理ならば, 
$\exists x(R) \leftrightarrow \exists x(S)$と
$\forall x(R) \leftrightarrow \forall x(S)$は共に$\mathscr{T}$の定理である.
\end{dedu}




\mathstrut
\begin{dedu}
\label{dedalleqquansepfree}%推論208
$R$と$S$を$\mathscr{T}$の関係式とし, $x$を1)では$R$の中に自由変数として現れない文字, 
2)では$S$の中に自由変数として現れない文字とする.

1)
$\forall x(R \leftrightarrow S)$が$\mathscr{T}$の定理ならば, 
$R \leftrightarrow \exists x(S)$と$R \leftrightarrow \forall x(S)$は共に$\mathscr{T}$の定理である.

2)
$\forall x(R \leftrightarrow S)$が$\mathscr{T}$の定理ならば, 
$\exists x(R) \leftrightarrow S$と$\forall x(R) \leftrightarrow S$は共に$\mathscr{T}$の定理である.
\end{dedu}




\mathstrut
\begin{dedu}
\label{dedalleqquansepfreeconst}%推論209
$R$と$S$を$\mathscr{T}$の関係式とし, $R \leftrightarrow S$が$\mathscr{T}$の定理であるとする.
また$x$を$\mathscr{T}$の定数でない文字とする.

1)
$x$が$R$の中に自由変数として現れなければ, 
$R \leftrightarrow \exists x(S)$と$R \leftrightarrow \forall x(S)$は共に$\mathscr{T}$の定理である.

2)
$x$が$S$の中に自由変数として現れなければ, 
$\exists x(R) \leftrightarrow S$と$\forall x(R) \leftrightarrow S$は共に$\mathscr{T}$の定理である.
\end{dedu}




\mathstrut
\begin{dedu}
\label{dedquanch}%推論210
$R$を$\mathscr{T}$の関係式とし, $x$と$y$を文字とする.

1)
$\exists x(\exists y(R))$が$\mathscr{T}$の定理ならば, 
$\exists y(\exists x(R))$は$\mathscr{T}$の定理である.

2)
$\forall x(\forall y(R))$が$\mathscr{T}$の定理ならば, 
$\forall y(\forall x(R))$は$\mathscr{T}$の定理である.

3)
$\exists x(\forall y(R))$が$\mathscr{T}$の定理ならば, 
$\forall y(\exists x(R))$は$\mathscr{T}$の定理である.
\end{dedu}




\mathstrut
\begin{dedu}
\label{dedgquanch}%推論211
$R$を$\mathscr{T}$の関係式とする.
また$n$を自然数とし, $x_{1}, x_{2}, \cdots, x_{n}$を文字とする.
また自然数$1, 2, \cdots, n$の順序を任意に入れ替えたものを
$i_{1}, i_{2}, \cdots, i_{n}$とする.

1)
$\exists x_{1}(\exists x_{2}( \cdots (\exists x_{n}(R)) \cdots ))$が$\mathscr{T}$の定理ならば, 
$\exists x_{i_{1}}(\exists x_{i_{2}}( \cdots (\exists x_{i_{n}}(R)) \cdots ))$は$\mathscr{T}$の定理である.

2)
$\forall x_{1}(\forall x_{2}( \cdots (\forall x_{n}(R)) \cdots ))$が$\mathscr{T}$の定理ならば, 
$\forall x_{i_{1}}(\forall x_{i_{2}}( \cdots (\forall x_{i_{n}}(R)) \cdots ))$は$\mathscr{T}$の定理である.
\end{dedu}




\mathstrut
\begin{dedu}
\label{dedspallfund}%推論212
$A$と$R$を$\mathscr{T}$の関係式とし, $x$を文字とする.

1)
$\forall_{A}x(R)$が$\mathscr{T}$の定理ならば, 
$\forall x(A \to R)$は$\mathscr{T}$の定理である.

2)
$\forall x(A \to R)$が$\mathscr{T}$の定理ならば, 
$\forall_{A}x(R)$は$\mathscr{T}$の定理である.

3)
$x$は$\mathscr{T}$の定数でないとする.
このとき$A \to R$が$\mathscr{T}$の定理ならば, 
$\forall_{A}x(R)$は$\mathscr{T}$の定理である.
\end{dedu}




\mathstrut
\begin{dedu}
\label{dedspallfund2}%推論213
$A$と$R$を$\mathscr{T}$の関係式とし, $x$を文字とする.

1)
$\forall_{A}x(R)$が$\mathscr{T}$の定理ならば, 
$(\tau_{x}(\neg (A \to R))|x)(A \to R)$は$\mathscr{T}$の定理である.

2)
$(\tau_{x}(\neg (A \to R))|x)(A \to R)$が$\mathscr{T}$の定理ならば, 
$\forall_{A}x(R)$は$\mathscr{T}$の定理である.
\end{dedu}




\mathstrut
\begin{dedu}
\label{dedspalldeduction}%推論214
$A$と$R$を$\mathscr{T}$の関係式とし, $x$を$\mathscr{T}$の定数でない文字とする.
また$\mathscr{T}$の明示的公理に$A$を追加して得られる理論を$\mathscr{T}'$とする.
$R$が$\mathscr{T}'$の定理ならば, $\forall_{A}x(R)$は$\mathscr{T}$の定理である.
\end{dedu}




\mathstrut
\begin{dedu}
\label{dedspallraa}%推論215
$A$と$R$を$\mathscr{T}$の関係式とし, $x$を$\mathscr{T}$の定数でない文字とする.
また$\mathscr{T}$の明示的公理に$A$と$\neg R$を追加して得られる理論を$\mathscr{T}'$とする.
$\mathscr{T}'$が矛盾すれば, $\forall_{A}x(R)$は$\mathscr{T}$の定理である.
\end{dedu}




\mathstrut
\begin{dedu}
\label{dedspex&quan}%推論216
$A$と$R$を$\mathscr{T}$の関係式とし, $x$を文字とする.

1)
$\exists_{A}x(R)$が$\mathscr{T}$の定理ならば, 
$\exists x(A)$と$\exists x(R)$は共に$\mathscr{T}$の定理である.

2)
$\exists x(A)$と$\forall x(R)$が共に$\mathscr{T}$の定理ならば, 
$\exists_{A}x(R)$は$\mathscr{T}$の定理である.

3)
$\forall x(A)$と$\exists x(R)$が共に$\mathscr{T}$の定理ならば, 
$\exists_{A}x(R)$は$\mathscr{T}$の定理である.
\end{dedu}




\mathstrut
\begin{dedu}
\label{dedspex&quanconst}%推論217
$A$と$R$を$\mathscr{T}$の関係式とし, $x$を$\mathscr{T}$の定数でない文字とする.

1)
$\exists x(A)$と$R$が共に$\mathscr{T}$の定理ならば, 
$\exists_{A}x(R)$は$\mathscr{T}$の定理である.

2)
$A$と$\exists x(R)$が共に$\mathscr{T}$の定理ならば, 
$\exists_{A}x(R)$は$\mathscr{T}$の定理である.
\end{dedu}




\mathstrut
\begin{dedu}
\label{dedspexneg}%推論218
$A$と$R$を$\mathscr{T}$の関係式とし, $x$を文字とする.

1)
$\neg \exists x(A)$が$\mathscr{T}$の定理ならば, 
$\neg \exists_{A}x(R)$は$\mathscr{T}$の定理である.

2)
$\neg \exists x(R)$が$\mathscr{T}$の定理ならば, 
$\neg \exists_{A}x(R)$は$\mathscr{T}$の定理である.

3)
$\forall x(\neg A)$が$\mathscr{T}$の定理ならば, 
$\neg \exists_{A}x(R)$は$\mathscr{T}$の定理である.

4)
$\forall x(\neg R)$が$\mathscr{T}$の定理ならば, 
$\neg \exists_{A}x(R)$は$\mathscr{T}$の定理である.
\end{dedu}




\mathstrut
\begin{dedu}
\label{dedspexnegconst}%推論219
$A$と$R$を$\mathscr{T}$の関係式とし, $x$を$\mathscr{T}$の定数でない文字とする.

1)
$\neg A$が$\mathscr{T}$の定理ならば, 
$\neg \exists_{A}x(R)$は$\mathscr{T}$の定理である.

2)
$\neg R$が$\mathscr{T}$の定理ならば, 
$\neg \exists_{A}x(R)$は$\mathscr{T}$の定理である.
\end{dedu}




\mathstrut
\begin{dedu}
\label{dedspall&quan}%推論220
$A$と$R$を$\mathscr{T}$の関係式とし, $x$を文字とする.

1)
$\forall_{A}x(R)$が$\mathscr{T}$の定理ならば, 
$\exists x(A) \to \exists x(R)$と$\forall x(A) \to \forall x(R)$は共に$\mathscr{T}$の定理である.

2)
$\exists x(A) \to \forall x(R)$が$\mathscr{T}$の定理ならば, 
$\forall_{A}x(R)$は$\mathscr{T}$の定理である.
\end{dedu}




\mathstrut
\begin{dedu}
\label{dedquantspall2}%推論221
$A$と$R$を$\mathscr{T}$の関係式とし, $x$を文字とする.

1)
$\forall x(R)$が$\mathscr{T}$の定理ならば, 
$\forall_{A}x(R)$は$\mathscr{T}$の定理である.

2)
$\neg \exists x(A)$が$\mathscr{T}$の定理ならば, 
$\forall_{A}x(R)$は$\mathscr{T}$の定理である.

3)
$\forall x(\neg A)$が$\mathscr{T}$の定理ならば, 
$\forall_{A}x(R)$は$\mathscr{T}$の定理である.
\end{dedu}




\mathstrut
\begin{dedu}
\label{dedquantspall2const}%推論222
$A$と$R$を$\mathscr{T}$の関係式とし, $x$を$\mathscr{T}$の定数でない文字とする.

1)
$R$が$\mathscr{T}$の定理ならば, 
$\forall_{A}x(R)$は$\mathscr{T}$の定理である.

2)
$\neg A$が$\mathscr{T}$の定理ならば, 
$\forall_{A}x(R)$は$\mathscr{T}$の定理である.
\end{dedu}




\mathstrut
\begin{dedu}
\label{dedspallneg}%推論223
$A$と$R$を$\mathscr{T}$の関係式とし, $x$を文字とする.

1)
$\exists x(A)$と$\neg \exists x(R)$が共に$\mathscr{T}$の定理ならば, 
$\neg \forall_{A}x(R)$は$\mathscr{T}$の定理である.

2)
$\forall x(A)$と$\neg \forall x(R)$が共に$\mathscr{T}$の定理ならば, 
$\neg \forall_{A}x(R)$は$\mathscr{T}$の定理である.

3)
$\exists x(A)$と$\forall x(\neg R)$が共に$\mathscr{T}$の定理ならば, 
$\neg \forall_{A}x(R)$は$\mathscr{T}$の定理である.

4)
$\forall x(A)$と$\exists x(\neg R)$が共に$\mathscr{T}$の定理ならば, 
$\neg \forall_{A}x(R)$は$\mathscr{T}$の定理である.
\end{dedu}




\mathstrut
\begin{dedu}
\label{dedspallnegconst}%推論224
$A$と$R$を$\mathscr{T}$の関係式とし, $x$を$\mathscr{T}$の定数でない文字とする.

1)
$\exists x(A)$と$\neg R$が共に$\mathscr{T}$の定理ならば, 
$\neg \forall_{A}x(R)$は$\mathscr{T}$の定理である.

2)
$A$と$\exists x(\neg R)$が共に$\mathscr{T}$の定理ならば, 
$\neg \forall_{A}x(R)$は$\mathscr{T}$の定理である.
\end{dedu}




\mathstrut
\begin{dedu}
\label{dedexeqspex}%推論225
$A$と$R$を$\mathscr{T}$の関係式, $x$を$\mathscr{T}$の定数でない文字とし, 
次のa), b), c)のいずれかが成り立つとする: 

a)
$R \to A$は$\mathscr{T}$の定理である.

b)
$A$は$\mathscr{T}$の定理である.

c)
$\neg R$は$\mathscr{T}$の定理である.

このとき$\exists x(R) \leftrightarrow \exists_{A}x(R)$は
$\mathscr{T}$の定理である.
\end{dedu}




\mathstrut
\begin{dedu}
\label{dedalleqspall}%推論226
$A$と$R$を$\mathscr{T}$の関係式, $x$を$\mathscr{T}$の定数でない文字とし, 
次のa), b), c)のいずれかが成り立つとする: 

a)
$\neg R \to A$は$\mathscr{T}$の定理である.

b)
$A$は$\mathscr{T}$の定理である.

c)
$R$は$\mathscr{T}$の定理である.

このとき$\forall x(R) \leftrightarrow \forall_{A}x(R)$は
$\mathscr{T}$の定理である.
\end{dedu}




\mathstrut
\begin{dedu}
\label{dedspquanafree}%推論227
$A$と$R$を$\mathscr{T}$の関係式とし, $x$を$A$の中に自由変数として現れない文字とする.

1)
$\exists_{A}x(R)$が$\mathscr{T}$の定理ならば, $A \wedge \exists x(R)$は$\mathscr{T}$の定理である.

2)
$A \wedge \exists x(R)$が$\mathscr{T}$の定理ならば, $\exists_{A}x(R)$は$\mathscr{T}$の定理である.

3)
$\forall_{A}x(R)$が$\mathscr{T}$の定理ならば, $A \to \forall x(R)$は$\mathscr{T}$の定理である.

4)
$A \to \forall x(R)$が$\mathscr{T}$の定理ならば, $\forall_{A}x(R)$は$\mathscr{T}$の定理である.
\end{dedu}




\mathstrut
\begin{dedu}
\label{dedspexafree}%推論228
$A$と$R$を$\mathscr{T}$の関係式とし, $x$を$A$の中に自由変数として現れない文字とする.
$A$と$\exists x(R)$が共に$\mathscr{T}$の定理ならば, $\exists_{A}x(R)$は$\mathscr{T}$の定理である.
\end{dedu}




\mathstrut
\begin{dedu}
\label{dedspquanafree2}%推論229
$A$と$R$を$\mathscr{T}$の関係式とし, $x$を$A$の中に自由変数として現れない文字とする.

1)
$\exists_{A}x(R)$が$\mathscr{T}$の定理ならば, $A$は$\mathscr{T}$の定理である.

2)
$\neg A$が$\mathscr{T}$の定理ならば, $\forall_{A}x(R)$は$\mathscr{T}$の定理である.
\end{dedu}




\mathstrut
\begin{dedu}
\label{dedspquanrfree}%推論230
$A$と$R$を$\mathscr{T}$の関係式とし, $x$を$R$の中に自由変数として現れない文字とする.

1)
$\exists_{A}x(R)$が$\mathscr{T}$の定理ならば, $\exists x(A) \wedge R$は$\mathscr{T}$の定理である.

2)
$\exists x(A) \wedge R$が$\mathscr{T}$の定理ならば, $\exists_{A}x(R)$は$\mathscr{T}$の定理である.

3)
$\forall_{A}x(R)$が$\mathscr{T}$の定理ならば, $\exists x(A) \to R$は$\mathscr{T}$の定理である.

4)
$\exists x(A) \to R$が$\mathscr{T}$の定理ならば, $\forall_{A}x(R)$は$\mathscr{T}$の定理である.
\end{dedu}




\mathstrut
\begin{dedu}
\label{dedspexrfree}%推論231
$A$と$R$を$\mathscr{T}$の関係式とし, $x$を$R$の中に自由変数として現れない文字とする.
$\exists x(A)$と$R$が共に$\mathscr{T}$の定理ならば, $\exists_{A}x(R)$は$\mathscr{T}$の定理である.
\end{dedu}




\mathstrut
\begin{dedu}
\label{dedspquanrfree2}%推論232
$A$と$R$を$\mathscr{T}$の関係式とし, $x$を$R$の中に自由変数として現れない文字とする.

1)
$\exists_{A}x(R)$が$\mathscr{T}$の定理ならば, $R$は$\mathscr{T}$の定理である.

2)
$R$が$\mathscr{T}$の定理ならば, $\forall_{A}x(R)$は$\mathscr{T}$の定理である.
\end{dedu}




\mathstrut
\begin{dedu}
\label{dedaespquandm}%推論233
$A$と$R$を論理的な理論$\mathscr{T}$の関係式とし, $x$を文字とする.

1)
$\neg \forall_{A}x(R)$が$\mathscr{T}$の定理ならば, 
$\exists_{A}x(\neg R)$は$\mathscr{T}$の定理である.

2)
$\exists_{A}x(\neg R)$が$\mathscr{T}$の定理ならば, 
$\neg \forall_{A}x(R)$は$\mathscr{T}$の定理である.
\end{dedu}




\mathstrut
\begin{dedu}
\label{dedeaspquandm}%推論234
$A$と$R$を$\mathscr{T}$の関係式とし, $x$を文字とする.

1)
$\neg \exists_{A}x(R)$が$\mathscr{T}$の定理ならば, 
$\forall_{A}x(\neg R)$は$\mathscr{T}$の定理である.

2)
$\forall_{A}x(\neg R)$が$\mathscr{T}$の定理ならば, 
$\neg \exists_{A}x(R)$は$\mathscr{T}$の定理である.
\end{dedu}




\mathstrut
\begin{dedu}
\label{dedspquangdm}%推論235
$R$を$\mathscr{T}$の関係式とする.
また$n$を自然数とし, $A_{1}, A_{2}, \cdots, A_{n}$を$\mathscr{T}$の関係式, 
$x_{1}, x_{2}, \cdots, x_{n}$を文字とする.
また$n$以下の各自然数$i$に対し, $p^{i}$を$\exists$, $\forall$のどちらかとし, 
$q^{i}$を, $p^{i}$が$\exists$ならば$\forall$, $p^{i}$が$\forall$ならば$\exists$とする.

1)
$\neg p^{1}_{A_{1}}x_{1}(p^{2}_{A_{2}}x_{2}( \cdots (p^{n}_{A_{n}}x_{n}(R)) \cdots ))$が
$\mathscr{T}$の定理ならば, 
$q^{1}_{A_{1}}x_{1}(q^{2}_{A_{2}}x_{2}( \cdots (q^{n}_{A_{n}}x_{n}(\neg R)) \cdots ))$は
$\mathscr{T}$の定理である.

2)
$q^{1}_{A_{1}}x_{1}(q^{2}_{A_{2}}x_{2}( \cdots (q^{n}_{A_{n}}x_{n}(\neg R)) \cdots ))$が
$\mathscr{T}$の定理ならば, 
$\neg p^{1}_{A_{1}}x_{1}(p^{2}_{A_{2}}x_{2}( \cdots (p^{n}_{A_{n}}x_{n}(R)) \cdots ))$は
$\mathscr{T}$の定理である.
\end{dedu}




\mathstrut
\begin{dedu}
\label{dedsps4}%推論236
$A$と$R$を$\mathscr{T}$の関係式, $T$を$\mathscr{T}$の対象式とし, 
$x$を文字とする.

1)
$(T|x)(A)$が$\mathscr{T}$の定理ならば, 
\[
  (T|x)(R) \to \exists_{A}x(R), ~~
  \forall_{A}x(R) \to (T|x)(R)
\]
は共に$\mathscr{T}$の定理である.

2)
$(T|x)(A)$と$(T|x)(R)$が共に$\mathscr{T}$の定理ならば, 
$\exists_{A}x(R)$は$\mathscr{T}$の定理である.

3)
$\forall_{A}x(R)$が$\mathscr{T}$の定理ならば, 
$(T|x)(A) \to (T|x)(R)$は$\mathscr{T}$の定理である.

4)
$\forall_{A}x(R)$と$(T|x)(A)$が共に$\mathscr{T}$の定理ならば, 
$(T|x)(R)$は$\mathscr{T}$の定理である.
\end{dedu}




\mathstrut
\begin{dedu}
\label{dedexalspquansep}%推論237
$A$と$R$を$\mathscr{T}$の関係式とし, $x$を文字とする.

1)
$\exists x(A)$が$\mathscr{T}$の定理ならば, 
$\forall_{A}x(R) \to \exists_{A}x(R)$は$\mathscr{T}$の定理である.

2)
$\forall_{A}x(R) \to \exists_{A}x(R)$が$\mathscr{T}$の定理ならば, 
$\exists x(A)$は$\mathscr{T}$の定理である.
\end{dedu}




\mathstrut
\begin{dedu}
\label{dedgsps4}%推論238
$R$を$\mathscr{T}$の関係式とする.
また$n$を自然数とし, $T_{1}, T_{2}, \cdots, T_{n}$を$\mathscr{T}$の対象式, 
$A_{1}, A_{2}, \cdots, A_{n}$を$\mathscr{T}$の関係式, 
$x_{1}, x_{2}, \cdots, x_{n}$をどの二つも互いに異なる文字とする.

1)
$(T_{1}|x_{1})(A_{1}), (T_{1}|x_{1}, T_{2}|x_{2})(A_{2}), \cdots, (T_{1}|x_{1}, T_{2}|x_{2}, \cdots, T_{n}|x_{n})(A_{n})$が
すべて$\mathscr{T}$の定理ならば, 
\[
  (T_{1}|x_{1}, T_{2}|x_{2}, \cdots, T_{n}|x_{n})(R) 
  \to \exists_{A_{1}}x_{1}(\exists_{A_{2}}x_{2}( \cdots (\exists_{A_{n}}x_{n}(R)) \cdots ))
\]
は$\mathscr{T}$の定理である.

2)
$(T_{1}|x_{1})(A_{1}), (T_{1}|x_{1}, T_{2}|x_{2})(A_{2}), \cdots, (T_{1}|x_{1}, T_{2}|x_{2}, \cdots, T_{n}|x_{n})(A_{n}), (T_{1}|x_{1}, T_{2}|x_{2}, \cdots, T_{n}|x_{n})(R)$が
すべて$\mathscr{T}$の定理ならば, 
$\exists_{A_{1}}x_{1}(\exists_{A_{2}}x_{2}( \cdots (\exists_{A_{n}}x_{n}(R)) \cdots ))$は
$\mathscr{T}$の定理である.
\end{dedu}




\mathstrut
\begin{dedu}
\label{dedgsps4free}%推論239
$R$を$\mathscr{T}$の関係式とする.
また$n$を自然数とし, $T_{1}, T_{2}, \cdots, T_{n}$を$\mathscr{T}$の対象式, 
$A_{1}, A_{2}, \cdots, A_{n}$を$\mathscr{T}$の関係式とする.
また$x_{1}, x_{2}, \cdots, x_{n}$をどの二つも互いに異なる文字とする.
いま$i < n$なる各自然数$i$に対し, $x_{1}, x_{2}, \cdots, x_{i}$はいずれも
$A_{i + 1}$の中に自由変数として現れないとする.

1)
$(T_{1}|x_{1})(A_{1}), (T_{2}|x_{2})(A_{2}), \cdots, (T_{n}|x_{n})(A_{n})$が
すべて$\mathscr{T}$の定理ならば, 
\[
  (T_{1}|x_{1}, T_{2}|x_{2}, \cdots, T_{n}|x_{n})(R) 
  \to \exists_{A_{1}}x_{1}(\exists_{A_{2}}x_{2}( \cdots (\exists_{A_{n}}x_{n}(R)) \cdots ))
\]
は$\mathscr{T}$の定理である.

2)
$(T_{1}|x_{1})(A_{1}), (T_{2}|x_{2})(A_{2}), \cdots, (T_{n}|x_{n})(A_{n}), (T_{1}|x_{1}, T_{2}|x_{2}, \cdots, T_{n}|x_{n})(R)$が
すべて$\mathscr{T}$の定理ならば, 
$\exists_{A_{1}}x_{1}(\exists_{A_{2}}x_{2}( \cdots (\exists_{A_{n}}x_{n}(R)) \cdots ))$は
$\mathscr{T}$の定理である.
\end{dedu}




\mathstrut
\begin{dedu}
\label{dedgspallfund2}%推論240
$R$を$\mathscr{T}$の関係式とする.
また$n$を自然数とし, $T_{1}, T_{2}, \cdots, T_{n}$を$\mathscr{T}$の対象式, 
$A_{1}, A_{2}, \cdots, A_{n}$を$\mathscr{T}$の関係式, 
$x_{1}, x_{2}, \cdots, x_{n}$をどの二つも互いに異なる文字とする.

1)
$(T_{1}|x_{1})(A_{1}), (T_{1}|x_{1}, T_{2}|x_{2})(A_{2}), \cdots, (T_{1}|x_{1}, T_{2}|x_{2}, \cdots, T_{n}|x_{n})(A_{n})$が
すべて$\mathscr{T}$の定理ならば, 
\[
  \forall_{A_{1}}x_{1}(\forall_{A_{2}}x_{2}( \cdots (\forall_{A_{n}}x_{n}(R)) \cdots )) 
  \to (T_{1}|x_{1}, T_{2}|x_{2}, \cdots, T_{n}|x_{n})(R)
\]
は$\mathscr{T}$の定理である.

2)
\[
  (T_{1}|x_{1})(A_{1}), (T_{1}|x_{1}, T_{2}|x_{2})(A_{2}), \cdots, (T_{1}|x_{1}, T_{2}|x_{2}, \cdots, T_{n}|x_{n})(A_{n}), 
  \forall_{A_{1}}x_{1}(\forall_{A_{2}}x_{2}( \cdots (\forall_{A_{n}}x_{n}(R)) \cdots ))
\]
がすべて$\mathscr{T}$の定理ならば, 
$(T_{1}|x_{1}, T_{2}|x_{2}, \cdots, T_{n}|x_{n})(R)$は
$\mathscr{T}$の定理である.
\end{dedu}




\mathstrut
\begin{dedu}
\label{dedgspallfund2free}%推論241
$R$を$\mathscr{T}$の関係式とする.
また$n$を自然数とし, $T_{1}, T_{2}, \cdots, T_{n}$を$\mathscr{T}$の対象式, 
$A_{1}, A_{2}, \cdots, A_{n}$を$\mathscr{T}$の関係式とする.
また$x_{1}, x_{2}, \cdots, x_{n}$をどの二つも互いに異なる文字とする.
いま$i < n$なる各自然数$i$に対し, $x_{1}, x_{2}, \cdots, x_{i}$はいずれも
$A_{i + 1}$の中に自由変数として現れないとする.

1)
$(T_{1}|x_{1})(A_{1}), (T_{2}|x_{2})(A_{2}), \cdots, (T_{n}|x_{n})(A_{n})$が
すべて$\mathscr{T}$の定理ならば, 
\[
  \forall_{A_{1}}x_{1}(\forall_{A_{2}}x_{2}( \cdots (\forall_{A_{n}}x_{n}(R)) \cdots )) 
  \to (T_{1}|x_{1}, T_{2}|x_{2}, \cdots, T_{n}|x_{n})(R)
\]
は$\mathscr{T}$の定理である.

2)
$(T_{1}|x_{1})(A_{1}), (T_{2}|x_{2})(A_{2}), \cdots, (T_{n}|x_{n})(A_{n}), \forall_{A_{1}}x_{1}(\forall_{A_{2}}x_{2}( \cdots (\forall_{A_{n}}x_{n}(R)) \cdots ))$が
すべて$\mathscr{T}$の定理ならば, 
$(T_{1}|x_{1}, T_{2}|x_{2}, \cdots, T_{n}|x_{n})(R)$は$\mathscr{T}$の定理である.
\end{dedu}




\mathstrut
\begin{dedu}
\label{dedtspquanfund}%推論242
$A$, $R$, $S$を$\mathscr{T}$の関係式とし, $x$を文字とする.

1)
$(\tau_{x}(A \wedge R)|x)(A \to (R \to S))$が$\mathscr{T}$の定理ならば, 
$\exists_{A}x(R) \to \exists_{A}x(S)$は$\mathscr{T}$の定理である.

2)
$(\tau_{x}(A \wedge R)|x)(R \to S)$が$\mathscr{T}$の定理ならば, 
$\exists_{A}x(R) \to \exists_{A}x(S)$は$\mathscr{T}$の定理である.

3)
$(\tau_{x}(\neg (A \to S))|x)(A \to (R \to S))$が$\mathscr{T}$の定理ならば, 
$\forall_{A}x(R) \to \forall_{A}x(S)$は$\mathscr{T}$の定理である.

4)
$(\tau_{x}(\neg (A \to S))|x)(R \to S)$が$\mathscr{T}$の定理ならば, 
$\forall_{A}x(R) \to \forall_{A}x(S)$は$\mathscr{T}$の定理である.
\end{dedu}




\mathstrut
\begin{dedu}
\label{dedtspquanfund2}%推論243
$A$, $R$, $S$を$\mathscr{T}$の関係式, $x$を文字とし, これらが次の性質($*$), ($**$)のいずれかを持つとする: 

($*$) ~~$\mathscr{T}$の任意の対象式$T$に対し, $(T|x)(A \to (R \to S))$は$\mathscr{T}$の定理となる.

($**$) ~~$\mathscr{T}$の任意の対象式$T$に対し, $(T|x)(R \to S)$は$\mathscr{T}$の定理となる.

このとき, $\exists_{A}x(R) \to \exists_{A}x(S)$と$\forall_{A}x(R) \to \forall_{A}x(S)$は共に
$\mathscr{T}$の定理である.
\end{dedu}




\mathstrut
\begin{dedu}
\label{dedeqspquanfund}%推論244
$A$, $R$, $S$を$\mathscr{T}$の関係式とし, $x$を文字とする.

1)
$(\tau_{x}(A \wedge R)|x)(A \to (R \to S))$と
$(\tau_{x}(A \wedge S)|x)(A \to (S \to R))$が
共に$\mathscr{T}$の定理ならば, 
$\exists_{A}x(R) \leftrightarrow \exists_{A}x(S)$は
$\mathscr{T}$の定理である.

2)
$(\tau_{x}(A \wedge R)|x)(R \to S)$と
$(\tau_{x}(A \wedge S)|x)(S \to R)$が
共に$\mathscr{T}$の定理ならば, 
$\exists_{A}x(R) \leftrightarrow \exists_{A}x(S)$は
$\mathscr{T}$の定理である.

3)
$(\tau_{x}(\neg (A \to S))|x)(A \to (R \to S))$と
$(\tau_{x}(\neg (A \to R))|x)(A \to (S \to R))$が
共に$\mathscr{T}$の定理ならば, 
$\forall_{A}x(R) \leftrightarrow \forall_{A}x(S)$は
$\mathscr{T}$の定理である.

4)
$(\tau_{x}(\neg (A \to S))|x)(R \to S)$と
$(\tau_{x}(\neg (A \to R))|x)(S \to R)$が
共に$\mathscr{T}$の定理ならば, 
$\forall_{A}x(R) \leftrightarrow \forall_{A}x(S)$は
$\mathscr{T}$の定理である.
\end{dedu}




\mathstrut
\begin{dedu}
\label{dedeqspquanfund2}%推論245
$A$, $R$, $S$を$\mathscr{T}$の関係式, $x$を文字とし, これらが次の性質($*$), ($**$)のいずれかを持つとする: 

($*$) ~~$\mathscr{T}$の任意の対象式$T$に対し, $(T|x)(A \to (R \leftrightarrow S))$は$\mathscr{T}$の定理となる.

($**$) ~~$\mathscr{T}$の任意の対象式$T$に対し, $(T|x)(R \leftrightarrow S)$は$\mathscr{T}$の定理となる.

このとき, $\exists_{A}x(R) \leftrightarrow \exists_{A}x(S)$と
$\forall_{A}x(R) \leftrightarrow \forall_{A}x(S)$は共に
$\mathscr{T}$の定理である.
\end{dedu}




\mathstrut
\begin{dedu}
\label{dedspquanvee}%推論246
$A$, $R$, $S$を$\mathscr{T}$の関係式とし, $x$を文字とする.

1)
$\exists_{A}x(R)$が$\mathscr{T}$の定理ならば, 
$\exists_{A}x(R \vee S)$は$\mathscr{T}$の定理である.

2)
$\exists_{A}x(S)$が$\mathscr{T}$の定理ならば, 
$\exists_{A}x(R \vee S)$は$\mathscr{T}$の定理である.

3)
$\forall_{A}x(R)$が$\mathscr{T}$の定理ならば, 
$\forall_{A}x(R \vee S)$は$\mathscr{T}$の定理である.

4)
$\forall_{A}x(S)$が$\mathscr{T}$の定理ならば, 
$\forall_{A}x(R \vee S)$は$\mathscr{T}$の定理である.
\end{dedu}




\mathstrut
\begin{dedu}
\label{dedspquanvch}%推論247
$A$, $R$, $S$を$\mathscr{T}$の関係式とし, $x$を文字とする.

1)
$\exists_{A}x(R \vee S)$が$\mathscr{T}$の定理ならば, 
$\exists_{A}x(S \vee R)$は$\mathscr{T}$の定理である.

2)
$\forall_{A}x(R \vee S)$が$\mathscr{T}$の定理ならば, 
$\forall_{A}x(S \vee R)$は$\mathscr{T}$の定理である.
\end{dedu}




\mathstrut
\begin{dedu}
\label{dedspquantveq}%推論248
$A$, $R$, $S$を$\mathscr{T}$の関係式とし, $x$を文字とする.

1)
$\exists_{A}x(R \to S)$が$\mathscr{T}$の定理ならば, 
$\exists_{A}x(\neg R \vee S)$は$\mathscr{T}$の定理である.

2)
$\exists_{A}x(\neg R \vee S)$が$\mathscr{T}$の定理ならば, 
$\exists_{A}x(R \to S)$は$\mathscr{T}$の定理である.

3)
$\forall_{A}x(R \to S)$が$\mathscr{T}$の定理ならば, 
$\forall_{A}x(\neg R \vee S)$は$\mathscr{T}$の定理である.

4)
$\forall_{A}x(\neg R \vee S)$が$\mathscr{T}$の定理ならば, 
$\forall_{A}x(R \to S)$は$\mathscr{T}$の定理である.
\end{dedu}




\mathstrut
\begin{dedu}
\label{dedspexv}%推論249
$A$, $R$, $S$を$\mathscr{T}$の関係式とし, $x$を文字とする.

1)
$\exists_{A}x(R \vee S)$が$\mathscr{T}$の定理ならば, 
$\exists_{A}x(R) \vee \exists_{A}x(S)$は$\mathscr{T}$の定理である.

2)
$\exists_{A}x(R) \vee \exists_{A}x(S)$が$\mathscr{T}$の定理ならば, 
$\exists_{A}x(R \vee S)$は$\mathscr{T}$の定理である.
\end{dedu}




\mathstrut
\begin{dedu}
\label{dedspexvrfree}%推論250
$A$, $R$, $S$を$\mathscr{T}$の関係式とし, $x$を$R$の中に自由変数として現れない文字とする.

1)
$\exists_{A}x(R \vee S)$が$\mathscr{T}$の定理ならば, 
$R \vee \exists_{A}x(S)$は$\mathscr{T}$の定理である.

2)
$\exists_{A}x(S \vee R)$が$\mathscr{T}$の定理ならば, 
$\exists_{A}x(S) \vee R$は$\mathscr{T}$の定理である.
\end{dedu}




\mathstrut
\begin{dedu}
\label{dedspexvrfreeeq}%推論251
$A$, $R$, $S$を$\mathscr{T}$の関係式とし, $x$を$R$の中に自由変数として現れない文字とする.

1)
$\exists x(A)$が$\mathscr{T}$の定理ならば, 
\[
  \exists_{A}x(R \vee S) \leftrightarrow R \vee \exists_{A}x(S), ~~
  \exists_{A}x(S \vee R) \leftrightarrow \exists_{A}x(S) \vee R
\]
は共に$\mathscr{T}$の定理である.

2)
$\exists x(A)$と$R \vee \exists_{A}x(S)$が共に$\mathscr{T}$の定理ならば, 
$\exists_{A}x(R \vee S)$は$\mathscr{T}$の定理である.

3)
$\exists x(A)$と$\exists_{A}x(S) \vee R$が共に$\mathscr{T}$の定理ならば, 
$\exists_{A}x(S \vee R)$は$\mathscr{T}$の定理である.
\end{dedu}




\mathstrut
\begin{dedu}
\label{dedspexvrfreeeq2}%推論252
$A$, $R$, $S$を$\mathscr{T}$の関係式とし, $x$を$R$の中に自由変数として現れない文字とする.
$\exists x(A)$と$R$が共に$\mathscr{T}$の定理ならば, 
$\exists_{A}x(R \vee S)$と$\exists_{A}x(S \vee R)$は共に$\mathscr{T}$の定理である.
\end{dedu}




\mathstrut
\begin{dedu}
\label{dedspallv}%推論253
$A$, $R$, $S$を$\mathscr{T}$の関係式とし, $x$を文字とする.

1)
$\forall_{A}x(R) \vee \forall_{A}x(S)$が$\mathscr{T}$の定理ならば, 
$\forall_{A}x(R \vee S)$は$\mathscr{T}$の定理である.

2)
$\forall_{A}x(R \vee S)$が$\mathscr{T}$の定理ならば, 
$\forall_{A}x(R) \vee \exists_{A}x(S)$と$\exists_{A}x(R) \vee \forall_{A}x(S)$は
共に$\mathscr{T}$の定理である.
\end{dedu}




\mathstrut
\begin{dedu}
\label{dedspallvrfree}%推論254
$A$, $R$, $S$を$\mathscr{T}$の関係式とし, $x$を$R$の中に自由変数として現れない文字とする.

1)
$\forall_{A}x(R \vee S)$が$\mathscr{T}$の定理ならば, 
$R \vee \forall_{A}x(S)$は$\mathscr{T}$の定理である.

2)
$R \vee \forall_{A}x(S)$が$\mathscr{T}$の定理ならば, 
$\forall_{A}x(R \vee S)$は$\mathscr{T}$の定理である.

3)
$\forall_{A}x(S \vee R)$が$\mathscr{T}$の定理ならば, 
$\forall_{A}x(S) \vee R$は$\mathscr{T}$の定理である.

4)
$\forall_{A}x(S) \vee R$が$\mathscr{T}$の定理ならば, 
$\forall_{A}x(S \vee R)$は$\mathscr{T}$の定理である.
\end{dedu}




\mathstrut
\begin{dedu}
\label{dedspallvrfree2}%推論255
$A$, $R$, $S$を$\mathscr{T}$の関係式とし, $x$を$R$の中に自由変数として現れない文字とする.
$R$が$\mathscr{T}$の定理ならば, 
$\forall_{A}x(R \vee S)$と$\forall_{A}x(S \vee R)$は共に$\mathscr{T}$の定理である.
\end{dedu}




\mathstrut
\begin{dedu}
\label{dedspquanwedge}%推論256
$A$, $R$, $S$を$\mathscr{T}$の関係式とし, $x$を文字とする.

1)
$\exists_{A}x(R \wedge S)$が$\mathscr{T}$の定理ならば, 
$\exists_{A}x(R)$と$\exists_{A}x(S)$は共に$\mathscr{T}$の定理である.

2)
$\forall_{A}x(R \wedge S)$が$\mathscr{T}$の定理ならば, 
$\forall_{A}x(R)$と$\forall_{A}x(S)$は共に$\mathscr{T}$の定理である.
\end{dedu}




\mathstrut
\begin{dedu}
\label{dedspquanwch}%推論257
$A$, $R$, $S$を$\mathscr{T}$の関係式とし, $x$を文字とする.

1)
$\exists_{A}x(R \wedge S)$が$\mathscr{T}$の定理ならば, 
$\exists_{A}x(S \wedge R)$は$\mathscr{T}$の定理である.

2)
$\forall_{A}x(R \wedge S)$が$\mathscr{T}$の定理ならば, 
$\forall_{A}x(S \wedge R)$は$\mathscr{T}$の定理である.
\end{dedu}




\mathstrut
\begin{dedu}
\label{dedspquantweq}%推論258
$A$, $R$, $S$を$\mathscr{T}$の関係式とし, $x$を文字とする.

1)
$\exists_{A}x(\neg (R \to S))$が$\mathscr{T}$の定理ならば, 
$\exists_{A}x(R \wedge \neg S)$は$\mathscr{T}$の定理である.

2)
$\exists_{A}x(R \wedge \neg S)$が$\mathscr{T}$の定理ならば, 
$\exists_{A}x(\neg (R \to S))$は$\mathscr{T}$の定理である.

3)
$\forall_{A}x(\neg (R \to S))$が$\mathscr{T}$の定理ならば, 
$\forall_{A}x(R \wedge \neg S)$は$\mathscr{T}$の定理である.

4)
$\forall_{A}x(R \wedge \neg S)$が$\mathscr{T}$の定理ならば, 
$\forall_{A}x(\neg (R \to S))$は$\mathscr{T}$の定理である.
\end{dedu}




\mathstrut
\begin{dedu}
\label{dedspexw}%推論259
$A$, $R$, $S$を$\mathscr{T}$の関係式とし, $x$を文字とする.

1)
$\exists_{A}x(R \wedge S)$が$\mathscr{T}$の定理ならば, 
$\exists_{A}x(R) \wedge \exists_{A}x(S)$は$\mathscr{T}$の定理である.

2)
$\exists_{A}x(R) \wedge \forall_{A}x(S)$が$\mathscr{T}$の定理ならば, 
$\exists_{A}x(R \wedge S)$は$\mathscr{T}$の定理である.

3)
$\forall_{A}x(R) \wedge \exists_{A}x(S)$が$\mathscr{T}$の定理ならば, 
$\exists_{A}x(R \wedge S)$は$\mathscr{T}$の定理である.
\end{dedu}




\mathstrut
\begin{dedu}
\label{dedspexw2}%推論260
$A$, $R$, $S$を$\mathscr{T}$の関係式とし, $x$を文字とする.

1)
$\exists_{A}x(R)$と$\forall_{A}x(S)$が共に$\mathscr{T}$の定理ならば, 
$\exists_{A}x(R \wedge S)$は$\mathscr{T}$の定理である.

2)
$\forall_{A}x(R)$と$\exists_{A}x(S)$が共に$\mathscr{T}$の定理ならば, 
$\exists_{A}x(R \wedge S)$は$\mathscr{T}$の定理である.
\end{dedu}




\mathstrut
\begin{dedu}
\label{dedspexwrfree}%推論261
$A$, $R$, $S$を$\mathscr{T}$の関係式とし, $x$を$R$の中に自由変数として現れない文字とする.

1)
$\exists_{A}x(R \wedge S)$が$\mathscr{T}$の定理ならば, 
$R \wedge \exists_{A}x(S)$は$\mathscr{T}$の定理である.

2)
$R \wedge \exists_{A}x(S)$が$\mathscr{T}$の定理ならば, 
$\exists_{A}x(R \wedge S)$は$\mathscr{T}$の定理である.

3)
$\exists_{A}x(S \wedge R)$が$\mathscr{T}$の定理ならば, 
$\exists_{A}x(S) \wedge R$は$\mathscr{T}$の定理である.

4)
$\exists_{A}x(S) \wedge R$が$\mathscr{T}$の定理ならば, 
$\exists_{A}x(S \wedge R)$は$\mathscr{T}$の定理である.
\end{dedu}




\mathstrut
\begin{dedu}
\label{dedspexwrfree2}%推論262
$A$, $R$, $S$を$\mathscr{T}$の関係式とし, $x$を$R$の中に自由変数として現れない文字とする.

1)
$\exists_{A}x(R \wedge S)$が$\mathscr{T}$の定理ならば, 
$R$は$\mathscr{T}$の定理である.

2)
$\exists_{A}x(S \wedge R)$が$\mathscr{T}$の定理ならば, 
$R$は$\mathscr{T}$の定理である.

3)
$R$と$\exists_{A}x(S)$が共に$\mathscr{T}$の定理ならば, 
$\exists_{A}x(R \wedge S)$と$\exists_{A}x(S \wedge R)$は共に$\mathscr{T}$の定理である.
\end{dedu}




\mathstrut
\begin{dedu}
\label{dedspallw}%推論263
$A$, $R$, $S$を$\mathscr{T}$の関係式とし, $x$を文字とする.

1)
$\forall_{A}x(R \wedge S)$が$\mathscr{T}$の定理ならば, 
$\forall_{A}x(R) \wedge \forall_{A}x(S)$は$\mathscr{T}$の定理である.

2)
$\forall_{A}x(R) \wedge \forall_{A}x(S)$が$\mathscr{T}$の定理ならば, 
$\forall_{A}x(R \wedge S)$は$\mathscr{T}$の定理である.
\end{dedu}




\mathstrut
\begin{dedu}
\label{dedspallw2}%推論264
$A$, $R$, $S$を$\mathscr{T}$の関係式とし, $x$を文字とする.
$\forall_{A}x(R)$と$\forall_{A}x(S)$が共に$\mathscr{T}$の定理ならば, 
$\forall_{A}x(R \wedge S)$は$\mathscr{T}$の定理である.
\end{dedu}




\mathstrut
\begin{dedu}
\label{dedspallwrfree}%推論265
$A$, $R$, $S$を$\mathscr{T}$の関係式とし, $x$を$R$の中に自由変数として現れない文字とする.

1)
$R \wedge \forall_{A}x(S)$が$\mathscr{T}$の定理ならば, 
$\forall_{A}x(R \wedge S)$は$\mathscr{T}$の定理である.

2)
$\forall_{A}x(S) \wedge R$が$\mathscr{T}$の定理ならば, 
$\forall_{A}x(S \wedge R)$は$\mathscr{T}$の定理である.
\end{dedu}




\mathstrut
\begin{dedu}
\label{dedspallwrfree2}%推論266
$A$, $R$, $S$を$\mathscr{T}$の関係式とし, $x$を$R$の中に自由変数として現れない文字とする.
$R$と$\forall_{A}x(S)$が共に$\mathscr{T}$の定理ならば, 
$\forall_{A}x(R \wedge S)$と$\forall_{A}x(S \wedge R)$は共に$\mathscr{T}$の定理である.
\end{dedu}




\mathstrut
\begin{dedu}
\label{dedspallwrfreeeq}%推論267
$A$, $R$, $S$を$\mathscr{T}$の関係式とし, $x$を$R$の中に自由変数として現れない文字とする.

1)
$\exists x(A)$が$\mathscr{T}$の定理ならば, 
\[
  \forall_{A}x(R \wedge S) \leftrightarrow R \wedge \forall_{A}x(S), ~~
  \forall_{A}x(S \wedge R) \leftrightarrow \forall_{A}x(S) \wedge R
\]
は共に$\mathscr{T}$の定理である.

2)
$\exists x(A)$と$\forall_{A}x(R \wedge S)$が共に$\mathscr{T}$の定理ならば, 
$R \wedge \forall_{A}x(S)$は$\mathscr{T}$の定理である.

3)
$\exists x(A)$と$\forall_{A}x(S \wedge R)$が共に$\mathscr{T}$の定理ならば, 
$\forall_{A}x(S) \wedge R$は$\mathscr{T}$の定理である.
\end{dedu}




\mathstrut
\begin{dedu}
\label{dedspallwrfreeeq2}%推論268
$A$, $R$, $S$を$\mathscr{T}$の関係式とし, $x$を$R$の中に自由変数として現れない文字とする.

1)
$\exists x(A)$と$\forall_{A}x(R \wedge S)$が共に$\mathscr{T}$の定理ならば, 
$R$は$\mathscr{T}$の定理である.

2)
$\exists x(A)$と$\forall_{A}x(S \wedge R)$が共に$\mathscr{T}$の定理ならば, 
$R$は$\mathscr{T}$の定理である.
\end{dedu}




\mathstrut
\begin{dedu}
\label{dedspexprevee}%推論269
$A$, $B$, $R$を$\mathscr{T}$の関係式とし, $x$を文字とする.

1)
$\exists_{A}x(R)$が$\mathscr{T}$の定理ならば, 
$\exists_{A \vee B}x(R)$は$\mathscr{T}$の定理である.

2)
$\exists_{B}x(R)$が$\mathscr{T}$の定理ならば, 
$\exists_{A \vee B}x(R)$は$\mathscr{T}$の定理である.
\end{dedu}




\mathstrut
\begin{dedu}
\label{dedspexprev}%推論270
$A$, $B$, $R$を$\mathscr{T}$の関係式とし, $x$を文字とする.

1)
$\exists_{A \vee B}x(R)$が$\mathscr{T}$の定理ならば, 
$\exists_{A}x(R) \vee \exists_{B}x(R)$は$\mathscr{T}$の定理である.

2)
$\exists_{A}x(R) \vee \exists_{B}x(R)$が$\mathscr{T}$の定理ならば, 
$\exists_{A \vee B}x(R)$は$\mathscr{T}$の定理である.
\end{dedu}




\mathstrut
\begin{dedu}
\label{dedspexprevafree}%推論271
$A$, $B$, $R$を$\mathscr{T}$の関係式とし, $x$を$A$の中に自由変数として現れない文字とする.

1)
$\exists_{A \vee B}x(R)$が$\mathscr{T}$の定理ならば, 
$A \vee \exists_{B}x(R)$は$\mathscr{T}$の定理である.

2)
$\exists_{B \vee A}x(R)$が$\mathscr{T}$の定理ならば, 
$\exists_{B}x(R) \vee A$は$\mathscr{T}$の定理である.
\end{dedu}




\mathstrut
\begin{dedu}
\label{dedspexprevafreeeq}%推論272
$A$, $B$, $R$を$\mathscr{T}$の関係式とし, $x$を$A$の中に自由変数として現れない文字とする.

1)
$\exists x(R)$が$\mathscr{T}$の定理ならば, 
\[
  \exists_{A \vee B}x(R) \leftrightarrow A \vee \exists_{B}x(R), ~~
  \exists_{B \vee A}x(R) \leftrightarrow \exists_{B}x(R) \vee A
\]
は共に$\mathscr{T}$の定理である.

2)
$\exists x(R)$と$A \vee \exists_{B}x(R)$が共に$\mathscr{T}$の定理ならば, 
$\exists_{A \vee B}x(R)$は$\mathscr{T}$の定理である.

3)
$\exists x(R)$と$\exists_{B}x(R) \vee A$が共に$\mathscr{T}$の定理ならば, 
$\exists_{B \vee A}x(R)$は$\mathscr{T}$の定理である.
\end{dedu}




\mathstrut
\begin{dedu}
\label{dedspexprevafreeeq2}%推論273
$A$, $B$, $R$を$\mathscr{T}$の関係式とし, $x$を$A$の中に自由変数として現れない文字とする.
$A$と$\exists x(R)$が共に$\mathscr{T}$の定理ならば, 
$\exists_{A \vee B}x(R)$と$\exists_{B \vee A}x(R)$は共に$\mathscr{T}$の定理である.
\end{dedu}




\mathstrut
\begin{dedu}
\label{dedspallprevee}%推論274
$A$, $B$, $R$を$\mathscr{T}$の関係式とし, $x$を文字とする.
$\forall_{A \vee B}x(R)$が$\mathscr{T}$の定理ならば, 
$\forall_{A}x(R)$と$\forall_{B}x(R)$は共に$\mathscr{T}$の定理である.
\end{dedu}




\mathstrut
\begin{dedu}
\label{dedspallprev}%推論275
$A$, $B$, $R$を$\mathscr{T}$の関係式とし, $x$を文字とする.

1)
$\forall_{A \vee B}x(R)$が$\mathscr{T}$の定理ならば, 
$\forall_{A}x(R) \wedge \forall_{B}x(R)$は$\mathscr{T}$の定理である.

2)
$\forall_{A}x(R) \wedge \forall_{B}x(R)$が$\mathscr{T}$の定理ならば, 
$\forall_{A \vee B}x(R)$は$\mathscr{T}$の定理である.
\end{dedu}




\mathstrut
\begin{dedu}
\label{dedspallprev2}%推論276
$A$, $B$, $R$を$\mathscr{T}$の関係式とし, $x$を文字とする.
$\forall_{A}x(R)$と$\forall_{B}x(R)$が共に$\mathscr{T}$の定理ならば, 
$\forall_{A \vee B}x(R)$は$\mathscr{T}$の定理である.
\end{dedu}




\mathstrut
\begin{dedu}
\label{dedspallprevafree}%推論277
$A$, $B$, $R$を$\mathscr{T}$の関係式とし, $x$を$A$の中に自由変数として現れない文字とする.

1)
$\neg A \wedge \forall_{B}x(R)$が$\mathscr{T}$の定理ならば, 
$\forall_{A \vee B}x(R)$は$\mathscr{T}$の定理である.

2)
$\forall_{B}x(R) \wedge \neg A$が$\mathscr{T}$の定理ならば, 
$\forall_{B \vee A}x(R)$は$\mathscr{T}$の定理である.
\end{dedu}




\mathstrut
\begin{dedu}
\label{dedspallprevafree2}%推論278
$A$, $B$, $R$を$\mathscr{T}$の関係式とし, $x$を$A$の中に自由変数として現れない文字とする.
$\neg A$と$\forall_{B}x(R)$が共に$\mathscr{T}$の定理ならば, 
$\forall_{A \vee B}x(R)$と$\forall_{B \vee A}x(R)$は共に$\mathscr{T}$の定理である.
\end{dedu}




\mathstrut
\begin{dedu}
\label{dedspallprevafreeeq}%推論279
$A$, $B$, $R$を$\mathscr{T}$の関係式とし, $x$を$A$の中に自由変数として現れない文字とする.

1)
$\exists x(\neg R)$が$\mathscr{T}$の定理ならば, 
\[
  \forall_{A \vee B}x(R) \leftrightarrow \neg A \wedge \forall_{B}x(R), ~~
  \forall_{B \vee A}x(R) \leftrightarrow \forall_{B}x(R) \wedge \neg A
\]
は共に$\mathscr{T}$の定理である.

2)
$\exists x(\neg R)$と$\forall_{A \vee B}x(R)$が共に$\mathscr{T}$の定理ならば, 
$\neg A \wedge \forall_{B}x(R)$は$\mathscr{T}$の定理である.

3)
$\exists x(\neg R)$と$\forall_{B \vee A}x(R)$が共に$\mathscr{T}$の定理ならば, 
$\forall_{B}x(R) \wedge \neg A$は$\mathscr{T}$の定理である.
\end{dedu}




\mathstrut
\begin{dedu}
\label{dedspallprevafreeeq2}%推論280
$A$, $B$, $R$を$\mathscr{T}$の関係式とし, $x$を$A$の中に自由変数として現れない文字とする.

1)
$\exists x(\neg R)$と$\forall_{A \vee B}x(R)$が共に$\mathscr{T}$の定理ならば, 
$\neg A$は$\mathscr{T}$の定理である.

2)
$\exists x(\neg R)$と$\forall_{B \vee A}x(R)$が共に$\mathscr{T}$の定理ならば, 
$\neg A$は$\mathscr{T}$の定理である.
\end{dedu}




\mathstrut
\begin{dedu}
\label{dedspexprewedge}%推論281
$A$, $B$, $R$を$\mathscr{T}$の関係式とし, $x$を文字とする.
$\exists_{A \wedge B}x(R)$が$\mathscr{T}$の定理ならば, 
$\exists_{A}x(R)$と$\exists_{B}x(R)$は共に$\mathscr{T}$の定理である.
\end{dedu}




\mathstrut
\begin{dedu}
\label{dedspexprew}%推論282
$A$, $B$, $R$を$\mathscr{T}$の関係式とし, $x$を文字とする.
$\exists_{A \wedge B}x(R)$が$\mathscr{T}$の定理ならば, 
$\exists_{A}x(R) \wedge \exists_{B}x(R)$は$\mathscr{T}$の定理である.
\end{dedu}




\mathstrut
\begin{dedu}
\label{dedspexprewafree}%推論283
$A$, $B$, $R$を$\mathscr{T}$の関係式とし, 
$x$を$A$の中に自由変数として現れない文字とする.

1)
$\exists_{A \wedge B}x(R)$が$\mathscr{T}$の定理ならば, 
$A \wedge \exists_{B}x(R)$は$\mathscr{T}$の定理である.

2)
$A \wedge \exists_{B}x(R)$が$\mathscr{T}$の定理ならば, 
$\exists_{A \wedge B}x(R)$は$\mathscr{T}$の定理である.

3)
$\exists_{B \wedge A}x(R)$が$\mathscr{T}$の定理ならば, 
$\exists_{B}x(R) \wedge A$は$\mathscr{T}$の定理である.

4)
$\exists_{B}x(R) \wedge A$が$\mathscr{T}$の定理ならば, 
$\exists_{B \wedge A}x(R)$は$\mathscr{T}$の定理である.
\end{dedu}




\mathstrut
\begin{dedu}
\label{dedspexprewafree2}%推論284
$A$, $B$, $R$を$\mathscr{T}$の関係式とし, 
$x$を$A$の中に自由変数として現れない文字とする.

1)
$\exists_{A \wedge B}x(R)$が$\mathscr{T}$の定理ならば, 
$A$は$\mathscr{T}$の定理である.

2)
$\exists_{B \wedge A}x(R)$が$\mathscr{T}$の定理ならば, 
$A$は$\mathscr{T}$の定理である.

3)
$A$と$\exists_{B}x(R)$が共に$\mathscr{T}$の定理ならば, 
$\exists_{A \wedge B}x(R)$と$\exists_{B \wedge A}x(R)$は共に$\mathscr{T}$の定理である.
\end{dedu}




\mathstrut
\begin{dedu}
\label{dedspallprewedge}%推論285
$A$, $B$, $R$を$\mathscr{T}$の関係式とし, $x$を文字とする.

1)
$\forall_{A}x(R)$が$\mathscr{T}$の定理ならば, 
$\forall_{A \wedge B}x(R)$は$\mathscr{T}$の定理である.

2)
$\forall_{B}x(R)$が$\mathscr{T}$の定理ならば, 
$\forall_{A \wedge B}x(R)$は$\mathscr{T}$の定理である.
\end{dedu}




\mathstrut
\begin{dedu}
\label{dedspallprew}%推論286
$A$, $B$, $R$を$\mathscr{T}$の関係式とし, $x$を文字とする.
$\forall_{A}x(R) \vee \forall_{B}x(R)$が$\mathscr{T}$の定理ならば, 
$\forall_{A \wedge B}x(R)$は$\mathscr{T}$の定理である.
\end{dedu}




\mathstrut
\begin{dedu}
\label{dedspallprewafree}%推論287
$A$, $B$, $R$を$\mathscr{T}$の関係式とし, 
$x$を$A$の中に自由変数として現れない文字とする.

1)
$\forall_{A \wedge B}x(R)$が$\mathscr{T}$の定理ならば, 
$A \to \forall_{B}x(R)$は$\mathscr{T}$の定理である.

2)
$\forall_{B \wedge A}x(R)$が$\mathscr{T}$の定理ならば, 
$A \to \forall_{B}x(R)$は$\mathscr{T}$の定理である.

3)
$A \to \forall_{B}x(R)$が$\mathscr{T}$の定理ならば, 
$\forall_{A \wedge B}x(R)$と$\forall_{B \wedge A}x(R)$は共に$\mathscr{T}$の定理である.
\end{dedu}




\mathstrut
\begin{dedu}
\label{dedspallprewafree2}%推論288
$A$, $B$, $R$を$\mathscr{T}$の関係式とし, 
$x$を$A$の中に自由変数として現れない文字とする.
$\neg A$が$\mathscr{T}$の定理ならば, 
$\forall_{A \wedge B}x(R)$と$\forall_{B \wedge A}x(R)$は共に$\mathscr{T}$の定理である.
\end{dedu}




\mathstrut
\begin{dedu}
\label{dedspquangvee}%推論289
$A$を$\mathscr{T}$の関係式とし, $x$を文字とする.
また$n$を自然数とし, $R_{1}, R_{2}, \cdots, R_{n}$を$\mathscr{T}$の関係式とする.
また$i$を$n$以下の自然数とする.

1)
$\exists_{A}x(R_{i})$が$\mathscr{T}$の定理ならば, 
$\exists_{A}x(R_{1} \vee R_{2} \vee \cdots \vee R_{n})$は$\mathscr{T}$の定理である.

2)
$\forall_{A}x(R_{i})$が$\mathscr{T}$の定理ならば, 
$\forall_{A}x(R_{1} \vee R_{2} \vee \cdots \vee R_{n})$は$\mathscr{T}$の定理である.
\end{dedu}




\mathstrut
\begin{dedu}
\label{dedspquangvee2}%推論290
$A$を$\mathscr{T}$の関係式とし, $x$を文字とする.
また$n$を自然数とし, $R_{1}, R_{2}, \cdots, R_{n}$を$\mathscr{T}$の関係式とする.
また$k$を自然数とし, $i_{1}, i_{2}, \cdots, i_{k}$を$n$以下の自然数とする.

1)
$\exists_{A}x(R_{i_{1}} \vee R_{i_{2}} \vee \cdots \vee R_{i_{k}})$が$\mathscr{T}$の定理ならば, 
$\exists_{A}x(R_{1} \vee R_{2} \vee \cdots \vee R_{n})$は$\mathscr{T}$の定理である.

2)
$\forall_{A}x(R_{i_{1}} \vee R_{i_{2}} \vee \cdots \vee R_{i_{k}})$が$\mathscr{T}$の定理ならば, 
$\forall_{A}x(R_{1} \vee R_{2} \vee \cdots \vee R_{n})$は$\mathscr{T}$の定理である.
\end{dedu}




\mathstrut
\begin{dedu}
\label{dedspexgv}%推論291
$A$を$\mathscr{T}$の関係式とし, $x$を文字とする.
また$n$を自然数とし, $R_{1}, R_{2}, \cdots, R_{n}$を$\mathscr{T}$の関係式とする.

1)
$\exists_{A}x(R_{1} \vee R_{2} \vee \cdots \vee R_{n})$が$\mathscr{T}$の定理ならば, 
$\exists_{A}x(R_{1}) \vee \exists_{A}x(R_{2}) \vee \cdots \vee \exists_{A}x(R_{n})$は$\mathscr{T}$の定理である.

2)
$\exists_{A}x(R_{1}) \vee \exists_{A}x(R_{2}) \vee \cdots \vee \exists_{A}x(R_{n})$が$\mathscr{T}$の定理ならば, 
$\exists_{A}x(R_{1} \vee R_{2} \vee \cdots \vee R_{n})$は$\mathscr{T}$の定理である.
\end{dedu}




\mathstrut
\begin{dedu}
\label{dedspexgvfree}%推論292
$A$を$\mathscr{T}$の関係式とする.
また$n$を自然数とし, $R_{1}, R_{2}, \cdots, R_{n}$を$\mathscr{T}$の関係式とする.
また$k$を$n$以下の自然数とし, $i_{1}, i_{2}, \cdots, i_{k}$を
$i_{1} < i_{2} < \cdots < i_{k} \LEQQ n$なる自然数とする.
また$x$を$R_{i_{1}}, R_{i_{2}}, \cdots, R_{i_{k}}$の中に自由変数として現れない文字とする.
このとき$\exists_{A}x(R_{1} \vee R_{2} \vee \cdots \vee R_{n})$が$\mathscr{T}$の定理ならば, 
\begin{multline*}
  \exists_{A}x(R_{1}) \vee \cdots \vee \exists_{A}x(R_{i_{1} - 1}) \vee R_{i_{1}} \vee \exists_{A}x(R_{i_{1} + 1}) \vee \cdots\cdots \\
  \vee \exists_{A}x(R_{i_{k} - 1}) \vee R_{i_{k}} \vee \exists_{A}x(R_{i_{k} + 1}) \vee \cdots \vee \exists_{A}x(R_{n})
\end{multline*}
は$\mathscr{T}$の定理である.
\end{dedu}




\mathstrut
\begin{dedu}
\label{dedspexgvfreeeq}%推論293
$A$を$\mathscr{T}$の関係式とする.
また$n$を自然数とし, $R_{1}, R_{2}, \cdots, R_{n}$を$\mathscr{T}$の関係式とする.
また$k$を$n$以下の自然数とし, $i_{1}, i_{2}, \cdots, i_{k}$を
$i_{1} < i_{2} < \cdots < i_{k} \LEQQ n$なる自然数とする.
また$x$を$R_{i_{1}}, R_{i_{2}}, \cdots, R_{i_{k}}$の中に自由変数として現れない文字とする.
$\exists x(A)$が$\mathscr{T}$の定理ならば, 
\begin{multline*}
  \exists_{A}x(R_{1} \vee R_{2} \vee \cdots \vee R_{n}) \\
  \leftrightarrow \exists_{A}x(R_{1}) \vee \cdots \vee \exists_{A}x(R_{i_{1} - 1}) \vee R_{i_{1}} \vee \exists_{A}x(R_{i_{1} + 1}) \vee \cdots\cdots \\
  \vee \exists_{A}x(R_{i_{k} - 1}) \vee R_{i_{k}} \vee \exists_{A}x(R_{i_{k} + 1}) \vee \cdots \vee \exists_{A}x(R_{n})
\end{multline*}
は$\mathscr{T}$の定理である.
\end{dedu}




\mathstrut
\begin{dedu}
\label{dedspexgvfreeeq2}%推論294
$A$を$\mathscr{T}$の関係式とする.
また$n$を自然数とし, $R_{1}, R_{2}, \cdots, R_{n}$を$\mathscr{T}$の関係式とする.
また$i$を$n$以下の自然数とし, $x$を$R_{i}$の中に自由変数として現れない文字とする.
$\exists x(A)$と$R_{i}$が共に$\mathscr{T}$の定理ならば, 
$\exists_{A}x(R_{1} \vee R_{2} \vee \cdots \vee R_{n})$は$\mathscr{T}$の定理である.
\end{dedu}




\mathstrut
\begin{dedu}
\label{dedspallgv}%推論295
$A$を$\mathscr{T}$の関係式とし, $x$を文字とする.
また$n$を自然数とし, $R_{1}, R_{2}, \cdots, R_{n}$を$\mathscr{T}$の関係式とする.

1)
$\forall_{A}x(R_{1}) \vee \forall_{A}x(R_{2}) \vee \cdots \vee \forall_{A}x(R_{n})$が$\mathscr{T}$の定理ならば, 
$\forall_{A}x(R_{1} \vee R_{2} \vee \cdots \vee R_{n})$は$\mathscr{T}$の定理である.

2)
$\forall_{A}x(R_{1} \vee R_{2} \vee \cdots \vee R_{n})$が$\mathscr{T}$の定理ならば, 
$n$以下の任意の自然数$i$に対して
\[
  \exists_{A}x(R_{1}) \vee \cdots \vee \exists_{A}x(R_{i - 1}) \vee \forall_{A}x(R_{i}) \vee \exists_{A}x(R_{i + 1}) \vee \cdots \vee \exists_{A}x(R_{n})
\]
は$\mathscr{T}$の定理である.
\end{dedu}




\mathstrut
\begin{dedu}
\label{dedspallgvfree}%推論296
$A$を$\mathscr{T}$の関係式とする.
また$n$を自然数とし, $R_{1}, R_{2}, \cdots, R_{n}$を$\mathscr{T}$の関係式とする.
また$k$を$n$以下の自然数とし, $i_{1}, i_{2}, \cdots, i_{k}$を
$i_{1} < i_{2} < \cdots < i_{k} \LEQQ n$なる自然数とする.
また$x$を$R_{i_{1}}, R_{i_{2}}, \cdots, R_{i_{k}}$の中に自由変数として現れない文字とする.
このとき
\begin{multline*}
  \forall_{A}x(R_{1}) \vee \cdots \vee \forall_{A}x(R_{i_{1} - 1}) \vee R_{i_{1}} \vee \forall_{A}x(R_{i_{1} + 1}) \vee \cdots\cdots \\
  \vee \forall_{A}x(R_{i_{k} - 1}) \vee R_{i_{k}} \vee \forall_{A}x(R_{i_{k} + 1}) \vee \cdots \vee \forall_{A}x(R_{n})
\end{multline*}
が$\mathscr{T}$の定理ならば, 
$\forall_{A}x(R_{1} \vee R_{2} \vee \cdots \vee R_{n})$は$\mathscr{T}$の定理である.
\end{dedu}




\mathstrut
\begin{dedu}
\label{dedspallgvfree2}%推論297
$A$を$\mathscr{T}$の関係式とする.
また$n$を自然数とし, $R_{1}, R_{2}, \cdots, R_{n}$を$\mathscr{T}$の関係式とする.
また$i$を$n$以下の自然数とし, $x$を$R_{i}$の中に自由変数として現れない文字とする.
$R_{i}$が$\mathscr{T}$の定理ならば, 
$\forall_{A}x(R_{1} \vee R_{2} \vee \cdots \vee R_{n})$は$\mathscr{T}$の定理である.
\end{dedu}




\mathstrut
\begin{dedu}
\label{dedspallgvfreeeq}%推論298
$A$を$\mathscr{T}$の関係式とし, $x$を文字とする.
また$n$を自然数とし, $R_{1}, R_{2}, \cdots, R_{n}$を$\mathscr{T}$の関係式とする.
また$i$を$n$以下の自然数とする.
いま$i$と異なる$n$以下の任意の自然数$j$に対し, $x$は$R_{j}$の中に自由変数として現れないとする.

1)
$\forall_{A}x(R_{1} \vee R_{2} \vee \cdots \vee R_{n})$が$\mathscr{T}$の定理ならば, 
$R_{1} \vee \cdots \vee R_{i - 1} \vee \forall_{A}x(R_{i}) \vee R_{i + 1} \vee \cdots \vee R_{n}$は$\mathscr{T}$の定理である.

2)
$R_{1} \vee \cdots \vee R_{i - 1} \vee \forall_{A}x(R_{i}) \vee R_{i + 1} \vee \cdots \vee R_{n}$が$\mathscr{T}$の定理ならば, 
$\forall_{A}x(R_{1} \vee R_{2} \vee \cdots \vee R_{n})$は$\mathscr{T}$の定理である.
\end{dedu}




\mathstrut
\begin{dedu}
\label{dedspquangwedge}%推論299
$A$を$\mathscr{T}$の関係式とし, $x$を文字とする.
また$n$を自然数とし, $R_{1}, R_{2}, \cdots, R_{n}$を$\mathscr{T}$の関係式とする.
また$i$を$n$以下の自然数とする.

1)
$\exists_{A}x(R_{1} \wedge R_{2} \wedge \cdots \wedge R_{n})$が$\mathscr{T}$の定理ならば, 
$\exists_{A}x(R_{i})$は$\mathscr{T}$の定理である.

2)
$\forall_{A}x(R_{1} \wedge R_{2} \wedge \cdots \wedge R_{n})$が$\mathscr{T}$の定理ならば, 
$\forall_{A}x(R_{i})$は$\mathscr{T}$の定理である.
\end{dedu}




\mathstrut
\begin{dedu}
\label{dedspquangwedge2}%推論300
$A$を$\mathscr{T}$の関係式とし, $x$を文字とする.
また$n$を自然数とし, $R_{1}, R_{2}, \cdots, R_{n}$を$\mathscr{T}$の関係式とする.
また$k$を自然数とし, $i_{1}, i_{2}, \cdots, i_{k}$を$n$以下の自然数とする.

1)
$\exists_{A}x(R_{1} \wedge R_{2} \wedge \cdots \wedge R_{n})$が$\mathscr{T}$の定理ならば, 
$\exists_{A}x(R_{i_{1}} \wedge R_{i_{2}} \wedge \cdots \wedge R_{i_{k}})$は$\mathscr{T}$の定理である.

2)
$\forall_{A}x(R_{1} \wedge R_{2} \wedge \cdots \wedge R_{n})$が$\mathscr{T}$の定理ならば, 
$\forall_{A}x(R_{i_{1}} \wedge R_{i_{2}} \wedge \cdots \wedge R_{i_{k}})$は$\mathscr{T}$の定理である.
\end{dedu}




\mathstrut
\begin{dedu}
\label{dedspexgw}%推論301
$A$を$\mathscr{T}$の関係式とし, $x$を文字とする.
また$n$を自然数とし, $R_{1}, R_{2}, \cdots, R_{n}$を$\mathscr{T}$の関係式とする.

1)
$\exists_{A}x(R_{1} \wedge R_{2} \wedge \cdots \wedge R_{n})$が$\mathscr{T}$の定理ならば, 
$\exists_{A}x(R_{1}) \wedge \exists_{A}x(R_{2}) \wedge \cdots \wedge \exists_{A}x(R_{n})$は$\mathscr{T}$の定理である.

2)
$i$を$n$以下の自然数とする.
\[
  \forall_{A}x(R_{1}) \wedge \cdots \wedge \forall_{A}x(R_{i - 1}) \wedge \exists_{A}x(R_{i}) \wedge \forall_{A}x(R_{i + 1}) \wedge \cdots \wedge \forall_{A}x(R_{n})
\]
が$\mathscr{T}$の定理ならば, 
$\exists_{A}x(R_{1} \wedge R_{2} \wedge \cdots \wedge R_{n})$は$\mathscr{T}$の定理である.
\end{dedu}




\mathstrut
\begin{dedu}
\label{dedspexgw2}%推論302
$A$を$\mathscr{T}$の関係式とし, $x$を文字とする.
また$n$を自然数とし, $R_{1}, R_{2}, \cdots, R_{n}$を$\mathscr{T}$の関係式とする.
また$i$を$n$以下の自然数とする.
$\forall_{A}x(R_{1}), \cdots, \forall_{A}x(R_{i - 1}), \exists_{A}x(R_{i}), \forall_{A}x(R_{i + 1}), \cdots, \forall_{A}x(R_{n})$が
すべて$\mathscr{T}$の定理ならば, 
$\exists_{A}x(R_{1} \wedge R_{2} \wedge \cdots \wedge R_{n})$は$\mathscr{T}$の定理である.
\end{dedu}




\mathstrut
\begin{dedu}
\label{dedspexgwfree}%推論303
$A$を$\mathscr{T}$の関係式とする.
また$n$を自然数とし, $R_{1}, R_{2}, \cdots, R_{n}$を$\mathscr{T}$の関係式とする.
また$k$を$n$以下の自然数とし, $i_{1}, i_{2}, \cdots, i_{k}$を
$i_{1} < i_{2} < \cdots < i_{k} \LEQQ n$なる自然数とする.
また$x$を$R_{i_{1}}, R_{i_{2}}, \cdots, R_{i_{k}}$の中に自由変数として現れない文字とする.
このとき$\exists_{A}x(R_{1} \wedge R_{2} \wedge \cdots \wedge R_{n})$が$\mathscr{T}$の定理ならば, 
\begin{multline*}
  \exists_{A}x(R_{1}) \wedge \cdots \wedge \exists_{A}x(R_{i_{1} - 1}) \wedge R_{i_{1}} \wedge \exists_{A}x(R_{i_{1} + 1}) \wedge \cdots\cdots \\
  \wedge \exists_{A}x(R_{i_{k} - 1}) \wedge R_{i_{k}} \wedge \exists_{A}x(R_{i_{k} + 1}) \wedge \cdots \wedge \exists_{A}x(R_{n})
\end{multline*}
は$\mathscr{T}$の定理である.
\end{dedu}




\mathstrut
\begin{dedu}
\label{dedspexgwfree2}%推論304
$A$を$\mathscr{T}$の関係式とする.
また$n$を自然数とし, $R_{1}, R_{2}, \cdots, R_{n}$を$\mathscr{T}$の関係式とする.
また$i$を$n$以下の自然数とし, $x$を$R_{i}$の中に自由変数として現れない文字とする.
$\exists_{A}x(R_{1} \wedge R_{2} \wedge \cdots \wedge R_{n})$が$\mathscr{T}$の定理ならば, 
$R_{i}$は$\mathscr{T}$の定理である.
\end{dedu}




\mathstrut
\begin{dedu}
\label{dedspexgwfreeeq}%推論305
$A$を$\mathscr{T}$の関係式とし, $x$を文字とする.
また$n$を自然数とし, $R_{1}, R_{2}, \cdots, R_{n}$を$\mathscr{T}$の関係式とする.
また$i$を$n$以下の自然数とする.
いま$i$と異なる$n$以下の任意の自然数$j$に対し, $x$は$R_{j}$の中に自由変数として現れないとする.

1)
$\exists_{A}x(R_{1} \wedge R_{2} \wedge \cdots \wedge R_{n})$が$\mathscr{T}$の定理ならば, 
$R_{1} \wedge \cdots \wedge R_{i - 1} \wedge \exists_{A}x(R_{i}) \wedge R_{i + 1} \wedge \cdots \wedge R_{n}$は$\mathscr{T}$の定理である.

2)
$R_{1} \wedge \cdots \wedge R_{i - 1} \wedge \exists_{A}x(R_{i}) \wedge R_{i + 1} \wedge \cdots \wedge R_{n}$が$\mathscr{T}$の定理ならば, 
$\exists_{A}x(R_{1} \wedge R_{2} \wedge \cdots \wedge R_{n})$は$\mathscr{T}$の定理である.
\end{dedu}




\mathstrut
\begin{dedu}
\label{dedspexgwfreeeq2}%推論306
$A$を$\mathscr{T}$の関係式とし, $x$を文字とする.
また$n$を自然数とし, $R_{1}, R_{2}, \cdots, R_{n}$を$\mathscr{T}$の関係式とする.
また$i$を$n$以下の自然数とする.
いま$i$と異なる$n$以下の任意の自然数$j$に対し, $x$は$R_{j}$の中に自由変数として現れないとする.
このとき$R_{1}, \cdots, R_{i - 1}, \exists_{A}x(R_{i}), R_{i + 1}, \cdots, R_{n}$が
すべて$\mathscr{T}$の定理ならば, 
$\exists_{A}x(R_{1} \wedge R_{2} \wedge \cdots \wedge R_{n})$は$\mathscr{T}$の定理である.
\end{dedu}




\mathstrut
\begin{dedu}
\label{dedspallgw}%推論307
$A$を$\mathscr{T}$の関係式とし, $x$を文字とする.
また$n$を自然数とし, $R_{1}, R_{2}, \cdots, R_{n}$を$\mathscr{T}$の関係式とする.

1)
$\forall_{A}x(R_{1} \wedge R_{2} \wedge \cdots \wedge R_{n})$が$\mathscr{T}$の定理ならば, 
$\forall_{A}x(R_{1}) \wedge \forall_{A}x(R_{2}) \wedge \cdots \wedge \forall_{A}x(R_{n})$は$\mathscr{T}$の定理である.

2)
$\forall_{A}x(R_{1}) \wedge \forall_{A}x(R_{2}) \wedge \cdots \wedge \forall_{A}x(R_{n})$が$\mathscr{T}$の定理ならば, 
$\forall_{A}x(R_{1} \wedge R_{2} \wedge \cdots \wedge R_{n})$は$\mathscr{T}$の定理である.
\end{dedu}




\mathstrut
\begin{dedu}
\label{dedspallgw2}%推論308
$A$を$\mathscr{T}$の関係式とし, $x$を文字とする.
また$n$を自然数とし, $R_{1}, R_{2}, \cdots, R_{n}$を$\mathscr{T}$の関係式とする.
$\forall_{A}x(R_{1}), \forall_{A}x(R_{2}), \cdots, \forall_{A}x(R_{n})$がすべて$\mathscr{T}$の定理ならば, 
$\forall_{A}x(R_{1} \wedge R_{2} \wedge \cdots \wedge R_{n})$は$\mathscr{T}$の定理である.
\end{dedu}




\mathstrut
\begin{dedu}
\label{dedspallgwfree}%推論309
$A$を$\mathscr{T}$の関係式とする.
また$n$を自然数とし, $R_{1}, R_{2}, \cdots, R_{n}$を$\mathscr{T}$の関係式とする.
また$k$を$n$以下の自然数とし, $i_{1}, i_{2}, \cdots, i_{k}$を
$i_{1} < i_{2} < \cdots < i_{k} \LEQQ n$なる自然数とする.
また$x$を$R_{i_{1}}, R_{i_{2}}, \cdots, R_{i_{k}}$の中に自由変数として現れない文字とする.
このとき
\begin{multline*}
  \forall_{A}x(R_{1}) \wedge \cdots \wedge \forall_{A}x(R_{i_{1} - 1}) \wedge R_{i_{1}} \wedge \forall_{A}x(R_{i_{1} + 1}) \wedge \cdots\cdots \\
  \wedge \forall_{A}x(R_{i_{k} - 1}) \wedge R_{i_{k}} \wedge \forall_{A}x(R_{i_{k} + 1}) \wedge \cdots \wedge \forall_{A}x(R_{n})
\end{multline*}
が$\mathscr{T}$の定理ならば, 
$\forall_{A}x(R_{1} \wedge R_{2} \wedge \cdots \wedge R_{n})$は$\mathscr{T}$の定理である.
\end{dedu}




\mathstrut
\begin{dedu}
\label{dedspallgwfree2}%推論310
$A$を$\mathscr{T}$の関係式とする.
また$n$を自然数とし, $R_{1}, R_{2}, \cdots, R_{n}$を$\mathscr{T}$の関係式とする.
また$k$を$n$以下の自然数とし, $i_{1}, i_{2}, \cdots, i_{k}$を
$i_{1} < i_{2} < \cdots < i_{k} \LEQQ n$なる自然数とする.
また$x$を$R_{i_{1}}, R_{i_{2}}, \cdots, R_{i_{k}}$の中に自由変数として現れない文字とする.
このとき
\[
  \forall_{A}x(R_{1}), \cdots, \forall_{A}x(R_{i_{1} - 1}), R_{i_{1}}, \forall_{A}x(R_{i_{1} + 1}), \cdots\cdots, 
  \forall_{A}x(R_{i_{k} - 1}), R_{i_{k}}, \forall_{A}x(R_{i_{k} + 1}), \cdots, \forall_{A}x(R_{n})
\]
がすべて$\mathscr{T}$の定理ならば, 
$\forall_{A}x(R_{1} \wedge R_{2} \wedge \cdots \wedge R_{n})$は$\mathscr{T}$の定理である.
\end{dedu}




\mathstrut
\begin{dedu}
\label{dedspallgwfreeeq}%推論311
$A$を$\mathscr{T}$の関係式とする.
また$n$を自然数とし, $R_{1}, R_{2}, \cdots, R_{n}$を$\mathscr{T}$の関係式とする.
また$k$を$n$以下の自然数とし, $i_{1}, i_{2}, \cdots, i_{k}$を
$i_{1} < i_{2} < \cdots < i_{k} \LEQQ n$なる自然数とする.
また$x$を$R_{i_{1}}, R_{i_{2}}, \cdots, R_{i_{k}}$の中に自由変数として現れない文字とする.
$\exists x(A)$が$\mathscr{T}$の定理ならば, 
\begin{multline*}
  \forall_{A}x(R_{1} \wedge R_{2} \wedge \cdots \wedge R_{n}) \\
  \leftrightarrow \forall_{A}x(R_{1}) \wedge \cdots \wedge \forall_{A}x(R_{i_{1} - 1}) \wedge R_{i_{1}} \wedge \forall_{A}x(R_{i_{1} + 1}) \wedge \cdots\cdots \\
  \wedge \forall_{A}x(R_{i_{k} - 1}) \wedge R_{i_{k}} \wedge \forall_{A}x(R_{i_{k} + 1}) \wedge \cdots \wedge \forall_{A}x(R_{n})
\end{multline*}
は$\mathscr{T}$の定理である.
\end{dedu}




\mathstrut
\begin{dedu}
\label{dedspallgwfreeeq2}%推論312
$A$を$\mathscr{T}$の関係式とする.
また$n$を自然数とし, $R_{1}, R_{2}, \cdots, R_{n}$を$\mathscr{T}$の関係式とする.
また$i$を$n$以下の自然数とし, $x$を$R_{i}$の中に自由変数として現れない文字とする.
$\exists x(A)$と$\forall_{A}x(R_{1} \wedge R_{2} \wedge \cdots \wedge R_{n})$が
共に$\mathscr{T}$の定理ならば, 
$R_{i}$は$\mathscr{T}$の定理である.
\end{dedu}




\mathstrut
\begin{dedu}
\label{dedspexpregvee}%推論313
$R$を$\mathscr{T}$の関係式とし, $x$を文字とする.
また$n$を自然数とし, $A_{1}, A_{2}, \cdots, A_{n}$を$\mathscr{T}$の関係式とする.
また$i$を$n$以下の自然数とする.
$\exists_{A_{i}}x(R)$が$\mathscr{T}$の定理ならば, 
$\exists_{A_{1} \vee A_{2} \vee \cdots \vee A_{n}}x(R)$は$\mathscr{T}$の定理である.
\end{dedu}




\mathstrut
\begin{dedu}
\label{dedspexpregvee2}%推論314
$R$を$\mathscr{T}$の関係式とし, $x$を文字とする.
また$n$を自然数とし, $A_{1}, A_{2}, \cdots, A_{n}$を$\mathscr{T}$の関係式とする.
また$k$を自然数とし, $i_{1}, i_{2}, \cdots, i_{k}$を$n$以下の自然数とする.
$\exists_{A_{i_{1}} \vee A_{i_{2}} \vee \cdots \vee A_{i_{k}}}x(R)$が$\mathscr{T}$の定理ならば, 
$\exists_{A_{1} \vee A_{2} \vee \cdots \vee A_{n}}x(R)$は$\mathscr{T}$の定理である.
\end{dedu}




\mathstrut
\begin{dedu}
\label{dedspexpregv}%推論315
$R$を$\mathscr{T}$の関係式とし, $x$を文字とする.
また$n$を自然数とし, $A_{1}, A_{2}, \cdots, A_{n}$を$\mathscr{T}$の関係式とする.

1)
$\exists_{A_{1} \vee A_{2} \vee \cdots \vee A_{n}}x(R)$が$\mathscr{T}$の定理ならば, 
$\exists_{A_{1}}x(R) \vee \exists_{A_{2}}x(R) \vee \cdots \vee \exists_{A_{n}}x(R)$は
$\mathscr{T}$の定理である.

2)
$\exists_{A_{1}}x(R) \vee \exists_{A_{2}}x(R) \vee \cdots \vee \exists_{A_{n}}x(R)$が
$\mathscr{T}$の定理ならば, 
$\exists_{A_{1} \vee A_{2} \vee \cdots \vee A_{n}}x(R)$は$\mathscr{T}$の定理である.
\end{dedu}




\mathstrut
\begin{dedu}
\label{dedspexpregvfree}%推論316
$R$を$\mathscr{T}$の関係式とする.
また$n$を自然数とし, $A_{1}, A_{2}, \cdots, A_{n}$を$\mathscr{T}$の関係式とする.
また$k$を$n$以下の自然数とし, $i_{1}, i_{2}, \cdots, i_{k}$を
$i_{1} < i_{2} < \cdots < i_{k} \LEQQ n$なる自然数とする.
また$x$を$A_{i_{1}}, A_{i_{2}}, \cdots, A_{i_{k}}$の中に自由変数として現れない文字とする.
このとき$\exists_{A_{1} \vee A_{2} \vee \cdots \vee A_{n}}x(R)$が$\mathscr{T}$の定理ならば, 
\begin{multline*}
  \exists_{A_{1}}x(R) \vee \cdots \vee \exists_{A_{i_{1} - 1}}x(R) \vee A_{i_{1}} \vee \exists_{A_{i_{1} + 1}}x(R) \vee \cdots\cdots \\
  \vee \exists_{A_{i_{k} - 1}}x(R) \vee A_{i_{k}} \vee \exists_{A_{i_{k} + 1}}x(R) \vee \cdots \vee \exists_{A_{n}}x(R)
\end{multline*}
は$\mathscr{T}$の定理である.
\end{dedu}




\mathstrut
\begin{dedu}
\label{dedspexpregvfreeeq}%推論317
$R$を$\mathscr{T}$の関係式とする.
また$n$を自然数とし, $A_{1}, A_{2}, \cdots, A_{n}$を$\mathscr{T}$の関係式とする.
また$k$を$n$以下の自然数とし, $i_{1}, i_{2}, \cdots, i_{k}$を
$i_{1} < i_{2} < \cdots < i_{k} \LEQQ n$なる自然数とする.
また$x$を$A_{i_{1}}, A_{i_{2}}, \cdots, A_{i_{k}}$の中に自由変数として現れない文字とする.
$\exists x(R)$が$\mathscr{T}$の定理ならば, 
\begin{multline*}
  \exists_{A_{1} \vee A_{2} \vee \cdots \vee A_{n}}x(R) 
  \leftrightarrow \exists_{A_{1}}x(R) \vee \cdots \vee \exists_{A_{i_{1} - 1}}x(R) \vee A_{i_{1}} \vee \exists_{A_{i_{1} + 1}}x(R) \vee \cdots\cdots \\
  \vee \exists_{A_{i_{k} - 1}}x(R) \vee A_{i_{k}} \vee \exists_{A_{i_{k} + 1}}x(R) \vee \cdots \vee \exists_{A_{n}}x(R)
\end{multline*}
は$\mathscr{T}$の定理である.
\end{dedu}




\mathstrut
\begin{dedu}
\label{dedspexpregvfreeeq2}%推論318
$R$を$\mathscr{T}$の関係式とする.
また$n$を自然数とし, $A_{1}, A_{2}, \cdots, A_{n}$を$\mathscr{T}$の関係式とする.
また$i$を$n$以下の自然数とし, $x$を$A_{i}$の中に自由変数として現れない文字とする.
$A_{i}$と$\exists x(R)$が共に$\mathscr{T}$の定理ならば, 
$\exists_{A_{1} \vee A_{2} \vee \cdots \vee A_{n}}x(R)$は$\mathscr{T}$の定理である.
\end{dedu}




\mathstrut
\begin{dedu}
\label{dedspallpregvee}%推論319
$R$を$\mathscr{T}$の関係式とし, $x$を文字とする.
また$n$を自然数とし, $A_{1}, A_{2}, \cdots, A_{n}$を$\mathscr{T}$の関係式とする.
また$i$を$n$以下の自然数とする.
$\forall_{A_{1} \vee A_{2} \vee \cdots \vee A_{n}}x(R)$が$\mathscr{T}$の定理ならば, 
$\forall_{A_{i}}x(R)$は$\mathscr{T}$の定理である.
\end{dedu}




\mathstrut
\begin{dedu}
\label{dedspallpregvee2}%推論320
$R$を$\mathscr{T}$の関係式とし, $x$を文字とする.
また$n$を自然数とし, $A_{1}, A_{2}, \cdots, A_{n}$を$\mathscr{T}$の関係式とする.
また$k$を自然数とし, $i_{1}, i_{2}, \cdots, i_{k}$を$n$以下の自然数とする.
$\forall_{A_{1} \vee A_{2} \vee \cdots \vee A_{n}}x(R)$が$\mathscr{T}$の定理ならば, 
$\forall_{A_{i_{1}} \vee A_{i_{2}} \vee \cdots \vee A_{i_{k}}}x(R)$は$\mathscr{T}$の定理である.
\end{dedu}




\mathstrut
\begin{dedu}
\label{dedspallpregv}%推論321
$R$を$\mathscr{T}$の関係式とし, $x$を文字とする.
また$n$を自然数とし, $A_{1}, A_{2}, \cdots, A_{n}$を$\mathscr{T}$の関係式とする.

1)
$\forall_{A_{1} \vee A_{2} \vee \cdots \vee A_{n}}x(R)$が$\mathscr{T}$の定理ならば, 
$\forall_{A_{1}}x(R) \wedge \forall_{A_{2}}x(R) \wedge \cdots \wedge \forall_{A_{n}}x(R)$は
$\mathscr{T}$の定理である.

2)
$\forall_{A_{1}}x(R) \wedge \forall_{A_{2}}x(R) \wedge \cdots \wedge \forall_{A_{n}}x(R)$が
$\mathscr{T}$の定理ならば, 
$\forall_{A_{1} \vee A_{2} \vee \cdots \vee A_{n}}x(R)$は$\mathscr{T}$の定理である.
\end{dedu}




\mathstrut
\begin{dedu}
\label{dedspallpregv2}%推論322
$R$を$\mathscr{T}$の関係式とし, $x$を文字とする.
また$n$を自然数とし, $A_{1}, A_{2}, \cdots, A_{n}$を$\mathscr{T}$の関係式とする.
$\forall_{A_{1}}x(R), \forall_{A_{2}}x(R), \cdots, \forall_{A_{n}}x(R)$が
すべて$\mathscr{T}$の定理ならば, 
$\forall_{A_{1} \vee A_{2} \vee \cdots \vee A_{n}}x(R)$は$\mathscr{T}$の定理である.
\end{dedu}




\mathstrut
\begin{dedu}
\label{dedspallpregvfree}%推論323
$R$を$\mathscr{T}$の関係式とする.
また$n$を自然数とし, $A_{1}, A_{2}, \cdots, A_{n}$を$\mathscr{T}$の関係式とする.
また$k$を$n$以下の自然数とし, $i_{1}, i_{2}, \cdots, i_{k}$を
$i_{1} < i_{2} < \cdots < i_{k} \LEQQ n$なる自然数とする.
また$x$を$A_{i_{1}}, A_{i_{2}}, \cdots, A_{i_{k}}$の中に自由変数として現れない文字とする.
このとき
\begin{multline*}
  \forall_{A_{1}}x(R) \wedge \cdots \wedge \forall_{A_{i_{1} - 1}}x(R) \wedge \neg A_{i_{1}} \wedge \forall_{A_{i_{1} + 1}}x(R) \wedge \cdots\cdots \\
  \wedge \forall_{A_{i_{k} - 1}}x(R) \wedge \neg A_{i_{k}} \wedge \forall_{A_{i_{k} + 1}}x(R) \wedge \cdots \wedge \forall_{A_{n}}x(R)
\end{multline*}
が$\mathscr{T}$の定理ならば, 
$\forall_{A_{1} \vee A_{2} \vee \cdots \vee A_{n}}x(R)$は$\mathscr{T}$の定理である.
\end{dedu}




\mathstrut
\begin{dedu}
\label{dedspallpregvfree2}%推論324
$R$を$\mathscr{T}$の関係式とする.
また$n$を自然数とし, $A_{1}, A_{2}, \cdots, A_{n}$を$\mathscr{T}$の関係式とする.
また$k$を$n$以下の自然数とし, $i_{1}, i_{2}, \cdots, i_{k}$を
$i_{1} < i_{2} < \cdots < i_{k} \LEQQ n$なる自然数とする.
また$x$を$A_{i_{1}}, A_{i_{2}}, \cdots, A_{i_{k}}$の中に自由変数として現れない文字とする.
このとき
\[
  \forall_{A_{1}}x(R), \cdots, \forall_{A_{i_{1} - 1}}x(R), \neg A_{i_{1}}, \forall_{A_{i_{1} + 1}}x(R), \cdots\cdots, 
  \forall_{A_{i_{k} - 1}}x(R), \neg A_{i_{k}}, \forall_{A_{i_{k} + 1}}x(R), \cdots, \forall_{A_{n}}x(R)
\]
がすべて$\mathscr{T}$の定理ならば, 
$\forall_{A_{1} \vee A_{2} \vee \cdots \vee A_{n}}x(R)$は$\mathscr{T}$の定理である.
\end{dedu}




\mathstrut
\begin{dedu}
\label{dedspallpregvfreeeq}%推論325
$R$を$\mathscr{T}$の関係式とする.
また$n$を自然数とし, $A_{1}, A_{2}, \cdots, A_{n}$を$\mathscr{T}$の関係式とする.
また$k$を$n$以下の自然数とし, $i_{1}, i_{2}, \cdots, i_{k}$を
$i_{1} < i_{2} < \cdots < i_{k} \LEQQ n$なる自然数とする.
また$x$を$A_{i_{1}}, A_{i_{2}}, \cdots, A_{i_{k}}$の中に自由変数として現れない文字とする.
$\exists x(\neg R)$が$\mathscr{T}$の定理ならば, 
\begin{multline*}
  \forall_{A_{1} \vee A_{2} \vee \cdots \vee A_{n}}x(R) 
  \leftrightarrow \forall_{A_{1}}x(R) \wedge \cdots \wedge \forall_{A_{i_{1} - 1}}x(R) \wedge \neg A_{i_{1}} \wedge \forall_{A_{i_{1} + 1}}x(R) \wedge \cdots\cdots \\
  \wedge \forall_{A_{i_{k} - 1}}x(R) \wedge \neg A_{i_{k}} \wedge \forall_{A_{i_{k} + 1}}x(R) \wedge \cdots \wedge \forall_{A_{n}}x(R)
\end{multline*}
は$\mathscr{T}$の定理である.
\end{dedu}




\mathstrut
\begin{dedu}
\label{dedspallpregvfreeeq2}%推論326
$R$を$\mathscr{T}$の関係式とする.
また$n$を自然数とし, $A_{1}, A_{2}, \cdots, A_{n}$を$\mathscr{T}$の関係式とする.
また$i$を$n$以下の自然数とし, $x$を$A_{i}$の中に自由変数として現れない文字とする.
$\exists x(\neg R)$と$\forall_{A_{1} \vee A_{2} \vee \cdots \vee A_{n}}x(R)$が
共に$\mathscr{T}$の定理ならば, 
$\neg A_{i}$は$\mathscr{T}$の定理である.
\end{dedu}




\mathstrut
\begin{dedu}
\label{dedspexpregwedge}%推論327
$R$を$\mathscr{T}$の関係式とし, $x$を文字とする.
また$n$を自然数とし, $A_{1}, A_{2}, \cdots, A_{n}$を$\mathscr{T}$の関係式とする.
また$i$を$n$以下の自然数とする.
$\exists_{A_{1} \wedge A_{2} \wedge \cdots \wedge A_{n}}x(R)$が$\mathscr{T}$の定理ならば, 
$\exists_{A_{i}}x(R)$は$\mathscr{T}$の定理である.
\end{dedu}




\mathstrut
\begin{dedu}
\label{dedspexpregwedge2}%推論328
$R$を$\mathscr{T}$の関係式とし, $x$を文字とする.
また$n$を自然数とし, $A_{1}, A_{2}, \cdots, A_{n}$を$\mathscr{T}$の関係式とする.
また$k$を自然数とし, $i_{1}, i_{2}, \cdots, i_{k}$を$n$以下の自然数とする.
$\exists_{A_{1} \wedge A_{2} \wedge \cdots \wedge A_{n}}x(R)$が$\mathscr{T}$の定理ならば, 
$\exists_{A_{i_{1}} \wedge A_{i_{2}} \wedge \cdots \wedge A_{i_{k}}}x(R)$は$\mathscr{T}$の定理である.
\end{dedu}




\mathstrut
\begin{dedu}
\label{dedspexpregw}%推論329
$R$を$\mathscr{T}$の関係式とし, $x$を文字とする.
また$n$を自然数とし, $A_{1}, A_{2}, \cdots, A_{n}$を$\mathscr{T}$の関係式とする.
$\exists_{A_{1} \wedge A_{2} \wedge \cdots \wedge A_{n}}x(R)$が$\mathscr{T}$の定理ならば, 
$\exists_{A_{1}}x(R) \wedge \exists_{A_{2}}x(R) \wedge \cdots \wedge \exists_{A_{n}}x(R)$は
$\mathscr{T}$の定理である.
\end{dedu}




\mathstrut
\begin{dedu}
\label{dedspexpregwfree}%推論330
$R$を$\mathscr{T}$の関係式とする.
また$n$を自然数とし, $A_{1}, A_{2}, \cdots, A_{n}$を$\mathscr{T}$の関係式とする.
また$k$と$l$を$k + l = n$なる自然数とし, 
$i_{1}, i_{2}, \cdots, i_{k}, j_{1}, j_{2}, \cdots, j_{l}$を
どの二つも互いに異なる$n$以下の自然数とする.
また$x$を$A_{i_{1}}, A_{i_{2}}, \cdots, A_{i_{k}}$の中に自由変数として現れない文字とする.

1)
$\exists_{A_{1} \wedge A_{2} \wedge \cdots \wedge A_{n}}x(R)$が$\mathscr{T}$の定理ならば, 
$A_{i_{1}} \wedge A_{i_{2}} \wedge \cdots \wedge A_{i_{k}} \wedge \exists_{A_{j_{1}} \wedge A_{j_{2}} \wedge \cdots \wedge A_{j_{l}}}x(R)$は
$\mathscr{T}$の定理である.

2)
$A_{i_{1}} \wedge A_{i_{2}} \wedge \cdots \wedge A_{i_{k}} \wedge \exists_{A_{j_{1}} \wedge A_{j_{2}} \wedge \cdots \wedge A_{j_{l}}}x(R)$が
$\mathscr{T}$の定理ならば, 
$\exists_{A_{1} \wedge A_{2} \wedge \cdots \wedge A_{n}}x(R)$は$\mathscr{T}$の定理である.
\end{dedu}




\mathstrut
\begin{dedu}
\label{dedspexpregwfree2}%推論331
$R$を$\mathscr{T}$の関係式とする.
また$n$を自然数とし, $A_{1}, A_{2}, \cdots, A_{n}$を$\mathscr{T}$の関係式とする.
また$k$と$l$を$k + l = n$なる自然数とし, 
$i_{1}, i_{2}, \cdots, i_{k}, j_{1}, j_{2}, \cdots, j_{l}$を
どの二つも互いに異なる$n$以下の自然数とする.
また$x$を$A_{i_{1}}, A_{i_{2}}, \cdots, A_{i_{k}}$の中に自由変数として現れない文字とする.

1)
$\exists_{A_{1} \wedge A_{2} \wedge \cdots \wedge A_{n}}x(R)$が$\mathscr{T}$の定理ならば, 
$A_{i_{1}}, A_{i_{2}}, \cdots, A_{i_{k}}$はいずれも$\mathscr{T}$の定理である.

2)
$A_{i_{1}}, A_{i_{2}}, \cdots, A_{i_{k}}, \exists_{A_{j_{1}} \wedge A_{j_{2}} \wedge \cdots \wedge A_{j_{l}}}x(R)$が
すべて$\mathscr{T}$の定理ならば, 
$\exists_{A_{1} \wedge A_{2} \wedge \cdots \wedge A_{n}}x(R)$は$\mathscr{T}$の定理である.
\end{dedu}




\mathstrut
\begin{dedu}
\label{dedspallpregwedge}%推論332
$R$を$\mathscr{T}$の関係式とし, $x$を文字とする.
また$n$を自然数とし, $A_{1}, A_{2}, \cdots, A_{n}$を$\mathscr{T}$の関係式とする.
また$i$を$n$以下の自然数とする.
$\forall_{A_{i}}x(R)$が$\mathscr{T}$の定理ならば, 
$\forall_{A_{1} \wedge A_{2} \wedge \cdots \wedge A_{n}}x(R)$は$\mathscr{T}$の定理である.
\end{dedu}




\mathstrut
\begin{dedu}
\label{dedspallpregwedge2}%推論333
$R$を$\mathscr{T}$の関係式とし, $x$を文字とする.
また$n$を自然数とし, $A_{1}, A_{2}, \cdots, A_{n}$を$\mathscr{T}$の関係式とする.
また$k$を自然数とし, $i_{1}, i_{2}, \cdots, i_{k}$を$n$以下の自然数とする.
$\forall_{A_{i_{1}} \wedge A_{i_{2}} \wedge \cdots \wedge A_{i_{k}}}x(R)$が$\mathscr{T}$の定理ならば, 
$\forall_{A_{1} \wedge A_{2} \wedge \cdots \wedge A_{n}}x(R)$は$\mathscr{T}$の定理である.
\end{dedu}




\mathstrut
\begin{dedu}
\label{dedspallpregw}%推論334
$R$を$\mathscr{T}$の関係式とし, $x$を文字とする.
また$n$を自然数とし, $A_{1}, A_{2}, \cdots, A_{n}$を$\mathscr{T}$の関係式とする.
$\forall_{A_{1}}x(R) \vee \forall_{A_{2}}x(R) \vee \cdots \vee \forall_{A_{n}}x(R)$が
$\mathscr{T}$の定理ならば, 
$\forall_{A_{1} \wedge A_{2} \wedge \cdots \wedge A_{n}}x(R)$は$\mathscr{T}$の定理である.
\end{dedu}




\mathstrut
\begin{dedu}
\label{dedspallpregwfree}%推論335
$R$を$\mathscr{T}$の関係式とする.
また$n$を自然数とし, $A_{1}, A_{2}, \cdots, A_{n}$を$\mathscr{T}$の関係式とする.
また$k$と$l$を$k + l = n$なる自然数とし, 
$i_{1}, i_{2}, \cdots, i_{k}, j_{1}, j_{2}, \cdots, j_{l}$を
どの二つも互いに異なる$n$以下の自然数とする.
また$x$を$A_{i_{1}}, A_{i_{2}}, \cdots, A_{i_{k}}$の中に自由変数として現れない文字とする.

1)
$\forall_{A_{1} \wedge A_{2} \wedge \cdots \wedge A_{n}}x(R)$が$\mathscr{T}$の定理ならば, 
$A_{i_{1}} \wedge A_{i_{2}} \wedge \cdots \wedge A_{i_{k}} \to \forall_{A_{j_{1}} \wedge A_{j_{2}} \wedge \cdots \wedge A_{j_{l}}}x(R)$は
$\mathscr{T}$の定理である.

2)
$A_{i_{1}} \wedge A_{i_{2}} \wedge \cdots \wedge A_{i_{k}} \to \forall_{A_{j_{1}} \wedge A_{j_{2}} \wedge \cdots \wedge A_{j_{l}}}x(R)$が
$\mathscr{T}$の定理ならば, 
$\forall_{A_{1} \wedge A_{2} \wedge \cdots \wedge A_{n}}x(R)$は$\mathscr{T}$の定理である.
\end{dedu}




\mathstrut
\begin{dedu}
\label{dedspallpregwfree2}%推論336
$R$を$\mathscr{T}$の関係式とする.
また$n$を自然数とし, $A_{1}, A_{2}, \cdots, A_{n}$を$\mathscr{T}$の関係式とする.
また$i$を$n$以下の自然数とし, $x$を$A_{i}$の中に自由変数として現れない文字とする.
$\neg A_{i}$が$\mathscr{T}$の定理ならば, 
$\forall_{A_{1} \wedge A_{2} \wedge \cdots \wedge A_{n}}x(R)$は$\mathscr{T}$の定理である.
\end{dedu}




\mathstrut
\begin{dedu}
\label{dedspextspquansep}%推論337
$A$, $R$, $S$を$\mathscr{T}$の関係式とし, $x$を文字とする.

1)
$\exists_{A}x(R \to S)$が$\mathscr{T}$の定理ならば, 
$\forall_{A}x(R) \to \exists_{A}x(S)$は$\mathscr{T}$の定理である.

2)
$\forall_{A}x(R) \to \exists_{A}x(S)$が$\mathscr{T}$の定理ならば, 
$\exists_{A}x(R \to S)$は$\mathscr{T}$の定理である.
\end{dedu}




\mathstrut
\begin{dedu}
\label{dedspextspquansep2}%推論338
$A$, $R$, $S$を$\mathscr{T}$の関係式とし, $x$を文字とする.

1)
$\exists_{A}x(S)$が$\mathscr{T}$の定理ならば, 
$\exists_{A}x(R \to S)$は$\mathscr{T}$の定理である.

2)
$\neg \forall_{A}x(R)$が$\mathscr{T}$の定理ならば, 
$\exists_{A}x(R \to S)$は$\mathscr{T}$の定理である.

3)
$\exists_{A}x(\neg R)$が$\mathscr{T}$の定理ならば, 
$\exists_{A}x(R \to S)$は$\mathscr{T}$の定理である.
\end{dedu}




\mathstrut
\begin{dedu}
\label{dedspextspquanseprfree}%推論339
$A$, $R$, $S$を$\mathscr{T}$の関係式とし, $x$を$R$の中に自由変数として現れない文字とする.
$\exists_{A}x(R \to S)$が$\mathscr{T}$の定理ならば, 
$R \to \exists_{A}x(S)$は$\mathscr{T}$の定理である.
\end{dedu}




\mathstrut
\begin{dedu}
\label{dedspextspquanseprfreeeq}%推論340
$A$, $R$, $S$を$\mathscr{T}$の関係式とし, $x$を$R$の中に自由変数として現れない文字とする.

1)
$\exists x(A)$が$\mathscr{T}$の定理ならば, 
$\exists_{A}x(R \to S) \leftrightarrow (R \to \exists_{A}x(S))$は$\mathscr{T}$の定理である.

2)
$\exists x(A)$と$R \to \exists_{A}x(S)$が共に$\mathscr{T}$の定理ならば, 
$\exists_{A}x(R \to S)$は$\mathscr{T}$の定理である.
\end{dedu}




\mathstrut
\begin{dedu}
\label{dedspextspquansepsfree}%推論341
$A$, $R$, $S$を$\mathscr{T}$の関係式とし, $x$を$S$の中に自由変数として現れない文字とする.
$\exists_{A}x(R \to S)$が$\mathscr{T}$の定理ならば, 
$\forall_{A}x(R) \to S$は$\mathscr{T}$の定理である.
\end{dedu}




\mathstrut
\begin{dedu}
\label{dedspextspquansepsfreeeq}%推論342
$A$, $R$, $S$を$\mathscr{T}$の関係式とし, $x$を$S$の中に自由変数として現れない文字とする.

1)
$\exists x(A)$が$\mathscr{T}$の定理ならば, 
$\exists_{A}x(R \to S) \leftrightarrow (\forall_{A}x(R) \to S)$は$\mathscr{T}$の定理である.

2)
$\exists x(A)$と$\forall_{A}x(R) \to S$が共に$\mathscr{T}$の定理ならば, 
$\exists_{A}x(R \to S)$は$\mathscr{T}$の定理である.
\end{dedu}




\mathstrut
\begin{dedu}
\label{dedspalltspquansep}%推論343
$A$, $R$, $S$を$\mathscr{T}$の関係式とし, $x$を文字とする.

1)
$\forall_{A}x(R \to S)$が$\mathscr{T}$の定理ならば, 
$\exists_{A}x(R) \to \exists_{A}x(S)$と$\forall_{A}x(R) \to \forall_{A}x(S)$は
共に$\mathscr{T}$の定理である.

2)
$\exists_{A}x(R) \to \forall_{A}x(S)$が$\mathscr{T}$の定理ならば, 
$\forall_{A}x(R \to S)$は$\mathscr{T}$の定理である.
\end{dedu}




\mathstrut
\begin{dedu}
\label{dedspalltspquansepconst}%推論344
$A$, $R$, $S$を$\mathscr{T}$の関係式とし, $x$を$\mathscr{T}$の定数でない文字とする.
$A \to (R \to S)$が$\mathscr{T}$の定理ならば, 
$\exists_{A}x(R) \to \exists_{A}x(S)$と$\forall_{A}x(R) \to \forall_{A}x(S)$は
共に$\mathscr{T}$の定理である.
\end{dedu}




\mathstrut
\begin{dedu}
\label{dedalltspquansep}%推論345
$A$, $R$, $S$を$\mathscr{T}$の関係式とし, $x$を文字とする.
$\forall x(R \to S)$が$\mathscr{T}$の定理ならば, 
$\exists_{A}x(R) \to \exists_{A}x(S)$と$\forall_{A}x(R) \to \forall_{A}x(S)$は
共に$\mathscr{T}$の定理である.
\end{dedu}




\mathstrut
\begin{dedu}
\label{dedalltspquansepconst}%推論346
$A$, $R$, $S$を$\mathscr{T}$の関係式とし, $x$を$\mathscr{T}$の定数でない文字とする.
$R \to S$が$\mathscr{T}$の定理ならば, 
$\exists_{A}x(R) \to \exists_{A}x(S)$と$\forall_{A}x(R) \to \forall_{A}x(S)$は
共に$\mathscr{T}$の定理である.
\end{dedu}




\mathstrut
\begin{dedu}
\label{dedspquanseptspall2}%推論347
$A$, $R$, $S$を$\mathscr{T}$の関係式とし, $x$を文字とする.

1)
$\forall_{A}x(S)$が$\mathscr{T}$の定理ならば, 
$\forall_{A}x(R \to S)$は$\mathscr{T}$の定理である.

2)
$\neg \exists_{A}x(R)$が$\mathscr{T}$の定理ならば, 
$\forall_{A}x(R \to S)$は$\mathscr{T}$の定理である.

3)
$\forall_{A}x(\neg R)$が$\mathscr{T}$の定理ならば, 
$\forall_{A}x(R \to S)$は$\mathscr{T}$の定理である.
\end{dedu}




\mathstrut
\begin{dedu}
\label{dedspquanseptspall2const}%推論348
$A$, $R$, $S$を$\mathscr{T}$の関係式とし, $x$を$\mathscr{T}$の定数でない文字とする.

1)
$A \to S$が$\mathscr{T}$の定理ならば, 
$\forall_{A}x(R \to S)$は$\mathscr{T}$の定理である.

2)
$S$が$\mathscr{T}$の定理ならば, 
$\forall_{A}x(R \to S)$は$\mathscr{T}$の定理である.

3)
$A \to \neg R$が$\mathscr{T}$の定理ならば, 
$\forall_{A}x(R \to S)$は$\mathscr{T}$の定理である.

4)
$\neg R$が$\mathscr{T}$の定理ならば, 
$\forall_{A}x(R \to S)$は$\mathscr{T}$の定理である.
\end{dedu}




\mathstrut
\begin{dedu}
\label{dedspalltspquansepfree}%推論349
$A$, $R$, $S$を$\mathscr{T}$の関係式とし, $x$を1), 2)では$R$の中に自由変数として現れない文字, 
3), 4)では$S$の中に自由変数として現れない文字とする.

1)
$\forall_{A}x(R \to S)$が$\mathscr{T}$の定理ならば, 
$R \to \forall_{A}x(S)$は$\mathscr{T}$の定理である.

2)
$R \to \forall_{A}x(S)$が$\mathscr{T}$の定理ならば, 
$\forall_{A}x(R \to S)$は$\mathscr{T}$の定理である.

3)
$\forall_{A}x(R \to S)$が$\mathscr{T}$の定理ならば, 
$\exists_{A}x(R) \to S$は$\mathscr{T}$の定理である.

4)
$\exists_{A}x(R) \to S$が$\mathscr{T}$の定理ならば, 
$\forall_{A}x(R \to S)$は$\mathscr{T}$の定理である.
\end{dedu}




\mathstrut
\begin{dedu}
\label{dedspalltspquansepfreeconst}%推論350
$A$, $R$, $S$を$\mathscr{T}$の関係式とし, 
$A \to (R \to S)$または$R \to S$が$\mathscr{T}$の定理であるとする.
また$x$を$\mathscr{T}$の定数でない文字とする.

1)
$x$が$R$の中に自由変数として現れなければ, 
$R \to \forall_{A}x(S)$は$\mathscr{T}$の定理である.

2)
$x$が$S$の中に自由変数として現れなければ, 
$\exists_{A}x(R) \to S$は$\mathscr{T}$の定理である.
\end{dedu}




\mathstrut
\begin{dedu}
\label{dedspallpretspquansep}%推論351
$A$, $B$, $R$を$\mathscr{T}$の関係式とし, $x$を文字とする.

1)
$\forall_{R}x(A \to B)$が$\mathscr{T}$の定理ならば, 
$\exists_{A}x(R) \to \exists_{B}x(R)$は$\mathscr{T}$の定理である.

2)
$\forall_{\neg R}x(A \to B)$が$\mathscr{T}$の定理ならば, 
$\forall_{B}x(R) \to \forall_{A}x(R)$は$\mathscr{T}$の定理である.
\end{dedu}




\mathstrut
\begin{dedu}
\label{dedspallpretspquansepconst}%推論352
$A$, $B$, $R$を$\mathscr{T}$の関係式とし, $x$を$\mathscr{T}$の定数でない文字とする.

1)
$R \to (A \to B)$が$\mathscr{T}$の定理ならば, 
$\exists_{A}x(R) \to \exists_{B}x(R)$は$\mathscr{T}$の定理である.

2)
$\neg R \to (A \to B)$が$\mathscr{T}$の定理ならば, 
$\forall_{B}x(R) \to \forall_{A}x(R)$は$\mathscr{T}$の定理である.
\end{dedu}




\mathstrut
\begin{dedu}
\label{dedallpretspquansep}%推論353
$A$, $B$, $R$を$\mathscr{T}$の関係式とし, $x$を文字とする.
$\forall x(A \to B)$が$\mathscr{T}$の定理ならば, 
$\exists_{A}x(R) \to \exists_{B}x(R)$と$\forall_{B}x(R) \to \forall_{A}x(R)$は
共に$\mathscr{T}$の定理である.
\end{dedu}




\mathstrut
\begin{dedu}
\label{dedallpretspquansepconst}%推論354
$A$, $B$, $R$を$\mathscr{T}$の関係式とし, $x$を$\mathscr{T}$の定数でない文字とする.
$A \to B$が$\mathscr{T}$の定理ならば, 
$\exists_{A}x(R) \to \exists_{B}x(R)$と$\forall_{B}x(R) \to \forall_{A}x(R)$は
共に$\mathscr{T}$の定理である.
\end{dedu}




\mathstrut
\begin{dedu}
\label{dedspquansepeqspex}%推論355
$A$, $R$, $S$を$\mathscr{T}$の関係式とし, $x$を文字とする.

1)
$\exists_{A}x(R) \leftrightarrow \forall_{A}x(S)$が$\mathscr{T}$の定理ならば, 
$\exists_{A}x(R \leftrightarrow S)$は$\mathscr{T}$の定理である.

2)
$\forall_{A}x(R) \leftrightarrow \exists_{A}x(S)$が$\mathscr{T}$の定理ならば, 
$\exists_{A}x(R \leftrightarrow S)$は$\mathscr{T}$の定理である.
\end{dedu}




\mathstrut
\begin{dedu}
\label{dedspquansepeqspexrfree}%推論356
$A$, $R$, $S$を$\mathscr{T}$の関係式とし, $x$を$R$の中に自由変数として現れない文字とする.

1)
$\exists x(A)$が$\mathscr{T}$の定理ならば, 
$(R \leftrightarrow \exists_{A}x(S)) \to \exists_{A}x(R \leftrightarrow S)$と
$(R \leftrightarrow \forall_{A}x(S)) \to \exists_{A}x(R \leftrightarrow S)$は
共に$\mathscr{T}$の定理である.

2)
$\exists x(A)$と$R \leftrightarrow \exists_{A}x(S)$が共に$\mathscr{T}$の定理ならば, 
$\exists_{A}x(R \leftrightarrow S)$は$\mathscr{T}$の定理である.

3)
$\exists x(A)$と$R \leftrightarrow \forall_{A}x(S)$が共に$\mathscr{T}$の定理ならば, 
$\exists_{A}x(R \leftrightarrow S)$は$\mathscr{T}$の定理である.
\end{dedu}




\mathstrut
\begin{dedu}
\label{dedspquansepeqspexsfree}%推論357
$A$, $R$, $S$を$\mathscr{T}$の関係式とし, $x$を$S$の中に自由変数として現れない文字とする.

1)
$\exists x(A)$が$\mathscr{T}$の定理ならば, 
$(\exists_{A}x(R) \leftrightarrow S) \to \exists_{A}x(R \leftrightarrow S)$と
$(\forall_{A}x(R) \leftrightarrow S) \to \exists_{A}x(R \leftrightarrow S)$は
共に$\mathscr{T}$の定理である.

2)
$\exists x(A)$と$\exists_{A}x(R) \leftrightarrow S$が共に$\mathscr{T}$の定理ならば, 
$\exists_{A}x(R \leftrightarrow S)$は$\mathscr{T}$の定理である.

3)
$\exists x(A)$と$\forall_{A}x(R) \leftrightarrow S$が共に$\mathscr{T}$の定理ならば, 
$\exists_{A}x(R \leftrightarrow S)$は$\mathscr{T}$の定理である.
\end{dedu}




\mathstrut
\begin{dedu}
\label{dedspalleqspquansep}%推論358
$A$, $R$, $S$を$\mathscr{T}$の関係式とし, $x$を文字とする.
$\forall_{A}x(R \leftrightarrow S)$が$\mathscr{T}$の定理ならば, 
$\exists_{A}x(R) \leftrightarrow \exists_{A}x(S)$と
$\forall_{A}x(R) \leftrightarrow \forall_{A}x(S)$は共に$\mathscr{T}$の定理である.
\end{dedu}




\mathstrut
\begin{dedu}
\label{dedspalleqspquansepconst}%推論359
$A$, $R$, $S$を$\mathscr{T}$の関係式とし, $x$を$\mathscr{T}$の定数でない文字とする.
$A \to (R \leftrightarrow S)$が$\mathscr{T}$の定理ならば, 
$\exists_{A}x(R) \leftrightarrow \exists_{A}x(S)$と
$\forall_{A}x(R) \leftrightarrow \forall_{A}x(S)$は共に$\mathscr{T}$の定理である.
\end{dedu}




\mathstrut
\begin{dedu}
\label{dedalleqspquansep}%推論360
$A$, $R$, $S$を$\mathscr{T}$の関係式とし, $x$を文字とする.
$\forall x(R \leftrightarrow S)$が$\mathscr{T}$の定理ならば, 
$\exists_{A}x(R) \leftrightarrow \exists_{A}x(S)$と
$\forall_{A}x(R) \leftrightarrow \forall_{A}x(S)$は
共に$\mathscr{T}$の定理である.
\end{dedu}




\mathstrut
\begin{dedu}
\label{dedalleqspquansepconst}%推論361
$A$, $R$, $S$を$\mathscr{T}$の関係式とし, $x$を$\mathscr{T}$の定数でない文字とする.
$R \leftrightarrow S$が$\mathscr{T}$の定理ならば, 
$\exists_{A}x(R) \leftrightarrow \exists_{A}x(S)$と
$\forall_{A}x(R) \leftrightarrow \forall_{A}x(S)$は
共に$\mathscr{T}$の定理である.
\end{dedu}




\mathstrut
\begin{dedu}
\label{dedspalleqspquanseprfree}%推論362
$A$, $R$, $S$を$\mathscr{T}$の関係式とし, $x$を$R$の中に自由変数として現れない文字とする.

1)
$\exists x(A)$が$\mathscr{T}$の定理ならば, 
$\forall_{A}x(R \leftrightarrow S) \to (R \leftrightarrow \exists_{A}x(S))$と
$\forall_{A}x(R \leftrightarrow S) \to (R \leftrightarrow \forall_{A}x(S))$は
共に$\mathscr{T}$の定理である.

2)
$\exists x(A)$と$\forall_{A}x(R \leftrightarrow S)$が共に$\mathscr{T}$の定理ならば, 
$R \leftrightarrow \exists_{A}x(S)$と$R \leftrightarrow \forall_{A}x(S)$は
共に$\mathscr{T}$の定理である.
\end{dedu}




\mathstrut
\begin{dedu}
\label{dedspalleqspquanseprfreeconst}%推論363
$A$, $R$, $S$を$\mathscr{T}$の関係式とし, 
$x$を$R$の中に自由変数として現れない, $\mathscr{T}$の定数でない文字とする.
$\exists x(A)$と$A \to (R \leftrightarrow S)$が共に$\mathscr{T}$の定理ならば, 
$R \leftrightarrow \exists_{A}x(S)$と$R \leftrightarrow \forall_{A}x(S)$は
共に$\mathscr{T}$の定理である.
\end{dedu}




\mathstrut
\begin{dedu}
\label{dedalleqspquanseprfree}%推論364
$A$, $R$, $S$を$\mathscr{T}$の関係式とし, $x$を$R$の中に自由変数として現れない文字とする.

1)
$\exists x(A)$が$\mathscr{T}$の定理ならば, 
$\forall x(R \leftrightarrow S) \to (R \leftrightarrow \exists_{A}x(S))$と
$\forall x(R \leftrightarrow S) \to (R \leftrightarrow \forall_{A}x(S))$は
共に$\mathscr{T}$の定理である.

2)
$\exists x(A)$と$\forall x(R \leftrightarrow S)$が共に$\mathscr{T}$の定理ならば, 
$R \leftrightarrow \exists_{A}x(S)$と$R \leftrightarrow \forall_{A}x(S)$は
共に$\mathscr{T}$の定理である.
\end{dedu}




\mathstrut
\begin{dedu}
\label{dedalleqspquanseprfreeconst}%推論365
$A$, $R$, $S$を$\mathscr{T}$の関係式とし, 
$x$を$R$の中に自由変数として現れない, $\mathscr{T}$の定数でない文字とする.
$\exists x(A)$と$R \leftrightarrow S$が共に$\mathscr{T}$の定理ならば, 
$R \leftrightarrow \exists_{A}x(S)$と$R \leftrightarrow \forall_{A}x(S)$は
共に$\mathscr{T}$の定理である.
\end{dedu}




\mathstrut
\begin{dedu}
\label{dedspalleqspquansepsfree}%推論366
$A$, $R$, $S$を$\mathscr{T}$の関係式とし, $x$を$S$の中に自由変数として現れない文字とする.

1)
$\exists x(A)$が$\mathscr{T}$の定理ならば, 
$\forall_{A}x(R \leftrightarrow S) \to (\exists_{A}x(R) \leftrightarrow S)$と
$\forall_{A}x(R \leftrightarrow S) \to (\forall_{A}x(R) \leftrightarrow S)$は
共に$\mathscr{T}$の定理である.

2)
$\exists x(A)$と$\forall_{A}x(R \leftrightarrow S)$が共に$\mathscr{T}$の定理ならば, 
$\exists_{A}x(R) \leftrightarrow S$と$\forall_{A}x(R) \leftrightarrow S$は
共に$\mathscr{T}$の定理である.
\end{dedu}




\mathstrut
\begin{dedu}
\label{dedspalleqspquansepsfreeconst}%推論367
$A$, $R$, $S$を$\mathscr{T}$の関係式とし, 
$x$を$S$の中に自由変数として現れない, $\mathscr{T}$の定数でない文字とする.
$\exists x(A)$と$A \to (R \leftrightarrow S)$が共に$\mathscr{T}$の定理ならば, 
$\exists_{A}x(R) \leftrightarrow S$と$\forall_{A}x(R) \leftrightarrow S$は
共に$\mathscr{T}$の定理である.
\end{dedu}




\mathstrut
\begin{dedu}
\label{dedalleqspquansepsfree}%推論368
$A$, $R$, $S$を$\mathscr{T}$の関係式とし, $x$を$S$の中に自由変数として現れない文字とする.

1)
$\exists x(A)$が$\mathscr{T}$の定理ならば, 
$\forall x(R \leftrightarrow S) \to (\exists_{A}x(R) \leftrightarrow S)$と
$\forall x(R \leftrightarrow S) \to (\forall_{A}x(R) \leftrightarrow S)$は
共に$\mathscr{T}$の定理である.

2)
$\exists x(A)$と$\forall x(R \leftrightarrow S)$が共に$\mathscr{T}$の定理ならば, 
$\exists_{A}x(R) \leftrightarrow S$と$\forall_{A}x(R) \leftrightarrow S$は
共に$\mathscr{T}$の定理である.
\end{dedu}




\mathstrut
\begin{dedu}
\label{dedalleqspquansepsfreeconst}%推論369
$A$, $R$, $S$を$\mathscr{T}$の関係式とし, 
$x$を$S$の中に自由変数として現れない, $\mathscr{T}$の定数でない文字とする.
$\exists x(A)$と$R \leftrightarrow S$が共に$\mathscr{T}$の定理ならば, 
$\exists_{A}x(R) \leftrightarrow S$と$\forall_{A}x(R) \leftrightarrow S$は
共に$\mathscr{T}$の定理である.
\end{dedu}




\mathstrut
\begin{dedu}
\label{dedspallpreeqspquansep}%推論370
$A$, $B$, $R$を$\mathscr{T}$の関係式とし, $x$を文字とする.

1)
$\forall_{R}x(A \leftrightarrow B)$が$\mathscr{T}$の定理ならば, 
$\exists_{A}x(R) \leftrightarrow \exists_{B}x(R)$は$\mathscr{T}$の定理である.

2)
$\forall_{\neg R}x(A \leftrightarrow B)$が$\mathscr{T}$の定理ならば, 
$\forall_{A}x(R) \leftrightarrow \forall_{B}x(R)$は$\mathscr{T}$の定理である.
\end{dedu}




\mathstrut
\begin{dedu}
\label{dedspallpreeqspquansepconst}%推論371
$A$, $B$, $R$を$\mathscr{T}$の関係式とし, $x$を$\mathscr{T}$の定数でない文字とする.

1)
$R \to (A \leftrightarrow B)$が$\mathscr{T}$の定理ならば, 
$\exists_{A}x(R) \leftrightarrow \exists_{B}x(R)$は$\mathscr{T}$の定理である.

2)
$\neg R \to (A \leftrightarrow B)$が$\mathscr{T}$の定理ならば, 
$\forall_{A}x(R) \leftrightarrow \forall_{B}x(R)$は$\mathscr{T}$の定理である.
\end{dedu}




\mathstrut
\begin{dedu}
\label{dedallpreeqspquansep}%推論372
$A$, $B$, $R$を$\mathscr{T}$の関係式とし, $x$を文字とする.
$\forall x(A \leftrightarrow B)$が$\mathscr{T}$の定理ならば, 
$\exists_{A}x(R) \leftrightarrow \exists_{B}x(R)$と
$\forall_{A}x(R) \leftrightarrow \forall_{B}x(R)$は
共に$\mathscr{T}$の定理である.
\end{dedu}




\mathstrut
\begin{dedu}
\label{dedallpreeqspquansepconst}%推論373
$A$, $B$, $R$を$\mathscr{T}$の関係式とし, $x$を$\mathscr{T}$の定数でない文字とする.
$A \leftrightarrow B$が$\mathscr{T}$の定理ならば, 
$\exists_{A}x(R) \leftrightarrow \exists_{B}x(R)$と
$\forall_{A}x(R) \leftrightarrow \forall_{B}x(R)$は
共に$\mathscr{T}$の定理である.
\end{dedu}




\mathstrut
\begin{dedu}
\label{dedquanspquanch}%推論374
$A$と$R$を$\mathscr{T}$の関係式とする.
また$x$と$y$を文字とし, $x$は$A$の中に自由変数として現れないとする.

1)
$\exists x(\exists_{A}y(R))$が$\mathscr{T}$の定理ならば, 
$\exists_{A}y(\exists x(R))$は$\mathscr{T}$の定理である.

2)
$\exists_{A}y(\exists x(R))$が$\mathscr{T}$の定理ならば, 
$\exists x(\exists_{A}y(R))$は$\mathscr{T}$の定理である.

3)
$\forall x(\forall_{A}y(R))$が$\mathscr{T}$の定理ならば, 
$\forall_{A}y(\forall x(R))$は$\mathscr{T}$の定理である.

4)
$\forall_{A}y(\forall x(R))$が$\mathscr{T}$の定理ならば, 
$\forall x(\forall_{A}y(R))$は$\mathscr{T}$の定理である.
\end{dedu}




\mathstrut
\begin{dedu}
\label{dedexspallch}%推論375
$A$と$R$を$\mathscr{T}$の関係式とする.
また$x$と$y$を文字とし, $x$は$A$の中に自由変数として現れないとする.
$\exists x(\forall_{A}y(R))$が$\mathscr{T}$の定理ならば, 
$\forall_{A}y(\exists x(R))$は$\mathscr{T}$の定理である.
\end{dedu}




\mathstrut
\begin{dedu}
\label{dedspexallch}%推論376
$A$と$R$を$\mathscr{T}$の関係式とする.
また$x$と$y$を文字とし, $y$は$A$の中に自由変数として現れないとする.
$\exists_{A}x(\forall y(R))$が$\mathscr{T}$の定理ならば, 
$\forall y(\exists_{A}x(R))$は$\mathscr{T}$の定理である.
\end{dedu}




\mathstrut
\begin{dedu}
\label{dedspexxeqexx}%推論377
$A$, $B$, $R$を$\mathscr{T}$の関係式とし, $x$を文字, $y$を$A$の中に自由変数として現れない文字とする.

1)
$\exists_{A}x(\exists_{B}y(R))$が$\mathscr{T}$の定理ならば, 
$\exists x(\exists y(A \wedge B \wedge R))$は$\mathscr{T}$の定理である.

2)
$\exists x(\exists y(A \wedge B \wedge R))$が$\mathscr{T}$の定理ならば, 
$\exists_{A}x(\exists_{B}y(R))$は$\mathscr{T}$の定理である.
\end{dedu}




\mathstrut
\begin{dedu}
\label{dedspallleqalll}%推論378
$A$, $B$, $R$を$\mathscr{T}$の関係式とし, $x$を文字, $y$を$A$の中に自由変数として現れない文字とする.

1)
$\forall_{A}x(\forall_{B}y(R))$が$\mathscr{T}$の定理ならば, 
$\forall x(\forall y(A \wedge B \to R))$は$\mathscr{T}$の定理である.

2)
$\forall x(\forall y(A \wedge B \to R))$が$\mathscr{T}$の定理ならば, 
$\forall_{A}x(\forall_{B}y(R))$は$\mathscr{T}$の定理である.
\end{dedu}




\mathstrut
\begin{dedu}
\label{dedspallleqalllconst}%推論379
$A$, $B$, $R$を$\mathscr{T}$の関係式とする.
また$x$と$y$を共に$\mathscr{T}$の定数でない文字とし, 
$y$は$A$の中に自由変数として現れないとする.
$A \wedge B \to R$が$\mathscr{T}$の定理ならば, 
$\forall_{A}x(\forall_{B}y(R))$は$\mathscr{T}$の定理である.
\end{dedu}




\mathstrut
\begin{dedu}
\label{dedspquanch}%推論380
$A$, $B$, $R$を$\mathscr{T}$の関係式とし, $x$を$B$の中に自由変数として現れない文字, 
$y$を$A$の中に自由変数として現れない文字とする.

1)
$\exists_{A}x(\exists_{B}y(R))$が$\mathscr{T}$の定理ならば, 
$\exists_{B}y(\exists_{A}x(R))$は$\mathscr{T}$の定理である.

2)
$\forall_{A}x(\forall_{B}y(R))$が$\mathscr{T}$の定理ならば, 
$\forall_{B}y(\forall_{A}x(R))$は$\mathscr{T}$の定理である.

3)
$\exists_{A}x(\forall_{B}y(R))$が$\mathscr{T}$の定理ならば, 
$\forall_{B}y(\exists_{A}x(R))$は$\mathscr{T}$の定理である.
\end{dedu}




\mathstrut
\begin{dedu}
\label{dedboss1}%推論381
$R$を$\mathscr{T}$の関係式とする.
また$n$を自然数とし, $A_{1}, A_{2}, \cdots, A_{n}$を$\mathscr{T}$の関係式, 
$x_{1}, x_{2}, \cdots, x_{n}$を文字とする.
いま$i < n$なる任意の自然数$i$に対し, 
$x_{i + 1}$は$A_{1}, A_{2}, \cdots, A_{i}$の中に自由変数として現れないとする.

1)
$\exists_{A_{1}}x_{1}(\exists_{A_{2}}x_{2}( \cdots (\exists_{A_{n}}x_{n}(R)) \cdots ))$が
$\mathscr{T}$の定理ならば, 
$\exists x_{1}(\exists x_{2}( \cdots (\exists x_{n}(A_{1} \wedge A_{2} \wedge \cdots \wedge A_{n} \wedge R)) \cdots ))$は
$\mathscr{T}$の定理である.

2)
$\exists x_{1}(\exists x_{2}( \cdots (\exists x_{n}(A_{1} \wedge A_{2} \wedge \cdots \wedge A_{n} \wedge R)) \cdots ))$が
$\mathscr{T}$の定理ならば, 
$\exists_{A_{1}}x_{1}(\exists_{A_{2}}x_{2}( \cdots (\exists_{A_{n}}x_{n}(R)) \cdots ))$は
$\mathscr{T}$の定理である.
\end{dedu}




\mathstrut
\begin{dedu}
\label{dedboss2}%推論382
$R$を$\mathscr{T}$の関係式とする.
また$n$を自然数とし, $A_{1}, A_{2}, \cdots, A_{n}$を$\mathscr{T}$の関係式, 
$x_{1}, x_{2}, \cdots, x_{n}$を文字とする.
いま$i < n$なる任意の自然数$i$に対し, 
$x_{i + 1}$は$A_{1}, A_{2}, \cdots, A_{i}$の中に自由変数として現れないとする.

1)
$\forall_{A_{1}}x_{1}(\forall_{A_{2}}x_{2}( \cdots (\forall_{A_{n}}x_{n}(R)) \cdots ))$が
$\mathscr{T}$の定理ならば, 
$\forall x_{1}(\forall x_{2}( \cdots (\forall x_{n}(A_{1} \wedge A_{2} \wedge \cdots \wedge A_{n} \to R)) \cdots ))$は
$\mathscr{T}$の定理である.

2)
$\forall x_{1}(\forall x_{2}( \cdots (\forall x_{n}(A_{1} \wedge A_{2} \wedge \cdots \wedge A_{n} \to R)) \cdots ))$が
$\mathscr{T}$の定理ならば, 
$\forall_{A_{1}}x_{1}(\forall_{A_{2}}x_{2}( \cdots (\forall_{A_{n}}x_{n}(R)) \cdots ))$は
$\mathscr{T}$の定理である.
\end{dedu}




\mathstrut
\begin{dedu}
\label{dedboss2const}%推論383
$R$を$\mathscr{T}$の関係式とする.
また$n$を自然数とし, $A_{1}, A_{2}, \cdots, A_{n}$を$\mathscr{T}$の関係式, 
$x_{1}, x_{2}, \cdots, x_{n}$を$\mathscr{T}$の定数でない文字とする.
いま$i < n$なる任意の自然数$i$に対し, 
$x_{i + 1}$は$A_{1}, A_{2}, \cdots, A_{i}$の中に自由変数として現れないとする.
$A_{1} \wedge A_{2} \wedge \cdots \wedge A_{n} \to R$が$\mathscr{T}$の定理ならば, 
$\forall_{A_{1}}x_{1}(\forall_{A_{2}}x_{2}( \cdots (\forall_{A_{n}}x_{n}(R)) \cdots ))$は
$\mathscr{T}$の定理である.
\end{dedu}




\mathstrut
\begin{dedu}
\label{dedgspquanch}%推論384
$R$を$\mathscr{T}$の関係式とする.
また$n$を自然数, $A_{1}, A_{2}, \cdots, A_{n}$を$\mathscr{T}$の関係式, 
$x_{1}, x_{2}, \cdots, x_{n}$を文字とし, 
$n$以下の互いに異なる任意の自然数$i$, $j$に対して$x_{i}$は$A_{j}$の中に自由変数として現れないとする.
また自然数$1, 2, \cdots, n$の順序を任意に入れ替えたものを$i_{1}, i_{2}, \cdots, i_{n}$とする.

1)
$\exists_{A_{1}}x_{1}(\exists_{A_{2}}x_{2}( \cdots (\exists_{A_{n}}x_{n}(R)) \cdots ))$が
$\mathscr{T}$の定理ならば, 
$\exists_{A_{i_{1}}}x_{i_{1}}(\exists_{A_{i_{2}}}x_{i_{2}}( \cdots (\exists_{A_{i_{n}}}x_{i_{n}}(R)) \cdots ))$は
$\mathscr{T}$の定理である.

2)
$\forall_{A_{1}}x_{1}(\forall_{A_{2}}x_{2}( \cdots (\forall_{A_{n}}x_{n}(R)) \cdots ))$が
$\mathscr{T}$の定理ならば, 
$\forall_{A_{i_{1}}}x_{i_{1}}(\forall_{A_{i_{2}}}x_{i_{2}}( \cdots (\forall_{A_{i_{n}}}x_{i_{n}}(R)) \cdots ))$は
$\mathscr{T}$の定理である.
\end{dedu}




\mathstrut
\begin{dedu}
\label{deds5}%推論385
$R$を$\mathscr{T}$の関係式, $T$と$U$を$\mathscr{T}$の対象式とし, 
$x$を文字とする.
$T = U$が$\mathscr{T}$の定理ならば, 
$(T|x)(R) \to (U|x)(R)$は$\mathscr{T}$の定理である.
\end{dedu}




\mathstrut
\begin{dedu}
\label{deds6}%推論386
$R$と$S$を$\mathscr{T}$の関係式とし, $x$を文字とする.
$\forall x(R \leftrightarrow S)$が$\mathscr{T}$の定理ならば, 
$\tau_{x}(R) = \tau_{x}(S)$は$\mathscr{T}$の定理である.
\end{dedu}




\mathstrut
\begin{dedu}
\label{deds6const}%推論387
$R$と$S$を$\mathscr{T}$の関係式とし, $x$を$\mathscr{T}$の定数でない文字とする.
$R \leftrightarrow S$が$\mathscr{T}$の定理ならば, 
$\tau_{x}(R) = \tau_{x}(S)$は$\mathscr{T}$の定理である.
\end{dedu}




\mathstrut
\begin{dedu}
\label{deds5n=}%推論388
$R$を$\mathscr{T}$の関係式, $T$と$U$を$\mathscr{T}$の対象式とし, 
$x$を文字とする.

1)
$(T|x)(R) \wedge \neg (U|x)(R)$が$\mathscr{T}$の定理ならば, 
$T \neq U$は$\mathscr{T}$の定理である.

2)
$(T|x)(R)$と$\neg (U|x)(R)$が共に$\mathscr{T}$の定理ならば, 
$T \neq U$は$\mathscr{T}$の定理である.

3)
$\neg (T|x)(R) \wedge (U|x)(R)$が$\mathscr{T}$の定理ならば, 
$T \neq U$は$\mathscr{T}$の定理である.

4)
$\neg (T|x)(R)$と$(U|x)(R)$が共に$\mathscr{T}$の定理ならば, 
$T \neq U$は$\mathscr{T}$の定理である.
\end{dedu}




\mathstrut
\begin{dedu}
\label{ded=ch}%推論389
$T$と$U$を$\mathscr{T}$の対象式とする.
$T = U$が$\mathscr{T}$の定理ならば, $U = T$は$\mathscr{T}$の定理である.
\end{dedu}




\mathstrut
\begin{dedu}
\label{dedn=ch}%推論390
$T$と$U$を$\mathscr{T}$の対象式とする.
$T \neq U$が$\mathscr{T}$の定理ならば, $U \neq T$は$\mathscr{T}$の定理である.
\end{dedu}




\mathstrut
\begin{dedu}
\label{deds5eq}%推論391
$R$を$\mathscr{T}$の関係式, $T$と$U$を$\mathscr{T}$の対象式とし, 
$x$を文字とする.

1)
$T = U$が$\mathscr{T}$の定理ならば, 
$(T|x)(R) \leftrightarrow (U|x)(R)$は$\mathscr{T}$の定理である.

2)
$T = U$が$\mathscr{T}$の定理であるとき, 
$(T|x)(R)$が$\mathscr{T}$の定理ならば$(U|x)(R)$は$\mathscr{T}$の定理であり, 
$(U|x)(R)$が$\mathscr{T}$の定理ならば$(T|x)(R)$は$\mathscr{T}$の定理である.
\end{dedu}




\mathstrut
\begin{dedu}
\label{dedgs5}%推論392
$R$を$\mathscr{T}$の関係式とする.
また$n$を自然数とし, $T_{1}, T_{2}, \cdots, T_{n}, U_{1}, U_{2}, \cdots, U_{n}$を
$\mathscr{T}$の対象式とする.
また$x_{1}, x_{2}, \cdots, x_{n}$を, どの二つも互いに異なる文字とする.

1)
$T_{1} = U_{1}, T_{2} = U_{2}, \cdots, T_{n} = U_{n}$がすべて$\mathscr{T}$の定理ならば, 
\[
  (T_{1}|x_{1}, T_{2}|x_{2}, \cdots, T_{n}|x_{n})(R) 
  \leftrightarrow (U_{1}|x_{1}, U_{2}|x_{2}, \cdots, U_{n}|x_{n})(R)
\]
は$\mathscr{T}$の定理である.

2)
$T_{1} = U_{1}, T_{2} = U_{2}, \cdots, T_{n} = U_{n}$がすべて$\mathscr{T}$の定理であるとき, 
$(T_{1}|x_{1}, T_{2}|x_{2}, \cdots, T_{n}|x_{n})(R)$が$\mathscr{T}$の定理ならば
$(U_{1}|x_{1}, U_{2}|x_{2}, \cdots, U_{n}|x_{n})(R)$は$\mathscr{T}$の定理であり, 
$(U_{1}|x_{1}, U_{2}|x_{2}, \cdots, U_{n}|x_{n})(R)$が$\mathscr{T}$の定理ならば
$(T_{1}|x_{1}, T_{2}|x_{2}, \cdots, T_{n}|x_{n})(R)$は$\mathscr{T}$の定理である.
\end{dedu}




\mathstrut
\begin{dedu}
\label{dedgs52}%推論393
$R$を$\mathscr{T}$の関係式とする.
また$n$を自然数とし, $T_{1}, T_{2}, \cdots, T_{n}, U_{1}, U_{2}, \cdots, U_{n}$を
$\mathscr{T}$の対象式とする.
また$x_{1}, x_{2}, \cdots, x_{n}$を, どの二つも互いに異なる文字とする.
いま$k$を自然数, $i_{1}, i_{2}, \cdots, i_{k}$を$n$以下の自然数とし, 
$i_{1}, i_{2}, \cdots, i_{k}$のいずれとも異なるような$n$以下の任意の自然数$i$に対して
$T_{i}$と$U_{i}$が同じ記号列であるとする.

1)
$T_{i_{1}} = U_{i_{1}}, T_{i_{2}} = U_{i_{2}}, \cdots, T_{i_{k}} = U_{i_{k}}$が
すべて$\mathscr{T}$の定理ならば, 
\[
  (T_{1}|x_{1}, T_{2}|x_{2}, \cdots, T_{n}|x_{n})(R) 
  \leftrightarrow (U_{1}|x_{1}, U_{2}|x_{2}, \cdots, U_{n}|x_{n})(R)
\]
は$\mathscr{T}$の定理である.

2)
$T_{i_{1}} = U_{i_{1}}, T_{i_{2}} = U_{i_{2}}, \cdots, T_{i_{k}} = U_{i_{k}}$が
すべて$\mathscr{T}$の定理であるとき, 
$(T_{1}|x_{1}, T_{2}|x_{2}, \cdots, T_{n}|x_{n})(R)$が$\mathscr{T}$の定理ならば
$(U_{1}|x_{1}, U_{2}|x_{2}, \cdots, U_{n}|x_{n})(R)$は$\mathscr{T}$の定理であり, 
$(U_{1}|x_{1}, U_{2}|x_{2}, \cdots, U_{n}|x_{n})(R)$が$\mathscr{T}$の定理ならば
$(T_{1}|x_{1}, T_{2}|x_{2}, \cdots, T_{n}|x_{n})(R)$は$\mathscr{T}$の定理である.
\end{dedu}




\mathstrut
\begin{dedu}
\label{ded=trans}%推論394
$T$, $U$, $V$を$\mathscr{T}$の対象式とする.
$T = U$と$U = V$が共に$\mathscr{T}$の定理ならば, 
$T = V$は$\mathscr{T}$の定理である.
\end{dedu}




\mathstrut
\begin{dedu}
\label{dedaddeq=}%推論395
\mbox{}

1)
$T$, $U$, $V$を$\mathscr{T}$の対象式とする.
$T = U$が$\mathscr{T}$の定理ならば, 
$T = V \leftrightarrow U = V$と$V = T \leftrightarrow V = U$は
共に$\mathscr{T}$の定理である.

2)
$T$, $U$, $V$, $W$を$\mathscr{T}$の対象式とする.
$T = U$と$V = W$が共に$\mathscr{T}$の定理ならば, 
$T = V \leftrightarrow U = W$は$\mathscr{T}$の定理である.
\end{dedu}




\mathstrut
\begin{dedu}
\label{ded=subst}%推論396
$T$, $U$, $V$を$\mathscr{T}$の対象式とし, $x$を文字とする.
$T = U$が$\mathscr{T}$の定理ならば, 
$(T|x)(V) = (U|x)(V)$は$\mathscr{T}$の定理である.
\end{dedu}




\mathstrut
\begin{dedu}
\label{ded=gsubst}%推論397
$n$を自然数とし, $T_{1}, T_{2}, \cdots, T_{n}, U_{1}, U_{2}, \cdots, U_{n}$を
$\mathscr{T}$の対象式とする.
また$x_{1}, x_{2}, \cdots, x_{n}$を, どの二つも互いに異なる文字とする.
また$V$を$\mathscr{T}$の対象式とする.
$T_{1} = U_{1}, T_{2} = U_{2}, \cdots, T_{n} = U_{n}$がすべて$\mathscr{T}$の定理ならば, 
$(T_{1}|x_{1}, T_{2}|x_{2}, \cdots, T_{n}|x_{n})(V) = (U_{1}|x_{1}, U_{2}|x_{2}, \cdots, U_{n}|x_{n})(V)$は
$\mathscr{T}$の定理である.
\end{dedu}




\mathstrut
\begin{dedu}
\label{ded=gsubst2}%推論398
$n$を自然数とし, $T_{1}, T_{2}, \cdots, T_{n}, U_{1}, U_{2}, \cdots, U_{n}$を
$\mathscr{T}$の対象式とする.
また$x_{1}, x_{2}, \cdots, x_{n}$を, どの二つも互いに異なる文字とする.
また$V$を$\mathscr{T}$の対象式とする.
いま$k$を自然数, $i_{1}, i_{2}, \cdots, i_{k}$を$n$以下の自然数とし, 
$i_{1}, i_{2}, \cdots, i_{k}$のいずれとも異なるような$n$以下の任意の自然数$i$に対して
$T_{i}$と$U_{i}$が同じ記号列であるとする.
このとき$T_{i_{1}} = U_{i_{1}}, T_{i_{2}} = U_{i_{2}}, \cdots, T_{i_{k}} = U_{i_{k}}$が
すべて$\mathscr{T}$の定理ならば, 
$(T_{1}|x_{1}, T_{2}|x_{2}, \cdots, T_{n}|x_{n})(V) = (U_{1}|x_{1}, U_{2}|x_{2}, \cdots, U_{n}|x_{n})(V)$は
$\mathscr{T}$の定理である.
\end{dedu}




\mathstrut
\begin{dedu}
\label{dedspquan=}%推論399
$R$を$\mathscr{T}$の関係式, $T$を$\mathscr{T}$の対象式とし, 
$x$を$T$の中に自由変数として現れない文字とする.

1)
$(T|x)(R)$が$\mathscr{T}$の定理ならば, 
$\exists_{x = T}x(R)$と$\forall_{x = T}x(R)$は共に$\mathscr{T}$の定理である.

2)
$\exists_{x = T}x(R)$が$\mathscr{T}$の定理ならば, 
$(T|x)(R)$は$\mathscr{T}$の定理である.

3)
$\forall_{x = T}x(R)$が$\mathscr{T}$の定理ならば, 
$(T|x)(R)$は$\mathscr{T}$の定理である.
\end{dedu}




\mathstrut
\begin{dedu}
\label{ded!const}%推論400
$R$を$\mathscr{T}$の関係式とする.

1)
$x$と$y$を, 互いに異なり, 共に$\mathscr{T}$の定数でない文字とする.
また$y$は$R$の中に自由変数として現れないとする.
$R \wedge (y|x)(R) \to x = y$が$\mathscr{T}$の定理ならば, 
$!x(R)$は$\mathscr{T}$の定理である.

2)
$x$を文字とする.
また$z$と$w$を, 互いに異なり, 共に$x$と異なり, $R$の中に自由変数として現れない, 
$\mathscr{T}$の定数でない文字とする.
$(z|x)(R) \wedge (w|x)(R) \to z = w$が$\mathscr{T}$の定理ならば, 
$!x(R)$は$\mathscr{T}$の定理である.
\end{dedu}




\mathstrut
\begin{dedu}
\label{ded!fund}%推論401
$R$を$\mathscr{T}$の関係式, $T$と$U$を$\mathscr{T}$の対象式とし, 
$x$を文字とする.

1)
$!x(R)$が$\mathscr{T}$の定理ならば, 
$(T|x)(R) \wedge (U|x)(R) \to T = U$は$\mathscr{T}$の定理である.

2)
$!x(R)$, $(T|x)(R)$, $(U|x)(R)$がいずれも$\mathscr{T}$の定理ならば, 
$T = U$は$\mathscr{T}$の定理である.
\end{dedu}




\mathstrut
\begin{dedu}
\label{dedn!}%推論402
$R$を$\mathscr{T}$の関係式, $T$と$U$を$\mathscr{T}$の対象式とし, 
$x$を文字とする.
$(T|x)(R)$, $(U|x)(R)$, $T \neq U$がいずれも$\mathscr{T}$の定理ならば, 
$\neg !x(R)$は$\mathscr{T}$の定理である.
\end{dedu}




\mathstrut
\begin{dedu}
\label{ded!free}%推論403
$R$を$\mathscr{T}$の関係式とし, $x$を$R$の中に自由変数として現れない文字とする.
また$y$と$z$を, 互いに異なる文字とする.

1)
$!x(R)$が$\mathscr{T}$の定理ならば, 
$R \to \forall y(\forall z(y = z))$は$\mathscr{T}$の定理である.

2)
$R \to \forall y(\forall z(y = z))$が$\mathscr{T}$の定理ならば, 
$!x(R)$は$\mathscr{T}$の定理である.
\end{dedu}




\mathstrut
\begin{dedu}
\label{ded!free2}%推論404
$R$を$\mathscr{T}$の関係式とし, $x$を$R$の中に自由変数として現れない文字とする.
$\neg R$が$\mathscr{T}$の定理ならば, $!x(R)$は$\mathscr{T}$の定理である.
\end{dedu}




\mathstrut
\begin{dedu}
\label{ded!free3}%推論405
$R$を$\mathscr{T}$の関係式とし, $x$を$R$の中に自由変数として現れない文字とする.
また$y$と$z$を, 互いに異なる文字とする.

1)
$\exists y(\exists z(y \neq z))$が$\mathscr{T}$の定理ならば, 
$!x(R) \leftrightarrow \neg R$は$\mathscr{T}$の定理である.

2)
$\exists y(\exists z(y \neq z))$と$!x(R)$が共に$\mathscr{T}$の定理ならば, 
$\neg R$は$\mathscr{T}$の定理である.
\end{dedu}




\mathstrut
\begin{dedu}
\label{ded!free4}%推論406
$R$を$\mathscr{T}$の関係式とし, $x$を$R$の中に自由変数として現れない文字とする.
いま$\mathscr{T}$の或る対象式$T$と$U$に対し, $T \neq U$が$\mathscr{T}$の定理であるとする.
このとき次の1), 2)が成り立つ.

1)
$!x(R) \leftrightarrow \neg R$は$\mathscr{T}$の定理である.

2)
$!x(R)$が$\mathscr{T}$の定理ならば, $\neg R$は$\mathscr{T}$の定理である.
\end{dedu}




\mathstrut
\begin{dedu}
\label{ded!tall}%推論407
$R$を$\mathscr{T}$の関係式とし, $x$を文字とする.
$!x(R)$が$\mathscr{T}$の定理ならば, 
$\forall x(R \to x = \tau_{x}(R))$は$\mathscr{T}$の定理である.
\end{dedu}




\mathstrut
\begin{dedu}
\label{ded!tTtau}%推論408
$R$を$\mathscr{T}$の関係式, $T$を$\mathscr{T}$の対象式とし, $x$を文字とする.

1)
$!x(R)$が$\mathscr{T}$の定理ならば, 
$(T|x)(R) \to T = \tau_{x}(R)$は$\mathscr{T}$の定理である.

2)
$!x(R)$と$(T|x)(R)$が共に$\mathscr{T}$の定理ならば, 
$T = \tau_{x}(R)$は$\mathscr{T}$の定理である.
\end{dedu}




\mathstrut
\begin{dedu}
\label{ded!thm}%推論409
$R$を$\mathscr{T}$の関係式, $T$を$\mathscr{T}$の対象式とし, 
$x$を$T$の中に自由変数として現れない文字とする.
$\forall x(R \to x = T)$が$\mathscr{T}$の定理ならば, 
$!x(R)$は$\mathscr{T}$の定理である.
\end{dedu}




\mathstrut
\begin{dedu}
\label{ded!thmconst}%推論410
$R$を$\mathscr{T}$の関係式, $T$を$\mathscr{T}$の対象式とし, 
$x$を$T$の中に自由変数として現れない, $\mathscr{T}$の定数でない文字とする.
$R \to x = T$が$\mathscr{T}$の定理ならば, 
$!x(R)$は$\mathscr{T}$の定理である.
\end{dedu}




\mathstrut
\begin{dedu}
\label{ded!equiv}%推論411
$R$を$\mathscr{T}$の関係式とし, $x$を文字とする.
また$y$を$x$と異なり, $R$の中に自由変数として現れない文字とする.

1)
$!x(R)$が$\mathscr{T}$の定理ならば, 
$\exists y(\forall x(R \to x = y))$は$\mathscr{T}$の定理である.

2)
$\exists y(\forall x(R \to x = y))$が$\mathscr{T}$の定理ならば, 
$!x(R)$は$\mathscr{T}$の定理である.
\end{dedu}




\mathstrut
\begin{dedu}
\label{dednquant!}%推論412
$R$を$\mathscr{T}$の関係式とし, $x$を文字とする.

1)
$\neg \exists x(R)$が$\mathscr{T}$の定理ならば, 
$!x(R)$は$\mathscr{T}$の定理である.

2)
$\forall x(\neg R)$が$\mathscr{T}$の定理ならば, 
$!x(R)$は$\mathscr{T}$の定理である.
\end{dedu}




\mathstrut
\begin{dedu}
\label{dednquant!const}%推論413
$R$を$\mathscr{T}$の関係式とし, $x$を$\mathscr{T}$の定数でない文字とする.
$\neg R$が$\mathscr{T}$の定理ならば, 
$!x(R)$は$\mathscr{T}$の定理である.
\end{dedu}




\mathstrut
\begin{dedu}
\label{ded!texn}%推論414
$R$を$\mathscr{T}$の関係式とし, $x$を文字とする.
いま$\mathscr{T}$の或る対象式$T$, $U$に対し, $T \neq U$が$\mathscr{T}$の定理であるとする.
このとき次の1), 2)が成り立つ.

1)
$!x(R) \to \exists x(\neg R)$は$\mathscr{T}$の定理である.

2)
$!x(R)$が$\mathscr{T}$の定理ならば, $\exists x(\neg R)$は$\mathscr{T}$の定理である.
\end{dedu}




\mathstrut
\begin{dedu}
\label{ded!ntex}%推論415
$R$を$\mathscr{T}$の関係式とし, $x$を文字とする.
いま$\mathscr{T}$の或る対象式$T$, $U$に対し, $T \neq U$が$\mathscr{T}$の定理であるとする.
このとき次の1), 2)が成り立つ.

1)
$!x(\neg R) \to \exists x(R)$は$\mathscr{T}$の定理である.

2)
$!x(\neg R)$が$\mathscr{T}$の定理ならば, $\exists x(R)$は$\mathscr{T}$の定理である.
\end{dedu}




\mathstrut
\begin{dedu}
\label{dedallt!sep}%推論416
$R$と$S$を$\mathscr{T}$の関係式とし, $x$を文字とする.
$\forall x(R \to S)$が$\mathscr{T}$の定理ならば, 
$!x(S) \to \ !x(R)$は$\mathscr{T}$の定理である.
\end{dedu}




\mathstrut
\begin{dedu}
\label{dedallt!sepconst}%推論417
$R$と$S$を$\mathscr{T}$の関係式とし, $x$を$\mathscr{T}$の定数でない文字とする.
$R \to S$が$\mathscr{T}$の定理ならば, 
$!x(S) \to \ !x(R)$は$\mathscr{T}$の定理である.
\end{dedu}




\mathstrut
\begin{dedu}
\label{ded!vee}%推論418
$R$と$S$を$\mathscr{T}$の関係式とし, $x$を文字とする.
$!x(R \vee S)$が$\mathscr{T}$の定理ならば, 
$!x(R)$と$!x(S)$は共に$\mathscr{T}$の定理である.
\end{dedu}




\mathstrut
\begin{dedu}
\label{ded!veew}%推論419
$R$と$S$を$\mathscr{T}$の関係式とし, $x$を文字とする.

1)
$!x(R \vee S)$が$\mathscr{T}$の定理ならば, 
$!x(R) \, \wedge \, !x(S)$は$\mathscr{T}$の定理である.

2)
$\forall x(R \to S) \, \wedge \, !x(S)$が$\mathscr{T}$の定理ならば, 
$!x(R \vee S)$は$\mathscr{T}$の定理である.

3)
$!x(R) \wedge \forall x(S \to R)$が$\mathscr{T}$の定理ならば, 
$!x(R \vee S)$は$\mathscr{T}$の定理である.
\end{dedu}




\mathstrut
\begin{dedu}
\label{ded!veew2}%推論420
$R$と$S$を$\mathscr{T}$の関係式とし, $x$を文字とする.

1)
$\forall x(R \to S)$と$!x(S)$が共に$\mathscr{T}$の定理ならば, 
$!x(R \vee S)$は$\mathscr{T}$の定理である.

2)
$!x(R)$と$\forall x(S \to R)$が共に$\mathscr{T}$の定理ならば, 
$!x(R \vee S)$は$\mathscr{T}$の定理である.
\end{dedu}




\mathstrut
\begin{dedu}
\label{ded!veew3}%推論421
$R$と$S$を$\mathscr{T}$の関係式とし, $x$を文字とする.

1)
$\neg \exists x(R) \, \wedge \, !x(S)$が$\mathscr{T}$の定理ならば, 
$!x(R \vee S)$は$\mathscr{T}$の定理である.

2)
$!x(R) \wedge \neg \exists x(S)$が$\mathscr{T}$の定理ならば, 
$!x(R \vee S)$は$\mathscr{T}$の定理である.

3)
$\forall x(\neg R) \, \wedge \, !x(S)$が$\mathscr{T}$の定理ならば, 
$!x(R \vee S)$は$\mathscr{T}$の定理である.

4)
$!x(R) \wedge \forall x(\neg S)$が$\mathscr{T}$の定理ならば, 
$!x(R \vee S)$は$\mathscr{T}$の定理である.
\end{dedu}




\mathstrut
\begin{dedu}
\label{ded!veew32}%推論422
$R$と$S$を$\mathscr{T}$の関係式とし, $x$を文字とする.

1)
$\neg \exists x(R)$と$!x(S)$が共に$\mathscr{T}$の定理ならば, 
$!x(R \vee S)$は$\mathscr{T}$の定理である.

2)
$!x(R)$と$\neg \exists x(S)$が共に$\mathscr{T}$の定理ならば, 
$!x(R \vee S)$は$\mathscr{T}$の定理である.

3)
$\forall x(\neg R)$と$!x(S)$が共に$\mathscr{T}$の定理ならば, 
$!x(R \vee S)$は$\mathscr{T}$の定理である.

4)
$!x(R)$と$\forall x(\neg S)$が共に$\mathscr{T}$の定理ならば, 
$!x(R \vee S)$は$\mathscr{T}$の定理である.
\end{dedu}




\mathstrut
\begin{dedu}
\label{ded!vfree}%推論423
$R$と$S$を$\mathscr{T}$の関係式とし, $x$を$R$の中に自由変数として現れない文字とする.

1)
$\neg R \, \wedge \, !x(S)$が$\mathscr{T}$の定理ならば, 
$!x(R \vee S)$は$\mathscr{T}$の定理である.

2)
$!x(S) \wedge \neg R$が$\mathscr{T}$の定理ならば, 
$!x(S \vee R)$は$\mathscr{T}$の定理である.
\end{dedu}




\mathstrut
\begin{dedu}
\label{ded!vfree2}%推論424
$R$と$S$を$\mathscr{T}$の関係式とし, $x$を$R$の中に自由変数として現れない文字とする.
$\neg R$と$!x(S)$が共に$\mathscr{T}$の定理ならば, 
$!x(R \vee S)$と$!x(S \vee R)$は共に$\mathscr{T}$の定理である.
\end{dedu}




\mathstrut
\begin{dedu}
\label{ded!vfreeeq}%推論425
$R$と$S$を$\mathscr{T}$の関係式とし, $x$を$R$の中に自由変数として現れない文字とする.
いま$\mathscr{T}$の或る対象式$T$, $U$に対し, $T \neq U$が$\mathscr{T}$の定理であるとする.
このとき
\[
  !x(R \vee S) \leftrightarrow \neg R \, \wedge \, !x(S), ~~
  !x(S \vee R) \leftrightarrow \ !x(S) \wedge \neg R
\]
は共に$\mathscr{T}$の定理である.
\end{dedu}




\mathstrut
\begin{dedu}
\label{ded!vfreeeq2}%推論426
$R$と$S$を$\mathscr{T}$の関係式とし, $x$を$R$の中に自由変数として現れない文字とする.
いま$\mathscr{T}$の或る対象式$T$, $U$に対し, $T \neq U$が$\mathscr{T}$の定理であるとする.
このとき次の1), 2)が成り立つ.

1)
$!x(R \vee S)$が$\mathscr{T}$の定理ならば, 
$\neg R \, \wedge \, !x(S)$は$\mathscr{T}$の定理である.

2)
$!x(S \vee R)$が$\mathscr{T}$の定理ならば, 
$!x(S) \wedge \neg R$は$\mathscr{T}$の定理である.
\end{dedu}




\mathstrut
\begin{dedu}
\label{ded!vfreeeq3}%推論427
$R$と$S$を$\mathscr{T}$の関係式とし, $x$を$R$の中に自由変数として現れない文字とする.
いま$\mathscr{T}$の或る対象式$T$, $U$に対し, $T \neq U$が$\mathscr{T}$の定理であるとする.
このとき次の1), 2)が成り立つ.

1)
$!x(R \vee S)$が$\mathscr{T}$の定理ならば, 
$\neg R$は$\mathscr{T}$の定理である.

2)
$!x(S \vee R)$が$\mathscr{T}$の定理ならば, 
$\neg R$は$\mathscr{T}$の定理である.
\end{dedu}




\mathstrut
\begin{dedu}
\label{ded!wedge}%推論428
$R$と$S$を$\mathscr{T}$の関係式とし, $x$を文字とする.

1)
$!x(R)$が$\mathscr{T}$の定理ならば, 
$!x(R \wedge S)$は$\mathscr{T}$の定理である.

2)
$!x(S)$が$\mathscr{T}$の定理ならば, 
$!x(R \wedge S)$は$\mathscr{T}$の定理である.
\end{dedu}




\mathstrut
\begin{dedu}
\label{ded!wedgev}%推論429
$R$と$S$を$\mathscr{T}$の関係式とし, $x$を文字とする.

1)
$!x(R) \, \vee \, !x(S)$が$\mathscr{T}$の定理ならば, 
$!x(R \wedge S)$は$\mathscr{T}$の定理である.

2)
$!x(R \wedge S)$が$\mathscr{T}$の定理ならば, 
$\exists x(\neg R \wedge S) \, \vee \, !x(S)$と
$!x(R) \vee \exists x(R \wedge \neg S)$は共に$\mathscr{T}$の定理である.
\end{dedu}




\mathstrut
\begin{dedu}
\label{ded!wedgev3}%推論430
$R$と$S$を$\mathscr{T}$の関係式とし, $x$を文字とする.
$!x(R \wedge S)$が$\mathscr{T}$の定理ならば, 
$\exists x(\neg R) \, \vee \, !x(S)$, $!x(R) \vee \exists x(\neg S)$, 
$\neg \forall x(R) \, \vee \, !x(S)$, $!x(R) \vee \neg \forall x(S)$は
いずれも$\mathscr{T}$の定理である.
\end{dedu}




\mathstrut
\begin{dedu}
\label{ded!wfree}%推論431
$R$と$S$を$\mathscr{T}$の関係式とし, $x$を$R$の中に自由変数として現れない文字とする.

1)
$!x(R \wedge S)$が$\mathscr{T}$の定理ならば, 
$R \to \ !x(S)$は$\mathscr{T}$の定理である.

2)
$!x(S \wedge R)$が$\mathscr{T}$の定理ならば, 
$R \to \ !x(S)$は$\mathscr{T}$の定理である.

3)
$R \to \ !x(S)$が$\mathscr{T}$の定理ならば, 
$!x(R \wedge S)$と$!x(S \wedge R)$は共に$\mathscr{T}$の定理である.
\end{dedu}




\mathstrut
\begin{dedu}
\label{ded!wfree2}%推論432
$R$と$S$を$\mathscr{T}$の関係式とし, $x$を$R$の中に自由変数として現れない文字とする.
$\neg R$が$\mathscr{T}$の定理ならば, 
$!x(R \wedge S)$と$!x(S \wedge R)$は共に$\mathscr{T}$の定理である.
\end{dedu}




\mathstrut
\begin{dedu}
\label{ded!gvee}%推論433
$x$を文字とする.
また$n$を自然数とし, $R_{1}, R_{2}, \cdots, R_{n}$を$\mathscr{T}$の関係式とする.
また$i$を$n$以下の自然数とする.
$!x(R_{1} \vee R_{2} \vee \cdots \vee R_{n})$が$\mathscr{T}$の定理ならば, 
$!x(R_{i})$は$\mathscr{T}$の定理である.
\end{dedu}




\mathstrut
\begin{dedu}
\label{ded!gvee2}%推論434
$x$を文字とする.
また$n$を自然数とし, $R_{1}, R_{2}, \cdots, R_{n}$を$\mathscr{T}$の関係式とする.
また$k$を自然数とし, $i_{1}, i_{2}, \cdots, i_{k}$を$n$以下の自然数とする.
$!x(R_{1} \vee R_{2} \vee \cdots \vee R_{n})$が$\mathscr{T}$の定理ならば, 
$!x(R_{i_{1}} \vee R_{i_{2}} \vee \cdots \vee R_{i_{k}})$は$\mathscr{T}$の定理である.
\end{dedu}




\mathstrut
\begin{dedu}
\label{ded!gv}%推論435
$x$を文字とする.
また$n$を自然数とし, $R_{1}, R_{2}, \cdots, R_{n}$を$\mathscr{T}$の関係式とする.

1)
$!x(R_{1} \vee R_{2} \vee \cdots \vee R_{n})$が$\mathscr{T}$の定理ならば, 
$!x(R_{1}) \, \wedge \, !x(R_{2}) \wedge \cdots \wedge \, !x(R_{n})$は$\mathscr{T}$の定理である.

2)
$i$を$n$以下の自然数とする.
\[
  \forall x(R_{1} \to R_{i}) \wedge \cdots 
  \wedge \forall x(R_{i - 1} \to R_{i}) \, \wedge \, !x(R_{i}) \wedge \forall x(R_{i + 1} \to R_{i}) \wedge \cdots 
  \wedge \forall x(R_{n} \to R_{i})
\]
が$\mathscr{T}$の定理ならば, 
$!x(R_{1} \vee R_{2} \vee \cdots \vee R_{n})$は$\mathscr{T}$の定理である.
\end{dedu}




\mathstrut
\begin{dedu}
\label{ded!gv2}%推論436
$x$を文字とする.
また$n$を自然数とし, $R_{1}, R_{2}, \cdots, R_{n}$を$\mathscr{T}$の関係式とする.
また$i$を$n$以下の自然数とする.
$\forall x(R_{1} \to R_{i}), \cdots, \forall x(R_{i - 1} \to R_{i}), !x(R_{i}), \forall x(R_{i + 1} \to R_{i}), \cdots, \forall x(R_{n} \to R_{i})$が
すべて$\mathscr{T}$の定理ならば, 
$!x(R_{1} \vee R_{2} \vee \cdots \vee R_{n})$は$\mathscr{T}$の定理である.
\end{dedu}




\mathstrut
\begin{dedu}
\label{ded!gv3}%推論437
$x$を文字とする.
また$n$を自然数とし, $R_{1}, R_{2}, \cdots, R_{n}$を$\mathscr{T}$の関係式とする.
また$i$を$n$以下の自然数とする.
$\neg \exists x(R_{1}) \wedge \cdots \wedge \neg \exists x(R_{i - 1}) \, \wedge \, !x(R_{i}) \wedge \neg \exists x(R_{i + 1}) \wedge \cdots \wedge \neg \exists x(R_{n})$が
$\mathscr{T}$の定理ならば, 
$!x(R_{1} \vee R_{2} \vee \cdots \vee R_{n})$は$\mathscr{T}$の定理である.
\end{dedu}




\mathstrut
\begin{dedu}
\label{ded!gv32}%推論438
$x$を文字とする.
また$n$を自然数とし, $R_{1}, R_{2}, \cdots, R_{n}$を$\mathscr{T}$の関係式とする.
また$i$を$n$以下の自然数とする.
$\neg \exists x(R_{1}), \cdots, \neg \exists x(R_{i - 1}), !x(R_{i}), \neg \exists x(R_{i + 1}), \cdots, \neg \exists x(R_{n})$が
すべて$\mathscr{T}$の定理ならば, 
$!x(R_{1} \vee R_{2} \vee \cdots \vee R_{n})$は$\mathscr{T}$の定理である.
\end{dedu}




\mathstrut
\begin{dedu}
\label{ded!gvfree}%推論439
$x$を文字とする.
また$n$を自然数とし, $R_{1}, R_{2}, \cdots, R_{n}$を$\mathscr{T}$の関係式とする.
また$i$を$n$以下の自然数とする.
いま$i$と異なる$n$以下の任意の自然数$j$に対し, $x$は$R_{j}$の中に自由変数として現れないとする.
このとき$\neg R_{1} \wedge \cdots \wedge \neg R_{i - 1} \, \wedge \, !x(R_{i}) \wedge \neg R_{i + 1} \wedge \cdots \wedge \neg R_{n}$が
$\mathscr{T}$の定理ならば, 
$!x(R_{1} \vee R_{2} \vee \cdots \vee R_{n})$は$\mathscr{T}$の定理である.
\end{dedu}




\mathstrut
\begin{dedu}
\label{ded!gvfree2}%推論440
$x$を文字とする.
また$n$を自然数とし, $R_{1}, R_{2}, \cdots, R_{n}$を$\mathscr{T}$の関係式とする.
また$i$を$n$以下の自然数とする.
いま$i$と異なる$n$以下の任意の自然数$j$に対し, $x$は$R_{j}$の中に自由変数として現れないとする.
このとき$\neg R_{1}, \cdots, \neg R_{i - 1}, !x(R_{i}), \neg R_{i + 1}, \cdots, \neg R_{n}$が
すべて$\mathscr{T}$の定理ならば, 
$!x(R_{1} \vee R_{2} \vee \cdots \vee R_{n})$は$\mathscr{T}$の定理である.
\end{dedu}




\mathstrut
\begin{dedu}
\label{ded!gvfreeeq}%推論441
$x$を文字とする.
また$n$を自然数とし, $R_{1}, R_{2}, \cdots, R_{n}$を$\mathscr{T}$の関係式とする.
また$i$を$n$以下の自然数とする.
いま$i$と異なる$n$以下の任意の自然数$j$に対し, $x$は$R_{j}$の中に自由変数として現れないとする.
またいま$\mathscr{T}$の或る対象式$T$, $U$に対し, $T \neq U$が$\mathscr{T}$の定理であるとする.
このとき
\[
  !x(R_{1} \vee R_{2} \vee \cdots \vee R_{n}) 
  \leftrightarrow \neg R_{1} \wedge \cdots \wedge \neg R_{i - 1} \, \wedge \, !x(R_{i}) \wedge \neg R_{i + 1} \wedge \cdots \wedge \neg R_{n}
\]
は$\mathscr{T}$の定理である.
\end{dedu}




\mathstrut
\begin{dedu}
\label{ded!gvfreeeq2}%推論442
$x$を文字とする.
また$n$を自然数とし, $R_{1}, R_{2}, \cdots, R_{n}$を$\mathscr{T}$の関係式とする.
また$i$を$n$以下の自然数とする.
いま$i$と異なる$n$以下の任意の自然数$j$に対し, $x$は$R_{j}$の中に自由変数として現れないとする.
またいま$\mathscr{T}$の或る対象式$T$, $U$に対し, $T \neq U$が$\mathscr{T}$の定理であるとする.
このとき$!x(R_{1} \vee R_{2} \vee \cdots \vee R_{n})$が$\mathscr{T}$の定理ならば, 
$\neg R_{1} \wedge \cdots \wedge \neg R_{i - 1} \, \wedge \, !x(R_{i}) \wedge \neg R_{i + 1} \wedge \cdots \wedge \neg R_{n}$は
$\mathscr{T}$の定理である.
\end{dedu}




\mathstrut
\begin{dedu}
\label{ded!gvfreeeq3}%推論443
$n$を自然数とし, $R_{1}, R_{2}, \cdots, R_{n}$を$\mathscr{T}$の関係式とする.
また$i$を$n$以下の自然数とし, $x$を$R_{i}$の中に自由変数として現れない文字とする.
いま$\mathscr{T}$の或る対象式$T$, $U$に対し, $T \neq U$が$\mathscr{T}$の定理であるとする.
このとき$!x(R_{1} \vee R_{2} \vee \cdots \vee R_{n})$が$\mathscr{T}$の定理ならば, 
$\neg R_{i}$は$\mathscr{T}$の定理である.
\end{dedu}




\mathstrut
\begin{dedu}
\label{ded!gwedge}%推論444
$x$を文字とする.
また$n$を自然数とし, $R_{1}, R_{2}, \cdots, R_{n}$を$\mathscr{T}$の関係式とする.
また$i$を$n$以下の自然数とする.
$!x(R_{i})$が$\mathscr{T}$の定理ならば, 
$!x(R_{1} \wedge R_{2} \wedge \cdots \wedge R_{n})$は$\mathscr{T}$の定理である.
\end{dedu}




\mathstrut
\begin{dedu}
\label{ded!gwedge2}%推論445
$x$を文字とする.
また$n$を自然数とし, $R_{1}, R_{2}, \cdots, R_{n}$を$\mathscr{T}$の関係式とする.
また$k$を自然数とし, $i_{1}, i_{2}, \cdots, i_{k}$を$n$以下の自然数とする.
$!x(R_{i_{1}} \wedge R_{i_{2}} \wedge \cdots \wedge R_{i_{k}})$が$\mathscr{T}$の定理ならば, 
$!x(R_{1} \wedge R_{2} \wedge \cdots \wedge R_{n})$は$\mathscr{T}$の定理である.
\end{dedu}




\mathstrut
\begin{dedu}
\label{ded!gw}%推論446
$x$を文字とする.
また$n$を自然数とし, $R_{1}, R_{2}, \cdots, R_{n}$を$\mathscr{T}$の関係式とする.

1)
$!x(R_{1}) \, \vee \, !x(R_{2}) \vee \cdots \vee \, !x(R_{n})$が$\mathscr{T}$の定理ならば, 
$!x(R_{1} \wedge R_{2} \wedge \cdots \wedge R_{n})$は$\mathscr{T}$の定理である.

2)
$i$を$n$以下の自然数とする.
$!x(R_{1} \wedge R_{2} \wedge \cdots \wedge R_{n})$が$\mathscr{T}$の定理ならば, 
\[
  \exists x(R_{i} \wedge \neg R_{1}) \vee \cdots 
  \vee \exists x(R_{i} \wedge \neg R_{i - 1}) \, \vee \, !x(R_{i}) \vee \exists x(R_{i} \wedge \neg R_{i + 1}) \vee \cdots 
  \vee \exists x(R_{i} \wedge \neg R_{n})
\]
は$\mathscr{T}$の定理である.
\end{dedu}




\mathstrut
\begin{dedu}
\label{ded!gw3}%推論447
$x$を文字とする.
また$n$を自然数とし, $R_{1}, R_{2}, \cdots, R_{n}$を$\mathscr{T}$の関係式とする.
また$i$を$n$以下の自然数とする.
$!x(R_{1} \wedge R_{2} \wedge \cdots \wedge R_{n})$が$\mathscr{T}$の定理ならば, 
$\exists x(\neg R_{1}) \vee \cdots \vee \exists x(\neg R_{i - 1}) \, \vee \, !x(R_{i}) \vee \exists x(\neg R_{i + 1}) \vee \cdots \vee \exists x(\neg R_{n})$は
$\mathscr{T}$の定理である.
\end{dedu}




\mathstrut
\begin{dedu}
\label{ded!gwfree}%推論448
$x$を文字とする.
また$n$を自然数とし, $R_{1}, R_{2}, \cdots, R_{n}$を$\mathscr{T}$の関係式とする.
また$i$を$n$以下の自然数とする.
いま$i$と異なる$n$以下の任意の自然数$j$に対し, $x$は$R_{j}$の中に自由変数として現れないとする.

1)
$!x(R_{1} \wedge R_{2} \wedge \cdots \wedge R_{n})$が$\mathscr{T}$の定理ならば, 
$\neg R_{1} \vee \cdots \vee \neg R_{i - 1} \, \vee \, !x(R_{i}) \vee \neg R_{i + 1} \vee \cdots \vee \neg R_{n}$は
$\mathscr{T}$の定理である.

2)
$\neg R_{1} \vee \cdots \vee \neg R_{i - 1} \, \vee \, !x(R_{i}) \vee \neg R_{i + 1} \vee \cdots \vee \neg R_{n}$が
$\mathscr{T}$の定理ならば, 
$!x(R_{1} \wedge R_{2} \wedge \cdots \wedge R_{n})$は$\mathscr{T}$の定理である.
\end{dedu}




\mathstrut
\begin{dedu}
\label{ded!gwfree2}%推論449
$n$を自然数とし, $R_{1}, R_{2}, \cdots, R_{n}$を$\mathscr{T}$の関係式とする.
また$i$を$n$以下の自然数とし, $x$を$R_{i}$の中に自由変数として現れない文字とする.
$\neg R_{i}$が$\mathscr{T}$の定理ならば, 
$!x(R_{1} \wedge R_{2} \wedge \cdots \wedge R_{n})$は$\mathscr{T}$の定理である.
\end{dedu}




\mathstrut
\begin{dedu}
\label{dedalleq!sep}%推論450
$R$と$S$を$\mathscr{T}$の関係式とし, $x$を文字とする.
$\forall x(R \leftrightarrow S)$が$\mathscr{T}$の定理ならば, 
$!x(R) \leftrightarrow \ !x(S)$は$\mathscr{T}$の定理である.
\end{dedu}




\mathstrut
\begin{dedu}
\label{dedalleq!sepconst}%推論451
$R$と$S$を$\mathscr{T}$の関係式とし, $x$を$\mathscr{T}$の定数でない文字とする.
$R \leftrightarrow S$が$\mathscr{T}$の定理ならば, 
$!x(R) \leftrightarrow \ !x(S)$は$\mathscr{T}$の定理である.
\end{dedu}




\mathstrut
\begin{dedu}
\label{ded!tspquansep}%推論452
$A$と$R$を$\mathscr{T}$の関係式とし, $x$を文字とする.
$!x(A)$が$\mathscr{T}$の定理ならば, 
$\exists_{A}x(R) \to \forall_{A}x(R)$は$\mathscr{T}$の定理である.
\end{dedu}




\mathstrut
\begin{dedu}
\label{ded!extall!}%推論453
$R$を$\mathscr{T}$の関係式とし, $x$と$y$を文字とする.
$!x(\exists y(R))$が$\mathscr{T}$の定理ならば, 
$\forall y(!x(R))$は$\mathscr{T}$の定理である.
\end{dedu}




\mathstrut
\begin{dedu}
\label{dedex!t!all}%推論454
$R$を$\mathscr{T}$の関係式とし, $x$と$y$を文字とする.
$\exists x(!y(R))$が$\mathscr{T}$の定理ならば, 
$!y(\forall x(R))$は$\mathscr{T}$の定理である.
\end{dedu}




\mathstrut
\begin{dedu}
\label{ded!quanch}%推論455
$R$を$\mathscr{T}$の関係式とし, $x$と$y$を文字とする.

1)
$!x(\exists y(R))$が$\mathscr{T}$の定理ならば, 
$\exists y(!x(R))$は$\mathscr{T}$の定理である.

2)
$\forall y(!x(R))$が$\mathscr{T}$の定理ならば, 
$!x(\forall y(R))$は$\mathscr{T}$の定理である.
\end{dedu}




\mathstrut
\begin{dedu}
\label{ded!spextspall!}%推論456
$A$と$R$を$\mathscr{T}$の関係式とする.
また$x$と$y$を文字とし, $x$は$A$の中に自由変数として現れないとする.
$!x(\exists_{A}y(R))$が$\mathscr{T}$の定理ならば, 
$\forall_{A}y(!x(R))$は$\mathscr{T}$の定理である.
\end{dedu}




\mathstrut
\begin{dedu}
\label{dedspex!t!spall}%推論457
$A$と$R$を$\mathscr{T}$の関係式とする.
また$x$と$y$を文字とし, $y$は$A$の中に自由変数として現れないとする.
$\exists_{A}x(!y(R))$が$\mathscr{T}$の定理ならば, 
$!y(\forall_{A}x(R))$は$\mathscr{T}$の定理である.
\end{dedu}




\mathstrut
\begin{dedu}
\label{ded!spquanch}%推論458
$A$と$R$を$\mathscr{T}$の関係式とする.
また$x$と$y$を文字とし, $x$は$A$の中に自由変数として現れないとする.

1)
$\exists y(A)$が$\mathscr{T}$の定理ならば, 
$!x(\exists_{A}y(R)) \to \exists_{A}y(!x(R))$と$\forall_{A}y(!x(R)) \to \ !x(\forall_{A}y(R))$は
共に$\mathscr{T}$の定理である.

2)
$!x(\exists_{A}y(R)) \to \exists_{A}y(!x(R))$が$\mathscr{T}$の定理ならば, 
$\exists y(A)$は$\mathscr{T}$の定理である.
\end{dedu}




\mathstrut
\begin{dedu}
\label{ded!spquanch2}%推論459
$A$と$R$を$\mathscr{T}$の関係式とする.
また$x$と$y$を文字とし, $x$は$A$の中に自由変数として現れないとする.

1)
$\exists y(A)$と$!x(\exists_{A}y(R))$が共に$\mathscr{T}$の定理ならば, 
$\exists_{A}y(!x(R))$は$\mathscr{T}$の定理である.

2)
$\exists y(A)$と$\forall_{A}y(!x(R))$が共に$\mathscr{T}$の定理ならば, 
$!x(\forall_{A}y(R))$は$\mathscr{T}$の定理である.
\end{dedu}




\mathstrut
\begin{dedu}
\label{ded!spallcheq}%推論460
$A$と$R$を$\mathscr{T}$の関係式とする.
また$x$と$y$を文字とし, $x$は$A$の中に自由変数として現れないとする.
いま$\mathscr{T}$の或る対象式$T$, $U$に対し, $T \neq U$が$\mathscr{T}$の定理であるとする.
このとき
\[
  \exists y(A) \leftrightarrow (\forall_{A}y(!x(R)) \to \ !x(\forall_{A}y(R)))
\]
は$\mathscr{T}$の定理である.
\end{dedu}




\mathstrut
\begin{dedu}
\label{ded!spallcheq2}%推論461
$A$と$R$を$\mathscr{T}$の関係式とする.
また$x$と$y$を文字とし, $x$は$A$の中に自由変数として現れないとする.
いま$\mathscr{T}$の或る対象式$T$, $U$に対し, $T \neq U$が$\mathscr{T}$の定理であるとする.
このとき$\forall_{A}y(!x(R)) \to \ !x(\forall_{A}y(R))$が$\mathscr{T}$の定理ならば, 
$\exists y(A)$は$\mathscr{T}$の定理である.
\end{dedu}




\mathstrut
\begin{dedu}
\label{ded!tsp!}%推論462
$A$と$R$を$\mathscr{T}$の関係式とし, $x$を文字とする.

1)
$!x(A)$が$\mathscr{T}$の定理ならば, 
$!_{A}x(R)$は$\mathscr{T}$の定理である.

2)
$!x(R)$が$\mathscr{T}$の定理ならば, 
$!_{A}x(R)$は$\mathscr{T}$の定理である.
\end{dedu}




\mathstrut
\begin{dedu}
\label{dednquantsp!}%推論463
$A$と$R$を$\mathscr{T}$の関係式とし, $x$を文字とする.

1)
$\neg \exists x(A)$が$\mathscr{T}$の定理ならば, 
$!_{A}x(R)$は$\mathscr{T}$の定理である.

2)
$\forall x(\neg A)$が$\mathscr{T}$の定理ならば, 
$!_{A}x(R)$は$\mathscr{T}$の定理である.

3)
$\neg \exists x(R)$が$\mathscr{T}$の定理ならば, 
$!_{A}x(R)$は$\mathscr{T}$の定理である.

4)
$\forall x(\neg R)$が$\mathscr{T}$の定理ならば, 
$!_{A}x(R)$は$\mathscr{T}$の定理である.
\end{dedu}




\mathstrut
\begin{dedu}
\label{dednquantsp!const}%推論464
$A$と$R$を$\mathscr{T}$の関係式とし, $x$を$\mathscr{T}$の定数でない文字とする.

1)
$\neg A$が$\mathscr{T}$の定理ならば, 
$!_{A}x(R)$は$\mathscr{T}$の定理である.

2)
$\neg R$が$\mathscr{T}$の定理ならば, 
$!_{A}x(R)$は$\mathscr{T}$の定理である.
\end{dedu}




\mathstrut
\begin{dedu}
\label{dedvtsp!}%推論465
$A$と$R$を$\mathscr{T}$の関係式とし, $x$を文字とする.

1)
$!x(A) \, \vee \, !x(R)$が$\mathscr{T}$の定理ならば, 
$!_{A}x(R)$は$\mathscr{T}$の定理である.

2)
$!_{A}x(R)$が$\mathscr{T}$の定理ならば, 
$\exists_{R}x(\neg A) \, \vee \, !x(R)$と
$!x(A) \vee \exists_{A}x(\neg R)$は共に$\mathscr{T}$の定理である.
\end{dedu}




\mathstrut
\begin{dedu}
\label{dedsp!tv2}%推論466
$A$と$R$を$\mathscr{T}$の関係式とし, $x$を文字とする.
$!_{A}x(R)$が$\mathscr{T}$の定理ならば, 
$\exists x(\neg A) \, \vee \, !x(R)$, $!x(A) \vee \exists x(\neg R)$, 
$\neg \forall x(A) \, \vee \, !x(R)$, $!x(A) \vee \neg \forall x(R)$は
いずれも$\mathscr{T}$の定理である.
\end{dedu}




\mathstrut
\begin{dedu}
\label{dedsp!eq!}%推論467
$A$と$R$を$\mathscr{T}$の関係式とし, $x$を文字とする.

1)
$\forall_{R}x(A)$が$\mathscr{T}$の定理ならば, 
$!_{A}x(R) \leftrightarrow \ !x(R)$は$\mathscr{T}$の定理である.

2)
$\forall_{R}x(A)$と$!_{A}x(R)$が共に$\mathscr{T}$の定理ならば, 
$!x(R)$は$\mathscr{T}$の定理である.

3)
$\forall_{A}x(R)$が$\mathscr{T}$の定理ならば, 
$!_{A}x(R) \leftrightarrow \ !x(A)$は$\mathscr{T}$の定理である.

4)
$\forall_{A}x(R)$と$!_{A}x(R)$が共に$\mathscr{T}$の定理ならば, 
$!x(A)$は$\mathscr{T}$の定理である.
\end{dedu}




\mathstrut
\begin{dedu}
\label{dedsp!eq!const}%推論468
$A$と$R$を$\mathscr{T}$の関係式とし, $x$を$\mathscr{T}$の定数でない文字とする.

1)
$R \to A$が$\mathscr{T}$の定理ならば, 
$!_{A}x(R) \leftrightarrow \ !x(R)$は$\mathscr{T}$の定理である.

2)
$R \to A$と$!_{A}x(R)$が共に$\mathscr{T}$の定理ならば, 
$!x(R)$は$\mathscr{T}$の定理である.

3)
$A \to R$が$\mathscr{T}$の定理ならば, 
$!_{A}x(R) \leftrightarrow \ !x(A)$は$\mathscr{T}$の定理である.

4)
$A \to R$と$!_{A}x(R)$が共に$\mathscr{T}$の定理ならば, 
$!x(A)$は$\mathscr{T}$の定理である.
\end{dedu}




\mathstrut
\begin{dedu}
\label{dedsp!eq!2}%推論469
$A$と$R$を$\mathscr{T}$の関係式とし, $x$を文字とする.

1)
$\forall x(A)$が$\mathscr{T}$の定理ならば, 
$!_{A}x(R) \leftrightarrow \ !x(R)$は$\mathscr{T}$の定理である.

2)
$\forall x(A)$と$!_{A}x(R)$が共に$\mathscr{T}$の定理ならば, 
$!x(R)$は$\mathscr{T}$の定理である.

3)
$\forall x(R)$が$\mathscr{T}$の定理ならば, 
$!_{A}x(R) \leftrightarrow \ !x(A)$は$\mathscr{T}$の定理である.

4)
$\forall x(R)$と$!_{A}x(R)$が共に$\mathscr{T}$の定理ならば, 
$!x(A)$は$\mathscr{T}$の定理である.
\end{dedu}




\mathstrut
\begin{dedu}
\label{dedsp!eq!2const}%推論470
$A$と$R$を$\mathscr{T}$の関係式とし, $x$を$\mathscr{T}$の定数でない文字とする.

1)
$A$が$\mathscr{T}$の定理ならば, 
$!_{A}x(R) \leftrightarrow \ !x(R)$は$\mathscr{T}$の定理である.

2)
$A$と$!_{A}x(R)$が共に$\mathscr{T}$の定理ならば, 
$!x(R)$は$\mathscr{T}$の定理である.

3)
$R$が$\mathscr{T}$の定理ならば, 
$!_{A}x(R) \leftrightarrow \ !x(A)$は$\mathscr{T}$の定理である.

4)
$R$と$!_{A}x(R)$が共に$\mathscr{T}$の定理ならば, 
$!x(A)$は$\mathscr{T}$の定理である.
\end{dedu}




\mathstrut
\begin{dedu}
\label{dedsp!afree}%推論471
$A$と$R$を$\mathscr{T}$の関係式とし, $x$を$A$の中に自由変数として現れない文字とする.

1)
$!_{A}x(R)$が$\mathscr{T}$の定理ならば, 
$A \to \ !x(R)$は$\mathscr{T}$の定理である.

2)
$A \to \ !x(R)$が$\mathscr{T}$の定理ならば, 
$!_{A}x(R)$は$\mathscr{T}$の定理である.
\end{dedu}




\mathstrut
\begin{dedu}
\label{dedsp!afree2}%推論472
$A$と$R$を$\mathscr{T}$の関係式とし, $x$を$A$の中に自由変数として現れない文字とする.
$\neg A$が$\mathscr{T}$の定理ならば, $!_{A}x(R)$は$\mathscr{T}$の定理である.
\end{dedu}




\mathstrut
\begin{dedu}
\label{dedsp!rfree}%推論473
$A$と$R$を$\mathscr{T}$の関係式とし, $x$を$R$の中に自由変数として現れない文字とする.

1)
$!_{A}x(R)$が$\mathscr{T}$の定理ならば, 
$!x(A) \vee \neg R$は$\mathscr{T}$の定理である.

2)
$!x(A) \vee \neg R$が$\mathscr{T}$の定理ならば, 
$!_{A}x(R)$は$\mathscr{T}$の定理である.
\end{dedu}




\mathstrut
\begin{dedu}
\label{dedsp!rfree2}%推論474
$A$と$R$を$\mathscr{T}$の関係式とし, $x$を$R$の中に自由変数として現れない文字とする.
$\neg R$が$\mathscr{T}$の定理ならば, $!_{A}x(R)$は$\mathscr{T}$の定理である.
\end{dedu}




\mathstrut
\begin{dedu}
\label{dedsp!rfree3}%推論475
$A$と$R$を$\mathscr{T}$の関係式とし, $x$を$R$の中に自由変数として現れない文字とする.

1)
$\neg !x(A)$が$\mathscr{T}$の定理ならば, 
$!_{A}x(R) \leftrightarrow \neg R$は$\mathscr{T}$の定理である.

2)
$\neg !x(A)$と$!_{A}x(R)$が共に$\mathscr{T}$の定理ならば, 
$\neg R$は$\mathscr{T}$の定理である.
\end{dedu}




\mathstrut
\begin{dedu}
\label{dedsp!const}%推論476
$A$と$R$を$\mathscr{T}$の関係式とする.

1)
$x$と$y$を, 互いに異なり, 共に$\mathscr{T}$の定数でない文字とする.
また$y$は$A$及び$R$の中に自由変数として現れないとする.
$A \wedge (y|x)(A) \wedge R \wedge (y|x)(R) \to x = y$が$\mathscr{T}$の定理ならば, 
$!_{A}x(R)$は$\mathscr{T}$の定理である.

2)
$x$を文字とする.
また$z$と$w$を, 互いに異なり, 共に$x$と異なり, $A$及び$R$の中に自由変数として現れない, 
$\mathscr{T}$の定数でない文字とする.
$(z|x)(A) \wedge (w|x)(A) \wedge (z|x)(R) \wedge (w|x)(R) \to z = w$が$\mathscr{T}$の定理ならば, 
$!_{A}x(R)$は$\mathscr{T}$の定理である.
\end{dedu}




\mathstrut
\begin{dedu}
\label{dedsp!fund}%推論477
$A$と$R$を$\mathscr{T}$の関係式, $T$と$U$を$\mathscr{T}$の対象式とし, 
$x$を文字とする.

1)
$!_{A}x(R)$が$\mathscr{T}$の定理ならば, 
$(T|x)(A) \wedge (U|x)(A) \wedge (T|x)(R) \wedge (U|x)(R) \to T = U$は$\mathscr{T}$の定理である.

2)
$!_{A}x(R)$, $(T|x)(A)$, $(U|x)(A)$, $(T|x)(R)$, $(U|x)(R)$がいずれも$\mathscr{T}$の定理ならば, 
$T = U$は$\mathscr{T}$の定理である.
\end{dedu}




\mathstrut
\begin{dedu}
\label{dednsp!}%推論478
$A$と$R$を$\mathscr{T}$の関係式, $T$と$U$を$\mathscr{T}$の対象式とし, 
$x$を文字とする.
$(T|x)(A)$, $(U|x)(A)$, $(T|x)(R)$, $(U|x)(R)$, $T \neq U$がいずれも$\mathscr{T}$の定理ならば, 
$\neg !_{A}x(R)$は$\mathscr{T}$の定理である.
\end{dedu}




\mathstrut
\begin{dedu}
\label{dedsp!lspall}%推論479
$A$と$R$を$\mathscr{T}$の関係式とし, $x$を文字とする.

1)
$!_{A}x(R)$が$\mathscr{T}$の定理ならば, 
$\forall_{A}x(R \to x = \tau_{x}(A \wedge R))$は$\mathscr{T}$の定理である.

2)
$\forall_{A}x(R \to x = \tau_{x}(A \wedge R))$が$\mathscr{T}$の定理ならば, 
$!_{A}x(R)$は$\mathscr{T}$の定理である.
\end{dedu}




\mathstrut
\begin{dedu}
\label{dedspalltsp!}%推論480
$A$と$R$を$\mathscr{T}$の関係式, $T$を$\mathscr{T}$の対象式とし, 
$x$を$T$の中に自由変数として現れない文字とする.
$\forall_{A}x(R \to x = T)$が$\mathscr{T}$の定理ならば, 
$!_{A}x(R)$は$\mathscr{T}$の定理である.
\end{dedu}




\mathstrut
\begin{dedu}
\label{dedsp!equiv}%推論481
$A$と$R$を$\mathscr{T}$の関係式とし, $x$を文字とする.
また$y$を$x$と異なり, $A$及び$R$の中に自由変数として現れない文字とする.

1)
$!_{A}x(R)$が$\mathscr{T}$の定理ならば, 
$\exists y(\forall_{A}x(R \to x = y))$は$\mathscr{T}$の定理である.

2)
$\exists y(\forall_{A}x(R \to x = y))$が$\mathscr{T}$の定理ならば, 
$!_{A}x(R)$は$\mathscr{T}$の定理である.
\end{dedu}




\mathstrut
\begin{dedu}
\label{dednspquantsp!}%推論482
$A$と$R$を$\mathscr{T}$の関係式とし, $x$を文字とする.

1)
$\neg \exists_{A}x(R)$が$\mathscr{T}$の定理ならば, 
$!_{A}x(R)$は$\mathscr{T}$の定理である.

2)
$\forall_{A}x(\neg R)$が$\mathscr{T}$の定理ならば, 
$!_{A}x(R)$は$\mathscr{T}$の定理である.
\end{dedu}




\mathstrut
\begin{dedu}
\label{dedsp!tspexn}%推論483
$A$と$R$を$\mathscr{T}$の関係式とし, $x$を文字とする.

1)
$\neg !x(A)$が$\mathscr{T}$の定理ならば, 
$!_{A}x(R) \to \exists_{A}x(\neg R)$は$\mathscr{T}$の定理である.

2)
$\neg !x(A)$と$!_{A}x(R)$が共に$\mathscr{T}$の定理ならば, 
$\exists_{A}x(\neg R)$は$\mathscr{T}$の定理である.
\end{dedu}




\mathstrut
\begin{dedu}
\label{dedsp!ntspex}%推論484
$A$と$R$を$\mathscr{T}$の関係式とし, $x$を文字とする.

1)
$\neg !x(A)$が$\mathscr{T}$の定理ならば, 
$!_{A}x(\neg R) \to \exists_{A}x(R)$は$\mathscr{T}$の定理である.

2)
$\neg !x(A)$と$!_{A}x(\neg R)$が共に$\mathscr{T}$の定理ならば, 
$\exists_{A}x(R)$は$\mathscr{T}$の定理である.
\end{dedu}




\mathstrut
\begin{dedu}
\label{dedspalltsp!sep}%推論485
$A$, $R$, $S$を$\mathscr{T}$の関係式とし, $x$を文字とする.
$\forall_{A}x(R \to S)$が$\mathscr{T}$の定理ならば, 
$!_{A}x(S) \to \ !_{A}x(R)$は$\mathscr{T}$の定理である.
\end{dedu}




\mathstrut
\begin{dedu}
\label{dedspalltsp!sepconst}%推論486
$A$, $R$, $S$を$\mathscr{T}$の関係式とし, $x$を$\mathscr{T}$の定数でない文字とする.
$A \to (R \to S)$が$\mathscr{T}$の定理ならば, 
$!_{A}x(S) \to \ !_{A}x(R)$は$\mathscr{T}$の定理である.
\end{dedu}




\mathstrut
\begin{dedu}
\label{dedalltsp!sep}%推論487
$A$, $R$, $S$を$\mathscr{T}$の関係式とし, $x$を文字とする.
$\forall x(R \to S)$が$\mathscr{T}$の定理ならば, 
$!_{A}x(S) \to \ !_{A}x(R)$は$\mathscr{T}$の定理である.
\end{dedu}




\mathstrut
\begin{dedu}
\label{dedalltsp!sepconst}%推論488
$A$, $R$, $S$を$\mathscr{T}$の関係式とし, $x$を$\mathscr{T}$の定数でない文字とする.
$R \to S$が$\mathscr{T}$の定理ならば, 
$!_{A}x(S) \to \ !_{A}x(R)$は$\mathscr{T}$の定理である.
\end{dedu}




\mathstrut
\begin{dedu}
\label{dedspallpretsp!sep}%推論489
$A$, $B$, $R$を$\mathscr{T}$の関係式とし, $x$を文字とする.
$\forall_{R}x(A \to B)$が$\mathscr{T}$の定理ならば, 
$!_{B}x(R) \to \ !_{A}x(R)$は$\mathscr{T}$の定理である.
\end{dedu}




\mathstrut
\begin{dedu}
\label{dedspallpretsp!sepconst}%推論490
$A$, $B$, $R$を$\mathscr{T}$の関係式とし, $x$を$\mathscr{T}$の定数でない文字とする.
$R \to (A \to B)$が$\mathscr{T}$の定理ならば, 
$!_{B}x(R) \to \ !_{A}x(R)$は$\mathscr{T}$の定理である.
\end{dedu}




\mathstrut
\begin{dedu}
\label{dedallpretsp!sep}%推論491
$A$, $B$, $R$を$\mathscr{T}$の関係式とし, $x$を文字とする.
$\forall x(A \to B)$が$\mathscr{T}$の定理ならば, 
$!_{B}x(R) \to \ !_{A}x(R)$は$\mathscr{T}$の定理である.
\end{dedu}




\mathstrut
\begin{dedu}
\label{dedallpretsp!sepconst}%推論492
$A$, $B$, $R$を$\mathscr{T}$の関係式とし, $x$を$\mathscr{T}$の定数でない文字とする.
$A \to B$が$\mathscr{T}$の定理ならば, 
$!_{B}x(R) \to \ !_{A}x(R)$は$\mathscr{T}$の定理である.
\end{dedu}




\mathstrut
\begin{dedu}
\label{dedsp!vee}%推論493
$A$, $R$, $S$を$\mathscr{T}$の関係式とし, $x$を文字とする.
$!_{A}x(R \vee S)$が$\mathscr{T}$の定理ならば, 
$!_{A}x(R)$と$!_{A}x(S)$は共に$\mathscr{T}$の定理である.
\end{dedu}




\mathstrut
\begin{dedu}
\label{dedsp!veew}%推論494
$A$, $R$, $S$を$\mathscr{T}$の関係式とし, $x$を文字とする.

1)
$!_{A}x(R \vee S)$が$\mathscr{T}$の定理ならば, 
$!_{A}x(R) \, \wedge \, !_{A}x(S)$は$\mathscr{T}$の定理である.

2)
$\forall_{A}x(R \to S) \, \wedge \, !_{A}x(S)$が$\mathscr{T}$の定理ならば, 
$!_{A}x(R \vee S)$は$\mathscr{T}$の定理である.

3)
$!_{A}x(R) \wedge \forall_{A}x(S \to R)$が$\mathscr{T}$の定理ならば, 
$!_{A}x(R \vee S)$は$\mathscr{T}$の定理である.
\end{dedu}




\mathstrut
\begin{dedu}
\label{dedsp!veew2}%推論495
$A$, $R$, $S$を$\mathscr{T}$の関係式とし, $x$を文字とする.

1)
$\forall_{A}x(R \to S)$と$!_{A}x(S)$が共に$\mathscr{T}$の定理ならば, 
$!_{A}x(R \vee S)$は$\mathscr{T}$の定理である.

2)
$!_{A}x(R)$と$\forall_{A}x(S \to R)$が共に$\mathscr{T}$の定理ならば, 
$!_{A}x(R \vee S)$は$\mathscr{T}$の定理である.
\end{dedu}




\mathstrut
\begin{dedu}
\label{dedsp!veew3}%推論496
$A$, $R$, $S$を$\mathscr{T}$の関係式とし, $x$を文字とする.

1)
$\neg \exists_{A}x(R) \, \wedge \, !_{A}x(S)$が$\mathscr{T}$の定理ならば, 
$!_{A}x(R \vee S)$は$\mathscr{T}$の定理である.

2)
$!_{A}x(R) \wedge \neg \exists_{A}x(S)$が$\mathscr{T}$の定理ならば, 
$!_{A}x(R \vee S)$は$\mathscr{T}$の定理である.

3)
$\forall_{A}x(\neg R) \, \wedge \, !_{A}x(S)$が$\mathscr{T}$の定理ならば, 
$!_{A}x(R \vee S)$は$\mathscr{T}$の定理である.

4)
$!_{A}x(R) \wedge \forall_{A}x(\neg S)$が$\mathscr{T}$の定理ならば, 
$!_{A}x(R \vee S)$は$\mathscr{T}$の定理である.
\end{dedu}




\mathstrut
\begin{dedu}
\label{dedsp!veew32}%推論497
$A$, $R$, $S$を$\mathscr{T}$の関係式とし, $x$を文字とする.

1)
$\neg \exists_{A}x(R)$と$!_{A}x(S)$が共に$\mathscr{T}$の定理ならば, 
$!_{A}x(R \vee S)$は$\mathscr{T}$の定理である.

2)
$!_{A}x(R)$と$\neg \exists_{A}x(S)$が共に$\mathscr{T}$の定理ならば, 
$!_{A}x(R \vee S)$は$\mathscr{T}$の定理である.

3)
$\forall_{A}x(\neg R)$と$!_{A}x(S)$が共に$\mathscr{T}$の定理ならば, 
$!_{A}x(R \vee S)$は$\mathscr{T}$の定理である.

4)
$!_{A}x(R)$と$\forall_{A}x(\neg S)$が共に$\mathscr{T}$の定理ならば, 
$!_{A}x(R \vee S)$は$\mathscr{T}$の定理である.
\end{dedu}




\mathstrut
\begin{dedu}
\label{dedsp!vfree}%推論498
$A$, $R$, $S$を$\mathscr{T}$の関係式とし, $x$を$R$の中に自由変数として現れない文字とする.

1)
$\neg R \, \wedge \, !_{A}x(S)$が$\mathscr{T}$の定理ならば, 
$!_{A}x(R \vee S)$は$\mathscr{T}$の定理である.

2)
$!_{A}x(S) \wedge \neg R$が$\mathscr{T}$の定理ならば, 
$!_{A}x(S \vee R)$は$\mathscr{T}$の定理である.
\end{dedu}




\mathstrut
\begin{dedu}
\label{dedsp!vfree2}%推論499
$A$, $R$, $S$を$\mathscr{T}$の関係式とし, $x$を$R$の中に自由変数として現れない文字とする.
$\neg R$と$!_{A}x(S)$が共に$\mathscr{T}$の定理ならば, 
$!_{A}x(R \vee S)$と$!_{A}x(S \vee R)$は共に$\mathscr{T}$の定理である.
\end{dedu}




\mathstrut
\begin{dedu}
\label{dedsp!vfreeeq}%推論500
$A$, $R$, $S$を$\mathscr{T}$の関係式とし, $x$を$R$の中に自由変数として現れない文字とする.

1)
$\neg !x(A)$が$\mathscr{T}$の定理ならば, 
\[
  !_{A}x(R \vee S) \leftrightarrow \neg R \, \wedge \, !_{A}x(S), ~~
  !_{A}x(S \vee R) \leftrightarrow \ !_{A}x(S) \wedge \neg R
\]
は共に$\mathscr{T}$の定理である.

2)
$\neg !x(A)$と$!_{A}x(R \vee S)$が共に$\mathscr{T}$の定理ならば, 
$\neg R \, \wedge \, !_{A}x(S)$は$\mathscr{T}$の定理である.

3)
$\neg !x(A)$と$!_{A}x(S \vee R)$が共に$\mathscr{T}$の定理ならば, 
$!_{A}x(S) \wedge \neg R$は$\mathscr{T}$の定理である.
\end{dedu}




\mathstrut
\begin{dedu}
\label{dedsp!vfreeeq2}%推論501
$A$, $R$, $S$を$\mathscr{T}$の関係式とし, $x$を$R$の中に自由変数として現れない文字とする.

1)
$\neg !x(A)$と$!_{A}x(R \vee S)$が共に$\mathscr{T}$の定理ならば, 
$\neg R$は$\mathscr{T}$の定理である.

2)
$\neg !x(A)$と$!_{A}x(S \vee R)$が共に$\mathscr{T}$の定理ならば, 
$\neg R$は$\mathscr{T}$の定理である.
\end{dedu}




\mathstrut
\begin{dedu}
\label{dedsp!wedge}%推論502
$A$, $R$, $S$を$\mathscr{T}$の関係式とし, $x$を文字とする.

1)
$!_{A}x(R)$が$\mathscr{T}$の定理ならば, 
$!_{A}x(R \wedge S)$は$\mathscr{T}$の定理である.

2)
$!_{A}x(S)$が$\mathscr{T}$の定理ならば, 
$!_{A}x(R \wedge S)$は$\mathscr{T}$の定理である.
\end{dedu}




\mathstrut
\begin{dedu}
\label{dedsp!wedgev}%推論503
$A$, $R$, $S$を$\mathscr{T}$の関係式とし, $x$を文字とする.

1)
$!_{A}x(R) \, \vee \, !_{A}x(S)$が$\mathscr{T}$の定理ならば, 
$!_{A}x(R \wedge S)$は$\mathscr{T}$の定理である.

2)
$!_{A}x(R \wedge S)$が$\mathscr{T}$の定理ならば, 
$\exists_{A}x(\neg R \wedge S) \, \vee \, !_{A}x(S)$と
$!_{A}x(R) \vee \exists_{A}x(R \wedge \neg S)$は共に$\mathscr{T}$の定理である.
\end{dedu}




\mathstrut
\begin{dedu}
\label{dedsp!wedgev3}%推論504
$A$, $R$, $S$を$\mathscr{T}$の関係式とし, $x$を文字とする.
$!_{A}x(R \wedge S)$が$\mathscr{T}$の定理ならば, 
$\exists_{A}x(\neg R) \, \vee \, !_{A}x(S)$, $!_{A}x(R) \vee \exists_{A}x(\neg S)$, 
$\neg \forall_{A}x(R) \, \vee \, !_{A}x(S)$, $!_{A}x(R) \vee \neg \forall_{A}x(S)$は
いずれも$\mathscr{T}$の定理である.
\end{dedu}




\mathstrut
\begin{dedu}
\label{dedsp!wfree}%推論505
$A$, $R$, $S$を$\mathscr{T}$の関係式とし, $x$を$R$の中に自由変数として現れない文字とする.

1)
$!_{A}x(R \wedge S)$が$\mathscr{T}$の定理ならば, 
$\neg R \, \vee \, !_{A}x(S)$は$\mathscr{T}$の定理である.

2)
$\neg R \, \vee \, !_{A}x(S)$が$\mathscr{T}$の定理ならば, 
$!_{A}x(R \wedge S)$は$\mathscr{T}$の定理である.

3)
$!_{A}x(S \wedge R)$が$\mathscr{T}$の定理ならば, 
$!_{A}x(S) \vee \neg R$は$\mathscr{T}$の定理である.

4)
$!_{A}x(S) \vee \neg R$が$\mathscr{T}$の定理ならば, 
$!_{A}x(S \wedge R)$は$\mathscr{T}$の定理である.
\end{dedu}




\mathstrut
\begin{dedu}
\label{dedsp!wfree2}%推論506
$A$, $R$, $S$を$\mathscr{T}$の関係式とし, $x$を$R$の中に自由変数として現れない文字とする.
$\neg R$が$\mathscr{T}$の定理ならば, 
$!_{A}x(R \wedge S)$と$!_{A}x(S \wedge R)$は共に$\mathscr{T}$の定理である.
\end{dedu}




\mathstrut
\begin{dedu}
\label{dedsp!prevee}%推論507
$A$, $B$, $R$を$\mathscr{T}$の関係式とし, $x$を文字とする.
$!_{A \vee B}x(R)$が$\mathscr{T}$の定理ならば, 
$!_{A}x(R)$と$!_{B}x(R)$は共に$\mathscr{T}$の定理である.
\end{dedu}




\mathstrut
\begin{dedu}
\label{dedsp!prev}%推論508
$A$, $B$, $R$を$\mathscr{T}$の関係式とし, $x$を文字とする.

1)
$!_{A \vee B}x(R)$が$\mathscr{T}$の定理ならば, 
$!_{A}x(R) \, \wedge \, !_{B}x(R)$は$\mathscr{T}$の定理である.

2)
$\forall_{R}x(A \to B) \, \wedge \, !_{B}x(R)$が$\mathscr{T}$の定理ならば, 
$!_{A \vee B}x(R)$は$\mathscr{T}$の定理である.

3)
$!_{A}x(R) \wedge \forall_{R}x(B \to A)$が$\mathscr{T}$の定理ならば, 
$!_{A \vee B}x(R)$は$\mathscr{T}$の定理である.
\end{dedu}




\mathstrut
\begin{dedu}
\label{dedsp!prev2}%推論509
$A$, $B$, $R$を$\mathscr{T}$の関係式とし, $x$を文字とする.

1)
$\forall_{R}x(A \to B)$と$!_{B}x(R)$が共に$\mathscr{T}$の定理ならば, 
$!_{A \vee B}x(R)$は$\mathscr{T}$の定理である.

2)
$!_{A}x(R)$と$\forall_{R}x(B \to A)$が共に$\mathscr{T}$の定理ならば, 
$!_{A \vee B}x(R)$は$\mathscr{T}$の定理である.
\end{dedu}




\mathstrut
\begin{dedu}
\label{dedsp!prev3}%推論510
$A$, $B$, $R$を$\mathscr{T}$の関係式とし, $x$を文字とする.

1)
$\neg \exists_{A}x(R) \, \wedge \, !_{B}x(R)$が$\mathscr{T}$の定理ならば, 
$!_{A \vee B}x(R)$は$\mathscr{T}$の定理である.

2)
$!_{A}x(R) \wedge \neg \exists_{B}x(R)$が$\mathscr{T}$の定理ならば, 
$!_{A \vee B}x(R)$は$\mathscr{T}$の定理である.

3)
$\forall_{A}x(\neg R) \, \wedge \, !_{B}x(R)$が$\mathscr{T}$の定理ならば, 
$!_{A \vee B}x(R)$は$\mathscr{T}$の定理である.

4)
$!_{A}x(R) \wedge \forall_{B}x(\neg R)$が$\mathscr{T}$の定理ならば, 
$!_{A \vee B}x(R)$は$\mathscr{T}$の定理である.
\end{dedu}




\mathstrut
\begin{dedu}
\label{dedsp!prev32}%推論511
$A$, $B$, $R$を$\mathscr{T}$の関係式とし, $x$を文字とする.

1)
$\neg \exists_{A}x(R)$と$!_{B}x(R)$が共に$\mathscr{T}$の定理ならば, 
$!_{A \vee B}x(R)$は$\mathscr{T}$の定理である.

2)
$!_{A}x(R)$と$\neg \exists_{B}x(R)$が共に$\mathscr{T}$の定理ならば, 
$!_{A \vee B}x(R)$は$\mathscr{T}$の定理である.

3)
$\forall_{A}x(\neg R)$と$!_{B}x(R)$が共に$\mathscr{T}$の定理ならば, 
$!_{A \vee B}x(R)$は$\mathscr{T}$の定理である.

4)
$!_{A}x(R)$と$\forall_{B}x(\neg R)$が共に$\mathscr{T}$の定理ならば, 
$!_{A \vee B}x(R)$は$\mathscr{T}$の定理である.
\end{dedu}




\mathstrut
\begin{dedu}
\label{dedsp!prevfree}%推論512
$A$, $B$, $R$を$\mathscr{T}$の関係式とし, $x$を$A$の中に自由変数として現れない文字とする.

1)
$\neg A \, \wedge \, !_{B}x(R)$が$\mathscr{T}$の定理ならば, 
$!_{A \vee B}x(R)$は$\mathscr{T}$の定理である.

2)
$!_{B}x(R) \wedge \neg A$が$\mathscr{T}$の定理ならば, 
$!_{B \vee A}x(R)$は$\mathscr{T}$の定理である.
\end{dedu}




\mathstrut
\begin{dedu}
\label{dedsp!prevfree2}%推論513
$A$, $B$, $R$を$\mathscr{T}$の関係式とし, $x$を$A$の中に自由変数として現れない文字とする.
$\neg A$と$!_{B}x(R)$が共に$\mathscr{T}$の定理ならば, 
$!_{A \vee B}x(R)$と$!_{B \vee A}x(R)$は共に$\mathscr{T}$の定理である.
\end{dedu}




\mathstrut
\begin{dedu}
\label{dedsp!prevfreeeq}%推論514
$A$, $B$, $R$を$\mathscr{T}$の関係式とし, $x$を$A$の中に自由変数として現れない文字とする.

1)
$\neg !x(R)$が$\mathscr{T}$の定理ならば, 
\[
  !_{A \vee B}x(R) \leftrightarrow \neg A \, \wedge \, !_{B}x(R), ~~
  !_{B \vee A}x(R) \leftrightarrow \ !_{B}x(R) \wedge \neg A
\]
は共に$\mathscr{T}$の定理である.

2)
$\neg !x(R)$と$!_{A \vee B}x(R)$が共に$\mathscr{T}$の定理ならば, 
$\neg A \, \wedge \, !_{B}x(R)$は$\mathscr{T}$の定理である.

3)
$\neg !x(R)$と$!_{B \vee A}x(R)$が共に$\mathscr{T}$の定理ならば, 
$!_{B}x(R) \wedge \neg A$は$\mathscr{T}$の定理である.
\end{dedu}




\mathstrut
\begin{dedu}
\label{dedsp!prevfreeeq2}%推論515
$A$, $B$, $R$を$\mathscr{T}$の関係式とし, $x$を$A$の中に自由変数として現れない文字とする.

1)
$\neg !x(R)$と$!_{A \vee B}x(R)$が共に$\mathscr{T}$の定理ならば, 
$\neg A$は$\mathscr{T}$の定理である.

2)
$\neg !x(R)$と$!_{B \vee A}x(R)$が共に$\mathscr{T}$の定理ならば, 
$\neg A$は$\mathscr{T}$の定理である.
\end{dedu}




\mathstrut
\begin{dedu}
\label{dedsp!prewedge}%推論516
$A$, $B$, $R$を$\mathscr{T}$の関係式とし, $x$を文字とする.

1)
$!_{A}x(R)$が$\mathscr{T}$の定理ならば, 
$!_{A \wedge B}x(R)$は$\mathscr{T}$の定理である.

2)
$!_{B}x(R)$が$\mathscr{T}$の定理ならば, 
$!_{A \wedge B}x(R)$は$\mathscr{T}$の定理である.
\end{dedu}




\mathstrut
\begin{dedu}
\label{dedsp!prew}%推論517
$A$, $B$, $R$を$\mathscr{T}$の関係式とし, $x$を文字とする.

1)
$!_{A}x(R) \, \vee \, !_{B}x(R)$が$\mathscr{T}$の定理ならば, 
$!_{A \wedge B}x(R)$は$\mathscr{T}$の定理である.

2)
$!_{A \wedge B}x(R)$が$\mathscr{T}$の定理ならば, 
$\exists_{R}x(\neg A \wedge B) \, \vee \, !_{B}x(R)$と
$!_{A}x(R) \vee \exists_{R}x(A \wedge \neg B)$は共に$\mathscr{T}$の定理である.
\end{dedu}




\mathstrut
\begin{dedu}
\label{dedsp!prew3}%推論518
$A$, $B$, $R$を$\mathscr{T}$の関係式とし, $x$を文字とする.
$!_{A \wedge B}x(R)$が$\mathscr{T}$の定理ならば, 
$\exists_{R}x(\neg A) \, \vee \, !_{B}x(R)$, $!_{A}x(R) \vee \exists_{R}x(\neg B)$, 
$\neg \forall_{R}x(A) \, \vee \, !_{B}x(R)$, $!_{A}x(R) \vee \neg \forall_{R}x(B)$は
いずれも$\mathscr{T}$の定理である.
\end{dedu}




\mathstrut
\begin{dedu}
\label{dedsp!prewfree}%推論519
$A$, $B$, $R$を$\mathscr{T}$の関係式とし, $x$を$A$の中に自由変数として現れない文字とする.

1)
$!_{A \wedge B}x(R)$が$\mathscr{T}$の定理ならば, 
$A \to \ !_{B}x(R)$は$\mathscr{T}$の定理である.

2)
$!_{B \wedge A}x(R)$が$\mathscr{T}$の定理ならば, 
$A \to \ !_{B}x(R)$は$\mathscr{T}$の定理である.

3)
$A \to \ !_{B}x(R)$が$\mathscr{T}$の定理ならば, 
$!_{A \wedge B}x(R)$と$!_{B \wedge A}x(R)$は共に$\mathscr{T}$の定理である.
\end{dedu}




\mathstrut
\begin{dedu}
\label{dedsp!prewfree2}%推論520
$A$, $B$, $R$を$\mathscr{T}$の関係式とし, $x$を$A$の中に自由変数として現れない文字とする.
$\neg A$が$\mathscr{T}$の定理ならば, 
$!_{A \wedge B}x(R)$と$!_{B \wedge A}x(R)$は共に$\mathscr{T}$の定理である.
\end{dedu}




\mathstrut
\begin{dedu}
\label{dedsp!gvee}%推論521
$A$を$\mathscr{T}$の関係式とし, $x$を文字とする.
また$n$を自然数とし, $R_{1}, R_{2}, \cdots, R_{n}$を$\mathscr{T}$の関係式とする.
また$i$を$n$以下の自然数とする.
$!_{A}x(R_{1} \vee R_{2} \vee \cdots \vee R_{n})$が$\mathscr{T}$の定理ならば, 
$!_{A}x(R_{i})$は$\mathscr{T}$の定理である.
\end{dedu}




\mathstrut
\begin{dedu}
\label{dedsp!gvee2}%推論522
$A$を$\mathscr{T}$の関係式とし, $x$を文字とする.
また$n$を自然数とし, $R_{1}, R_{2}, \cdots, R_{n}$を$\mathscr{T}$の関係式とする.
また$k$を自然数とし, $i_{1}, i_{2}, \cdots, i_{k}$を$n$以下の自然数とする.
$!_{A}x(R_{1} \vee R_{2} \vee \cdots \vee R_{n})$が$\mathscr{T}$の定理ならば, 
$!_{A}x(R_{i_{1}} \vee R_{i_{2}} \vee \cdots \vee R_{i_{k}})$は$\mathscr{T}$の定理である.
\end{dedu}




\mathstrut
\begin{dedu}
\label{dedsp!gv}%推論523
$A$を$\mathscr{T}$の関係式とし, $x$を文字とする.
また$n$を自然数とし, $R_{1}, R_{2}, \cdots, R_{n}$を$\mathscr{T}$の関係式とする.

1)
$!_{A}x(R_{1} \vee R_{2} \vee \cdots \vee R_{n})$が$\mathscr{T}$の定理ならば, 
$!_{A}x(R_{1}) \, \wedge \, !_{A}x(R_{2}) \wedge \cdots \wedge \, !_{A}x(R_{n})$は$\mathscr{T}$の定理である.

2)
$i$を$n$以下の自然数とする.
\[
  \forall_{A}x(R_{1} \to R_{i}) \wedge \cdots 
  \wedge \forall_{A}x(R_{i - 1} \to R_{i}) \, \wedge \, !_{A}x(R_{i}) \wedge \forall_{A}x(R_{i + 1} \to R_{i}) \wedge \cdots 
  \wedge \forall_{A}x(R_{n} \to R_{i})
\]
が$\mathscr{T}$の定理ならば, 
$!_{A}x(R_{1} \vee R_{2} \vee \cdots \vee R_{n})$は$\mathscr{T}$の定理である.
\end{dedu}




\mathstrut
\begin{dedu}
\label{dedsp!gv2}%推論524
$A$を$\mathscr{T}$の関係式とし, $x$を文字とする.
また$n$を自然数とし, $R_{1}, R_{2}, \cdots, R_{n}$を$\mathscr{T}$の関係式とする.
また$i$を$n$以下の自然数とする.
$\forall_{A}x(R_{1} \to R_{i}), \cdots, \forall_{A}x(R_{i - 1} \to R_{i}), !_{A}x(R_{i}), \forall_{A}x(R_{i + 1} \to R_{i}), \cdots, \forall_{A}x(R_{n} \to R_{i})$が
すべて$\mathscr{T}$の定理ならば, 
$!_{A}x(R_{1} \vee R_{2} \vee \cdots \vee R_{n})$は$\mathscr{T}$の定理である.
\end{dedu}




\mathstrut
\begin{dedu}
\label{dedsp!gv3}%推論525
$A$を$\mathscr{T}$の関係式とし, $x$を文字とする.
また$n$を自然数とし, $R_{1}, R_{2}, \cdots, R_{n}$を$\mathscr{T}$の関係式とする.
また$i$を$n$以下の自然数とする.
$\neg \exists_{A}x(R_{1}) \wedge \cdots \wedge \neg \exists_{A}x(R_{i - 1}) \, \wedge \, !_{A}x(R_{i}) \wedge \neg \exists_{A}x(R_{i + 1}) \wedge \cdots \wedge \neg \exists_{A}x(R_{n})$が
$\mathscr{T}$の定理ならば, 
$!_{A}x(R_{1} \vee R_{2} \vee \cdots \vee R_{n})$は$\mathscr{T}$の定理である.
\end{dedu}




\mathstrut
\begin{dedu}
\label{dedsp!gv32}%推論526
$A$を$\mathscr{T}$の関係式とし, $x$を文字とする.
また$n$を自然数とし, $R_{1}, R_{2}, \cdots, R_{n}$を$\mathscr{T}$の関係式とする.
また$i$を$n$以下の自然数とする.
$\neg \exists_{A}x(R_{1}), \cdots, \neg \exists_{A}x(R_{i - 1}), !_{A}x(R_{i}), \neg \exists_{A}x(R_{i + 1}), \cdots, \neg \exists_{A}x(R_{n})$が
すべて$\mathscr{T}$の定理ならば, 
$!_{A}x(R_{1} \vee R_{2} \vee \cdots \vee R_{n})$は$\mathscr{T}$の定理である.
\end{dedu}




\mathstrut
\begin{dedu}
\label{dedsp!gvfree}%推論527
$A$を$\mathscr{T}$の関係式とし, $x$を文字とする.
また$n$を自然数とし, $R_{1}, R_{2}, \cdots, R_{n}$を$\mathscr{T}$の関係式とする.
また$i$を$n$以下の自然数とする.
いま$i$と異なる$n$以下の任意の自然数$j$に対し, $x$は$R_{j}$の中に自由変数として現れないとする.
このとき$\neg R_{1} \wedge \cdots \wedge \neg R_{i - 1} \, \wedge \, !_{A}x(R_{i}) \wedge \neg R_{i + 1} \wedge \cdots \wedge \neg R_{n}$が
$\mathscr{T}$の定理ならば, 
$!_{A}x(R_{1} \vee R_{2} \vee \cdots \vee R_{n})$は$\mathscr{T}$の定理である.
\end{dedu}




\mathstrut
\begin{dedu}
\label{dedsp!gvfree2}%推論528
$A$を$\mathscr{T}$の関係式とし, $x$を文字とする.
また$n$を自然数とし, $R_{1}, R_{2}, \cdots, R_{n}$を$\mathscr{T}$の関係式とする.
また$i$を$n$以下の自然数とする.
いま$i$と異なる$n$以下の任意の自然数$j$に対し, $x$は$R_{j}$の中に自由変数として現れないとする.
このとき$\neg R_{1}, \cdots, \neg R_{i - 1}, !_{A}x(R_{i}), \neg R_{i + 1}, \cdots, \neg R_{n}$が
すべて$\mathscr{T}$の定理ならば, 
$!_{A}x(R_{1} \vee R_{2} \vee \cdots \vee R_{n})$は$\mathscr{T}$の定理である.
\end{dedu}




\mathstrut
\begin{dedu}
\label{dedsp!gvfreeeq}%推論529
$A$を$\mathscr{T}$の関係式とし, $x$を文字とする.
また$n$を自然数とし, $R_{1}, R_{2}, \cdots, R_{n}$を$\mathscr{T}$の関係式とする.
また$i$を$n$以下の自然数とする.
いま$i$と異なる$n$以下の任意の自然数$j$に対し, $x$は$R_{j}$の中に自由変数として現れないとする.

1)
$\neg !x(A)$が$\mathscr{T}$の定理ならば, 
\[
  !_{A}x(R_{1} \vee R_{2} \vee \cdots \vee R_{n}) 
  \leftrightarrow \neg R_{1} \wedge \cdots \wedge \neg R_{i - 1} \, \wedge \, !_{A}x(R_{i}) \wedge \neg R_{i + 1} \wedge \cdots \wedge \neg R_{n}
\]
は$\mathscr{T}$の定理である.

2)
$\neg !x(A)$と$!_{A}x(R_{1} \vee R_{2} \vee \cdots \vee R_{n})$が共に$\mathscr{T}$の定理ならば, 
\[
  \neg R_{1} \wedge \cdots \wedge \neg R_{i - 1} \, \wedge \, !_{A}x(R_{i}) \wedge \neg R_{i + 1} \wedge \cdots \wedge \neg R_{n}
\]
は$\mathscr{T}$の定理である.
\end{dedu}




\mathstrut
\begin{dedu}
\label{dedsp!gvfreeeq2}%推論530
$A$を$\mathscr{T}$の関係式とする.
また$n$を自然数とし, $R_{1}, R_{2}, \cdots, R_{n}$を$\mathscr{T}$の関係式とする.
また$i$を$n$以下の自然数とし, $x$を$R_{i}$の中に自由変数として現れない文字とする.
このとき$\neg !x(A)$と$!_{A}x(R_{1} \vee R_{2} \vee \cdots \vee R_{n})$が共に$\mathscr{T}$の定理ならば, 
$\neg R_{i}$は$\mathscr{T}$の定理である.
\end{dedu}




\mathstrut
\begin{dedu}
\label{dedsp!gwedge}%推論531
$A$を$\mathscr{T}$の関係式とし, $x$を文字とする.
また$n$を自然数とし, $R_{1}, R_{2}, \cdots, R_{n}$を$\mathscr{T}$の関係式とする.
また$i$を$n$以下の自然数とする.
$!_{A}x(R_{i})$が$\mathscr{T}$の定理ならば, 
$!_{A}x(R_{1} \wedge R_{2} \wedge \cdots \wedge R_{n})$は$\mathscr{T}$の定理である.
\end{dedu}




\mathstrut
\begin{dedu}
\label{dedsp!gwedge2}%推論532
$A$を$\mathscr{T}$の関係式とし, $x$を文字とする.
また$n$を自然数とし, $R_{1}, R_{2}, \cdots, R_{n}$を$\mathscr{T}$の関係式とする.
また$k$を自然数とし, $i_{1}, i_{2}, \cdots, i_{k}$を$n$以下の自然数とする.
$!_{A}x(R_{i_{1}} \wedge R_{i_{2}} \wedge \cdots \wedge R_{i_{k}})$が$\mathscr{T}$の定理ならば, 
$!_{A}x(R_{1} \wedge R_{2} \wedge \cdots \wedge R_{n})$は$\mathscr{T}$の定理である.
\end{dedu}




\mathstrut
\begin{dedu}
\label{dedsp!gw}%推論533
$A$を$\mathscr{T}$の関係式とし, $x$を文字とする.
また$n$を自然数とし, $R_{1}, R_{2}, \cdots, R_{n}$を$\mathscr{T}$の関係式とする.

1)
$!_{A}x(R_{1}) \, \vee \, !_{A}x(R_{2}) \vee \cdots \vee \, !_{A}x(R_{n})$が$\mathscr{T}$の定理ならば, 
$!_{A}x(R_{1} \wedge R_{2} \wedge \cdots \wedge R_{n})$は$\mathscr{T}$の定理である.

2)
$i$を$n$以下の自然数とする.
$!_{A}x(R_{1} \wedge R_{2} \wedge \cdots \wedge R_{n})$が$\mathscr{T}$の定理ならば, 
\[
  \exists_{A}x(R_{i} \wedge \neg R_{1}) \vee \cdots 
  \vee \exists_{A}x(R_{i} \wedge \neg R_{i - 1}) \, \vee \, !_{A}x(R_{i}) \vee \exists_{A}x(R_{i} \wedge \neg R_{i + 1}) \vee \cdots 
  \vee \exists_{A}x(R_{i} \wedge \neg R_{n})
\]
は$\mathscr{T}$の定理である.
\end{dedu}




\mathstrut
\begin{dedu}
\label{dedsp!gw3}%推論534
$A$を$\mathscr{T}$の関係式とし, $x$を文字とする.
また$n$を自然数とし, $R_{1}, R_{2}, \cdots, R_{n}$を$\mathscr{T}$の関係式とする.
また$i$を$n$以下の自然数とする.
$!_{A}x(R_{1} \wedge R_{2} \wedge \cdots \wedge R_{n})$が$\mathscr{T}$の定理ならば, 
$\exists_{A}x(\neg R_{1}) \vee \cdots \vee \exists_{A}x(\neg R_{i - 1}) \, \vee \, !_{A}x(R_{i}) \vee \exists_{A}x(\neg R_{i + 1}) \vee \cdots \vee \exists_{A}x(\neg R_{n})$は
$\mathscr{T}$の定理である.
\end{dedu}




\mathstrut
\begin{dedu}
\label{dedsp!gwfree}%推論535
$A$を$\mathscr{T}$の関係式とし, $x$を文字とする.
また$n$を自然数とし, $R_{1}, R_{2}, \cdots, R_{n}$を$\mathscr{T}$の関係式とする.
また$i$を$n$以下の自然数とする.
いま$i$と異なる$n$以下の任意の自然数$j$に対し, $x$は$R_{j}$の中に自由変数として現れないとする.

1)
$!_{A}x(R_{1} \wedge R_{2} \wedge \cdots \wedge R_{n})$が$\mathscr{T}$の定理ならば, 
$\neg R_{1} \vee \cdots \vee \neg R_{i - 1} \, \vee \, !_{A}x(R_{i}) \vee \neg R_{i + 1} \vee \cdots \vee \neg R_{n}$は
$\mathscr{T}$の定理である.

2)
$\neg R_{1} \vee \cdots \vee \neg R_{i - 1} \, \vee \, !_{A}x(R_{i}) \vee \neg R_{i + 1} \vee \cdots \vee \neg R_{n}$が
$\mathscr{T}$の定理ならば, 
$!_{A}x(R_{1} \wedge R_{2} \wedge \cdots \wedge R_{n})$は$\mathscr{T}$の定理である.
\end{dedu}




\mathstrut
\begin{dedu}
\label{dedsp!gwfree2}%推論536
$A$を$\mathscr{T}$の関係式とする.
また$n$を自然数とし, $R_{1}, R_{2}, \cdots, R_{n}$を$\mathscr{T}$の関係式とする.
また$i$を$n$以下の自然数とし, $x$を$R_{i}$の中に自由変数として現れない文字とする.
$\neg R_{i}$が$\mathscr{T}$の定理ならば, 
$!_{A}x(R_{1} \wedge R_{2} \wedge \cdots \wedge R_{n})$は$\mathscr{T}$の定理である.
\end{dedu}




\mathstrut
\begin{dedu}
\label{dedsp!pregvee}%推論537
$R$を$\mathscr{T}$の関係式とし, $x$を文字とする.
また$n$を自然数とし, $A_{1}, A_{2}, \cdots, A_{n}$を$\mathscr{T}$の関係式とする.
また$i$を$n$以下の自然数とする.
$!_{A_{1} \vee A_{2} \vee \cdots \vee A_{n}}x(R)$が$\mathscr{T}$の定理ならば, 
$!_{A_{i}}x(R)$は$\mathscr{T}$の定理である.
\end{dedu}




\mathstrut
\begin{dedu}
\label{dedsp!pregvee2}%推論538
$R$を$\mathscr{T}$の関係式とし, $x$を文字とする.
また$n$を自然数とし, $A_{1}, A_{2}, \cdots, A_{n}$を$\mathscr{T}$の関係式とする.
また$k$を自然数とし, $i_{1}, i_{2}, \cdots, i_{k}$を$n$以下の自然数とする.
$!_{A_{1} \vee A_{2} \vee \cdots \vee A_{n}}x(R)$が$\mathscr{T}$の定理ならば, 
$!_{A_{i_{1}} \vee A_{i_{2}} \vee \cdots \vee A_{i_{k}}}x(R)$は$\mathscr{T}$の定理である.
\end{dedu}




\mathstrut
\begin{dedu}
\label{dedsp!pregv}%推論539
$R$を$\mathscr{T}$の関係式とし, $x$を文字とする.
また$n$を自然数とし, $A_{1}, A_{2}, \cdots, A_{n}$を$\mathscr{T}$の関係式とする.

1)
$!_{A_{1} \vee A_{2} \vee \cdots \vee A_{n}}x(R)$が$\mathscr{T}$の定理ならば, 
$!_{A_{1}}x(R) \, \wedge \, !_{A_{2}}x(R) \wedge \cdots \wedge \, !_{A_{n}}x(R)$は$\mathscr{T}$の定理である.

2)
$i$を$n$以下の自然数とする.
\[
  \forall_{R}x(A_{1} \to A_{i}) \wedge \cdots 
  \wedge \forall_{R}x(A_{i - 1} \to A_{i}) \, \wedge \, !_{A_{i}}x(R) \wedge \forall_{R}x(A_{i + 1} \to A_{i}) \wedge \cdots 
  \wedge \forall_{R}x(A_{n} \to A_{i})
\]
が$\mathscr{T}$の定理ならば, 
$!_{A_{1} \vee A_{2} \vee \cdots \vee A_{n}}x(R)$は$\mathscr{T}$の定理である.
\end{dedu}




\mathstrut
\begin{dedu}
\label{dedsp!pregv2}%推論540
$R$を$\mathscr{T}$の関係式とし, $x$を文字とする.
また$n$を自然数とし, $A_{1}, A_{2}, \cdots, A_{n}$を$\mathscr{T}$の関係式とする.
また$i$を$n$以下の自然数とする.
$\forall_{R}x(A_{1} \to A_{i}), \cdots, \forall_{R}x(A_{i - 1} \to A_{i}), !_{A_{i}}x(R), \forall_{R}x(A_{i + 1} \to A_{i}), \cdots, \forall_{R}x(A_{n} \to A_{i})$が
すべて$\mathscr{T}$の定理ならば, 
$!_{A_{1} \vee A_{2} \vee \cdots \vee A_{n}}x(R)$は$\mathscr{T}$の定理である.
\end{dedu}




\mathstrut
\begin{dedu}
\label{dedsp!pregv3}%推論541
$R$を$\mathscr{T}$の関係式とし, $x$を文字とする.
また$n$を自然数とし, $A_{1}, A_{2}, \cdots, A_{n}$を$\mathscr{T}$の関係式とする.
また$i$を$n$以下の自然数とする.
\[
  \neg \exists_{A_{1}}x(R) \wedge \cdots 
  \wedge \neg \exists_{A_{i - 1}}x(R) \, \wedge \, !_{A_{i}}x(R) \wedge \neg \exists_{A_{i + 1}}x(R) \wedge \cdots 
  \wedge \neg \exists_{A_{n}}x(R)
\]
が$\mathscr{T}$の定理ならば, 
$!_{A_{1} \vee A_{2} \vee \cdots \vee A_{n}}x(R)$は$\mathscr{T}$の定理である.
\end{dedu}




\mathstrut
\begin{dedu}
\label{dedsp!pregv32}%推論542
$R$を$\mathscr{T}$の関係式とし, $x$を文字とする.
また$n$を自然数とし, $A_{1}, A_{2}, \cdots, A_{n}$を$\mathscr{T}$の関係式とする.
また$i$を$n$以下の自然数とする.
$\neg \exists_{A_{1}}x(R), \cdots, \neg \exists_{A_{i - 1}}x(R), !_{A_{i}}x(R), \neg \exists_{A_{i + 1}}x(R), \cdots, \neg \exists_{A_{n}}x(R)$が
すべて$\mathscr{T}$の定理ならば, 
$!_{A_{1} \vee A_{2} \vee \cdots \vee A_{n}}x(R)$は$\mathscr{T}$の定理である.
\end{dedu}




\mathstrut
\begin{dedu}
\label{dedsp!pregvfree}%推論543
$R$を$\mathscr{T}$の関係式とし, $x$を文字とする.
また$n$を自然数とし, $A_{1}, A_{2}, \cdots, A_{n}$を$\mathscr{T}$の関係式とする.
また$i$を$n$以下の自然数とする.
いま$i$と異なる$n$以下の任意の自然数$j$に対し, $x$は$A_{j}$の中に自由変数として現れないとする.
このとき$\neg A_{1} \wedge \cdots \wedge \neg A_{i - 1} \, \wedge \, !_{A_{i}}x(R) \wedge \neg A_{i + 1} \wedge \cdots \wedge \neg A_{n}$が
$\mathscr{T}$の定理ならば, 
$!_{A_{1} \vee A_{2} \vee \cdots \vee A_{n}}x(R)$は$\mathscr{T}$の定理である.
\end{dedu}




\mathstrut
\begin{dedu}
\label{dedsp!pregvfree2}%推論544
$R$を$\mathscr{T}$の関係式とし, $x$を文字とする.
また$n$を自然数とし, $A_{1}, A_{2}, \cdots, A_{n}$を$\mathscr{T}$の関係式とする.
また$i$を$n$以下の自然数とする.
いま$i$と異なる$n$以下の任意の自然数$j$に対し, $x$は$A_{j}$の中に自由変数として現れないとする.
このとき$\neg A_{1}, \cdots, \neg A_{i - 1}, !_{A_{i}}x(R), \neg A_{i + 1}, \cdots, \neg A_{n}$が
すべて$\mathscr{T}$の定理ならば, 
$!_{A_{1} \vee A_{2} \vee \cdots \vee A_{n}}x(R)$は$\mathscr{T}$の定理である.
\end{dedu}




\mathstrut
\begin{dedu}
\label{dedsp!pregvfreeeq}%推論545
$R$を$\mathscr{T}$の関係式とし, $x$を文字とする.
また$n$を自然数とし, $A_{1}, A_{2}, \cdots, A_{n}$を$\mathscr{T}$の関係式とする.
また$i$を$n$以下の自然数とする.
いま$i$と異なる$n$以下の任意の自然数$j$に対し, $x$は$A_{j}$の中に自由変数として現れないとする.

1)
$\neg !x(R)$が$\mathscr{T}$の定理ならば, 
\[
  !_{A_{1} \vee A_{2} \vee \cdots \vee A_{n}}x(R) 
  \leftrightarrow \neg A_{1} \wedge \cdots \wedge \neg A_{i - 1} \, \wedge \, !_{A_{i}}x(R) \wedge \neg A_{i + 1} \wedge \cdots \wedge \neg A_{n}
\]
は$\mathscr{T}$の定理である.

2)
$\neg !x(R)$と$!_{A_{1} \vee A_{2} \vee \cdots \vee A_{n}}x(R)$が共に$\mathscr{T}$の定理ならば, 
\[
  \neg A_{1} \wedge \cdots \wedge \neg A_{i - 1} \, \wedge \, !_{A_{i}}x(R) \wedge \neg A_{i + 1} \wedge \cdots \wedge \neg A_{n}
\]
は$\mathscr{T}$の定理である.
\end{dedu}




\mathstrut
\begin{dedu}
\label{dedsp!pregvfreeeq2}%推論546
$R$を$\mathscr{T}$の関係式とする.
また$n$を自然数とし, $A_{1}, A_{2}, \cdots, A_{n}$を$\mathscr{T}$の関係式とする.
また$i$を$n$以下の自然数とし, $x$を$A_{i}$の中に自由変数として現れない文字とする.
このとき$\neg !x(R)$と$!_{A_{1} \vee A_{2} \vee \cdots \vee A_{n}}x(R)$が共に$\mathscr{T}$の定理ならば, 
$\neg A_{i}$は$\mathscr{T}$の定理である.
\end{dedu}




\mathstrut
\begin{dedu}
\label{dedsp!pregwedge}%推論547
$R$を$\mathscr{T}$の関係式とし, $x$を文字とする.
また$n$を自然数とし, $A_{1}, A_{2}, \cdots, A_{n}$を$\mathscr{T}$の関係式とする.
また$i$を$n$以下の自然数とする.
$!_{A_{i}}x(R)$が$\mathscr{T}$の定理ならば, 
$!_{A_{1} \wedge A_{2} \wedge \cdots \wedge A_{n}}x(R)$は$\mathscr{T}$の定理である.
\end{dedu}




\mathstrut
\begin{dedu}
\label{dedsp!pregwedge2}%推論548
$R$を$\mathscr{T}$の関係式とし, $x$を文字とする.
また$n$を自然数とし, $A_{1}, A_{2}, \cdots, A_{n}$を$\mathscr{T}$の関係式とする.
また$k$を自然数とし, $i_{1}, i_{2}, \cdots, i_{k}$を$n$以下の自然数とする.
$!_{A_{i_{1}} \wedge A_{i_{2}} \wedge \cdots \wedge A_{i_{k}}}x(R)$が$\mathscr{T}$の定理ならば, 
$!_{A_{1} \wedge A_{2} \wedge \cdots \wedge A_{n}}x(R)$は$\mathscr{T}$の定理である.
\end{dedu}




\mathstrut
\begin{dedu}
\label{dedsp!pregw}%推論549
$R$を$\mathscr{T}$の関係式とし, $x$を文字とする.
また$n$を自然数とし, $A_{1}, A_{2}, \cdots, A_{n}$を$\mathscr{T}$の関係式とする.

1)
$!_{A_{1}}x(R) \, \vee \, !_{A_{2}}x(R) \vee \cdots \vee \, !_{A_{n}}x(R)$が$\mathscr{T}$の定理ならば, 
$!_{A_{1} \wedge A_{2} \wedge \cdots \wedge A_{n}}x(R)$は$\mathscr{T}$の定理である.

2)
$i$を$n$以下の自然数とする.
$!_{A_{1} \wedge A_{2} \wedge \cdots \wedge A_{n}}x(R)$が$\mathscr{T}$の定理ならば, 
\[
  \exists_{R}x(A_{i} \wedge \neg A_{1}) \vee \cdots 
  \vee \exists_{R}x(A_{i} \wedge \neg A_{i - 1}) \, \vee \, !_{A_{i}}x(R) \vee \exists_{R}x(A_{i} \wedge \neg A_{i + 1}) \vee \cdots 
  \vee \exists_{R}x(A_{i} \wedge \neg A_{n})
\]
は$\mathscr{T}$の定理である.
\end{dedu}




\mathstrut
\begin{dedu}
\label{dedsp!pregw3}%推論550
$R$を$\mathscr{T}$の関係式とし, $x$を文字とする.
また$n$を自然数とし, $A_{1}, A_{2}, \cdots, A_{n}$を$\mathscr{T}$の関係式とする.
また$i$を$n$以下の自然数とする.
$!_{A_{1} \wedge A_{2} \wedge \cdots \wedge A_{n}}x(R)$が$\mathscr{T}$の定理ならば, 
$\exists_{R}x(\neg A_{1}) \vee \cdots \vee \exists_{R}x(\neg A_{i - 1}) \, \vee \, !_{A_{i}}x(R) \vee \exists_{R}x(\neg A_{i + 1}) \vee \cdots \vee \exists_{R}x(\neg A_{n})$は
$\mathscr{T}$の定理である.
\end{dedu}




\mathstrut
\begin{dedu}
\label{dedsp!pregwfree}%推論551
$R$を$\mathscr{T}$の関係式とし, $x$を文字とする.
また$n$を自然数とし, $A_{1}, A_{2}, \cdots, A_{n}$を$\mathscr{T}$の関係式とする.
また$i$を$n$以下の自然数とする.
いま$i$と異なる$n$以下の任意の自然数$j$に対し, $x$は$A_{j}$の中に自由変数として現れないとする.

1)
$!_{A_{1} \wedge A_{2} \wedge \cdots \wedge A_{n}}x(R)$が$\mathscr{T}$の定理ならば, 
$\neg A_{1} \vee \cdots \vee \neg A_{i - 1} \, \vee \, !_{A_{i}}x(R) \vee \neg A_{i + 1} \vee \cdots \vee \neg A_{n}$は
$\mathscr{T}$の定理である.

2)
$\neg A_{1} \vee \cdots \vee \neg A_{i - 1} \, \vee \, !_{A_{i}}x(R) \vee \neg A_{i + 1} \vee \cdots \vee \neg A_{n}$が
$\mathscr{T}$の定理ならば, 
$!_{A_{1} \wedge A_{2} \wedge \cdots \wedge A_{n}}x(R)$は$\mathscr{T}$の定理である.
\end{dedu}




\mathstrut
\begin{dedu}
\label{dedsp!pregwfree2}%推論552
$R$を$\mathscr{T}$の関係式とする.
また$n$を自然数とし, $A_{1}, A_{2}, \cdots, A_{n}$を$\mathscr{T}$の関係式とする.
また$i$を$n$以下の自然数とし, $x$を$A_{i}$の中に自由変数として現れない文字とする.
$\neg A_{i}$が$\mathscr{T}$の定理ならば, 
$!_{A_{1} \wedge A_{2} \wedge \cdots \wedge A_{n}}x(R)$は$\mathscr{T}$の定理である.
\end{dedu}




\mathstrut
\begin{dedu}
\label{dedspalleqsp!sep}%推論553
$A$, $R$, $S$を$\mathscr{T}$の関係式とし, $x$を文字とする.
$\forall_{A}x(R \leftrightarrow S)$が$\mathscr{T}$の定理ならば, 
$!_{A}x(R) \leftrightarrow \ !_{A}x(S)$は$\mathscr{T}$の定理である.
\end{dedu}




\mathstrut
\begin{dedu}
\label{dedspalleqsp!sepconst}%推論554
$A$, $R$, $S$を$\mathscr{T}$の関係式とし, $x$を$\mathscr{T}$の定数でない文字とする.
$A \to (R \leftrightarrow S)$が$\mathscr{T}$の定理ならば, 
$!_{A}x(R) \leftrightarrow \ !_{A}x(S)$は$\mathscr{T}$の定理である.
\end{dedu}




\mathstrut
\begin{dedu}
\label{dedalleqsp!sep}%推論555
$A$, $R$, $S$を$\mathscr{T}$の関係式とし, $x$を文字とする.
$\forall x(R \leftrightarrow S)$が$\mathscr{T}$の定理ならば, 
$!_{A}x(R) \leftrightarrow \ !_{A}x(S)$は$\mathscr{T}$の定理である.
\end{dedu}




\mathstrut
\begin{dedu}
\label{dedalleqsp!sepconst}%推論556
$A$, $R$, $S$を$\mathscr{T}$の関係式とし, $x$を$\mathscr{T}$の定数でない文字とする.
$R \leftrightarrow S$が$\mathscr{T}$の定理ならば, 
$!_{A}x(R) \leftrightarrow \ !_{A}x(S)$は$\mathscr{T}$の定理である.
\end{dedu}




\mathstrut
\begin{dedu}
\label{dedspallpreeqsp!sep}%推論557
$A$, $B$, $R$を$\mathscr{T}$の関係式とし, $x$を文字とする.
$\forall_{R}x(A \leftrightarrow B)$が$\mathscr{T}$の定理ならば, 
$!_{A}x(R) \leftrightarrow \ !_{B}x(R)$は$\mathscr{T}$の定理である.
\end{dedu}




\mathstrut
\begin{dedu}
\label{dedspallpreeqsp!sepconst}%推論558
$A$, $B$, $R$を$\mathscr{T}$の関係式とし, $x$を$\mathscr{T}$の定数でない文字とする.
$R \to (A \leftrightarrow B)$が$\mathscr{T}$の定理ならば, 
$!_{A}x(R) \leftrightarrow \ !_{B}x(R)$は$\mathscr{T}$の定理である.
\end{dedu}




\mathstrut
\begin{dedu}
\label{dedallpreeqsp!sep}%推論559
$A$, $B$, $R$を$\mathscr{T}$の関係式とし, $x$を文字とする.
$\forall x(A \leftrightarrow B)$が$\mathscr{T}$の定理ならば, 
$!_{A}x(R) \leftrightarrow \ !_{B}x(R)$は$\mathscr{T}$の定理である.
\end{dedu}




\mathstrut
\begin{dedu}
\label{dedallpreeqsp!sepconst}%推論560
$A$, $B$, $R$を$\mathscr{T}$の関係式とし, $x$を$\mathscr{T}$の定数でない文字とする.
$A \leftrightarrow B$が$\mathscr{T}$の定理ならば, 
$!_{A}x(R) \leftrightarrow \ !_{B}x(R)$は$\mathscr{T}$の定理である.
\end{dedu}




\mathstrut
\begin{dedu}
\label{dedsp!extallsp!}%推論561
$A$と$R$を$\mathscr{T}$の関係式とする.
また$x$と$y$を文字とし, $y$は$A$の中に自由変数として現れないとする.
$!_{A}x(\exists y(R))$が$\mathscr{T}$の定理ならば, 
$\forall y(!_{A}x(R))$は$\mathscr{T}$の定理である.
\end{dedu}




\mathstrut
\begin{dedu}
\label{dedexsp!tsp!all}%推論562
$A$と$R$を$\mathscr{T}$の関係式とする.
また$x$と$y$を文字とし, $x$は$A$の中に自由変数として現れないとする.
$\exists x(!_{A}y(R))$が$\mathscr{T}$の定理ならば, 
$!_{A}y(\forall x(R))$は$\mathscr{T}$の定理である.
\end{dedu}




\mathstrut
\begin{dedu}
\label{dedsp!quanch}%推論563
$A$と$R$を$\mathscr{T}$の関係式とする.
また$x$と$y$を文字とし, $y$は$A$の中に自由変数として現れないとする.

1)
$!_{A}x(\exists y(R))$が$\mathscr{T}$の定理ならば, 
$\exists y(!_{A}x(R))$は$\mathscr{T}$の定理である.

2)
$\forall y(!_{A}x(R))$が$\mathscr{T}$の定理ならば, 
$!_{A}x(\forall y(R))$は$\mathscr{T}$の定理である.
\end{dedu}




\mathstrut
\begin{dedu}
\label{dedsp!spextspallsp!}%推論564
$A$, $B$, $R$を$\mathscr{T}$の関係式とし, $x$を$B$の中に自由変数として現れない文字, 
$y$を$A$の中に自由変数として現れない文字とする.
$!_{A}x(\exists_{B}y(R))$が$\mathscr{T}$の定理ならば, 
$\forall_{B}y(!_{A}x(R))$は$\mathscr{T}$の定理である.
\end{dedu}




\mathstrut
\begin{dedu}
\label{dedspexsp!tsp!spall}%推論565
$A$, $B$, $R$を$\mathscr{T}$の関係式とし, $x$を$B$の中に自由変数として現れない文字, 
$y$を$A$の中に自由変数として現れない文字とする.
$\exists_{A}x(!_{B}y(R))$が$\mathscr{T}$の定理ならば, 
$!_{B}y(\forall_{A}x(R))$は$\mathscr{T}$の定理である.
\end{dedu}




\mathstrut
\begin{dedu}
\label{dedsp!spquanch}%推論566
$A$, $B$, $R$を$\mathscr{T}$の関係式とし, $x$を$B$の中に自由変数として現れない文字, 
$y$を$A$の中に自由変数として現れない文字とする.

1)
$\exists y(B)$が$\mathscr{T}$の定理ならば, 
$!_{A}x(\exists_{B}y(R)) \to \exists_{B}y(!_{A}x(R))$と
$\forall_{B}y(!_{A}x(R)) \to \ !_{A}x(\forall_{B}y(R))$は
共に$\mathscr{T}$の定理である.

2)
$!_{A}x(\exists_{B}y(R)) \to \exists_{B}y(!_{A}x(R))$が$\mathscr{T}$の定理ならば, 
$\exists y(B)$は$\mathscr{T}$の定理である.
\end{dedu}




\mathstrut
\begin{dedu}
\label{dedsp!spquanch2}%推論567
$A$, $B$, $R$を$\mathscr{T}$の関係式とし, $x$を$B$の中に自由変数として現れない文字, 
$y$を$A$の中に自由変数として現れない文字とする.

1)
$\exists y(B)$と$!_{A}x(\exists_{B}y(R))$が共に$\mathscr{T}$の定理ならば, 
$\exists_{B}y(!_{A}x(R))$は$\mathscr{T}$の定理である.

2)
$\exists y(B)$と$\forall_{B}y(!_{A}x(R))$が共に$\mathscr{T}$の定理ならば, 
$!_{A}x(\forall_{B}y(R))$は$\mathscr{T}$の定理である.
\end{dedu}




\mathstrut
\begin{dedu}
\label{dedsp!spallcheq}%推論568
$A$, $B$, $R$を$\mathscr{T}$の関係式とし, $x$を$B$の中に自由変数として現れない文字, 
$y$を$A$の中に自由変数として現れない文字とする.

1)
$\neg !x(A)$が$\mathscr{T}$の定理ならば, 
$\exists y(B) \leftrightarrow (\forall_{B}y(!_{A}x(R)) \to \ !_{A}x(\forall_{B}y(R)))$は
$\mathscr{T}$の定理である.

2)
$\neg !x(A)$と$\forall_{B}y(!_{A}x(R)) \to \ !_{A}x(\forall_{B}y(R))$が共に$\mathscr{T}$の定理ならば, 
$\exists y(B)$は$\mathscr{T}$の定理である.
\end{dedu}




\mathstrut
\begin{dedu}
\label{ded!lnexvex!}%推論569
$R$を$\mathscr{T}$の関係式とし, $x$を文字とする.

1)
$!x(R)$が$\mathscr{T}$の定理ならば, 
$\neg \exists x(R) \vee \exists !x(R)$は$\mathscr{T}$の定理である.

2)
$\neg \exists x(R) \vee \exists !x(R)$が$\mathscr{T}$の定理ならば, 
$!x(R)$は$\mathscr{T}$の定理である.
\end{dedu}




\mathstrut
\begin{dedu}
\label{dedex!free}%推論570
$R$を$\mathscr{T}$の関係式とし, $x$を$R$の中に自由変数として現れない文字とする.
また$y$と$z$を, 互いに異なる文字とする.

1)
$\exists !x(R)$が$\mathscr{T}$の定理ならば, 
$R \wedge \forall y(\forall z(y = z))$は$\mathscr{T}$の定理である.

2)
$R \wedge \forall y(\forall z(y = z))$が$\mathscr{T}$の定理ならば, 
$\exists !x(R)$は$\mathscr{T}$の定理である.
\end{dedu}




\mathstrut
\begin{dedu}
\label{dedex!free2}%推論571
$R$を$\mathscr{T}$の関係式とし, $x$を$R$の中に自由変数として現れない文字とする.
また$y$と$z$を, 互いに異なる文字とする.

1)
$\exists !x(R)$が$\mathscr{T}$の定理ならば, 
$R$と$\forall y(\forall z(y = z))$は共に$\mathscr{T}$の定理である.

2)
$R$と$\forall y(\forall z(y = z))$が共に$\mathscr{T}$の定理ならば, 
$\exists !x(R)$は$\mathscr{T}$の定理である.
\end{dedu}




\mathstrut
\begin{dedu}
\label{dedex!free3}%推論572
$R$を$\mathscr{T}$の関係式とし, $x$を$R$の中に自由変数として現れない文字とする.

1)
$y$と$z$を互いに異なる文字とする.
$\exists y(\exists z(y \neq z))$が$\mathscr{T}$の定理ならば, 
$\neg \exists !x(R)$は$\mathscr{T}$の定理である.

2)
$\mathscr{T}$の或る対象式$T$, $U$に対して$T \neq U$が$\mathscr{T}$の定理ならば, 
$\neg \exists !x(R)$は$\mathscr{T}$の定理である.
\end{dedu}




\mathstrut
\begin{dedu}
\label{dedex!lall}%推論573
$R$を$\mathscr{T}$の関係式とし, $x$を文字とする.

1)
$\exists !x(R)$が$\mathscr{T}$の定理ならば, 
$\forall x(R \leftrightarrow x = \tau_{x}(R))$は$\mathscr{T}$の定理である.

2)
$\forall x(R \leftrightarrow x = \tau_{x}(R))$が$\mathscr{T}$の定理ならば, 
$\exists !x(R)$は$\mathscr{T}$の定理である.
\end{dedu}




\mathstrut
\begin{dedu}
\label{dedex!lallconst}%推論574
$R$を$\mathscr{T}$の関係式とし, $x$を$\mathscr{T}$の定数でない文字とする.
$R \leftrightarrow x = \tau_{x}(R)$が$\mathscr{T}$の定理ならば, 
$\exists !x(R)$は$\mathscr{T}$の定理である.
\end{dedu}




\mathstrut
\begin{dedu}
\label{dedex!tTtau}%推論575
$R$を$\mathscr{T}$の関係式, $T$を$\mathscr{T}$の対象式とし, $x$を文字とする.

1)
$\exists !x(R)$が$\mathscr{T}$の定理ならば, 
$(T|x)(R) \leftrightarrow T = \tau_{x}(R)$は$\mathscr{T}$の定理である.

2)
$\exists !x(R)$と$(T|x)(R)$が共に$\mathscr{T}$の定理ならば, 
$T = \tau_{x}(R)$は$\mathscr{T}$の定理である.

3)
$\exists !x(R)$と$T = \tau_{x}(R)$が共に$\mathscr{T}$の定理ならば, 
$(T|x)(R)$は$\mathscr{T}$の定理である.
\end{dedu}




\mathstrut
\begin{dedu}
\label{dedex!thm}%推論576
$R$を$\mathscr{T}$の関係式, $T$を$\mathscr{T}$の対象式とし, 
$x$を$T$の中に自由変数として現れない文字とする.
$\forall x(R \leftrightarrow x = T)$が$\mathscr{T}$の定理ならば, 
$\exists !x(R)$は$\mathscr{T}$の定理である.
\end{dedu}




\mathstrut
\begin{dedu}
\label{dedex!thmconst}%推論577
$R$を$\mathscr{T}$の関係式, $T$を$\mathscr{T}$の対象式とし, 
$x$を$T$の中に自由変数として現れない, $\mathscr{T}$の定数でない文字とする.
$R \leftrightarrow x = T$が$\mathscr{T}$の定理ならば, 
$\exists !x(R)$は$\mathscr{T}$の定理である.
\end{dedu}




\mathstrut
\begin{dedu}
\label{dedalltT=tau}%推論578
$R$を$\mathscr{T}$の関係式, $T$を$\mathscr{T}$の対象式とし, 
$x$を$T$の中に自由変数として現れない文字とする.
$\forall x(R \leftrightarrow x = T)$が$\mathscr{T}$の定理ならば, 
$T = \tau_{x}(R)$は$\mathscr{T}$の定理である.
\end{dedu}




\mathstrut
\begin{dedu}
\label{dedalltT=tauconst}%推論579
$R$を$\mathscr{T}$の関係式, $T$を$\mathscr{T}$の対象式とし, 
$x$を$T$の中に自由変数として現れない, $\mathscr{T}$の定数でない文字とする.
$R \leftrightarrow x = T$が$\mathscr{T}$の定理ならば, 
$T = \tau_{x}(R)$は$\mathscr{T}$の定理である.
\end{dedu}




\mathstrut
\begin{dedu}
\label{dedex!equiv}%推論580
$R$を$\mathscr{T}$の関係式とし, $x$を文字とする.
また$y$を$x$と異なり, $R$の中に自由変数として現れない文字とする.

1)
$\exists !x(R)$が$\mathscr{T}$の定理ならば, 
$\exists y(\forall x(R \leftrightarrow x = y))$は$\mathscr{T}$の定理である.

2)
$\exists y(\forall x(R \leftrightarrow x = y))$が$\mathscr{T}$の定理ならば, 
$\exists !x(R)$は$\mathscr{T}$の定理である.
\end{dedu}




\mathstrut
\begin{dedu}
\label{dedex!equiv2}%推論581
$R$を$\mathscr{T}$の関係式とし, $x$を文字とする.
また$y$を$x$と異なり, $R$の中に自由変数として現れない文字とする.

1)
$\exists !x(R)$が$\mathscr{T}$の定理ならば, 
$\exists y((y|x)(R) \wedge \forall x(R \to x = y))$は$\mathscr{T}$の定理である.

2)
$\exists y((y|x)(R) \wedge \forall x(R \to x = y))$が$\mathscr{T}$の定理ならば, 
$\exists !x(R)$は$\mathscr{T}$の定理である.
\end{dedu}




\mathstrut
\begin{dedu}
\label{dedalleqex!sep}%推論582
$R$と$S$を$\mathscr{T}$の関係式とし, $x$を文字とする.
$\forall x(R \leftrightarrow S)$が$\mathscr{T}$の定理ならば, 
$\exists !x(R) \leftrightarrow \exists !x(S)$は$\mathscr{T}$の定理である.
\end{dedu}




\mathstrut
\begin{dedu}
\label{dedalleqex!sepconst}%推論583
$R$と$S$を$\mathscr{T}$の関係式とし, $x$を$\mathscr{T}$の定数でない文字とする.
$R \leftrightarrow S$が$\mathscr{T}$の定理ならば, 
$\exists !x(R) \leftrightarrow \exists !x(S)$は$\mathscr{T}$の定理である.
\end{dedu}




\mathstrut
\begin{dedu}
\label{dedex!tspquaneq}%推論584
$A$と$R$を$\mathscr{T}$の関係式とし, $x$を文字とする.
$\exists !x(A)$が$\mathscr{T}$の定理ならば, 
$\exists_{A}x(R) \leftrightarrow \forall_{A}x(R)$は$\mathscr{T}$の定理である.
\end{dedu}




\mathstrut
\begin{dedu}
\label{dedex!ttauspquaneq}%推論585
$A$と$R$を$\mathscr{T}$の関係式とし, $x$を文字とする.
$\exists !x(A)$が$\mathscr{T}$の定理ならば, 
$(\tau_{x}(A)|x)(R) \leftrightarrow \exists_{A}x(R)$と
$(\tau_{x}(A)|x)(R) \leftrightarrow \forall_{A}x(R)$は共に$\mathscr{T}$の定理である.
\end{dedu}




\mathstrut
\begin{dedu}
\label{dedex!exch}%推論586
$R$を$\mathscr{T}$の関係式とし, $x$と$y$を文字とする.
$\exists !x(\exists y(R))$が$\mathscr{T}$の定理ならば, 
$\exists y(\exists !x(R))$は$\mathscr{T}$の定理である.
\end{dedu}




\mathstrut
\begin{dedu}
\label{dedex!spexch}%推論587
$A$と$R$を$\mathscr{T}$の関係式とする.
また$x$と$y$を文字とし, $x$は$A$の中に自由変数として現れないとする.
$\exists !x(\exists_{A}y(R))$が$\mathscr{T}$の定理ならば, 
$\exists_{A}y(\exists !x(R))$は$\mathscr{T}$の定理である.
\end{dedu}




\mathstrut
\begin{dedu}
\label{dedsp!lnspexvspex!}%推論588
$A$と$R$を$\mathscr{T}$の関係式とし, $x$を文字とする.

1)
$!_{A}x(R)$が$\mathscr{T}$の定理ならば, 
$\neg \exists_{A}x(R) \vee \exists !_{A}x(R)$は$\mathscr{T}$の定理である.

2)
$\neg \exists_{A}x(R) \vee \exists !_{A}x(R)$が$\mathscr{T}$の定理ならば, 
$!_{A}x(R)$は$\mathscr{T}$の定理である.
\end{dedu}




\mathstrut
\begin{dedu}
\label{dedspex!afree}%推論589
$A$と$R$を$\mathscr{T}$の関係式とし, $x$を$A$の中に自由変数として現れない文字とする.

1)
$\exists !_{A}x(R)$が$\mathscr{T}$の定理ならば, 
$A \wedge \exists !x(R)$は$\mathscr{T}$の定理である.

2)
$\exists !_{A}x(R)$が$\mathscr{T}$の定理ならば, 
$A$と$\exists !x(R)$は共に$\mathscr{T}$の定理である.

3)
$A \wedge \exists !x(R)$が$\mathscr{T}$の定理ならば, 
$\exists !_{A}x(R)$は$\mathscr{T}$の定理である.

4)
$A$と$\exists !x(R)$が共に$\mathscr{T}$の定理ならば, 
$\exists !_{A}x(R)$は$\mathscr{T}$の定理である.
\end{dedu}




\mathstrut
\begin{dedu}
\label{dedspex!rfree}%推論590
$A$と$R$を$\mathscr{T}$の関係式とし, $x$を$R$の中に自由変数として現れない文字とする.

1)
$\exists !_{A}x(R)$が$\mathscr{T}$の定理ならば, 
$\exists !x(A) \wedge R$は$\mathscr{T}$の定理である.

2)
$\exists !_{A}x(R)$が$\mathscr{T}$の定理ならば, 
$\exists !x(A)$と$R$は共に$\mathscr{T}$の定理である.

3)
$\exists !x(A) \wedge R$が$\mathscr{T}$の定理ならば, 
$\exists !_{A}x(R)$は$\mathscr{T}$の定理である.

4)
$\exists !x(A)$と$R$が共に$\mathscr{T}$の定理ならば, 
$\exists !_{A}x(R)$は$\mathscr{T}$の定理である.
\end{dedu}




\mathstrut
\begin{dedu}
\label{dedspalltspex!}%推論591
$A$と$R$を$\mathscr{T}$の関係式, $T$を$\mathscr{T}$の対象式とし, 
$x$を$T$の中に自由変数として現れない文字とする.

1)
$(T|x)(A) \wedge \forall_{A}x(R \leftrightarrow x = T)$が$\mathscr{T}$の定理ならば, 
$\exists !_{A}x(R)$は$\mathscr{T}$の定理である.

2)
$(T|x)(A)$と$\forall_{A}x(R \leftrightarrow x = T)$が共に$\mathscr{T}$の定理ならば, 
$\exists !_{A}x(R)$は$\mathscr{T}$の定理である.
\end{dedu}




\mathstrut
\begin{dedu}
\label{dedspalltspex!const}%推論592
$A$と$R$を$\mathscr{T}$の関係式, $T$を$\mathscr{T}$の対象式とし, 
$x$を$T$の中に自由変数として現れない, $\mathscr{T}$の定数でない文字とする.
$(T|x)(A)$と$A \to (R \leftrightarrow x = T)$が共に$\mathscr{T}$の定理ならば, 
$\exists !_{A}x(R)$は$\mathscr{T}$の定理である.
\end{dedu}




\mathstrut
\begin{dedu}
\label{dedalltspex!}%推論593
$A$と$R$を$\mathscr{T}$の関係式, $T$を$\mathscr{T}$の対象式とし, 
$x$を$T$の中に自由変数として現れない文字とする.

1)
$(T|x)(A) \wedge \forall x(R \leftrightarrow x = T)$が$\mathscr{T}$の定理ならば, 
$\exists !_{A}x(R)$は$\mathscr{T}$の定理である.

2)
$(T|x)(A)$と$\forall x(R \leftrightarrow x = T)$が共に$\mathscr{T}$の定理ならば, 
$\exists !_{A}x(R)$は$\mathscr{T}$の定理である.
\end{dedu}




\mathstrut
\begin{dedu}
\label{dedalltspex!const}%推論594
$A$と$R$を$\mathscr{T}$の関係式, $T$を$\mathscr{T}$の対象式とし, 
$x$を$T$の中に自由変数として現れない, $\mathscr{T}$の定数でない文字とする.
$(T|x)(A)$と$R \leftrightarrow x = T$が共に$\mathscr{T}$の定理ならば, 
$\exists !_{A}x(R)$は$\mathscr{T}$の定理である.
\end{dedu}




\mathstrut
\begin{dedu}
\label{dedspalltT=sptau}%推論595
$A$と$R$を$\mathscr{T}$の関係式, $T$を$\mathscr{T}$の対象式とし, 
$x$を$T$の中に自由変数として現れない文字とする.

1)
$(T|x)(A) \wedge \forall_{A}x(R \leftrightarrow x = T)$が$\mathscr{T}$の定理ならば, 
$T = \tau_{x}(A \wedge R)$は$\mathscr{T}$の定理である.

2)
$(T|x)(A)$と$\forall_{A}x(R \leftrightarrow x = T)$が共に$\mathscr{T}$の定理ならば, 
$T = \tau_{x}(A \wedge R)$は$\mathscr{T}$の定理である.
\end{dedu}




\mathstrut
\begin{dedu}
\label{dedspalltT=sptauconst}%推論596
$A$と$R$を$\mathscr{T}$の関係式, $T$を$\mathscr{T}$の対象式とし, 
$x$を$T$の中に自由変数として現れない, $\mathscr{T}$の定数でない文字とする.
$(T|x)(A)$と$A \to (R \leftrightarrow x = T)$が共に$\mathscr{T}$の定理ならば, 
$T = \tau_{x}(A \wedge R)$は$\mathscr{T}$の定理である.
\end{dedu}




\mathstrut
\begin{dedu}
\label{dedalltT=sptau}%推論597
$A$と$R$を$\mathscr{T}$の関係式, $T$を$\mathscr{T}$の対象式とし, 
$x$を$T$の中に自由変数として現れない文字とする.

1)
$(T|x)(A) \wedge \forall x(R \leftrightarrow x = T)$が$\mathscr{T}$の定理ならば, 
$T = \tau_{x}(A \wedge R)$と$\tau_{x}(R) = \tau_{x}(A \wedge R)$は共に$\mathscr{T}$の定理である.

2)
$(T|x)(A)$と$\forall x(R \leftrightarrow x = T)$が共に$\mathscr{T}$の定理ならば, 
$T = \tau_{x}(A \wedge R)$と$\tau_{x}(R) = \tau_{x}(A \wedge R)$は共に$\mathscr{T}$の定理である.
\end{dedu}




\mathstrut
\begin{dedu}
\label{dedalltT=sptauconst}%推論598
$A$と$R$を$\mathscr{T}$の関係式, $T$を$\mathscr{T}$の対象式とし, 
$x$を$T$の中に自由変数として現れない, $\mathscr{T}$の定数でない文字とする.
$(T|x)(A)$と$R \leftrightarrow x = T$が共に$\mathscr{T}$の定理ならば, 
$T = \tau_{x}(A \wedge R)$と$\tau_{x}(R) = \tau_{x}(A \wedge R)$は共に$\mathscr{T}$の定理である.
\end{dedu}




\mathstrut
\begin{dedu}
\label{dedspex!tspall}%推論599
$A$と$R$を$\mathscr{T}$の関係式とし, $x$を文字とする.
$\exists !_{A}x(R)$が$\mathscr{T}$の定理ならば, 
$\forall_{A}x(R \leftrightarrow x = \tau_{x}(A \wedge R))$は$\mathscr{T}$の定理である.
\end{dedu}




\mathstrut
\begin{dedu}
\label{dedspex!lspall}%推論600
$A$と$R$を$\mathscr{T}$の関係式とし, $x$を文字とする.

1)
$\exists !_{A}x(R)$が$\mathscr{T}$の定理ならば, 
$(\tau_{x}(A \wedge R)|x)(A) \wedge \forall_{A}x(R \leftrightarrow x = \tau_{x}(A \wedge R))$は
$\mathscr{T}$の定理である.

2)
$(\tau_{x}(A \wedge R)|x)(A) \wedge \forall_{A}x(R \leftrightarrow x = \tau_{x}(A \wedge R))$が
$\mathscr{T}$の定理ならば, 
$\exists !_{A}x(R)$は$\mathscr{T}$の定理である.
\end{dedu}




\mathstrut
\begin{dedu}
\label{dedspex!equiv}%推論601
$A$と$R$を$\mathscr{T}$の関係式とし, $x$を文字とする.
また$y$を$x$と異なり, $A$及び$R$の中に自由変数として現れない文字とする.

1)
$\exists !_{A}x(R)$が$\mathscr{T}$の定理ならば, 
$\exists_{(y|x)(A)}y(\forall_{A}x(R \leftrightarrow x = y))$は$\mathscr{T}$の定理である.

2)
$\exists_{(y|x)(A)}y(\forall_{A}x(R \leftrightarrow x = y))$が$\mathscr{T}$の定理ならば, 
$\exists !_{A}x(R)$は$\mathscr{T}$の定理である.
\end{dedu}




\mathstrut
\begin{dedu}
\label{dedspex!equiv2}%推論602
$A$と$R$を$\mathscr{T}$の関係式とし, $x$を文字とする.
また$y$を$x$と異なり, $A$及び$R$の中に自由変数として現れない文字とする.

1)
$\exists !_{A}x(R)$が$\mathscr{T}$の定理ならば, 
$\exists_{(y|x)(A)}y((y|x)(R) \wedge \forall_{A}x(R \to x = y))$は$\mathscr{T}$の定理である.

2)
$\exists_{(y|x)(A)}y((y|x)(R) \wedge \forall_{A}x(R \to x = y))$が$\mathscr{T}$の定理ならば, 
$\exists !_{A}x(R)$は$\mathscr{T}$の定理である.
\end{dedu}




\mathstrut
\begin{dedu}
\label{dedspalleqspex!sep}%推論603
$A$, $R$, $S$を$\mathscr{T}$の関係式とし, $x$を文字とする.
$\forall_{A}x(R \leftrightarrow S)$が$\mathscr{T}$の定理ならば, 
$\exists !_{A}x(R) \leftrightarrow \exists !_{A}x(S)$は$\mathscr{T}$の定理である.
\end{dedu}




\mathstrut
\begin{dedu}
\label{dedspalleqspex!sepconst}%推論604
$A$, $R$, $S$を$\mathscr{T}$の関係式とし, $x$を$\mathscr{T}$の定数でない文字とする.
$A \to (R \leftrightarrow S)$が$\mathscr{T}$の定理ならば, 
$\exists !_{A}x(R) \leftrightarrow \exists !_{A}x(S)$は$\mathscr{T}$の定理である.
\end{dedu}




\mathstrut
\begin{dedu}
\label{dedalleqspex!sep}%推論605
$A$, $R$, $S$を$\mathscr{T}$の関係式とし, $x$を文字とする.
$\forall x(R \leftrightarrow S)$が$\mathscr{T}$の定理ならば, 
$\exists !_{A}x(R) \leftrightarrow \exists !_{A}x(S)$は$\mathscr{T}$の定理である.
\end{dedu}




\mathstrut
\begin{dedu}
\label{dedalleqspex!sepconst}%推論606
$A$, $R$, $S$を$\mathscr{T}$の関係式とし, $x$を$\mathscr{T}$の定数でない文字とする.
$R \leftrightarrow S$が$\mathscr{T}$の定理ならば, 
$\exists !_{A}x(R) \leftrightarrow \exists !_{A}x(S)$は$\mathscr{T}$の定理である.
\end{dedu}




\mathstrut
\begin{dedu}
\label{dedspallpreeqspex!sep}%推論607
$A$, $B$, $R$を$\mathscr{T}$の関係式とし, $x$を文字とする.
$\forall_{R}x(A \leftrightarrow B)$が$\mathscr{T}$の定理ならば, 
$\exists !_{A}x(R) \leftrightarrow \exists !_{B}x(R)$は$\mathscr{T}$の定理である.
\end{dedu}




\mathstrut
\begin{dedu}
\label{dedspallpreeqspex!sepconst}%推論608
$A$, $B$, $R$を$\mathscr{T}$の関係式とし, $x$を$\mathscr{T}$の定数でない文字とする.
$R \to (A \leftrightarrow B)$が$\mathscr{T}$の定理ならば, 
$\exists !_{A}x(R) \leftrightarrow \exists !_{B}x(R)$は$\mathscr{T}$の定理である.
\end{dedu}




\mathstrut
\begin{dedu}
\label{dedallpreeqspex!sep}%推論609
$A$, $B$, $R$を$\mathscr{T}$の関係式とし, $x$を文字とする.
$\forall x(A \leftrightarrow B)$が$\mathscr{T}$の定理ならば, 
$\exists !_{A}x(R) \leftrightarrow \exists !_{B}x(R)$は$\mathscr{T}$の定理である.
\end{dedu}




\mathstrut
\begin{dedu}
\label{dedallpreeqspex!sepconst}%推論610
$A$, $B$, $R$を$\mathscr{T}$の関係式とし, $x$を$\mathscr{T}$の定数でない文字とする.
$A \leftrightarrow B$が$\mathscr{T}$の定理ならば, 
$\exists !_{A}x(R) \leftrightarrow \exists !_{B}x(R)$は$\mathscr{T}$の定理である.
\end{dedu}




\mathstrut
\begin{dedu}
\label{dedspex!exch}%推論611
$A$と$R$を$\mathscr{T}$の関係式とする.
また$x$と$y$を文字とし, $y$は$A$の中に自由変数として現れないとする.
$\exists !_{A}x(\exists y(R))$が$\mathscr{T}$の定理ならば, 
$\exists y(\exists !_{A}x(R))$は$\mathscr{T}$の定理である.
\end{dedu}




\mathstrut
\begin{dedu}
\label{dedspex!spexch}%推論612
$A$, $B$, $R$を$\mathscr{T}$の関係式とし, $x$を$B$の中に自由変数として現れない文字, 
$y$を$A$の中に自由変数として現れない文字とする.
$\exists !_{A}x(\exists_{B}y(R))$が$\mathscr{T}$の定理ならば, 
$\exists_{B}y(\exists !_{A}x(R))$は$\mathscr{T}$の定理である.
\end{dedu}




\newpage




\section{Thm}




\mathstrut
\begin{theo}
\label{ata}%Thm1
$A$が$\mathscr{T}$の関係式ならば, 
\[
  A \to A
\]
は$\mathscr{T}$の定理である.
\end{theo}




\mathstrut
\begin{theo}
\label{at11atb1tb1}%Thm2
$A$と$B$が$\mathscr{T}$の関係式ならば, 
\[
  A \to ((A \to B) \to B)
\]
は$\mathscr{T}$の定理である.
\end{theo}




\mathstrut
\begin{theo}
\label{1atb1t11cta1t1ctb11}%Thm3
$A$, $B$, $C$が$\mathscr{T}$の関係式ならば, 
\[
  (A \to B) \to ((C \to A) \to (C \to B))
\]
は$\mathscr{T}$の定理である.
\end{theo}




\mathstrut
\begin{theo}
\label{1atb1t11btc1t1atc11}%Thm4
$A$, $B$, $C$が$\mathscr{T}$の関係式ならば, 
\[
  (A \to B) \to ((B \to C) \to (A \to C))
\]
は$\mathscr{T}$の定理である.
\end{theo}




\mathstrut
\begin{theo}
\label{1at1btc11t1bt1atc11}%Thm5
$A$, $B$, $C$が$\mathscr{T}$の関係式ならば, 
\[
  (A \to (B \to C)) \to (B \to (A \to C))
\]
は$\mathscr{T}$の定理である.
\end{theo}




\mathstrut
\begin{theo}
\label{1at1atb11t1atb1}%Thm6
$A$と$B$が$\mathscr{T}$の関係式ならば, 
\[
  (A \to (A \to B)) \to (A \to B)
\]
は$\mathscr{T}$の定理である.
\end{theo}




\mathstrut
\begin{theo}
\label{1atb1t11at1btc11t1atc11}%Thm7
$A$, $B$, $C$が$\mathscr{T}$の関係式ならば, 
\[
  (A \to B) \to ((A \to (B \to C)) \to (A \to C))
\]
は$\mathscr{T}$の定理である.
\end{theo}




\mathstrut
\begin{theo}
\label{11atb1t1atc11t1at1btc11}%Thm8
$A$, $B$, $C$が$\mathscr{T}$の関係式ならば, 
\[
  ((A \to B) \to (A \to C)) \to (A \to (B \to C))
\]
は$\mathscr{T}$の定理である.
\end{theo}




\mathstrut
\begin{theo}
\label{nat1atb1}%Thm9
$A$と$B$が$\mathscr{T}$の関係式ならば, 
\[
  \neg A \to (A \to B), ~~
  A \to (\neg A \to B)
\]
は共に$\mathscr{T}$の定理である.
\end{theo}




\mathstrut
\begin{theo}
\label{nnata}%Thm10
{\bf (二重否定の除去定理)}~
$A$が$\mathscr{T}$の関係式ならば, 
\[
  \neg \neg A \to A
\]
は$\mathscr{T}$の定理である.
\end{theo}




\mathstrut
\begin{theo}
\label{atnna}%Thm11
{\bf (二重否定の導入定理)}~
$A$が$\mathscr{T}$の関係式ならば, 
\[
  A \to \neg \neg A
\]
は$\mathscr{T}$の定理である.
\end{theo}




\mathstrut
\begin{theo}
\label{1atb1t1nbtna1}%Thm12
$A$と$B$が$\mathscr{T}$の関係式ならば, 
\[
  (A \to B) \to (\neg B \to \neg A)
\]
は$\mathscr{T}$の定理である.
\end{theo}




\mathstrut
\begin{theo}
\label{1natb1t1nbta1}%Thm13
$A$と$B$が$\mathscr{T}$の関係式ならば, 
\[
  (\neg A \to B) \to (\neg B \to A)
\]
は$\mathscr{T}$の定理である.
\end{theo}




\mathstrut
\begin{theo}
\label{1atnb1t1btna1}%Thm14
$A$と$B$が$\mathscr{T}$の関係式ならば, 
\[
  (A \to \neg B) \to (B \to \neg A)
\]
は$\mathscr{T}$の定理である.
\end{theo}




\mathstrut
\begin{theo}
\label{1nat1btc11t11cta1t1bta11}%Thm15
$A$, $B$, $C$が$\mathscr{T}$の関係式ならば, 
\[
  (\neg A \to (B \to C)) \to ((C \to A) \to (B \to A))
\]
は$\mathscr{T}$の定理である.
\end{theo}




\mathstrut
\begin{theo}
\label{11cta1t1bta11t1nat1btc11}%Thm16
$A$, $B$, $C$が$\mathscr{T}$の関係式ならば, 
\[
  ((C \to A) \to (B \to A)) \to (\neg A \to (B \to C))
\]
は$\mathscr{T}$の定理である.
\end{theo}




\mathstrut
\begin{theo}
\label{1atna1tna}%Thm17
$A$が$\mathscr{T}$の関係式ならば, 
\[
  (A \to \neg A) \to \neg A
\]
は$\mathscr{T}$の定理である.
\end{theo}




\mathstrut
\begin{theo}
\label{1nata1ta}%Thm18
$A$が$\mathscr{T}$の関係式ならば, 
\[
  (\neg A \to A) \to A
\]
は$\mathscr{T}$の定理である.
\end{theo}




\mathstrut
\begin{theo}
\label{11atb1ta1ta}%Thm19
{\bf (Peirceの法則)}~
$A$と$B$が$\mathscr{T}$の関係式ならば, 
\[
  ((A \to B) \to A) \to A
\]
は$\mathscr{T}$の定理である.
\end{theo}




\mathstrut
\begin{theo}
\label{1atb1t11natb1tb1}%Thm20
$A$と$B$が$\mathscr{T}$の関係式ならば, 
\[
  (A \to B) \to ((\neg A \to B) \to B)
\]
は$\mathscr{T}$の定理である.
\end{theo}




\mathstrut
\begin{theo}
\label{1atb1t11atnb1tna1}%Thm21
$A$と$B$が$\mathscr{T}$の関係式ならば, 
\[
  (A \to B) \to ((A \to \neg B) \to \neg A)
\]
は$\mathscr{T}$の定理である.
\end{theo}




\mathstrut
\begin{theo}
\label{1natb1t11natnb1ta1}%Thm22
$A$と$B$が$\mathscr{T}$の関係式ならば, 
\[
  (\neg A \to B) \to ((\neg A \to \neg B) \to A)
\]
は$\mathscr{T}$の定理である.
\end{theo}




\mathstrut
\begin{theo}
\label{at1nbtn1atb11}%Thm23
$A$と$B$が$\mathscr{T}$の関係式ならば, 
\[
  A \to (\neg B \to \neg (A \to B))
\]
は$\mathscr{T}$の定理である.
\end{theo}




\mathstrut
\begin{theo}
\label{at1btn1atnb11}%Thm24
$A$と$B$が$\mathscr{T}$の関係式ならば, 
\[
  A \to (B \to \neg (A \to \neg B))
\]
は$\mathscr{T}$の定理である.
\end{theo}




\mathstrut
\begin{theo}
\label{atavb}%Thm25
$A$と$B$が$\mathscr{T}$の関係式ならば, 
\[
  A \to A \vee B, ~~
  B \to A \vee B
\]
は共に$\mathscr{T}$の定理である.
\end{theo}




\mathstrut
\begin{theo}
\label{1atc1t11btc1t1avbtc11}%Thm26
$A$, $B$, $C$が$\mathscr{T}$の関係式ならば, 
\[
  (A \to C) \to ((B \to C) \to (A \vee B \to C))
\]
は$\mathscr{T}$の定理である.
\end{theo}




\mathstrut
\begin{theo}
\label{1atb1t1avbtb1}%Thm27
$A$と$B$が$\mathscr{T}$の関係式ならば, 
\[
  (A \to B) \to (A \vee B \to B), ~~
  (B \to A) \to (A \vee B \to A)
\]
は共に$\mathscr{T}$の定理である.
\end{theo}




\mathstrut
\begin{theo}
\label{1atb1vct1avctbvc1}%Thm28
$A$, $B$, $C$が$\mathscr{T}$の関係式ならば, 
\begin{align*}
  (A \to B) \vee C \to (A \vee C \to B \vee C)&, ~~
  (A \to B) \vee C \to (C \vee A \to C \vee B), \\
  \mbox{} \\
  C \vee (A \to B) \to (A \vee C \to B \vee C)&, ~~
  C \vee (A \to B) \to (C \vee A \to C \vee B)
\end{align*}
はいずれも$\mathscr{T}$の定理である.
\end{theo}




\mathstrut
\begin{theo}
\label{1avctbvc1t1atb1vc}%Thm29
$A$, $B$, $C$が$\mathscr{T}$の関係式ならば, 
\begin{align*}
  (A \vee C \to B \vee C) \to (A \to B) \vee C&, ~~
  (C \vee A \to C \vee B) \to (A \to B) \vee C, \\
  \mbox{} \\
  (A \vee C \to B \vee C) \to C \vee (A \to B)&, ~~
  (C \vee A \to C \vee B) \to C \vee (A \to B)
\end{align*}
はいずれも$\mathscr{T}$の定理である.
\end{theo}




\mathstrut
\begin{theo}
\label{1atb1t1avctbvc1}%Thm30
$A$, $B$, $C$が$\mathscr{T}$の関係式ならば, 
\[
  (A \to B) \to (A \vee C \to B \vee C), ~~
  (A \to B) \to (C \vee A \to C \vee B)
\]
は共に$\mathscr{T}$の定理である.
\end{theo}




\mathstrut
\begin{theo}
\label{avna}%Thm31
{\bf (排中律)}~
$A$が$\mathscr{T}$の関係式ならば, 
\[
  A \vee \neg A
\]
は$\mathscr{T}$の定理である.
\end{theo}




\mathstrut
\begin{theo}
\label{avata}%Thm32
$A$が$\mathscr{T}$の関係式ならば, 
\[
  A \vee A \to A
\]
は$\mathscr{T}$の定理である.
\end{theo}




\mathstrut
\begin{theo}
\label{avbtbva}%Thm33
$A$と$B$が$\mathscr{T}$の関係式ならば, 
\[
  A \vee B \to B \vee A
\]
は$\mathscr{T}$の定理である.
\end{theo}




\mathstrut
\begin{theo}
\label{1avb1vctav1bvc1}%Thm34
$A$, $B$, $C$が$\mathscr{T}$の関係式ならば, 
\[
  (A \vee B) \vee C \to A \vee (B \vee C)
\]
は$\mathscr{T}$の定理である.
\end{theo}




\mathstrut
\begin{theo}
\label{av1bvc1t1avb1vc}%Thm35
$A$, $B$, $C$が$\mathscr{T}$の関係式ならば, 
\[
  A \vee (B \vee C) \to (A \vee B) \vee C
\]
は$\mathscr{T}$の定理である.
\end{theo}




\mathstrut
\begin{theo}
\label{av1bvc1t1avb1v1avc1}%Thm36
$A$, $B$, $C$が$\mathscr{T}$の関係式ならば, 
\[
  A \vee (B \vee C) \to (A \vee B) \vee (A \vee C), ~~
  (A \vee B) \vee C \to (A \vee C) \vee (B \vee C)
\]
は共に$\mathscr{T}$の定理である.
\end{theo}




\mathstrut
\begin{theo}
\label{1avb1v1avc1tav1bvc1}%Thm37
$A$, $B$, $C$が$\mathscr{T}$の関係式ならば, 
\[
  (A \vee B) \vee (A \vee C) \to A \vee (B \vee C), ~~
  (A \vee C) \vee (B \vee C) \to (A \vee B) \vee C
\]
は共に$\mathscr{T}$の定理である.
\end{theo}




\mathstrut
\begin{theo}
\label{1atb1tnavb}%Thm38
$A$と$B$が$\mathscr{T}$の関係式ならば, 
\[
  (A \to B) \to \neg A \vee B, ~~
  \neg A \vee B \to (A \to B)
\]
は共に$\mathscr{T}$の定理である.
\end{theo}




\mathstrut
\begin{theo}
\label{1atb1vct1atbvc1}%Thm39
$A$, $B$, $C$が$\mathscr{T}$の関係式ならば, 
\[
  (A \to B) \vee C \to (A \to B \vee C), ~~
  C \vee (A \to B) \to (A \to C \vee B)
\]
は共に$\mathscr{T}$の定理である.
\end{theo}




\mathstrut
\begin{theo}
\label{1atbvc1t1atb1vc}%Thm40
$A$, $B$, $C$が$\mathscr{T}$の関係式ならば, 
\[
  (A \to B \vee C) \to (A \to B) \vee C, ~~
  (A \to C \vee B) \to C \vee (A \to B)
\]
は共に$\mathscr{T}$の定理である.
\end{theo}




\mathstrut
\begin{theo}
\label{1atbvc1t1avctbvc1}%Thm41
$A$, $B$, $C$が$\mathscr{T}$の関係式ならば, 
\[
  (A \to B \vee C) \to (A \vee C \to B \vee C), ~~
  (A \to C \vee B) \to (C \vee A \to C \vee B)
\]
は共に$\mathscr{T}$の定理である.
\end{theo}




\mathstrut
\begin{theo}
\label{1avctbvc1t1atbvc1}%Thm42
$A$, $B$, $C$が$\mathscr{T}$の関係式ならば, 
\[
  (A \vee C \to B \vee C) \to (A \to B \vee C), ~~
  (C \vee A \to C \vee B) \to (A \to C \vee B)
\]
は共に$\mathscr{T}$の定理である.
\end{theo}




\mathstrut
\begin{theo}
\label{1atbvc1t1atb1v1atc1}%Thm43
$A$, $B$, $C$が$\mathscr{T}$の関係式ならば, 
\[
  (A \to B \vee C) \to (A \to B) \vee (A \to C)
\]
は$\mathscr{T}$の定理である.
\end{theo}




\mathstrut
\begin{theo}
\label{1atb1v1atc1t1atbvc1}%Thm44
$A$, $B$, $C$が$\mathscr{T}$の関係式ならば, 
\[
  (A \to B) \vee (A \to C) \to (A \to B \vee C)
\]
は$\mathscr{T}$の定理である.
\end{theo}




\mathstrut
\begin{theo}
\label{11atb1tb1tavb}%Thm45
$A$と$B$が$\mathscr{T}$の関係式ならば, 
\[
  ((A \to B) \to B) \to A \vee B
\]
は$\mathscr{T}$の定理である.
\end{theo}




\mathstrut
\begin{theo}
\label{avbt11atb1tb1}%Thm46
$A$と$B$が$\mathscr{T}$の関係式ならば, 
\[
  A \vee B \to ((A \to B) \to B)
\]
は$\mathscr{T}$の定理である.
\end{theo}




\mathstrut
\begin{theo}
\label{awbta}%Thm47
$A$と$B$が$\mathscr{T}$の関係式ならば, 
\[
  A \wedge B \to A, ~~
  A \wedge B \to B
\]
は共に$\mathscr{T}$の定理である.
\end{theo}




\mathstrut
\begin{theo}
\label{at1btawb1}%Thm48
$A$と$B$が$\mathscr{T}$の関係式ならば, 
\[
  A \to (B \to A \wedge B)
\]
は$\mathscr{T}$の定理である.
\end{theo}




\mathstrut
\begin{theo}
\label{1cta1t11ctb1t1ctawb11}%Thm49
$A$, $B$, $C$が$\mathscr{T}$の関係式ならば, 
\[
  (C \to A) \to ((C \to B) \to (C \to A \wedge B))
\]
は$\mathscr{T}$の定理である.
\end{theo}




\mathstrut
\begin{theo}
\label{1atb1t1atawb1}%Thm50
$A$と$B$が$\mathscr{T}$の関係式ならば, 
\[
  (A \to B) \to (A \to A \wedge B), ~~
  (B \to A) \to (B \to A \wedge B)
\]
は共に$\mathscr{T}$の定理である.
\end{theo}




\mathstrut
\begin{theo}
\label{1ct1atb11t1awctbwc1}%Thm51
$A$, $B$, $C$が$\mathscr{T}$の関係式ならば, 
\[
  (C \to (A \to B)) \to (A \wedge C \to B \wedge C), ~~
  (C \to (A \to B)) \to (C \wedge A \to C \wedge B)
\]
は共に$\mathscr{T}$の定理である.
\end{theo}




\mathstrut
\begin{theo}
\label{1awctbwc1t1ct1atb11}%Thm52
$A$, $B$, $C$が$\mathscr{T}$の関係式ならば, 
\[
  (A \wedge C \to B \wedge C) \to (C \to (A \to B)), ~~
  (C \wedge A \to C \wedge B) \to (C \to (A \to B))
\]
は共に$\mathscr{T}$の定理である.
\end{theo}




\mathstrut
\begin{theo}
\label{1atb1t1awctbwc1}%Thm53
$A$, $B$, $C$が$\mathscr{T}$の関係式ならば, 
\[
  (A \to B) \to (A \wedge C \to B \wedge C), ~~
  (A \to B) \to (C \wedge A \to C \wedge B)
\]
は共に$\mathscr{T}$の定理である.
\end{theo}




\mathstrut
\begin{theo}
\label{n1awna1}%Thm54
{\bf (矛盾律)}~
$A$が$\mathscr{T}$の関係式ならば, 
\[
  \neg (A \wedge \neg A)
\]
は$\mathscr{T}$の定理である.
\end{theo}




\mathstrut
\begin{theo}
\label{atawa}%Thm55
$A$が$\mathscr{T}$の関係式ならば, 
\[
  A \to A \wedge A
\]
は$\mathscr{T}$の定理である.
\end{theo}




\mathstrut
\begin{theo}
\label{awbtbwa}%Thm56
$A$と$B$が$\mathscr{T}$の関係式ならば, 
\[
  A \wedge B \to B \wedge A
\]
は$\mathscr{T}$の定理である.
\end{theo}




\mathstrut
\begin{theo}
\label{1awb1wctaw1bwc1}%Thm57
$A$, $B$, $C$が$\mathscr{T}$の関係式ならば, 
\[
  (A \wedge B) \wedge C \to A \wedge (B \wedge C)
\]
は$\mathscr{T}$の定理である.
\end{theo}




\mathstrut
\begin{theo}
\label{aw1bwc1t1awb1wc}%Thm58
$A$, $B$, $C$が$\mathscr{T}$の関係式ならば, 
\[
  A \wedge (B \wedge C) \to (A \wedge B) \wedge C
\]
は$\mathscr{T}$の定理である.
\end{theo}




\mathstrut
\begin{theo}
\label{aw1bwc1t1awb1w1awc1}%Thm59
$A$, $B$, $C$が$\mathscr{T}$の関係式ならば, 
\[
  A \wedge (B \wedge C) \to (A \wedge B) \wedge (A \wedge C), ~~
  (A \wedge B) \wedge C \to (A \wedge C) \wedge (B \wedge C)
\]
は共に$\mathscr{T}$の定理である.
\end{theo}




\mathstrut
\begin{theo}
\label{1awb1w1awc1taw1bwc1}%Thm60
$A$, $B$, $C$が$\mathscr{T}$の関係式ならば, 
\[
  (A \wedge B) \wedge (A \wedge C) \to A \wedge (B \wedge C), ~~
  (A \wedge C) \wedge (B \wedge C) \to (A \wedge B) \wedge C
\]
は共に$\mathscr{T}$の定理である.
\end{theo}




\mathstrut
\begin{theo}
\label{n1atb1tawnb}%Thm61
$A$と$B$が$\mathscr{T}$の関係式ならば, 
\[
  \neg (A \to B) \to A \wedge \neg B, ~~
  A \wedge \neg B \to \neg (A \to B)
\]
は共に$\mathscr{T}$の定理である.
\end{theo}




\mathstrut
\begin{theo}
\label{1at1btc11t1awbtc1}%Thm62
$A$, $B$, $C$が$\mathscr{T}$の関係式ならば, 
\[
  (A \to (B \to C)) \to (A \wedge B \to C)
\]
は$\mathscr{T}$の定理である.
\end{theo}




\mathstrut
\begin{theo}
\label{1awbtc1t1at1btc11}%Thm63
$A$, $B$, $C$が$\mathscr{T}$の関係式ならば, 
\[
  (A \wedge B \to C) \to (A \to (B \to C))
\]
は$\mathscr{T}$の定理である.
\end{theo}




\mathstrut
\begin{theo}
\label{1awctb1t1awctbwc1}%Thm64
$A$, $B$, $C$が$\mathscr{T}$の関係式ならば, 
\[
  (A \wedge C \to B) \to (A \wedge C \to B \wedge C), ~~
  (C \wedge A \to B) \to (C \wedge A \to C \wedge B)
\]
は共に$\mathscr{T}$の定理である.
\end{theo}




\mathstrut
\begin{theo}
\label{1awctbwc1t1awctb1}%Thm65
$A$, $B$, $C$が$\mathscr{T}$の関係式ならば, 
\[
  (A \wedge C \to B \wedge C) \to (A \wedge C \to B), ~~
  (C \wedge A \to C \wedge B) \to (C \wedge A \to B)
\]
は共に$\mathscr{T}$の定理である.
\end{theo}




\mathstrut
\begin{theo}
\label{1atbwc1t1atb1w1atc1}%Thm66
$A$, $B$, $C$が$\mathscr{T}$の関係式ならば, 
\[
  (A \to B \wedge C) \to (A \to B) \wedge (A \to C)
\]
は$\mathscr{T}$の定理である.
\end{theo}




\mathstrut
\begin{theo}
\label{1atb1w1atc1t1atbwc1}%Thm67
$A$, $B$, $C$が$\mathscr{T}$の関係式ならば, 
\[
  (A \to B) \wedge (A \to C) \to (A \to B \wedge C)
\]
は$\mathscr{T}$の定理である.
\end{theo}




\mathstrut
\begin{theo}
\label{aw1btc1t1btawc1}%Thm68
$A$, $B$, $C$が$\mathscr{T}$の関係式ならば, 
\[
  A \wedge (B \to C) \to (B \to A \wedge C)
\]
は$\mathscr{T}$の定理である.
\end{theo}




\mathstrut
\begin{theo}
\label{n1awb1tnavnb}%Thm69
$A$と$B$が$\mathscr{T}$の関係式ならば, 
\[
  \neg (A \wedge B) \to \neg A \vee \neg B, ~~
  \neg A \vee \neg B \to \neg (A \wedge B)
\]
は共に$\mathscr{T}$の定理である.
\end{theo}




\mathstrut
\begin{theo}
\label{n1avb1tnawnb}%Thm70
$A$と$B$が$\mathscr{T}$の関係式ならば, 
\[
  \neg (A \vee B) \to \neg A \wedge \neg B, ~~
  \neg A \wedge \neg B \to \neg (A \vee B)
\]
は共に$\mathscr{T}$の定理である.
\end{theo}




\mathstrut
\begin{theo}
\label{1avb1wata}%Thm71
$A$と$B$が$\mathscr{T}$の関係式ならば, 
\[
  (A \vee B) \wedge A \to A, ~~
  A \to (A \vee B) \wedge A
\]
は共に$\mathscr{T}$の定理である.
\end{theo}




\mathstrut
\begin{theo}
\label{1awb1vata}%Thm72
$A$と$B$が$\mathscr{T}$の関係式ならば, 
\[
  (A \wedge B) \vee A \to A, ~~
  A \to (A \wedge B) \vee A
\]
は共に$\mathscr{T}$の定理である.
\end{theo}




\mathstrut
\begin{theo}
\label{1avbtc1t1atc1w1btc1}%Thm73
$A$, $B$, $C$が$\mathscr{T}$の関係式ならば, 
\[
  (A \vee B \to C) \to (A \to C) \wedge (B \to C)
\]
は$\mathscr{T}$の定理である.
\end{theo}




\mathstrut
\begin{theo}
\label{1atc1w1btc1t1avbtc1}%Thm74
$A$, $B$, $C$が$\mathscr{T}$の関係式ならば, 
\[
  (A \to C) \wedge (B \to C) \to (A \vee B \to C)
\]
は$\mathscr{T}$の定理である.
\end{theo}




\mathstrut
\begin{theo}
\label{1awbtc1t1atc1v1btc1}%Thm75
$A$, $B$, $C$が$\mathscr{T}$の関係式ならば, 
\[
  (A \wedge B \to C) \to (A \to C) \vee (B \to C)
\]
は$\mathscr{T}$の定理である.
\end{theo}




\mathstrut
\begin{theo}
\label{1atc1v1btc1t1awbtc1}%Thm76
$A$, $B$, $C$が$\mathscr{T}$の関係式ならば, 
\[
  (A \to C) \vee (B \to C) \to (A \wedge B \to C)
\]
は$\mathscr{T}$の定理である.
\end{theo}




\mathstrut
\begin{theo}
\label{1atb1w1ctd1t1avctbvd1}%Thm77
$A$, $B$, $C$, $D$が$\mathscr{T}$の関係式ならば, 
\[
  (A \to B) \wedge (C \to D) \to (A \vee C \to B \vee D)
\]
は$\mathscr{T}$の定理である.
\end{theo}




\mathstrut
\begin{theo}
\label{1atb1w1ctd1t1awctbwd1}%Thm78
$A$, $B$, $C$, $D$が$\mathscr{T}$の関係式ならば, 
\[
  (A \to B) \wedge (C \to D) \to (A \wedge C \to B \wedge D)
\]
は$\mathscr{T}$の定理である.
\end{theo}




\mathstrut
\begin{theo}
\label{aw1bvc1t1awb1v1awc1}%Thm79
$A$, $B$, $C$が$\mathscr{T}$の関係式ならば, 
\[
  A \wedge (B \vee C) \to (A \wedge B) \vee (A \wedge C), ~~
  (A \vee B) \wedge C \to (A \wedge C) \vee (B \wedge C)
\]
は共に$\mathscr{T}$の定理である.
\end{theo}




\mathstrut
\begin{theo}
\label{1awb1v1awc1taw1bvc1}%Thm80
$A$, $B$, $C$が$\mathscr{T}$の関係式ならば, 
\[
  (A \wedge B) \vee (A \wedge C) \to A \wedge (B \vee C), ~~
  (A \wedge C) \vee (B \wedge C) \to (A \vee B) \wedge C
\]
は共に$\mathscr{T}$の定理である.
\end{theo}




\mathstrut
\begin{theo}
\label{av1bwc1t1avb1w1avc1}%Thm81
$A$, $B$, $C$が$\mathscr{T}$の関係式ならば, 
\[
  A \vee (B \wedge C) \to (A \vee B) \wedge (A \vee C), ~~
  (A \wedge B) \vee C \to (A \vee C) \wedge (B \vee C)
\]
は共に$\mathscr{T}$の定理である.
\end{theo}




\mathstrut
\begin{theo}
\label{1avb1w1avc1tav1bwc1}%Thm82
$A$, $B$, $C$が$\mathscr{T}$の関係式ならば, 
\[
  (A \vee B) \wedge (A \vee C) \to A \vee (B \wedge C), ~~
  (A \vee C) \wedge (B \vee C) \to (A \wedge B) \vee C
\]
は共に$\mathscr{T}$の定理である.
\end{theo}




\mathstrut
\begin{theo}
\label{thmgvee}%Thm83
$n$を自然数とし, $A_{1}, A_{2}, \cdots, A_{n}$を$\mathscr{T}$の関係式とする.
このとき$n$以下の任意の自然数$i$に対し, 
\[
  A_{i} \to A_{1} \vee A_{2} \vee \cdots \vee A_{n}
\]
は$\mathscr{T}$の定理である.
\end{theo}




\mathstrut
\begin{theo}
\label{thmgvee2}%Thm84
$n$を自然数とし, $A_{1}, A_{2}, \cdots, A_{n}$を$\mathscr{T}$の関係式とする.
また$k$を自然数とし, $i_{1}, i_{2}, \cdots, i_{k}$を$n$以下の自然数とする.
このとき
\[
  A_{i_{1}} \vee A_{i_{2}} \vee \cdots \vee A_{i_{k}} 
  \to A_{1} \vee A_{2} \vee \cdots \vee A_{n}
\]
は$\mathscr{T}$の定理である.
\end{theo}




\mathstrut
\begin{theo}
\label{thmgvgw}%Thm85
$n$を自然数とし, $A_{1}, A_{2}, \cdots, A_{n}$を$\mathscr{T}$の関係式とする.
このとき
\begin{align*}
  &A_{1} \wedge A_{2} \wedge \cdots \wedge A_{n} 
  \to \neg (\neg A_{1} \vee \neg A_{2} \vee \cdots \vee \neg A_{n}), \\
  \mbox{} \\
  &\neg (\neg A_{1} \vee \neg A_{2} \vee \cdots \vee \neg A_{n}) 
  \to A_{1} \wedge A_{2} \wedge \cdots \wedge A_{n}
\end{align*}
は共に$\mathscr{T}$の定理である.
\end{theo}




\mathstrut
\begin{theo}
\label{thmgwedge}%Thm86
$n$を自然数とし, $A_{1}, A_{2}, \cdots, A_{n}$を$\mathscr{T}$の関係式とする.
このとき$n$以下の任意の自然数$i$に対し, 
\[
  A_{1} \wedge A_{2} \wedge \cdots \wedge A_{n} \to A_{i}
\]
は$\mathscr{T}$の定理である.
\end{theo}




\mathstrut
\begin{theo}
\label{thmgwedge2}%Thm87
$n$を自然数とし, $A_{1}, A_{2}, \cdots, A_{n}$を$\mathscr{T}$の関係式とする.
また$k$を自然数とし, $i_{1}, i_{2}, \cdots, i_{k}$を$n$以下の自然数とする.
このとき
\[
  A_{1} \wedge A_{2} \wedge \cdots \wedge A_{n} 
  \to A_{i_{1}} \wedge A_{i_{2}} \wedge \cdots \wedge A_{i_{k}}
\]
は$\mathscr{T}$の定理である.
\end{theo}




\mathstrut
\begin{theo}
\label{thmgvloem}%Thm88
$n$を自然数とし, $A_{1}, A_{2}, \cdots, A_{n}$を$\mathscr{T}$の関係式とする.
いま$i$と$j$が互いに異なる$n$以下の自然数で, $A_{j}$が$\neg A_{i}$であるとする.
このとき
\[
  A_{1} \vee A_{2} \vee \cdots \vee A_{n}
\]
は$\mathscr{T}$の定理である.
\end{theo}




\mathstrut
\begin{theo}
\label{thmgwloc}%Thm89
$n$を自然数とし, $A_{1}, A_{2}, \cdots, A_{n}$を$\mathscr{T}$の関係式とする.
いま$i$と$j$が互いに異なる$n$以下の自然数で, $A_{j}$が$\neg A_{i}$であるとする.
このとき
\[
  \neg (A_{1} \wedge A_{2} \wedge \cdots \wedge A_{n})
\]
は$\mathscr{T}$の定理である.
\end{theo}




\mathstrut
\begin{theo}
\label{thmgvidempotent}%Thm90
$A$が$\mathscr{T}$の関係式ならば, 
\[
  \underbrace{A \vee A \vee \cdots \vee A}_{Aの個数は任意} \to A
\]
は$\mathscr{T}$の定理である.
\end{theo}




\mathstrut
\begin{theo}
\label{thmgwidempotent}%Thm91
$A$が$\mathscr{T}$の関係式ならば, 
\[
  A \to \underbrace{A \wedge A \wedge \cdots \wedge A}_{Aの個数は任意}
\]
は$\mathscr{T}$の定理である.
\end{theo}




\mathstrut
\begin{theo}
\label{thmgvch}%Thm92
$n$を自然数とし, $A_{1}, A_{2}, \cdots, A_{n}$を$\mathscr{T}$の関係式とする.
また自然数$1, 2, \cdots, n$の順序を任意に入れ替えたものを
$i_{1}, i_{2}, \cdots, i_{n}$とする.
このとき
\[
  A_{1} \vee A_{2} \vee \cdots \vee A_{n} 
  \to A_{i_{1}} \vee A_{i_{2}} \vee \cdots \vee A_{i_{n}}
\]
は$\mathscr{T}$の定理である.
\end{theo}




\mathstrut
\begin{theo}
\label{thmgwch}%Thm93
$n$を自然数とし, $A_{1}, A_{2}, \cdots, A_{n}$を$\mathscr{T}$の関係式とする.
また自然数$1, 2, \cdots, n$の順序を任意に入れ替えたものを
$i_{1}, i_{2}, \cdots, i_{n}$とする.
このとき
\[
  A_{1} \wedge A_{2} \wedge \cdots \wedge A_{n} 
  \to A_{i_{1}} \wedge A_{i_{2}} \wedge \cdots \wedge A_{i_{n}}
\]
は$\mathscr{T}$の定理である.
\end{theo}




\mathstrut
\begin{theo}
\label{thmgvass}%Thm94
$n$を自然数とし, $A_{1}, A_{2}, \cdots, A_{n}$を$\mathscr{T}$の関係式とする.
また$k$を$k < n$なる自然数とし, $i_{1}, i_{2}, \cdots, i_{k}$を
$i_{1} < i_{2} < \cdots < i_{k} < n$なる自然数とする.
同様に, $l$を$l < n$なる自然数とし, $j_{1}, j_{2}, \cdots, j_{l}$を
$j_{1} < j_{2} < \cdots < j_{l} < n$なる自然数とする.
このとき
\begin{multline*}
  (A_{1} \vee \cdots \vee A_{i_{1}}) \vee (A_{i_{1} + 1} \vee \cdots \vee A_{i_{2}}) \vee \cdots\cdots \vee (A_{i_{k} + 1} \vee \cdots \vee A_{n}) \\
  \to (A_{1} \vee \cdots \vee A_{j_{1}}) \vee (A_{j_{1} + 1} \vee \cdots \vee A_{j_{2}}) \vee \cdots\cdots \vee (A_{j_{l} + 1} \vee \cdots \vee A_{n})
\end{multline*}
は$\mathscr{T}$の定理である.
\end{theo}




\mathstrut
\begin{theo}
\label{thmgwass}%Thm95
$n$を自然数とし, $A_{1}, A_{2}, \cdots, A_{n}$を$\mathscr{T}$の関係式とする.
また$k$を$k < n$なる自然数とし, $i_{1}, i_{2}, \cdots, i_{k}$を
$i_{1} < i_{2} < \cdots < i_{k} < n$なる自然数とする.
同様に, $l$を$l < n$なる自然数とし, $j_{1}, j_{2}, \cdots, j_{l}$を
$j_{1} < j_{2} < \cdots < j_{l} < n$なる自然数とする.
このとき
\begin{multline*}
  (A_{1} \wedge \cdots \wedge A_{i_{1}}) \wedge (A_{i_{1} + 1} \wedge \cdots \wedge A_{i_{2}}) \wedge \cdots\cdots \wedge (A_{i_{k} + 1} \wedge \cdots \wedge A_{n}) \\
  \to (A_{1} \wedge \cdots \wedge A_{j_{1}}) \wedge (A_{j_{1} + 1} \wedge \cdots \wedge A_{j_{2}}) \wedge \cdots\cdots \wedge (A_{j_{l} + 1} \wedge \cdots \wedge A_{n})
\end{multline*}
は$\mathscr{T}$の定理である.
\end{theo}




\mathstrut
\begin{theo}
\label{thmgvdist1}%Thm96
$n$を自然数とし, $A_{1}, A_{2}, \cdots, A_{n}$を$\mathscr{T}$の関係式とする.
また$B$を$\mathscr{T}$の関係式とする.
このとき
\begin{align*}
  &B \vee (A_{1} \vee A_{2} \vee \cdots \vee A_{n}) 
  \to (B \vee A_{1}) \vee (B \vee A_{2}) \vee \cdots \vee (B \vee A_{n}), \\
  \mbox{} \\
  &(A_{1} \vee A_{2} \vee \cdots \vee A_{n}) \vee B 
  \to (A_{1} \vee B) \vee (A_{2} \vee B) \vee \cdots \vee (A_{n} \vee B)
\end{align*}
は共に$\mathscr{T}$の定理である.
\end{theo}




\mathstrut
\begin{theo}
\label{thmgvdist2}%Thm97
$n$を自然数とし, $A_{1}, A_{2}, \cdots, A_{n}$を$\mathscr{T}$の関係式とする.
また$B$を$\mathscr{T}$の関係式とする.
このとき
\begin{align*}
  &(B \vee A_{1}) \vee (B \vee A_{2}) \vee \cdots \vee (B \vee A_{n}) 
  \to B \vee (A_{1} \vee A_{2} \vee \cdots \vee A_{n}), \\
  \mbox{} \\
  &(A_{1} \vee B) \vee (A_{2} \vee B) \vee \cdots \vee (A_{n} \vee B) 
  \to (A_{1} \vee A_{2} \vee \cdots \vee A_{n}) \vee B
\end{align*}
は共に$\mathscr{T}$の定理である.
\end{theo}




\mathstrut
\begin{theo}
\label{thmgwdist1}%Thm98
$n$を自然数とし, $A_{1}, A_{2}, \cdots, A_{n}$を$\mathscr{T}$の関係式とする.
また$B$を$\mathscr{T}$の関係式とする.
このとき
\begin{align*}
  &B \wedge (A_{1} \wedge A_{2} \wedge \cdots \wedge A_{n}) 
  \to (B \wedge A_{1}) \wedge (B \wedge A_{2}) \wedge \cdots \wedge (B \wedge A_{n}), \\
  \mbox{} \\
  &(A_{1} \wedge A_{2} \wedge \cdots \wedge A_{n}) \wedge B 
  \to (A_{1} \wedge B) \wedge (A_{2} \wedge B) \wedge \cdots \wedge (A_{n} \wedge B)
\end{align*}
は共に$\mathscr{T}$の定理である.
\end{theo}




\mathstrut
\begin{theo}
\label{thmgwdist2}%Thm99
$n$を自然数とし, $A_{1}, A_{2}, \cdots, A_{n}$を$\mathscr{T}$の関係式とする.
また$B$を$\mathscr{T}$の関係式とする.
このとき
\begin{align*}
  &(B \wedge A_{1}) \wedge (B \wedge A_{2}) \wedge \cdots \wedge (B \wedge A_{n}) 
  \to B \wedge (A_{1} \wedge A_{2} \wedge \cdots \wedge A_{n}), \\
  \mbox{} \\
  &(A_{1} \wedge B) \wedge (A_{2} \wedge B) \wedge \cdots \wedge (A_{n} \wedge B) 
  \to (A_{1} \wedge A_{2} \wedge \cdots \wedge A_{n}) \wedge B
\end{align*}
は共に$\mathscr{T}$の定理である.
\end{theo}




\mathstrut
\begin{theo}
\label{thmwgvdist1}%Thm100
$n$を自然数とし, $A_{1}, A_{2}, \cdots, A_{n}$を$\mathscr{T}$の関係式とする.
また$B$を$\mathscr{T}$の関係式とする.
このとき
\begin{align*}
  &B \wedge (A_{1} \vee A_{2} \vee \cdots \vee A_{n}) 
  \to (B \wedge A_{1}) \vee (B \wedge A_{2}) \vee \cdots \vee (B \wedge A_{n}), \\
  \mbox{} \\
  &(A_{1} \vee A_{2} \vee \cdots \vee A_{n}) \wedge B 
  \to (A_{1} \wedge B) \vee (A_{2} \wedge B) \vee \cdots \vee (A_{n} \wedge B)
\end{align*}
は共に$\mathscr{T}$の定理である.
\end{theo}




\mathstrut
\begin{theo}
\label{thmwgvdist2}%Thm101
$n$を自然数とし, $A_{1}, A_{2}, \cdots, A_{n}$を$\mathscr{T}$の関係式とする.
また$B$を$\mathscr{T}$の関係式とする.
このとき
\begin{align*}
  &(B \wedge A_{1}) \vee (B \wedge A_{2}) \vee \cdots \vee (B \wedge A_{n}) 
  \to B \wedge (A_{1} \vee A_{2} \vee \cdots \vee A_{n}), \\
  \mbox{} \\
  &(A_{1} \wedge B) \vee (A_{2} \wedge B) \vee \cdots \vee (A_{n} \wedge B) 
  \to (A_{1} \vee A_{2} \vee \cdots \vee A_{n}) \wedge B
\end{align*}
は共に$\mathscr{T}$の定理である.
\end{theo}




\mathstrut
\begin{theo}
\label{thmvgwdist1}%Thm102
$n$を自然数とし, $A_{1}, A_{2}, \cdots, A_{n}$を$\mathscr{T}$の関係式とする.
また$B$を$\mathscr{T}$の関係式とする.
このとき
\begin{align*}
  &B \vee (A_{1} \wedge A_{2} \wedge \cdots \wedge A_{n}) 
  \to (B \vee A_{1}) \wedge (B \vee A_{2}) \wedge \cdots \wedge (B \vee A_{n}), \\
  \mbox{} \\
  &(A_{1} \wedge A_{2} \wedge \cdots \wedge A_{n}) \vee B 
  \to (A_{1} \vee B) \wedge (A_{2} \vee B) \wedge \cdots \wedge (A_{n} \vee B)
\end{align*}
は共に$\mathscr{T}$の定理である.
\end{theo}




\mathstrut
\begin{theo}
\label{thmvgwdist2}%Thm103
$n$を自然数とし, $A_{1}, A_{2}, \cdots, A_{n}$を$\mathscr{T}$の関係式とする.
また$B$を$\mathscr{T}$の関係式とする.
このとき
\begin{align*}
  &(B \vee A_{1}) \wedge (B \vee A_{2}) \wedge \cdots \wedge (B \vee A_{n}) 
  \to B \vee (A_{1} \wedge A_{2} \wedge \cdots \wedge A_{n}), \\
  \mbox{} \\
  &(A_{1} \vee B) \wedge (A_{2} \vee B) \wedge \cdots \wedge (A_{n} \vee B) 
  \to (A_{1} \wedge A_{2} \wedge \cdots \wedge A_{n}) \vee B
\end{align*}
は共に$\mathscr{T}$の定理である.
\end{theo}




\mathstrut
\begin{theo}
\label{thmtgvdist}%Thm104
$n$を自然数とし, $A_{1}, A_{2}, \cdots, A_{n}$を$\mathscr{T}$の関係式とする.
また$B$を$\mathscr{T}$の関係式とする.
このとき
\begin{align*}
  &(B \to A_{1} \vee A_{2} \vee \cdots \vee A_{n}) 
  \to (B \to A_{1}) \vee (B \to A_{2}) \vee \cdots \vee (B \to A_{n}), \\
  \mbox{} \\
  &(B \to A_{1}) \vee (B \to A_{2}) \vee \cdots \vee (B \to A_{n}) 
  \to (B \to A_{1} \vee A_{2} \vee \cdots \vee A_{n})
\end{align*}
は共に$\mathscr{T}$の定理である.
\end{theo}




\mathstrut
\begin{theo}
\label{thmtgwdist}%Thm105
$n$を自然数とし, $A_{1}, A_{2}, \cdots, A_{n}$を$\mathscr{T}$の関係式とする.
また$B$を$\mathscr{T}$の関係式とする.
このとき
\begin{align*}
  &(B \to A_{1} \wedge A_{2} \wedge \cdots \wedge A_{n}) 
  \to (B \to A_{1}) \wedge (B \to A_{2}) \wedge \cdots \wedge (B \to A_{n}), \\
  \mbox{} \\
  &(B \to A_{1}) \wedge (B \to A_{2}) \wedge \cdots \wedge (B \to A_{n}) 
  \to (B \to A_{1} \wedge A_{2} \wedge \cdots \wedge A_{n})
\end{align*}
は共に$\mathscr{T}$の定理である.
\end{theo}




\mathstrut
\begin{theo}
\label{thmgvtgwdist}%Thm106
$n$を自然数とし, $A_{1}, A_{2}, \cdots, A_{n}$を$\mathscr{T}$の関係式とする.
また$B$を$\mathscr{T}$の関係式とする.
このとき
\begin{align*}
  &(A_{1} \vee A_{2} \vee \cdots \vee A_{n} \to B) 
  \to (A_{1} \to B) \wedge (A_{2} \to B) \wedge \cdots \wedge (A_{n} \to B), \\
  \mbox{} \\
  &(A_{1} \to B) \wedge (A_{2} \to B) \wedge \cdots \wedge (A_{n} \to B) 
  \to (A_{1} \vee A_{2} \vee \cdots \vee A_{n} \to B)
\end{align*}
は共に$\mathscr{T}$の定理である.
\end{theo}




\mathstrut
\begin{theo}
\label{thmgwtgvdist}%Thm107
$n$を自然数とし, $A_{1}, A_{2}, \cdots, A_{n}$を$\mathscr{T}$の関係式とする.
また$B$を$\mathscr{T}$の関係式とする.
このとき
\begin{align*}
  &(A_{1} \wedge A_{2} \wedge \cdots \wedge A_{n} \to B) 
  \to (A_{1} \to B) \vee (A_{2} \to B) \vee \cdots \vee (A_{n} \to B), \\
  \mbox{} \\
  &(A_{1} \to B) \vee (A_{2} \to B) \vee \cdots \vee (A_{n} \to B) 
  \to (A_{1} \wedge A_{2} \wedge \cdots \wedge A_{n} \to B)
\end{align*}
は共に$\mathscr{T}$の定理である.
\end{theo}




\mathstrut
\begin{theo}
\label{thmgvdemorgan}%Thm108
$n$を自然数とし, $A_{1}, A_{2}, \cdots, A_{n}$を$\mathscr{T}$の関係式とする.
このとき
\begin{align*}
  &\neg (A_{1} \vee A_{2} \vee \cdots \vee A_{n}) 
  \to \neg A_{1} \wedge \neg A_{2} \wedge \cdots \wedge \neg A_{n}, \\
  \mbox{} \\
  &\neg A_{1} \wedge \neg A_{2} \wedge \cdots \wedge \neg A_{n} 
  \to \neg (A_{1} \vee A_{2} \vee \cdots \vee A_{n})
\end{align*}
は共に$\mathscr{T}$の定理である.
\end{theo}




\mathstrut
\begin{theo}
\label{thmgwdemorgan}%Thm109
$n$を自然数とし, $A_{1}, A_{2}, \cdots, A_{n}$を$\mathscr{T}$の関係式とする.
このとき
\begin{align*}
  &\neg (A_{1} \wedge A_{2} \wedge \cdots \wedge A_{n}) 
  \to \neg A_{1} \vee \neg A_{2} \vee \cdots \vee \neg A_{n}, \\
  \mbox{} \\
  &\neg A_{1} \vee \neg A_{2} \vee \cdots \vee \neg A_{n} 
  \to \neg (A_{1} \wedge A_{2} \wedge \cdots \wedge A_{n})
\end{align*}
は共に$\mathscr{T}$の定理である.
\end{theo}




\mathstrut
\begin{theo}
\label{thmgvabs}%Thm110
$n$を自然数とし, $A_{1}, A_{2}, \cdots, A_{n}$を$\mathscr{T}$の関係式とする.
また$i$を$n$以下の自然数とする.
このとき
\begin{align*}
  &(A_{1} \vee A_{2} \vee \cdots \vee A_{n}) \wedge A_{i} \to A_{i}, \\
  \mbox{} \\
  &A_{i} \to (A_{1} \vee A_{2} \vee \cdots \vee A_{n}) \wedge A_{i}
\end{align*}
は共に$\mathscr{T}$の定理である.
\end{theo}




\mathstrut
\begin{theo}
\label{thmgwabs}%Thm111
$n$を自然数とし, $A_{1}, A_{2}, \cdots, A_{n}$を$\mathscr{T}$の関係式とする.
また$i$を$n$以下の自然数とする.
このとき
\begin{align*}
  &(A_{1} \wedge A_{2} \wedge \cdots \wedge A_{n}) \vee A_{i} \to A_{i}, \\
  \mbox{} \\
  &A_{i} \to (A_{1} \wedge A_{2} \wedge \cdots \wedge A_{n}) \vee A_{i}
\end{align*}
は共に$\mathscr{T}$の定理である.
\end{theo}




\mathstrut
\begin{theo}
\label{thmfromaddgv}%Thm112
$n$を自然数とし, $A_{1}, A_{2}, \cdots, A_{n}, B_{1}, B_{2}, \cdots, B_{n}$を$\mathscr{T}$の関係式とする.
このとき
\[
  (A_{1} \to B_{1}) \wedge (A_{2} \to B_{2}) \wedge \cdots \wedge (A_{n} \to B_{n}) 
  \to (A_{1} \vee A_{2} \vee \cdots \vee A_{n} \to B_{1} \vee B_{2} \vee \cdots \vee B_{n})
\]
は$\mathscr{T}$の定理である.
\end{theo}




\mathstrut
\begin{theo}
\label{thmfromaddgw}%Thm113
$n$を自然数とし, $A_{1}, A_{2}, \cdots, A_{n}, B_{1}, B_{2}, \cdots, B_{n}$を$\mathscr{T}$の関係式とする.
このとき
\[
  (A_{1} \to B_{1}) \wedge (A_{2} \to B_{2}) \wedge \cdots \wedge (A_{n} \to B_{n}) 
  \to (A_{1} \wedge A_{2} \wedge \cdots \wedge A_{n} \to B_{1} \wedge B_{2} \wedge \cdots \wedge B_{n})
\]
は$\mathscr{T}$の定理である.
\end{theo}




\mathstrut
\begin{theo}
\label{ala}%Thm114
{\bf (反射律)}~
$A$が$\mathscr{T}$の関係式ならば, 
\[
  A \leftrightarrow A
\]
は$\mathscr{T}$の定理である.
\end{theo}




\mathstrut
\begin{theo}
\label{1alb1t1bla1}%Thm115
$A$と$B$が$\mathscr{T}$の関係式ならば, 
\[
  (A \leftrightarrow B) \to (B \leftrightarrow A)
\]
は$\mathscr{T}$の定理である.
\end{theo}




\mathstrut
\begin{theo}
\label{1alb1l1bla1}%Thm116
$A$と$B$が$\mathscr{T}$の関係式ならば, 
\[
  (A \leftrightarrow B) \leftrightarrow (B \leftrightarrow A)
\]
は$\mathscr{T}$の定理である.
\end{theo}




\mathstrut
\begin{theo}
\label{1alb1w1blc1t1alc1}%Thm117
$A$, $B$, $C$が$\mathscr{T}$の関係式ならば, 
\[
  (A \leftrightarrow B) \wedge (B \leftrightarrow C) \to (A \leftrightarrow C)
\]
は$\mathscr{T}$の定理である.
\end{theo}




\mathstrut
\begin{theo}
\label{1at1btc11l1bt1atc11}%Thm118
$A$, $B$, $C$が$\mathscr{T}$の関係式ならば, 
\[
  (A \to (B \to C)) \leftrightarrow (B \to (A \to C))
\]
は$\mathscr{T}$の定理である.
\end{theo}




\mathstrut
\begin{theo}
\label{1atb1l1at1atb11}%Thm119
$A$と$B$が$\mathscr{T}$の関係式ならば, 
\[
  (A \to B) \leftrightarrow (A \to (A \to B))
\]
は$\mathscr{T}$の定理である.
\end{theo}




\mathstrut
\begin{theo}
\label{1at1btc11l11atb1t1atc11}%Thm120
$A$, $B$, $C$が$\mathscr{T}$の関係式ならば, 
\[
  (A \to (B \to C)) \leftrightarrow ((A \to B) \to (A \to C))
\]
は$\mathscr{T}$の定理である.
\end{theo}




\mathstrut
\begin{theo}
\label{nnala}%Thm121
$A$が$\mathscr{T}$の関係式ならば, 
\[
  \neg \neg A \leftrightarrow A
\]
は$\mathscr{T}$の定理である.
\end{theo}




\mathstrut
\begin{theo}
\label{1atb1l1nbtna1}%Thm122
$A$と$B$が$\mathscr{T}$の関係式ならば, 
\[
  (A \to B) \leftrightarrow (\neg B \to \neg A), ~~
  (\neg A \to B) \leftrightarrow (\neg B \to A), ~~
  (A \to \neg B) \leftrightarrow (B \to \neg A)
\]
はいずれも$\mathscr{T}$の定理である.
\end{theo}




\mathstrut
\begin{theo}
\label{1nat1btc11l11cta1t1bta11}%Thm123
$A$, $B$, $C$が$\mathscr{T}$の関係式ならば, 
\[
  (\neg A \to (B \to C)) \leftrightarrow ((C \to A) \to (B \to A))
\]
は$\mathscr{T}$の定理である.
\end{theo}




\mathstrut
\begin{theo}
\label{nal1atna1}%Thm124
$A$が$\mathscr{T}$の関係式ならば, 
\[
  \neg A \leftrightarrow (A \to \neg A), ~~
  A \leftrightarrow (\neg A \to A)
\]
は共に$\mathscr{T}$の定理である.
\end{theo}




\mathstrut
\begin{theo}
\label{al11atb1ta1}%Thm125
$A$と$B$が$\mathscr{T}$の関係式ならば, 
\[
  A \leftrightarrow ((A \to B) \to A)
\]
は$\mathscr{T}$の定理である.
\end{theo}




\mathstrut
\begin{theo}
\label{1atb1l1avbtb1}%Thm126
$A$と$B$が$\mathscr{T}$の関係式ならば, 
\[
  (A \to B) \leftrightarrow (A \vee B \to B), ~~
  (B \to A) \leftrightarrow (A \vee B \to A)
\]
は共に$\mathscr{T}$の定理である.
\end{theo}




\mathstrut
\begin{theo}
\label{1atb1l1avblb1}%Thm127
$A$と$B$が$\mathscr{T}$の関係式ならば, 
\[
  (A \to B) \leftrightarrow (A \vee B \leftrightarrow B), ~~
  (B \to A) \leftrightarrow (A \vee B \leftrightarrow A)
\]
は共に$\mathscr{T}$の定理である.
\end{theo}




\mathstrut
\begin{theo}
\label{1atb1vcl1avctbvc1}%Thm128
$A$, $B$, $C$が$\mathscr{T}$の関係式ならば, 
\begin{align*}
  (A \to B) \vee C \leftrightarrow (A \vee C \to B \vee C)&, ~~
  (A \to B) \vee C \leftrightarrow (C \vee A \to C \vee B), \\
  \mbox{} \\
  C \vee (A \to B) \leftrightarrow (A \vee C \to B \vee C)&, ~~
  C \vee (A \to B) \leftrightarrow (C \vee A \to C \vee B)
\end{align*}
はいずれも$\mathscr{T}$の定理である.
\end{theo}




\mathstrut
\begin{theo}
\label{avala}%Thm129
{\bf (論理和の冪等律)}~
$A$が$\mathscr{T}$の関係式ならば, 
\[
  A \vee A \leftrightarrow A
\]
は$\mathscr{T}$の定理である.
\end{theo}




\mathstrut
\begin{theo}
\label{avblbva}%Thm130
{\bf (論理和の交換律)}~
$A$と$B$が$\mathscr{T}$の関係式ならば, 
\[
  A \vee B \leftrightarrow B \vee A
\]
は$\mathscr{T}$の定理である.
\end{theo}




\mathstrut
\begin{theo}
\label{1avb1vclav1bvc1}%Thm131
{\bf (論理和の結合律)}~
$A$, $B$, $C$が$\mathscr{T}$の関係式ならば, 
\[
  (A \vee B) \vee C \leftrightarrow A \vee (B \vee C)
\]
は$\mathscr{T}$の定理である.
\end{theo}




\mathstrut
\begin{theo}
\label{av1bvc1l1avb1v1avc1}%Thm132
$A$, $B$, $C$が$\mathscr{T}$の関係式ならば, 
\[
  A \vee (B \vee C) \leftrightarrow (A \vee B) \vee (A \vee C), ~~
  (A \vee B) \vee C \leftrightarrow (A \vee C) \vee (B \vee C)
\]
は共に$\mathscr{T}$の定理である.
\end{theo}




\mathstrut
\begin{theo}
\label{1atb1lnavb}%Thm133
$A$と$B$が$\mathscr{T}$の関係式ならば, 
\[
  (A \to B) \leftrightarrow \neg A \vee B
\]
は$\mathscr{T}$の定理である.
\end{theo}




\mathstrut
\begin{theo}
\label{1atb1vcl1atbvc1}%Thm134
$A$, $B$, $C$が$\mathscr{T}$の関係式ならば, 
\[
  (A \to B) \vee C \leftrightarrow (A \to B \vee C), ~~
  C \vee (A \to B) \leftrightarrow (A \to C \vee B)
\]
は共に$\mathscr{T}$の定理である.
\end{theo}




\mathstrut
\begin{theo}
\label{1atbvc1l1avctbvc1}%Thm135
$A$, $B$, $C$が$\mathscr{T}$の関係式ならば, 
\[
  (A \to B \vee C) \leftrightarrow (A \vee C \to B \vee C), ~~
  (A \to C \vee B) \leftrightarrow (C \vee A \to C \vee B)
\]
は共に$\mathscr{T}$の定理である.
\end{theo}




\mathstrut
\begin{theo}
\label{1atbvc1l1atb1v1atc1}%Thm136
$A$, $B$, $C$が$\mathscr{T}$の関係式ならば, 
\[
  (A \to B \vee C) \leftrightarrow (A \to B) \vee (A \to C)
\]
は$\mathscr{T}$の定理である.
\end{theo}




\mathstrut
\begin{theo}
\label{11atb1tb1lavb}%Thm137
$A$と$B$が$\mathscr{T}$の関係式ならば, 
\[
  ((A \to B) \to B) \leftrightarrow A \vee B
\]
は$\mathscr{T}$の定理である.
\end{theo}




\mathstrut
\begin{theo}
\label{11atb1lb1lavb}%Thm138
$A$と$B$が$\mathscr{T}$の関係式ならば, 
\[
  ((A \to B) \leftrightarrow B) \leftrightarrow A \vee B
\]
は$\mathscr{T}$の定理である.
\end{theo}




\mathstrut
\begin{theo}
\label{1atb1l1atawb1}%Thm139
$A$と$B$が$\mathscr{T}$の関係式ならば, 
\[
  (A \to B) \leftrightarrow (A \to A \wedge B), ~~
  (B \to A) \leftrightarrow (B \to A \wedge B)
\]
は共に$\mathscr{T}$の定理である.
\end{theo}




\mathstrut
\begin{theo}
\label{1atb1l1awbla1}%Thm140
$A$と$B$が$\mathscr{T}$の関係式ならば, 
\[
  (A \to B) \leftrightarrow (A \wedge B \leftrightarrow A), ~~
  (B \to A) \leftrightarrow (A \wedge B \leftrightarrow B)
\]
は共に$\mathscr{T}$の定理である.
\end{theo}




\mathstrut
\begin{theo}
\label{1ct1atb11l1awctbwc1}%Thm141
$A$, $B$, $C$が$\mathscr{T}$の関係式ならば, 
\[
  (C \to (A \to B)) \leftrightarrow (A \wedge C \to B \wedge C), ~~
  (C \to (A \to B)) \leftrightarrow (C \wedge A \to C \wedge B)
\]
は共に$\mathscr{T}$の定理である.
\end{theo}




\mathstrut
\begin{theo}
\label{awala}%Thm142
{\bf (論理積の冪等律)}~
$A$が$\mathscr{T}$の関係式ならば, 
\[
  A \wedge A \leftrightarrow A
\]
は$\mathscr{T}$の定理である.
\end{theo}




\mathstrut
\begin{theo}
\label{awblbwa}%Thm143
{\bf (論理積の交換律)}~
$A$と$B$が$\mathscr{T}$の関係式ならば, 
\[
  A \wedge B \leftrightarrow B \wedge A
\]
は$\mathscr{T}$の定理である.
\end{theo}




\mathstrut
\begin{theo}
\label{1awb1wclaw1bwc1}%Thm144
{\bf (論理積の結合律)}~
$A$, $B$, $C$が$\mathscr{T}$の関係式ならば, 
\[
  (A \wedge B) \wedge C \leftrightarrow A \wedge (B \wedge C)
\]
は$\mathscr{T}$の定理である.
\end{theo}




\mathstrut
\begin{theo}
\label{aw1bwc1l1awb1w1awc1}%Thm145
$A$, $B$, $C$が$\mathscr{T}$の関係式ならば, 
\[
  A \wedge (B \wedge C) \leftrightarrow (A \wedge B) \wedge (A \wedge C), ~~
  (A \wedge B) \wedge C \leftrightarrow (A \wedge C) \wedge (B \wedge C)
\]
は共に$\mathscr{T}$の定理である.
\end{theo}




\mathstrut
\begin{theo}
\label{n1atb1lawnb}%Thm146
$A$と$B$が$\mathscr{T}$の関係式ならば, 
\[
  \neg (A \to B) \leftrightarrow A \wedge \neg B
\]
は$\mathscr{T}$の定理である.
\end{theo}




\mathstrut
\begin{theo}
\label{1at1btc11l1awbtc1}%Thm147
$A$, $B$, $C$が$\mathscr{T}$の関係式ならば, 
\[
  (A \to (B \to C)) \leftrightarrow (A \wedge B \to C)
\]
は$\mathscr{T}$の定理である.
\end{theo}




\mathstrut
\begin{theo}
\label{1awctb1l1awctbwc1}%Thm148
$A$, $B$, $C$が$\mathscr{T}$の関係式ならば, 
\[
  (A \wedge C \to B) \leftrightarrow (A \wedge C \to B \wedge C), ~~
  (C \wedge A \to B) \leftrightarrow (C \wedge A \to C \wedge B)
\]
は共に$\mathscr{T}$の定理である.
\end{theo}




\mathstrut
\begin{theo}
\label{1atbwc1l1atb1w1atc1}%Thm149
$A$, $B$, $C$が$\mathscr{T}$の関係式ならば, 
\[
  (A \to B \wedge C) \leftrightarrow (A \to B) \wedge (A \to C)
\]
は$\mathscr{T}$の定理である.
\end{theo}




\mathstrut
\begin{theo}
\label{n1awb1lnavnb}%Thm150
{\bf (de Morganの法則)}~
$A$と$B$が$\mathscr{T}$の関係式ならば, 
\[
  \neg (A \wedge B) \leftrightarrow \neg A \vee \neg B, ~~
  \neg (A \vee B) \leftrightarrow \neg A \wedge \neg B
\]
は共に$\mathscr{T}$の定理である.
\end{theo}




\mathstrut
\begin{theo}
\label{1avb1wala}%Thm151
{\bf (吸収律)}~
$A$と$B$が$\mathscr{T}$の関係式ならば, 
\[
  (A \vee B) \wedge A \leftrightarrow A, ~~
  (A \wedge B) \vee A \leftrightarrow A
\]
は共に$\mathscr{T}$の定理である.
\end{theo}




\mathstrut
\begin{theo}
\label{1avbtc1l1atc1w1btc1}%Thm152
$A$, $B$, $C$が$\mathscr{T}$の関係式ならば, 
\[
  (A \vee B \to C) \leftrightarrow (A \to C) \wedge (B \to C)
\]
は$\mathscr{T}$の定理である.
\end{theo}




\mathstrut
\begin{theo}
\label{1awbtc1l1atc1v1btc1}%Thm153
$A$, $B$, $C$が$\mathscr{T}$の関係式ならば, 
\[
  (A \wedge B \to C) \leftrightarrow (A \to C) \vee (B \to C)
\]
は$\mathscr{T}$の定理である.
\end{theo}




\mathstrut
\begin{theo}
\label{aw1bvc1l1awb1v1awc1}%Thm154
{\bf (分配律)}~
$A$, $B$, $C$が$\mathscr{T}$の関係式ならば, 
\begin{align*}
  &A \wedge (B \vee C) \leftrightarrow (A \wedge B) \vee (A \wedge C), ~~
  (A \vee B) \wedge C \leftrightarrow (A \wedge C) \vee (B \wedge C), \\
  \mbox{} \\
  &A \vee (B \wedge C) \leftrightarrow (A \vee B) \wedge (A \vee C), ~~
  (A \wedge B) \vee C \leftrightarrow (A \vee C) \wedge (B \vee C)
\end{align*}
はいずれも$\mathscr{T}$の定理である.
\end{theo}




\mathstrut
\begin{theo}
\label{thmgvgweq}%Thm155
$n$を自然数とし, $A_{1}, A_{2}, \cdots, A_{n}$を$\mathscr{T}$の関係式とする.
このとき
\[
  A_{1} \wedge A_{2} \wedge \cdots \wedge A_{n} 
  \leftrightarrow \neg (\neg A_{1} \vee \neg A_{2} \vee \cdots \vee \neg A_{n})
\]
は$\mathscr{T}$の定理である.
\end{theo}




\mathstrut
\begin{theo}
\label{thmgvidempotenteq}%Thm156
$A$が$\mathscr{T}$の関係式ならば, 
\[
  \underbrace{A \vee A \vee \cdots \vee A}_{Aの個数は任意} \leftrightarrow A
\]
は$\mathscr{T}$の定理である.
\end{theo}




\mathstrut
\begin{theo}
\label{thmgwidempotenteq}%Thm157
$A$が$\mathscr{T}$の関係式ならば, 
\[
  \underbrace{A \wedge A \wedge \cdots \wedge A}_{Aの個数は任意} \leftrightarrow A
\]
は$\mathscr{T}$の定理である.
\end{theo}




\mathstrut
\begin{theo}
\label{thmgvcheq}%Thm158
$n$を自然数とし, $A_{1}, A_{2}, \cdots, A_{n}$を$\mathscr{T}$の関係式とする.
また自然数$1, 2, \cdots, n$の順序を任意に入れ替えたものを
$i_{1}, i_{2}, \cdots, i_{n}$とする.
このとき
\[
  A_{1} \vee A_{2} \vee \cdots \vee A_{n} 
  \leftrightarrow A_{i_{1}} \vee A_{i_{2}} \vee \cdots \vee A_{i_{n}}
\]
は$\mathscr{T}$の定理である.
\end{theo}




\mathstrut
\begin{theo}
\label{thmgwcheq}%Thm159
$n$を自然数とし, $A_{1}, A_{2}, \cdots, A_{n}$を$\mathscr{T}$の関係式とする.
また自然数$1, 2, \cdots, n$の順序を任意に入れ替えたものを
$i_{1}, i_{2}, \cdots, i_{n}$とする.
このとき
\[
  A_{1} \wedge A_{2} \wedge \cdots \wedge A_{n} 
  \leftrightarrow A_{i_{1}} \wedge A_{i_{2}} \wedge \cdots \wedge A_{i_{n}}
\]
は$\mathscr{T}$の定理である.
\end{theo}




\mathstrut
\begin{theo}
\label{thmgvasseq}%Thm160
$n$を自然数とし, $A_{1}, A_{2}, \cdots, A_{n}$を$\mathscr{T}$の関係式とする.
また$k$を$k < n$なる自然数とし, $i_{1}, i_{2}, \cdots, i_{k}$を
$i_{1} < i_{2} < \cdots < i_{k} < n$なる自然数とする.
同様に, $l$を$l < n$なる自然数とし, $j_{1}, j_{2}, \cdots, j_{l}$を
$j_{1} < j_{2} < \cdots < j_{l} < n$なる自然数とする.
このとき
\begin{multline*}
  (A_{1} \vee \cdots \vee A_{i_{1}}) \vee (A_{i_{1} + 1} \vee \cdots \vee A_{i_{2}}) \vee \cdots\cdots \vee (A_{i_{k} + 1} \vee \cdots \vee A_{n}) \\
  \leftrightarrow (A_{1} \vee \cdots \vee A_{j_{1}}) \vee (A_{j_{1} + 1} \vee \cdots \vee A_{j_{2}}) \vee \cdots\cdots \vee (A_{j_{l} + 1} \vee \cdots \vee A_{n})
\end{multline*}
は$\mathscr{T}$の定理である.
\end{theo}




\mathstrut
\begin{theo}
\label{thmgwasseq}%Thm161
$n$を自然数とし, $A_{1}, A_{2}, \cdots, A_{n}$を$\mathscr{T}$の関係式とする.
また$k$を$k < n$なる自然数とし, $i_{1}, i_{2}, \cdots, i_{k}$を
$i_{1} < i_{2} < \cdots < i_{k} < n$なる自然数とする.
同様に, $l$を$l < n$なる自然数とし, $j_{1}, j_{2}, \cdots, j_{l}$を
$j_{1} < j_{2} < \cdots < j_{l} < n$なる自然数とする.
このとき
\begin{multline*}
  (A_{1} \wedge \cdots \wedge A_{i_{1}}) \wedge (A_{i_{1} + 1} \wedge \cdots \wedge A_{i_{2}}) \wedge \cdots\cdots \wedge (A_{i_{k} + 1} \wedge \cdots \wedge A_{n}) \\
  \leftrightarrow (A_{1} \wedge \cdots \wedge A_{j_{1}}) \wedge (A_{j_{1} + 1} \wedge \cdots \wedge A_{j_{2}}) \wedge \cdots\cdots \wedge (A_{j_{l} + 1} \wedge \cdots \wedge A_{n})
\end{multline*}
は$\mathscr{T}$の定理である.
\end{theo}




\mathstrut
\begin{theo}
\label{thmgvdisteq}%Thm162
$n$を自然数とし, $A_{1}, A_{2}, \cdots, A_{n}$を$\mathscr{T}$の関係式とする.
また$B$を$\mathscr{T}$の関係式とする.
このとき
\begin{align*}
  &B \vee (A_{1} \vee A_{2} \vee \cdots \vee A_{n}) 
  \leftrightarrow (B \vee A_{1}) \vee (B \vee A_{2}) \vee \cdots \vee (B \vee A_{n}), \\
  \mbox{} \\
  &(A_{1} \vee A_{2} \vee \cdots \vee A_{n}) \vee B 
  \leftrightarrow (A_{1} \vee B) \vee (A_{2} \vee B) \vee \cdots \vee (A_{n} \vee B)
\end{align*}
は共に$\mathscr{T}$の定理である.
\end{theo}




\mathstrut
\begin{theo}
\label{thmgwdisteq}%Thm163
$n$を自然数とし, $A_{1}, A_{2}, \cdots, A_{n}$を$\mathscr{T}$の関係式とする.
また$B$を$\mathscr{T}$の関係式とする.
このとき
\begin{align*}
  &B \wedge (A_{1} \wedge A_{2} \wedge \cdots \wedge A_{n}) 
  \leftrightarrow (B \wedge A_{1}) \wedge (B \wedge A_{2}) \wedge \cdots \wedge (B \wedge A_{n}), \\
  \mbox{} \\
  &(A_{1} \wedge A_{2} \wedge \cdots \wedge A_{n}) \wedge B 
  \leftrightarrow (A_{1} \wedge B) \wedge (A_{2} \wedge B) \wedge \cdots \wedge (A_{n} \wedge B)
\end{align*}
は共に$\mathscr{T}$の定理である.
\end{theo}




\mathstrut
\begin{theo}
\label{thmgdisteq}%Thm164
$n$を自然数とし, $A_{1}, A_{2}, \cdots, A_{n}$を$\mathscr{T}$の関係式とする.
また$B$を$\mathscr{T}$の関係式とする.
このとき
\begin{align*}
  &B \wedge (A_{1} \vee A_{2} \vee \cdots \vee A_{n}) 
  \leftrightarrow (B \wedge A_{1}) \vee (B \wedge A_{2}) \vee \cdots \vee (B \wedge A_{n}), \\
  \mbox{} \\
  &(A_{1} \vee A_{2} \vee \cdots \vee A_{n}) \wedge B 
  \leftrightarrow (A_{1} \wedge B) \vee (A_{2} \wedge B) \vee \cdots \vee (A_{n} \wedge B), \\
  \mbox{} \\
  &B \vee (A_{1} \wedge A_{2} \wedge \cdots \wedge A_{n}) 
  \leftrightarrow (B \vee A_{1}) \wedge (B \vee A_{2}) \wedge \cdots \wedge (B \vee A_{n}), \\
  \mbox{} \\
  &(A_{1} \wedge A_{2} \wedge \cdots \wedge A_{n}) \vee B 
  \leftrightarrow (A_{1} \vee B) \wedge (A_{2} \vee B) \wedge \cdots \wedge (A_{n} \vee B)
\end{align*}
はいずれも$\mathscr{T}$の定理である.
\end{theo}




\mathstrut
\begin{theo}
\label{thmtgvdisteq}%Thm165
$n$を自然数とし, $A_{1}, A_{2}, \cdots, A_{n}$を$\mathscr{T}$の関係式とする.
また$B$を$\mathscr{T}$の関係式とする.
このとき
\[
  (B \to A_{1} \vee A_{2} \vee \cdots \vee A_{n}) 
  \leftrightarrow (B \to A_{1}) \vee (B \to A_{2}) \vee \cdots \vee (B \to A_{n})
\]
は$\mathscr{T}$の定理である.
\end{theo}




\mathstrut
\begin{theo}
\label{thmtgwdisteq}%Thm166
$n$を自然数とし, $A_{1}, A_{2}, \cdots, A_{n}$を$\mathscr{T}$の関係式とする.
また$B$を$\mathscr{T}$の関係式とする.
このとき
\[
  (B \to A_{1} \wedge A_{2} \wedge \cdots \wedge A_{n}) 
  \leftrightarrow (B \to A_{1}) \wedge (B \to A_{2}) \wedge \cdots \wedge (B \to A_{n})
\]
は$\mathscr{T}$の定理である.
\end{theo}




\mathstrut
\begin{theo}
\label{thmtgvgwdisteq}%Thm167
$n$を自然数とし, $A_{1}, A_{2}, \cdots, A_{n}$を$\mathscr{T}$の関係式とする.
また$B$を$\mathscr{T}$の関係式とする.
このとき
\begin{align*}
  &(A_{1} \vee A_{2} \vee \cdots \vee A_{n} \to B) 
  \leftrightarrow (A_{1} \to B) \wedge (A_{2} \to B) \wedge \cdots \wedge (A_{n} \to B), \\
  \mbox{} \\
  &(A_{1} \wedge A_{2} \wedge \cdots \wedge A_{n} \to B) 
  \leftrightarrow (A_{1} \to B) \vee (A_{2} \to B) \vee \cdots \vee (A_{n} \to B)
\end{align*}
は共に$\mathscr{T}$の定理である.
\end{theo}




\mathstrut
\begin{theo}
\label{thmgdemorganeq}%Thm168
$n$を自然数とし, $A_{1}, A_{2}, \cdots, A_{n}$を$\mathscr{T}$の関係式とする.
このとき
\begin{align*}
  &\neg (A_{1} \vee A_{2} \vee \cdots \vee A_{n}) 
  \leftrightarrow \neg A_{1} \wedge \neg A_{2} \wedge \cdots \wedge \neg A_{n}, \\
  \mbox{} \\
  &\neg (A_{1} \wedge A_{2} \wedge \cdots \wedge A_{n}) 
  \leftrightarrow \neg A_{1} \vee \neg A_{2} \vee \cdots \vee \neg A_{n}
\end{align*}
は共に$\mathscr{T}$の定理である.
\end{theo}




\mathstrut
\begin{theo}
\label{thmgabseq}%Thm169
$n$を自然数とし, $A_{1}, A_{2}, \cdots, A_{n}$を$\mathscr{T}$の関係式とする.
また$i$を$n$以下の自然数とする.
このとき
\begin{align*}
  &(A_{1} \vee A_{2} \vee \cdots \vee A_{n}) \wedge A_{i} \leftrightarrow A_{i}, \\
  \mbox{} \\
  &(A_{1} \wedge A_{2} \wedge \cdots \wedge A_{n}) \vee A_{i} \leftrightarrow A_{i}
\end{align*}
は共に$\mathscr{T}$の定理である.
\end{theo}




\mathstrut
\begin{theo}
\label{1alb1l1nalnb1}%Thm170
$A$と$B$が$\mathscr{T}$の関係式ならば, 
\[
  (A \leftrightarrow B) \leftrightarrow (\neg A \leftrightarrow \neg B)
\]
は$\mathscr{T}$の定理である.
\end{theo}




\mathstrut
\begin{theo}
\label{1nalb1l1alnb1}%Thm171
$A$と$B$が$\mathscr{T}$の関係式ならば, 
\[
  (\neg A \leftrightarrow B) \leftrightarrow (A \leftrightarrow \neg B)
\]
は$\mathscr{T}$の定理である.
\end{theo}




\mathstrut
\begin{theo}
\label{1alb1t11atc1l1btc11}%Thm172
$A$, $B$, $C$が$\mathscr{T}$の関係式ならば, 
\[
  (A \leftrightarrow B) \to ((A \to C) \leftrightarrow (B \to C)), ~~
  (A \leftrightarrow B) \to ((C \to A) \leftrightarrow (C \to B))
\]
は共に$\mathscr{T}$の定理である.
\end{theo}




\mathstrut
\begin{theo}
\label{1alb1w1cld1t11atc1l1btd11}%Thm173
$A$, $B$, $C$, $D$が$\mathscr{T}$の関係式ならば, 
\[
  (A \leftrightarrow B) \wedge (C \leftrightarrow D) \to ((A \to C) \leftrightarrow (B \to D))
\]
は$\mathscr{T}$の定理である.
\end{theo}




\mathstrut
\begin{theo}
\label{1alb1t1avclbvc1}%Thm174
$A$, $B$, $C$が$\mathscr{T}$の関係式ならば, 
\[
  (A \leftrightarrow B) \to (A \vee C \leftrightarrow B \vee C), ~~
  (A \leftrightarrow B) \to (C \vee A \leftrightarrow C \vee B)
\]
は共に$\mathscr{T}$の定理である.
\end{theo}




\mathstrut
\begin{theo}
\label{1alb1w1cld1t1avclbvd1}%Thm175
$A$, $B$, $C$, $D$が$\mathscr{T}$の関係式ならば, 
\[
  (A \leftrightarrow B) \wedge (C \leftrightarrow D) \to (A \vee C \leftrightarrow B \vee D)
\]
は$\mathscr{T}$の定理である.
\end{theo}




\mathstrut
\begin{theo}
\label{1alb1t1awclbwc1}%Thm176
$A$, $B$, $C$が$\mathscr{T}$の関係式ならば, 
\[
  (A \leftrightarrow B) \to (A \wedge C \leftrightarrow B \wedge C), ~~
  (A \leftrightarrow B) \to (C \wedge A \leftrightarrow C \wedge B)
\]
は共に$\mathscr{T}$の定理である.
\end{theo}




\mathstrut
\begin{theo}
\label{1alb1w1cld1t1awclbwd1}%Thm177
$A$, $B$, $C$, $D$が$\mathscr{T}$の関係式ならば, 
\[
  (A \leftrightarrow B) \wedge (C \leftrightarrow D) \to (A \wedge C \leftrightarrow B \wedge D)
\]
は$\mathscr{T}$の定理である.
\end{theo}




\mathstrut
\begin{theo}
\label{thmfromaddeqgv}%Thm178
$n$を自然数とし, $A_{1}, A_{2}, \cdots, A_{n}, B_{1}, B_{2}, \cdots, B_{n}$を$\mathscr{T}$の関係式とする.
このとき
\[
  (A_{1} \leftrightarrow B_{1}) \wedge (A_{2} \leftrightarrow B_{2}) \wedge \cdots \wedge (A_{n} \leftrightarrow B_{n}) 
  \to (A_{1} \vee A_{2} \vee \cdots \vee A_{n} \leftrightarrow B_{1} \vee B_{2} \vee \cdots \vee B_{n})
\]
は$\mathscr{T}$の定理である.
\end{theo}




\mathstrut
\begin{theo}
\label{thmfromaddeqgw}%Thm179
$n$を自然数とし, $A_{1}, A_{2}, \cdots, A_{n}, B_{1}, B_{2}, \cdots, B_{n}$を$\mathscr{T}$の関係式とする.
このとき
\[
  (A_{1} \leftrightarrow B_{1}) \wedge (A_{2} \leftrightarrow B_{2}) \wedge \cdots \wedge (A_{n} \leftrightarrow B_{n}) 
  \to (A_{1} \wedge A_{2} \wedge \cdots \wedge A_{n} \leftrightarrow B_{1} \wedge B_{2} \wedge \cdots \wedge B_{n})
\]
は$\mathscr{T}$の定理である.
\end{theo}




\mathstrut
\begin{theo}
\label{1alb1t11alc1l1blc11}%Thm180
$A$, $B$, $C$が$\mathscr{T}$の関係式ならば, 
\[
  (A \leftrightarrow B) \to ((A \leftrightarrow C) \leftrightarrow (B \leftrightarrow C)), ~~
  (A \leftrightarrow B) \to ((C \leftrightarrow A) \leftrightarrow (C \leftrightarrow B))
\]
は共に$\mathscr{T}$の定理である.
\end{theo}




\mathstrut
\begin{theo}
\label{1alb1w1cld1t11alc1l1bld11}%Thm181
$A$, $B$, $C$, $D$が$\mathscr{T}$の関係式ならば, 
\[
  (A \leftrightarrow B) \wedge (C \leftrightarrow D) \to ((A \leftrightarrow C) \leftrightarrow (B \leftrightarrow D))
\]
は$\mathscr{T}$の定理である.
\end{theo}




\mathstrut
\begin{theo}
\label{1ct1alb11l11cta1l1ctb11}%Thm182
$A$, $B$, $C$が$\mathscr{T}$の関係式ならば, 
\[
  (C \to (A \leftrightarrow B)) \leftrightarrow ((C \to A) \leftrightarrow (C \to B))
\]
は$\mathscr{T}$の定理である.
\end{theo}




\mathstrut
\begin{theo}
\label{1nct1alb11l11atc1l1btc11}%Thm183
$A$, $B$, $C$が$\mathscr{T}$の関係式ならば, 
\[
  (\neg C \to (A \leftrightarrow B)) \leftrightarrow ((A \to C) \leftrightarrow (B \to C))
\]
は$\mathscr{T}$の定理である.
\end{theo}




\mathstrut
\begin{theo}
\label{1alb1vcl1avclbvc1}%Thm184
$A$, $B$, $C$が$\mathscr{T}$の関係式ならば, 
\begin{align*}
  (A \leftrightarrow B) \vee C \leftrightarrow (A \vee C \leftrightarrow B \vee C)&, ~~
  (A \leftrightarrow B) \vee C \leftrightarrow (C \vee A \leftrightarrow C \vee B), \\
  \mbox{} \\
  C \vee (A \leftrightarrow B) \leftrightarrow (A \vee C \leftrightarrow B \vee C)&, ~~
  C \vee (A \leftrightarrow B) \leftrightarrow (C \vee A \leftrightarrow C \vee B)
\end{align*}
はいずれも$\mathscr{T}$の定理である.
\end{theo}




\mathstrut
\begin{theo}
\label{1ct1alb11l1awclbwc1}%Thm185
$A$, $B$, $C$が$\mathscr{T}$の関係式ならば, 
\[
  (C \to (A \leftrightarrow B)) \leftrightarrow (A \wedge C \leftrightarrow B \wedge C), ~~
  (C \to (A \leftrightarrow B)) \leftrightarrow (C \wedge A \leftrightarrow C \wedge B)
\]
は共に$\mathscr{T}$の定理である.
\end{theo}




\mathstrut
\begin{theo}
\label{11alc1t1blc11t1ct1atb11}%Thm186
$A$, $B$, $C$が$\mathscr{T}$の関係式ならば, 
\[
  ((A \leftrightarrow C) \to (B \leftrightarrow C)) \to (C \to (A \to B)), ~~
  ((A \leftrightarrow C) \to (B \leftrightarrow C)) \to (\neg C \to (B \to A))
\]
は共に$\mathscr{T}$の定理である.
\end{theo}




\mathstrut
\begin{theo}
\label{1alb1l11alc1l1blc11}%Thm187
$A$, $B$, $C$が$\mathscr{T}$の関係式ならば, 
\[
  (A \leftrightarrow B) \leftrightarrow ((A \leftrightarrow C) \leftrightarrow (B \leftrightarrow C)), ~~
  (A \leftrightarrow B) \leftrightarrow ((C \leftrightarrow A) \leftrightarrow (C \leftrightarrow B))
\]
は共に$\mathscr{T}$の定理である.
\end{theo}




\mathstrut
\begin{theo}
\label{1atb1l1at1alb11}%Thm188
$A$と$B$が$\mathscr{T}$の関係式ならば, 
\[
  (A \to B) \leftrightarrow (A \to (A \leftrightarrow B)), ~~
  (B \to A) \leftrightarrow (B \to (A \leftrightarrow B))
\]
は共に$\mathscr{T}$の定理である.
\end{theo}




\mathstrut
\begin{theo}
\label{avbl11alb1ta1}%Thm189
$A$と$B$が$\mathscr{T}$の関係式ならば, 
\[
  A \vee B \leftrightarrow ((A \leftrightarrow B) \to A), ~~
  A \vee B \leftrightarrow ((A \leftrightarrow B) \to B)
\]
は共に$\mathscr{T}$の定理である.
\end{theo}




\mathstrut
\begin{theo}
\label{al11alb1lb1}%Thm190
$A$と$B$が$\mathscr{T}$の関係式ならば, 
\[
  A \leftrightarrow ((A \leftrightarrow B) \leftrightarrow B), ~~
  B \leftrightarrow ((A \leftrightarrow B) \leftrightarrow A)
\]
は共に$\mathscr{T}$の定理である.
\end{theo}




\mathstrut
\begin{theo}
\label{1alb1l1awb1v1nawnb1}%Thm191
$A$と$B$が$\mathscr{T}$の関係式ならば, 
\[
  (A \leftrightarrow B) \leftrightarrow (A \wedge B) \vee (\neg A \wedge \neg B)
\]
は$\mathscr{T}$の定理である.
\end{theo}




\mathstrut
\begin{theo}
\label{thmquanfree}%Thm192
$\mathscr{T}$を論理的な理論とし, $R$を$\mathscr{T}$の関係式, $x$を文字とする.
$x$が$R$の中に自由変数として現れなければ, 
\[
  \exists x(R) \leftrightarrow R, ~~
  \forall x(R) \leftrightarrow R
\]
は共に$\mathscr{T}$の定理である.
\end{theo}




\mathstrut
\begin{theo}
\label{thmallfund1}%Thm193
$\mathscr{T}$を論理的な理論とし, $R$を$\mathscr{T}$の関係式, $x$を文字とする.
このとき
\[
  \forall x(R) \leftrightarrow (\tau_{x}(\neg R)|x)(R)
\]
は$\mathscr{T}$の定理である.
\end{theo}




\mathstrut
\begin{theo}
\label{allx1rtr1}%Thm194
$\mathscr{T}$を論理的な理論とし, $R$を$\mathscr{T}$の関係式, $x$を文字とする.
このとき
\[
  \exists x(R \to R), ~~
  \forall x(R \to R)
\]
は共に$\mathscr{T}$の定理である.
\end{theo}




\mathstrut
\begin{theo}
\label{allx1rlr1}%Thm195
$\mathscr{T}$を論理的な理論とし, $R$を$\mathscr{T}$の関係式, $x$を文字とする.
このとき
\[
  \exists x(R \leftrightarrow R), ~~
  \forall x(R \leftrightarrow R)
\]
は共に$\mathscr{T}$の定理である.
\end{theo}




\mathstrut
\begin{theo}
\label{thmaequandm}%Thm196
$\mathscr{T}$を論理的な理論とし, $R$を$\mathscr{T}$の関係式, $x$を文字とする.
このとき
\[
  \neg \forall x(R) \leftrightarrow \exists x(\neg R)
\]
は$\mathscr{T}$の定理である.
\end{theo}




\mathstrut
\begin{theo}
\label{thmallfund2}%Thm197
$R$を$\mathscr{T}$の関係式, $T$を$\mathscr{T}$の対象式とし, 
$x$を文字とする.
このとき
\[
  \forall x(R) \to (T|x)(R)
\]
は$\mathscr{T}$の定理である.
\end{theo}




\mathstrut
\begin{theo}
\label{thmallfund3}%Thm198
$R$を$\mathscr{T}$の関係式とし, $x$を文字とするとき, 
\[
  \forall x(R) \to R
\]
は$\mathscr{T}$の定理である.
\end{theo}




\mathstrut
\begin{theo}
\label{thmalltoexist}%Thm199
$R$を$\mathscr{T}$の関係式とし, $x$を文字とするとき, 
\[
  \forall x(R) \to \exists x(R)
\]
は$\mathscr{T}$の定理である.
\end{theo}




\mathstrut
\begin{theo}
\label{thmgs4}%Thm200
$R$を$\mathscr{T}$の関係式とする.
また$n$を自然数とし, $T_{1}, T_{2}, \cdots, T_{n}$を$\mathscr{T}$の対象式とする.
また$x_{1}, x_{2}, \cdots, x_{n}$を, どの二つも互いに異なる文字とする.
このとき
\[
  (T_{1}|x_{1}, T_{2}|x_{2}, \cdots, T_{n}|x_{n})(R) 
  \to \exists x_{1}(\exists x_{2}( \cdots (\exists x_{n}(R)) \cdots ))
\]
は$\mathscr{T}$の定理である.
\end{theo}




\mathstrut
\begin{theo}
\label{thmgallfund2}%Thm201
$R$を$\mathscr{T}$の関係式とする.
また$n$を自然数とし, $T_{1}, T_{2}, \cdots, T_{n}$を$\mathscr{T}$の対象式とする.
また$x_{1}, x_{2}, \cdots, x_{n}$を, どの二つも互いに異なる文字とする.
このとき
\[
  \forall x_{1}(\forall x_{2}( \cdots (\forall x_{n}(R)) \cdots )) 
  \to (T_{1}|x_{1}, T_{2}|x_{2}, \cdots, T_{n}|x_{n})(R)
\]
は$\mathscr{T}$の定理である.
\end{theo}




\mathstrut
\begin{theo}
\label{thmquannn}%Thm202
$R$を$\mathscr{T}$の関係式とし, $x$を文字とするとき, 
\[
  \exists x(\neg \neg R) \leftrightarrow \exists x(R), ~~
  \forall x(\neg \neg R) \leftrightarrow \forall x(R)
\]
は共に$\mathscr{T}$の定理である.
\end{theo}




\mathstrut
\begin{theo}
\label{thmeaquandm}%Thm203
$R$を$\mathscr{T}$の関係式とし, $x$を文字とするとき, 
\[
  \neg \exists x(R) \leftrightarrow \forall x(\neg R)
\]
は$\mathscr{T}$の定理である.
\end{theo}




\mathstrut
\begin{theo}
\label{thmquangdm}%Thm204
$R$を$\mathscr{T}$の関係式とする.
また$n$を自然数とし, $x_{1}, x_{2}, \cdots, x_{n}$を文字とする.
また$n$以下の各自然数$i$に対し, $p_{i}$を$\exists$, $\forall$のどちらかとし, 
$q_{i}$を, $p_{i}$が$\exists$ならば$\forall$, $p_{i}$が$\forall$ならば$\exists$とする.
このとき
\[
  \neg p_{1}x_{1}(p_{2}x_{2}( \cdots (p_{n}x_{n}(R)) \cdots )) 
  \leftrightarrow q_{1}x_{1}(q_{2}x_{2}( \cdots (q_{n}x_{n}(\neg R)) \cdots ))
\]
は$\mathscr{T}$の定理である.
\end{theo}




\mathstrut
\begin{theo}
\label{thmquanvee}%Thm205
$R$と$S$を$\mathscr{T}$の関係式とし, $x$を文字とする.
このとき
\begin{align*}
  &\exists x(R) \to \exists x(R \vee S), ~~
  \exists x(S) \to \exists x(R \vee S), \\
  \mbox{} \\
  &\forall x(R) \to \forall x(R \vee S), ~~
  \forall x(S) \to \forall x(R \vee S)
\end{align*}
はいずれも$\mathscr{T}$の定理である.
\end{theo}




\mathstrut
\begin{theo}
\label{thmquanvch}%Thm206
$R$と$S$を$\mathscr{T}$の関係式とし, $x$を文字とする.
このとき
\[
  \exists x(R \vee S) \leftrightarrow \exists x(S \vee R), ~~
  \forall x(R \vee S) \leftrightarrow \forall x(S \vee R)
\]
は共に$\mathscr{T}$の定理である.
\end{theo}




\mathstrut
\begin{theo}
\label{thmquantveq}%Thm207
$R$と$S$を$\mathscr{T}$の関係式とし, $x$を文字とする.
このとき
\[
  \exists x(R \to S) \leftrightarrow \exists x(\neg R \vee S), ~~
  \forall x(R \to S) \leftrightarrow \forall x(\neg R \vee S)
\]
は共に$\mathscr{T}$の定理である.
\end{theo}




\mathstrut
\begin{theo}
\label{thmexv}%Thm208
$R$と$S$を$\mathscr{T}$の関係式とし, $x$を文字とする.
このとき
\[
  \exists x(R \vee S) \leftrightarrow \exists x(R) \vee \exists x(S)
\]
は$\mathscr{T}$の定理である.
\end{theo}




\mathstrut
\begin{theo}
\label{thmexvrfree}%Thm209
$R$と$S$を$\mathscr{T}$の関係式とし, $x$を文字とする.
$x$が$R$の中に自由変数として現れなければ, 
\[
  \exists x(R \vee S) \leftrightarrow R \vee \exists x(S), ~~
  \exists x(S \vee R) \leftrightarrow \exists x(S) \vee R
\]
は共に$\mathscr{T}$の定理である.
\end{theo}




\mathstrut
\begin{theo}
\label{thmallv}%Thm210
$R$と$S$を$\mathscr{T}$の関係式とし, $x$を文字とする.
このとき
\[
  \forall x(R) \vee \forall x(S) \to \forall x(R \vee S)
\]
は$\mathscr{T}$の定理である.
\end{theo}




\mathstrut
\begin{theo}
\label{thmallv2}%Thm211
$R$と$S$を$\mathscr{T}$の関係式とし, $x$を文字とする.
このとき
\[
  \forall x(R \vee S) \to \forall x(R) \vee \exists x(S), ~~
  \forall x(R \vee S) \to \exists x(R) \vee \forall x(S)
\]
は共に$\mathscr{T}$の定理である.
\end{theo}




\mathstrut
\begin{theo}
\label{thmallvrfree}%Thm212
$R$と$S$を$\mathscr{T}$の関係式とし, $x$を文字とする.
$x$が$R$の中に自由変数として現れなければ, 
\[
  \forall x(R \vee S) \leftrightarrow R \vee \forall x(S), ~~
  \forall x(S \vee R) \leftrightarrow \forall x(S) \vee R
\]
は共に$\mathscr{T}$の定理である.
\end{theo}




\mathstrut
\begin{theo}
\label{thmquanwedge}%Thm213
$R$と$S$を$\mathscr{T}$の関係式とし, $x$を文字とする.
このとき
\begin{align*}
  &\exists x(R \wedge S) \to \exists x(R), ~~
  \exists x(R \wedge S) \to \exists x(S), \\
  \mbox{} \\
  &\forall x(R \wedge S) \to \forall x(R), ~~
  \forall x(R \wedge S) \to \forall x(S)
\end{align*}
はいずれも$\mathscr{T}$の定理である.
\end{theo}




\mathstrut
\begin{theo}
\label{thmquanwch}%Thm214
$R$と$S$を$\mathscr{T}$の関係式とし, $x$を文字とする.
このとき
\[
  \exists x(R \wedge S) \leftrightarrow \exists x(S \wedge R), ~~
  \forall x(R \wedge S) \leftrightarrow \forall x(S \wedge R)
\]
は共に$\mathscr{T}$の定理である.
\end{theo}




\mathstrut
\begin{theo}
\label{thmquantweq}%Thm215
$R$と$S$を$\mathscr{T}$の関係式とし, $x$を文字とする.
このとき
\[
  \exists x(\neg (R \to S)) \leftrightarrow \exists x(R \wedge \neg S), ~~
  \forall x(\neg (R \to S)) \leftrightarrow \forall x(R \wedge \neg S)
\]
は共に$\mathscr{T}$の定理である.
\end{theo}




\mathstrut
\begin{theo}
\label{thmexw}%Thm216
$R$と$S$を$\mathscr{T}$の関係式とし, $x$を文字とする.
このとき
\[
  \exists x(R \wedge S) \to \exists x(R) \wedge \exists x(S)
\]
は$\mathscr{T}$の定理である.
\end{theo}




\mathstrut
\begin{theo}
\label{thmexw2}%Thm217
$R$と$S$を$\mathscr{T}$の関係式とし, $x$を文字とする.
このとき
\[
  \exists x(R) \wedge \forall x(S) \to \exists x(R \wedge S), ~~
  \forall x(R) \wedge \exists x(S) \to \exists x(R \wedge S)
\]
は共に$\mathscr{T}$の定理である.
\end{theo}




\mathstrut
\begin{theo}
\label{thmexwrfree}%Thm218
$R$と$S$を$\mathscr{T}$の関係式とし, $x$を文字とする.
$x$が$R$の中に自由変数として現れなければ, 
\[
  \exists x(R \wedge S) \leftrightarrow R \wedge \exists x(S), ~~
  \exists x(S \wedge R) \leftrightarrow \exists x(S) \wedge R
\]
は共に$\mathscr{T}$の定理である.
\end{theo}




\mathstrut
\begin{theo}
\label{thmallw}%Thm219
$R$と$S$を$\mathscr{T}$の関係式とし, $x$を文字とする.
このとき
\[
  \forall x(R \wedge S) \leftrightarrow \forall x(R) \wedge \forall x(S)
\]
は$\mathscr{T}$の定理である.
\end{theo}




\mathstrut
\begin{theo}
\label{thmallwrfree}%Thm220
$R$と$S$を$\mathscr{T}$の関係式とし, $x$を文字とする.
$x$が$R$の中に自由変数として現れなければ, 
\[
  \forall x(R \wedge S) \leftrightarrow R \wedge \forall x(S), ~~
  \forall x(S \wedge R) \leftrightarrow \forall x(S) \wedge R
\]
は共に$\mathscr{T}$の定理である.
\end{theo}




\mathstrut
\begin{theo}
\label{thmquangvee}%Thm221
$x$を文字とする.
また$n$を自然数とし, $R_{1}, R_{2}, \cdots, R_{n}$を$\mathscr{T}$の関係式とする.
このとき$n$以下の任意の自然数$i$に対し, 
\[
  \exists x(R_{i}) \to \exists x(R_{1} \vee R_{2} \vee \cdots \vee R_{n}), ~~
  \forall x(R_{i}) \to \forall x(R_{1} \vee R_{2} \vee \cdots \vee R_{n})
\]
は共に$\mathscr{T}$の定理である.
\end{theo}




\mathstrut
\begin{theo}
\label{thmquangvee2}%Thm222
$x$を文字とする.
また$n$を自然数とし, $R_{1}, R_{2}, \cdots, R_{n}$を$\mathscr{T}$の関係式とする.
また$k$を自然数とし, $i_{1}, i_{2}, \cdots, i_{k}$を$n$以下の自然数とする.
このとき
\begin{align*}
  &\exists x(R_{i_{1}} \vee R_{i_{2}} \vee \cdots \vee R_{i_{k}}) 
  \to \exists x(R_{1} \vee R_{2} \vee \cdots \vee R_{n}), \\
  \mbox{} \\
  &\forall x(R_{i_{1}} \vee R_{i_{2}} \vee \cdots \vee R_{i_{k}}) 
  \to \forall x(R_{1} \vee R_{2} \vee \cdots \vee R_{n})
\end{align*}
は共に$\mathscr{T}$の定理である.
\end{theo}




\mathstrut
\begin{theo}
\label{thmexgv}%Thm223
$x$を文字とする.
また$n$を自然数とし, $R_{1}, R_{2}, \cdots, R_{n}$を$\mathscr{T}$の関係式とする.
このとき
\[
  \exists x(R_{1} \vee R_{2} \vee \cdots \vee R_{n}) 
  \leftrightarrow \exists x(R_{1}) \vee \exists x(R_{2}) \vee \cdots \vee \exists x(R_{n})
\]
は$\mathscr{T}$の定理である.
\end{theo}




\mathstrut
\begin{theo}
\label{thmexgvfree}%Thm224
$x$を文字とする.
また$n$を自然数とし, $R_{1}, R_{2}, \cdots, R_{n}$を$\mathscr{T}$の関係式とする.
また$k$を$n$以下の自然数とし, $i_{1}, i_{2}, \cdots, i_{k}$を
$i_{1} < i_{2} < \cdots < i_{k} \LEQQ n$なる自然数とする.
$x$が$R_{i_{1}}, R_{i_{2}}, \cdots, R_{i_{k}}$の中に自由変数として現れなければ, 
\begin{multline*}
  \exists x(R_{1} \vee R_{2} \vee \cdots \vee R_{n}) 
  \leftrightarrow \exists x(R_{1}) \vee \cdots \vee \exists x(R_{i_{1} - 1}) \vee R_{i_{1}} \vee \exists x(R_{i_{1} + 1}) \vee \cdots\cdots \\
  \vee \exists x(R_{i_{k} - 1}) \vee R_{i_{k}} \vee \exists x(R_{i_{k} + 1}) \vee \cdots \vee \exists x(R_{n})
\end{multline*}
は$\mathscr{T}$の定理である.
\end{theo}




\mathstrut
\begin{theo}
\label{thmallgv}%Thm225
$x$を文字とする.
また$n$を自然数とし, $R_{1}, R_{2}, \cdots, R_{n}$を$\mathscr{T}$の関係式とする.
このとき
\[
  \forall x(R_{1}) \vee \forall x(R_{2}) \vee \cdots \vee \forall x(R_{n}) 
  \to \forall x(R_{1} \vee R_{2} \vee \cdots \vee R_{n})
\]
は$\mathscr{T}$の定理である.
\end{theo}




\mathstrut
\begin{theo}
\label{thmallgv2}%Thm226
$x$を文字とする.
また$n$を自然数とし, $R_{1}, R_{2}, \cdots, R_{n}$を$\mathscr{T}$の関係式とする.
また$i$を$n$以下の自然数とする.
このとき
\[
  \forall x(R_{1} \vee R_{2} \vee \cdots \vee R_{n}) 
  \to \exists x(R_{1}) \vee \cdots \vee \exists x(R_{i - 1}) \vee \forall x(R_{i}) \vee \exists x(R_{i + 1}) \vee \cdots \vee \exists x(R_{n})
\]
は$\mathscr{T}$の定理である.
\end{theo}




\mathstrut
\begin{theo}
\label{thmallgvfree}%Thm227
$x$を文字とする.
また$n$を自然数とし, $R_{1}, R_{2}, \cdots, R_{n}$を$\mathscr{T}$の関係式とする.
また$k$を$n$以下の自然数とし, $i_{1}, i_{2}, \cdots, i_{k}$を
$i_{1} < i_{2} < \cdots < i_{k} \LEQQ n$なる自然数とする.
$x$が$R_{i_{1}}, R_{i_{2}}, \cdots, R_{i_{k}}$の中に自由変数として現れなければ, 
\begin{multline*}
  \forall x(R_{1}) \vee \cdots \vee \forall x(R_{i_{1} - 1}) \vee R_{i_{1}} \vee \forall x(R_{i_{1} + 1}) \vee \cdots\cdots \\
  \vee \forall x(R_{i_{k} - 1}) \vee R_{i_{k}} \vee \forall x(R_{i_{k} + 1}) \vee \cdots \vee \forall x(R_{n}) 
  \to \forall x(R_{1} \vee R_{2} \vee \cdots \vee R_{n})
\end{multline*}
は$\mathscr{T}$の定理である.
\end{theo}




\mathstrut
\begin{theo}
\label{thmallgvfreeeq}%Thm228
$x$を文字とする.
また$n$を自然数とし, $R_{1}, R_{2}, \cdots, R_{n}$を$\mathscr{T}$の関係式とする.
また$i$を$n$以下の自然数とする.
$i$と異なる$n$以下の任意の自然数$j$に対し, $x$が$R_{j}$の中に自由変数として現れなければ, 
\[
  \forall x(R_{1} \vee R_{2} \vee \cdots \vee R_{n}) 
  \leftrightarrow R_{1} \vee \cdots \vee R_{i - 1} \vee \forall x(R_{i}) \vee R_{i + 1} \vee \cdots \vee R_{n}
\]
は$\mathscr{T}$の定理である.
\end{theo}




\mathstrut
\begin{theo}
\label{thmquangwedge}%Thm229
$x$を文字とする.
また$n$を自然数とし, $R_{1}, R_{2}, \cdots, R_{n}$を$\mathscr{T}$の関係式とする.
このとき$n$以下の任意の自然数$i$に対し, 
\[
  \exists x(R_{1} \wedge R_{2} \wedge \cdots \wedge R_{n}) \to \exists x(R_{i}), ~~
  \forall x(R_{1} \wedge R_{2} \wedge \cdots \wedge R_{n}) \to \forall x(R_{i})
\]
は共に$\mathscr{T}$の定理である.
\end{theo}




\mathstrut
\begin{theo}
\label{thmquangwedge2}%Thm230
$x$を文字とする.
また$n$を自然数とし, $R_{1}, R_{2}, \cdots, R_{n}$を$\mathscr{T}$の関係式とする.
また$k$を自然数とし, $i_{1}, i_{2}, \cdots, i_{k}$を$n$以下の自然数とする.
このとき
\begin{align*}
  &\exists x(R_{1} \wedge R_{2} \wedge \cdots \wedge R_{n}) 
  \to \exists x(R_{i_{1}} \wedge R_{i_{2}} \wedge \cdots \wedge R_{i_{k}}), \\
  \mbox{} \\
  &\forall x(R_{1} \wedge R_{2} \wedge \cdots \wedge R_{n}) 
  \to \forall x(R_{i_{1}} \wedge R_{i_{2}} \wedge \cdots \wedge R_{i_{k}})
\end{align*}
は共に$\mathscr{T}$の定理である.
\end{theo}




\mathstrut
\begin{theo}
\label{thmexgw}%Thm231
$x$を文字とする.
また$n$を自然数とし, $R_{1}, R_{2}, \cdots, R_{n}$を$\mathscr{T}$の関係式とする.
このとき
\[
  \exists x(R_{1} \wedge R_{2} \wedge \cdots \wedge R_{n}) 
  \to \exists x(R_{1}) \wedge \exists x(R_{2}) \wedge \cdots \wedge \exists x(R_{n})
\]
は$\mathscr{T}$の定理である.
\end{theo}




\mathstrut
\begin{theo}
\label{thmexgw2}%Thm232
$x$を文字とする.
また$n$を自然数とし, $R_{1}, R_{2}, \cdots, R_{n}$を$\mathscr{T}$の関係式とする.
また$i$を$n$以下の自然数とする.
このとき
\[
  \forall x(R_{1}) \wedge \cdots \wedge \forall x(R_{i - 1}) \wedge \exists x(R_{i}) \wedge \forall x(R_{i + 1}) \wedge \cdots \wedge \forall x(R_{n}) 
  \to \exists x(R_{1} \wedge R_{2} \wedge \cdots \wedge R_{n})
\]
は$\mathscr{T}$の定理である.
\end{theo}




\mathstrut
\begin{theo}
\label{thmexgwfree}%Thm233
$x$を文字とする.
また$n$を自然数とし, $R_{1}, R_{2}, \cdots, R_{n}$を$\mathscr{T}$の関係式とする.
また$k$を$n$以下の自然数とし, $i_{1}, i_{2}, \cdots, i_{k}$を
$i_{1} < i_{2} < \cdots < i_{k} \LEQQ n$なる自然数とする.
$x$が$R_{i_{1}}, R_{i_{2}}, \cdots, R_{i_{k}}$の中に自由変数として現れなければ, 
\begin{multline*}
  \exists x(R_{1} \wedge R_{2} \wedge \cdots \wedge R_{n}) 
  \to \exists x(R_{1}) \wedge \cdots \wedge \exists x(R_{i_{1} - 1}) \wedge R_{i_{1}} \wedge \exists x(R_{i_{1} + 1}) \wedge \cdots\cdots \\
  \wedge \exists x(R_{i_{k} - 1}) \wedge R_{i_{k}} \wedge \exists x(R_{i_{k} + 1}) \wedge \cdots \wedge \exists x(R_{n})
\end{multline*}
は$\mathscr{T}$の定理である.
\end{theo}




\mathstrut
\begin{theo}
\label{thmexgwfreeeq}%Thm234
$x$を文字とする.
また$n$を自然数とし, $R_{1}, R_{2}, \cdots, R_{n}$を$\mathscr{T}$の関係式とする.
また$i$を$n$以下の自然数とする.
$i$と異なる$n$以下の任意の自然数$j$に対し, $x$が$R_{j}$の中に自由変数として現れなければ, 
\[
  \exists x(R_{1} \wedge R_{2} \wedge \cdots \wedge R_{n}) 
  \leftrightarrow R_{1} \wedge \cdots \wedge R_{i - 1} \wedge \exists x(R_{i}) \wedge R_{i + 1} \wedge \cdots \wedge R_{n}
\]
は$\mathscr{T}$の定理である.
\end{theo}




\mathstrut
\begin{theo}
\label{thmallgw}%Thm235
$x$を文字とする.
また$n$を自然数とし, $R_{1}, R_{2}, \cdots, R_{n}$を$\mathscr{T}$の関係式とする.
このとき
\[
  \forall x(R_{1} \wedge R_{2} \wedge \cdots \wedge R_{n}) 
  \leftrightarrow \forall x(R_{1}) \wedge \forall x(R_{2}) \wedge \cdots \wedge \forall x(R_{n})
\]
は$\mathscr{T}$の定理である.
\end{theo}




\mathstrut
\begin{theo}
\label{thmallgwfree}%Thm236
$x$を文字とする.
また$n$を自然数とし, $R_{1}, R_{2}, \cdots, R_{n}$を$\mathscr{T}$の関係式とする.
また$k$を$n$以下の自然数とし, $i_{1}, i_{2}, \cdots, i_{k}$を
$i_{1} < i_{2} < \cdots < i_{k} \LEQQ n$なる自然数とする.
$x$が$R_{i_{1}}, R_{i_{2}}, \cdots, R_{i_{k}}$の中に自由変数として現れなければ, 
\begin{multline*}
  \forall x(R_{1} \wedge R_{2} \wedge \cdots \wedge R_{n}) 
  \leftrightarrow \forall x(R_{1}) \wedge \cdots \wedge \forall x(R_{i_{1} - 1}) \wedge R_{i_{1}} \wedge \forall x(R_{i_{1} + 1}) \wedge \cdots\cdots \\
  \wedge \forall x(R_{i_{k} - 1}) \wedge R_{i_{k}} \wedge \forall x(R_{i_{k} + 1}) \wedge \cdots \wedge \forall x(R_{n})
\end{multline*}
は$\mathscr{T}$の定理である.
\end{theo}




\mathstrut
\begin{theo}
\label{thmextquansep}%Thm237
$R$と$S$を$\mathscr{T}$の関係式とし, $x$を文字とする.
このとき
\[
  \exists x(R \to S) \leftrightarrow (\forall x(R) \to \exists x(S))
\]
は$\mathscr{T}$の定理である.
\end{theo}




\mathstrut
\begin{theo}
\label{thmextquansep2}%Thm238
$R$と$S$を$\mathscr{T}$の関係式とし, $x$を文字とする.
このとき
\[
  \exists x(S) \to \exists x(R \to S), ~~
  \neg \forall x(R) \to \exists x(R \to S), ~~
  \exists x(\neg R) \to \exists x(R \to S)
\]
はいずれも$\mathscr{T}$の定理である.
\end{theo}




\mathstrut
\begin{theo}
\label{thmextquanseprfree}%Thm239
$R$と$S$を$\mathscr{T}$の関係式とし, $x$を文字とする.
$x$が$R$の中に自由変数として現れなければ, 
\[
  \exists x(R \to S) \leftrightarrow (R \to \exists x(S))
\]
は$\mathscr{T}$の定理である.
\end{theo}




\mathstrut
\begin{theo}
\label{thmextquansepsfree}%Thm240
$R$と$S$を$\mathscr{T}$の関係式とし, $x$を文字とする.
$x$が$S$の中に自由変数として現れなければ, 
\[
  \exists x(R \to S) \leftrightarrow (\forall x(R) \to S)
\]
は$\mathscr{T}$の定理である.
\end{theo}




\mathstrut
\begin{theo}
\label{thmalltexsep}%Thm241
$R$と$S$を$\mathscr{T}$の関係式とし, $x$を文字とする.
このとき
\[
  \forall x(R \to S) \to (\exists x(R) \to \exists x(S))
\]
は$\mathscr{T}$の定理である.
\end{theo}




\mathstrut
\begin{theo}
\label{thmalltallsep}%Thm242
$R$と$S$を$\mathscr{T}$の関係式とし, $x$を文字とする.
このとき
\[
  \forall x(R \to S) \to (\forall x(R) \to \forall x(S))
\]
は$\mathscr{T}$の定理である.
\end{theo}




\mathstrut
\begin{theo}
\label{thmquanseptall}%Thm243
$R$と$S$を$\mathscr{T}$の関係式とし, $x$を文字とする.
このとき
\[
  (\exists x(R) \to \forall x(S)) \to \forall x(R \to S)
\]
は$\mathscr{T}$の定理である.
\end{theo}




\mathstrut
\begin{theo}
\label{thmquanseptall2}%Thm244
$R$と$S$を$\mathscr{T}$の関係式とし, $x$を文字とする.
このとき
\[
  \forall x(S) \to \forall x(R \to S), ~~
  \neg \exists x(R) \to \forall x(R \to S), ~~
  \forall x(\neg R) \to \forall x(R \to S)
\]
はいずれも$\mathscr{T}$の定理である.
\end{theo}




\mathstrut
\begin{theo}
\label{thmalltallseprfree}%Thm245
$R$と$S$を$\mathscr{T}$の関係式とし, $x$を文字とする.
$x$が$R$の中に自由変数として現れなければ, 
\[
  \forall x(R \to S) \leftrightarrow (R \to \forall x(S))
\]
は$\mathscr{T}$の定理である.
\end{theo}




\mathstrut
\begin{theo}
\label{thmalltexsepsfree}%Thm246
$R$と$S$を$\mathscr{T}$の関係式とし, $x$を文字とする.
$x$が$S$の中に自由変数として現れなければ, 
\[
  \forall x(R \to S) \leftrightarrow (\exists x(R) \to S)
\]
は$\mathscr{T}$の定理である.
\end{theo}




\mathstrut
\begin{theo}
\label{thmquansepeqex}%Thm247
$R$と$S$を$\mathscr{T}$の関係式とし, $x$を文字とする.
このとき
\[
  (\exists x(R) \leftrightarrow \forall x(S)) \to \exists x(R \leftrightarrow S), ~~
  (\forall x(R) \leftrightarrow \exists x(S)) \to \exists x(R \leftrightarrow S)
\]
は共に$\mathscr{T}$の定理である.
\end{theo}




\mathstrut
\begin{theo}
\label{thmquansepeqexrfree}%Thm248
$R$と$S$を$\mathscr{T}$の関係式とし, $x$を文字とする.
$x$が$R$の中に自由変数として現れなければ, 
\[
  (R \leftrightarrow \forall x(S)) \to \exists x(R \leftrightarrow S), ~~
  (R \leftrightarrow \exists x(S)) \to \exists x(R \leftrightarrow S)
\]
は共に$\mathscr{T}$の定理である.
\end{theo}




\mathstrut
\begin{theo}
\label{thmquansepeqexsfree}%Thm249
$R$と$S$を$\mathscr{T}$の関係式とし, $x$を文字とする.
$x$が$S$の中に自由変数として現れなければ, 
\[
  (\exists x(R) \leftrightarrow S) \to \exists x(R \leftrightarrow S), ~~
  (\forall x(R) \leftrightarrow S) \to \exists x(R \leftrightarrow S)
\]
は共に$\mathscr{T}$の定理である.
\end{theo}




\mathstrut
\begin{theo}
\label{thmalleqexsep}%Thm250
$R$と$S$を$\mathscr{T}$の関係式とし, $x$を文字とする.
このとき
\[
  \forall x(R \leftrightarrow S) \to (\exists x(R) \leftrightarrow \exists x(S))
\]
は$\mathscr{T}$の定理である.
\end{theo}




\mathstrut
\begin{theo}
\label{thmalleqallsep}%Thm251
$R$と$S$を$\mathscr{T}$の関係式とし, $x$を文字とする.
このとき
\[
  \forall x(R \leftrightarrow S) \to (\forall x(R) \leftrightarrow \forall x(S))
\]
は$\mathscr{T}$の定理である.
\end{theo}




\mathstrut
\begin{theo}
\label{thmalleqquanseprfree}%Thm252
$R$と$S$を$\mathscr{T}$の関係式とし, $x$を文字とする.
$x$が$R$の中に自由変数として現れなければ, 
\[
  \forall x(R \leftrightarrow S) \to (R \leftrightarrow \exists x(S)), ~~
  \forall x(R \leftrightarrow S) \to (R \leftrightarrow \forall x(S))
\]
は共に$\mathscr{T}$の定理である.
\end{theo}




\mathstrut
\begin{theo}
\label{thmalleqquansepsfree}%Thm253
$R$と$S$を$\mathscr{T}$の関係式とし, $x$を文字とする.
$x$が$S$の中に自由変数として現れなければ, 
\[
  \forall x(R \leftrightarrow S) \to (\exists x(R) \leftrightarrow S), ~~
  \forall x(R \leftrightarrow S) \to (\forall x(R) \leftrightarrow S)
\]
は共に$\mathscr{T}$の定理である.
\end{theo}




\mathstrut
\begin{theo}
\label{thmexch}%Thm254
$R$を$\mathscr{T}$の関係式とし, $x$と$y$を文字とする.
このとき
\[
  \exists x(\exists y(R)) \leftrightarrow \exists y(\exists x(R))
\]
は$\mathscr{T}$の定理である.
\end{theo}




\mathstrut
\begin{theo}
\label{thmallch}%Thm255
$R$を$\mathscr{T}$の関係式とし, $x$と$y$を文字とする.
このとき
\[
  \forall x(\forall y(R)) \leftrightarrow \forall y(\forall x(R))
\]
は$\mathscr{T}$の定理である.
\end{theo}




\mathstrut
\begin{theo}
\label{thmquanch}%Thm256
$R$を$\mathscr{T}$の関係式とし, $x$と$y$を文字とする.
このとき
\[
  \exists x(\forall y(R)) \to \forall y(\exists x(R))
\]
は$\mathscr{T}$の定理である.
\end{theo}




\mathstrut
\begin{theo}
\label{thmgquanchlem1}%Thm257
$R$を$\mathscr{T}$の関係式とし, $x$と$y$を文字とする.
また$n$を自然数とし, $x_{1}, x_{2}, \cdots, x_{n}$を文字とする.
このとき
\begin{align*}
  &\exists x_{1}(\exists x_{2}( \cdots (\exists x_{n}(\exists x(\exists y(R)))) \cdots )) 
  \leftrightarrow \exists x_{1}(\exists x_{2}( \cdots (\exists x_{n}(\exists y(\exists x(R)))) \cdots )), \\
  \mbox{} \\
  &\forall x_{1}(\forall x_{2}( \cdots (\forall x_{n}(\forall x(\forall y(R)))) \cdots )) 
  \leftrightarrow \forall x_{1}(\forall x_{2}( \cdots (\forall x_{n}(\forall y(\forall x(R)))) \cdots ))
\end{align*}
は共に$\mathscr{T}$の定理である.
\end{theo}




\mathstrut
\begin{theo}
\label{thmgquanchlem2}%Thm258
$R$を$\mathscr{T}$の関係式とする.
また$n$を自然数とし, $x_{1}, x_{2}, \cdots, x_{n}$を文字とする.
また$i$, $j$を$i < j \LEQQ n$なる自然数とする.
このとき
\begin{align*}
  &\exists x_{1}( \cdots (\exists x_{i}( \cdots (\exists x_{j}( \cdots (\exists x_{n}(R)) \cdots )) \cdots )) \cdots ) 
  \leftrightarrow \exists x_{1}( \cdots (\exists x_{j}( \cdots (\exists x_{i}( \cdots (\exists x_{n}(R)) \cdots )) \cdots )) \cdots ), \\
  \mbox{} \\
  &\forall x_{1}( \cdots (\forall x_{i}( \cdots (\forall x_{j}( \cdots (\forall x_{n}(R)) \cdots )) \cdots )) \cdots ) 
  \leftrightarrow \forall x_{1}( \cdots (\forall x_{j}( \cdots (\forall x_{i}( \cdots (\forall x_{n}(R)) \cdots )) \cdots )) \cdots )
\end{align*}
は共に$\mathscr{T}$の定理である.
\end{theo}




\mathstrut
\begin{theo}
\label{thmgexch}%Thm259
$R$を$\mathscr{T}$の関係式とする.
また$n$を自然数とし, $x_{1}, x_{2}, \cdots, x_{n}$を文字とする.
また自然数$1, 2, \cdots, n$の順序を任意に入れ替えたものを
$i_{1}, i_{2}, \cdots, i_{n}$とする.
このとき
\[
  \exists x_{1}(\exists x_{2}( \cdots (\exists x_{n}(R)) \cdots )) 
  \leftrightarrow \exists x_{i_{1}}(\exists x_{i_{2}}( \cdots (\exists x_{i_{n}}(R)) \cdots ))
\]
は$\mathscr{T}$の定理である.
\end{theo}




\mathstrut
\begin{theo}
\label{thmgallch}%Thm260
$R$を$\mathscr{T}$の関係式とする.
また$n$を自然数とし, $x_{1}, x_{2}, \cdots, x_{n}$を文字とする.
また自然数$1, 2, \cdots, n$の順序を任意に入れ替えたものを
$i_{1}, i_{2}, \cdots, i_{n}$とする.
このとき
\[
  \forall x_{1}(\forall x_{2}( \cdots (\forall x_{n}(R)) \cdots )) 
  \leftrightarrow \forall x_{i_{1}}(\forall x_{i_{2}}( \cdots (\forall x_{i_{n}}(R)) \cdots ))
\]
は$\mathscr{T}$の定理である.
\end{theo}




\mathstrut
\begin{theo}
\label{thmspallfund}%Thm261
$A$と$R$を$\mathscr{T}$の関係式とし, $x$を文字とする.
このとき
\[
  \forall_{A}x(R) \leftrightarrow \forall x(A \to R)
\]
は$\mathscr{T}$の定理である.
\end{theo}




\mathstrut
\begin{theo}
\label{thmspallfund2}%Thm262
$A$と$R$を$\mathscr{T}$の関係式とし, $x$を文字とする.
このとき
\[
  \forall_{A}x(R) \leftrightarrow (\tau_{x}(\neg (A \to R))|x)(A \to R)
\]
は$\mathscr{T}$の定理である.
\end{theo}




\mathstrut
\begin{theo}
\label{thmspextex}%Thm263
$A$と$R$を$\mathscr{T}$の関係式とし, $x$を文字とする.
このとき
\[
  \exists_{A}x(R) \to \exists x(A), ~~
  \exists_{A}x(R) \to \exists x(R)
\]
は共に$\mathscr{T}$の定理である.
\end{theo}




\mathstrut
\begin{theo}
\label{thmquantspex}%Thm264
$A$と$R$を$\mathscr{T}$の関係式とし, $x$を文字とする.
このとき
\[
  \exists x(A) \wedge \forall x(R) \to \exists_{A}x(R), ~~
  \forall x(A) \wedge \exists x(R) \to \exists_{A}x(R)
\]
は共に$\mathscr{T}$の定理である.
\end{theo}




\mathstrut
\begin{theo}
\label{thmspexneg}%Thm265
$A$と$R$を$\mathscr{T}$の関係式とし, $x$を文字とする.
このとき
\begin{align*}
  &\neg \exists x(A) \to \neg \exists_{A}x(R), ~~
  \neg \exists x(R) \to \neg \exists_{A}x(R), \\
  \mbox{} \\
  &\forall x(\neg A) \to \neg \exists_{A}x(R), ~~
  \forall x(\neg R) \to \neg \exists_{A}x(R)
\end{align*}
はいずれも$\mathscr{T}$の定理である.
\end{theo}




\mathstrut
\begin{theo}
\label{thmspalltquan}%Thm266
$A$と$R$を$\mathscr{T}$の関係式とし, $x$を文字とする.
このとき
\[
  \forall_{A}x(R) \to (\exists x(A) \to \exists x(R)), ~~
  \forall_{A}x(R) \to (\forall x(A) \to \forall x(R))
\]
は共に$\mathscr{T}$の定理である.
\end{theo}




\mathstrut
\begin{theo}
\label{thmquantspall}%Thm267
$A$と$R$を$\mathscr{T}$の関係式とし, $x$を文字とする.
このとき
\[
  (\exists x(A) \to \forall x(R)) \to \forall_{A}x(R)
\]
は$\mathscr{T}$の定理である.
\end{theo}




\mathstrut
\begin{theo}
\label{thmquantspall2}%Thm268
$A$と$R$を$\mathscr{T}$の関係式とし, $x$を文字とする.
このとき
\[
  \forall x(R) \to \forall_{A}x(R), ~~
  \neg \exists x(A) \to \forall_{A}x(R), ~~
  \forall x(\neg A) \to \forall_{A}x(R)
\]
はいずれも$\mathscr{T}$の定理である.
\end{theo}




\mathstrut
\begin{theo}
\label{thmspallneg}%Thm269
$A$と$R$を$\mathscr{T}$の関係式とし, $x$を文字とする.
このとき
\begin{align*}
  &\exists x(A) \wedge \neg \exists x(R) \to \neg \forall_{A}x(R), ~~
  \forall x(A) \wedge \neg \forall x(R) \to \neg \forall_{A}x(R), \\
  \mbox{} \\
  &\exists x(A) \wedge \forall x(\neg R) \to \neg \forall_{A}x(R), ~~
  \forall x(A) \wedge \exists x(\neg R) \to \neg \forall_{A}x(R)
\end{align*}
はいずれも$\mathscr{T}$の定理である.
\end{theo}




\mathstrut
\begin{theo}
\label{thmspquanafree}%Thm270
$A$と$R$を$\mathscr{T}$の関係式とし, $x$を文字とする.
$x$が$A$の中に自由変数として現れなければ, 
\[
  \exists_{A}x(R) \leftrightarrow A \wedge \exists x(R), ~~
  \forall_{A}x(R) \leftrightarrow (A \to \forall x(R))
\]
は共に$\mathscr{T}$の定理である.
\end{theo}




\mathstrut
\begin{theo}
\label{thmspquanafree2}%Thm271
$A$と$R$を$\mathscr{T}$の関係式とし, $x$を文字とする.
$x$が$A$の中に自由変数として現れなければ, 
\[
  \exists_{A}x(R) \to A, ~~
  \neg A \to \forall_{A}x(R)
\]
は共に$\mathscr{T}$の定理である.
\end{theo}




\mathstrut
\begin{theo}
\label{thmspquanrfree}%Thm272
$A$と$R$を$\mathscr{T}$の関係式とし, $x$を文字とする.
$x$が$R$の中に自由変数として現れなければ, 
\[
  \exists_{A}x(R) \leftrightarrow \exists x(A) \wedge R, ~~
  \forall_{A}x(R) \leftrightarrow (\exists x(A) \to R)
\]
は共に$\mathscr{T}$の定理である.
\end{theo}




\mathstrut
\begin{theo}
\label{thmspquanrfree2}%Thm273
$A$と$R$を$\mathscr{T}$の関係式とし, $x$を文字とする.
$x$が$R$の中に自由変数として現れなければ, 
\[
  \exists_{A}x(R) \to R, ~~
  R \to \forall_{A}x(R)
\]
は共に$\mathscr{T}$の定理である.
\end{theo}




\mathstrut
\begin{theo}
\label{thmaespquandm}%Thm274
$A$と$R$を論理的な理論$\mathscr{T}$の関係式とし, $x$を文字とする.
このとき
\[
  \neg \forall_{A}x(R) \leftrightarrow \exists_{A}x(\neg R)
\]
は$\mathscr{T}$の定理である.
\end{theo}




\mathstrut
\begin{theo}
\label{thmspquannn}%Thm275
$A$と$R$を$\mathscr{T}$の関係式とし, $x$を文字とする.
このとき
\[
  \exists_{A}x(\neg \neg R) \leftrightarrow \exists_{A}x(R), ~~
  \forall_{A}x(\neg \neg R) \leftrightarrow \forall_{A}x(R)
\]
は共に$\mathscr{T}$の定理である.
\end{theo}




\mathstrut
\begin{theo}
\label{thmeaspquandm}%Thm276
$A$と$R$を$\mathscr{T}$の関係式とし, $x$を文字とする.
このとき
\[
  \neg \exists_{A}x(R) \leftrightarrow \forall_{A}x(\neg R)
\]
は$\mathscr{T}$の定理である.
\end{theo}




\mathstrut
\begin{theo}
\label{thmspquangdm}%Thm277
$R$を$\mathscr{T}$の関係式とする.
また$n$を自然数とし, $A_{1}, A_{2}, \cdots, A_{n}$を$\mathscr{T}$の関係式, 
$x_{1}, x_{2}, \cdots, x_{n}$を文字とする.
また$n$以下の各自然数$i$に対し, $p^{i}$を$\exists$, $\forall$のどちらかとし, 
$q^{i}$を, $p^{i}$が$\exists$ならば$\forall$, $p^{i}$が$\forall$ならば$\exists$とする.
このとき
\[
  \neg p^{1}_{A_{1}}x_{1}(p^{2}_{A_{2}}x_{2}( \cdots (p^{n}_{A_{n}}x_{n}(R)) \cdots )) 
  \leftrightarrow q^{1}_{A_{1}}x_{1}(q^{2}_{A_{2}}x_{2}( \cdots (q^{n}_{A_{n}}x_{n}(\neg R)) \cdots ))
\]
は$\mathscr{T}$の定理である.
\end{theo}




\mathstrut
\begin{theo}
\label{thmsps4}%Thm278
$A$と$R$を$\mathscr{T}$の関係式, $T$を$\mathscr{T}$の対象式とし, 
$x$を文字とする.
このとき
\[
  (T|x)(A) \wedge (T|x)(R) \to \exists_{A}x(R), ~~
  \forall_{A}x(R) \to ((T|x)(A) \to (T|x)(R))
\]
は共に$\mathscr{T}$の定理である.
\end{theo}




\mathstrut
\begin{theo}
\label{thmexalspquansep}%Thm279
$A$と$R$を$\mathscr{T}$の関係式とし, $x$を文字とする.
このとき
\[
  \exists x(A) \leftrightarrow (\forall_{A}x(R) \to \exists_{A}x(R))
\]
は$\mathscr{T}$の定理である.
\end{theo}




\mathstrut
\begin{theo}
\label{thmgsps4}%Thm280
$R$を$\mathscr{T}$の関係式とする.
また$n$を自然数とし, $T_{1}, T_{2}, \cdots, T_{n}$を$\mathscr{T}$の対象式, 
$A_{1}, A_{2}, \cdots, A_{n}$を$\mathscr{T}$の関係式, 
$x_{1}, x_{2}, \cdots, x_{n}$をどの二つも互いに異なる文字とする.
このとき
\begin{multline*}
  (T_{1}|x_{1})(A_{1}) \wedge (T_{1}|x_{1}, T_{2}|x_{2})(A_{2}) \wedge \cdots \wedge (T_{1}|x_{1}, T_{2}|x_{2}, \cdots, T_{n}|x_{n})(A_{n}) \\
  \wedge (T_{1}|x_{1}, T_{2}|x_{2}, \cdots, T_{n}|x_{n})(R) 
  \to \exists_{A_{1}}x_{1}(\exists_{A_{2}}x_{2}( \cdots (\exists_{A_{n}}x_{n}(R)) \cdots ))
\end{multline*}
は$\mathscr{T}$の定理である.
\end{theo}




\mathstrut
\begin{theo}
\label{thmgsps4free}%Thm281
$R$を$\mathscr{T}$の関係式とする.
また$n$を自然数とし, $T_{1}, T_{2}, \cdots, T_{n}$を$\mathscr{T}$の対象式, 
$A_{1}, A_{2}, \cdots, A_{n}$を$\mathscr{T}$の関係式とする.
また$x_{1}, x_{2}, \cdots, x_{n}$をどの二つも互いに異なる文字とする.
いま$i < n$なる各自然数$i$に対し, $x_{1}, x_{2}, \cdots, x_{i}$はいずれも
$A_{i + 1}$の中に自由変数として現れないとする.
このとき
\begin{multline*}
  (T_{1}|x_{1})(A_{1}) \wedge (T_{2}|x_{2})(A_{2}) \wedge \cdots \wedge (T_{n}|x_{n})(A_{n}) 
  \wedge (T_{1}|x_{1}, T_{2}|x_{2}, \cdots, T_{n}|x_{n})(R) \\
  \to \exists_{A_{1}}x_{1}(\exists_{A_{2}}x_{2}( \cdots (\exists_{A_{n}}x_{n}(R)) \cdots ))
\end{multline*}
は$\mathscr{T}$の定理である.
\end{theo}




\mathstrut
\begin{theo}
\label{thmgspallfund2}%Thm282
$R$を$\mathscr{T}$の関係式とする.
また$n$を自然数とし, $T_{1}, T_{2}, \cdots, T_{n}$を$\mathscr{T}$の対象式, 
$A_{1}, A_{2}, \cdots, A_{n}$を$\mathscr{T}$の関係式, 
$x_{1}, x_{2}, \cdots, x_{n}$をどの二つも互いに異なる文字とする.
このとき
\begin{multline*}
  \forall_{A_{1}}x_{1}(\forall_{A_{2}}x_{2}( \cdots (\forall_{A_{n}}x_{n}(R)) \cdots )) \\
  \to ((T_{1}|x_{1})(A_{1}) \wedge (T_{1}|x_{1}, T_{2}|x_{2})(A_{2}) \wedge \cdots \wedge (T_{1}|x_{1}, T_{2}|x_{2}, \cdots, T_{n}|x_{n})(A_{n}) \\
  \to (T_{1}|x_{1}, T_{2}|x_{2}, \cdots, T_{n}|x_{n})(R))
\end{multline*}
は$\mathscr{T}$の定理である.
\end{theo}




\mathstrut
\begin{theo}
\label{thmgspallfund2free}%Thm283
$R$を$\mathscr{T}$の関係式とする.
また$n$を自然数とし, $T_{1}, T_{2}, \cdots, T_{n}$を$\mathscr{T}$の対象式, 
$A_{1}, A_{2}, \cdots, A_{n}$を$\mathscr{T}$の関係式とする.
また$x_{1}, x_{2}, \cdots, x_{n}$をどの二つも互いに異なる文字とする.
いま$i < n$なる各自然数$i$に対し, $x_{1}, x_{2}, \cdots, x_{i}$はいずれも
$A_{i + 1}$の中に自由変数として現れないとする.
このとき
\begin{multline*}
  \forall_{A_{1}}x_{1}(\forall_{A_{2}}x_{2}( \cdots (\forall_{A_{n}}x_{n}(R)) \cdots )) \\
  \to ((T_{1}|x_{1})(A_{1}) \wedge (T_{2}|x_{2})(A_{2}) \wedge \cdots \wedge (T_{n}|x_{n})(A_{n}) 
  \to (T_{1}|x_{1}, T_{2}|x_{2}, \cdots, T_{n}|x_{n})(R))
\end{multline*}
は$\mathscr{T}$の定理である.
\end{theo}




\mathstrut
\begin{theo}
\label{thmspquanvee}%Thm284
$A$, $R$, $S$を$\mathscr{T}$の関係式とし, $x$を文字とする.
このとき
\begin{align*}
  &\exists_{A}x(R) \to \exists_{A}x(R \vee S), ~~
  \exists_{A}x(S) \to \exists_{A}x(R \vee S), \\
  \mbox{} \\
  &\forall_{A}x(R) \to \forall_{A}x(R \vee S), ~~
  \forall_{A}x(S) \to \forall_{A}x(R \vee S)
\end{align*}
はいずれも$\mathscr{T}$の定理である.
\end{theo}




\mathstrut
\begin{theo}
\label{thmspquanvch}%Thm285
$A$, $R$, $S$を$\mathscr{T}$の関係式とし, $x$を文字とする.
このとき
\[
  \exists_{A}x(R \vee S) \leftrightarrow \exists_{A}x(S \vee R), ~~
  \forall_{A}x(R \vee S) \leftrightarrow \forall_{A}x(S \vee R)
\]
は共に$\mathscr{T}$の定理である.
\end{theo}




\mathstrut
\begin{theo}
\label{thmspquantveq}%Thm286
$A$, $R$, $S$を$\mathscr{T}$の関係式とし, $x$を文字とする.
このとき
\[
  \exists_{A}x(R \to S) \leftrightarrow \exists_{A}x(\neg R \vee S), ~~
  \forall_{A}x(R \to S) \leftrightarrow \forall_{A}x(\neg R \vee S)
\]
は共に$\mathscr{T}$の定理である.
\end{theo}




\mathstrut
\begin{theo}
\label{thmspexv}%Thm287
$A$, $R$, $S$を$\mathscr{T}$の関係式とし, $x$を文字とする.
このとき
\[
  \exists_{A}x(R \vee S) \leftrightarrow \exists_{A}x(R) \vee \exists_{A}x(S)
\]
は$\mathscr{T}$の定理である.
\end{theo}




\mathstrut
\begin{theo}
\label{thmspexvrfree}%Thm288
$A$, $R$, $S$を$\mathscr{T}$の関係式とし, $x$を文字とする.
$x$が$R$の中に自由変数として現れなければ, 
\[
  \exists_{A}x(R \vee S) \to R \vee \exists_{A}x(S), ~~
  \exists_{A}x(S \vee R) \to \exists_{A}x(S) \vee R
\]
は共に$\mathscr{T}$の定理である.
\end{theo}




\mathstrut
\begin{theo}
\label{thmspexvrfreeeq}%Thm289
$A$, $R$, $S$を$\mathscr{T}$の関係式とし, $x$を文字とする.
$x$が$R$の中に自由変数として現れなければ, 
\[
  \exists x(A) \to (\exists_{A}x(R \vee S) \leftrightarrow R \vee \exists_{A}x(S)), ~~
  \exists x(A) \to (\exists_{A}x(S \vee R) \leftrightarrow \exists_{A}x(S) \vee R)
\]
は共に$\mathscr{T}$の定理である.
\end{theo}




\mathstrut
\begin{theo}
\label{thmspallv}%Thm290
$A$, $R$, $S$を$\mathscr{T}$の関係式とし, $x$を文字とする.
このとき
\[
  \forall_{A}x(R) \vee \forall_{A}x(S) \to \forall_{A}x(R \vee S)
\]
は$\mathscr{T}$の定理である.
\end{theo}




\mathstrut
\begin{theo}
\label{thmspallv2}%Thm291
$A$, $R$, $S$を$\mathscr{T}$の関係式とし, $x$を文字とする.
このとき
\[
  \forall_{A}x(R \vee S) \to \forall_{A}x(R) \vee \exists_{A}x(S), ~~
  \forall_{A}x(R \vee S) \to \exists_{A}x(R) \vee \forall_{A}x(S)
\]
は共に$\mathscr{T}$の定理である.
\end{theo}




\mathstrut
\begin{theo}
\label{thmspallvrfree}%Thm292
$A$, $R$, $S$を$\mathscr{T}$の関係式とし, $x$を文字とする.
$x$が$R$の中に自由変数として現れなければ, 
\[
  \forall_{A}x(R \vee S) \leftrightarrow R \vee \forall_{A}x(S), ~~
  \forall_{A}x(S \vee R) \leftrightarrow \forall_{A}x(S) \vee R
\]
は共に$\mathscr{T}$の定理である.
\end{theo}




\mathstrut
\begin{theo}
\label{thmspquanwedge}%Thm293
$A$, $R$, $S$を$\mathscr{T}$の関係式とし, $x$を文字とする.
このとき
\begin{align*}
  &\exists_{A}x(R \wedge S) \to \exists_{A}x(R), ~~
  \exists_{A}x(R \wedge S) \to \exists_{A}x(S), \\
  \mbox{} \\
  &\forall_{A}x(R \wedge S) \to \forall_{A}x(R), ~~
  \forall_{A}x(R \wedge S) \to \forall_{A}x(S)
\end{align*}
はいずれも$\mathscr{T}$の定理である.
\end{theo}




\mathstrut
\begin{theo}
\label{thmspquanwch}%Thm294
$A$, $R$, $S$を$\mathscr{T}$の関係式とし, $x$を文字とする.
このとき
\[
  \exists_{A}x(R \wedge S) \leftrightarrow \exists_{A}x(S \wedge R), ~~
  \forall_{A}x(R \wedge S) \leftrightarrow \forall_{A}x(S \wedge R)
\]
は共に$\mathscr{T}$の定理である.
\end{theo}




\mathstrut
\begin{theo}
\label{thmspquantweq}%Thm295
$A$, $R$, $S$を$\mathscr{T}$の関係式とし, $x$を文字とする.
このとき
\[
  \exists_{A}x(\neg (R \to S)) \leftrightarrow \exists_{A}x(R \wedge \neg S), ~~
  \forall_{A}x(\neg (R \to S)) \leftrightarrow \forall_{A}x(R \wedge \neg S)
\]
は共に$\mathscr{T}$の定理である.
\end{theo}




\mathstrut
\begin{theo}
\label{thmspexw}%Thm296
$A$, $R$, $S$を$\mathscr{T}$の関係式とし, $x$を文字とする.
このとき
\[
  \exists_{A}x(R \wedge S) \to \exists_{A}x(R) \wedge \exists_{A}x(S)
\]
は$\mathscr{T}$の定理である.
\end{theo}




\mathstrut
\begin{theo}
\label{thmspexw2}%Thm297
$A$, $R$, $S$を$\mathscr{T}$の関係式とし, $x$を文字とする.
このとき
\[
  \exists_{A}x(R) \wedge \forall_{A}x(S) \to \exists_{A}x(R \wedge S), ~~
  \forall_{A}x(R) \wedge \exists_{A}x(S) \to \exists_{A}x(R \wedge S)
\]
は共に$\mathscr{T}$の定理である.
\end{theo}




\mathstrut
\begin{theo}
\label{thmspexwrfree}%Thm298
$A$, $R$, $S$を$\mathscr{T}$の関係式とし, $x$を文字とする.
$x$が$R$の中に自由変数として現れなければ, 
\[
  \exists_{A}x(R \wedge S) \leftrightarrow R \wedge \exists_{A}x(S), ~~
  \exists_{A}x(S \wedge R) \leftrightarrow \exists_{A}x(S) \wedge R
\]
は共に$\mathscr{T}$の定理である.
\end{theo}




\mathstrut
\begin{theo}
\label{thmspallw}%Thm299
$A$, $R$, $S$を$\mathscr{T}$の関係式とし, $x$を文字とする.
このとき
\[
  \forall_{A}x(R \wedge S) \leftrightarrow \forall_{A}x(R) \wedge \forall_{A}x(S)
\]
は$\mathscr{T}$の定理である.
\end{theo}




\mathstrut
\begin{theo}
\label{thmspallwrfree}%Thm300
$A$, $R$, $S$を$\mathscr{T}$の関係式とし, $x$を文字とする.
$x$が$R$の中に自由変数として現れなければ, 
\[
  R \wedge \forall_{A}x(S) \to \forall_{A}x(R \wedge S), ~~
  \forall_{A}x(S) \wedge R \to \forall_{A}x(S \wedge R)
\]
は共に$\mathscr{T}$の定理である.
\end{theo}




\mathstrut
\begin{theo}
\label{thmspallwrfreeeq}%Thm301
$A$, $R$, $S$を$\mathscr{T}$の関係式とし, $x$を文字とする.
$x$が$R$の中に自由変数として現れなければ, 
\[
  \exists x(A) \to (\forall_{A}x(R \wedge S) \leftrightarrow R \wedge \forall_{A}x(S)), ~~
  \exists x(A) \to (\forall_{A}x(S \wedge R) \leftrightarrow \forall_{A}x(S) \wedge R)
\]
は共に$\mathscr{T}$の定理である.
\end{theo}




\mathstrut
\begin{theo}
\label{thmspexprevee}%Thm302
$A$, $B$, $R$を$\mathscr{T}$の関係式とし, $x$を文字とする.
このとき
\[
  \exists_{A}x(R) \to \exists_{A \vee B}x(R), ~~
  \exists_{B}x(R) \to \exists_{A \vee B}x(R)
\]
は共に$\mathscr{T}$の定理である.
\end{theo}




\mathstrut
\begin{theo}
\label{thmspexprev}%Thm303
$A$, $B$, $R$を$\mathscr{T}$の関係式とし, $x$を文字とする.
このとき
\[
  \exists_{A \vee B}x(R) \leftrightarrow \exists_{A}x(R) \vee \exists_{B}x(R)
\]
は$\mathscr{T}$の定理である.
\end{theo}




\mathstrut
\begin{theo}
\label{thmspexprevafree}%Thm304
$A$, $B$, $R$を$\mathscr{T}$の関係式とし, $x$を文字とする.
$x$が$A$の中に自由変数として現れなければ, 
\[
  \exists_{A \vee B}x(R) \to A \vee \exists_{B}x(R), ~~
  \exists_{B \vee A}x(R) \to \exists_{B}x(R) \vee A
\]
は共に$\mathscr{T}$の定理である.
\end{theo}




\mathstrut
\begin{theo}
\label{thmspexprevafreeeq}%Thm305
$A$, $B$, $R$を$\mathscr{T}$の関係式とし, $x$を文字とする.
$x$が$A$の中に自由変数として現れなければ, 
\[
  \exists x(R) \to (\exists_{A \vee B}x(R) \leftrightarrow A \vee \exists_{B}x(R)), ~~
  \exists x(R) \to (\exists_{B \vee A}x(R) \leftrightarrow \exists_{B}x(R) \vee A)
\]
は共に$\mathscr{T}$の定理である.
\end{theo}




\mathstrut
\begin{theo}
\label{thmspallprevee}%Thm306
$A$, $B$, $R$を$\mathscr{T}$の関係式とし, $x$を文字とする.
このとき
\[
  \forall_{A \vee B}x(R) \to \forall_{A}x(R), ~~
  \forall_{A \vee B}x(R) \to \forall_{B}x(R)
\]
は共に$\mathscr{T}$の定理である.
\end{theo}




\mathstrut
\begin{theo}
\label{thmspallprev}%Thm307
$A$, $B$, $R$を$\mathscr{T}$の関係式とし, $x$を文字とする.
このとき
\[
  \forall_{A \vee B}x(R) \leftrightarrow \forall_{A}x(R) \wedge \forall_{B}x(R)
\]
は$\mathscr{T}$の定理である.
\end{theo}




\mathstrut
\begin{theo}
\label{thmspallprevafree}%Thm308
$A$, $B$, $R$を$\mathscr{T}$の関係式とし, $x$を文字とする.
$x$が$A$の中に自由変数として現れなければ, 
\[
  \neg A \wedge \forall_{B}x(R) \to \forall_{A \vee B}x(R), ~~
  \forall_{B}x(R) \wedge \neg A \to \forall_{B \vee A}x(R)
\]
は共に$\mathscr{T}$の定理である.
\end{theo}




\mathstrut
\begin{theo}
\label{thmspallprevafreeeq}%Thm309
$A$, $B$, $R$を$\mathscr{T}$の関係式とし, $x$を文字とする.
$x$が$A$の中に自由変数として現れなければ, 
\[
  \exists x(\neg R) \to (\forall_{A \vee B}x(R) \leftrightarrow \neg A \wedge \forall_{B}x(R)), ~~
  \exists x(\neg R) \to (\forall_{B \vee A}x(R) \leftrightarrow \forall_{B}x(R) \wedge \neg A)
\]
は共に$\mathscr{T}$の定理である.
\end{theo}




\mathstrut
\begin{theo}
\label{thmspexprewedge}%Thm310
$A$, $B$, $R$を$\mathscr{T}$の関係式とし, $x$を文字とする.
このとき
\[
  \exists_{A \wedge B}x(R) \to \exists_{A}x(R), ~~
  \exists_{A \wedge B}x(R) \to \exists_{B}x(R)
\]
は共に$\mathscr{T}$の定理である.
\end{theo}




\mathstrut
\begin{theo}
\label{thmspexprew}%Thm311
$A$, $B$, $R$を$\mathscr{T}$の関係式とし, $x$を文字とする.
このとき
\[
  \exists_{A \wedge B}x(R) \to \exists_{A}x(R) \wedge \exists_{B}x(R)
\]
は$\mathscr{T}$の定理である.
\end{theo}




\mathstrut
\begin{theo}
\label{thmspexprewafree}%Thm312
$A$, $B$, $R$を$\mathscr{T}$の関係式とし, $x$を文字とする.
$x$が$A$の中に自由変数として現れなければ, 
\[
  \exists_{A \wedge B}x(R) \leftrightarrow A \wedge \exists_{B}x(R), ~~
  \exists_{B \wedge A}x(R) \leftrightarrow \exists_{B}x(R) \wedge A
\]
は共に$\mathscr{T}$の定理である.
\end{theo}




\mathstrut
\begin{theo}
\label{thmspallprewedge}%Thm313
$A$, $B$, $R$を$\mathscr{T}$の関係式とし, $x$を文字とする.
このとき
\[
  \forall_{A}x(R) \to \forall_{A \wedge B}x(R), ~~
  \forall_{B}x(R) \to \forall_{A \wedge B}x(R)
\]
は共に$\mathscr{T}$の定理である.
\end{theo}




\mathstrut
\begin{theo}
\label{thmspallprew}%Thm314
$A$, $B$, $R$を$\mathscr{T}$の関係式とし, $x$を文字とする.
このとき
\[
  \forall_{A}x(R) \vee \forall_{B}x(R) \to \forall_{A \wedge B}x(R)
\]
は$\mathscr{T}$の定理である.
\end{theo}




\mathstrut
\begin{theo}
\label{thmspallprewafree}%Thm315
$A$, $B$, $R$を$\mathscr{T}$の関係式とし, $x$を文字とする.
$x$が$A$の中に自由変数として現れなければ, 
\[
  \forall_{A \wedge B}x(R) \leftrightarrow (A \to \forall_{B}x(R)), ~~
  \forall_{B \wedge A}x(R) \leftrightarrow (A \to \forall_{B}x(R))
\]
は共に$\mathscr{T}$の定理である.
\end{theo}




\mathstrut
\begin{theo}
\label{thmspquangvee}%Thm316
$A$を$\mathscr{T}$の関係式とし, $x$を文字とする.
また$n$を自然数とし, $R_{1}, R_{2}, \cdots, R_{n}$を$\mathscr{T}$の関係式とする.
このとき$n$以下の任意の自然数$i$に対し, 
\[
  \exists_{A}x(R_{i}) \to \exists_{A}x(R_{1} \vee R_{2} \vee \cdots \vee R_{n}), ~~
  \forall_{A}x(R_{i}) \to \forall_{A}x(R_{1} \vee R_{2} \vee \cdots \vee R_{n})
\]
は共に$\mathscr{T}$の定理である.
\end{theo}




\mathstrut
\begin{theo}
\label{thmspquangvee2}%Thm317
$A$を$\mathscr{T}$の関係式とし, $x$を文字とする.
また$n$を自然数とし, $R_{1}, R_{2}, \cdots, R_{n}$を$\mathscr{T}$の関係式とする.
また$k$を自然数とし, $i_{1}, i_{2}, \cdots, i_{k}$を$n$以下の自然数とする.
このとき
\begin{align*}
  &\exists_{A}x(R_{i_{1}} \vee R_{i_{2}} \vee \cdots \vee R_{i_{k}}) 
  \to \exists_{A}x(R_{1} \vee R_{2} \vee \cdots \vee R_{n}), \\
  \mbox{} \\
  &\forall_{A}x(R_{i_{1}} \vee R_{i_{2}} \vee \cdots \vee R_{i_{k}}) 
  \to \forall_{A}x(R_{1} \vee R_{2} \vee \cdots \vee R_{n})
\end{align*}
は共に$\mathscr{T}$の定理である.
\end{theo}




\mathstrut
\begin{theo}
\label{thmspexgv}%Thm318
$A$を$\mathscr{T}$の関係式とし, $x$を文字とする.
また$n$を自然数とし, $R_{1}, R_{2}, \cdots, R_{n}$を$\mathscr{T}$の関係式とする.
このとき
\[
  \exists_{A}x(R_{1} \vee R_{2} \vee \cdots \vee R_{n}) 
  \leftrightarrow \exists_{A}x(R_{1}) \vee \exists_{A}x(R_{2}) \vee \cdots \vee \exists_{A}x(R_{n})
\]
は$\mathscr{T}$の定理である.
\end{theo}




\mathstrut
\begin{theo}
\label{thmspexgvfree}%Thm319
$A$を$\mathscr{T}$の関係式とし, $x$を文字とする.
また$n$を自然数とし, $R_{1}, R_{2}, \cdots, R_{n}$を$\mathscr{T}$の関係式とする.
また$k$を$n$以下の自然数とし, $i_{1}, i_{2}, \cdots, i_{k}$を
$i_{1} < i_{2} < \cdots < i_{k} \LEQQ n$なる自然数とする.
$x$が$R_{i_{1}}, R_{i_{2}}, \cdots, R_{i_{k}}$の中に自由変数として現れなければ, 
\begin{multline*}
  \exists_{A}x(R_{1} \vee R_{2} \vee \cdots \vee R_{n}) \\
  \to \exists_{A}x(R_{1}) \vee \cdots \vee \exists_{A}x(R_{i_{1} - 1}) \vee R_{i_{1}} \vee \exists_{A}x(R_{i_{1} + 1}) \vee \cdots\cdots \\
  \vee \exists_{A}x(R_{i_{k} - 1}) \vee R_{i_{k}} \vee \exists_{A}x(R_{i_{k} + 1}) \vee \cdots \vee \exists_{A}x(R_{n})
\end{multline*}
は$\mathscr{T}$の定理である.
\end{theo}




\mathstrut
\begin{theo}
\label{thmspexgvfreeeq}%Thm320
$A$を$\mathscr{T}$の関係式とし, $x$を文字とする.
また$n$を自然数とし, $R_{1}, R_{2}, \cdots, R_{n}$を$\mathscr{T}$の関係式とする.
また$k$を$n$以下の自然数とし, $i_{1}, i_{2}, \cdots, i_{k}$を
$i_{1} < i_{2} < \cdots < i_{k} \LEQQ n$なる自然数とする.
$x$が$R_{i_{1}}, R_{i_{2}}, \cdots, R_{i_{k}}$の中に自由変数として現れなければ, 
\begin{multline*}
  \exists x(A) 
  \to (\exists_{A}x(R_{1} \vee R_{2} \vee \cdots \vee R_{n}) \\
  \leftrightarrow \exists_{A}x(R_{1}) \vee \cdots \vee \exists_{A}x(R_{i_{1} - 1}) \vee R_{i_{1}} \vee \exists_{A}x(R_{i_{1} + 1}) \vee \cdots\cdots \\
  \vee \exists_{A}x(R_{i_{k} - 1}) \vee R_{i_{k}} \vee \exists_{A}x(R_{i_{k} + 1}) \vee \cdots \vee \exists_{A}x(R_{n}))
\end{multline*}
は$\mathscr{T}$の定理である.
\end{theo}




\mathstrut
\begin{theo}
\label{thmspallgv}%Thm321
$A$を$\mathscr{T}$の関係式とし, $x$を文字とする.
また$n$を自然数とし, $R_{1}, R_{2}, \cdots, R_{n}$を$\mathscr{T}$の関係式とする.
このとき
\[
  \forall_{A}x(R_{1}) \vee \forall_{A}x(R_{2}) \vee \cdots \vee \forall_{A}x(R_{n}) 
  \to \forall_{A}x(R_{1} \vee R_{2} \vee \cdots \vee R_{n})
\]
は$\mathscr{T}$の定理である.
\end{theo}




\mathstrut
\begin{theo}
\label{thmspallgv2}%Thm322
$A$を$\mathscr{T}$の関係式とし, $x$を文字とする.
また$n$を自然数とし, $R_{1}, R_{2}, \cdots, R_{n}$を$\mathscr{T}$の関係式とする.
また$i$を$n$以下の自然数とする.
このとき
\[
  \forall_{A}x(R_{1} \vee R_{2} \vee \cdots \vee R_{n}) 
  \to \exists_{A}x(R_{1}) \vee \cdots \vee \exists_{A}x(R_{i - 1}) \vee \forall_{A}x(R_{i}) \vee \exists_{A}x(R_{i + 1}) \vee \cdots \vee \exists_{A}x(R_{n})
\]
は$\mathscr{T}$の定理である.
\end{theo}




\mathstrut
\begin{theo}
\label{thmspallgvfree}%Thm323
$A$を$\mathscr{T}$の関係式とし, $x$を文字とする.
また$n$を自然数とし, $R_{1}, R_{2}, \cdots, R_{n}$を$\mathscr{T}$の関係式とする.
また$k$を$n$以下の自然数とし, $i_{1}, i_{2}, \cdots, i_{k}$を
$i_{1} < i_{2} < \cdots < i_{k} \LEQQ n$なる自然数とする.
$x$が$R_{i_{1}}, R_{i_{2}}, \cdots, R_{i_{k}}$の中に自由変数として現れなければ, 
\begin{multline*}
  \forall_{A}x(R_{1}) \vee \cdots \vee \forall_{A}x(R_{i_{1} - 1}) \vee R_{i_{1}} \vee \forall_{A}x(R_{i_{1} + 1}) \vee \cdots\cdots \\
  \vee \forall_{A}x(R_{i_{k} - 1}) \vee R_{i_{k}} \vee \forall_{A}x(R_{i_{k} + 1}) \vee \cdots \vee \forall_{A}x(R_{n}) 
  \to \forall_{A}x(R_{1} \vee R_{2} \vee \cdots \vee R_{n})
\end{multline*}
は$\mathscr{T}$の定理である.
\end{theo}




\mathstrut
\begin{theo}
\label{thmspallgvfreeeq}%Thm324
$A$を$\mathscr{T}$の関係式とし, $x$を文字とする.
また$n$を自然数とし, $R_{1}, R_{2}, \cdots, R_{n}$を$\mathscr{T}$の関係式とする.
また$i$を$n$以下の自然数とする.
$i$と異なる$n$以下の任意の自然数$j$に対し, $x$が$R_{j}$の中に自由変数として現れなければ, 
\[
  \forall_{A}x(R_{1} \vee R_{2} \vee \cdots \vee R_{n}) 
  \leftrightarrow R_{1} \vee \cdots \vee R_{i - 1} \vee \forall_{A}x(R_{i}) \vee R_{i + 1} \vee \cdots \vee R_{n}
\]
は$\mathscr{T}$の定理である.
\end{theo}




\mathstrut
\begin{theo}
\label{thmspquangwedge}%Thm325
$A$を$\mathscr{T}$の関係式とし, $x$を文字とする.
また$n$を自然数とし, $R_{1}, R_{2}, \cdots, R_{n}$を$\mathscr{T}$の関係式とする.
このとき$n$以下の任意の自然数$i$に対し, 
\[
  \exists_{A}x(R_{1} \wedge R_{2} \wedge \cdots \wedge R_{n}) \to \exists_{A}x(R_{i}), ~~
  \forall_{A}x(R_{1} \wedge R_{2} \wedge \cdots \wedge R_{n}) \to \forall_{A}x(R_{i})
\]
は共に$\mathscr{T}$の定理である.
\end{theo}




\mathstrut
\begin{theo}
\label{thmspquangwedge2}%Thm326
$A$を$\mathscr{T}$の関係式とし, $x$を文字とする.
また$n$を自然数とし, $R_{1}, R_{2}, \cdots, R_{n}$を$\mathscr{T}$の関係式とする.
また$k$を自然数とし, $i_{1}, i_{2}, \cdots, i_{k}$を$n$以下の自然数とする.
このとき
\begin{align*}
  &\exists_{A}x(R_{1} \wedge R_{2} \wedge \cdots \wedge R_{n}) 
  \to \exists_{A}x(R_{i_{1}} \wedge R_{i_{2}} \wedge \cdots \wedge R_{i_{k}}), \\
  \mbox{} \\
  &\forall_{A}x(R_{1} \wedge R_{2} \wedge \cdots \wedge R_{n}) 
  \to \forall_{A}x(R_{i_{1}} \wedge R_{i_{2}} \wedge \cdots \wedge R_{i_{k}})
\end{align*}
は共に$\mathscr{T}$の定理である.
\end{theo}




\mathstrut
\begin{theo}
\label{thmspexgw}%Thm327
$A$を$\mathscr{T}$の関係式とし, $x$を文字とする.
また$n$を自然数とし, $R_{1}, R_{2}, \cdots, R_{n}$を$\mathscr{T}$の関係式とする.
このとき
\[
  \exists_{A}x(R_{1} \wedge R_{2} \wedge \cdots \wedge R_{n}) 
  \to \exists_{A}x(R_{1}) \wedge \exists_{A}x(R_{2}) \wedge \cdots \wedge \exists_{A}x(R_{n})
\]
は$\mathscr{T}$の定理である.
\end{theo}




\mathstrut
\begin{theo}
\label{thmspexgw2}%Thm328
$A$を$\mathscr{T}$の関係式とし, $x$を文字とする.
また$n$を自然数とし, $R_{1}, R_{2}, \cdots, R_{n}$を$\mathscr{T}$の関係式とする.
また$i$を$n$以下の自然数とする.
このとき
\[
  \forall_{A}x(R_{1}) \wedge \cdots \wedge \forall_{A}x(R_{i - 1}) \wedge \exists_{A}x(R_{i}) \wedge \forall_{A}x(R_{i + 1}) \wedge \cdots \wedge \forall_{A}x(R_{n}) 
  \to \exists_{A}x(R_{1} \wedge R_{2} \wedge \cdots \wedge R_{n})
\]
は$\mathscr{T}$の定理である.
\end{theo}




\mathstrut
\begin{theo}
\label{thmspexgwfree}%Thm329
$A$を$\mathscr{T}$の関係式とし, $x$を文字とする.
また$n$を自然数とし, $R_{1}, R_{2}, \cdots, R_{n}$を$\mathscr{T}$の関係式とする.
また$k$を$n$以下の自然数とし, $i_{1}, i_{2}, \cdots, i_{k}$を
$i_{1} < i_{2} < \cdots < i_{k} \LEQQ n$なる自然数とする.
$x$が$R_{i_{1}}, R_{i_{2}}, \cdots, R_{i_{k}}$の中に自由変数として現れなければ, 
\begin{multline*}
  \exists_{A}x(R_{1} \wedge R_{2} \wedge \cdots \wedge R_{n}) 
  \to \exists_{A}x(R_{1}) \wedge \cdots \wedge \exists_{A}x(R_{i_{1} - 1}) \wedge R_{i_{1}} \wedge \exists_{A}x(R_{i_{1} + 1}) \wedge \cdots\cdots \\
  \wedge \exists_{A}x(R_{i_{k} - 1}) \wedge R_{i_{k}} \wedge \exists_{A}x(R_{i_{k} + 1}) \wedge \cdots \wedge \exists_{A}x(R_{n})
\end{multline*}
は$\mathscr{T}$の定理である.
\end{theo}




\mathstrut
\begin{theo}
\label{thmspexgwfreeeq}%Thm330
$A$を$\mathscr{T}$の関係式とし, $x$を文字とする.
また$n$を自然数とし, $R_{1}, R_{2}, \cdots, R_{n}$を$\mathscr{T}$の関係式とする.
また$i$を$n$以下の自然数とする.
$i$と異なる$n$以下の任意の自然数$j$に対し, $x$が$R_{j}$の中に自由変数として現れなければ, 
\[
  \exists_{A}x(R_{1} \wedge R_{2} \wedge \cdots \wedge R_{n}) 
  \leftrightarrow R_{1} \wedge \cdots \wedge R_{i - 1} \wedge \exists_{A}x(R_{i}) \wedge R_{i + 1} \wedge \cdots \wedge R_{n}
\]
は$\mathscr{T}$の定理である.
\end{theo}




\mathstrut
\begin{theo}
\label{thmspallgw}%Thm331
$A$を$\mathscr{T}$の関係式とし, $x$を文字とする.
また$n$を自然数とし, $R_{1}, R_{2}, \cdots, R_{n}$を$\mathscr{T}$の関係式とする.
このとき
\[
  \forall_{A}x(R_{1} \wedge R_{2} \wedge \cdots \wedge R_{n}) 
  \leftrightarrow \forall_{A}x(R_{1}) \wedge \forall_{A}x(R_{2}) \wedge \cdots \wedge \forall_{A}x(R_{n})
\]
は$\mathscr{T}$の定理である.
\end{theo}




\mathstrut
\begin{theo}
\label{thmspallgwfree}%Thm332
$A$を$\mathscr{T}$の関係式とし, $x$を文字とする.
また$n$を自然数とし, $R_{1}, R_{2}, \cdots, R_{n}$を$\mathscr{T}$の関係式とする.
また$k$を$n$以下の自然数とし, $i_{1}, i_{2}, \cdots, i_{k}$を
$i_{1} < i_{2} < \cdots < i_{k} \LEQQ n$なる自然数とする.
$x$が$R_{i_{1}}, R_{i_{2}}, \cdots, R_{i_{k}}$の中に自由変数として現れなければ, 
\begin{multline*}
  \forall_{A}x(R_{1}) \wedge \cdots \wedge \forall_{A}x(R_{i_{1} - 1}) \wedge R_{i_{1}} \wedge \forall_{A}x(R_{i_{1} + 1}) \wedge \cdots\cdots \\
  \wedge \forall_{A}x(R_{i_{k} - 1}) \wedge R_{i_{k}} \wedge \forall_{A}x(R_{i_{k} + 1}) \wedge \cdots \wedge \forall_{A}x(R_{n}) \\
  \to \forall_{A}x(R_{1} \wedge R_{2} \wedge \cdots \wedge R_{n})
\end{multline*}
は$\mathscr{T}$の定理である.
\end{theo}




\mathstrut
\begin{theo}
\label{thmspallgwfreeeq}%Thm333
$A$を$\mathscr{T}$の関係式とし, $x$を文字とする.
また$n$を自然数とし, $R_{1}, R_{2}, \cdots, R_{n}$を$\mathscr{T}$の関係式とする.
また$k$を$n$以下の自然数とし, $i_{1}, i_{2}, \cdots, i_{k}$を
$i_{1} < i_{2} < \cdots < i_{k} \LEQQ n$なる自然数とする.
$x$が$R_{i_{1}}, R_{i_{2}}, \cdots, R_{i_{k}}$の中に自由変数として現れなければ, 
\begin{multline*}
  \exists x(A) 
  \to (\forall_{A}x(R_{1} \wedge R_{2} \wedge \cdots \wedge R_{n}) \\
  \leftrightarrow \forall_{A}x(R_{1}) \wedge \cdots \wedge \forall_{A}x(R_{i_{1} - 1}) \wedge R_{i_{1}} \wedge \forall_{A}x(R_{i_{1} + 1}) \wedge \cdots\cdots \\
  \wedge \forall_{A}x(R_{i_{k} - 1}) \wedge R_{i_{k}} \wedge \forall_{A}x(R_{i_{k} + 1}) \wedge \cdots \wedge \forall_{A}x(R_{n}))
\end{multline*}
は$\mathscr{T}$の定理である.
\end{theo}




\mathstrut
\begin{theo}
\label{thmspexpregvee}%Thm334
$R$を$\mathscr{T}$の関係式とし, $x$を文字とする.
また$n$を自然数とし, $A_{1}, A_{2}, \cdots, A_{n}$を$\mathscr{T}$の関係式とする.
このとき$n$以下の任意の自然数$i$に対し, 
\[
  \exists_{A_{i}}x(R) \to \exists_{A_{1} \vee A_{2} \vee \cdots \vee A_{n}}x(R)
\]
は$\mathscr{T}$の定理である.
\end{theo}




\mathstrut
\begin{theo}
\label{thmspexpregvee2}%Thm335
$R$を$\mathscr{T}$の関係式とし, $x$を文字とする.
また$n$を自然数とし, $A_{1}, A_{2}, \cdots, A_{n}$を$\mathscr{T}$の関係式とする.
また$k$を自然数とし, $i_{1}, i_{2}, \cdots, i_{k}$を$n$以下の自然数とする.
このとき
\[
  \exists_{A_{i_{1}} \vee A_{i_{2}} \vee \cdots \vee A_{i_{k}}}x(R) 
  \to \exists_{A_{1} \vee A_{2} \vee \cdots \vee A_{n}}x(R)
\]
は$\mathscr{T}$の定理である.
\end{theo}




\mathstrut
\begin{theo}
\label{thmspexpregv}%Thm336
$R$を$\mathscr{T}$の関係式とし, $x$を文字とする.
また$n$を自然数とし, $A_{1}, A_{2}, \cdots, A_{n}$を$\mathscr{T}$の関係式とする.
このとき
\[
  \exists_{A_{1} \vee A_{2} \vee \cdots \vee A_{n}}x(R) 
  \leftrightarrow \exists_{A_{1}}x(R) \vee \exists_{A_{2}}x(R) \vee \cdots \vee \exists_{A_{n}}x(R)
\]
は$\mathscr{T}$の定理である.
\end{theo}




\mathstrut
\begin{theo}
\label{thmspexpregvfree}%Thm337
$R$を$\mathscr{T}$の関係式とし, $x$を文字とする.
また$n$を自然数とし, $A_{1}, A_{2}, \cdots, A_{n}$を$\mathscr{T}$の関係式とする.
また$k$を$n$以下の自然数とし, $i_{1}, i_{2}, \cdots, i_{k}$を
$i_{1} < i_{2} < \cdots < i_{k} \LEQQ n$なる自然数とする.
$x$が$A_{i_{1}}, A_{i_{2}}, \cdots, A_{i_{k}}$の中に自由変数として現れなければ, 
\begin{multline*}
  \exists_{A_{1} \vee A_{2} \vee \cdots \vee A_{n}}x(R) 
  \to \exists_{A_{1}}x(R) \vee \cdots \vee \exists_{A_{i_{1} - 1}}x(R) \vee A_{i_{1}} \vee \exists_{A_{i_{1} + 1}}x(R) \vee \cdots\cdots \\
  \vee \exists_{A_{i_{k} - 1}}x(R) \vee A_{i_{k}} \vee \exists_{A_{i_{k} + 1}}x(R) \vee \cdots \vee \exists_{A_{n}}x(R)
\end{multline*}
は$\mathscr{T}$の定理である.
\end{theo}




\mathstrut
\begin{theo}
\label{thmspexpregvfreeeq}%Thm338
$R$を$\mathscr{T}$の関係式とし, $x$を文字とする.
また$n$を自然数とし, $A_{1}, A_{2}, \cdots, A_{n}$を$\mathscr{T}$の関係式とする.
また$k$を$n$以下の自然数とし, $i_{1}, i_{2}, \cdots, i_{k}$を
$i_{1} < i_{2} < \cdots < i_{k} \LEQQ n$なる自然数とする.
$x$が$A_{i_{1}}, A_{i_{2}}, \cdots, A_{i_{k}}$の中に自由変数として現れなければ, 
\begin{multline*}
  \exists x(R) 
  \to (\exists_{A_{1} \vee A_{2} \vee \cdots \vee A_{n}}x(R) 
  \leftrightarrow \exists_{A_{1}}x(R) \vee \cdots \vee \exists_{A_{i_{1} - 1}}x(R) \vee A_{i_{1}} \vee \exists_{A_{i_{1} + 1}}x(R) \vee \cdots\cdots \\
  \vee \exists_{A_{i_{k} - 1}}x(R) \vee A_{i_{k}} \vee \exists_{A_{i_{k} + 1}}x(R) \vee \cdots \vee \exists_{A_{n}}x(R))
\end{multline*}
は$\mathscr{T}$の定理である.
\end{theo}




\mathstrut
\begin{theo}
\label{thmspallpregvee}%Thm339
$R$を$\mathscr{T}$の関係式とし, $x$を文字とする.
また$n$を自然数とし, $A_{1}, A_{2}, \cdots, A_{n}$を$\mathscr{T}$の関係式とする.
このとき$n$以下の任意の自然数$i$に対し, 
\[
  \forall_{A_{1} \vee A_{2} \vee \cdots \vee A_{n}}x(R) \to \forall_{A_{i}}x(R)
\]
は$\mathscr{T}$の定理である.
\end{theo}




\mathstrut
\begin{theo}
\label{thmspallpregvee2}%Thm340
$R$を$\mathscr{T}$の関係式とし, $x$を文字とする.
また$n$を自然数とし, $A_{1}, A_{2}, \cdots, A_{n}$を$\mathscr{T}$の関係式とする.
また$k$を自然数とし, $i_{1}, i_{2}, \cdots, i_{k}$を$n$以下の自然数とする.
このとき
\[
  \forall_{A_{1} \vee A_{2} \vee \cdots \vee A_{n}}x(R) 
  \to \forall_{A_{i_{1}} \vee A_{i_{2}} \vee \cdots \vee A_{i_{k}}}x(R)
\]
は$\mathscr{T}$の定理である.
\end{theo}




\mathstrut
\begin{theo}
\label{thmspallpregv}%Thm341
$R$を$\mathscr{T}$の関係式とし, $x$を文字とする.
また$n$を自然数とし, $A_{1}, A_{2}, \cdots, A_{n}$を$\mathscr{T}$の関係式とする.
このとき
\[
  \forall_{A_{1} \vee A_{2} \vee \cdots \vee A_{n}}x(R) 
  \leftrightarrow \forall_{A_{1}}x(R) \wedge \forall_{A_{2}}x(R) \wedge \cdots \wedge \forall_{A_{n}}x(R)
\]
は$\mathscr{T}$の定理である.
\end{theo}




\mathstrut
\begin{theo}
\label{thmspallpregvfree}%Thm342
$R$を$\mathscr{T}$の関係式とし, $x$を文字とする.
また$n$を自然数とし, $A_{1}, A_{2}, \cdots, A_{n}$を$\mathscr{T}$の関係式とする.
また$k$を$n$以下の自然数とし, $i_{1}, i_{2}, \cdots, i_{k}$を
$i_{1} < i_{2} < \cdots < i_{k} \LEQQ n$なる自然数とする.
$x$が$A_{i_{1}}, A_{i_{2}}, \cdots, A_{i_{k}}$の中に自由変数として現れなければ, 
\begin{multline*}
  \forall_{A_{1}}x(R) \wedge \cdots \wedge \forall_{A_{i_{1} - 1}}x(R) \wedge \neg A_{i_{1}} \wedge \forall_{A_{i_{1} + 1}}x(R) \wedge \cdots\cdots \\
  \wedge \forall_{A_{i_{k} - 1}}x(R) \wedge \neg A_{i_{k}} \wedge \forall_{A_{i_{k} + 1}}x(R) \wedge \cdots \wedge \forall_{A_{n}}x(R) 
  \to \forall_{A_{1} \vee A_{2} \vee \cdots \vee A_{n}}x(R)
\end{multline*}
は$\mathscr{T}$の定理である.
\end{theo}




\mathstrut
\begin{theo}
\label{thmspallpregvfreeeq}%Thm343
$R$を$\mathscr{T}$の関係式とし, $x$を文字とする.
また$n$を自然数とし, $A_{1}, A_{2}, \cdots, A_{n}$を$\mathscr{T}$の関係式とする.
また$k$を$n$以下の自然数とし, $i_{1}, i_{2}, \cdots, i_{k}$を
$i_{1} < i_{2} < \cdots < i_{k} \LEQQ n$なる自然数とする.
$x$が$A_{i_{1}}, A_{i_{2}}, \cdots, A_{i_{k}}$の中に自由変数として現れなければ, 
\begin{multline*}
  \exists x(\neg R) 
  \to (\forall_{A_{1} \vee A_{2} \vee \cdots \vee A_{n}}x(R) 
  \leftrightarrow \forall_{A_{1}}x(R) \wedge \cdots \wedge \forall_{A_{i_{1} - 1}}x(R) \wedge \neg A_{i_{1}} \wedge \forall_{A_{i_{1} + 1}}x(R) \wedge \cdots\cdots \\
  \wedge \forall_{A_{i_{k} - 1}}x(R) \wedge \neg A_{i_{k}} \wedge \forall_{A_{i_{k} + 1}}x(R) \wedge \cdots \wedge \forall_{A_{n}}x(R))
\end{multline*}
は$\mathscr{T}$の定理である.
\end{theo}




\mathstrut
\begin{theo}
\label{thmspexpregwedge}%Thm344
$R$を$\mathscr{T}$の関係式とし, $x$を文字とする.
また$n$を自然数とし, $A_{1}, A_{2}, \cdots, A_{n}$を$\mathscr{T}$の関係式とする.
このとき$n$以下の任意の自然数$i$に対し, 
\[
  \exists_{A_{1} \wedge A_{2} \wedge \cdots \wedge A_{n}}x(R) \to \exists_{A_{i}}x(R)
\]
は$\mathscr{T}$の定理である.
\end{theo}




\mathstrut
\begin{theo}
\label{thmspexpregwedge2}%Thm345
$R$を$\mathscr{T}$の関係式とし, $x$を文字とする.
また$n$を自然数とし, $A_{1}, A_{2}, \cdots, A_{n}$を$\mathscr{T}$の関係式とする.
また$k$を自然数とし, $i_{1}, i_{2}, \cdots, i_{k}$を$n$以下の自然数とする.
このとき
\[
  \exists_{A_{1} \wedge A_{2} \wedge \cdots \wedge A_{n}}x(R) 
  \to \exists_{A_{i_{1}} \wedge A_{i_{2}} \wedge \cdots \wedge A_{i_{k}}}x(R)
\]
は$\mathscr{T}$の定理である.
\end{theo}




\mathstrut
\begin{theo}
\label{thmspexpregw}%Thm346
$R$を$\mathscr{T}$の関係式とし, $x$を文字とする.
また$n$を自然数とし, $A_{1}, A_{2}, \cdots, A_{n}$を$\mathscr{T}$の関係式とする.
このとき
\[
  \exists_{A_{1} \wedge A_{2} \wedge \cdots \wedge A_{n}}x(R) 
  \to \exists_{A_{1}}x(R) \wedge \exists_{A_{2}}x(R) \wedge \cdots \wedge \exists_{A_{n}}x(R)
\]
は$\mathscr{T}$の定理である.
\end{theo}




\mathstrut
\begin{theo}
\label{thmspexpregwfree}%Thm347
$R$を$\mathscr{T}$の関係式とし, $x$を文字とする.
また$n$を自然数とし, $A_{1}, A_{2}, \cdots, A_{n}$を$\mathscr{T}$の関係式とする.
また$k$と$l$を$k + l = n$なる自然数とし, 
$i_{1}, i_{2}, \cdots, i_{k}, j_{1}, j_{2}, \cdots, j_{l}$を
どの二つも互いに異なる$n$以下の自然数とする.
$x$が$A_{i_{1}}, A_{i_{2}}, \cdots, A_{i_{k}}$の中に自由変数として現れなければ, 
\[
  \exists_{A_{1} \wedge A_{2} \wedge \cdots \wedge A_{n}}x(R) 
  \leftrightarrow A_{i_{1}} \wedge A_{i_{2}} \wedge \cdots \wedge A_{i_{k}} 
  \wedge \exists_{A_{j_{1}} \wedge A_{j_{2}} \wedge \cdots \wedge A_{j_{l}}}x(R)
\]
は$\mathscr{T}$の定理である.
\end{theo}




\mathstrut
\begin{theo}
\label{thmspallpregwedge}%Thm348
$R$を$\mathscr{T}$の関係式とし, $x$を文字とする.
また$n$を自然数とし, $A_{1}, A_{2}, \cdots, A_{n}$を$\mathscr{T}$の関係式とする.
このとき$n$以下の任意の自然数$i$に対し, 
\[
  \forall_{A_{i}}x(R) \to \forall_{A_{1} \wedge A_{2} \wedge \cdots \wedge A_{n}}x(R)
\]
は$\mathscr{T}$の定理である.
\end{theo}




\mathstrut
\begin{theo}
\label{thmspallpregwedge2}%Thm349
$R$を$\mathscr{T}$の関係式とし, $x$を文字とする.
また$n$を自然数とし, $A_{1}, A_{2}, \cdots, A_{n}$を$\mathscr{T}$の関係式とする.
また$k$を自然数とし, $i_{1}, i_{2}, \cdots, i_{k}$を$n$以下の自然数とする.
このとき
\[
  \forall_{A_{i_{1}} \wedge A_{i_{2}} \wedge \cdots \wedge A_{i_{k}}}x(R) 
  \to \forall_{A_{1} \wedge A_{2} \wedge \cdots \wedge A_{n}}x(R)
\]
は$\mathscr{T}$の定理である.
\end{theo}




\mathstrut
\begin{theo}
\label{thmspallpregw}%Thm350
$R$を$\mathscr{T}$の関係式とし, $x$を文字とする.
また$n$を自然数とし, $A_{1}, A_{2}, \cdots, A_{n}$を$\mathscr{T}$の関係式とする.
このとき
\[
  \forall_{A_{1}}x(R) \vee \forall_{A_{2}}x(R) \vee \cdots \vee \forall_{A_{n}}x(R) 
  \to \forall_{A_{1} \wedge A_{2} \wedge \cdots \wedge A_{n}}x(R)
\]
は$\mathscr{T}$の定理である.
\end{theo}




\mathstrut
\begin{theo}
\label{thmspallpregwfree}%Thm351
$R$を$\mathscr{T}$の関係式とし, $x$を文字とする.
また$n$を自然数とし, $A_{1}, A_{2}, \cdots, A_{n}$を$\mathscr{T}$の関係式とする.
また$k$と$l$を$k + l = n$なる自然数とし, 
$i_{1}, i_{2}, \cdots, i_{k}, j_{1}, j_{2}, \cdots, j_{l}$を
どの二つも互いに異なる$n$以下の自然数とする.
$x$が$A_{i_{1}}, A_{i_{2}}, \cdots, A_{i_{k}}$の中に自由変数として現れなければ, 
\[
  \forall_{A_{1} \wedge A_{2} \wedge \cdots \wedge A_{n}}x(R) 
  \leftrightarrow (A_{i_{1}} \wedge A_{i_{2}} \wedge \cdots \wedge A_{i_{k}} 
  \to \forall_{A_{j_{1}} \wedge A_{j_{2}} \wedge \cdots \wedge A_{j_{l}}}x(R))
\]
は$\mathscr{T}$の定理である.
\end{theo}




\mathstrut
\begin{theo}
\label{thmspextspquansep}%Thm352
$A$, $R$, $S$を$\mathscr{T}$の関係式とし, $x$を文字とする.
このとき
\[
  \exists_{A}x(R \to S) \leftrightarrow (\forall_{A}x(R) \to \exists_{A}x(S))
\]
は$\mathscr{T}$の定理である.
\end{theo}




\mathstrut
\begin{theo}
\label{thmspextspquansep2}%Thm353
$A$, $R$, $S$を$\mathscr{T}$の関係式とし, $x$を文字とする.
このとき
\[
  \exists_{A}x(S) \to \exists_{A}x(R \to S), ~~
  \neg \forall_{A}x(R) \to \exists_{A}x(R \to S), ~~
  \exists_{A}x(\neg R) \to \exists_{A}x(R \to S)
\]
はいずれも$\mathscr{T}$の定理である.
\end{theo}




\mathstrut
\begin{theo}
\label{thmspextspquanseprfree}%Thm354
$A$, $R$, $S$を$\mathscr{T}$の関係式とし, $x$を文字とする.
$x$が$R$の中に自由変数として現れなければ, 
\[
  \exists_{A}x(R \to S) \to (R \to \exists_{A}x(S))
\]
は$\mathscr{T}$の定理である.
\end{theo}




\mathstrut
\begin{theo}
\label{thmspextspquanseprfreeeq}%Thm355
$A$, $R$, $S$を$\mathscr{T}$の関係式とし, $x$を文字とする.
$x$が$R$の中に自由変数として現れなければ, 
\[
  \exists x(A) \to (\exists_{A}x(R \to S) \leftrightarrow (R \to \exists_{A}x(S)))
\]
は$\mathscr{T}$の定理である.
\end{theo}




\mathstrut
\begin{theo}
\label{thmspextspquansepsfree}%Thm356
$A$, $R$, $S$を$\mathscr{T}$の関係式とし, $x$を文字とする.
$x$が$S$の中に自由変数として現れなければ, 
\[
  \exists_{A}x(R \to S) \to (\forall_{A}x(R) \to S)
\]
は$\mathscr{T}$の定理である.
\end{theo}




\mathstrut
\begin{theo}
\label{thmspextspquansepsfreeeq}%Thm357
$A$, $R$, $S$を$\mathscr{T}$の関係式とし, $x$を文字とする.
$x$が$S$の中に自由変数として現れなければ, 
\[
  \exists x(A) \to (\exists_{A}x(R \to S) \leftrightarrow (\forall_{A}x(R) \to S))
\]
は$\mathscr{T}$の定理である.
\end{theo}




\mathstrut
\begin{theo}
\label{thmspalltspexsep}%Thm358
$A$, $R$, $S$を$\mathscr{T}$の関係式とし, $x$を文字とする.
このとき
\[
  \forall_{A}x(R \to S) \to (\exists_{A}x(R) \to \exists_{A}x(S))
\]
は$\mathscr{T}$の定理である.
\end{theo}




\mathstrut
\begin{theo}
\label{thmspalltspallsep}%Thm359
$A$, $R$, $S$を$\mathscr{T}$の関係式とし, $x$を文字とする.
このとき
\[
  \forall_{A}x(R \to S) \to (\forall_{A}x(R) \to \forall_{A}x(S))
\]
は$\mathscr{T}$の定理である.
\end{theo}




\mathstrut
\begin{theo}
\label{thmspquanseptspall}%Thm360
$A$, $R$, $S$を$\mathscr{T}$の関係式とし, $x$を文字とする.
このとき
\[
  (\exists_{A}x(R) \to \forall_{A}x(S)) \to \forall_{A}x(R \to S)
\]
は$\mathscr{T}$の定理である.
\end{theo}




\mathstrut
\begin{theo}
\label{thmalltspquansep}%Thm361
$A$, $R$, $S$を$\mathscr{T}$の関係式とし, $x$を文字とする.
このとき
\[
  \forall x(R \to S) \to (\exists_{A}x(R) \to \exists_{A}x(S)), ~~
  \forall x(R \to S) \to (\forall_{A}x(R) \to \forall_{A}x(S))
\]
は共に$\mathscr{T}$の定理である.
\end{theo}




\mathstrut
\begin{theo}
\label{thmspquanseptspall2}%Thm362
$A$, $R$, $S$を$\mathscr{T}$の関係式とし, $x$を文字とする.
このとき
\[
  \forall_{A}x(S) \to \forall_{A}x(R \to S), ~~
  \neg \exists_{A}x(R) \to \forall_{A}x(R \to S), ~~
  \forall_{A}x(\neg R) \to \forall_{A}x(R \to S)
\]
はいずれも$\mathscr{T}$の定理である.
\end{theo}




\mathstrut
\begin{theo}
\label{thmspalltspallseprfree}%Thm363
$A$, $R$, $S$を$\mathscr{T}$の関係式とし, $x$を文字とする.
$x$が$R$の中に自由変数として現れなければ, 
\[
  \forall_{A}x(R \to S) \leftrightarrow (R \to \forall_{A}x(S))
\]
は$\mathscr{T}$の定理である.
\end{theo}




\mathstrut
\begin{theo}
\label{thmspalltspexsepsfree}%Thm364
$A$, $R$, $S$を$\mathscr{T}$の関係式とし, $x$を文字とする.
$x$が$S$の中に自由変数として現れなければ, 
\[
  \forall_{A}x(R \to S) \leftrightarrow (\exists_{A}x(R) \to S)
\]
は$\mathscr{T}$の定理である.
\end{theo}




\mathstrut
\begin{theo}
\label{thmspallpretspexsep}%Thm365
$A$, $B$, $R$を$\mathscr{T}$の関係式とし, $x$を文字とする.
このとき
\[
  \forall_{R}x(A \to B) \to (\exists_{A}x(R) \to \exists_{B}x(R))
\]
は$\mathscr{T}$の定理である.
\end{theo}




\mathstrut
\begin{theo}
\label{thmspallpretspallsep}%Thm366
$A$, $B$, $R$を$\mathscr{T}$の関係式とし, $x$を文字とする.
このとき
\[
  \forall_{\neg R}x(A \to B) \to (\forall_{B}x(R) \to \forall_{A}x(R))
\]
は$\mathscr{T}$の定理である.
\end{theo}




\mathstrut
\begin{theo}
\label{thmallpretspquansep}%Thm367
$A$, $B$, $R$を$\mathscr{T}$の関係式とし, $x$を文字とする.
このとき
\[
  \forall x(A \to B) \to (\exists_{A}x(R) \to \exists_{B}x(R)), ~~
  \forall x(A \to B) \to (\forall_{B}x(R) \to \forall_{A}x(R))
\]
は共に$\mathscr{T}$の定理である.
\end{theo}




\mathstrut
\begin{theo}
\label{thmspquansepeqspex}%Thm368
$A$, $R$, $S$を$\mathscr{T}$の関係式とし, $x$を文字とする.
このとき
\[
  (\exists_{A}x(R) \leftrightarrow \forall_{A}x(S)) \to \exists_{A}x(R \leftrightarrow S), ~~
  (\forall_{A}x(R) \leftrightarrow \exists_{A}x(S)) \to \exists_{A}x(R \leftrightarrow S)
\]
は共に$\mathscr{T}$の定理である.
\end{theo}




\mathstrut
\begin{theo}
\label{thmspquansepeqspexrfree}%Thm369
$A$, $R$, $S$を$\mathscr{T}$の関係式とし, $x$を文字とする.
$x$が$R$の中に自由変数として現れなければ, 
\[
  \exists x(A) \to ((R \leftrightarrow \exists_{A}x(S)) \to \exists_{A}x(R \leftrightarrow S)), ~~
  \exists x(A) \to ((R \leftrightarrow \forall_{A}x(S)) \to \exists_{A}x(R \leftrightarrow S))
\]
は共に$\mathscr{T}$の定理である.
\end{theo}




\mathstrut
\begin{theo}
\label{thmspquansepeqspexsfree}%Thm370
$A$, $R$, $S$を$\mathscr{T}$の関係式とし, $x$を文字とする.
$x$が$S$の中に自由変数として現れなければ, 
\[
  \exists x(A) \to ((\exists_{A}x(R) \leftrightarrow S) \to \exists_{A}x(R \leftrightarrow S)), ~~
  \exists x(A) \to ((\forall_{A}x(R) \leftrightarrow S) \to \exists_{A}x(R \leftrightarrow S))
\]
は共に$\mathscr{T}$の定理である.
\end{theo}




\mathstrut
\begin{theo}
\label{thmspalleqspexsep}%Thm371
$A$, $R$, $S$を$\mathscr{T}$の関係式とし, $x$を文字とする.
このとき
\[
  \forall_{A}x(R \leftrightarrow S) \to (\exists_{A}x(R) \leftrightarrow \exists_{A}x(S))
\]
は$\mathscr{T}$の定理である.
\end{theo}




\mathstrut
\begin{theo}
\label{thmspalleqspallsep}%Thm372
$A$, $R$, $S$を$\mathscr{T}$の関係式とし, $x$を文字とする.
このとき
\[
  \forall_{A}x(R \leftrightarrow S) \to (\forall_{A}x(R) \leftrightarrow \forall_{A}x(S))
\]
は$\mathscr{T}$の定理である.
\end{theo}




\mathstrut
\begin{theo}
\label{thmalleqspquansep}%Thm373
$A$, $R$, $S$を$\mathscr{T}$の関係式とし, $x$を文字とする.
このとき
\[
  \forall x(R \leftrightarrow S) \to (\exists_{A}x(R) \leftrightarrow \exists_{A}x(S)), ~~
  \forall x(R \leftrightarrow S) \to (\forall_{A}x(R) \leftrightarrow \forall_{A}x(S))
\]
は共に$\mathscr{T}$の定理である.
\end{theo}




\mathstrut
\begin{theo}
\label{thmspalleqspquanseprfree}%Thm374
$A$, $R$, $S$を$\mathscr{T}$の関係式とし, $x$を文字とする.
$x$が$R$の中に自由変数として現れなければ, 
\[
  \exists x(A) \to (\forall_{A}x(R \leftrightarrow S) \to (R \leftrightarrow \exists_{A}x(S))), ~~
  \exists x(A) \to (\forall_{A}x(R \leftrightarrow S) \to (R \leftrightarrow \forall_{A}x(S)))
\]
は共に$\mathscr{T}$の定理である.
\end{theo}




\mathstrut
\begin{theo}
\label{thmalleqspquanseprfree}%Thm375
$A$, $R$, $S$を$\mathscr{T}$の関係式とし, $x$を文字とする.
$x$が$R$の中に自由変数として現れなければ, 
\[
  \exists x(A) \to (\forall x(R \leftrightarrow S) \to (R \leftrightarrow \exists_{A}x(S))), ~~
  \exists x(A) \to (\forall x(R \leftrightarrow S) \to (R \leftrightarrow \forall_{A}x(S)))
\]
は共に$\mathscr{T}$の定理である.
\end{theo}




\mathstrut
\begin{theo}
\label{thmspalleqspquansepsfree}%Thm376
$A$, $R$, $S$を$\mathscr{T}$の関係式とし, $x$を文字とする.
$x$が$S$の中に自由変数として現れなければ, 
\[
  \exists x(A) \to (\forall_{A}x(R \leftrightarrow S) \to (\exists_{A}x(R) \leftrightarrow S)), ~~
  \exists x(A) \to (\forall_{A}x(R \leftrightarrow S) \to (\forall_{A}x(R) \leftrightarrow S))
\]
は共に$\mathscr{T}$の定理である.
\end{theo}




\mathstrut
\begin{theo}
\label{thmalleqspquansepsfree}%Thm377
$A$, $R$, $S$を$\mathscr{T}$の関係式とし, $x$を文字とする.
$x$が$S$の中に自由変数として現れなければ, 
\[
  \exists x(A) \to (\forall x(R \leftrightarrow S) \to (\exists_{A}x(R) \leftrightarrow S)), ~~
  \exists x(A) \to (\forall x(R \leftrightarrow S) \to (\forall_{A}x(R) \leftrightarrow S))
\]
は共に$\mathscr{T}$の定理である.
\end{theo}




\mathstrut
\begin{theo}
\label{thmspallpreeqspexsep}%Thm378
$A$, $B$, $R$を$\mathscr{T}$の関係式とし, $x$を文字とする.
このとき
\[
  \forall_{R}x(A \leftrightarrow B) \to (\exists_{A}x(R) \leftrightarrow \exists_{B}x(R))
\]
は$\mathscr{T}$の定理である.
\end{theo}




\mathstrut
\begin{theo}
\label{thmspallpreeqspallsep}%Thm379
$A$, $B$, $R$を$\mathscr{T}$の関係式とし, $x$を文字とする.
このとき
\[
  \forall_{\neg R}x(A \leftrightarrow B) \to (\forall_{A}x(R) \leftrightarrow \forall_{B}x(R))
\]
は$\mathscr{T}$の定理である.
\end{theo}




\mathstrut
\begin{theo}
\label{thmallpreeqspquansep}%Thm380
$A$, $B$, $R$を$\mathscr{T}$の関係式とし, $x$を文字とする.
このとき
\[
  \forall x(A \leftrightarrow B) \to (\exists_{A}x(R) \leftrightarrow \exists_{B}x(R)), ~~
  \forall x(A \leftrightarrow B) \to (\forall_{A}x(R) \leftrightarrow \forall_{B}x(R))
\]
は共に$\mathscr{T}$の定理である.
\end{theo}




\mathstrut
\begin{theo}
\label{thmexspexch}%Thm381
$A$と$R$を$\mathscr{T}$の関係式とし, $x$と$y$を文字とする.
$x$が$A$の中に自由変数として現れなければ, 
\[
  \exists x(\exists_{A}y(R)) \leftrightarrow \exists_{A}y(\exists x(R))
\]
は$\mathscr{T}$の定理である.
\end{theo}




\mathstrut
\begin{theo}
\label{thmallspallch}%Thm382
$A$と$R$を$\mathscr{T}$の関係式とし, $x$と$y$を文字とする.
$x$が$A$の中に自由変数として現れなければ, 
\[
  \forall x(\forall_{A}y(R)) \leftrightarrow \forall_{A}y(\forall x(R))
\]
は$\mathscr{T}$の定理である.
\end{theo}




\mathstrut
\begin{theo}
\label{thmexspallch}%Thm383
$A$と$R$を$\mathscr{T}$の関係式とし, $x$と$y$を文字とする.
$x$が$A$の中に自由変数として現れなければ, 
\[
  \exists x(\forall_{A}y(R)) \to \forall_{A}y(\exists x(R))
\]
は$\mathscr{T}$の定理である.
\end{theo}




\mathstrut
\begin{theo}
\label{thmspexallch}%Thm384
$A$と$R$を$\mathscr{T}$の関係式とし, $x$と$y$を文字とする.
$y$が$A$の中に自由変数として現れなければ, 
\[
  \exists_{A}x(\forall y(R)) \to \forall y(\exists_{A}x(R))
\]
は$\mathscr{T}$の定理である.
\end{theo}




\mathstrut
\begin{theo}
\label{thmspexxeqexx}%Thm385
$A$, $B$, $R$を$\mathscr{T}$の関係式とし, $x$と$y$を文字とする.
$y$が$A$の中に自由変数として現れなければ, 
\[
  \exists_{A}x(\exists_{B}y(R)) \leftrightarrow \exists x(\exists y(A \wedge B \wedge R))
\]
は$\mathscr{T}$の定理である.
\end{theo}




\mathstrut
\begin{theo}
\label{thmspallleqalll}%Thm386
$A$, $B$, $R$を$\mathscr{T}$の関係式とし, $x$と$y$を文字とする.
$y$が$A$の中に自由変数として現れなければ, 
\[
  \forall_{A}x(\forall_{B}y(R)) \leftrightarrow \forall x(\forall y(A \wedge B \to R))
\]
は$\mathscr{T}$の定理である.
\end{theo}




\mathstrut
\begin{theo}
\label{thmspexch}%Thm387
$A$, $B$, $R$を$\mathscr{T}$の関係式とし, $x$と$y$を文字とする.
$x$が$B$の中に自由変数として現れず, $y$が$A$の中に自由変数として現れなければ, 
\[
  \exists_{A}x(\exists_{B}y(R)) \leftrightarrow \exists_{B}y(\exists_{A}x(R))
\]
は$\mathscr{T}$の定理である.
\end{theo}




\mathstrut
\begin{theo}
\label{thmspallch}%Thm388
$A$, $B$, $R$を$\mathscr{T}$の関係式とし, $x$と$y$を文字とする.
$x$が$B$の中に自由変数として現れず, $y$が$A$の中に自由変数として現れなければ, 
\[
  \forall_{A}x(\forall_{B}y(R)) \leftrightarrow \forall_{B}y(\forall_{A}x(R))
\]
は$\mathscr{T}$の定理である.
\end{theo}




\mathstrut
\begin{theo}
\label{thmspquanch}%Thm389
$A$, $B$, $R$を$\mathscr{T}$の関係式とし, $x$と$y$を文字とする.
$x$が$B$の中に自由変数として現れず, $y$が$A$の中に自由変数として現れなければ, 
\[
  \exists_{A}x(\forall_{B}y(R)) \to \forall_{B}y(\exists_{A}x(R))
\]
は$\mathscr{T}$の定理である.
\end{theo}




\mathstrut
\begin{theo}
\label{thmboss1}%Thm390
$R$を$\mathscr{T}$の関係式とする.
また$n$を自然数とし, $A_{1}, A_{2}, \cdots, A_{n}$を$\mathscr{T}$の関係式, 
$x_{1}, x_{2}, \cdots, x_{n}$を文字とする.
$i < n$なる任意の自然数$i$に対し, 
$x_{i + 1}$が$A_{1}, A_{2}, \cdots, A_{i}$の中に自由変数として現れなければ, 
\[
  \exists_{A_{1}}x_{1}(\exists_{A_{2}}x_{2}( \cdots (\exists_{A_{n}}x_{n}(R)) \cdots )) 
  \leftrightarrow \exists x_{1}(\exists x_{2}( \cdots (\exists x_{n}(A_{1} \wedge A_{2} \wedge \cdots \wedge A_{n} \wedge R)) \cdots ))
\]
は$\mathscr{T}$の定理である.
\end{theo}




\mathstrut
\begin{theo}
\label{thmboss2}%Thm391
$R$を$\mathscr{T}$の関係式とする.
また$n$を自然数とし, $A_{1}, A_{2}, \cdots, A_{n}$を$\mathscr{T}$の関係式, 
$x_{1}, x_{2}, \cdots, x_{n}$を文字とする.
$i < n$なる任意の自然数$i$に対し, 
$x_{i + 1}$が$A_{1}, A_{2}, \cdots, A_{i}$の中に自由変数として現れなければ, 
\[
  \forall_{A_{1}}x_{1}(\forall_{A_{2}}x_{2}( \cdots (\forall_{A_{n}}x_{n}(R)) \cdots )) 
  \leftrightarrow \forall x_{1}(\forall x_{2}( \cdots (\forall x_{n}(A_{1} \wedge A_{2} \wedge \cdots \wedge A_{n} \to R)) \cdots ))
\]
は$\mathscr{T}$の定理である.
\end{theo}




\mathstrut
\begin{theo}
\label{thmgspexch}%Thm392
$R$を$\mathscr{T}$の関係式とする.
また$n$を自然数, $A_{1}, A_{2}, \cdots, A_{n}$を$\mathscr{T}$の関係式, 
$x_{1}, x_{2}, \cdots, x_{n}$を文字とし, 
$n$以下の互いに異なる任意の自然数$i$, $j$に対して$x_{i}$は$A_{j}$の中に自由変数として現れないとする.
また自然数$1, 2, \cdots, n$の順序を任意に入れ替えたものを$i_{1}, i_{2}, \cdots, i_{n}$とする.
このとき
\[
  \exists_{A_{1}}x_{1}(\exists_{A_{2}}x_{2}( \cdots (\exists_{A_{n}}x_{n}(R)) \cdots )) 
  \leftrightarrow \exists_{A_{i_{1}}}x_{i_{1}}(\exists_{A_{i_{2}}}x_{i_{2}}( \cdots (\exists_{A_{i_{n}}}x_{i_{n}}(R)) \cdots ))
\]
は$\mathscr{T}$の定理である.
\end{theo}




\mathstrut
\begin{theo}
\label{thmgspallch}%Thm393
$R$を$\mathscr{T}$の関係式とする.
また$n$を自然数, $A_{1}, A_{2}, \cdots, A_{n}$を$\mathscr{T}$の関係式, 
$x_{1}, x_{2}, \cdots, x_{n}$を文字とし, 
$n$以下の互いに異なる任意の自然数$i$, $j$に対して$x_{i}$は$A_{j}$の中に自由変数として現れないとする.
また自然数$1, 2, \cdots, n$の順序を任意に入れ替えたものを$i_{1}, i_{2}, \cdots, i_{n}$とする.
このとき
\[
  \forall_{A_{1}}x_{1}(\forall_{A_{2}}x_{2}( \cdots (\forall_{A_{n}}x_{n}(R)) \cdots )) 
  \leftrightarrow \forall_{A_{i_{1}}}x_{i_{1}}(\forall_{A_{i_{2}}}x_{i_{2}}( \cdots (\forall_{A_{i_{n}}}x_{i_{n}}(R)) \cdots ))
\]
は$\mathscr{T}$の定理である.
\end{theo}




\mathstrut
\begin{theo}
\label{thms5n=}%Thm394
$R$を$\mathscr{T}$の関係式, $T$と$U$を$\mathscr{T}$の対象式とし, 
$x$を文字とする.
このとき
\[
  (T|x)(R) \wedge \neg (U|x)(R) \to T \neq U, ~~
  \neg (T|x)(R) \wedge (U|x)(R) \to T \neq U
\]
は共に$\mathscr{T}$の定理である.
\end{theo}




\mathstrut
\begin{theo}
\label{x=x}%Thm395
$T$を$\mathscr{T}$の対象式とするとき, 
\[
  T = T
\]
は$\mathscr{T}$の定理である.
\end{theo}




\mathstrut
\begin{theo}
\label{allx1x=x1}%Thm396
$x$を文字とするとき, 
\[
  \forall x(x = x)
\]
は$\mathscr{T}$の定理である.
\end{theo}




\mathstrut
\begin{theo}
\label{thmex=}%Thm397
$T$を$\mathscr{T}$の対象式とし, $x$を$T$の中に自由変数として現れない文字とする.
このとき
\[
  \exists x(x = T), ~~
  \exists x(T = x)
\]
は共に$\mathscr{T}$の定理である.
\end{theo}




\mathstrut
\begin{theo}
\label{thmex=2}%Thm398
$x$と$y$を文字とするとき, 
\[
  \exists x(x = y), ~~
  \exists y(x = y)
\]
は共に$\mathscr{T}$の定理である.
\end{theo}




\mathstrut
\begin{theo}
\label{x=yty=x}%Thm399
$T$と$U$を$\mathscr{T}$の対象式とするとき, 
\[
  T = U \to U = T
\]
は$\mathscr{T}$の定理である.
\end{theo}




\mathstrut
\begin{theo}
\label{x=yly=x}%Thm400
$T$と$U$を$\mathscr{T}$の対象式とするとき, 
\[
  T = U \leftrightarrow U = T
\]
は$\mathscr{T}$の定理である.
\end{theo}




\mathstrut
\begin{theo}
\label{xn=ylyn=x}%Thm401
$T$と$U$を$\mathscr{T}$の対象式とするとき, 
\[
  T \neq U \leftrightarrow U \neq T
\]
は$\mathscr{T}$の定理である.
\end{theo}




\mathstrut
\begin{theo}
\label{tau1x=T1=T}%Thm402
$T$を$\mathscr{T}$の対象式とし, $x$を$T$の中に自由変数として現れない文字とする.
このとき
\[
  \tau_{x}(x = T) = T, ~~
  \tau_{x}(T = x) = T
\]
は共に$\mathscr{T}$の定理である.
\end{theo}




\mathstrut
\begin{theo}
\label{thms5eq}%Thm403
$R$を$\mathscr{T}$の関係式, $T$と$U$を$\mathscr{T}$の対象式とし, 
$x$を文字とする.
このとき
\[
  T = U \to ((T|x)(R) \leftrightarrow (U|x)(R))
\]
は$\mathscr{T}$の定理である.
\end{theo}




\mathstrut
\begin{theo}
\label{thmfroms5t}%Thm404
$R$を$\mathscr{T}$の関係式, $T$と$U$を$\mathscr{T}$の対象式とし, 
$x$を文字とする.
このとき
\[
  T = U \wedge (T|x)(R) \to (U|x)(R), ~~
  T = U \wedge (U|x)(R) \to (T|x)(R)
\]
は共に$\mathscr{T}$の定理である.
\end{theo}




\mathstrut
\begin{theo}
\label{thmfroms5eq}%Thm405
$R$を$\mathscr{T}$の関係式, $T$と$U$を$\mathscr{T}$の対象式とし, 
$x$を文字とする.
このとき
\[
  T = U \wedge (T|x)(R) \leftrightarrow T = U \wedge (U|x)(R)
\]
は$\mathscr{T}$の定理である.
\end{theo}




\mathstrut
\begin{theo}
\label{thmgs5}%Thm406
$R$を$\mathscr{T}$の関係式とする.
また$n$を自然数とし, $T_{1}, T_{2}, \cdots, T_{n}, U_{1}, U_{2}, \cdots, U_{n}$を
$\mathscr{T}$の対象式とする.
また$x_{1}, x_{2}, \cdots, x_{n}$を, どの二つも互いに異なる文字とする.
このとき
\begin{multline*}
  T_{1} = U_{1} \wedge T_{2} = U_{2} \wedge \cdots \wedge T_{n} = U_{n} \\
  \to ((T_{1}|x_{1}, T_{2}|x_{2}, \cdots, T_{n}|x_{n})(R) \leftrightarrow (U_{1}|x_{1}, U_{2}|x_{2}, \cdots, U_{n}|x_{n})(R))
\end{multline*}
は$\mathscr{T}$の定理である.
\end{theo}




\mathstrut
\begin{theo}
\label{thmgs52}%Thm407
$R$を$\mathscr{T}$の関係式とする.
また$n$を自然数とし, $T_{1}, T_{2}, \cdots, T_{n}, U_{1}, U_{2}, \cdots, U_{n}$を
$\mathscr{T}$の対象式とする.
また$x_{1}, x_{2}, \cdots, x_{n}$を, どの二つも互いに異なる文字とする.
いま$k$を自然数, $i_{1}, i_{2}, \cdots, i_{k}$を$n$以下の自然数とし, 
$i_{1}, i_{2}, \cdots, i_{k}$のいずれとも異なるような$n$以下の任意の自然数$i$に対して
$T_{i}$と$U_{i}$が同じ記号列であるとする.
このとき
\begin{multline*}
  T_{i_{1}} = U_{i_{1}} \wedge T_{i_{2}} = U_{i_{2}} \wedge \cdots \wedge T_{i_{k}} = U_{i_{k}} \\
  \to ((T_{1}|x_{1}, T_{2}|x_{2}, \cdots, T_{n}|x_{n})(R) \leftrightarrow (U_{1}|x_{1}, U_{2}|x_{2}, \cdots, U_{n}|x_{n})(R))
\end{multline*}
は$\mathscr{T}$の定理である.
\end{theo}




\mathstrut
\begin{theo}
\label{x=ywy=ztx=z}%Thm408
$T$, $U$, $V$を$\mathscr{T}$の対象式とするとき, 
\[
  T = U \wedge U = V \to T = V
\]
は$\mathscr{T}$の定理である.
\end{theo}




\mathstrut
\begin{theo}
\label{x=yt1x=zly=z1}%Thm409
$T$, $U$, $V$を$\mathscr{T}$の対象式とするとき, 
\[
  T = U \to (T = V \leftrightarrow U = V), ~~
  T = U \to (V = T \leftrightarrow V = U)
\]
は共に$\mathscr{T}$の定理である.
\end{theo}




\mathstrut
\begin{theo}
\label{x=ywz=ut1x=zly=u1}%Thm410
$T$, $U$, $V$, $W$を$\mathscr{T}$の対象式とするとき, 
\[
  T = U \wedge V = W \to (T = V \leftrightarrow U = W)
\]
は$\mathscr{T}$の定理である.
\end{theo}




\mathstrut
\begin{theo}
\label{T=Ut1TV=UV1}%Thm411
$T$, $U$, $V$を$\mathscr{T}$の対象式とし, $x$を文字とする.
このとき
\[
  T = U \to (T|x)(V) = (U|x)(V)
\]
は$\mathscr{T}$の定理である.
\end{theo}




\mathstrut
\begin{theo}
\label{thm=gsubst}%Thm412
$n$を自然数とし, $T_{1}, T_{2}, \cdots, T_{n}, U_{1}, U_{2}, \cdots, U_{n}$を
$\mathscr{T}$の対象式とする.
また$x_{1}, x_{2}, \cdots, x_{n}$を, どの二つも互いに異なる文字とする.
また$V$を$\mathscr{T}$の対象式とする.
このとき
\[
  T_{1} = U_{1} \wedge T_{2} = U_{2} \wedge \cdots \wedge T_{n} = U_{n} 
  \to (T_{1}|x_{1}, T_{2}|x_{2}, \cdots, T_{n}|x_{n})(V) = (U_{1}|x_{1}, U_{2}|x_{2}, \cdots, U_{n}|x_{n})(V)
\]
は$\mathscr{T}$の定理である.
\end{theo}




\mathstrut
\begin{theo}
\label{thm=gsubst2}%Thm413
$n$を自然数とし, $T_{1}, T_{2}, \cdots, T_{n}, U_{1}, U_{2}, \cdots, U_{n}$を
$\mathscr{T}$の対象式とする.
また$x_{1}, x_{2}, \cdots, x_{n}$を, どの二つも互いに異なる文字とする.
また$V$を$\mathscr{T}$の対象式とする.
いま$k$を自然数, $i_{1}, i_{2}, \cdots, i_{k}$を$n$以下の自然数とし, 
$i_{1}, i_{2}, \cdots, i_{k}$のいずれとも異なるような$n$以下の任意の自然数$i$に対して
$T_{i}$と$U_{i}$が同じ記号列であるとする.
このとき
\begin{multline*}
  T_{i_{1}} = U_{i_{1}} \wedge T_{i_{2}} = U_{i_{2}} \wedge \cdots \wedge T_{i_{k}} = U_{i_{k}} \\
  \to (T_{1}|x_{1}, T_{2}|x_{2}, \cdots, T_{n}|x_{n})(V) = (U_{1}|x_{1}, U_{2}|x_{2}, \cdots, U_{n}|x_{n})(V)
\end{multline*}
は$\mathscr{T}$の定理である.
\end{theo}




\mathstrut
\begin{theo}
\label{thmspquan=}%Thm414
$R$を$\mathscr{T}$の関係式, $T$を$\mathscr{T}$の対象式とし, 
$x$を$T$の中に自由変数として現れない文字とする.
このとき
\[
  (T|x)(R) \leftrightarrow \exists_{x = T}x(R), ~~
  (T|x)(R) \leftrightarrow \forall_{x = T}x(R)
\]
は共に$\mathscr{T}$の定理である.
\end{theo}




\mathstrut
\begin{theo}
\label{thm!fund}%Thm415
$R$を$\mathscr{T}$の関係式, $T$と$U$を$\mathscr{T}$の対象式とし, 
$x$を文字とする.
このとき
\[
  !x(R) \to ((T|x)(R) \wedge (U|x)(R) \to T = U)
\]
は$\mathscr{T}$の定理である.
\end{theo}




\mathstrut
\begin{theo}
\label{thmn!}%Thm416
$R$を$\mathscr{T}$の関係式, $T$と$U$を$\mathscr{T}$の対象式とし, 
$x$を文字とする.
このとき
\[
  (T|x)(R) \wedge (U|x)(R) \wedge T \neq U \to \neg !x(R)
\]
は$\mathscr{T}$の定理である.
\end{theo}




\mathstrut
\begin{theo}
\label{thm!free}%Thm417
$R$を$\mathscr{T}$の関係式とし, $x$を$R$の中に自由変数として現れない文字とする.
また$y$と$z$を, 互いに異なる文字とする.
このとき
\[
  !x(R) \leftrightarrow (R \to \forall y(\forall z(y = z)))
\]
は$\mathscr{T}$の定理である.
\end{theo}




\mathstrut
\begin{theo}
\label{thm!free2}%Thm418
$R$を$\mathscr{T}$の関係式とし, $x$を$R$の中に自由変数として現れない文字とする.
このとき
\[
  \neg R \to \ !x(R)
\]
は$\mathscr{T}$の定理である.
\end{theo}




\mathstrut
\begin{theo}
\label{thm!tall}%Thm419
$R$を$\mathscr{T}$の関係式とし, $x$を文字とする.
このとき
\[
  !x(R) \to \forall x(R \to x = \tau_{x}(R))
\]
は$\mathscr{T}$の定理である.
\end{theo}




\mathstrut
\begin{theo}
\label{thm!tTtau}%Thm420
$R$を$\mathscr{T}$の関係式, $T$を$\mathscr{T}$の対象式とし, 
$x$を文字とする.
このとき
\[
  !x(R) \to ((T|x)(R) \to T = \tau_{x}(R))
\]
は$\mathscr{T}$の定理である.
\end{theo}




\mathstrut
\begin{theo}
\label{thmallt!}%Thm421
$R$を$\mathscr{T}$の関係式, $T$を$\mathscr{T}$の対象式とし, 
$x$を$T$の中に自由変数として現れない文字とする.
このとき
\[
  \forall x(R \to x = T) \to \ !x(R)
\]
は$\mathscr{T}$の定理である.
\end{theo}




\mathstrut
\begin{theo}
\label{thm!lall}%Thm422
$R$を$\mathscr{T}$の関係式とし, $x$を文字とする.
このとき
\[
  !x(R) \leftrightarrow \forall x(R \to x = \tau_{x}(R))
\]
は$\mathscr{T}$の定理である.
\end{theo}




\mathstrut
\begin{theo}
\label{thm!equiv}%Thm423
$R$を$\mathscr{T}$の関係式とし, $x$を文字とする.
また$y$を$x$と異なり, $R$の中に自由変数として現れない文字とする.
このとき
\[
  !x(R) \leftrightarrow \exists y(\forall x(R \to x = y))
\]
は$\mathscr{T}$の定理である.
\end{theo}




\mathstrut
\begin{theo}
\label{thmnquant!}%Thm424
$R$を$\mathscr{T}$の関係式とし, $x$を文字とする.
このとき
\[
  \neg \exists x(R) \to \ !x(R), ~~
  \forall x(\neg R) \to \ !x(R)
\]
は共に$\mathscr{T}$の定理である.
\end{theo}




\mathstrut
\begin{theo}
\label{thmallt!sep}%Thm425
$R$と$S$を$\mathscr{T}$の関係式とし, $x$を文字とする.
このとき
\[
  \forall x(R \to S) \to (!x(S) \to \ !x(R))
\]
は$\mathscr{T}$の定理である.
\end{theo}




\mathstrut
\begin{theo}
\label{thm!vee}%Thm426
$R$と$S$を$\mathscr{T}$の関係式とし, $x$を文字とする.
このとき
\[
  !x(R \vee S) \to \ !x(R), ~~
  !x(R \vee S) \to \ !x(S)
\]
は共に$\mathscr{T}$の定理である.
\end{theo}




\mathstrut
\begin{theo}
\label{thm!veew}%Thm427
$R$と$S$を$\mathscr{T}$の関係式とし, $x$を文字とする.
このとき
\[
  !x(R \vee S) \to \ !x(R) \, \wedge \, !x(S)
\]
は$\mathscr{T}$の定理である.
\end{theo}




\mathstrut
\begin{theo}
\label{thm!veew2}%Thm428
$R$と$S$を$\mathscr{T}$の関係式とし, $x$を文字とする.
このとき
\[
  \forall x(R \to S) \, \wedge \, !x(S) \to \ !x(R \vee S), ~~
  !x(R) \wedge \forall x(S \to R) \to \ !x(R \vee S)
\]
は共に$\mathscr{T}$の定理である.
\end{theo}




\mathstrut
\begin{theo}
\label{thm!veew3}%Thm429
$R$と$S$を$\mathscr{T}$の関係式とし, $x$を文字とする.
このとき
\begin{align*}
  &\neg \exists x(R) \, \wedge \, !x(S) \to \ !x(R \vee S), ~~
  !x(R) \wedge \neg \exists x(S) \to \ !x(R \vee S), \\
  \mbox{} \\
  &\forall x(\neg R) \, \wedge \, !x(S) \to \ !x(R \vee S), ~~
  !x(R) \wedge \forall x(\neg S) \to \ !x(R \vee S)
\end{align*}
はいずれも$\mathscr{T}$の定理である.
\end{theo}




\mathstrut
\begin{theo}
\label{thm!vfree}%Thm430
$R$と$S$を$\mathscr{T}$の関係式とし, $x$を$R$の中に自由変数として現れない文字とする.
このとき
\[
  \neg R \, \wedge \, !x(S) \to \ !x(R \vee S), ~~
  !x(S) \wedge \neg R \to \ !x(S \vee R)
\]
は共に$\mathscr{T}$の定理である.
\end{theo}




\mathstrut
\begin{theo}
\label{thm!wedge}%Thm431
$R$と$S$を$\mathscr{T}$の関係式とし, $x$を文字とする.
このとき
\[
  !x(R) \to \ !x(R \wedge S), ~~
  !x(S) \to \ !x(R \wedge S)
\]
は共に$\mathscr{T}$の定理である.
\end{theo}




\mathstrut
\begin{theo}
\label{thm!wedgev}%Thm432
$R$と$S$を$\mathscr{T}$の関係式とし, $x$を文字とする.
このとき
\[
  !x(R) \, \vee \, !x(S) \to \ !x(R \wedge S)
\]
は$\mathscr{T}$の定理である.
\end{theo}




\mathstrut
\begin{theo}
\label{thm!wedgev2}%Thm433
$R$と$S$を$\mathscr{T}$の関係式とし, $x$を文字とする.
このとき
\[
  !x(R \wedge S) \to \exists x(\neg R \wedge S) \, \vee \, !x(S), ~~
  !x(R \wedge S) \to \ !x(R) \vee \exists x(R \wedge \neg S)
\]
は共に$\mathscr{T}$の定理である.
\end{theo}




\mathstrut
\begin{theo}
\label{thm!wedgev3}%Thm434
$R$と$S$を$\mathscr{T}$の関係式とし, $x$を文字とする.
このとき
\begin{align*}
  &!x(R \wedge S) \to \exists x(\neg R) \, \vee \, !x(S), ~~
  !x(R \wedge S) \to \ !x(R) \vee \exists x(\neg S), \\
  \mbox{} \\
  &!x(R \wedge S) \to \neg \forall x(R) \, \vee \, !x(S), ~~
  !x(R \wedge S) \to \ !x(R) \vee \neg \forall x(S)
\end{align*}
はいずれも$\mathscr{T}$の定理である.
\end{theo}




\mathstrut
\begin{theo}
\label{thm!wfree}%Thm435
$R$と$S$を$\mathscr{T}$の関係式とし, $x$を$R$の中に自由変数として現れない文字とする.
このとき
\[
  !x(R \wedge S) \leftrightarrow (R \to \ !x(S)), ~~
  !x(S \wedge R) \leftrightarrow (R \to \ !x(S))
\]
は共に$\mathscr{T}$の定理である.
\end{theo}




\mathstrut
\begin{theo}
\label{thm!gvee}%Thm436
$x$を文字とする.
また$n$を自然数とし, $R_{1}, R_{2}, \cdots, R_{n}$を$\mathscr{T}$の関係式とする.
このとき$n$以下の任意の自然数$i$に対し, 
\[
  !x(R_{1} \vee R_{2} \vee \cdots \vee R_{n}) \to \ !x(R_{i})
\]
は$\mathscr{T}$の定理である.
\end{theo}




\mathstrut
\begin{theo}
\label{thm!gvee2}%Thm437
$x$を文字とする.
また$n$を自然数とし, $R_{1}, R_{2}, \cdots, R_{n}$を$\mathscr{T}$の関係式とする.
また$k$を自然数とし, $i_{1}, i_{2}, \cdots, i_{k}$を$n$以下の自然数とする.
このとき
\[
  !x(R_{1} \vee R_{2} \vee \cdots \vee R_{n}) 
  \to \ !x(R_{i_{1}} \vee R_{i_{2}} \vee \cdots \vee R_{i_{k}})
\]
は$\mathscr{T}$の定理である.
\end{theo}




\mathstrut
\begin{theo}
\label{thm!gv}%Thm438
$x$を文字とする.
また$n$を自然数とし, $R_{1}, R_{2}, \cdots, R_{n}$を$\mathscr{T}$の関係式とする.
このとき
\[
  !x(R_{1} \vee R_{2} \vee \cdots \vee R_{n}) 
  \to \ !x(R_{1}) \, \wedge \, !x(R_{2}) \wedge \cdots \wedge \, !x(R_{n})
\]
は$\mathscr{T}$の定理である.
\end{theo}




\mathstrut
\begin{theo}
\label{thm!gv2}%Thm439
$x$を文字とする.
また$n$を自然数とし, $R_{1}, R_{2}, \cdots, R_{n}$を$\mathscr{T}$の関係式とする.
また$i$を$n$以下の自然数とする.
このとき
\begin{multline*}
  \forall x(R_{1} \to R_{i}) \wedge \cdots 
  \wedge \forall x(R_{i - 1} \to R_{i}) \, \wedge \, !x(R_{i}) \wedge \forall x(R_{i + 1} \to R_{i}) \wedge \cdots 
  \wedge \forall x(R_{n} \to R_{i}) \\
  \to \ !x(R_{1} \vee R_{2} \vee \cdots \vee R_{n})
\end{multline*}
は$\mathscr{T}$の定理である.
\end{theo}




\mathstrut
\begin{theo}
\label{thm!gv3}%Thm440
$x$を文字とする.
また$n$を自然数とし, $R_{1}, R_{2}, \cdots, R_{n}$を$\mathscr{T}$の関係式とする.
また$i$を$n$以下の自然数とする.
このとき
\[
  \neg \exists x(R_{1}) \wedge \cdots 
  \wedge \neg \exists x(R_{i - 1}) \, \wedge \, !x(R_{i}) \wedge \neg \exists x(R_{i + 1}) \wedge \cdots 
  \wedge \neg \exists x(R_{n}) 
  \to \ !x(R_{1} \vee R_{2} \vee \cdots \vee R_{n})
\]
は$\mathscr{T}$の定理である.
\end{theo}




\mathstrut
\begin{theo}
\label{thm!gvfree}%Thm441
$x$を文字とする.
また$n$を自然数とし, $R_{1}, R_{2}, \cdots, R_{n}$を$\mathscr{T}$の関係式とする.
また$i$を$n$以下の自然数とする.
$i$と異なる$n$以下の任意の自然数$j$に対し, $x$が$R_{j}$の中に自由変数として現れなければ, 
\[
  \neg R_{1} \wedge \cdots \wedge \neg R_{i - 1} \, \wedge \, !x(R_{i}) \wedge \neg R_{i + 1} \wedge \cdots \wedge \neg R_{n} 
  \to \ !x(R_{1} \vee R_{2} \vee \cdots \vee R_{n})
\]
は$\mathscr{T}$の定理である.
\end{theo}




\mathstrut
\begin{theo}
\label{thm!gwedge}%Thm442
$x$を文字とする.
また$n$を自然数とし, $R_{1}, R_{2}, \cdots, R_{n}$を$\mathscr{T}$の関係式とする.
このとき$n$以下の任意の自然数$i$に対し, 
\[
  !x(R_{i}) \to \ !x(R_{1} \wedge R_{2} \wedge \cdots \wedge R_{n})
\]
は$\mathscr{T}$の定理である.
\end{theo}




\mathstrut
\begin{theo}
\label{thm!gwedge2}%Thm443
$x$を文字とする.
また$n$を自然数とし, $R_{1}, R_{2}, \cdots, R_{n}$を$\mathscr{T}$の関係式とする.
また$k$を自然数とし, $i_{1}, i_{2}, \cdots, i_{k}$を$n$以下の自然数とする.
このとき
\[
  !x(R_{i_{1}} \wedge R_{i_{2}} \wedge \cdots \wedge R_{i_{k}}) 
  \to \ !x(R_{1} \wedge R_{2} \wedge \cdots \wedge R_{n})
\]
は$\mathscr{T}$の定理である.
\end{theo}




\mathstrut
\begin{theo}
\label{thm!gw}%Thm444
$x$を文字とする.
また$n$を自然数とし, $R_{1}, R_{2}, \cdots, R_{n}$を$\mathscr{T}$の関係式とする.
このとき
\[
  !x(R_{1}) \, \vee \, !x(R_{2}) \vee \cdots \vee \, !x(R_{n}) 
  \to \ !x(R_{1} \wedge R_{2} \wedge \cdots \wedge R_{n})
\]
は$\mathscr{T}$の定理である.
\end{theo}




\mathstrut
\begin{theo}
\label{thm!gw2}%Thm445
$x$を文字とする.
また$n$を自然数とし, $R_{1}, R_{2}, \cdots, R_{n}$を$\mathscr{T}$の関係式とする.
また$i$を$n$以下の自然数とする.
このとき
\begin{multline*}
  !x(R_{1} \wedge R_{2} \wedge \cdots \wedge R_{n}) \\
  \to \exists x(R_{i} \wedge \neg R_{1}) \vee \cdots 
  \vee \exists x(R_{i} \wedge \neg R_{i - 1}) \, \vee \, !x(R_{i}) \vee \exists x(R_{i} \wedge \neg R_{i + 1}) \vee \cdots 
  \vee \exists x(R_{i} \wedge \neg R_{n})
\end{multline*}
は$\mathscr{T}$の定理である.
\end{theo}




\mathstrut
\begin{theo}
\label{thm!gw3}%Thm446
$x$を文字とする.
また$n$を自然数とし, $R_{1}, R_{2}, \cdots, R_{n}$を$\mathscr{T}$の関係式とする.
また$i$を$n$以下の自然数とする.
このとき
\[
  !x(R_{1} \wedge R_{2} \wedge \cdots \wedge R_{n}) 
  \to \exists x(\neg R_{1}) \vee \cdots 
  \vee \exists x(\neg R_{i - 1}) \, \vee \, !x(R_{i}) \vee \exists x(\neg R_{i + 1}) \vee \cdots 
  \vee \exists x(\neg R_{n})
\]
は$\mathscr{T}$の定理である.
\end{theo}




\mathstrut
\begin{theo}
\label{thm!gwfree}%Thm447
$x$を文字とする.
また$n$を自然数とし, $R_{1}, R_{2}, \cdots, R_{n}$を$\mathscr{T}$の関係式とする.
また$i$を$n$以下の自然数とする.
$i$と異なる$n$以下の任意の自然数$j$に対し, $x$が$R_{j}$の中に自由変数として現れなければ, 
\[
  !x(R_{1} \wedge R_{2} \wedge \cdots \wedge R_{n}) 
  \leftrightarrow \neg R_{1} \vee \cdots \vee \neg R_{i - 1} \, \vee \, !x(R_{i}) \vee \neg R_{i + 1} \vee \cdots \vee \neg R_{n}
\]
は$\mathscr{T}$の定理である.
\end{theo}




\mathstrut
\begin{theo}
\label{thmalleq!sep}%Thm448
$R$と$S$を$\mathscr{T}$の関係式とし, $x$を文字とする.
このとき
\[
  \forall x(R \leftrightarrow S) \to (!x(R) \leftrightarrow \ !x(S))
\]
は$\mathscr{T}$の定理である.
\end{theo}




\mathstrut
\begin{theo}
\label{thm!tspquansep}%Thm449
$A$と$R$を$\mathscr{T}$の関係式とし, $x$を文字とする.
このとき
\[
  !x(A) \to (\exists_{A}x(R) \to \forall_{A}x(R))
\]
は$\mathscr{T}$の定理である.
\end{theo}




\mathstrut
\begin{theo}
\label{thm!extall!}%Thm450
$R$を$\mathscr{T}$の関係式とし, $x$と$y$を文字とする.
このとき
\[
  !x(\exists y(R)) \to \forall y(!x(R))
\]
は$\mathscr{T}$の定理である.
\end{theo}




\mathstrut
\begin{theo}
\label{thmex!t!all}%Thm451
$R$を$\mathscr{T}$の関係式とし, $x$と$y$を文字とする.
このとき
\[
  \exists x(!y(R)) \to \ !y(\forall x(R))
\]
は$\mathscr{T}$の定理である.
\end{theo}




\mathstrut
\begin{theo}
\label{thm!quanch}%Thm452
$R$を$\mathscr{T}$の関係式とし, $x$と$y$を文字とする.
このとき
\[
  !x(\exists y(R)) \to \exists y(!x(R)), ~~
  \forall y(!x(R)) \to \ !x(\forall y(R))
\]
は共に$\mathscr{T}$の定理である.
\end{theo}




\mathstrut
\begin{theo}
\label{thm!spextspall!}%Thm453
$A$と$R$を$\mathscr{T}$の関係式とし, $x$と$y$を文字とする.
$x$が$A$の中に自由変数として現れなければ, 
\[
  !x(\exists_{A}y(R)) \to \forall_{A}y(!x(R))
\]
は$\mathscr{T}$の定理である.
\end{theo}




\mathstrut
\begin{theo}
\label{thmspex!t!spall}%Thm454
$A$と$R$を$\mathscr{T}$の関係式とし, $x$と$y$を文字とする.
$y$が$A$の中に自由変数として現れなければ, 
\[
  \exists_{A}x(!y(R)) \to \ !y(\forall_{A}x(R))
\]
は$\mathscr{T}$の定理である.
\end{theo}




\mathstrut
\begin{theo}
\label{thm!spexch}%Thm455
$A$と$R$を$\mathscr{T}$の関係式とし, $x$と$y$を文字とする.
$x$が$A$の中に自由変数として現れなければ, 
\[
  \exists y(A) \leftrightarrow (!x(\exists_{A}y(R)) \to \exists_{A}y(!x(R)))
\]
は$\mathscr{T}$の定理である.
\end{theo}




\mathstrut
\begin{theo}
\label{thm!spallch}%Thm456
$A$と$R$を$\mathscr{T}$の関係式とし, $x$と$y$を文字とする.
$x$が$A$の中に自由変数として現れなければ, 
\[
  \exists y(A) \to (\forall_{A}y(!x(R)) \to \ !x(\forall_{A}y(R)))
\]
は$\mathscr{T}$の定理である.
\end{theo}




\mathstrut
\begin{theo}
\label{thm!tsp!}%Thm457
$A$と$R$を$\mathscr{T}$の関係式とし, $x$を文字とする.
このとき
\[
  !x(A) \to \ !_{A}x(R), ~~
  !x(R) \to \ !_{A}x(R)
\]
は共に$\mathscr{T}$の定理である.
\end{theo}




\mathstrut
\begin{theo}
\label{thmnquantsp!}%Thm458
$A$と$R$を$\mathscr{T}$の関係式とし, $x$を文字とする.
このとき
\begin{align*}
  &\neg \exists x(A) \to \ !_{A}x(R), ~~
  \forall x(\neg A) \to \ !_{A}x(R), \\
  \mbox{} \\
  &\neg \exists x(R) \to \ !_{A}x(R), ~~
  \forall x(\neg R) \to \ !_{A}x(R)
\end{align*}
はいずれも$\mathscr{T}$の定理である.
\end{theo}




\mathstrut
\begin{theo}
\label{thmvtsp!}%Thm459
$A$と$R$を$\mathscr{T}$の関係式とし, $x$を文字とする.
このとき
\[
  !x(A) \, \vee \, !x(R) \to \ !_{A}x(R)
\]
は$\mathscr{T}$の定理である.
\end{theo}




\mathstrut
\begin{theo}
\label{thmsp!tv}%Thm460
$A$と$R$を$\mathscr{T}$の関係式とし, $x$を文字とする.
このとき
\[
  !_{A}x(R) \to \exists_{R}x(\neg A) \, \vee \, !x(R), ~~
  !_{A}x(R) \to \ !x(A) \vee \exists_{A}x(\neg R)
\]
は共に$\mathscr{T}$の定理である.
\end{theo}




\mathstrut
\begin{theo}
\label{thmsp!tv2}%Thm461
$A$と$R$を$\mathscr{T}$の関係式とし, $x$を文字とする.
このとき
\begin{align*}
  &!_{A}x(R) \to \exists x(\neg A) \, \vee \, !x(R), ~~
  !_{A}x(R) \to \ !x(A) \vee \exists x(\neg R), \\
  \mbox{} \\
  &!_{A}x(R) \to \neg \forall x(A) \, \vee \, !x(R), ~~
  !_{A}x(R) \to \ !x(A) \vee \neg \forall x(R)
\end{align*}
はいずれも$\mathscr{T}$の定理である.
\end{theo}




\mathstrut
\begin{theo}
\label{thmsp!eq!}%Thm462
$A$と$R$を$\mathscr{T}$の関係式とし, $x$を文字とする.
このとき
\[
  \forall_{R}x(A) \to (!_{A}x(R) \leftrightarrow \ !x(R)), ~~
  \forall_{A}x(R) \to (!_{A}x(R) \leftrightarrow \ !x(A))
\]
は共に$\mathscr{T}$の定理である.
\end{theo}




\mathstrut
\begin{theo}
\label{thmsp!eq!2}%Thm463
$A$と$R$を$\mathscr{T}$の関係式とし, $x$を文字とする.
このとき
\[
  \forall x(A) \to (!_{A}x(R) \leftrightarrow \ !x(R)), ~~
  \forall x(R) \to (!_{A}x(R) \leftrightarrow \ !x(A))
\]
は共に$\mathscr{T}$の定理である.
\end{theo}




\mathstrut
\begin{theo}
\label{thmsp!afree}%Thm464
$A$と$R$を$\mathscr{T}$の関係式とし, $x$を$A$の中に自由変数として現れない文字とする.
このとき
\[
  !_{A}x(R) \leftrightarrow (A \to \ !x(R))
\]
は$\mathscr{T}$の定理である.
\end{theo}




\mathstrut
\begin{theo}
\label{thmsp!afree2}%Thm465
$A$と$R$を$\mathscr{T}$の関係式とし, $x$を$A$の中に自由変数として現れない文字とする.
このとき
\[
  \neg A \to \ !_{A}x(R)
\]
は$\mathscr{T}$の定理である.
\end{theo}




\mathstrut
\begin{theo}
\label{thmsp!rfree}%Thm466
$A$と$R$を$\mathscr{T}$の関係式とし, $x$を$R$の中に自由変数として現れない文字とする.
このとき
\[
  !_{A}x(R) \leftrightarrow \ !x(A) \vee \neg R
\]
は$\mathscr{T}$の定理である.
\end{theo}




\mathstrut
\begin{theo}
\label{thmsp!rfree2}%Thm467
$A$と$R$を$\mathscr{T}$の関係式とし, $x$を$R$の中に自由変数として現れない文字とする.
このとき
\[
  \neg R \to \ !_{A}x(R)
\]
は$\mathscr{T}$の定理である.
\end{theo}




\mathstrut
\begin{theo}
\label{thmsp!rfree3}%Thm468
$A$と$R$を$\mathscr{T}$の関係式とし, $x$を$R$の中に自由変数として現れない文字とする.
このとき
\[
  \neg !x(A) \to (!_{A}x(R) \leftrightarrow \neg R)
\]
は$\mathscr{T}$の定理である.
\end{theo}




\mathstrut
\begin{theo}
\label{thmsp!fund}%Thm469
$A$と$R$を$\mathscr{T}$の関係式, $T$と$U$を$\mathscr{T}$の対象式とし, 
$x$を文字とする.
このとき
\[
  !_{A}x(R) \to ((T|x)(A) \wedge (U|x)(A) \wedge (T|x)(R) \wedge (U|x)(R) \to T = U)
\]
は$\mathscr{T}$の定理である.
\end{theo}




\mathstrut
\begin{theo}
\label{thmnsp!}%Thm470
$A$と$R$を$\mathscr{T}$の関係式, $T$と$U$を$\mathscr{T}$の対象式とし, 
$x$を文字とする.
このとき
\[
  (T|x)(A) \wedge (U|x)(A) \wedge (T|x)(R) \wedge (U|x)(R) \wedge T \neq U \to \neg !_{A}x(R)
\]
は$\mathscr{T}$の定理である.
\end{theo}




\mathstrut
\begin{theo}
\label{thmsp!lspall}%Thm471
$A$と$R$を$\mathscr{T}$の関係式とし, $x$を文字とする.
このとき
\[
  !_{A}x(R) \leftrightarrow \forall_{A}x(R \to x = \tau_{x}(A \wedge R))
\]
は$\mathscr{T}$の定理である.
\end{theo}




\mathstrut
\begin{theo}
\label{thmspalltsp!}%Thm472
$A$と$R$を$\mathscr{T}$の関係式, $T$を$\mathscr{T}$の対象式とし, 
$x$を$T$の中に自由変数として現れない文字とする.
このとき
\[
  \forall_{A}x(R \to x = T) \to \ !_{A}x(R)
\]
は$\mathscr{T}$の定理である.
\end{theo}




\mathstrut
\begin{theo}
\label{thmsp!equiv}%Thm473
$A$と$R$を$\mathscr{T}$の関係式とし, $x$を文字とする.
また$y$を$x$と異なり, $A$及び$R$の中に自由変数として現れない文字とする.
このとき
\[
  !_{A}x(R) \leftrightarrow \exists y(\forall_{A}x(R \to x = y))
\]
は$\mathscr{T}$の定理である.
\end{theo}




\mathstrut
\begin{theo}
\label{thmnspquantsp!}%Thm474
$A$と$R$を$\mathscr{T}$の関係式とし, $x$を文字とする.
このとき
\[
  \neg \exists_{A}x(R) \to \ !_{A}x(R), ~~
  \forall_{A}x(\neg R) \to \ !_{A}x(R)
\]
は共に$\mathscr{T}$の定理である.
\end{theo}




\mathstrut
\begin{theo}
\label{thmsp!tspexn}%Thm475
$A$と$R$を$\mathscr{T}$の関係式とし, $x$を文字とする.
このとき
\[
  \neg !x(A) \to (!_{A}x(R) \to \exists_{A}x(\neg R))
\]
は$\mathscr{T}$の定理である.
\end{theo}




\mathstrut
\begin{theo}
\label{thmsp!ntspex}%Thm476
$A$と$R$を$\mathscr{T}$の関係式とし, $x$を文字とする.
このとき
\[
  \neg !x(A) \to (!_{A}x(\neg R) \to \exists_{A}x(R))
\]
は$\mathscr{T}$の定理である.
\end{theo}




\mathstrut
\begin{theo}
\label{thmspalltsp!sep}%Thm477
$A$, $R$, $S$を$\mathscr{T}$の関係式とし, $x$を文字とする.
このとき
\[
  \forall_{A}x(R \to S) \to (!_{A}x(S) \to \ !_{A}x(R))
\]
は$\mathscr{T}$の定理である.
\end{theo}




\mathstrut
\begin{theo}
\label{thmalltsp!sep}%Thm478
$A$, $R$, $S$を$\mathscr{T}$の関係式とし, $x$を文字とする.
このとき
\[
  \forall x(R \to S) \to (!_{A}x(S) \to \ !_{A}x(R))
\]
は$\mathscr{T}$の定理である.
\end{theo}




\mathstrut
\begin{theo}
\label{thmspallpretsp!sep}%Thm479
$A$, $B$, $R$を$\mathscr{T}$の関係式とし, $x$を文字とする.
このとき
\[
  \forall_{R}x(A \to B) \to (!_{B}x(R) \to \ !_{A}x(R))
\]
は$\mathscr{T}$の定理である.
\end{theo}




\mathstrut
\begin{theo}
\label{thmallpretsp!sep}%Thm480
$A$, $B$, $R$を$\mathscr{T}$の関係式とし, $x$を文字とする.
このとき
\[
  \forall x(A \to B) \to (!_{B}x(R) \to \ !_{A}x(R))
\]
は$\mathscr{T}$の定理である.
\end{theo}




\mathstrut
\begin{theo}
\label{thmsp!vee}%Thm481
$A$, $R$, $S$を$\mathscr{T}$の関係式とし, $x$を文字とする.
このとき
\[
  !_{A}x(R \vee S) \to \ !_{A}x(R), ~~
  !_{A}x(R \vee S) \to \ !_{A}x(S)
\]
は共に$\mathscr{T}$の定理である.
\end{theo}




\mathstrut
\begin{theo}
\label{thmsp!veew}%Thm482
$A$, $R$, $S$を$\mathscr{T}$の関係式とし, $x$を文字とする.
このとき
\[
  !_{A}x(R \vee S) \to \ !_{A}x(R) \, \wedge \, !_{A}x(S)
\]
は$\mathscr{T}$の定理である.
\end{theo}




\mathstrut
\begin{theo}
\label{thmsp!veew2}%Thm483
$A$, $R$, $S$を$\mathscr{T}$の関係式とし, $x$を文字とする.
このとき
\[
  \forall_{A}x(R \to S) \, \wedge \, !_{A}x(S) \to \ !_{A}x(R \vee S), ~~
  !_{A}x(R) \wedge \forall_{A}x(S \to R) \to \ !_{A}x(R \vee S)
\]
は共に$\mathscr{T}$の定理である.
\end{theo}




\mathstrut
\begin{theo}
\label{thmsp!veew3}%Thm484
$A$, $R$, $S$を$\mathscr{T}$の関係式とし, $x$を文字とする.
このとき
\begin{align*}
  &\neg \exists_{A}x(R) \, \wedge \, !_{A}x(S) \to \ !_{A}x(R \vee S), ~~
  !_{A}x(R) \wedge \neg \exists_{A}x(S) \to \ !_{A}x(R \vee S), \\
  \mbox{} \\
  &\forall_{A}x(\neg R) \, \wedge \, !_{A}x(S) \to \ !_{A}x(R \vee S), ~~
  !_{A}x(R) \wedge \forall_{A}x(\neg S) \to \ !_{A}x(R \vee S)
\end{align*}
はいずれも$\mathscr{T}$の定理である.
\end{theo}




\mathstrut
\begin{theo}
\label{thmsp!vfree}%Thm485
$A$, $R$, $S$を$\mathscr{T}$の関係式とし, $x$を$R$の中に自由変数として現れない文字とする.
このとき
\[
  \neg R \, \wedge \, !_{A}x(S) \to \ !_{A}x(R \vee S), ~~
  !_{A}x(S) \wedge \neg R \to \ !_{A}x(S \vee R)
\]
は共に$\mathscr{T}$の定理である.
\end{theo}




\mathstrut
\begin{theo}
\label{thmsp!vfreeeq}%Thm486
$A$, $R$, $S$を$\mathscr{T}$の関係式とし, $x$を$R$の中に自由変数として現れない文字とする.
このとき
\[
  \neg !x(A) \to (!_{A}x(R \vee S) \leftrightarrow \neg R \, \wedge \, !_{A}x(S)), ~~
  \neg !x(A) \to (!_{A}x(S \vee R) \leftrightarrow \ !_{A}x(S) \wedge \neg R)
\]
は共に$\mathscr{T}$の定理である.
\end{theo}




\mathstrut
\begin{theo}
\label{thmsp!wedge}%Thm487
$A$, $R$, $S$を$\mathscr{T}$の関係式とし, $x$を文字とする.
このとき
\[
  !_{A}x(R) \to \ !_{A}x(R \wedge S), ~~
  !_{A}x(S) \to \ !_{A}x(R \wedge S)
\]
は共に$\mathscr{T}$の定理である.
\end{theo}




\mathstrut
\begin{theo}
\label{thmsp!wedgev}%Thm488
$A$, $R$, $S$を$\mathscr{T}$の関係式とし, $x$を文字とする.
このとき
\[
  !_{A}x(R) \, \vee \, !_{A}x(S) \to \ !_{A}x(R \wedge S)
\]
は$\mathscr{T}$の定理である.
\end{theo}




\mathstrut
\begin{theo}
\label{thmsp!wedgev2}%Thm489
$A$, $R$, $S$を$\mathscr{T}$の関係式とし, $x$を文字とする.
このとき
\[
  !_{A}x(R \wedge S) \to \exists_{A}x(\neg R \wedge S) \, \vee \, !_{A}x(S), ~~
  !_{A}x(R \wedge S) \to \ !_{A}x(R) \vee \exists_{A}x(R \wedge \neg S)
\]
は共に$\mathscr{T}$の定理である.
\end{theo}




\mathstrut
\begin{theo}
\label{thmsp!wedgev3}%Thm490
$A$, $R$, $S$を$\mathscr{T}$の関係式とし, $x$を文字とする.
このとき
\begin{align*}
  &!_{A}x(R \wedge S) \to \exists_{A}x(\neg R) \, \vee \, !_{A}x(S), ~~
  !_{A}x(R \wedge S) \to \ !_{A}x(R) \vee \exists_{A}x(\neg S), \\
  \mbox{} \\
  &!_{A}x(R \wedge S) \to \neg \forall_{A}x(R) \, \vee \, !_{A}x(S), ~~
  !_{A}x(R \wedge S) \to \ !_{A}x(R) \vee \neg \forall_{A}x(S)
\end{align*}
はいずれも$\mathscr{T}$の定理である.
\end{theo}




\mathstrut
\begin{theo}
\label{thmsp!wfree}%Thm491
$A$, $R$, $S$を$\mathscr{T}$の関係式とし, $x$を$R$の中に自由変数として現れない文字とする.
このとき
\[
  !_{A}x(R \wedge S) \leftrightarrow \neg R \, \vee \, !_{A}x(S), ~~
  !_{A}x(S \wedge R) \leftrightarrow \ !_{A}x(S) \vee \neg R
\]
は共に$\mathscr{T}$の定理である.
\end{theo}




\mathstrut
\begin{theo}
\label{thmsp!prevee}%Thm492
$A$, $B$, $R$を$\mathscr{T}$の関係式とし, $x$を文字とする.
このとき
\[
  !_{A \vee B}x(R) \to \ !_{A}x(R), ~~
  !_{A \vee B}x(R) \to \ !_{B}x(R)
\]
は共に$\mathscr{T}$の定理である.
\end{theo}




\mathstrut
\begin{theo}
\label{thmsp!prev}%Thm493
$A$, $B$, $R$を$\mathscr{T}$の関係式とし, $x$を文字とする.
このとき
\[
  !_{A \vee B}x(R) \to \ !_{A}x(R) \, \wedge \, !_{B}x(R)
\]
は$\mathscr{T}$の定理である.
\end{theo}




\mathstrut
\begin{theo}
\label{thmsp!prev2}%Thm494
$A$, $B$, $R$を$\mathscr{T}$の関係式とし, $x$を文字とする.
このとき
\[
  \forall_{R}x(A \to B) \, \wedge \, !_{B}x(R) \to \ !_{A \vee B}x(R), ~~
  !_{A}x(R) \wedge \forall_{R}x(B \to A) \to \ !_{A \vee B}x(R)
\]
は共に$\mathscr{T}$の定理である.
\end{theo}




\mathstrut
\begin{theo}
\label{thmsp!prev3}%Thm495
$A$, $B$, $R$を$\mathscr{T}$の関係式とし, $x$を文字とする.
このとき
\begin{align*}
  &\neg \exists_{A}x(R) \, \wedge \, !_{B}x(R) \to \ !_{A \vee B}x(R), ~~
  !_{A}x(R) \wedge \neg \exists_{B}x(R) \to \ !_{A \vee B}x(R), \\
  \mbox{} \\
  &\forall_{A}x(\neg R) \, \wedge \, !_{B}x(R) \to \ !_{A \vee B}x(R), ~~
  !_{A}x(R) \wedge \forall_{B}x(\neg R) \to \ !_{A \vee B}x(R)
\end{align*}
はいずれも$\mathscr{T}$の定理である.
\end{theo}




\mathstrut
\begin{theo}
\label{thmsp!prevfree}%Thm496
$A$, $B$, $R$を$\mathscr{T}$の関係式とし, $x$を$A$の中に自由変数として現れない文字とする.
このとき
\[
  \neg A \, \wedge \, !_{B}x(R) \to \ !_{A \vee B}x(R), ~~
  !_{B}x(R) \wedge \neg A \to \ !_{B \vee A}x(R)
\]
は共に$\mathscr{T}$の定理である.
\end{theo}




\mathstrut
\begin{theo}
\label{thmsp!prevfreeeq}%Thm497
$A$, $B$, $R$を$\mathscr{T}$の関係式とし, $x$を$A$の中に自由変数として現れない文字とする.
このとき
\[
  \neg !x(R) \to (!_{A \vee B}x(R) \leftrightarrow \neg A \, \wedge \, !_{B}x(R)), ~~
  \neg !x(R) \to (!_{B \vee A}x(R) \leftrightarrow \ !_{B}x(R) \wedge \neg A)
\]
は共に$\mathscr{T}$の定理である.
\end{theo}




\mathstrut
\begin{theo}
\label{thmsp!prewedge}%Thm498
$A$, $B$, $R$を$\mathscr{T}$の関係式とし, $x$を文字とする.
このとき
\[
  !_{A}x(R) \to \ !_{A \wedge B}x(R), ~~
  !_{B}x(R) \to \ !_{A \wedge B}x(R)
\]
は共に$\mathscr{T}$の定理である.
\end{theo}




\mathstrut
\begin{theo}
\label{thmsp!prew}%Thm499
$A$, $B$, $R$を$\mathscr{T}$の関係式とし, $x$を文字とする.
このとき
\[
  !_{A}x(R) \, \vee \, !_{B}x(R) \to \ !_{A \wedge B}x(R)
\]
は$\mathscr{T}$の定理である.
\end{theo}




\mathstrut
\begin{theo}
\label{thmsp!prew2}%Thm500
$A$, $B$, $R$を$\mathscr{T}$の関係式とし, $x$を文字とする.
このとき
\[
  !_{A \wedge B}x(R) \to \exists_{R}x(\neg A \wedge B) \, \vee \, !_{B}x(R), ~~
  !_{A \wedge B}x(R) \to \ !_{A}x(R) \vee \exists_{R}x(A \wedge \neg B)
\]
は共に$\mathscr{T}$の定理である.
\end{theo}




\mathstrut
\begin{theo}
\label{thmsp!prew3}%Thm501
$A$, $B$, $R$を$\mathscr{T}$の関係式とし, $x$を文字とする.
このとき
\begin{align*}
  &!_{A \wedge B}x(R) \to \exists_{R}x(\neg A) \, \vee \, !_{B}x(R), ~~
  !_{A \wedge B}x(R) \to \ !_{A}x(R) \vee \exists_{R}x(\neg B), \\
  \mbox{} \\
  &!_{A \wedge B}x(R) \to \neg \forall_{R}x(A) \, \vee \, !_{B}x(R), ~~
  !_{A \wedge B}x(R) \to \ !_{A}x(R) \vee \neg \forall_{R}x(B)
\end{align*}
はいずれも$\mathscr{T}$の定理である.
\end{theo}




\mathstrut
\begin{theo}
\label{thmsp!prewfree}%Thm502
$A$, $B$, $R$を$\mathscr{T}$の関係式とし, $x$を$A$の中に自由変数として現れない文字とする.
このとき
\[
  !_{A \wedge B}x(R) \leftrightarrow (A \to \ !_{B}x(R)), ~~
  !_{B \wedge A}x(R) \leftrightarrow (A \to \ !_{B}x(R))
\]
は共に$\mathscr{T}$の定理である.
\end{theo}




\mathstrut
\begin{theo}
\label{thmsp!gvee}%Thm503
$A$を$\mathscr{T}$の関係式とし, $x$を文字とする.
また$n$を自然数とし, $R_{1}, R_{2}, \cdots, R_{n}$を$\mathscr{T}$の関係式とする.
このとき$n$以下の任意の自然数$i$に対し, 
\[
  !_{A}x(R_{1} \vee R_{2} \vee \cdots \vee R_{n}) \to \ !_{A}x(R_{i})
\]
は$\mathscr{T}$の定理である.
\end{theo}




\mathstrut
\begin{theo}
\label{thmsp!gvee2}%Thm504
$A$を$\mathscr{T}$の関係式とし, $x$を文字とする.
また$n$を自然数とし, $R_{1}, R_{2}, \cdots, R_{n}$を$\mathscr{T}$の関係式とする.
また$k$を自然数とし, $i_{1}, i_{2}, \cdots, i_{k}$を$n$以下の自然数とする.
このとき
\[
  !_{A}x(R_{1} \vee R_{2} \vee \cdots \vee R_{n}) 
  \to \ !_{A}x(R_{i_{1}} \vee R_{i_{2}} \vee \cdots \vee R_{i_{k}})
\]
は$\mathscr{T}$の定理である.
\end{theo}




\mathstrut
\begin{theo}
\label{thmsp!gv}%Thm505
$A$を$\mathscr{T}$の関係式とし, $x$を文字とする.
また$n$を自然数とし, $R_{1}, R_{2}, \cdots, R_{n}$を$\mathscr{T}$の関係式とする.
このとき
\[
  !_{A}x(R_{1} \vee R_{2} \vee \cdots \vee R_{n}) 
  \to \ !_{A}x(R_{1}) \, \wedge \, !_{A}x(R_{2}) \wedge \cdots \wedge \, !_{A}x(R_{n})
\]
は$\mathscr{T}$の定理である.
\end{theo}




\mathstrut
\begin{theo}
\label{thmsp!gv2}%Thm506
$A$を$\mathscr{T}$の関係式とし, $x$を文字とする.
また$n$を自然数とし, $R_{1}, R_{2}, \cdots, R_{n}$を$\mathscr{T}$の関係式とする.
また$i$を$n$以下の自然数とする.
このとき
\begin{multline*}
  \forall_{A}x(R_{1} \to R_{i}) \wedge \cdots 
  \wedge \forall_{A}x(R_{i - 1} \to R_{i}) \, \wedge \, !_{A}x(R_{i}) \wedge \forall_{A}x(R_{i + 1} \to R_{i}) \wedge \cdots 
  \wedge \forall_{A}x(R_{n} \to R_{i}) \\
  \to \ !_{A}x(R_{1} \vee R_{2} \vee \cdots \vee R_{n})
\end{multline*}
は$\mathscr{T}$の定理である.
\end{theo}




\mathstrut
\begin{theo}
\label{thmsp!gv3}%Thm507
$A$を$\mathscr{T}$の関係式とし, $x$を文字とする.
また$n$を自然数とし, $R_{1}, R_{2}, \cdots, R_{n}$を$\mathscr{T}$の関係式とする.
また$i$を$n$以下の自然数とする.
このとき
\[
  \neg \exists_{A}x(R_{1}) \wedge \cdots 
  \wedge \neg \exists_{A}x(R_{i - 1}) \, \wedge \, !_{A}x(R_{i}) \wedge \neg \exists_{A}x(R_{i + 1}) \wedge \cdots 
  \wedge \neg \exists_{A}x(R_{n}) 
  \to \ !_{A}x(R_{1} \vee R_{2} \vee \cdots \vee R_{n})
\]
は$\mathscr{T}$の定理である.
\end{theo}




\mathstrut
\begin{theo}
\label{thmsp!gvfree}%Thm508
$A$を$\mathscr{T}$の関係式とし, $x$を文字とする.
また$n$を自然数とし, $R_{1}, R_{2}, \cdots, R_{n}$を$\mathscr{T}$の関係式とする.
また$i$を$n$以下の自然数とする.
$i$と異なる$n$以下の任意の自然数$j$に対し, $x$が$R_{j}$の中に自由変数として現れなければ, 
\[
  \neg R_{1} \wedge \cdots \wedge \neg R_{i - 1} \, \wedge \, !_{A}x(R_{i}) \wedge \neg R_{i + 1} \wedge \cdots \wedge \neg R_{n} 
  \to \ !_{A}x(R_{1} \vee R_{2} \vee \cdots \vee R_{n})
\]
は$\mathscr{T}$の定理である.
\end{theo}




\mathstrut
\begin{theo}
\label{thmsp!gvfreeeq}%Thm509
$A$を$\mathscr{T}$の関係式とし, $x$を文字とする.
また$n$を自然数とし, $R_{1}, R_{2}, \cdots, R_{n}$を$\mathscr{T}$の関係式とする.
また$i$を$n$以下の自然数とする.
$i$と異なる$n$以下の任意の自然数$j$に対し, $x$が$R_{j}$の中に自由変数として現れなければ, 
\[
  \neg !x(A) 
  \to (!_{A}x(R_{1} \vee R_{2} \vee \cdots \vee R_{n}) 
  \leftrightarrow \neg R_{1} \wedge \cdots \wedge \neg R_{i - 1} \, \wedge \, !_{A}x(R_{i}) \wedge \neg R_{i + 1} \wedge \cdots \wedge \neg R_{n})
\]
は$\mathscr{T}$の定理である.
\end{theo}




\mathstrut
\begin{theo}
\label{thmsp!gwedge}%Thm510
$A$を$\mathscr{T}$の関係式とし, $x$を文字とする.
また$n$を自然数とし, $R_{1}, R_{2}, \cdots, R_{n}$を$\mathscr{T}$の関係式とする.
このとき$n$以下の任意の自然数$i$に対し, 
\[
  !_{A}x(R_{i}) \to \ !_{A}x(R_{1} \wedge R_{2} \wedge \cdots \wedge R_{n})
\]
は$\mathscr{T}$の定理である.
\end{theo}




\mathstrut
\begin{theo}
\label{thmsp!gwedge2}%Thm511
$A$を$\mathscr{T}$の関係式とし, $x$を文字とする.
また$n$を自然数とし, $R_{1}, R_{2}, \cdots, R_{n}$を$\mathscr{T}$の関係式とする.
また$k$を自然数とし, $i_{1}, i_{2}, \cdots, i_{k}$を$n$以下の自然数とする.
このとき
\[
  !_{A}x(R_{i_{1}} \wedge R_{i_{2}} \wedge \cdots \wedge R_{i_{k}}) 
  \to \ !_{A}x(R_{1} \wedge R_{2} \wedge \cdots \wedge R_{n})
\]
は$\mathscr{T}$の定理である.
\end{theo}




\mathstrut
\begin{theo}
\label{thmsp!gw}%Thm512
$A$を$\mathscr{T}$の関係式とし, $x$を文字とする.
また$n$を自然数とし, $R_{1}, R_{2}, \cdots, R_{n}$を$\mathscr{T}$の関係式とする.
このとき
\[
  !_{A}x(R_{1}) \, \vee \, !_{A}x(R_{2}) \vee \cdots \vee \, !_{A}x(R_{n}) 
  \to \ !_{A}x(R_{1} \wedge R_{2} \wedge \cdots \wedge R_{n})
\]
は$\mathscr{T}$の定理である.
\end{theo}




\mathstrut
\begin{theo}
\label{thmsp!gw2}%Thm513
$A$を$\mathscr{T}$の関係式とし, $x$を文字とする.
また$n$を自然数とし, $R_{1}, R_{2}, \cdots, R_{n}$を$\mathscr{T}$の関係式とする.
また$i$を$n$以下の自然数とする.
このとき
\begin{multline*}
  !_{A}x(R_{1} \wedge R_{2} \wedge \cdots \wedge R_{n}) \\
  \to \exists_{A}x(R_{i} \wedge \neg R_{1}) \vee \cdots 
  \vee \exists_{A}x(R_{i} \wedge \neg R_{i - 1}) \, \vee \, !_{A}x(R_{i}) \vee \exists_{A}x(R_{i} \wedge \neg R_{i + 1}) \vee \cdots 
  \vee \exists_{A}x(R_{i} \wedge \neg R_{n})
\end{multline*}
は$\mathscr{T}$の定理である.
\end{theo}




\mathstrut
\begin{theo}
\label{thmsp!gw3}%Thm514
$A$を$\mathscr{T}$の関係式とし, $x$を文字とする.
また$n$を自然数とし, $R_{1}, R_{2}, \cdots, R_{n}$を$\mathscr{T}$の関係式とする.
また$i$を$n$以下の自然数とする.
このとき
\[
  !_{A}x(R_{1} \wedge R_{2} \wedge \cdots \wedge R_{n}) 
  \to \exists_{A}x(\neg R_{1}) \vee \cdots 
  \vee \exists_{A}x(\neg R_{i - 1}) \, \vee \, !_{A}x(R_{i}) \vee \exists_{A}x(\neg R_{i + 1}) \vee \cdots 
  \vee \exists_{A}x(\neg R_{n})
\]
は$\mathscr{T}$の定理である.
\end{theo}




\mathstrut
\begin{theo}
\label{thmsp!gwfree}%Thm515
$A$を$\mathscr{T}$の関係式とし, $x$を文字とする.
また$n$を自然数とし, $R_{1}, R_{2}, \cdots, R_{n}$を$\mathscr{T}$の関係式とする.
また$i$を$n$以下の自然数とする.
$i$と異なる$n$以下の任意の自然数$j$に対し, $x$が$R_{j}$の中に自由変数として現れなければ, 
\[
  !_{A}x(R_{1} \wedge R_{2} \wedge \cdots \wedge R_{n}) 
  \leftrightarrow \neg R_{1} \vee \cdots \vee \neg R_{i - 1} \, \vee \, !_{A}x(R_{i}) \vee \neg R_{i + 1} \vee \cdots \vee \neg R_{n}
\]
は$\mathscr{T}$の定理である.
\end{theo}




\mathstrut
\begin{theo}
\label{thmsp!pregvee}%Thm516
$R$を$\mathscr{T}$の関係式とし, $x$を文字とする.
また$n$を自然数とし, $A_{1}, A_{2}, \cdots, A_{n}$を$\mathscr{T}$の関係式とする.
このとき$n$以下の任意の自然数$i$に対し, 
\[
  !_{A_{1} \vee A_{2} \vee \cdots \vee A_{n}}x(R) \to \ !_{A_{i}}x(R)
\]
は$\mathscr{T}$の定理である.
\end{theo}




\mathstrut
\begin{theo}
\label{thmsp!pregvee2}%Thm517
$R$を$\mathscr{T}$の関係式とし, $x$を文字とする.
また$n$を自然数とし, $A_{1}, A_{2}, \cdots, A_{n}$を$\mathscr{T}$の関係式とする.
また$k$を自然数とし, $i_{1}, i_{2}, \cdots, i_{k}$を$n$以下の自然数とする.
このとき
\[
  !_{A_{1} \vee A_{2} \vee \cdots \vee A_{n}}x(R) 
  \to \ !_{A_{i_{1}} \vee A_{i_{2}} \vee \cdots \vee A_{i_{k}}}x(R)
\]
は$\mathscr{T}$の定理である.
\end{theo}




\mathstrut
\begin{theo}
\label{thmsp!pregv}%Thm518
$R$を$\mathscr{T}$の関係式とし, $x$を文字とする.
また$n$を自然数とし, $A_{1}, A_{2}, \cdots, A_{n}$を$\mathscr{T}$の関係式とする.
このとき
\[
  !_{A_{1} \vee A_{2} \vee \cdots \vee A_{n}}x(R) 
  \to \ !_{A_{1}}x(R) \, \wedge \, !_{A_{2}}x(R) \wedge \cdots \wedge \, !_{A_{n}}x(R)
\]
は$\mathscr{T}$の定理である.
\end{theo}




\mathstrut
\begin{theo}
\label{thmsp!pregv2}%Thm519
$R$を$\mathscr{T}$の関係式とし, $x$を文字とする.
また$n$を自然数とし, $A_{1}, A_{2}, \cdots, A_{n}$を$\mathscr{T}$の関係式とする.
また$i$を$n$以下の自然数とする.
このとき
\begin{multline*}
  \forall_{R}x(A_{1} \to A_{i}) \wedge \cdots 
  \wedge \forall_{R}x(A_{i - 1} \to A_{i}) \, \wedge \, !_{A_{i}}x(R) \wedge \forall_{R}x(A_{i + 1} \to A_{i}) \wedge \cdots 
  \wedge \forall_{R}x(A_{n} \to A_{i}) \\
  \to \ !_{A_{1} \vee A_{2} \vee \cdots \vee A_{n}}x(R)
\end{multline*}
は$\mathscr{T}$の定理である.
\end{theo}




\mathstrut
\begin{theo}
\label{thmsp!pregv3}%Thm520
$R$を$\mathscr{T}$の関係式とし, $x$を文字とする.
また$n$を自然数とし, $A_{1}, A_{2}, \cdots, A_{n}$を$\mathscr{T}$の関係式とする.
また$i$を$n$以下の自然数とする.
このとき
\[
  \neg \exists_{A_{1}}x(R) \wedge \cdots 
  \wedge \neg \exists_{A_{i - 1}}x(R) \, \wedge \, !_{A_{i}}x(R) \wedge \neg \exists_{A_{i + 1}}x(R) \wedge \cdots 
  \wedge \neg \exists_{A_{n}}x(R) 
  \to \ !_{A_{1} \vee A_{2} \vee \cdots \vee A_{n}}x(R)
\]
は$\mathscr{T}$の定理である.
\end{theo}




\mathstrut
\begin{theo}
\label{thmsp!pregvfree}%Thm521
$R$を$\mathscr{T}$の関係式とし, $x$を文字とする.
また$n$を自然数とし, $A_{1}, A_{2}, \cdots, A_{n}$を$\mathscr{T}$の関係式とする.
また$i$を$n$以下の自然数とする.
$i$と異なる$n$以下の任意の自然数$j$に対し, $x$が$A_{j}$の中に自由変数として現れなければ, 
\[
  \neg A_{1} \wedge \cdots 
  \wedge \neg A_{i - 1} \, \wedge \, !_{A_{i}}x(R) \wedge \neg A_{i + 1} \wedge \cdots 
  \wedge \neg A_{n} 
  \to \ !_{A_{1} \vee A_{2} \vee \cdots \vee A_{n}}x(R)
\]
は$\mathscr{T}$の定理である.
\end{theo}




\mathstrut
\begin{theo}
\label{thmsp!pregvfreeeq}%Thm522
$R$を$\mathscr{T}$の関係式とし, $x$を文字とする.
また$n$を自然数とし, $A_{1}, A_{2}, \cdots, A_{n}$を$\mathscr{T}$の関係式とする.
また$i$を$n$以下の自然数とする.
$i$と異なる$n$以下の任意の自然数$j$に対し, $x$が$A_{j}$の中に自由変数として現れなければ, 
\[
  \neg !x(R) 
  \to (!_{A_{1} \vee A_{2} \vee \cdots \vee A_{n}}x(R) 
  \leftrightarrow \neg A_{1} \wedge \cdots \wedge \neg A_{i - 1} \, \wedge \, !_{A_{i}}x(R) \wedge \neg A_{i + 1} \wedge \cdots \wedge \neg A_{n})
\]
は$\mathscr{T}$の定理である.
\end{theo}




\mathstrut
\begin{theo}
\label{thmsp!pregwedge}%Thm523
$R$を$\mathscr{T}$の関係式とし, $x$を文字とする.
また$n$を自然数とし, $A_{1}, A_{2}, \cdots, A_{n}$を$\mathscr{T}$の関係式とする.
このとき$n$以下の任意の自然数$i$に対し, 
\[
  !_{A_{i}}x(R) \to \ !_{A_{1} \wedge A_{2} \wedge \cdots \wedge A_{n}}x(R)
\]
は$\mathscr{T}$の定理である.
\end{theo}




\mathstrut
\begin{theo}
\label{thmsp!pregwedge2}%Thm524
$R$を$\mathscr{T}$の関係式とし, $x$を文字とする.
また$n$を自然数とし, $A_{1}, A_{2}, \cdots, A_{n}$を$\mathscr{T}$の関係式とする.
また$k$を自然数とし, $i_{1}, i_{2}, \cdots, i_{k}$を$n$以下の自然数とする.
このとき
\[
  !_{A_{i_{1}} \wedge A_{i_{2}} \wedge \cdots \wedge A_{i_{k}}}x(R) 
  \to \ !_{A_{1} \wedge A_{2} \wedge \cdots \wedge A_{n}}x(R)
\]
は$\mathscr{T}$の定理である.
\end{theo}




\mathstrut
\begin{theo}
\label{thmsp!pregw}%Thm525
$R$を$\mathscr{T}$の関係式とし, $x$を文字とする.
また$n$を自然数とし, $A_{1}, A_{2}, \cdots, A_{n}$を$\mathscr{T}$の関係式とする.
このとき
\[
  !_{A_{1}}x(R) \, \vee \, !_{A_{2}}x(R) \vee \cdots \vee \, !_{A_{n}}x(R) 
  \to \ !_{A_{1} \wedge A_{2} \wedge \cdots \wedge A_{n}}x(R)
\]
は$\mathscr{T}$の定理である.
\end{theo}




\mathstrut
\begin{theo}
\label{thmsp!pregw2}%Thm526
$R$を$\mathscr{T}$の関係式とし, $x$を文字とする.
また$n$を自然数とし, $A_{1}, A_{2}, \cdots, A_{n}$を$\mathscr{T}$の関係式とする.
また$i$を$n$以下の自然数とする.
このとき
\begin{multline*}
  !_{A_{1} \wedge A_{2} \wedge \cdots \wedge A_{n}}x(R) \\
  \to \exists_{R}x(A_{i} \wedge \neg A_{1}) \vee \cdots 
  \vee \exists_{R}x(A_{i} \wedge \neg A_{i - 1}) \, \vee \, !_{A_{i}}x(R) \vee \exists_{R}x(A_{i} \wedge \neg A_{i + 1}) \vee \cdots 
  \vee \exists_{R}x(A_{i} \wedge \neg A_{n})
\end{multline*}
は$\mathscr{T}$の定理である.
\end{theo}




\mathstrut
\begin{theo}
\label{thmsp!pregw3}%Thm527
$R$を$\mathscr{T}$の関係式とし, $x$を文字とする.
また$n$を自然数とし, $A_{1}, A_{2}, \cdots, A_{n}$を$\mathscr{T}$の関係式とする.
また$i$を$n$以下の自然数とする.
このとき
\[
  !_{A_{1} \wedge A_{2} \wedge \cdots \wedge A_{n}}x(R) 
  \to \exists_{R}x(\neg A_{1}) \vee \cdots 
  \vee \exists_{R}x(\neg A_{i - 1}) \, \vee \, !_{A_{i}}x(R) \vee \exists_{R}x(\neg A_{i + 1}) \vee \cdots 
  \vee \exists_{R}x(\neg A_{n})
\]
は$\mathscr{T}$の定理である.
\end{theo}




\mathstrut
\begin{theo}
\label{thmsp!pregwfree}%Thm528
$R$を$\mathscr{T}$の関係式とし, $x$を文字とする.
また$n$を自然数とし, $A_{1}, A_{2}, \cdots, A_{n}$を$\mathscr{T}$の関係式とする.
また$i$を$n$以下の自然数とする.
$i$と異なる$n$以下の任意の自然数$j$に対し, $x$が$A_{j}$の中に自由変数として現れなければ, 
\[
  !_{A_{1} \wedge A_{2} \wedge \cdots \wedge A_{n}}x(R) 
  \leftrightarrow \neg A_{1} \vee \cdots \vee \neg A_{i - 1} \, \vee \, !_{A_{i}}x(R) \vee \neg A_{i + 1} \vee \cdots \vee \neg A_{n}
\]
は$\mathscr{T}$の定理である.
\end{theo}




\mathstrut
\begin{theo}
\label{thmspalleqsp!sep}%Thm529
$A$, $R$, $S$を$\mathscr{T}$の関係式とし, $x$を文字とする.
このとき
\[
  \forall_{A}x(R \leftrightarrow S) \to (!_{A}x(R) \leftrightarrow \ !_{A}x(S))
\]
は$\mathscr{T}$の定理である.
\end{theo}




\mathstrut
\begin{theo}
\label{thmalleqsp!sep}%Thm530
$A$, $R$, $S$を$\mathscr{T}$の関係式とし, $x$を文字とする.
このとき
\[
  \forall x(R \leftrightarrow S) \to (!_{A}x(R) \leftrightarrow \ !_{A}x(S))
\]
は$\mathscr{T}$の定理である.
\end{theo}




\mathstrut
\begin{theo}
\label{thmspallpreeqsp!sep}%Thm531
$A$, $B$, $R$を$\mathscr{T}$の関係式とし, $x$を文字とする.
このとき
\[
  \forall_{R}x(A \leftrightarrow B) \to (!_{A}x(R) \leftrightarrow \ !_{B}x(R))
\]
は$\mathscr{T}$の定理である.
\end{theo}




\mathstrut
\begin{theo}
\label{thmallpreeqsp!sep}%Thm532
$A$, $B$, $R$を$\mathscr{T}$の関係式とし, $x$を文字とする.
このとき
\[
  \forall x(A \leftrightarrow B) \to (!_{A}x(R) \leftrightarrow \ !_{B}x(R))
\]
は$\mathscr{T}$の定理である.
\end{theo}




\mathstrut
\begin{theo}
\label{thmsp!extallsp!}%Thm533
$A$と$R$を$\mathscr{T}$の関係式とし, $x$と$y$を文字とする.
$y$が$A$の中に自由変数として現れなければ, 
\[
  !_{A}x(\exists y(R)) \to \forall y(!_{A}x(R))
\]
は$\mathscr{T}$の定理である.
\end{theo}




\mathstrut
\begin{theo}
\label{thmexsp!tsp!all}%Thm534
$A$と$R$を$\mathscr{T}$の関係式とし, $x$と$y$を文字とする.
$x$が$A$の中に自由変数として現れなければ, 
\[
  \exists x(!_{A}y(R)) \to \ !_{A}y(\forall x(R))
\]
は$\mathscr{T}$の定理である.
\end{theo}




\mathstrut
\begin{theo}
\label{thmsp!quanch}%Thm535
$A$と$R$を$\mathscr{T}$の関係式とし, $x$と$y$を文字とする.
$y$が$A$の中に自由変数として現れなければ, 
\[
  !_{A}x(\exists y(R)) \to \exists y(!_{A}x(R)), ~~
  \forall y(!_{A}x(R)) \to \ !_{A}x(\forall y(R))
\]
は共に$\mathscr{T}$の定理である.
\end{theo}




\mathstrut
\begin{theo}
\label{thmsp!spextspallsp!}%Thm536
$A$, $B$, $R$を$\mathscr{T}$の関係式とし, $x$と$y$を文字とする.
$x$が$B$の中に自由変数として現れず, $y$が$A$の中に自由変数として現れなければ, 
\[
  !_{A}x(\exists_{B}y(R)) \to \forall_{B}y(!_{A}x(R))
\]
は$\mathscr{T}$の定理である.
\end{theo}




\mathstrut
\begin{theo}
\label{thmspexsp!tsp!spall}%Thm537
$A$, $B$, $R$を$\mathscr{T}$の関係式とし, $x$と$y$を文字とする.
$x$が$B$の中に自由変数として現れず, $y$が$A$の中に自由変数として現れなければ, 
\[
  \exists_{A}x(!_{B}y(R)) \to \ !_{B}y(\forall_{A}x(R))
\]
は$\mathscr{T}$の定理である.
\end{theo}




\mathstrut
\begin{theo}
\label{thmsp!spexch}%Thm538
$A$, $B$, $R$を$\mathscr{T}$の関係式とし, $x$と$y$を文字とする.
$x$が$B$の中に自由変数として現れず, $y$が$A$の中に自由変数として現れなければ, 
\[
  \exists y(B) \leftrightarrow (!_{A}x(\exists_{B}y(R)) \to \exists_{B}y(!_{A}x(R)))
\]
は$\mathscr{T}$の定理である.
\end{theo}




\mathstrut
\begin{theo}
\label{thmsp!spallch}%Thm539
$A$, $B$, $R$を$\mathscr{T}$の関係式とし, $x$と$y$を文字とする.
$x$が$B$の中に自由変数として現れず, $y$が$A$の中に自由変数として現れなければ, 
\[
  \exists y(B) \to (\forall_{B}y(!_{A}x(R)) \to \ !_{A}x(\forall_{B}y(R)))
\]
は$\mathscr{T}$の定理である.
\end{theo}




\mathstrut
\begin{theo}
\label{thmsp!spallcheq}%Thm540
$A$, $B$, $R$を$\mathscr{T}$の関係式とし, $x$と$y$を文字とする.
$x$が$B$の中に自由変数として現れず, $y$が$A$の中に自由変数として現れなければ, 
\[
  \neg !x(A) \to (\exists y(B) \leftrightarrow (\forall_{B}y(!_{A}x(R)) \to \ !_{A}x(\forall_{B}y(R))))
\]
は$\mathscr{T}$の定理である.
\end{theo}




\mathstrut
\begin{theo}
\label{thm!lnexvex!}%Thm541
$R$を$\mathscr{T}$の関係式とし, $x$を文字とする.
このとき
\[
  !x(R) \leftrightarrow \neg \exists x(R) \vee \exists !x(R)
\]
は$\mathscr{T}$の定理である.
\end{theo}




\mathstrut
\begin{theo}
\label{thmex!free}%Thm542
$R$を$\mathscr{T}$の関係式とし, $x$を$R$の中に自由変数として現れない文字とする.
また$y$と$z$を, 互いに異なる文字とする.
このとき
\[
  \exists !x(R) \leftrightarrow R \wedge \forall y(\forall z(y = z))
\]
は$\mathscr{T}$の定理である.
\end{theo}




\mathstrut
\begin{theo}
\label{thmex!lall}%Thm543
$R$を$\mathscr{T}$の関係式とし, $x$を文字とする.
このとき
\[
  \exists !x(R) \leftrightarrow \forall x(R \leftrightarrow x = \tau_{x}(R))
\]
は$\mathscr{T}$の定理である.
\end{theo}




\mathstrut
\begin{theo}
\label{thmex!tTtau}%Thm544
$R$を$\mathscr{T}$の関係式, $T$を$\mathscr{T}$の対象式とし, 
$x$を文字とする.
このとき
\[
  \exists !x(R) \to ((T|x)(R) \leftrightarrow T = \tau_{x}(R))
\]
は$\mathscr{T}$の定理である.
\end{theo}




\mathstrut
\begin{theo}
\label{thmalltex!}%Thm545
$R$を$\mathscr{T}$の関係式, $T$を$\mathscr{T}$の対象式とし, 
$x$を$T$の中に自由変数として現れない文字とする.
このとき
\[
  \forall x(R \leftrightarrow x = T) \to \exists !x(R)
\]
は$\mathscr{T}$の定理である.
\end{theo}




\mathstrut
\begin{theo}
\label{thmalltT=tau}%Thm546
$R$を$\mathscr{T}$の関係式, $T$を$\mathscr{T}$の対象式とし, 
$x$を$T$の中に自由変数として現れない文字とする.
このとき
\[
  \forall x(R \leftrightarrow x = T) \to T = \tau_{x}(R)
\]
は$\mathscr{T}$の定理である.
\end{theo}




\mathstrut
\begin{theo}
\label{thmex!equiv}%Thm547
$R$を$\mathscr{T}$の関係式とし, $x$を文字とする.
また$y$を$x$と異なり, $R$の中に自由変数として現れない文字とする.
このとき
\[
  \exists !x(R) \leftrightarrow \exists y(\forall x(R \leftrightarrow x = y))
\]
は$\mathscr{T}$の定理である.
\end{theo}




\mathstrut
\begin{theo}
\label{thmex!equiv2}%Thm548
$R$を$\mathscr{T}$の関係式とし, $x$を文字とする.
また$y$を$x$と異なり, $R$の中に自由変数として現れない文字とする.
このとき
\[
  \exists !x(R) \leftrightarrow \exists y((y|x)(R) \wedge \forall x(R \to x = y))
\]
は$\mathscr{T}$の定理である.
\end{theo}




\mathstrut
\begin{theo}
\label{thmalleqex!sep}%Thm549
$R$と$S$を$\mathscr{T}$の関係式とし, $x$を文字とする.
このとき
\[
  \forall x(R \leftrightarrow S) \to (\exists !x(R) \leftrightarrow \exists !x(S))
\]
は$\mathscr{T}$の定理である.
\end{theo}




\mathstrut
\begin{theo}
\label{thmex!tspquaneq}%Thm550
$A$と$R$を$\mathscr{T}$の関係式とし, $x$を文字とする.
このとき
\[
  \exists !x(A) \to (\exists_{A}x(R) \leftrightarrow \forall_{A}x(R))
\]
は$\mathscr{T}$の定理である.
\end{theo}




\mathstrut
\begin{theo}
\label{thmex!ttauspexeq}%Thm551
$A$と$R$を$\mathscr{T}$の関係式とし, $x$を文字とする.
このとき
\[
  \exists !x(A) \to ((\tau_{x}(A)|x)(R) \leftrightarrow \exists_{A}x(R))
\]
は$\mathscr{T}$の定理である.
\end{theo}




\mathstrut
\begin{theo}
\label{thmex!ttauspalleq}%Thm552
$A$と$R$を$\mathscr{T}$の関係式とし, $x$を文字とする.
このとき
\[
  \exists !x(A) \to ((\tau_{x}(A)|x)(R) \leftrightarrow \forall_{A}x(R))
\]
は$\mathscr{T}$の定理である.
\end{theo}




\mathstrut
\begin{theo}
\label{thmex!=}%Thm553
$T$を$\mathscr{T}$の対象式とし, 
$x$を$T$の中に自由変数として現れない文字とする.
このとき
\[
  \exists !x(x = T), ~~
  \exists !x(T = x)
\]
は共に$\mathscr{T}$の定理である.
\end{theo}




\mathstrut
\begin{theo}
\label{thmex!exch}%Thm554
$R$を$\mathscr{T}$の関係式とし, $x$と$y$を文字とする.
このとき
\[
  \exists !x(\exists y(R)) \to \exists y(\exists !x(R))
\]
は$\mathscr{T}$の定理である.
\end{theo}




\mathstrut
\begin{theo}
\label{thmex!spexch}%Thm555
$A$と$R$を$\mathscr{T}$の関係式とし, $x$と$y$を文字とする.
$x$が$A$の中に自由変数として現れなければ, 
\[
  \exists !x(\exists_{A}y(R)) \to \exists_{A}y(\exists !x(R))
\]
は$\mathscr{T}$の定理である.
\end{theo}




\mathstrut
\begin{theo}
\label{thmsp!lnspexvspex!}%Thm556
$A$と$R$を$\mathscr{T}$の関係式とし, $x$を文字とする.
このとき
\[
  !_{A}x(R) \leftrightarrow \neg \exists_{A}x(R) \vee \exists !_{A}x(R)
\]
は$\mathscr{T}$の定理である.
\end{theo}




\mathstrut
\begin{theo}
\label{thmspex!afree}%Thm557
$A$と$R$を$\mathscr{T}$の関係式とし, $x$を$A$の中に自由変数として現れない文字とする.
このとき
\[
  \exists !_{A}x(R) \leftrightarrow A \wedge \exists !x(R)
\]
は$\mathscr{T}$の定理である.
\end{theo}




\mathstrut
\begin{theo}
\label{thmspex!afree2}%Thm558
$A$と$R$を$\mathscr{T}$の関係式とし, $x$を$A$の中に自由変数として現れない文字とする.
このとき
\[
  \exists !_{A}x(R) \to A, ~~
  \exists !_{A}x(R) \to \exists !x(R)
\]
は共に$\mathscr{T}$の定理である.
\end{theo}




\mathstrut
\begin{theo}
\label{thmspex!rfree}%Thm559
$A$と$R$を$\mathscr{T}$の関係式とし, $x$を$R$の中に自由変数として現れない文字とする.
このとき
\[
  \exists !_{A}x(R) \leftrightarrow \exists !x(A) \wedge R
\]
は$\mathscr{T}$の定理である.
\end{theo}




\mathstrut
\begin{theo}
\label{thmspex!rfree2}%Thm560
$A$と$R$を$\mathscr{T}$の関係式とし, $x$を$R$の中に自由変数として現れない文字とする.
このとき
\[
  \exists !_{A}x(R) \to \exists !x(A), ~~
  \exists !_{A}x(R) \to R
\]
は共に$\mathscr{T}$の定理である.
\end{theo}




\mathstrut
\begin{theo}
\label{thmspalltspex!}%Thm561
$A$と$R$を$\mathscr{T}$の関係式, $T$を$\mathscr{T}$の対象式とし, 
$x$を$T$の中に自由変数として現れない文字とする.
このとき
\[
  (T|x)(A) \wedge \forall_{A}x(R \leftrightarrow x = T) \to \exists !_{A}x(R)
\]
は$\mathscr{T}$の定理である.
\end{theo}




\mathstrut
\begin{theo}
\label{thmalltspex!}%Thm562
$A$と$R$を$\mathscr{T}$の関係式, $T$を$\mathscr{T}$の対象式とし, 
$x$を$T$の中に自由変数として現れない文字とする.
このとき
\[
  (T|x)(A) \wedge \forall x(R \leftrightarrow x = T) \to \exists !_{A}x(R)
\]
は$\mathscr{T}$の定理である.
\end{theo}




\mathstrut
\begin{theo}
\label{thmspalltT=sptau}%Thm563
$A$と$R$を$\mathscr{T}$の関係式, $T$を$\mathscr{T}$の対象式とし, 
$x$を$T$の中に自由変数として現れない文字とする.
このとき
\[
  (T|x)(A) \wedge \forall_{A}x(R \leftrightarrow x = T) \to T = \tau_{x}(A \wedge R)
\]
は$\mathscr{T}$の定理である.
\end{theo}




\mathstrut
\begin{theo}
\label{thmalltT=sptau}%Thm564
$A$と$R$を$\mathscr{T}$の関係式, $T$を$\mathscr{T}$の対象式とし, 
$x$を$T$の中に自由変数として現れない文字とする.
このとき
\[
  (T|x)(A) \wedge \forall x(R \leftrightarrow x = T) \to T = \tau_{x}(A \wedge R), ~~
  (T|x)(A) \wedge \forall x(R \leftrightarrow x = T) \to \tau_{x}(R) = \tau_{x}(A \wedge R)
\]
は共に$\mathscr{T}$の定理である.
\end{theo}




\mathstrut
\begin{theo}
\label{thmspex!tspall}%Thm565
$A$と$R$を$\mathscr{T}$の関係式とし, $x$を文字とする.
このとき
\[
  \exists !_{A}x(R) \to \forall_{A}x(R \leftrightarrow x = \tau_{x}(A \wedge R))
\]
は$\mathscr{T}$の定理である.
\end{theo}




\mathstrut
\begin{theo}
\label{thmspex!lspall}%Thm566
$A$と$R$を$\mathscr{T}$の関係式とし, $x$を文字とする.
このとき
\[
  \exists !_{A}x(R) \leftrightarrow (\tau_{x}(A \wedge R)|x)(A) \wedge \forall_{A}x(R \leftrightarrow x = \tau_{x}(A \wedge R))
\]
は$\mathscr{T}$の定理である.
\end{theo}




\mathstrut
\begin{theo}
\label{thmspex!equiv}%Thm567
$A$と$R$を$\mathscr{T}$の関係式とし, $x$を文字とする.
また$y$を$x$と異なり, $A$及び$R$の中に自由変数として現れない文字とする.
このとき
\[
  \exists !_{A}x(R) \leftrightarrow \exists_{(y|x)(A)}y(\forall_{A}x(R \leftrightarrow x = y))
\]
は$\mathscr{T}$の定理である.
\end{theo}




\mathstrut
\begin{theo}
\label{thmspex!equiv2}%Thm568
$A$と$R$を$\mathscr{T}$の関係式とし, $x$を文字とする.
また$y$を$x$と異なり, $A$及び$R$の中に自由変数として現れない文字とする.
このとき
\[
  \exists !_{A}x(R) \leftrightarrow \exists_{(y|x)(A)}y((y|x)(R) \wedge \forall_{A}x(R \to x = y))
\]
は$\mathscr{T}$の定理である.
\end{theo}




\mathstrut
\begin{theo}
\label{thmspalleqspex!sep}%Thm569
$A$, $R$, $S$を$\mathscr{T}$の関係式とし, $x$を文字とする.
このとき
\[
  \forall_{A}x(R \leftrightarrow S) \to (\exists !_{A}x(R) \leftrightarrow \exists !_{A}x(S))
\]
は$\mathscr{T}$の定理である.
\end{theo}




\mathstrut
\begin{theo}
\label{thmalleqspex!sep}%Thm570
$A$, $R$, $S$を$\mathscr{T}$の関係式とし, $x$を文字とする.
このとき
\[
  \forall x(R \leftrightarrow S) \to (\exists !_{A}x(R) \leftrightarrow \exists !_{A}x(S))
\]
は$\mathscr{T}$の定理である.
\end{theo}




\mathstrut
\begin{theo}
\label{thmspallpreeqspex!sep}%Thm571
$A$, $B$, $R$を$\mathscr{T}$の関係式とし, $x$を文字とする.
このとき
\[
  \forall_{R}x(A \leftrightarrow B) \to (\exists !_{A}x(R) \leftrightarrow \exists !_{B}x(R))
\]
は$\mathscr{T}$の定理である.
\end{theo}




\mathstrut
\begin{theo}
\label{thmallpreeqspex!sep}%Thm572
$A$, $B$, $R$を$\mathscr{T}$の関係式とし, $x$を文字とする.
このとき
\[
  \forall x(A \leftrightarrow B) \to (\exists !_{A}x(R) \leftrightarrow \exists !_{B}x(R))
\]
は$\mathscr{T}$の定理である.
\end{theo}




\mathstrut
\begin{theo}
\label{thmspex!exch}%Thm573
$A$と$R$を$\mathscr{T}$の関係式とし, $x$と$y$を文字とする.
$y$が$A$の中に自由変数として現れなければ, 
\[
  \exists !_{A}x(\exists y(R)) \to \exists y(\exists !_{A}x(R))
\]
は$\mathscr{T}$の定理である.
\end{theo}




\mathstrut
\begin{theo}
\label{thmspex!spexch}%Thm574
$A$, $B$, $R$を$\mathscr{T}$の関係式とし, $x$と$y$を文字とする.
$x$が$B$の中に自由変数として現れず, $y$が$A$の中に自由変数として現れなければ, 
\[
  \exists !_{A}x(\exists_{B}y(R)) \to \exists_{B}y(\exists !_{A}x(R))
\]
は$\mathscr{T}$の定理である.
\end{theo}




\newpage




\part{集合論}




\setcounter{section}{0}
\setcounter{defi}{0}
\section{用語, 省略記法の準備}%新規




\textbf{集合論}というのは, $=$と$\in$という二つの記号のみを特殊記号として持ち, 
そのすべてのschemaが既述のS1---S6及び後述のS7で与えられ, 
そのすべての明示的公理が後述のA1---A4で与えられる理論のことをいう.
後で見るように, これらの明示的公理の中には自由変数が現れない.
言い換えれば, 集合論は定数を持たない理論である.
この節では, 集合論を記述するための準備を行う.




\mathstrut
$\mathscr{T}$を特殊記号として$\in$を持つ或る理論とし, $a$と$b$を$\mathscr{T}$の記号列とする.
記号列の本来の書き方では$\in a b$と書かれる記号列を, 
以後見やすさのため$(a) \in (b)$または$(b) \ni (a)$と書き表す.
括弧は適宜省略する.
また$\neg \in$という記号列を$\notin$と書き表し, 
記号列の本来の書き方では$\notin a b$と書かれる記号列を, 
以後$(a) \notin (b)$または$(b) \not\ni (a)$と書き表す.
これについても括弧は適宜省略する.
これらの記法によれば, $a \notin b$は$\neg (a \in b)$と同じ記号列である.

$a$と$b$が$\mathscr{T}$の対象式ならば, 構成法則 \ref{formfund}により, 
$a \in b$と$a \notin b$は共に$\mathscr{T}$の関係式である.
特に$\mathscr{T}$の対象式$b$に対し, $a \in b$が$\mathscr{T}$の定理となるような
$\mathscr{T}$の対象式$a$をすべて, ($\mathscr{T}$における) $b$の\textbf{元} (member) または
\textbf{要素} (element) という.




\mathstrut
さて集合論 (或いはそれよりも強い理論) では, $\exists_{x \in a}x(R)$, $\forall_{x \in a}x(R)$, 
$!_{x \in a}x(R)$, $\exists !_{x \in a}x(R)$という形の記号列を扱うことが多い.
そこで次の定義 \ref{defspquanin}, \ref{defsp!spex!in}でこれらの記号列の省略記法を導入する.




\mathstrut
\begin{defi}
\label{defspquanin}%定義1%確認済
$\mathscr{T}$を特殊記号として$\in$を持つ理論とし, 
$a$と$R$を$\mathscr{T}$の記号列, $x$を文字とする.
$\exists_{x \in a}x(R)$, $\forall_{x \in a}x(R)$という記号列を, 
それぞれ$(\exists x \in a)(R)$, $(\forall x \in a)(R)$とも書き表す.
\end{defi}




\mathstrut%確認済
以下の変数法則 \ref{valspquanin}, 一般代入法則 \ref{gsubstspquanin}, 
代入法則 \ref{substspquanintrans}, \ref{substspquanin}, 
構成法則 \ref{formspquanin}では, $\mathscr{T}$を特殊記号として$\in$を持つ理論とし, 
これらの法則における``記号列'', ``対象式'', ``関係式''とは, 
それぞれ$\mathscr{T}$の記号列, $\mathscr{T}$の対象式, $\mathscr{T}$の関係式のこととする.




\mathstrut
\begin{valu}
\label{valspquanin}%変数18%確認済
$a$と$R$を記号列とし, $x$を文字とする.

1)
$x$は$(\exists x \in a)(R)$及び$(\forall x \in a)(R)$の中に自由変数として現れない.

2)
$y$を文字とする.
$y$が$a$及び$R$の中に自由変数として現れなければ, 
$y$は$(\exists x \in a)(R)$及び$(\forall x \in a)(R)$の中に自由変数として現れない.
\end{valu}


\noindent{\bf 証明}
~1)
定義より$(\exists x \in a)(R)$, $(\forall x \in a)(R)$はそれぞれ
$\exists_{x \in a}x(R)$, $\forall_{x \in a}x(R)$という記号列である.
変数法則 \ref{valspquan}により, $x$はこれらの記号列の中に自由変数として現れない.

\noindent
2)
$y$が$x$のときは1)により2)が成り立つ.
$y$が$x$と異なる文字であるとき, $y$が$a$の中に自由変数として現れないことから, 
変数法則 \ref{valfund}により, $y$は$x \in a$の中に自由変数として現れない.
このことと$y$が$R$の中に自由変数として現れないことから, 
変数法則 \ref{valspquan}により, $y$は$\exists_{x \in a}x(R)$及び$\forall_{x \in a}x(R)$, 
即ち$(\exists x \in a)(R)$及び$(\forall x \in a)(R)$の中に自由変数として現れない.
\halmos




\mathstrut
\begin{gsub}
\label{gsubstspquanin}%一般代入24%確認済
$a$と$R$を記号列とし, $x$を文字とする.
また$n$を自然数とし, $T_{1}, T_{2}, \cdots, T_{n}$を記号列とする.
また$y_{1}, y_{2}, \cdots, y_{n}$を, どの二つも互いに異なる文字とする.
$x$が$y_{1}, y_{2}, \cdots, y_{n}$のいずれとも異なり, かつ
$T_{1}, T_{2}, \cdots, T_{n}$のいずれの記号列の中にも自由変数として現れなければ, 
\begin{multline*}
  (T_{1}|y_{1}, T_{2}|y_{2}, \cdots, T_{n}|y_{n})((\exists x \in a)(R)) \\
  \equiv (\exists x \in (T_{1}|y_{1}, T_{2}|y_{2}, \cdots, T_{n}|y_{n})(a))((T_{1}|y_{1}, T_{2}|y_{2}, \cdots, T_{n}|y_{n})(R)), 
\end{multline*}
\begin{multline*}
  (T_{1}|y_{1}, T_{2}|y_{2}, \cdots, T_{n}|y_{n})((\forall x \in a)(R)) \\
  \equiv (\forall x \in (T_{1}|y_{1}, T_{2}|y_{2}, \cdots, T_{n}|y_{n})(a))((T_{1}|y_{1}, T_{2}|y_{2}, \cdots, T_{n}|y_{n})(R))
\end{multline*}
が成り立つ.
\end{gsub}


\noindent{\bf 証明}
~$x$が$y_{1}, y_{2}, \cdots, y_{n}$のいずれとも異なり, かつ
$T_{1}, T_{2}, \cdots, T_{n}$のいずれの記号列の中にも自由変数として現れないことから, 
一般代入法則 \ref{gsubstspquan}により
\begin{align}
  \label{gsubstspquanin1}
  (T_{1}|y_{1}, T_{2}|y_{2}, \cdots, T_{n}|y_{n})(\exists_{x \in a}x(R)) 
  &\equiv \exists_{(T_{1}|y_{1}, T_{2}|y_{2}, \cdots, T_{n}|y_{n})(x \in a)}x((T_{1}|y_{1}, T_{2}|y_{2}, \cdots, T_{n}|y_{n})(R)), \\
  \mbox{} \notag \\
  \label{gsubstspquanin2}
  (T_{1}|y_{1}, T_{2}|y_{2}, \cdots, T_{n}|y_{n})(\forall_{x \in a}x(R)) 
  &\equiv \forall_{(T_{1}|y_{1}, T_{2}|y_{2}, \cdots, T_{n}|y_{n})(x \in a)}x((T_{1}|y_{1}, T_{2}|y_{2}, \cdots, T_{n}|y_{n})(R))
\end{align}
が共に成り立つ.
また$x$が$y_{1}, y_{2}, \cdots, y_{n}$のいずれとも異なることと
一般代入法則 \ref{gsubstfund}から, 
\begin{equation}
\label{gsubstspquanin3}
  (T_{1}|y_{1}, T_{2}|y_{2}, \cdots, T_{n}|y_{n})(x \in a) 
  \equiv x \in (T_{1}|y_{1}, T_{2}|y_{2}, \cdots, T_{n}|y_{n})(a)
\end{equation}
が成り立つ.
そこで(\ref{gsubstspquanin1})と(\ref{gsubstspquanin3}), 
(\ref{gsubstspquanin2})と(\ref{gsubstspquanin3})から, 
\begin{align*}
  (T_{1}|y_{1}, T_{2}|y_{2}, \cdots, T_{n}|y_{n})(\exists_{x \in a}x(R)) 
  &\equiv \exists_{x \in (T_{1}|y_{1}, T_{2}|y_{2}, \cdots, T_{n}|y_{n})(a)}x((T_{1}|y_{1}, T_{2}|y_{2}, \cdots, T_{n}|y_{n})(R)), \\
  \mbox{} \\
  (T_{1}|y_{1}, T_{2}|y_{2}, \cdots, T_{n}|y_{n})(\forall_{x \in a}x(R)) 
  &\equiv \forall_{x \in (T_{1}|y_{1}, T_{2}|y_{2}, \cdots, T_{n}|y_{n})(a)}x((T_{1}|y_{1}, T_{2}|y_{2}, \cdots, T_{n}|y_{n})(R)), 
\end{align*}
即ち
\begin{multline*}
  (T_{1}|y_{1}, T_{2}|y_{2}, \cdots, T_{n}|y_{n})((\exists x \in a)(R)) \\
  \equiv (\exists x \in (T_{1}|y_{1}, T_{2}|y_{2}, \cdots, T_{n}|y_{n})(a))((T_{1}|y_{1}, T_{2}|y_{2}, \cdots, T_{n}|y_{n})(R)), 
\end{multline*}
\begin{multline*}
  (T_{1}|y_{1}, T_{2}|y_{2}, \cdots, T_{n}|y_{n})((\forall x \in a)(R)) \\
  \equiv (\forall x \in (T_{1}|y_{1}, T_{2}|y_{2}, \cdots, T_{n}|y_{n})(a))((T_{1}|y_{1}, T_{2}|y_{2}, \cdots, T_{n}|y_{n})(R))
\end{multline*}
が共に成り立つ.
\halmos




\mathstrut
\begin{subs}
\label{substspquanintrans}%代入25%確認済
$a$と$R$を記号列とし, $x$と$y$を文字とする.
$y$が$a$及び$R$の中に自由変数として現れなければ, 
\[
  (\exists x \in a)(R) \equiv (\exists y \in (y|x)(a))((y|x)(R)), ~~
  (\forall x \in a)(R) \equiv (\forall y \in (y|x)(a))((y|x)(R))
\]
が成り立つ.
更に, $x$が$a$の中に自由変数として現れなければ, 
\[
  (\exists x \in a)(R) \equiv (\exists y \in a)((y|x)(R)), ~~
  (\forall x \in a)(R) \equiv (\forall y \in a)((y|x)(R))
\]
が成り立つ.
\end{subs}


\noindent{\bf 証明}
~$y$が$x$ならば, 本法則が成り立つことは代入法則 \ref{substsame}によって明らかである.
そこで以下$y$は$x$と異なる文字であるとする.
このとき$y$が$a$の中に自由変数として現れないことから, 
変数法則 \ref{valfund}により, $y$は$x \in a$の中に自由変数として現れない.
このことと$y$が$R$の中に自由変数として現れないことから, 
代入法則 \ref{substspquantrans}により
\begin{align}
  \label{substspquanintrans1}
  &\exists_{x \in a}x(R) \equiv \exists_{(y|x)(x \in a)}y((y|x)(R)), \\
  \mbox{} \notag \\
  \label{substspquanintrans2}
  &\forall_{x \in a}x(R) \equiv \forall_{(y|x)(x \in a)}y((y|x)(R))
\end{align}
が共に成り立つ.
また代入法則 \ref{substfund}により
\begin{equation}
\label{substspquanintrans3}
  (y|x)(x \in a) \equiv y \in (y|x)(a)
\end{equation}
が成り立つ.
そこで(\ref{substspquanintrans1})と(\ref{substspquanintrans3}), 
(\ref{substspquanintrans2})と(\ref{substspquanintrans3})から, 
\[
  \exists_{x \in a}x(R) \equiv \exists_{y \in (y|x)(a)}y((y|x)(R)), ~~
  \forall_{x \in a}x(R) \equiv \forall_{y \in (y|x)(a)}y((y|x)(R)), 
\]
即ち
\[
  (\exists x \in a)(R) \equiv (\exists y \in (y|x)(a))((y|x)(R)), ~~
  (\forall x \in a)(R) \equiv (\forall y \in (y|x)(a))((y|x)(R))
\]
が共に成り立つ.
特にここで$x$が$a$の中に自由変数として現れなければ, 代入法則 \ref{substfree}により
$(y|x)(a)$は$a$と一致するから, 
\[
  (\exists x \in a)(R) \equiv (\exists y \in a)((y|x)(R)), ~~
  (\forall x \in a)(R) \equiv (\forall y \in a)((y|x)(R))
\]
が共に成り立つ.
\halmos




\mathstrut
\begin{subs}
\label{substspquanin}%代入26%確認済
$a$, $R$, $T$を記号列とし, $x$と$y$を異なる文字とする.
$x$が$T$の中に自由変数として現れなければ, 
\[
  (T|y)((\exists x \in a)(R)) \equiv (\exists x \in (T|y)(a))((T|y)(R)), ~~
  (T|y)((\forall x \in a)(R)) \equiv (\forall x \in (T|y)(a))((T|y)(R))
\]
が成り立つ.
\end{subs}


\noindent{\bf 証明}
~一般代入法則 \ref{gsubstspquanin}において, $n$が$1$の場合である.
\halmos




\mathstrut
\begin{form}
\label{formspquanin}%構成35%確認済
$a$が対象式, $R$が関係式, $x$が文字ならば, 
$(\exists x \in a)(R)$と$(\forall x \in a)(R)$は共に関係式である.
\end{form}


\noindent{\bf 証明}
~構成法則 \ref{formfund}, \ref{formspquan}によって明らかである.
\halmos




\mathstrut
\begin{defi}
\label{defsp!spex!in}%定義2%確認済
$\mathscr{T}$を特殊記号として$=$及び$\in$を持つ理論とし, 
$a$と$R$を$\mathscr{T}$の記号列, $x$を文字とする.
$!_{x \in a}x(R)$, $\exists !_{x \in a}x(R)$という記号列を, 
それぞれ$(!x \in a)(R)$, $(\exists !x \in a)(R)$とも書き表す.
\end{defi}




\mathstrut%確認済
以下の変数法則 \ref{valsp!spex!in}, 一般代入法則 \ref{gsubstsp!spex!in}, 
代入法則 \ref{substsp!spex!intrans}, \ref{substsp!spex!in}, 
構成法則 \ref{formsp!spex!in}では, $\mathscr{T}$を特殊記号として$=$及び$\in$を持つ理論とし, 
これらの法則における``記号列'', ``対象式'', ``関係式''とは, 
それぞれ$\mathscr{T}$の記号列, $\mathscr{T}$の対象式, $\mathscr{T}$の関係式のこととする.




\mathstrut
\begin{valu}
\label{valsp!spex!in}%変数19%確認済
$a$と$R$を記号列とし, $x$を文字とする.

1)
$x$は$(!x \in a)(R)$及び$(\exists !x \in a)(R)$の中に自由変数として現れない.

2)
$y$を文字とする.
$y$が$a$及び$R$の中に自由変数として現れなければ, 
$y$は$(!x \in a)(R)$及び$(\exists !x \in a)(R)$の中に自由変数として現れない.
\end{valu}


\noindent{\bf 証明}
~1)
定義より$(!x \in a)(R)$, $(\exists !x \in a)(R)$はそれぞれ
$!_{x \in a}x(R)$, $\exists !_{x \in a}x(R)$という記号列である.
変数法則 \ref{valsp!}, \ref{valspex!}により, $x$はこれらの記号列の中に自由変数として現れない.

\noindent
2)
$y$が$x$のときは1)により2)が成り立つ.
$y$が$x$と異なる文字であるとき, $y$が$a$の中に自由変数として現れないことから, 
変数法則 \ref{valfund}により, $y$は$x \in a$の中に自由変数として現れない.
このことと$y$が$R$の中に自由変数として現れないことから, 
変数法則 \ref{valsp!}, \ref{valspex!}により, $y$は$!_{x \in a}x(R)$及び$\exists !_{x \in a}x(R)$, 
即ち$(!x \in a)(R)$及び$(\exists !x \in a)(R)$の中に自由変数として現れない.
\halmos




\mathstrut
\begin{gsub}
\label{gsubstsp!spex!in}%一般代入25%確認済
$a$と$R$を記号列とし, $x$を文字とする.
また$n$を自然数とし, $T_{1}, T_{2}, \cdots, T_{n}$を記号列とする.
また$y_{1}, y_{2}, \cdots, y_{n}$を, どの二つも互いに異なる文字とする.
$x$が$y_{1}, y_{2}, \cdots, y_{n}$のいずれとも異なり, かつ
$T_{1}, T_{2}, \cdots, T_{n}$のいずれの記号列の中にも自由変数として現れなければ, 
\begin{multline*}
  (T_{1}|y_{1}, T_{2}|y_{2}, \cdots, T_{n}|y_{n})((!x \in a)(R)) \\
  \equiv (!x \in (T_{1}|y_{1}, T_{2}|y_{2}, \cdots, T_{n}|y_{n})(a))((T_{1}|y_{1}, T_{2}|y_{2}, \cdots, T_{n}|y_{n})(R)), 
\end{multline*}
\begin{multline*}
  (T_{1}|y_{1}, T_{2}|y_{2}, \cdots, T_{n}|y_{n})((\exists !x \in a)(R)) \\
  \equiv (\exists !x \in (T_{1}|y_{1}, T_{2}|y_{2}, \cdots, T_{n}|y_{n})(a))((T_{1}|y_{1}, T_{2}|y_{2}, \cdots, T_{n}|y_{n})(R))
\end{multline*}
が成り立つ.
\end{gsub}


\noindent{\bf 証明}
~$x$が$y_{1}, y_{2}, \cdots, y_{n}$のいずれとも異なり, かつ
$T_{1}, T_{2}, \cdots, T_{n}$のいずれの記号列の中にも自由変数として現れないことから, 
一般代入法則 \ref{gsubstsp!}, \ref{gsubstspex!}により
\begin{align}
  \label{gsubstsp!spex!in1}
  (T_{1}|y_{1}, T_{2}|y_{2}, \cdots, T_{n}|y_{n})(!_{x \in a}x(R)) 
  &\equiv \ !_{(T_{1}|y_{1}, T_{2}|y_{2}, \cdots, T_{n}|y_{n})(x \in a)}x((T_{1}|y_{1}, T_{2}|y_{2}, \cdots, T_{n}|y_{n})(R)), \\
  \mbox{} \notag \\
  \label{gsubstsp!spex!in2}
  (T_{1}|y_{1}, T_{2}|y_{2}, \cdots, T_{n}|y_{n})(\exists !_{x \in a}x(R)) 
  &\equiv \exists !_{(T_{1}|y_{1}, T_{2}|y_{2}, \cdots, T_{n}|y_{n})(x \in a)}x((T_{1}|y_{1}, T_{2}|y_{2}, \cdots, T_{n}|y_{n})(R))
\end{align}
が共に成り立つ.
また$x$が$y_{1}, y_{2}, \cdots, y_{n}$のいずれとも異なることと
一般代入法則 \ref{gsubstfund}から, 
\begin{equation}
\label{gsubstsp!spex!in3}
  (T_{1}|y_{1}, T_{2}|y_{2}, \cdots, T_{n}|y_{n})(x \in a) 
  \equiv x \in (T_{1}|y_{1}, T_{2}|y_{2}, \cdots, T_{n}|y_{n})(a)
\end{equation}
が成り立つ.
そこで(\ref{gsubstsp!spex!in1})と(\ref{gsubstsp!spex!in3}), 
(\ref{gsubstsp!spex!in2})と(\ref{gsubstsp!spex!in3})から, 
\begin{align*}
  (T_{1}|y_{1}, T_{2}|y_{2}, \cdots, T_{n}|y_{n})(!_{x \in a}x(R)) 
  &\equiv \ !_{x \in (T_{1}|y_{1}, T_{2}|y_{2}, \cdots, T_{n}|y_{n})(a)}x((T_{1}|y_{1}, T_{2}|y_{2}, \cdots, T_{n}|y_{n})(R)), \\
  \mbox{} \\
  (T_{1}|y_{1}, T_{2}|y_{2}, \cdots, T_{n}|y_{n})(\exists !_{x \in a}x(R)) 
  &\equiv \exists !_{x \in (T_{1}|y_{1}, T_{2}|y_{2}, \cdots, T_{n}|y_{n})(a)}x((T_{1}|y_{1}, T_{2}|y_{2}, \cdots, T_{n}|y_{n})(R)), 
\end{align*}
即ち
\begin{multline*}
  (T_{1}|y_{1}, T_{2}|y_{2}, \cdots, T_{n}|y_{n})((!x \in a)(R)) \\
  \equiv (!x \in (T_{1}|y_{1}, T_{2}|y_{2}, \cdots, T_{n}|y_{n})(a))((T_{1}|y_{1}, T_{2}|y_{2}, \cdots, T_{n}|y_{n})(R)), 
\end{multline*}
\begin{multline*}
  (T_{1}|y_{1}, T_{2}|y_{2}, \cdots, T_{n}|y_{n})((\exists !x \in a)(R)) \\
  \equiv (\exists !x \in (T_{1}|y_{1}, T_{2}|y_{2}, \cdots, T_{n}|y_{n})(a))((T_{1}|y_{1}, T_{2}|y_{2}, \cdots, T_{n}|y_{n})(R))
\end{multline*}
が共に成り立つ.
\halmos




\mathstrut
\begin{subs}
\label{substsp!spex!intrans}%代入27%確認済
$a$と$R$を記号列とし, $x$と$y$を文字とする.
$y$が$a$及び$R$の中に自由変数として現れなければ, 
\[
  (!x \in a)(R) \equiv (!y \in (y|x)(a))((y|x)(R)), ~~
  (\exists !x \in a)(R) \equiv (\exists !y \in (y|x)(a))((y|x)(R))
\]
が成り立つ.
更に, $x$が$a$の中に自由変数として現れなければ, 
\[
  (!x \in a)(R) \equiv (!y \in a)((y|x)(R)), ~~
  (\exists !x \in a)(R) \equiv (\exists !y \in a)((y|x)(R))
\]
が成り立つ.
\end{subs}


\noindent{\bf 証明}
~$y$が$x$ならば, 本法則が成り立つことは代入法則 \ref{substsame}によって明らかである.
そこで以下$y$は$x$と異なる文字であるとする.
このとき$y$が$a$の中に自由変数として現れないことから, 
変数法則 \ref{valfund}により, $y$は$x \in a$の中に自由変数として現れない.
このことと$y$が$R$の中に自由変数として現れないことから, 
代入法則 \ref{substsp!trans}, \ref{substspex!trans}により
\begin{align}
  \label{substsp!spex!intrans1}
  &!_{x \in a}x(R) \equiv \ !_{(y|x)(x \in a)}y((y|x)(R)), \\
  \mbox{} \notag \\
  \label{substsp!spex!intrans2}
  &\exists !_{x \in a}x(R) \equiv \exists !_{(y|x)(x \in a)}y((y|x)(R))
\end{align}
が共に成り立つ.
また代入法則 \ref{substfund}により
\begin{equation}
\label{substsp!spex!intrans3}
  (y|x)(x \in a) \equiv y \in (y|x)(a)
\end{equation}
が成り立つ.
そこで(\ref{substsp!spex!intrans1})と(\ref{substsp!spex!intrans3}), 
(\ref{substsp!spex!intrans2})と(\ref{substsp!spex!intrans3})から, 
\[
  !_{x \in a}x(R) \equiv \ !_{y \in (y|x)(a)}y((y|x)(R)), ~~
  \exists !_{x \in a}x(R) \equiv \exists !_{y \in (y|x)(a)}y((y|x)(R)), 
\]
即ち
\[
  (!x \in a)(R) \equiv (!y \in (y|x)(a))((y|x)(R)), ~~
  (\exists !x \in a)(R) \equiv (\exists !y \in (y|x)(a))((y|x)(R))
\]
が共に成り立つ.
特にここで$x$が$a$の中に自由変数として現れなければ, 代入法則 \ref{substfree}により
$(y|x)(a)$は$a$と一致するから, 
\[
  (!x \in a)(R) \equiv (!y \in a)((y|x)(R)), ~~
  (\exists !x \in a)(R) \equiv (\exists !y \in a)((y|x)(R))
\]
が共に成り立つ.
\halmos




\mathstrut
\begin{subs}
\label{substsp!spex!in}%代入28%確認済
$a$, $R$, $T$を記号列とし, $x$と$y$を異なる文字とする.
$x$が$T$の中に自由変数として現れなければ, 
\[
  (T|y)((!x \in a)(R)) \equiv (!x \in (T|y)(a))((T|y)(R)), ~~
  (T|y)((\exists !x \in a)(R)) \equiv (\exists !x \in (T|y)(a))((T|y)(R))
\]
が成り立つ.
\end{subs}


\noindent{\bf 証明}
~一般代入法則 \ref{gsubstsp!spex!in}において, $n$が$1$の場合である.
\halmos




\mathstrut
\begin{form}
\label{formsp!spex!in}%構成36%確認済
$a$が対象式, $R$が関係式, $x$が文字ならば, 
$(!x \in a)(R)$と$(\exists !x \in a)(R)$は共に関係式である.
\end{form}


\noindent{\bf 証明}
~構成法則 \ref{formfund}, \ref{formsp!}, \ref{formspex!}によって明らかである.
\halmos




\mathstrut
$\mathscr{T}$を理論とし, $R$を$\mathscr{T}$の関係式とする.
以後``$R$が$\mathscr{T}$の定理である''という代わりに, 
``$\mathscr{T}$において$R$が成り立つ'', ``$\mathscr{T}$において$R$ (である)''などともいう.
また``$\neg R$が$\mathscr{T}$の定理である''という代わりに, 
``$\mathscr{T}$において$R$が成り立たない'', ``$\mathscr{T}$において$R$ではない''
などということもある (これらの表現を``$R$は$\mathscr{T}$の定理ではない''という意味で用いることはない).

また今後は``$\mathscr{T}$の'', ``$\mathscr{T}$において'', 
といった表現は (変数法則等の各法則や定義などを除いて) 基本的に用いない.
一連の議論の中で理論が明示されていないとき, 
そこで問題にしているのは, 集合論よりも強い或る一つの理論である (即ち$=$と$\in$を特殊記号に持ち, 
S1---S6及びこれから導入するS7をschemaに持ち, これから導入するA1---A4が定理となるような理論である).

また今後対象式の同義語として\textbf{集合}という言葉も用いる.




\newpage
\setcounter{defi}{0}
\section{包含関係, 外延公理}




この節では, 集合の間の最も基本的な関係である包含関係を定義する.
また集合論の明示的公理の一つである外延公理を導入する.

まず$=$と$\in$という二つの特殊記号に関して成り立つ最も基本的な定理を与えるところから始める.




\mathstrut
\begin{thm}
\label{sthm=tineq}%定理2.1%確認済
$a$, $b$, $c$を集合とするとき, 
\begin{equation}
\label{sthm=tineq1}
  a = b \to (a \in c \leftrightarrow b \in c), ~~
  a = b \to (c \in a \leftrightarrow c \in b)
\end{equation}
が成り立つ.
またこのことから, 次の1), 2)が成り立つ.

1)
$a = b$ならば, $a \in c \leftrightarrow b \in c$, $c \in a \leftrightarrow c \in b$.

2)
$a = b$とする.
このとき$a \in c$ならば, $b \in c$である.
また$b \in c$ならば, $a \in c$である.
また$c \in a$ならば, $c \in b$である.
また$c \in b$ならば, $c \in a$である.
\end{thm}


\noindent{\bf 証明}
~$x$を$c$の中に自由変数として現れない文字とするとき, Thm \ref{thms5eq}より
\[
  a = b \to ((a|x)(x \in c) \leftrightarrow (b|x)(x \in c)), ~~
  a = b \to ((a|x)(c \in x) \leftrightarrow (b|x)(c \in x))
\]
が共に成り立つが, 代入法則 \ref{substfree}, \ref{substfund}によれば
これらの記号列はそれぞれ
\[
  a = b \to (a \in c \leftrightarrow b \in c), ~~
  a = b \to (c \in a \leftrightarrow c \in b)
\]
と一致するから, これらが共に成り立つ.

\noindent
1)
(\ref{sthm=tineq1})と推論法則 \ref{dedmp}によって明らか.

\noindent
2)
1)と推論法則 \ref{dedeqfund}によって明らか.
\halmos




\mathstrut%確認済
次の定理 \ref{sthm=&in}は内容的には定理 \ref{sthm=tineq}と重複しているが, 
この形で後に参照することが多いので定理として述べておく.




\mathstrut
\begin{thm}
\label{sthm=&in}%定理2.2%確認済
$a$, $b$, $c$を集合とするとき, 
\begin{align*}
  &a = b \wedge a \in c \to b \in c, ~~
  a = b \wedge b \in c \to a \in c, \\
  \mbox{} \\
  &a = b \wedge c \in a \to c \in b, ~~
  a = b \wedge c \in b \to c \in a
\end{align*}
がすべて成り立つ.
\end{thm}


\noindent{\bf 証明}
~$x$を$c$の中に自由変数として現れない文字とするとき, 
Thm \ref{thmfroms5t}より
\begin{align*}
  &a = b \wedge (a|x)(x \in c) \to (b|x)(x \in c), ~~
  a = b \wedge (b|x)(x \in c) \to (a|x)(x \in c), \\
  \mbox{} \\
  &a = b \wedge (a|x)(c \in x) \to (b|x)(c \in x), ~~
  a = b \wedge (b|x)(c \in x) \to (a|x)(c \in x)
\end{align*}
がすべて成り立つが, 代入法則 \ref{substfree}, \ref{substfund}によれば
これらの記号列はそれぞれ
\begin{align*}
  &a = b \wedge a \in c \to b \in c, ~~
  a = b \wedge b \in c \to a \in c, \\
  \mbox{} \\
  &a = b \wedge c \in a \to c \in b, ~~
  a = b \wedge c \in b \to c \in a
\end{align*}
と一致するから, これらがすべて成り立つ.
\halmos




\mathstrut
\begin{thm}
\label{sthmn=&in}%定理2.3%新規%確認済
$a$, $b$, $c$を集合とするとき, 
\begin{align*}
  &a \in c \wedge b \notin c \to a \neq b, ~~
  a \notin c \wedge b \in c \to a \neq b, \\
  \mbox{} \\
  &c \in a \wedge c \notin b \to a \neq b, ~~
  c \notin a \wedge c \in b \to a \neq b
\end{align*}
がすべて成り立つ.
またこのことから, 次の(\ref{sthmn=&in1})が成り立つ.
\begin{align}
\label{sthmn=&in1}
  &a \in c \text{と} b \notin c \text{が共に成り立てば,} ~a \neq b. ~
  \text{また} a \notin c \text{と} b \in c \text{が共に成り立てば,} ~a \neq b. \\
  &\text{また} c \in a \text{と} c \notin b \text{が共に成り立てば,} ~a \neq b. ~
  \text{また} c \notin a \text{と} c \in b \text{が共に成り立てば,} ~a \neq b. \notag
\end{align}
\end{thm}


\noindent{\bf 証明}
~$x$を$c$の中に自由変数として現れない文字とするとき, 
Thm \ref{thms5n=}より
\begin{align*}
  &(a|x)(x \in c) \wedge \neg (b|x)(x \in c) \to a \neq b, ~~
  \neg (a|x)(x \in c) \wedge (b|x)(x \in c) \to a \neq b, \\
  \mbox{} \\
  &(a|x)(c \in x) \wedge \neg (b|x)(c \in x) \to a \neq b, ~~
  \neg (a|x)(c \in x) \wedge (b|x)(c \in x) \to a \neq b
\end{align*}
がすべて成り立つが, 代入法則 \ref{substfree}, \ref{substfund}によれば
これらの記号列はそれぞれ
\begin{align*}
  &a \in c \wedge b \notin c \to a \neq b, ~~
  a \notin c \wedge b \in c \to a \neq b, \\
  \mbox{} \\
  &c \in a \wedge c \notin b \to a \neq b, ~~
  c \notin a \wedge c \in b \to a \neq b
\end{align*}
と一致するから, これらがすべて成り立つ.
(\ref{sthmn=&in1})が成り立つことは, これらと推論法則 \ref{dedmp}, \ref{dedwedge}によって明らかである.
\halmos




\mathstrut
\begin{thm}
\label{sthmspin=}%定理2.4%新規%確認済
$a$と$b$を集合, $R$を関係式とし, $x$を$a$と$b$の中に自由変数として現れない文字とする.
このとき
\begin{align*}
  &a = b \to ((\exists x \in a)(R) \leftrightarrow (\exists x \in b)(R)), ~~
  a = b \to ((\forall x \in a)(R) \leftrightarrow (\forall x \in b)(R)), \\
  \mbox{} \\
  &a = b \to ((!x \in a)(R) \leftrightarrow (!x \in b)(R)), ~~
  a = b \to ((\exists !x \in a)(R) \leftrightarrow (\exists !x \in b)(R))
\end{align*}
がすべて成り立つ.
またこれらから, 次の(\ref{sthmspin=1})が成り立つ.
\begin{align}
\label{sthmspin=1}
  &a = b \text{ならば,} ~(\exists x \in a)(R) \leftrightarrow (\exists x \in b)(R), ~
  (\forall x \in a)(R) \leftrightarrow (\forall x \in b)(R), \\
  &(!x \in a)(R) \leftrightarrow (!x \in b)(R), ~
  (\exists !x \in a)(R) \leftrightarrow (\exists !x \in b)(R) \text{がすべて成り立つ.} \notag
\end{align}
\end{thm}


\noindent{\bf 証明}
~$y$を$x$と異なり, $R$の中に自由変数として現れない文字とするとき, Thm \ref{thms5eq}より
\begin{align*}
  &a = b \to ((a|y)((\exists x \in y)(R)) \leftrightarrow (b|y)((\exists x \in y)(R))), \\
  \mbox{} \\
  &a = b \to ((a|y)((\forall x \in y)(R)) \leftrightarrow (b|y)((\forall x \in y)(R))), \\
  \mbox{} \\
  &a = b \to ((a|y)((!x \in y)(R)) \leftrightarrow (b|y)((!x \in y)(R))), \\
  \mbox{} \\
  &a = b \to ((a|y)((\exists !x \in y)(R)) \leftrightarrow (b|y)((\exists !x \in y)(R)))
\end{align*}
がすべて成り立つ.
ここで$x$と$y$が互いに異なり, $x$が$a$, $b$の中に自由変数として現れず, 
$y$が$R$の中に自由変数として現れないことから, 
代入法則 \ref{substfree}, \ref{substspquanin}, \ref{substsp!spex!in}によってわかるように, 
これらの記号列はそれぞれ
\begin{align*}
  &a = b \to ((\exists x \in a)(R) \leftrightarrow (\exists x \in b)(R)), ~~
  a = b \to ((\forall x \in a)(R) \leftrightarrow (\forall x \in b)(R)), \\
  \mbox{} \\
  &a = b \to ((!x \in a)(R) \leftrightarrow (!x \in b)(R)), ~~
  a = b \to ((\exists !x \in a)(R) \leftrightarrow (\exists !x \in b)(R))
\end{align*}
と一致する.
故にこれらがすべて成り立つ.
(\ref{sthmspin=1})が成り立つことはこれらと推論法則 \ref{dedmp}から直ちにわかる.
\halmos




\mathstrut%変更する%koko
次に$a \subset b$という記号列を定義する.
次の変形法則から始めよう.




\mathstrut
\begin{defo}
\label{subset}%変形10%確認済
$\mathscr{T}$を特殊記号として$\in$を持つ理論とし, $a$と$b$を$\mathscr{T}$の記号列とする.
また$x$と$y$を共に$a$及び$b$の中に自由変数として現れない文字とする.
このとき
\[
  \forall x(x \in a \to x \in b) \equiv \forall y(y \in a \to y \in b)
\]
が成り立つ.
\end{defo}


\noindent{\bf 証明}
~$y$が$x$と同じ文字ならば明らか.
$y$が$x$と異なる文字ならば, このことと$y$が$a$, $b$の中に自由変数として現れないことから, 
変数法則 \ref{valfund}により$y$は$x \in a \to x \in b$の中に自由変数として現れない.
故に代入法則 \ref{substquantrans}により
\[
  \forall x(x \in a \to x \in b) \equiv \forall y((y|x)(x \in a \to x \in b))
\]
が成り立つ.
また$x$が$a$, $b$の中に自由変数として現れないことから, 
代入法則 \ref{substfree}, \ref{substfund}により
\[
  (y|x)(x \in a \to x \in b) \equiv y \in a \to y \in b
\]
が成り立つ.
故に本法則が成り立つ.
\halmos




\mathstrut
\begin{defi}
\label{defsubset}%定義1%確認済
$\mathscr{T}$を特殊記号として$\in$を持つ理論とし, $a$と$b$を$\mathscr{T}$の記号列とする.
また$x$と$y$を共に$a$及び$b$の中に自由変数として現れない文字とする.
このとき変形法則 \ref{subset}により, 
$\forall x(x \in a \to x \in b)$と$\forall y(y \in a \to y \in b)$は同じ記号列となる.
$a$と$b$に対して定まるこの記号列を, $(a) \subset (b)$または$(b) \supset (a)$と書き表す.
また$\neg ((a) \subset (b))$という記号列を, 
$(a) \not\subset (b)$または$(b) \not\supset (a)$と書き表す.
これらの括弧は適宜省略する.
\end{defi}




\mathstrut%確認済
以下の変数法則 \ref{valsubset}, 一般代入法則 \ref{gsubstsubset}, 代入法則 \ref{substsubset}, 
構成法則 \ref{formsubset}では, $\mathscr{T}$を特殊記号として$\in$を持つ理論とし, 
これらの法則における``記号列'', ``集合'', ``関係式''とは, 
それぞれ$\mathscr{T}$の記号列, $\mathscr{T}$の対象式, $\mathscr{T}$の関係式のこととする.




\mathstrut
\begin{valu}
\label{valsubset}%変数20%確認済
$a$と$b$を記号列とし, $x$を文字とする.
$x$が$a$及び$b$の中に自由変数として現れなければ, 
$x$は$a \subset b$の中に自由変数として現れない.
\end{valu}


\noindent{\bf 証明}
~このとき定義から$a \subset b$は$\forall x(x \in a \to x \in b)$と同じである.
変数法則 \ref{valquan}によれば, $x$はこの中に自由変数として現れない.
\halmos




\mathstrut
\begin{gsub}
\label{gsubstsubset}%一般代入26%確認済
$a$と$b$を記号列とする.
また$n$を自然数とし, $T_{1}, T_{2}, \cdots, T_{n}$を記号列とする.
また$x_{1}, x_{2}, \cdots, x_{n}$を, どの二つも互いに異なる文字とする.
このとき
\[
  (T_{1}|x_{1}, T_{2}|x_{2}, \cdots, T_{n}|x_{n})(a \subset b) 
  \equiv (T_{1}|x_{1}, T_{2}|x_{2}, \cdots, T_{n}|x_{n})(a) \subset (T_{1}|x_{1}, T_{2}|x_{2}, \cdots, T_{n}|x_{n})(b)
\]
が成り立つ.
\end{gsub}


\noindent{\bf 証明}
~$y$を$x_{1}, x_{2}, \cdots, x_{n}$のいずれの文字とも異なり, 
$a$, $b$, $T_{1}, T_{2}, \cdots, T_{n}$のいずれの記号列の中にも自由変数として現れない文字とする.
このとき定義から$a \subset b$は$\forall y(y \in a \to y \in b)$と同じだから, 
\begin{equation}
\label{gsubstsubset1}
  (T_{1}|x_{1}, T_{2}|x_{2}, \cdots, T_{n}|x_{n})(a \subset b) 
  \equiv (T_{1}|x_{1}, T_{2}|x_{2}, \cdots, T_{n}|x_{n})(\forall y(y \in a \to y \in b))
\end{equation}
である.
また$y$が$x_{1}, x_{2}, \cdots, x_{n}$のいずれとも異なり, 
$T_{1}, T_{2}, \cdots, T_{n}$のいずれの中にも自由変数として現れないことから, 
一般代入法則 \ref{gsubstquan}により
\begin{equation}
\label{gsubstsubset2}
  (T_{1}|x_{1}, T_{2}|x_{2}, \cdots, T_{n}|x_{n})(\forall y(y \in a \to y \in b)) 
  \equiv \forall y((T_{1}|x_{1}, T_{2}|x_{2}, \cdots, T_{n}|x_{n})(y \in a \to y \in b))
\end{equation}
が成り立つ.
また$y$が$x_{1}, x_{2}, \cdots, x_{n}$のいずれとも異なることと
一般代入法則 \ref{gsubstfund}により, 
\begin{multline}
\label{gsubstsubset3}
  (T_{1}|x_{1}, T_{2}|x_{2}, \cdots, T_{n}|x_{n})(y \in a \to y \in b) \\
  \equiv y \in (T_{1}|x_{1}, T_{2}|x_{2}, \cdots, T_{n}|x_{n})(a) \to y \in (T_{1}|x_{1}, T_{2}|x_{2}, \cdots, T_{n}|x_{n})(b)
\end{multline}
が成り立つ.
そこで(\ref{gsubstsubset1})---(\ref{gsubstsubset3})からわかるように, 
$(T_{1}|x_{1}, T_{2}|x_{2}, \cdots, T_{n}|x_{n})(a \subset b)$は
\begin{equation}
\label{gsubstsubset4}
  \forall y(y \in (T_{1}|x_{1}, T_{2}|x_{2}, \cdots, T_{n}|x_{n})(a) \to y \in (T_{1}|x_{1}, T_{2}|x_{2}, \cdots, T_{n}|x_{n})(b))
\end{equation}
と一致する.
ここで$y$が$a$, $b$, $T_{1}, T_{2}, \cdots, T_{n}$のいずれの中にも
自由変数として現れないことから, 変数法則 \ref{valgsubst}により, 
$y$は$(T_{1}|x_{1}, T_{2}|x_{2}, \cdots, T_{n}|x_{n})(a)$及び
$(T_{1}|x_{1}, T_{2}|x_{2}, \cdots, T_{n}|x_{n})(b)$の中に自由変数として現れない.
故に定義から, (\ref{gsubstsubset4})は
$(T_{1}|x_{1}, T_{2}|x_{2}, \cdots, T_{n}|x_{n})(a) \subset (T_{1}|x_{1}, T_{2}|x_{2}, \cdots, T_{n}|x_{n})(b)$と同じである.
故に本法則が成り立つ.
\halmos




\mathstrut
\begin{subs}
\label{substsubset}%代入29%確認済
$a$, $b$, $T$を記号列とし, $x$を文字とする.
このとき
\[
  (T|x)(a \subset b) \equiv (T|x)(a) \subset (T|x)(b)
\]
が成り立つ.
\end{subs}


\noindent{\bf 証明}
~一般代入法則 \ref{gsubstsubset}において, $n$が$1$の場合である.
\halmos




\mathstrut
\begin{form}
\label{formsubset}%構成37%確認済
$a$と$b$が集合ならば, $a \subset b$は関係式である.
\end{form}


\noindent{\bf 証明}
~$x$を$a$, $b$の中に自由変数として現れない文字とするとき, 
$a \subset b$は$\forall x(x \in a \to x \in b)$である.
$a$と$b$が集合のとき, これが関係式となることは, 
構成法則 \ref{formfund}, \ref{formquan}によって直ちにわかる.
\halmos




\mathstrut%確認済
$\mathscr{T}$を特殊記号として$\in$を持つ或る理論とする.
$\mathscr{T}$の対象式$a$, $b$によって
$a \subset b$と書かれる関係式 (構成法則 \ref{formsubset}) をすべて, 
($\mathscr{T}$における) \textbf{包含関係}という.
また$\mathscr{T}$の対象式$b$に対し, $a \subset b$が$\mathscr{T}$の定理となるような
$\mathscr{T}$の対象式$a$をすべて, ($\mathscr{T}$における) $b$の\textbf{部分集合} (subset) という.




\mathstrut
\begin{thm}
\label{sthmsubsetconst}%定理2.5%確認済
$a$と$b$を集合とする.
また$x$を$a$及び$b$の中に自由変数として現れない, 定数でない文字とする.
このとき$x \in a \to x \in b$ならば, $a \subset b$である.
\end{thm}


\noindent{\bf 証明}
~このとき推論法則 \ref{dedltthmquan}により$\forall x(x \in a \to x \in b)$が成り立つが, 
$x$が$a$及び$b$の中に自由変数として現れないことから, この記号列は$a \subset b$と同じである.
故にこれが成り立つ.
\halmos




\mathstrut
\begin{thm}
\label{sthmsubsetbasis}%定理2.6%確認済
$a$, $b$, $c$を集合とするとき, 
\[
  a \subset b \to (c \in a \to c \in b)
\]
が成り立つ.
またこのことから, 次の1), 2)が成り立つ.

1)
$a \subset b$ならば, $c \in a \to c \in b$.

2)
$a \subset b$と$c \in a$が共に成り立つならば, $c \in b$.
\end{thm}


\noindent{\bf 証明}
~$x$を$a$及び$b$の中に自由変数として現れない文字とするとき, 
定義から$a \subset b$は$\forall x(x \in a \to x \in b)$だから, 
Thm \ref{thmallfund2}より
\[
  a \subset b \to (c|x)(x \in a \to x \in b)
\]
が成り立つ.
ここで$x$が$a$, $b$の中に自由変数として現れないことから, 
代入法則 \ref{substfree}, \ref{substfund}によりこの記号列は
\[
  a \subset b \to (c \in a \to c \in b)
\]
と一致する.
故にこれが成り立つ.
このことと推論法則 \ref{dedmp}によって1), 2)は直ちに得られる.
\halmos




\mathstrut
\begin{thm}
\label{sthmnotsubset}%定理2.7%1)と2)は新規%確認済
$a$と$b$を集合とし, $x$をこれらの中に自由変数として現れない文字とする.
このとき
\begin{equation}
\label{sthmnotsubset1}
  a \not\subset b \leftrightarrow \exists x(x \in a \wedge x \notin b)
\end{equation}
が成り立つ.
またこのことから, 次の1), 2)が成り立つ.

1)
$a \not\subset b$ならば, $\exists x(x \in a \wedge x \notin b)$.

2)
$\exists x(x \in a \wedge x \notin b)$ならば, $a \not\subset b$.
\end{thm}


\noindent{\bf 証明}
~このとき定義から$a \subset b$は$\forall x(x \in a \to x \in b)$と同じだから, 
Thm \ref{thmaequandm}より
\begin{equation}
\label{sthmnotsubset2}
  a \not\subset b \leftrightarrow \exists x(\neg (x \in a \to x \in b))
\end{equation}
が成り立つ.
またThm \ref{thmquantweq}より
\begin{equation}
\label{sthmnotsubset3}
  \exists x(\neg (x \in a \to x \in b)) \leftrightarrow \exists x(x \in a \wedge x \notin b)
\end{equation}
が成り立つ.
そこで(\ref{sthmnotsubset2}), (\ref{sthmnotsubset3})から, 
推論法則 \ref{dedeqtrans}によって(\ref{sthmnotsubset1})が成り立つ.
1), 2)は(\ref{sthmnotsubset1})と推論法則 \ref{dedeqfund}によって直ちに得られる.
\halmos




\mathstrut
\begin{thm}
\label{sthmnotsubsetbasis}%定理2.8%確認済
$a$, $b$, $c$を集合とするとき, 
\begin{equation}
\label{sthmnotsubsetbasis1}
  c \in a \wedge c \notin b \to a \not\subset b
\end{equation}
が成り立つ.
またこのことから, 次の(\ref{sthmnotsubsetbasis2})が成り立つ.
\begin{equation}
\label{sthmnotsubsetbasis2}
  c \in a \text{と} c \notin b \text{が共に成り立てば,} ~a \not\subset b.
\end{equation}
\end{thm}


\noindent{\bf 証明}
~Thm \ref{n1atb1tawnb}より
\begin{equation}
\label{sthmnotsubsetbasis3}
  c \in a \wedge c \notin b \to \neg (c \in a \to c \in b)
\end{equation}
が成り立つ.
また定理 \ref{sthmsubsetbasis}より
\[
  a \subset b \to (c \in a \to c \in b)
\]
が成り立つから, 推論法則 \ref{dedcp}により
\begin{equation}
\label{sthmnotsubsetbasis4}
  \neg (c \in a \to c \in b) \to a \not\subset b
\end{equation}
が成り立つ.
そこで(\ref{sthmnotsubsetbasis3}), (\ref{sthmnotsubsetbasis4})から, 
推論法則 \ref{dedmmp}によって(\ref{sthmnotsubsetbasis1})が成り立つ.
(\ref{sthmnotsubsetbasis2})が成り立つことは, 
(\ref{sthmnotsubsetbasis1})と推論法則 \ref{dedmp}, \ref{dedwedge}によって明らかである.
\halmos




\mathstrut
\begin{thm}
\label{sthm=tsubseteq}%定理2.9%確認済
$a$, $b$, $c$を集合とするとき, 
\begin{equation}
\label{sthm=tsubseteq1}
  a = b \to (a \subset c \leftrightarrow b \subset c), ~~
  a = b \to (c \subset a \leftrightarrow c \subset b)
\end{equation}
が成り立つ.
またこのことから, 次の1), 2)が成り立つ.

1)
$a = b$ならば, $a \subset c \leftrightarrow b \subset c$, $c \subset a \leftrightarrow c \subset b$.

2)
$a = b$とする.
このとき$a \subset c$ならば, $b \subset c$である.
また$b \subset c$ならば, $a \subset c$である.
また$c \subset a$ならば, $c \subset b$である.
また$c \subset b$ならば, $c \subset a$である.
\end{thm}


\noindent{\bf 証明}
~$x$を$c$の中に自由変数として現れない文字とするとき, Thm \ref{thms5eq}より
\[
  a = b \to ((a|x)(x \subset c) \leftrightarrow (b|x)(x \subset c)), ~~
  a = b \to ((a|x)(c \subset x) \leftrightarrow (b|x)(c \subset x))
\]
が共に成り立つが, 代入法則 \ref{substfree}, \ref{substsubset}によれば
これらの記号列はそれぞれ
\[
  a = b \to (a \subset c \leftrightarrow b \subset c), ~~
  a = b \to (c \subset a \leftrightarrow c \subset b)
\]
と一致するから, これらが共に成り立つ.

\noindent
1)
(\ref{sthm=tsubseteq1})と推論法則 \ref{dedmp}によって明らか.

\noindent
2)
1)と推論法則 \ref{dedeqfund}によって明らか.
\halmos




\mathstrut
\begin{thm}
\label{sthm=&subset}%定理2.10%確認済
$a$, $b$, $c$を集合とするとき, 
\begin{align*}
  &a = b \wedge a \subset c \to b \subset c, ~~
  a = b \wedge b \subset c \to a \subset c, \\
  \mbox{} \\
  &a = b \wedge c \subset a \to c \subset b, ~~
  a = b \wedge c \subset b \to c \subset a
\end{align*}
がすべて成り立つ.
\end{thm}


\noindent{\bf 証明}
~$x$を$c$の中に自由変数として現れない文字とするとき, 
Thm \ref{thmfroms5t}より
\begin{align*}
  &a = b \wedge (a|x)(x \subset c) \to (b|x)(x \subset c), ~~
  a = b \wedge (b|x)(x \subset c) \to (a|x)(x \subset c), \\
  \mbox{} \\
  &a = b \wedge (a|x)(c \subset x) \to (b|x)(c \subset x), ~~
  a = b \wedge (b|x)(c \subset x) \to (a|x)(c \subset x)
\end{align*}
がすべて成り立つが, 代入法則 \ref{substfree}, \ref{substsubset}によれば
これらの記号列はそれぞれ
\begin{align*}
  &a = b \wedge a \subset c \to b \subset c, ~~
  a = b \wedge b \subset c \to a \subset c, \\
  \mbox{} \\
  &a = b \wedge c \subset a \to c \subset b, ~~
  a = b \wedge c \subset b \to c \subset a
\end{align*}
と一致するから, これらがすべて成り立つ.
\halmos




\mathstrut
\begin{thm}
\label{sthmn=&subset}%定理2.11%新規%確認済
$a$, $b$, $c$を集合とするとき, 
\begin{align*}
  &a \subset c \wedge b \not\subset c \to a \neq b, ~~
  a \not\subset c \wedge b \subset c \to a \neq b, \\
  \mbox{} \\
  &c \subset a \wedge c \not\subset b \to a \neq b, ~~
  c \not\subset a \wedge c \subset b \to a \neq b
\end{align*}
がすべて成り立つ.
またこのことから, 次の(\ref{sthmn=&subset1})が成り立つ.
\begin{align}
\label{sthmn=&subset1}
  &a \subset c \text{と} b \not\subset c \text{が共に成り立てば,} ~a \neq b. ~
  \text{また} a \not\subset c \text{と} b \subset c \text{が共に成り立てば,} ~a \neq b. \\
  &\text{また} c \subset a \text{と} c \not\subset b \text{が共に成り立てば,} ~a \neq b. ~
  \text{また} c \not\subset a \text{と} c \subset b \text{が共に成り立てば,} ~a \neq b. \notag
\end{align}
\end{thm}


\noindent{\bf 証明}
~$x$を$c$の中に自由変数として現れない文字とするとき, 
Thm \ref{thms5n=}より
\begin{align*}
  &(a|x)(x \subset c) \wedge \neg (b|x)(x \subset c) \to a \neq b, ~~
  \neg (a|x)(x \subset c) \wedge (b|x)(x \subset c) \to a \neq b, \\
  \mbox{} \\
  &(a|x)(c \subset x) \wedge \neg (b|x)(c \subset x) \to a \neq b, ~~
  \neg (a|x)(c \subset x) \wedge (b|x)(c \subset x) \to a \neq b
\end{align*}
がすべて成り立つが, 代入法則 \ref{substfree}, \ref{substsubset}によれば
これらの記号列はそれぞれ
\begin{align*}
  &a \subset c \wedge b \not\subset c \to a \neq b, ~~
  a \not\subset c \wedge b \subset c \to a \neq b, \\
  \mbox{} \\
  &c \subset a \wedge c \not\subset b \to a \neq b, ~~
  c \not\subset a \wedge c \subset b \to a \neq b
\end{align*}
と一致するから, これらがすべて成り立つ.
(\ref{sthmn=&subset1})が成り立つことは, これらと推論法則 \ref{dedmp}, \ref{dedwedge}によって明らかである.
\halmos




\mathstrut
\begin{thm}
\label{sthmsubsetself}%定理2.12%確認済
$a$を集合とするとき, 
\[
  a \subset a
\]
が成り立つ.
\end{thm}


\noindent{\bf 証明}
~$x$を$a$の中に自由変数として現れない文字とするとき, 
$a \subset a$は$\forall x(x \in a \to x \in a)$である.
Thm \ref{allx1rtr1}より, これは定理である.
\halmos




\mathstrut
\begin{thm}
\label{sthm=tsubset}%定理2.13%確認済
$a$と$b$を集合とするとき, 
\begin{equation}
\label{sthm=tsubset1}
  a = b \to a \subset b, ~~
  a = b \to b \subset a
\end{equation}
が成り立つ.
またこのことから, 次の(\ref{sthm=tsubset2})が成り立つ.
\begin{equation}
\label{sthm=tsubset2}
  a = b \text{ならば,} ~a \subset b, ~b \subset a.
\end{equation}
\end{thm}


\noindent{\bf 証明}
~定理 \ref{sthmsubsetself}より$a \subset a$が成り立つから, 
推論法則 \ref{dedatawbtrue2}により
\begin{equation}
\label{sthm=tsubset3}
  a = b \to a = b \wedge a \subset a
\end{equation}
が成り立つ.
また定理 \ref{sthm=&subset}より
\begin{align}
  \label{sthm=tsubset4}
  &a = b \wedge a \subset a \to a \subset b, \\
  \mbox{} \notag \\
  \label{sthm=tsubset5}
  &a = b \wedge a \subset a \to b \subset a
\end{align}
が共に成り立つ.
そこで(\ref{sthm=tsubset3})と(\ref{sthm=tsubset4}), (\ref{sthm=tsubset3})と(\ref{sthm=tsubset5})から, 
それぞれ推論法則 \ref{dedmmp}によって(\ref{sthm=tsubset1})が成り立つ.
このことと推論法則 \ref{dedmp}によって(\ref{sthm=tsubset2})は直ちに得られる.
\halmos




\mathstrut
\begin{thm}
\label{sthmsubsettrans}%定理2.14%確認済
$a$, $b$, $c$を集合とするとき, 
\[
  a \subset b \wedge b \subset c \to a \subset c
\]
が成り立つ.
またこのことから, 次の(\ref{sthmsubsettrans1})が成り立つ.
\begin{equation}
\label{sthmsubsettrans1}
  a \subset b \text{と} b \subset c \text{が共に成り立てば,} ~a \subset c.
\end{equation}
\end{thm}


\noindent{\bf 証明}
~$x$を$a$, $b$, $c$の中に自由変数として現れない, 定数でない文字とする.
このとき定理 \ref{sthmsubsetbasis}より
\[
  a \subset b \to (x \in a \to x \in b), ~~
  b \subset c \to (x \in b \to x \in c)
\]
が共に成り立つから, 推論法則 \ref{dedfromaddw}により
\begin{equation}
\label{sthmsubsettrans2}
  a \subset b \wedge b \subset c \to (x \in a \to x \in b) \wedge (x \in b \to x \in c)
\end{equation}
が成り立つ.
またThm \ref{1atb1t11btc1t1atc11}より
\[
  (x \in a \to x \in b) \to ((x \in b \to x \in c) \to (x \in a \to x \in c))
\]
が成り立つから, 推論法則 \ref{dedtwch}により
\begin{equation}
\label{sthmsubsettrans3}
  (x \in a \to x \in b) \wedge (x \in b \to x \in c) \to (x \in a \to x \in c)
\end{equation}
が成り立つ.
そこで(\ref{sthmsubsettrans2}), (\ref{sthmsubsettrans3})から, 推論法則 \ref{dedmmp}によって
\begin{equation}
\label{sthmsubsettrans4}
  a \subset b \wedge b \subset c \to (x \in a \to x \in c)
\end{equation}
が成り立つ.
ここで$x$が$a$, $b$, $c$の中に自由変数として現れないことから, 
変数法則 \ref{valwedge}, \ref{valsubset}により, 
$x$は$a \subset b \wedge b \subset c$の中に自由変数として現れない.
また$x$は定数でない.
これらのことと(\ref{sthmsubsettrans4})が成り立つことから, 推論法則 \ref{dedalltquansepfreeconst}により
\[
  a \subset b \wedge b \subset c \to \forall x(x \in a \to x \in c)
\]
が成り立つ.
ここで$x$が$a$, $c$の中に自由変数として現れないことから, この記号列は
$a \subset b \wedge b \subset c \to a \subset c$と同じである.
故にこれが成り立つ.
(\ref{sthmsubsettrans1})はこれと推論法則 \ref{dedmp}, \ref{dedwedge}から直ちに得られる.
\halmos




\mathstrut
\begin{thm}
\label{sthmspinsubset}%定理2.15%新規%確認済
$a$と$b$を集合, $R$を関係式とし, $x$を$a$と$b$の中に自由変数として現れない文字とする.
このとき
\begin{align*}
  &a \subset b \to ((\exists x \in a)(R) \to (\exists x \in b)(R)), \\
  \mbox{} \\
  &a \subset b \to ((\forall x \in b)(R) \to (\forall x \in a)(R)), \\
  \mbox{} \\
  &a \subset b \to ((!x \in b)(R) \to (!x \in a)(R))
\end{align*}
がすべて成り立つ.
またこれらから, 次の(\ref{sthmspinsubset1})が成り立つ.
\begin{align}
\label{sthmspinsubset1}
  &a \subset b \text{ならば,} ~(\exists x \in a)(R) \to (\exists x \in b)(R), ~
  (\forall x \in b)(R) \to (\forall x \in a)(R), \\
  &(!x \in b)(R) \to (!x \in a)(R) \text{がすべて成り立つ.} \notag
\end{align}
\end{thm}


\noindent{\bf 証明}
~このとき$a \subset b$は$\forall x(x \in a \to x \in b)$と同じである.
故に前半はThm \ref{thmallpretspquansep}, \ref{thmallpretsp!sep}より明らか.
(\ref{sthmspinsubset1})は前半と推論法則 \ref{dedmp}から直ちに得られる.
\halmos




\mathstrut%確認済
次に, 集合論における最も基本的な公理である外延公理を導入する.

以下の変形法則 \ref{axiom1}, 変数法則 \ref{valaxiom1}, 構成法則 \ref{formaxiom1}では, 
$\mathscr{T}$を特殊記号として$=$と$\in$を持つ理論とし, 
これらの法則における``記号列'', ``関係式''とは, 
それぞれ$\mathscr{T}$の記号列, $\mathscr{T}$の関係式のこととする.




\mathstrut
\begin{defo}
\label{axiom1}%変形11%確認済
$x$, $y$, $z$, $w$を文字とし, 
$x$と$y$は互いに異なり, $z$と$w$も互いに異なるとする.
このとき
\[
  \forall x(\forall y(x \subset y \wedge y \subset x \to x = y)) 
  \equiv \forall z(\forall w(z \subset w \wedge w \subset z \to z = w))
\]
が成り立つ.
\end{defo}


\noindent{\bf 証明}
~$u$, $v$を, 互いに異なり,共に$x$, $y$, $z$, $w$のいずれとも異なる文字とする.
このとき変数法則 \ref{valfund}, \ref{valwedge}, \ref{valquan}, \ref{valsubset}によってわかるように, 
$u$は$\forall y(x \subset y \wedge y \subset x \to x = y)$の中に自由変数として現れない.
故に代入法則 \ref{substquantrans}により
\begin{equation}
\label{axiom11}
  \forall x(\forall y(x \subset y \wedge y \subset x \to x = y)) 
  \equiv \forall u((u|x)(\forall y(x \subset y \wedge y \subset x \to x = y)))
\end{equation}
が成り立つ.
また$y$が$x$, $u$と異なることから, 代入法則 \ref{substquan}により
\begin{equation}
\label{axiom12}
  (u|x)(\forall y(x \subset y \wedge y \subset x \to x = y)) 
  \equiv \forall y((u|x)(x \subset y \wedge y \subset x \to x = y))
\end{equation}
が成り立つ.
また$x$が$y$と異なることと代入法則 \ref{substfund}, \ref{substwedge}, \ref{substsubset}により, 
\begin{equation}
\label{axiom13}
  (u|x)(x \subset y \wedge y \subset x \to x = y) 
  \equiv u \subset y \wedge y \subset u \to u = y
\end{equation}
が成り立つ.
そこで(\ref{axiom11}), (\ref{axiom12}), (\ref{axiom13})から, 
\begin{equation}
\label{axiom14}
  \forall x(\forall y(x \subset y \wedge y \subset x \to x = y)) 
  \equiv \forall u(\forall y(u \subset y \wedge y \subset u \to u = y))
\end{equation}
が成り立つことがわかる.
また$v$が$u$, $y$と異なることから, 
変数法則 \ref{valfund}, \ref{valwedge}, \ref{valsubset}によってわかるように, 
$v$は$u \subset y \wedge y \subset u \to u = y$の中に自由変数として現れない.
故に代入法則 \ref{substquantrans}により
\begin{equation}
\label{axiom15}
  \forall y(u \subset y \wedge y \subset u \to u = y) 
  \equiv \forall v((v|y)(u \subset y \wedge y \subset u \to u = y))
\end{equation}
が成り立つ.
また$y$が$u$と異なることと代入法則 \ref{substfund}, \ref{substwedge}, \ref{substsubset}により, 
\begin{equation}
\label{axiom16}
  (v|y)(u \subset y \wedge y \subset u \to u = y) 
  \equiv u \subset v \wedge v \subset u \to u = v
\end{equation}
が成り立つ.
そこで(\ref{axiom15}), (\ref{axiom16})から, 
\begin{equation}
\label{axiom17}
  \forall u(\forall y(u \subset y \wedge y \subset u \to u = y)) 
  \equiv \forall u(\forall v(u \subset v \wedge v \subset u \to u = v))
\end{equation}
が成り立つことがわかる.
故に(\ref{axiom14}), (\ref{axiom17})から, 
\[
  \forall x(\forall y(x \subset y \wedge y \subset x \to x = y)) 
  \equiv \forall u(\forall v(u \subset v \wedge v \subset u \to u = v))
\]
が成り立つ.
以上の議論と全く同様にして, 
\[
  \forall z(\forall w(z \subset w \wedge w \subset z \to z = w)) 
  \equiv \forall u(\forall v(u \subset v \wedge v \subset u \to u = v))
\]
も成り立つ.
従って, $\forall x(\forall y(x \subset y \wedge y \subset x \to x = y))$と
$\forall z(\forall w(z \subset w \wedge w \subset z \to z = w))$は同一の記号列である.
\halmos




\mathstrut
\begin{valu}
\label{valaxiom1}%変数21%確認済
$x$と$y$を文字とするとき, 
$\forall x(\forall y(x \subset y \wedge y \subset x \to x = y))$は
自由変数を持たない.
\end{valu}


\noindent{\bf 証明}
~変数法則 \ref{valquan}によってわかるように, $x$と$y$は共に
$\forall x(\forall y(x \subset y \wedge y \subset x \to x = y))$の中に自由変数として現れない.
また$z$を$x$, $y$と異なる文字とするとき, 
変数法則 \ref{valfund}, \ref{valwedge}, \ref{valquan}, \ref{valsubset}によってわかるように, 
$z$は$\forall x(\forall y(x \subset y \wedge y \subset x \to x = y))$の中に自由変数として現れない.
故に本法則が成り立つ.
\halmos




\mathstrut
\begin{form}
\label{formaxiom1}%構成38%確認済
$x$と$y$を文字とするとき, 
$\forall x(\forall y(x \subset y \wedge y \subset x \to x = y))$は
関係式である.
\end{form}


\noindent{\bf 証明}
~構成法則 \ref{formfund}, \ref{formwedge}, \ref{formquan}, \ref{formsubset}によって明らか.
\halmos




\mathstrut
\begin{defi}
\label{defaxiom1}%定義2%確認済
$x$と$y$を異なる文字とするとき, 次の記号列A1は集合論の明示的公理である: 
\begin{center}
  A1. ~~$\forall x(\forall y(x \subset y \wedge y \subset x \to x = y))$
\end{center}
これを\textbf{外延公理} (axiom of extensionality) という.
A1は構成法則 \ref{formaxiom1}により確かに関係式である.
また変数法則 \ref{valaxiom1}により, A1は自由変数を持たない.
また変形法則 \ref{axiom1}により, A1は仮定を満たす文字$x$, $y$の取り方に依らずに定まる記号列である.
即ち$z$と$w$を異なる文字とするとき, 
A1は$\forall z(\forall w(z \subset w \wedge w \subset z \to z = w))$と一致する.
\end{defi}




\mathstrut%確認済
以後特に断らない限り, 上記のA1は定理であるとする.




\mathstrut
\begin{thm}
\label{sthmaxiom1}%定理2.16%確認済
$a$と$b$を集合とするとき, 
\[
  a \subset b \wedge b \subset a \leftrightarrow a = b
\]
が成り立つ.
またこのことから特に, 次の(\ref{sthmaxiom11})が成り立つ.
\begin{equation}
\label{sthmaxiom11}
  a \subset b \text{と} b \subset a \text{が共に成り立てば,} ~a = b.
\end{equation}
\end{thm}


\noindent{\bf 証明}
~$x$と$y$を互いに異なる文字とする.
このとき外延公理A1より
\[
  \forall x(\forall y(x \subset y \wedge y \subset x \to x = y))
\]
が成り立つから, 推論法則 \ref{dedgallfund2}により
\[
  (a|x, b|y)(x \subset y \wedge y \subset x \to x = y)
\]
が成り立つ.
一般代入法則 \ref{gsubstfund}, \ref{gsubstwedge}, \ref{gsubstsubset}によれば, この記号列は
\begin{equation}
\label{sthmaxiom12}
  a \subset b \wedge b \subset a \to a = b
\end{equation}
と一致する.
故にこれが定理となる.
また定理 \ref{sthm=tsubset}より
\[
  a = b \to a \subset b, ~~
  a = b \to b \subset a
\]
が共に成り立つから, 推論法則 \ref{dedprewedge}により
\begin{equation}
\label{sthmaxiom13}
  a = b \to a \subset b \wedge b \subset a
\end{equation}
が成り立つ.
そこで(\ref{sthmaxiom12}), (\ref{sthmaxiom13})から, 推論法則 \ref{dedequiv}により
\[
  a \subset b \wedge b \subset a \leftrightarrow a = b
\]
が成り立つ.
(\ref{sthmaxiom11})が成り立つことは, 
これと推論法則 \ref{dedwedge}, \ref{dedeqfund}によって明らかである.
\halmos




\mathstrut
\begin{thm}
\label{sthmset=}%定理2.17%確認済
$a$と$b$を集合とし, $x$をこれらの中に自由変数として現れない文字とする.
このとき
\begin{equation}
\label{sthmset=1}
  \forall x(x \in a \leftrightarrow x \in b) \leftrightarrow a = b
\end{equation}
が成り立つ.
またこのことから, 次の1), 2)が成り立つ.

1)
$\forall x(x \in a \leftrightarrow x \in b)$ならば, $a = b$.
また$a = b$ならば, $\forall x(x \in a \leftrightarrow x \in b)$.

2)
$x$が定数でなく, $x \in a \leftrightarrow x \in b$が成り立てば, $a = b$.
\end{thm}


\noindent{\bf 証明}
~Thm \ref{thmallw}より
\[
  \forall x(x \in a \leftrightarrow x \in b) 
  \leftrightarrow \forall x(x \in a \to x \in b) \wedge \forall x(x \in b \to x \in a)
\]
が成り立つが, $x$が$a$, $b$の中に自由変数として現れないことからこの記号列は
\begin{equation}
\label{sthmset=2}
  \forall x(x \in a \leftrightarrow x \in b) 
  \leftrightarrow a \subset b \wedge b \subset a
\end{equation}
と同じだから, これが定理となる.
また定理 \ref{sthmaxiom1}より
\begin{equation}
\label{sthmset=3}
  a \subset b \wedge b \subset a \leftrightarrow a = b
\end{equation}
が成り立つ.
そこで(\ref{sthmset=2}), (\ref{sthmset=3})から, 
推論法則 \ref{dedeqtrans}によって(\ref{sthmset=1})が成り立つ.

\noindent
1)
(\ref{sthmset=1})と推論法則 \ref{dedeqfund}によって明らか.

\noindent
2)
1)と推論法則 \ref{dedltthmquan}によって明らか.
\halmos




\mathstrut
\begin{defi}
\label{defpsubset}%定義3%新規%確認済
$\mathscr{T}$を特殊記号として$=$と$\in$を持つ理論とし, $a$と$b$を$\mathscr{T}$の記号列とする.
$a \subset b \wedge a \neq b$という記号列を, $(a) \subsetneqq (b)$または$(b) \supsetneqq (a)$とも書く.
括弧は適宜省略する.
$a$と$b$が集合ならば, 構成法則 \ref{formfund}, \ref{formwedge}, \ref{formsubset}からわかるように, 
$a \subsetneqq b$は関係式である.
集合$b$に対して$a \subsetneqq b$が定理となるような集合$a$をすべて, 
($\mathscr{T}$における) $b$の\textbf{真部分集合} (proper subset) という.
\end{defi}




\mathstrut
\begin{thm}
\label{sthmpsubsetbasis}%定理2.18%新規%確認済
$a$と$b$を集合とするとき, 
\begin{equation}
\label{sthmpsubsetbasis1}
  a \subset b \leftrightarrow a \subsetneqq b \vee a = b
\end{equation}
が成り立つ.
\end{thm}


\noindent{\bf 証明}
~$a \subsetneqq b$の定義から, Thm \ref{at1btawb1}より
\[
  a \subset b \to (a \neq b \to a \subsetneqq b), 
\]
即ち
\begin{equation}
\label{sthmpsubsetbasis2}
  a \subset b \to a = b \vee a \subsetneqq b
\end{equation}
が成り立つ.
またThm \ref{avbtbva}より
\begin{equation}
\label{sthmpsubsetbasis3}
  a = b \vee a \subsetneqq b \to a \subsetneqq b \vee a = b
\end{equation}
が成り立つ.
そこで(\ref{sthmpsubsetbasis2}), (\ref{sthmpsubsetbasis3})から, 
推論法則 \ref{dedmmp}によって
\begin{equation}
\label{sthmpsubsetbasis4}
  a \subset b \to a \subsetneqq b \vee a = b
\end{equation}
が成り立つ.
またThm \ref{awbta}より
\[
  a \subsetneqq b \to a \subset b
\]
が成り立ち, 定理 \ref{sthm=tsubset}より
\[
  a = b \to a \subset b
\]
が成り立つから, 推論法則 \ref{deddil}により
\begin{equation}
\label{sthmpsubsetbasis5}
  a \subsetneqq b \vee a = b \to a \subset b
\end{equation}
が成り立つ.
そこで(\ref{sthmpsubsetbasis4}), (\ref{sthmpsubsetbasis5})から, 
推論法則 \ref{dedequiv}により(\ref{sthmpsubsetbasis1})が成り立つ.
\halmos




\mathstrut
\begin{thm}
\label{sthmpsubset&axiom1}%定理2.19%新規%確認済
$a$と$b$を集合とするとき, 
\begin{equation}
\label{sthmpsubset&axiom11}
  a \subsetneqq b \leftrightarrow a \subset b \wedge b \not\subset a
\end{equation}
が成り立つ.
特に, 
\begin{equation}
\label{sthmpsubset&axiom12}
  a \subsetneqq b \to b \not\subset a
\end{equation}
が成り立つ.
またこれらから, 次の1), 2)が成り立つ.

1)
$a \subsetneqq b$ならば, $b \not\subset a$.

2)
$a \subset b$と$b \not\subset a$が共に成り立てば, $a \subsetneqq b$.
\end{thm}


\noindent{\bf 証明}
~Thm \ref{awbta}より
\begin{equation}
\label{sthmpsubset&axiom13}
  a \subsetneqq b \to a \subset b
\end{equation}
が成り立つ.
また定理 \ref{sthmaxiom1}と推論法則 \ref{dedequiv}により
\[
  a \subset b \wedge b \subset a \to a = b
\]
が成り立つから, 推論法則 \ref{dedtwch}により
\begin{equation}
\label{sthmpsubset&axiom14}
  a \subset b \to (b \subset a \to a = b)
\end{equation}
が成り立つ.
またThm \ref{1atb1t1nbtna1}より
\begin{equation}
\label{sthmpsubset&axiom15}
  (b \subset a \to a = b) \to (a \neq b \to b \not\subset a)
\end{equation}
が成り立つ.
そこで(\ref{sthmpsubset&axiom14}), (\ref{sthmpsubset&axiom15})から, 推論法則 \ref{dedmmp}によって
\[
  a \subset b \to (a \neq b \to b \not\subset a)
\]
が成り立つ.
故に推論法則 \ref{dedtwch}により(\ref{sthmpsubset&axiom12})が成り立つ.
そこで(\ref{sthmpsubset&axiom12}), (\ref{sthmpsubset&axiom13})から, 推論法則 \ref{dedprewedge}により
\begin{equation}
\label{sthmpsubset&axiom16}
  a \subsetneqq b \to a \subset b \wedge b \not\subset a
\end{equation}
が成り立つ.
また定理 \ref{sthm=tsubset}より
\[
  a = b \to b \subset a
\]
が成り立つから, 推論法則 \ref{dedcp}により
\[
  b \not\subset a \to a \neq b
\]
が成り立つ.
故に推論法則 \ref{dedaddw}により
\begin{equation}
\label{sthmpsubset&axiom17}
  a \subset b \wedge b \not\subset a \to a \subsetneqq b
\end{equation}
が成り立つ.
(\ref{sthmpsubset&axiom16}), (\ref{sthmpsubset&axiom17})から, 
推論法則 \ref{dedequiv}により(\ref{sthmpsubset&axiom11})が成り立つ.

\noindent
1)
(\ref{sthmpsubset&axiom12})と推論法則 \ref{dedmp}によって明らか.

\noindent
2)
(\ref{sthmpsubset&axiom11})と推論法則 \ref{dedwedge}, \ref{dedeqfund}によって明らか.
\halmos




\mathstrut
\begin{thm}
\label{sthmpsubsettrans}%定理2.20%新規%確認済
$a$, $b$, $c$を集合とするとき, 
\begin{align}
  \label{sthmpsubsettrans1}
  &a \subset b \wedge b \subsetneqq c \to a \subsetneqq c, \\
  \mbox{} \notag \\
  \label{sthmpsubsettrans2}
  &a \subsetneqq b \wedge b \subset c \to a \subsetneqq c, \\
  \mbox{} \notag \\
  \label{sthmpsubsettrans3}
  &a \subsetneqq b \wedge b \subsetneqq c \to a \subsetneqq c
\end{align}
が成り立つ.
またこのことから, 次の(\ref{sthmpsubsettrans4})が成り立つ.
\begin{align}
\label{sthmpsubsettrans4}
  &a \subset b \text{と} b \subsetneqq c \text{が共に成り立てば,} ~a \subsetneqq c. ~
  \text{また} a \subsetneqq b \text{と} b \subset c \text{が共に成り立てば,} ~a \subsetneqq c. \\
  &\text{また} a \subsetneqq b \text{と} b \subsetneqq c \text{が共に成り立てば,} ~a \subsetneqq c. \notag
\end{align}
\end{thm}


\noindent{\bf 証明}
~まず(\ref{sthmpsubsettrans1}), (\ref{sthmpsubsettrans2})が成り立つことを示す.
Thm \ref{awbta}より
\[
  b \subsetneqq c \to b \subset c, ~~
  a \subsetneqq b \to a \subset b
\]
が共に成り立つから, 推論法則 \ref{dedaddw}により
\begin{align}
  \label{sthmpsubsettrans5}
  &a \subset b \wedge b \subsetneqq c \to a \subset b \wedge b \subset c, \\
  \mbox{} \notag \\
  \label{sthmpsubsettrans6}
  &a \subsetneqq b \wedge b \subset c \to a \subset b \wedge b \subset c
\end{align}
が共に成り立つ.
また定理 \ref{sthmsubsettrans}より
\begin{equation}
\label{sthmpsubsettrans7}
  a \subset b \wedge b \subset c \to a \subset c
\end{equation}
が成り立つ.
そこで(\ref{sthmpsubsettrans5})と(\ref{sthmpsubsettrans7}), 
(\ref{sthmpsubsettrans6})と(\ref{sthmpsubsettrans7})から, それぞれ推論法則 \ref{dedmmp}によって
\begin{align}
  \label{sthmpsubsettrans8}
  &a \subset b \wedge b \subsetneqq c \to a \subset c, \\
  \mbox{} \notag \\
  \label{sthmpsubsettrans9}
  &a \subsetneqq b \wedge b \subset c \to a \subset c
\end{align}
が成り立つ.
また定理 \ref{sthmpsubset&axiom1}より
\[
  b \subsetneqq c \to c \not\subset b, ~~
  a \subsetneqq b \to b \not\subset a
\]
が共に成り立つから, 推論法則 \ref{dedaddw}により
\begin{align}
  \label{sthmpsubsettrans10}
  &a \subset b \wedge b \subsetneqq c \to a \subset b \wedge c \not\subset b, \\
  \mbox{} \notag \\
  \label{sthmpsubsettrans11}
  &a \subsetneqq b \wedge b \subset c \to b \not\subset a \wedge b \subset c
\end{align}
が共に成り立つ.
また定理 \ref{sthmn=&subset}より
\begin{align}
  \label{sthmpsubsettrans12}
  &a \subset b \wedge c \not\subset b \to a \neq c, \\
  \mbox{} \notag \\
  \label{sthmpsubsettrans13}
  &b \not\subset a \wedge b \subset c \to a \neq c
\end{align}
が共に成り立つ.
そこで(\ref{sthmpsubsettrans10})と(\ref{sthmpsubsettrans12}), 
(\ref{sthmpsubsettrans11})と(\ref{sthmpsubsettrans13})から, それぞれ推論法則 \ref{dedmmp}によって
\begin{align}
  \label{sthmpsubsettrans14}
  &a \subset b \wedge b \subsetneqq c \to a \neq c, \\
  \mbox{} \notag \\
  \label{sthmpsubsettrans15}
  &a \subsetneqq b \wedge b \subset c \to a \neq c
\end{align}
が成り立つ.
故に(\ref{sthmpsubsettrans8})と(\ref{sthmpsubsettrans14}), 
(\ref{sthmpsubsettrans9})と(\ref{sthmpsubsettrans15})から, それぞれ推論法則 \ref{dedprewedge}により
(\ref{sthmpsubsettrans1}), (\ref{sthmpsubsettrans2})が成り立つ.

次に(\ref{sthmpsubsettrans3})が成り立つことを示す.
Thm \ref{awbta}より$a \subsetneqq b \to a \subset b$が成り立つから, 推論法則 \ref{dedaddw}により
\[
  a \subsetneqq b \wedge b \subsetneqq c \to a \subset b \wedge b \subsetneqq c
\]
が成り立つ.
これと(\ref{sthmpsubsettrans1})から, 推論法則 \ref{dedmmp}によって(\ref{sthmpsubsettrans3})が成り立つ.

(\ref{sthmpsubsettrans4})が成り立つことは, 
(\ref{sthmpsubsettrans1}), (\ref{sthmpsubsettrans2}), (\ref{sthmpsubsettrans3})と
推論法則 \ref{dedmp}, \ref{dedwedge}から直ちにわかる.
\halmos
%ここまで確認



\newpage
\setcounter{defi}{0}
\section{集合を作り得る関係式}




集合論では$\{x \mid R\}$という形の集合を扱うことが多い.
ここで$x$は文字, $R$は関係式であり, $\{x \mid R\}$は
\begin{equation}
\label{sm}
  \forall x(x \in \{x \mid R\} \leftrightarrow R)
\end{equation}
を満たす集合である.
即ち$R$を$x$に関する命題とみるならば, 
$\{x \mid R\}$は直観的には``$R$を満たすような$x$の全体から成る集合''を表す.
しかし任意の関係式$R$に対して$\{x \mid R\}$が(\ref{sm})を満たすわけではない.
そのためには$R$に関する適当な条件が必要がある.
それらの条件については次節以降で述べる.

(\ref{sm})が成り立つとき, $R$は$x$について集合を作り得るという.
この節ではこの概念, 及び記号列$\{x \mid R\}$の正確な定義を与え, 
そのごく基本的な性質について述べる.




\mathstrut
\begin{thm}
\label{sthmsm!}%定理3.1%確認済
$R$を関係式とし, $x$を文字とする.
また$y$を$x$と異なり, $R$の中に自由変数として現れない文字とする.
このとき
\begin{equation}
\label{sthmsm!1}
  !y(\forall x(x \in y \leftrightarrow R))
\end{equation}
が成り立つ.
\end{thm}


\noindent{\bf 証明}
~関係式$\forall x(x \in y \leftrightarrow R)$を$S$と書く.
また$z$と$w$を互いに異なり, 共に$x$, $y$と異なり, 
$R$の中に自由変数として現れない, 定数でない文字とする.
このとき変数法則 \ref{valequiv}, \ref{valquan}からわかるように, 
$z$と$w$は共に$S$の中に自由変数として現れない.
また$x$が$y$, $z$と異なり, $y$が$R$の中に自由変数として現れないことから, 
代入法則 \ref{substfree}, \ref{substequiv}, \ref{substquan}により
\[
  (z|y)(S) \equiv \forall x(x \in z \leftrightarrow R)
\]
が成り立つことがわかる.
そこでいま$\tau_{x}(\neg (x \in z \leftrightarrow x \in w))$を$T$と書けば, 
$T$は対象式であり, Thm \ref{thmallfund2}より
\[
  (z|y)(S) \to (T|x)(x \in z \leftrightarrow R)
\]
が成り立つ.
ここで$x$が$z$と異なることと代入法則 \ref{substequiv}によれば, この記号列は
\begin{equation}
\label{sthmsm!2}
  (z|y)(S) \to (T \in z \leftrightarrow (T|x)(R))
\end{equation}
と一致する.
故にこれが成り立つ.
以上と全く同様にして, 
\begin{equation}
\label{sthmsm!3}
  (w|y)(S) \to (T \in w \leftrightarrow (T|x)(R))
\end{equation}
も成り立つ.
またThm \ref{1alb1t1bla1}より
\begin{equation}
\label{sthmsm!4}
  (T \in w \leftrightarrow (T|x)(R)) \to ((T|x)(R) \leftrightarrow T \in w)
\end{equation}
が成り立つ.
そこで(\ref{sthmsm!3}), (\ref{sthmsm!4})から, 推論法則 \ref{dedmmp}によって
\[
  (w|y)(S) \to ((T|x)(R) \leftrightarrow T \in w)
\]
が成り立つ.
故にこれと(\ref{sthmsm!2})から, 推論法則 \ref{dedfromaddw}により
\begin{equation}
\label{sthmsm!5}
  (z|y)(S) \wedge (w|y)(S) 
  \to (T \in z \leftrightarrow (T|x)(R)) \wedge ((T|x)(R) \leftrightarrow T \in w)
\end{equation}
が成り立つ.
またThm \ref{1alb1w1blc1t1alc1}より
\begin{equation}
\label{sthmsm!6}
  (T \in z \leftrightarrow (T|x)(R)) \wedge ((T|x)(R) \leftrightarrow T \in w) 
  \to (T \in z \leftrightarrow T \in w)
\end{equation}
が成り立つ.
また$T$の定義から, Thm \ref{thmallfund1}と推論法則 \ref{dedequiv}により
\[
  (T|x)(x \in z \leftrightarrow x \in w) \to \forall x(x \in z \leftrightarrow x \in w)
\]
が成り立つが, $x$が$z$, $w$と異なることと代入法則 \ref{substequiv}によりこの記号列は
\begin{equation}
\label{sthmsm!7}
  (T \in z \leftrightarrow T \in w) \to \forall x(x \in z \leftrightarrow x \in w)
\end{equation}
と一致するから, これが成り立つ.
また$x$が$z$, $w$と異なることから, 定理 \ref{sthmset=}と推論法則 \ref{dedequiv}により
\begin{equation}
\label{sthmsm!8}
  \forall x(x \in z \leftrightarrow x \in w) \to z = w
\end{equation}
が成り立つ.
そこで(\ref{sthmsm!5})---(\ref{sthmsm!8})から, 推論法則 \ref{dedmmp}によって
\[
  (z|y)(S) \wedge (w|y)(S) \to z = w
\]
が成り立つことがわかる.
さてここで$z$と$w$は互いに異なり, 共に$y$と異なり, 
上述のように$S$の中に自由変数として現れず, また定数でない.
故に推論法則 \ref{ded!const}により, $!y(S)$, 即ち(\ref{sthmsm!1})が成り立つ.
\halmos




\mathstrut
\begin{defo}
\label{setmake}%変形12%確認済
$\mathscr{T}$を特殊記号として$\in$を持つ理論とし, $R$を$\mathscr{T}$の記号列, $x$を文字とする.
また$y$と$z$を共に$x$と異なり, $R$の中に自由変数として現れない文字とする.
このとき
\[
  \exists y(\forall x(x \in y \leftrightarrow R)) 
  \equiv \exists z(\forall x(x \in z \leftrightarrow R))
\]
が成り立つ.
\end{defo}


\noindent{\bf 証明}
~$y$と$z$が同じ文字ならば明らかだから, 以下$y$と$z$は異なる文字であるとする.
このとき$z$が$x$, $y$と異なり, $R$の中に自由変数として現れないことから, 
変数法則 \ref{valequiv}, \ref{valquan}により, 
$z$は$\forall x(x \in y \leftrightarrow R)$の中に自由変数として現れない.
故に代入法則 \ref{substquantrans}により
\[
  \exists y(\forall x(x \in y \leftrightarrow R)) \equiv
  \exists z((z|y)(\forall x(x \in y \leftrightarrow R)))
\]
が成り立つ.
また$x$が$y$, $z$と異なり, $y$が$R$の中に自由変数として現れないことから, 
代入法則 \ref{substfree}, \ref{substequiv}, \ref{substquan}によってわかるように
\[
  (z|y)(\forall x(x \in y \leftrightarrow R)) 
  \equiv \forall x(x \in z \leftrightarrow R)
\]
が成り立つ.
故に本法則が成り立つ.
\halmos




\mathstrut
\begin{defi}
\label{defsm}%定義1%確認済
$\mathscr{T}$を特殊記号として$\in$を持つ理論とし, $R$を$\mathscr{T}$の記号列, $x$を文字とする.
また$y$と$z$を共に$x$と異なり, $R$の中に自由変数として現れない文字とする.
このとき変形法則 \ref{setmake}によれば, $\exists y(\forall x(x \in y \leftrightarrow R))$と
$\exists z(\forall x(x \in z \leftrightarrow R))$は同じ記号列となる.
$R$と$x$に対して定まるこの記号列を, ${\rm Set}_{x}(R)$と書き表す.
\end{defi}




\mathstrut%確認済
以下の変数法則 \ref{valsm}, 一般代入法則 \ref{gsubstsm}, 代入法則 \ref{substsmtrans}, \ref{substsm}, 
構成法則 \ref{formsm}では, $\mathscr{T}$を特殊記号として$\in$を持つ理論とし, 
これらの法則における``記号列'', ``関係式''とは, 
それぞれ$\mathscr{T}$の記号列, $\mathscr{T}$の関係式のこととする.




\mathstrut
\begin{valu}
\label{valsm}%変数22%確認済
$R$を記号列とし, $x$を文字とする.

1)
$x$は${\rm Set}_{x}(R)$の中に自由変数として現れない.

2)
$y$を文字とする.
$y$が$R$の中に自由変数として現れなければ, 
$y$は${\rm Set}_{x}(R)$の中に自由変数として現れない.
\end{valu}


\noindent{\bf 証明}
~1)
$z$を$x$と異なり, $R$の中に自由変数として現れない文字とするとき, 
定義から${\rm Set}_{x}(R)$は$\exists z(\forall x(x \in z \leftrightarrow R))$と同じである.
変数法則 \ref{valquan}によれば, $x$はこの中に自由変数として現れない.

\noindent
2)
$y$が$x$と同じ文字ならば1)により明らか.
$y$が$x$と異なる文字ならば, このことと$y$が$R$の中に自由変数として現れないことから, 
定義より${\rm Set}_{x}(R)$は$\exists y(\forall x(x \in y \leftrightarrow R))$と同じである.
変数法則 \ref{valquan}によれば, $y$はこの中に自由変数として現れない.
\halmos




\mathstrut
\begin{gsub}
\label{gsubstsm}%一般代入27%確認済
$R$を記号列とし, $x$を文字とする.
また$n$を自然数とし, $T_{1}, T_{2}, \cdots, T_{n}$を記号列とする.
また$y_{1}, y_{2}, \cdots, y_{n}$を, どの二つも互いに異なる文字とする.
$x$が$y_{1}, y_{2}, \cdots, y_{n}$のいずれとも異なり, かつ
$T_{1}, T_{2}, \cdots, T_{n}$のいずれの記号列の中にも自由変数として現れなければ, 
\[
  (T_{1}|y_{1}, T_{2}|y_{2}, \cdots, T_{n}|y_{n})({\rm Set}_{x}(R)) 
  \equiv {\rm Set}_{x}((T_{1}|y_{1}, T_{2}|y_{2}, \cdots, T_{n}|y_{n})(R))
\]
が成り立つ.
\end{gsub}


\noindent{\bf 証明}
~$z$を$x, y_{1}, y_{2}, \cdots, y_{n}$のいずれとも異なり, 
$R, T_{1}, T_{2}, \cdots, T_{n}$のいずれの記号列の中にも自由変数として現れない文字とする.
このとき定義から${\rm Set}_{x}(R)$は$\exists z(\forall x(x \in z \leftrightarrow R))$だから, 
\begin{equation}
\label{gsubstsm1}
  (T_{1}|y_{1}, T_{2}|y_{2}, \cdots, T_{n}|y_{n})({\rm Set}_{x}(R)) 
  \equiv (T_{1}|y_{1}, T_{2}|y_{2}, \cdots, T_{n}|y_{n})(\exists z(\forall x(x \in z \leftrightarrow R)))
\end{equation}
である.
また$z$が$y_{1}, y_{2}, \cdots, y_{n}$のいずれとも異なり, かつ
$T_{1}, T_{2}, \cdots, T_{n}$のいずれの記号列の中にも自由変数として現れないことから, 
一般代入法則 \ref{gsubstquan}により
\begin{equation}
\label{gsubstsm2}
  (T_{1}|y_{1}, T_{2}|y_{2}, \cdots, T_{n}|y_{n})(\exists z(\forall x(x \in z \leftrightarrow R))) 
  \equiv \exists z((T_{1}|y_{1}, T_{2}|y_{2}, \cdots, T_{n}|y_{n})(\forall x(x \in z \leftrightarrow R)))
\end{equation}
が成り立つ.
また$x$も$y_{1}, y_{2}, \cdots, y_{n}$のいずれとも異なり, かつ
$T_{1}, T_{2}, \cdots, T_{n}$のいずれの記号列の中にも自由変数として現れないから, 
同じく一般代入法則 \ref{gsubstquan}により
\begin{equation}
\label{gsubstsm3}
  (T_{1}|y_{1}, T_{2}|y_{2}, \cdots, T_{n}|y_{n})(\forall x(x \in z \leftrightarrow R)) 
  \equiv \forall x((T_{1}|y_{1}, T_{2}|y_{2}, \cdots, T_{n}|y_{n})(x \in z \leftrightarrow R))
\end{equation}
が成り立つ.
また$x$と$z$が共に$y_{1}, y_{2}, \cdots, y_{n}$のいずれとも異なることと
一般代入法則 \ref{gsubstequiv}から, 
\begin{equation}
\label{gsubstsm4}
  (T_{1}|y_{1}, T_{2}|y_{2}, \cdots, T_{n}|y_{n})(x \in z \leftrightarrow R) 
  \equiv x \in z \leftrightarrow (T_{1}|y_{1}, T_{2}|y_{2}, \cdots, T_{n}|y_{n})(R)
\end{equation}
が成り立つ.
以上の(\ref{gsubstsm1})---(\ref{gsubstsm4})からわかるように, 
$(T_{1}|y_{1}, T_{2}|y_{2}, \cdots, T_{n}|y_{n})({\rm Set}_{x}(R))$は
\begin{equation}
\label{gsubstsm5}
  \exists z(\forall x(x \in z \leftrightarrow (T_{1}|y_{1}, T_{2}|y_{2}, \cdots, T_{n}|y_{n})(R)))
\end{equation}
と一致する.
ここで$z$が$R, T_{1}, T_{2}, \cdots, T_{n}$のいずれの記号列の中にも
自由変数として現れないことから, 変数法則 \ref{valgsubst}により, 
$z$は$(T_{1}|y_{1}, T_{2}|y_{2}, \cdots, T_{n}|y_{n})(R)$の中に自由変数として現れない.
このことと$z$が$x$と異なることから, 定義より(\ref{gsubstsm5})は
${\rm Set}_{x}((T_{1}|y_{1}, T_{2}|y_{2}, \cdots, T_{n}|y_{n})(R))$と同じである.
故に本法則が成り立つ.
\halmos




\mathstrut
\begin{subs}
\label{substsmtrans}%代入30%確認済
$R$を記号列とし, $x$と$y$を文字とする.
$y$が$R$の中に自由変数として現れなければ, 
\[
  {\rm Set}_{x}(R) \equiv {\rm Set}_{y}((y|x)(R))
\]
が成り立つ.
\end{subs}


\noindent{\bf 証明}
~$y$が$x$と同じ文字ならば, 代入法則 \ref{substsame}によって本法則が成り立つから, 
以下では$y$は$x$と異なる文字であるとする.
いま$z$を$x$とも$y$とも異なり, $R$の中に自由変数として現れない文字とする.
このとき変数法則 \ref{valsubst}により, $z$は$(y|x)(R)$の中に自由変数として現れない.
また定義から
\begin{equation}
\label{substsmtrans1}
  {\rm Set}_{x}(R) \equiv \exists z(\forall x(x \in z \leftrightarrow R))
\end{equation}
である.
また$y$が$x$, $z$と異なり, $R$の中に自由変数として現れないことから, 
変数法則 \ref{valequiv}により, $y$は$x \in z \leftrightarrow R$の中に自由変数として現れない.
故に代入法則 \ref{substquantrans}により
\begin{equation}
\label{substsmtrans2}
  \forall x(x \in z \leftrightarrow R) \equiv \forall y((y|x)(x \in z \leftrightarrow R))
\end{equation}
が成り立つ.
また$x$が$z$と異なることと代入法則 \ref{substequiv}により, 
\begin{equation}
\label{substsmtrans3}
  (y|x)(x \in z \leftrightarrow R) \equiv y \in z \leftrightarrow (y|x)(R)
\end{equation}
が成り立つ.
以上の(\ref{substsmtrans1}), (\ref{substsmtrans2}), (\ref{substsmtrans3})からわかるように, 
${\rm Set}_{x}(R)$は
\begin{equation}
\label{substsmtrans4}
  \exists z(\forall y(y \in z \leftrightarrow (y|x)(R)))
\end{equation}
と同じである.
ここで$z$が$y$と異なり, 上述のように$(y|x)(R)$の中に自由変数として現れないことから, 
定義より(\ref{substsmtrans4})は${\rm Set}_{y}((y|x)(R))$と同じである.
故に本法則が成り立つ.
\halmos




\mathstrut
\begin{subs}
\label{substsm}%代入31%確認済
$R$と$T$を記号列とし, $x$と$y$を異なる文字とする.
$x$が$T$の中に自由変数として現れなければ, 
\[
  (T|y)({\rm Set}_{x}(R)) \equiv {\rm Set}_{x}((T|y)(R))
\]
が成り立つ.
\end{subs}


\noindent{\bf 証明}
~一般代入法則 \ref{gsubstsm}において, $n$が$1$の場合である.
\halmos




\mathstrut
\begin{form}
\label{formsm}%構成39%確認済
$R$が関係式, $x$が文字ならば, ${\rm Set}_{x}(R)$は関係式である.
\end{form}


\noindent{\bf 証明}
~$y$を$x$と異なり, $R$の中に自由変数として現れない文字とするとき, 
定義から${\rm Set}_{x}(R)$は$\exists y(\forall x(x \in y \leftrightarrow R))$である.
これが関係式であることは構成法則 \ref{formfund}, \ref{formequiv}, \ref{formquan}から直ちにわかる.
\halmos




\mathstrut%確認済
$\mathscr{T}$を特殊記号として$\in$を持つ理論とし, $R$を$\mathscr{T}$の関係式, $x$を文字とする.
このとき上記の構成法則 \ref{formsm}によれば, ${\rm Set}_{x}(R)$は$\mathscr{T}$の関係式である.
これが$\mathscr{T}$の定理となるとき, 
$R$は ($\mathscr{T}$において) \textbf{${\bm x}$について集合を作り得る}という.

さて${\rm Set}_{x}(R)$の定義と定理 \ref{sthmsm!}から直ちに次の定理を得る.




\mathstrut
\begin{thm}
\label{sthmsmex!}%定理3.2%確認済
$R$を関係式とし, $x$を文字とする.
また$y$を$x$と異なり, $R$の中に自由変数として現れない文字とする.
このとき
\begin{equation}
\label{sthmsmex!1}
  {\rm Set}_{x}(R) \leftrightarrow \exists !y(\forall x(x \in y \leftrightarrow R))
\end{equation}
が成り立つ.
\end{thm}


\noindent{\bf 証明}
~このとき${\rm Set}_{x}(R)$は$\exists y(\forall x(x \in y \leftrightarrow R))$と同じだから, 
定理 \ref{sthmsm!}と推論法則 \ref{dedeqch}, \ref{dedawblatrue2}により
(\ref{sthmsmex!1})が成り立つことがわかる.
\halmos




\mathstrut%確認済
次の定理も${\rm Set}_{x}(R)$の定義から直ちに得られる.




\mathstrut
\begin{thm}
\label{sthmsmbasis}%定理3.3%新規%確認済
$a$を集合, $R$を関係式とし, $x$を$a$の中に自由変数として現れない文字とする.
このとき
\[
  \forall x(x \in a \leftrightarrow R) \to {\rm Set}_{x}(R)
\]
が成り立つ.
またこのことから, 次の1), 2)が成り立つ.

1)
$\forall x(x \in a \leftrightarrow R)$ならば, $R$は$x$について集合を作り得る.

2)
$x$が定数でなく, $x \in a \leftrightarrow R$が成り立てば, $R$は$x$について集合を作り得る.
\end{thm}


\noindent{\bf 証明}
~$y$を$x$と異なり, $R$の中に自由変数として現れない文字とする.
このとき${\rm Set}_{x}(R)$は$\exists y(\forall x(x \in y \leftrightarrow R))$と同じだから, 
schema S4の適用により
\[
  (a|y)(\forall x(x \in y \leftrightarrow R)) \to {\rm Set}_{x}(R)
\]
が成り立つ.
ここで$x$が$y$と異なり, $a$の中に自由変数として現れないことから, 
代入法則 \ref{substquan}によりこの記号列は
\[
  \forall x((a|y)(x \in y \leftrightarrow R)) \to {\rm Set}_{x}(R)
\]
と一致する.
また$y$が$x$と異なり, $R$の中に自由変数として現れないことから, 
代入法則 \ref{substfree}, \ref{substequiv}によりこの記号列は
\[
  \forall x(x \in a \leftrightarrow R) \to {\rm Set}_{x}(R)
\]
と一致する.
故にこれが成り立つ.
1)が成り立つことはこれと推論法則 \ref{dedmp}によって明らかである.
また2)が成り立つことは1)と推論法則 \ref{dedltthmquan}によって明らかである.
\halmos




\mathstrut%確認済%koko
集合論の明示的公理A2, A3及びschema S7は, 特定の関係式が集合を作り得ることを保証するものである.
それらについては次節以降で論じる.

集合を作り得る関係式とそうでない関係式の例を一つずつ挙げておく.




\mathstrut
{\small
\noindent
{\bf 例 1.}~%例3.1%確認済
$a$を集合とし, $x$を$a$の中に自由変数として現れない文字とする.
このとき関係式$x \in a$は$x$について集合を作り得る.

実際これは$\forall x(x \in a \leftrightarrow x \in a)$が成り立つ (Thm \ref{allx1rlr1}) ことと
定理 \ref{sthmsmbasis}から直ちにわかる. ------
}




\mathstrut
{\small
\noindent
{\bf 例 2.}~%例3.2%確認済
$x$を文字とするとき, 関係式$x \notin x$は$x$について集合を作り得ない.
即ち, 
\[
  \neg {\rm Set}_{x}(x \notin x)
\]
が成り立つ.

実際$y$を$x$と異なる定数でない文字とすれば, $y$は$x \notin x$の中に自由変数として現れないから, 
${\rm Set}_{x}(x \notin x)$は$\exists y(\forall x(x \in y \leftrightarrow x \notin x))$である.
そこでThm \ref{thmquangdm}より
\begin{equation}
\label{ex3.2.1}
  \neg {\rm Set}_{x}(x \notin x) 
  \leftrightarrow \forall y(\exists x(\neg (x \in y \leftrightarrow x \notin x)))
\end{equation}
が成り立つ.
またThm \ref{awblbwa}より
\begin{equation}
\label{ex3.2.2}
  y \in y \wedge y \notin y \leftrightarrow y \notin y \wedge y \in y
\end{equation}
が成り立つ.
またThm \ref{nal1atna1}より
\[
  y \notin y \leftrightarrow (y \in y \to y \notin y), ~~
  y \in y \leftrightarrow (y \notin y \to y \in y)
\]
が共に成り立つから, 推論法則 \ref{dedaddeqw}により
\begin{equation}
\label{ex3.2.3}
  y \notin y \wedge y \in y \leftrightarrow (y \in y \leftrightarrow y \notin y)
\end{equation}
が成り立つ.
そこで(\ref{ex3.2.2}), (\ref{ex3.2.3})から, 推論法則 \ref{dedeqtrans}によって
\[
  y \in y \wedge y \notin y \leftrightarrow (y \in y \leftrightarrow y \notin y)
\]
が成り立つ.
これとThm \ref{n1awna1}より$\neg (y \in y \wedge y \notin y)$が成り立つことから, 
推論法則 \ref{dedeqfund}により
\[
  \neg (y \in y \leftrightarrow y \notin y)
\]
が成り立つ.
ここで$y$が$x$と異なることと代入法則 \ref{substfund}, \ref{substequiv}により, この記号列が
\[
  (y|x)(\neg (x \in y \leftrightarrow x \notin x))
\]
と一致することがわかる.
故にこれが成り立つ.
そこで推論法則 \ref{deds4}により
\[
  \exists x(\neg (x \in y \leftrightarrow x \notin x))
\]
が成り立つ.
このことと$y$が定数でないことから, 推論法則 \ref{dedltthmquan}により
\[
  \forall y(\exists x(\neg (x \in y \leftrightarrow x \notin x)))
\]
が成り立つ.
従ってこれと(\ref{ex3.2.1})から, 推論法則 \ref{dedeqfund}により
$\neg {\rm Set}_{x}(x \notin x)$が成り立つ. ------
}




\mathstrut%確認済
次の定理 \ref{sthmalleqsmlem}は, 定理 \ref{sthmalleqsm}を示すための補題である.
後で定理 \ref{sthmalleqiset=}を示すときにもこれを用いる.




\mathstrut
\begin{thm}
\label{sthmalleqsmlem}%定理3.4%新規%確認済
$R$と$S$を関係式とし, $x$を文字とする.
また$y$を$x$と異なり, $R$及び$S$の中に自由変数として現れない, 定数でない文字とする.
このとき
\begin{equation}
\label{sthmalleqsmlem1}
  \forall x(R \leftrightarrow S) 
  \to \forall y(\forall x(x \in y \leftrightarrow R) \leftrightarrow \forall x(x \in y \leftrightarrow S))
\end{equation}
が成り立つ.
\end{thm}


\noindent{\bf 証明}
~$\tau_{x}(\neg ((x \in y \leftrightarrow R) \leftrightarrow (x \in y \leftrightarrow S)))$を
$T$と書けば, $T$は対象式であり, Thm \ref{thmallfund2}より
\[
  \forall x(R \leftrightarrow S) \to (T|x)(R \leftrightarrow S)
\]
が成り立つ.
ここで代入法則 \ref{substequiv}によればこの記号列は
\begin{equation}
\label{sthmalleqsmlem2}
  \forall x(R \leftrightarrow S) \to ((T|x)(R) \leftrightarrow (T|x)(S))
\end{equation}
と一致するから, これが成り立つ.
またThm \ref{1alb1t11alc1l1blc11}より
\begin{equation}
\label{sthmalleqsmlem3}
  ((T|x)(R) \leftrightarrow (T|x)(S)) 
  \to ((T \in y \leftrightarrow (T|x)(R)) \leftrightarrow (T \in y \leftrightarrow (T|x)(S)))
\end{equation}
が成り立つ.
また$T$の定義から, Thm \ref{thmallfund1}と推論法則 \ref{dedequiv}により
\[
  (T|x)((x \in y \leftrightarrow R) \leftrightarrow (x \in y \leftrightarrow S)) 
  \to \forall x((x \in y \leftrightarrow R) \leftrightarrow (x \in y \leftrightarrow S))
\]
が成り立つが, $x$が$y$と異なることと代入法則 \ref{substequiv}によればこの記号列は
\begin{equation}
\label{sthmalleqsmlem4}
  ((T \in y \leftrightarrow (T|x)(R)) \leftrightarrow (T \in y \leftrightarrow (T|x)(S))) 
  \to \forall x((x \in y \leftrightarrow R) \leftrightarrow (x \in y \leftrightarrow S))
\end{equation}
と一致するから, これが成り立つ.
またThm \ref{thmalleqallsep}より
\begin{equation}
\label{sthmalleqsmlem5}
  \forall x((x \in y \leftrightarrow R) \leftrightarrow (x \in y \leftrightarrow S)) 
  \to (\forall x(x \in y \leftrightarrow R) \leftrightarrow \forall x(x \in y \leftrightarrow S))
\end{equation}
が成り立つ.
そこで(\ref{sthmalleqsmlem2})---(\ref{sthmalleqsmlem5})から, 推論法則 \ref{dedmmp}によって
\begin{equation}
\label{sthmalleqsmlem6}
  \forall x(R \leftrightarrow S) 
  \to (\forall x(x \in y \leftrightarrow R) \leftrightarrow \forall x(x \in y \leftrightarrow S))
\end{equation}
が成り立つことがわかる.
ここで$y$が$R$と$S$の中に自由変数として現れないことから, 変数法則 \ref{valequiv}, \ref{valquan}により, 
$y$は$\forall x(R \leftrightarrow S)$の中に自由変数として現れない.
また$y$は定数でない.
これらのことと(\ref{sthmalleqsmlem6})が成り立つことから, 
推論法則 \ref{dedalltquansepfreeconst}により(\ref{sthmalleqsmlem1})が成り立つ.
\halmos




\mathstrut
\begin{thm}
\label{sthmalleqsm}%定理3.5%1)と2)は新規%確認済
$R$と$S$を関係式とし, $x$を文字とする.
このとき
\begin{equation}
\label{sthmalleqsm1}
  \forall x(R \leftrightarrow S) \to ({\rm Set}_{x}(R) \leftrightarrow {\rm Set}_{x}(S))
\end{equation}
が成り立つ.
またこのことから, 次の1)---4)が成り立つ.

1)
$\forall x(R \leftrightarrow S)$ならば, 
${\rm Set}_{x}(R) \leftrightarrow {\rm Set}_{x}(S)$.

2)
$x$が定数でなく, $R \leftrightarrow S$が成り立てば, 
${\rm Set}_{x}(R) \leftrightarrow {\rm Set}_{x}(S)$.

3)
$\forall x(R \leftrightarrow S)$であるとする.
このとき$R$, $S$のうちの一方が$x$について集合を作り得るならば, 他方も$x$について集合を作り得る.

4)
$x$が定数でなく, $R \leftrightarrow S$が成り立つとする.
このとき$R$, $S$のうちの一方が$x$について集合を作り得るならば, 他方も$x$について集合を作り得る.
\end{thm}


\noindent{\bf 証明}
~$y$を$x$と異なり, $R$及び$S$の中に自由変数として現れない, 定数でない文字とする.
このとき${\rm Set}_{x}(R)$, ${\rm Set}_{x}(S)$はそれぞれ
$\exists y(\forall x(x \in y \leftrightarrow R))$, $\exists y(\forall x(x \in y \leftrightarrow S))$と同じである.
また定理 \ref{sthmalleqsmlem}より
\begin{equation}
\label{sthmalleqsm2}
  \forall x(R \leftrightarrow S) 
  \to \forall y(\forall x(x \in y \leftrightarrow R) \leftrightarrow \forall x(x \in y \leftrightarrow S))
\end{equation}
が成り立つ.
またThm \ref{thmalleqexsep}より
\[
  \forall y(\forall x(x \in y \leftrightarrow R) \leftrightarrow \forall x(x \in y \leftrightarrow S)) 
  \to (\exists y(\forall x(x \in y \leftrightarrow R)) \leftrightarrow \exists y(\forall x(x \in y \leftrightarrow S)))
\]
が成り立つが, 上述のことよりこの記号列は
\begin{equation}
\label{sthmalleqsm3}
  \forall y(\forall x(x \in y \leftrightarrow R) \leftrightarrow \forall x(x \in y \leftrightarrow S)) 
  \to ({\rm Set}_{x}(R) \leftrightarrow {\rm Set}_{x}(S))
\end{equation}
と同じだから, これが成り立つ.
そこで(\ref{sthmalleqsm2}), (\ref{sthmalleqsm3})から, 
推論法則 \ref{dedmmp}によって(\ref{sthmalleqsm1})が成り立つ.

\noindent
1)
(\ref{sthmalleqsm1})と推論法則 \ref{dedmp}によって明らか.

\noindent
2)
1)と推論法則 \ref{dedltthmquan}によって明らか.

\noindent
3)
1)と推論法則 \ref{dedeqfund}によって明らか.

\noindent
4)
3)と推論法則 \ref{dedltthmquan}によって明らか.
\halmos




\mathstrut%確認済%koko
次に$\{x \mid R\}$という記号列の定義を与える.
そのために次の法則を証明する.




\mathstrut
\begin{defo}
\label{intension}%変形13%確認済
$\mathscr{T}$を特殊記号として$\in$を持つ理論とし, $R$を$\mathscr{T}$の記号列, $x$を文字とする.
また$y$と$z$を共に$x$と異なり, $R$の中に自由変数として現れない文字とする.
このとき
\[
  \tau_{y}(\forall x(x \in y \leftrightarrow R)) 
  \equiv \tau_{z}(\forall x(x \in z \leftrightarrow R))
\]
が成り立つ.
\end{defo}


\noindent{\bf 証明}
~$y$と$z$が同じ文字ならば明らかだから, 以下$y$と$z$は異なる文字であるとする.
このとき$z$が$x$, $y$と異なり, $R$の中に自由変数として現れないことから, 
変数法則 \ref{valequiv}, \ref{valquan}により, 
$z$は$\forall x(x \in y \leftrightarrow R)$の中に自由変数として現れない.
故に代入法則 \ref{substtautrans}により
\[
  \tau_{y}(\forall x(x \in y \leftrightarrow R)) 
  \equiv \tau_{z}((z|y)(\forall x(x \in y \leftrightarrow R)))
\]
が成り立つ.
また$x$が$y$, $z$と異なり, $y$が$R$の中に自由変数として現れないことから, 
代入法則 \ref{substfree}, \ref{substequiv}, \ref{substquan}によってわかるように
\[
  (z|y)(\forall x(x \in y \leftrightarrow R)) 
  \equiv \forall x(x \in z \leftrightarrow R)
\]
が成り立つ.
故に本法則が成り立つ.
\halmos




\mathstrut
\begin{defi}
\label{defiset}%定義2%確認済
$\mathscr{T}$を特殊記号として$\in$を持つ理論とし, $R$を$\mathscr{T}$の記号列, $x$を文字とする.
また$y$と$z$を共に$x$と異なり, $R$の中に自由変数として現れない文字とする.
このとき変形法則 \ref{intension}によれば, $\tau_{y}(\forall x(x \in y \leftrightarrow R))$と
$\tau_{z}(\forall x(x \in z \leftrightarrow R))$は同じ記号列となる.
$R$と$x$に対して定まるこの記号列を, $\{x \mid R\}$と書き表す.
\end{defi}




\mathstrut%確認済
以下の変数法則 \ref{valiset}, 一般代入法則 \ref{gsubstiset}, 代入法則 \ref{substisettrans}, \ref{substiset}, 
構成法則 \ref{formiset}では, $\mathscr{T}$を特殊記号として$\in$を持つ理論とし, 
これらの法則における``記号列'', ``関係式'', ``集合''とは, 
それぞれ$\mathscr{T}$の記号列, $\mathscr{T}$の関係式, $\mathscr{T}$の対象式のこととする.




\mathstrut
\begin{valu}
\label{valiset}%変数23%確認済
$R$を記号列とし, $x$を文字とする.

1)
$x$は$\{x \mid R\}$の中に自由変数として現れない.

2)
$y$を文字とする.
$y$が$R$の中に自由変数として現れなければ, 
$y$は$\{x \mid R\}$の中に自由変数として現れない.
\end{valu}


\noindent{\bf 証明}
~1)
$z$を$x$と異なり, $R$の中に自由変数として現れない文字とするとき, 
定義から$\{x \mid R\}$は$\tau_{z}(\forall x(x \in z \leftrightarrow R))$と同じである.
変数法則 \ref{valtau}, \ref{valquan}によれば, $x$はこの中に自由変数として現れない.

\noindent
2)
$y$が$x$と同じ文字ならば1)により明らか.
$y$が$x$と異なる文字ならば, このことと$y$が$R$の中に自由変数として現れないことから, 
定義より$\{x \mid R\}$は$\tau_{y}(\forall x(x \in y \leftrightarrow R))$と同じである.
変数法則 \ref{valtau}によれば, $y$はこの中に自由変数として現れない.
\halmos




\mathstrut
\begin{gsub}
\label{gsubstiset}%一般代入28%確認済
$R$を記号列とし, $x$を文字とする.
また$n$を自然数とし, $T_{1}, T_{2}, \cdots, T_{n}$を記号列とする.
また$y_{1}, y_{2}, \cdots, y_{n}$を, どの二つも互いに異なる文字とする.
$x$が$y_{1}, y_{2}, \cdots, y_{n}$のいずれとも異なり, かつ
$T_{1}, T_{2}, \cdots, T_{n}$のいずれの記号列の中にも自由変数として現れなければ, 
\[
  (T_{1}|y_{1}, T_{2}|y_{2}, \cdots, T_{n}|y_{n})(\{x \mid R\}) 
  \equiv \{x \mid (T_{1}|y_{1}, T_{2}|y_{2}, \cdots, T_{n}|y_{n})(R)\}
\]
が成り立つ.
\end{gsub}


\noindent{\bf 証明}
~$z$を$x, y_{1}, y_{2}, \cdots, y_{n}$のいずれとも異なり, 
$R, T_{1}, T_{2}, \cdots, T_{n}$のいずれの記号列の中にも自由変数として現れない文字とする.
このとき定義から$\{x \mid R\}$は$\tau_{z}(\forall x(x \in z \leftrightarrow R))$だから, 
\begin{equation}
\label{gsubstiset1}
  (T_{1}|y_{1}, T_{2}|y_{2}, \cdots, T_{n}|y_{n})(\{x \mid R\}) 
  \equiv (T_{1}|y_{1}, T_{2}|y_{2}, \cdots, T_{n}|y_{n})(\tau_{z}(\forall x(x \in z \leftrightarrow R)))
\end{equation}
である.
また$z$が$y_{1}, y_{2}, \cdots, y_{n}$のいずれとも異なり, かつ
$T_{1}, T_{2}, \cdots, T_{n}$のいずれの記号列の中にも自由変数として現れないことから, 
一般代入法則 \ref{gsubsttau}により
\begin{equation}
\label{gsubstiset2}
  (T_{1}|y_{1}, T_{2}|y_{2}, \cdots, T_{n}|y_{n})(\tau_{z}(\forall x(x \in z \leftrightarrow R))) 
  \equiv \tau_{z}((T_{1}|y_{1}, T_{2}|y_{2}, \cdots, T_{n}|y_{n})(\forall x(x \in z \leftrightarrow R)))
\end{equation}
が成り立つ.
また$x$も$y_{1}, y_{2}, \cdots, y_{n}$のいずれとも異なり, かつ
$T_{1}, T_{2}, \cdots, T_{n}$のいずれの記号列の中にも自由変数として現れないから, 
一般代入法則 \ref{gsubstquan}により
\begin{equation}
\label{gsubstiset3}
  (T_{1}|y_{1}, T_{2}|y_{2}, \cdots, T_{n}|y_{n})(\forall x(x \in z \leftrightarrow R)) 
  \equiv \forall x((T_{1}|y_{1}, T_{2}|y_{2}, \cdots, T_{n}|y_{n})(x \in z \leftrightarrow R))
\end{equation}
が成り立つ.
また$x$と$z$が共に$y_{1}, y_{2}, \cdots, y_{n}$のいずれとも異なることと
一般代入法則 \ref{gsubstequiv}から, 
\begin{equation}
\label{gsubstiset4}
  (T_{1}|y_{1}, T_{2}|y_{2}, \cdots, T_{n}|y_{n})(x \in z \leftrightarrow R) 
  \equiv x \in z \leftrightarrow (T_{1}|y_{1}, T_{2}|y_{2}, \cdots, T_{n}|y_{n})(R)
\end{equation}
が成り立つ.
以上の(\ref{gsubstiset1})---(\ref{gsubstiset4})からわかるように, 
$(T_{1}|y_{1}, T_{2}|y_{2}, \cdots, T_{n}|y_{n})(\{x \mid R\})$は
\begin{equation}
\label{gsubstiset5}
  \tau_{z}(\forall x(x \in z \leftrightarrow (T_{1}|y_{1}, T_{2}|y_{2}, \cdots, T_{n}|y_{n})(R)))
\end{equation}
と一致する.
ここで$z$が$R, T_{1}, T_{2}, \cdots, T_{n}$のいずれの記号列の中にも
自由変数として現れないことから, 変数法則 \ref{valgsubst}により, 
$z$は$(T_{1}|y_{1}, T_{2}|y_{2}, \cdots, T_{n}|y_{n})(R)$の中に自由変数として現れない.
このことと$z$が$x$と異なることから, 定義より(\ref{gsubstiset5})は
$\{x \mid (T_{1}|y_{1}, T_{2}|y_{2}, \cdots, T_{n}|y_{n})(R)\}$と同じである.
故に本法則が成り立つ.
\halmos




\mathstrut
\begin{subs}
\label{substisettrans}%代入32%確認済
$R$を記号列とし, $x$と$y$を文字とする.
$y$が$R$の中に自由変数として現れなければ, 
\[
  \{x \mid R\} \equiv \{y \mid (y|x)(R)\}
\]
が成り立つ.
\end{subs}


\noindent{\bf 証明}
~$y$が$x$と同じ文字ならば, 代入法則 \ref{substsame}によって本法則が成り立つから, 
以下では$y$は$x$と異なる文字であるとする.
いま$z$を$x$とも$y$とも異なり, $R$の中に自由変数として現れない文字とする.
このとき変数法則 \ref{valsubst}により, $z$は$(y|x)(R)$の中に自由変数として現れない.
また定義から
\begin{equation}
\label{substisettrans1}
  \{x \mid R\} \equiv \tau_{z}(\forall x(x \in z \leftrightarrow R))
\end{equation}
である.
また$y$が$x$, $z$と異なり, $R$の中に自由変数として現れないことから, 
変数法則 \ref{valequiv}により, $y$は$x \in z \leftrightarrow R$の中に自由変数として現れない.
故に代入法則 \ref{substquantrans}により
\begin{equation}
\label{substisettrans2}
  \forall x(x \in z \leftrightarrow R) \equiv \forall y((y|x)(x \in z \leftrightarrow R))
\end{equation}
が成り立つ.
また$x$が$z$と異なることと代入法則 \ref{substequiv}により, 
\begin{equation}
\label{substisettrans3}
  (y|x)(x \in z \leftrightarrow R) \equiv y \in z \leftrightarrow (y|x)(R)
\end{equation}
が成り立つ.
以上の(\ref{substisettrans1}), (\ref{substisettrans2}), (\ref{substisettrans3})からわかるように, 
$\{x \mid R\}$は
\begin{equation}
\label{substisettrans4}
  \tau_{z}(\forall y(y \in z \leftrightarrow (y|x)(R)))
\end{equation}
と同じである.
ここで$z$が$y$と異なり, 上述のように$(y|x)(R)$の中に自由変数として現れないことから, 
定義より(\ref{substisettrans4})は$\{y \mid (y|x)(R)\}$と同じである.
故に本法則が成り立つ.
\halmos




\mathstrut
\begin{subs}
\label{substiset}%代入33%確認済
$R$と$T$を記号列とし, $x$と$y$を異なる文字とする.
$x$が$T$の中に自由変数として現れなければ, 
\[
  (T|y)(\{x \mid R\}) \equiv \{x \mid (T|y)(R)\}
\]
が成り立つ.
\end{subs}


\noindent{\bf 証明}
~一般代入法則 \ref{gsubstiset}において, $n$が$1$の場合である.
\halmos




\mathstrut
\begin{form}
\label{formiset}%構成40%確認済
$R$が関係式, $x$が文字ならば, $\{x \mid R\}$は集合である.
\end{form}


\noindent{\bf 証明}
~$y$を$x$と異なり, $R$の中に自由変数として現れない文字とするとき, 
定義から$\{x \mid R\}$は$\tau_{y}(\forall x(x \in y \leftrightarrow R))$である.
これが集合であることは構成法則 \ref{formfund}, \ref{formequiv}, \ref{formquan}から直ちにわかる.
\halmos




\mathstrut%註%確認済
{\small
\noindent
\textbf{註.} 
$R$を記号列, $x$を文字とするとき, 
\begin{equation}
\label{sm&iset1}
  {\rm Set}_{x}(R) \equiv \forall x(x \in \{x \mid R\} \leftrightarrow R)
\end{equation}
が成り立つ.

実際$y$を$x$と異なり, $R$の中に自由変数として現れない文字とするとき, 
$\{x \mid R\}$は$\tau_{y}(\forall x(x \in y \leftrightarrow R))$と同じであり, 
${\rm Set}_{x}(R)$は$\exists y(\forall x(x \in y \leftrightarrow R))$, 即ち
$(\tau_{y}(\forall x(x \in y \leftrightarrow R))|y)(\forall x(x \in y \leftrightarrow R))$と
同じである.
そこで
\begin{equation}
\label{sm&iset2}
  {\rm Set}_{x}(R) \equiv (\{x \mid R\}|y)(\forall x(x \in y \leftrightarrow R))
\end{equation}
が成り立つ.
また$x$は$y$と異なり, 変数法則 \ref{valiset}により$\{x \mid R\}$の中に
自由変数として現れないから, 代入法則 \ref{substquan}により
\begin{equation}
\label{sm&iset3}
  (\{x \mid R\}|y)(\forall x(x \in y \leftrightarrow R)) 
  \equiv \forall x((\{x \mid R\}|y)(x \in y \leftrightarrow R))
\end{equation}
が成り立つ.
また$y$が$x$と異なり, $R$の中に自由変数として現れないことから, 
代入法則 \ref{substfree}, \ref{substequiv}により
\begin{equation}
\label{sm&iset4}
  (\{x \mid R\}|y)(x \in y \leftrightarrow R) \equiv x \in \{x \mid R\} \leftrightarrow R
\end{equation}
が成り立つ.
そこで(\ref{sm&iset2})---(\ref{sm&iset4})から, (\ref{sm&iset1})が成り立つことがわかる.

(\ref{sm&iset1})は以後特に断りなく引用する.
}




\mathstrut
さてこの註から直ちに次の定理を得る.




\mathstrut
\begin{thm}
\label{sthmisetbasis}%定理3.6%sthmsetintelementから変更%確認済
$R$を関係式, $T$を対象式とし, $x$を文字とする.
このとき
\begin{equation}
\label{sthmisetbasis1}
  {\rm Set}_{x}(R) \to (T \in \{x \mid R\} \leftrightarrow (T|x)(R))
\end{equation}
が成り立つ.
またこのことから, 次の1), 2)が成り立つ.

1)
$R$が$x$について集合を作り得るならば, $T \in \{x \mid R\} \leftrightarrow (T|x)(R)$.

2)
$R$が$x$について集合を作り得るとする.
このとき$T \in \{x \mid R\}$ならば, $(T|x)(R)$である.
またこのとき$(T|x)(R)$ならば, $T \in \{x \mid R\}$である.
\end{thm}


\noindent{\bf 証明}
~このとき${\rm Set}_{x}(R)$は$\forall x(x \in \{x \mid R\} \leftrightarrow R)$と同じである.
故にThm \ref{thmallfund2}より
\[
  {\rm Set}_{x}(R) \to (T|x)(x \in \{x \mid R\} \leftrightarrow R)
\]
が成り立つが, 変数法則 \ref{valiset}によれば$x$は$\{x \mid R\}$の中に自由変数として現れないから, 
代入法則 \ref{substfree}, \ref{substfund}, \ref{substequiv}によれば
この記号列は(\ref{sthmisetbasis1})と一致する.
故に(\ref{sthmisetbasis1})が成り立つ.

\noindent
1)
(\ref{sthmisetbasis1})と推論法則 \ref{dedmp}によって明らか.

\noindent
2)
1)と推論法則 \ref{dedeqfund}によって明らか.
\halmos




\mathstrut
\begin{thm}
\label{sthma=iset}%定理3.7%新規%確認済
$a$を集合, $R$を関係式とし, $x$を$a$の中に自由変数として現れない文字とする.
このとき
\[
  \forall x(x \in a \leftrightarrow R) \to a = \{x \mid R\}
\]
が成り立つ.
またこのことから, 次の1), 2)が成り立つ.

1)
$\forall x(x \in a \leftrightarrow R)$ならば, $a = \{x \mid R\}$.

2)
$x$が定数でなく, $x \in a \leftrightarrow R$が成り立てば, $a = \{x \mid R\}$.
\end{thm}


\noindent{\bf 証明}
~$y$を$x$と異なり, $R$の中に自由変数として現れない文字とする.
このとき定理 \ref{sthmsm!}より$!y(\forall x(x \in y \leftrightarrow R))$が成り立つから, 
推論法則 \ref{ded!tTtau}により
\[
  (a|y)(\forall x(x \in y \leftrightarrow R)) \to a = \tau_{y}(\forall x(x \in y \leftrightarrow R))
\]
が成り立つ.
ここで$y$が$x$と異なり, $R$の中に自由変数として現れないことから, 
この記号列は
\[
  (a|y)(\forall x(x \in y \leftrightarrow R)) \to a = \{x \mid R\}
\]
と同じである.
また$x$が$y$と異なり, $a$の中に自由変数として現れないことから, 
代入法則 \ref{substquan}によりこの記号列は
\[
  \forall x((a|y)(x \in y \leftrightarrow R)) \to a = \{x \mid R\}
\]
と一致する.
また$y$が$x$と異なり, $R$の中に自由変数として現れないことから, 
代入法則 \ref{substfree}, \ref{substequiv}によりこの記号列は
\[
  \forall x(x \in a \leftrightarrow R) \to a = \{x \mid R\}
\]
と一致する.
故にこれが成り立つ.
1)が成り立つことはこれと推論法則 \ref{dedmp}によって明らかである.
また2)が成り立つことは1)と推論法則 \ref{dedltthmquan}によって明らかである.
\halmos




\mathstrut
{\small
\noindent
{\bf 例 3.}~%例3.3%確認済
$a$を集合とし, $x$を$a$の中に自由変数として現れない文字とする.
このとき$\{x \mid x \in a\} = a$が成り立つ.

実際Thm \ref{allx1rlr1}より$\forall x(x \in a \leftrightarrow x \in a)$が成り立つから, 
このことと$x$が$a$の中に自由変数として現れないことから, 
定理 \ref{sthma=iset}より$a = \{x \mid x \in a\}$が成り立つ.
故に推論法則 \ref{ded=ch}により$\{x \mid x \in a\} = a$が成り立つ. ------
}




\mathstrut
\begin{thm}
\label{sthmsmbasis&iset=a}%定理3.8%新規%確認済
$a$を集合, $R$を関係式とし, $x$を$a$の中に自由変数として現れない文字とする.
このとき
\begin{equation}
\label{sthmsmbasis&iset=a1}
  \forall x(x \in a \leftrightarrow R) \leftrightarrow {\rm Set}_{x}(R) \wedge \{x \mid R\} = a
\end{equation}
が成り立つ.
またこのことから, 次の(\ref{sthmsmbasis&iset=a2})が成り立つ.
\begin{equation}
\label{sthmsmbasis&iset=a2}
  R \text{が} x \text{について集合を作り得るとき,} ~\{x \mid R\} = a \text{ならば,} ~\forall x(x \in a \leftrightarrow R).
\end{equation}
\end{thm}


\noindent{\bf 証明}
~$x$が$a$の中に自由変数として現れないことから, 定理 \ref{sthmsmbasis}より
\begin{equation}
\label{sthmsmbasis&iset=a3}
  \forall x(x \in a \leftrightarrow R) \to {\rm Set}_{x}(R)
\end{equation}
が成り立ち, 定理 \ref{sthma=iset}より
\begin{equation}
\label{sthmsmbasis&iset=a4}
  \forall x(x \in a \leftrightarrow R) \to a = \{x \mid R\}
\end{equation}
が成り立つ.
またThm \ref{x=yty=x}より
\begin{equation}
\label{sthmsmbasis&iset=a5}
  a = \{x \mid R\} \to \{x \mid R\} = a
\end{equation}
が成り立つ.
そこで(\ref{sthmsmbasis&iset=a4}), (\ref{sthmsmbasis&iset=a5})から, 推論法則 \ref{dedmmp}によって
\begin{equation}
\label{sthmsmbasis&iset=a6}
  \forall x(x \in a \leftrightarrow R) \to \{x \mid R\} = a
\end{equation}
が成り立つ.
故に(\ref{sthmsmbasis&iset=a3}), (\ref{sthmsmbasis&iset=a6})から, 推論法則 \ref{dedprewedge}により
\begin{equation}
\label{sthmsmbasis&iset=a7}
  \forall x(x \in a \leftrightarrow R) \to {\rm Set}_{x}(R) \wedge \{x \mid R\} = a
\end{equation}
が成り立つ.
また$y$を$x$と異なり, $R$の中に自由変数として現れない文字とするとき, schema S5の適用により
\begin{equation}
\label{sthmsmbasis&iset=a8}
  \{x \mid R\} = a 
  \to ((\{x \mid R\}|y)(\forall x(x \in y \leftrightarrow R)) \to (a|y)(\forall x(x \in y \leftrightarrow R)))
\end{equation}
が成り立つ.
ここで$y$が$x$と異なり, $R$の中に自由変数として現れないことから, 
${\rm Set}_{x}(R)$は$\exists y(\forall x(x \in y \leftrightarrow R))$, 
即ち$(\tau_{y}(\forall x(x \in y \leftrightarrow R))|y)(\forall x(x \in y \leftrightarrow R))$と一致し, 
$\{x \mid R\}$は$\tau_{y}(\forall x(x \in y \leftrightarrow R))$と一致するから, 
$(\{x \mid R\}|y)(\forall x(x \in y \leftrightarrow R))$は${\rm Set}_{x}(R)$と一致する.
また$x$と$y$が互いに異なり, $x$, $y$がそれぞれ$a$, $R$の中に自由変数として現れないことから, 
代入法則 \ref{substfree}, \ref{substequiv}, \ref{substquan}によってわかるように, 
$(a|y)(\forall x(x \in y \leftrightarrow R))$は$\forall x(x \in a \leftrightarrow R)$と一致する.
従って(\ref{sthmsmbasis&iset=a8})は
\[
  \{x \mid R\} = a \to ({\rm Set}_{x}(R) \to \forall x(x \in a \leftrightarrow R))
\]
と一致する.
故にこれが成り立つ.
そこで推論法則 \ref{dedch}, \ref{dedtwch}により
\begin{equation}
\label{sthmsmbasis&iset=a9}
  {\rm Set}_{x}(R) \wedge \{x \mid R\} = a \to \forall x(x \in a \leftrightarrow R)
\end{equation}
が成り立つことがわかる.
(\ref{sthmsmbasis&iset=a7}), (\ref{sthmsmbasis&iset=a9})から, 
推論法則 \ref{dedequiv}により(\ref{sthmsmbasis&iset=a1})が成り立つ.
(\ref{sthmsmbasis&iset=a2})が成り立つことは, 
(\ref{sthmsmbasis&iset=a1})と推論法則 \ref{dedwedge}, \ref{dedeqfund}によって明らかである.
\halmos




\mathstrut
\begin{thm}
\label{sthmsmtiset&asubset}%定理3.9%新規%確認済
$a$を集合, $R$を関係式とし, $x$を$a$の中に自由変数として現れない文字とする.
このとき
\begin{align}
  \label{sthmsmtiset&asubset1}
  &{\rm Set}_{x}(R) \to (\forall x(x \in a \to R) \leftrightarrow a \subset \{x \mid R\}), \\
  \mbox{} \notag \\
  \label{sthmsmtiset&asubset2}
  &{\rm Set}_{x}(R) \to (\forall x(R \to x \in a) \leftrightarrow \{x \mid R\} \subset a)
\end{align}
が共に成り立つ.
またこれらから, 次の1)が成り立つ.

1)
$R$が$x$について集合を作り得るならば, 
\[
  \forall x(x \in a \to R) \leftrightarrow a \subset \{x \mid R\}, ~~
  \forall x(R \to x \in a) \leftrightarrow \{x \mid R\} \subset a
\]
が共に成り立つ.

更に, $R$が$x$について集合を作り得るとき, 次の2)---5)が成り立つ.

2)
$\forall x(x \in a \to R)$ならば, $a \subset \{x \mid R\}$.
また$a \subset \{x \mid R\}$ならば, $\forall x(x \in a \to R)$.

3)
$x$が定数でなく, $x \in a \to R$が成り立てば, $a \subset \{x \mid R\}$.

4)
$\forall x(R \to x \in a)$ならば, $\{x \mid R\} \subset a$.
また$\{x \mid R\} \subset a$ならば, $\forall x(R \to x \in a)$.

5)
$x$が定数でなく, $R \to x \in a$が成り立てば, $\{x \mid R\} \subset a$.
\end{thm}


\noindent{\bf 証明}
~$y$を$x$と異なり, $a$及び$R$の中に自由変数として現れない, 定数でない文字とする.
このとき変数法則 \ref{valsm}により, $y$は${\rm Set}_{x}(R)$の中に自由変数として現れない.
また変数法則 \ref{valfund}により, 
$y$は$x \in a \to R$及び$R \to x \in a$の中に自由変数として現れない.
また変数法則 \ref{valiset}により, $y$は$\{x \mid R\}$の中に自由変数として現れない.
また定理 \ref{sthmisetbasis}より
\begin{equation}
\label{sthmsmtiset&asubset3}
  {\rm Set}_{x}(R) \to (y \in \{x \mid R\} \leftrightarrow (y|x)(R))
\end{equation}
が成り立つ.
またThm \ref{1alb1t1bla1}より
\begin{equation}
\label{sthmsmtiset&asubset4}
  (y \in \{x \mid R\} \leftrightarrow (y|x)(R)) \to ((y|x)(R) \leftrightarrow y \in \{x \mid R\})
\end{equation}
が成り立つ.
またThm \ref{1alb1t11atc1l1btc11}より
\begin{align*}
  &((y|x)(R) \leftrightarrow y \in \{x \mid R\}) 
  \to ((y \in a \to (y|x)(R)) \leftrightarrow (y \in a \to y \in \{x \mid R\})), \\
  \mbox{} \notag \\
  &((y|x)(R) \leftrightarrow y \in \{x \mid R\}) 
  \to (((y|x)(R) \to y \in a) \leftrightarrow (y \in \{x \mid R\} \to y \in a))
\end{align*}
が共に成り立つ.
ここで$x$が$a$の中に自由変数として現れないことと代入法則 \ref{substfree}, \ref{substfund}から, 
これらの記号列はそれぞれ
\begin{align}
  \label{sthmsmtiset&asubset5}
  &((y|x)(R) \leftrightarrow y \in \{x \mid R\}) 
  \to ((y|x)(x \in a \to R) \leftrightarrow (y \in a \to y \in \{x \mid R\})), \\
  \mbox{} \notag \\
  \label{sthmsmtiset&asubset6}
  &((y|x)(R) \leftrightarrow y \in \{x \mid R\}) 
  \to ((y|x)(R \to x \in a) \leftrightarrow (y \in \{x \mid R\} \to y \in a))
\end{align}
と一致する.
故にこれらが共に成り立つ.
そこで(\ref{sthmsmtiset&asubset3}), (\ref{sthmsmtiset&asubset4}), (\ref{sthmsmtiset&asubset5})から, 
推論法則 \ref{dedmmp}によって
\begin{equation}
\label{sthmsmtiset&asubset7}
  {\rm Set}_{x}(R) \to ((y|x)(x \in a \to R) \leftrightarrow (y \in a \to y \in \{x \mid R\}))
\end{equation}
が成り立つことがわかる.
また(\ref{sthmsmtiset&asubset3}), (\ref{sthmsmtiset&asubset4}), (\ref{sthmsmtiset&asubset6})から, 
同じく推論法則 \ref{dedmmp}によって
\begin{equation}
\label{sthmsmtiset&asubset8}
  {\rm Set}_{x}(R) \to ((y|x)(R \to x \in a) \leftrightarrow (y \in \{x \mid R\} \to y \in a))
\end{equation}
が成り立つことがわかる.
ここで$y$は定数でなく, 上述のように${\rm Set}_{x}(R)$の中に自由変数として現れないから, 
このことと(\ref{sthmsmtiset&asubset7}), (\ref{sthmsmtiset&asubset8})から, 
推論法則 \ref{dedalltquansepfreeconst}により
\begin{align}
  \label{sthmsmtiset&asubset9}
  &{\rm Set}_{x}(R) 
  \to \forall y((y|x)(x \in a \to R) \leftrightarrow (y \in a \to y \in \{x \mid R\})), \\
  \mbox{} \notag \\
  \label{sthmsmtiset&asubset10}
  &{\rm Set}_{x}(R) 
  \to \forall y((y|x)(R \to x \in a) \leftrightarrow (y \in \{x \mid R\} \to y \in a))
\end{align}
が共に成り立つ.
またThm \ref{thmalleqallsep}より
\begin{multline*}
  \forall y((y|x)(x \in a \to R) \leftrightarrow (y \in a \to y \in \{x \mid R\})) \\
  \to (\forall y((y|x)(x \in a \to R)) \leftrightarrow \forall y(y \in a \to y \in \{x \mid R\})), 
\end{multline*}
\begin{multline*}
  \forall y((y|x)(R \to x \in a) \leftrightarrow (y \in \{x \mid R\} \to y \in a)) \\
  \to (\forall y((y|x)(R \to x \in a)) \leftrightarrow \forall y(y \in \{x \mid R\} \to y \in a))
\end{multline*}
が共に成り立つ.
ここで上述のように$y$は$x \in a \to R$, $R \to x \in a$, $a$, $\{x \mid R\}$の
いずれの記号列の中にも自由変数として現れないから, 
定義と代入法則 \ref{substquantrans}によればこれらの記号列はそれぞれ
\begin{align}
  \label{sthmsmtiset&asubset11}
  &\forall y((y|x)(x \in a \to R) \leftrightarrow (y \in a \to y \in \{x \mid R\})) 
  \to (\forall x(x \in a \to R) \leftrightarrow a \subset \{x \mid R\}), \\
  \mbox{} \notag \\
  \label{sthmsmtiset&asubset12}
  &\forall y((y|x)(R \to x \in a) \leftrightarrow (y \in \{x \mid R\} \to y \in a)) 
  \to (\forall x(R \to x \in a) \leftrightarrow \{x \mid R\} \subset a)
\end{align}
と一致する.
従ってこれらが共に成り立つ.
そこで(\ref{sthmsmtiset&asubset9})と(\ref{sthmsmtiset&asubset11}), 
(\ref{sthmsmtiset&asubset10})と(\ref{sthmsmtiset&asubset12})から, 
それぞれ推論法則 \ref{dedmmp}によって
(\ref{sthmsmtiset&asubset1}), (\ref{sthmsmtiset&asubset2})が成り立つ.

\noindent
1)
(\ref{sthmsmtiset&asubset1}), (\ref{sthmsmtiset&asubset2})と推論法則 \ref{dedmp}によって明らか.

\noindent
2), 4)
1)と推論法則 \ref{dedeqfund}によって明らか.

\noindent
3)
2)と推論法則 \ref{dedltthmquan}によって明らか.

\noindent
5)
4)と推論法則 \ref{dedltthmquan}によって明らか.
\halmos




\mathstrut
\begin{thm}
\label{sthmspiniset}%定理3.10%新規%確認済
$A$と$R$を関係式とし, $x$を文字とする.
このとき
\begin{align*}
  &{\rm Set}_{x}(A) \to ((\exists x \in \{x \mid A\})(R) \leftrightarrow \exists_{A}x(R)), ~~
  {\rm Set}_{x}(A) \to ((\forall x \in \{x \mid A\})(R) \leftrightarrow \forall_{A}x(R)), \\
  \mbox{} \\
  &{\rm Set}_{x}(A) \to ((!x \in \{x \mid A\})(R) \leftrightarrow \ !_{A}x(R)), ~~
  {\rm Set}_{x}(A) \to ((\exists !x \in \{x \mid A\})(R) \leftrightarrow \exists !_{A}x(R))
\end{align*}
がすべて成り立つ.
またこれらから, 次の(\ref{sthmspiniset1})が成り立つ.
\begin{align}
\label{sthmspiniset1}
  &A \text{が} x \text{について集合を作り得るならば,} \\
  &(\exists x \in \{x \mid A\})(R) \leftrightarrow \exists_{A}x(R), ~
  (\forall x \in \{x \mid A\})(R) \leftrightarrow \forall_{A}x(R), \notag \\
  &(!x \in \{x \mid A\})(R) \leftrightarrow \ !_{A}x(R), ~
  (\exists !x \in \{x \mid A\})(R) \leftrightarrow \exists !_{A}x(R) \text{がすべて成り立つ.} \notag
\end{align}
\end{thm}


\noindent{\bf 証明}
~このとき${\rm Set}_{x}(A)$は$\forall x(x \in \{x \mid A\} \leftrightarrow A)$と同じである.
故に前半はThm \ref{thmallpreeqspquansep}, \ref{thmallpreeqsp!sep}, \ref{thmallpreeqspex!sep}より明らか.
(\ref{sthmspiniset1})は前半と推論法則 \ref{dedmp}から直ちに得られる.
\halmos




\mathstrut
\begin{thm}
\label{sthmalleqiset=}%定理3.11%新規%確認済
$R$と$S$を関係式とし, $x$を文字とする.
このとき
\begin{equation}
\label{sthmalleqiset=1}
  \forall x(R \leftrightarrow S) \to \{x \mid R\} = \{x \mid S\}
\end{equation}
が成り立つ.
またこのことから, 次の1), 2)が成り立つ.

1)
$\forall x(R \leftrightarrow S)$ならば, $\{x \mid R\} = \{x \mid S\}$.

2)
$x$が定数でなく, $R \leftrightarrow S$が成り立てば, $\{x \mid R\} = \{x \mid S\}$.
\end{thm}


\noindent{\bf 証明}
~$y$を$x$と異なり, $R$及び$S$の中に自由変数として現れない, 定数でない文字とする.
このとき$\{x \mid R\}$, $\{x \mid S\}$はそれぞれ$\tau_{y}(\forall x(x \in y \leftrightarrow R))$, 
$\tau_{y}(\forall x(x \in y \leftrightarrow S))$と同じである.
また定理 \ref{sthmalleqsmlem}より
\begin{equation}
\label{sthmalleqiset=2}
  \forall x(R \leftrightarrow S) 
  \to \forall y(\forall x(x \in y \leftrightarrow R) \leftrightarrow \forall x(x \in y \leftrightarrow S))
\end{equation}
が成り立つ.
またschema S6の適用により
\[
  \forall y(\forall x(x \in y \leftrightarrow R) \leftrightarrow \forall x(x \in y \leftrightarrow S)) 
  \to \tau_{y}(\forall x(x \in y \leftrightarrow R)) = \tau_{y}(\forall x(x \in y \leftrightarrow S))
\]
が成り立つが, 上述のことよりこの記号列は
\begin{equation}
\label{sthmalleqiset=3}
  \forall y(\forall x(x \in y \leftrightarrow R) \leftrightarrow \forall x(x \in y \leftrightarrow S)) 
  \to \{x \mid R\} = \{x \mid S\}
\end{equation}
と一致するから, これが成り立つ.
そこで(\ref{sthmalleqiset=2}), (\ref{sthmalleqiset=3})から, 
推論法則 \ref{dedmmp}によって(\ref{sthmalleqiset=1})が成り立つ.

\noindent
1)
(\ref{sthmalleqiset=1})と推論法則 \ref{dedmp}によって明らか.

\noindent
2)
1)と推論法則 \ref{dedltthmquan}によって明らか.
\halmos




\mathstrut
\begin{thm}
\label{sthmsmtalltisetsubseteq}%定理3.12%新規%確認済
$R$と$S$を関係式とし, $x$を文字とする.
このとき
\begin{equation}
\label{sthmsmtalltisetsubseteq1}
  {\rm Set}_{x}(R) \wedge {\rm Set}_{x}(S) 
  \to (\forall x(R \to S) \leftrightarrow \{x \mid R\} \subset \{x \mid S\})
\end{equation}
が成り立つ.
またこのことから, 次の1), 2), 3)が成り立つ.

1)
$R$と$S$が共に$x$について集合を作り得るならば, 
$\forall x(R \to S) \leftrightarrow \{x \mid R\} \subset \{x \mid S\}$.

2)
$R$と$S$が共に$x$について集合を作り得るとする.
このとき$\forall x(R \to S)$ならば, $\{x \mid R\} \subset \{x \mid S\}$.
またこのとき$\{x \mid R\} \subset \{x \mid S\}$ならば, $\forall x(R \to S)$.

3)
$x$は定数でないとする.
また$R$と$S$は共に$x$について集合を作り得るとする.
このとき$R \to S$ならば, $\{x \mid R\} \subset \{x \mid S\}$.
\end{thm}


\noindent{\bf 証明}
~$y$を$R$と$S$の中に自由変数として現れない, 定数でない文字とする.
このとき定理 \ref{sthmisetbasis}より
\[
  {\rm Set}_{x}(R) \to (y \in \{x \mid R\} \leftrightarrow (y|x)(R)), ~~
  {\rm Set}_{x}(S) \to (y \in \{x \mid S\} \leftrightarrow (y|x)(S))
\]
が共に成り立つから, 推論法則 \ref{dedfromaddw}により
\begin{equation}
\label{sthmsmtalltisetsubseteq2}
  {\rm Set}_{x}(R) \wedge {\rm Set}_{x}(S) 
  \to (y \in \{x \mid R\} \leftrightarrow (y|x)(R)) \wedge (y \in \{x \mid S\} \leftrightarrow (y|x)(S))
\end{equation}
が成り立つ.
またThm \ref{1alb1w1cld1t11atc1l1btd11}より
\begin{multline}
\label{sthmsmtalltisetsubseteq3}
  (y \in \{x \mid R\} \leftrightarrow (y|x)(R)) \wedge (y \in \{x \mid S\} \leftrightarrow (y|x)(S)) \\
  \to ((y \in \{x \mid R\} \to y \in \{x \mid S\}) \leftrightarrow ((y|x)(R) \to (y|x)(S)))
\end{multline}
が成り立つ.
そこで(\ref{sthmsmtalltisetsubseteq2}), (\ref{sthmsmtalltisetsubseteq3})から, 
推論法則 \ref{dedmmp}によって
\begin{equation}
\label{sthmsmtalltisetsubseteq4}
  {\rm Set}_{x}(R) \wedge {\rm Set}_{x}(S) 
  \to ((y \in \{x \mid R\} \to y \in \{x \mid S\}) \leftrightarrow ((y|x)(R) \to (y|x)(S)))
\end{equation}
が成り立つ.
ここで$y$が$R$と$S$の中に自由変数として現れないことから, 
変数法則 \ref{valwedge}, \ref{valsm}により, 
$y$は${\rm Set}_{x}(R) \wedge {\rm Set}_{x}(S)$の中に自由変数として現れない.
また$y$は定数でない.
そこでこれらのことと(\ref{sthmsmtalltisetsubseteq4})が成り立つことから, 
推論法則 \ref{dedalltquansepfreeconst}により
\begin{equation}
\label{sthmsmtalltisetsubseteq5}
  {\rm Set}_{x}(R) \wedge {\rm Set}_{x}(S) 
  \to \forall y((y \in \{x \mid R\} \to y \in \{x \mid S\}) \leftrightarrow ((y|x)(R) \to (y|x)(S)))
\end{equation}
が成り立つ.
またThm \ref{thmalleqallsep}より
\begin{multline}
\label{sthmsmtalltisetsubseteq6}
  \forall y((y \in \{x \mid R\} \to y \in \{x \mid S\}) \leftrightarrow ((y|x)(R) \to (y|x)(S))) \\
  \to (\forall y(y \in \{x \mid R\} \to y \in \{x \mid S\}) \leftrightarrow \forall y((y|x)(R) \to (y|x)(S)))
\end{multline}
が成り立つ.
ここで$y$が$R$と$S$の中に自由変数として現れないことから, 
変数法則 \ref{valiset}により$y$は$\{x \mid R\}$と$\{x \mid S\}$の中に自由変数として現れないから, 
$\forall y(y \in \{x \mid R\} \to y \in \{x \mid S\})$は$\{x \mid R\} \subset \{x \mid S\}$と同じである.
また変数法則 \ref{valfund}により$y$は$R \to S$の中に自由変数として現れないから, 
代入法則 \ref{substfund}, \ref{substquantrans}によってわかるように, 
$\forall y((y|x)(R) \to (y|x)(S))$は$\forall x(R \to S)$と同じである.
従って(\ref{sthmsmtalltisetsubseteq6})は
\begin{equation}
\label{sthmsmtalltisetsubseteq7}
  \forall y((y \in \{x \mid R\} \to y \in \{x \mid S\}) \leftrightarrow ((y|x)(R) \to (y|x)(S))) 
  \to (\{x \mid R\} \subset \{x \mid S\} \leftrightarrow \forall x(R \to S))
\end{equation}
と同じである.
故にこれが成り立つ.
またThm \ref{1alb1t1bla1}より
\begin{equation}
\label{sthmsmtalltisetsubseteq8}
  (\{x \mid R\} \subset \{x \mid S\} \leftrightarrow \forall x(R \to S)) 
  \to (\forall x(R \to S) \leftrightarrow \{x \mid R\} \subset \{x \mid S\})
\end{equation}
が成り立つ.
そこで(\ref{sthmsmtalltisetsubseteq5}), (\ref{sthmsmtalltisetsubseteq7}), 
(\ref{sthmsmtalltisetsubseteq8})から, 
推論法則 \ref{dedmmp}によって(\ref{sthmsmtalltisetsubseteq1})が成り立つことがわかる.

\noindent
1)
(\ref{sthmsmtalltisetsubseteq1})と推論法則 \ref{dedmp}, \ref{dedwedge}によって明らか.

\noindent
2)
1)と推論法則 \ref{dedeqfund}によって明らか.

\noindent
3)
2)と推論法則 \ref{dedltthmquan}によって明らか.
\halmos




\mathstrut
\begin{thm}
\label{sthmsmtalleqiset=eq}%定理3.13%新規%確認済
$R$と$S$を関係式とし, $x$を文字とする.
このとき
\begin{equation}
\label{sthmsmtalleqiset=eq1}
  {\rm Set}_{x}(R) \wedge {\rm Set}_{x}(S) 
  \to (\forall x(R \leftrightarrow S) \leftrightarrow \{x \mid R\} = \{x \mid S\})
\end{equation}
が成り立つ.
またこのことから, 次の1), 2)が成り立つ.

1)
$R$と$S$が共に$x$について集合を作り得るならば, 
$\forall x(R \leftrightarrow S) \leftrightarrow \{x \mid R\} = \{x \mid S\}$.

2)
$R$と$S$が共に$x$について集合を作り得るとする.
このとき$\{x \mid R\} = \{x \mid S\}$ならば, $\forall x(R \leftrightarrow S)$.
\end{thm}


\noindent{\bf 証明}
~Thm \ref{awbtbwa}より
\begin{equation}
\label{sthmsmtalleqiset=eq2}
  {\rm Set}_{x}(R) \wedge {\rm Set}_{x}(S) \to {\rm Set}_{x}(S) \wedge {\rm Set}_{x}(R)
\end{equation}
が成り立つ.
また定理 \ref{sthmsmtalltisetsubseteq}より
\begin{align}
  \label{sthmsmtalleqiset=eq3}
  &{\rm Set}_{x}(R) \wedge {\rm Set}_{x}(S) 
  \to (\forall x(R \to S) \leftrightarrow \{x \mid R\} \subset \{x \mid S\}), \\
  \mbox{} \notag \\
  \label{sthmsmtalleqiset=eq4}
  &{\rm Set}_{x}(S) \wedge {\rm Set}_{x}(R) 
  \to (\forall x(S \to R) \leftrightarrow \{x \mid S\} \subset \{x \mid R\})
\end{align}
が共に成り立つ.
そこで(\ref{sthmsmtalleqiset=eq2}), (\ref{sthmsmtalleqiset=eq4})から, 推論法則 \ref{dedmmp}によって
\[
  {\rm Set}_{x}(R) \wedge {\rm Set}_{x}(S) 
  \to (\forall x(S \to R) \leftrightarrow \{x \mid S\} \subset \{x \mid R\})
\]
が成り立つ.
故にこれと(\ref{sthmsmtalleqiset=eq3})から, 推論法則 \ref{dedprewedge}により
\begin{equation}
\label{sthmsmtalleqiset=eq5}
  {\rm Set}_{x}(R) \wedge {\rm Set}_{x}(S) 
  \to (\forall x(R \to S) \leftrightarrow \{x \mid R\} \subset \{x \mid S\}) \wedge (\forall x(S \to R) \leftrightarrow \{x \mid S\} \subset \{x \mid R\})
\end{equation}
が成り立つ.
またThm \ref{1alb1w1cld1t1awclbwd1}より
\begin{multline}
\label{sthmsmtalleqiset=eq6}
  (\forall x(R \to S) \leftrightarrow \{x \mid R\} \subset \{x \mid S\}) \wedge (\forall x(S \to R) \leftrightarrow \{x \mid S\} \subset \{x \mid R\}) \\
  \to (\forall x(R \to S) \wedge \forall x(S \to R) \leftrightarrow \{x \mid R\} \subset \{x \mid S\} \wedge \{x \mid S\} \subset \{x \mid R\})
\end{multline}
が成り立つ.
またThm \ref{thmallw}と推論法則 \ref{dedeqch}により
\begin{equation}
\label{sthmsmtalleqiset=eq7}
  \forall x(R \to S) \wedge \forall x(S \to R) \leftrightarrow \forall x(R \leftrightarrow S)
\end{equation}
が成り立つ.
また定理 \ref{sthmaxiom1}より
\begin{equation}
\label{sthmsmtalleqiset=eq8}
  \{x \mid R\} \subset \{x \mid S\} \wedge \{x \mid S\} \subset \{x \mid R\} 
  \leftrightarrow \{x \mid R\} = \{x \mid S\}
\end{equation}
が成り立つ.
そこで(\ref{sthmsmtalleqiset=eq7}), (\ref{sthmsmtalleqiset=eq8})から, 推論法則 \ref{dedaddeqeq}により
\begin{multline*}
  (\forall x(R \to S) \wedge \forall x(S \to R) \leftrightarrow \{x \mid R\} \subset \{x \mid S\} \wedge \{x \mid S\} \subset \{x \mid R\}) \\
  \leftrightarrow (\forall x(R \leftrightarrow S) \leftrightarrow \{x \mid R\} = \{x \mid S\})
\end{multline*}
が成り立つ.
故に推論法則 \ref{dedequiv}により
\begin{multline}
\label{sthmsmtalleqiset=eq9}
  (\forall x(R \to S) \wedge \forall x(S \to R) \leftrightarrow \{x \mid R\} \subset \{x \mid S\} \wedge \{x \mid S\} \subset \{x \mid R\}) \\
  \to (\forall x(R \leftrightarrow S) \leftrightarrow \{x \mid R\} = \{x \mid S\})
\end{multline}
が成り立つ.
そこで(\ref{sthmsmtalleqiset=eq5}), (\ref{sthmsmtalleqiset=eq6}), (\ref{sthmsmtalleqiset=eq9})から, 
推論法則 \ref{dedmmp}によって(\ref{sthmsmtalleqiset=eq1})が成り立つことがわかる.

\noindent
1)
(\ref{sthmsmtalleqiset=eq1})と推論法則 \ref{dedmp}, \ref{dedwedge}によって明らか.

\noindent
2)
1)と推論法則 \ref{dedeqfund}によって明らか.
\halmos
%ここまで確認



\newpage
\setcounter{defi}{0}
\section{非順序対}



%確認済%koko
この節では, 集合論の明示的公理の一つである対公理を導入する.
これは二つの集合$a$と$b$のみから成る集合$\{a, b\}$が存在することを
保証する公理である.
この公理によって, 唯一つの集合$a$のみから成る集合$\{a\}$の存在も
保証される.

まず対公理を導入するための準備を行う.
以下の変形法則 \ref{axiom2}, 変数法則 \ref{valaxiom2}, 構成法則 \ref{formaxiom2}では, 
$\mathscr{T}$を特殊記号として$=$と$\in$を持つ理論とし, 
これらの法則における``記号列'', ``関係式''とは, 
それぞれ$\mathscr{T}$の記号列, $\mathscr{T}$の関係式のこととする.




\mathstrut
\begin{defo}
\label{axiom2}%変形14%確認済
$x$, $y$, $z$を, どの二つも互いに異なる文字とする.
同じく$u$, $v$, $w$も, どの二つも互いに異なる文字とする.
このとき
\[
  \forall x(\forall y({\rm Set}_{z}(z = x \vee z = y))) 
  \equiv \forall u(\forall v({\rm Set}_{w}(w = u \vee w = v)))
\]
が成り立つ.
\end{defo}


\noindent{\bf 証明}
~$p$, $q$, $r$を, どの二つも互いに異なる文字とし, これらのうちのどの一つも
$x$, $y$, $z$, $u$, $v$, $w$のいずれとも異なるとする.
このとき変数法則 \ref{valfund}, \ref{valquan}, \ref{valsm}からわかるように, 
$p$は$\forall y({\rm Set}_{z}(z = x \vee z = y))$の中に自由変数として現れないから, 
代入法則 \ref{substquantrans}により
\begin{equation}
\label{axiom21}
  \forall x(\forall y({\rm Set}_{z}(z = x \vee z = y))) 
  \equiv \forall p((p|x)(\forall y({\rm Set}_{z}(z = x \vee z = y))))
\end{equation}
が成り立つ.
また$y$が$x$, $p$と異なることから, 代入法則 \ref{substquan}により
\begin{equation}
\label{axiom22}
  (p|x)(\forall y({\rm Set}_{z}(z = x \vee z = y))) 
  \equiv \forall y((p|x)({\rm Set}_{z}(z = x \vee z = y)))
\end{equation}
が成り立つ.
また$z$が$x$, $p$と異なることから, 代入法則 \ref{substsm}により
\begin{equation}
\label{axiom23}
  (p|x)({\rm Set}_{z}(z = x \vee z = y)) \equiv {\rm Set}_{z}((p|x)(z = x \vee z = y))
\end{equation}
が成り立つ.
また$x$が$y$, $z$と異なることと代入法則 \ref{substfund}により, 
\begin{equation}
\label{axiom24}
  (p|x)(z = x \vee z = y) \equiv z = p \vee z = y
\end{equation}
が成り立つ.
そこで(\ref{axiom21})---(\ref{axiom24})から, 
\begin{equation}
\label{axiom25}
  \forall x(\forall y({\rm Set}_{z}(z = x \vee z = y))) 
  \equiv \forall p(\forall y({\rm Set}_{z}(z = p \vee z = y)))
\end{equation}
が成り立つことがわかる.
また$q$が$y$, $z$, $p$と異なることから, 変数法則 \ref{valfund}, \ref{valsm}によってわかるように
$q$は${\rm Set}_{z}(z = p \vee z = y)$の中に自由変数として現れないから, 
代入法則 \ref{substquantrans}により
\begin{equation}
\label{axiom26}
  \forall y({\rm Set}_{z}(z = p \vee z = y)) \equiv \forall q((q|y)({\rm Set}_{z}(z = p \vee z = y)))
\end{equation}
が成り立つ.
また$z$が$y$, $q$と異なることから, 代入法則 \ref{substsm}により
\begin{equation}
\label{axiom27}
  (q|y)({\rm Set}_{z}(z = p \vee z = y)) \equiv {\rm Set}_{z}((q|y)(z = p \vee z = y))
\end{equation}
が成り立つ.
また$y$が$z$, $p$と異なることと代入法則 \ref{substfund}により, 
\begin{equation}
\label{axiom28}
  (q|y)(z = p \vee z = y) \equiv z = p \vee z = q
\end{equation}
が成り立つ.
そこで(\ref{axiom26})---(\ref{axiom28})から, 
\begin{equation}
\label{axiom29}
  \forall y({\rm Set}_{z}(z = p \vee z = y)) \equiv \forall q({\rm Set}_{z}(z = p \vee z = q))
\end{equation}
が成り立つことがわかる.
また$r$が$z$, $p$, $q$と異なることから, 変数法則 \ref{valfund}により
$r$は$z = p \vee z = q$の中に自由変数として現れないから, 代入法則 \ref{substsmtrans}により
\begin{equation}
\label{axiom210}
  {\rm Set}_{z}(z = p \vee z = q) \equiv {\rm Set}_{r}((r|z)(z = p \vee z = q))
\end{equation}
が成り立つ.
また$z$が$p$, $q$と異なることと代入法則 \ref{substfund}により, 
\begin{equation}
\label{axiom211}
  (r|z)(z = p \vee z = q) \equiv r = p \vee r = q
\end{equation}
が成り立つ.
そこで(\ref{axiom210}), (\ref{axiom211})から, 
\begin{equation}
\label{axiom212}
  {\rm Set}_{z}(z = p \vee z = q) \equiv {\rm Set}_{r}(r = p \vee r = q)
\end{equation}
が成り立つ.
故に(\ref{axiom25}), (\ref{axiom29}), (\ref{axiom212})から, 
\[
  \forall x(\forall y({\rm Set}_{z}(z = x \vee z = y))) 
  \equiv \forall p(\forall q({\rm Set}_{r}(r = p \vee r = q)))
\]
が成り立つことがわかる.
以上の議論と全く同様にして, 
\[
  \forall u(\forall v({\rm Set}_{w}(w = u \vee w = v))) 
  \equiv \forall p(\forall q({\rm Set}_{r}(r = p \vee r = q)))
\]
も成り立つ.
従って, $\forall x(\forall y({\rm Set}_{z}(z = x \vee z = y)))$と
$\forall u(\forall v({\rm Set}_{w}(w = u \vee w = v)))$は同一の記号列である.
\halmos




\mathstrut
\begin{valu}
\label{valaxiom2}%変数24%確認済
$x$, $y$, $z$を文字とするとき, 
$\forall x(\forall y({\rm Set}_{z}(z = x \vee z = y)))$は自由変数を持たない.
\end{valu}


\noindent{\bf 証明}
~変数法則 \ref{valquan}によってわかるように, 
$x$と$y$は共に$\forall x(\forall y({\rm Set}_{z}(z = x \vee z = y)))$の中に自由変数として現れない.
また変数法則 \ref{valquan}, \ref{valsm}によってわかるように, 
$z$は$\forall x(\forall y({\rm Set}_{z}(z = x \vee z = y)))$の中に自由変数として現れない.
また$w$を$x$, $y$, $z$と異なる文字とするとき, 
変数法則 \ref{valfund}, \ref{valquan}, \ref{valsm}によってわかるように, 
$w$は$\forall x(\forall y({\rm Set}_{z}(z = x \vee z = y)))$の中に自由変数として現れない.
故に本法則が成り立つ.
\halmos




\mathstrut
\begin{form}
\label{formaxiom2}%構成41%確認済
$x$, $y$, $z$を文字とするとき, 
$\forall x(\forall y({\rm Set}_{z}(z = x \vee z = y)))$は関係式である.
\end{form}


\noindent{\bf 証明}
~構成法則 \ref{formfund}, \ref{formquan}, \ref{formsm}によって明らか.
\halmos




\mathstrut
\begin{defi}
\label{defaxiom2}%定義1%確認済
$x$, $y$, $z$をどの二つも互いに異なる文字とするとき, 次の記号列A2は集合論の明示的公理である: 
\begin{center}
  A2. ~~$\forall x(\forall y({\rm Set}_{z}(z = x \vee z = y)))$
\end{center}
これを\textbf{対公理} (axiom of unordered pair) という.
A2は構成法則 \ref{formaxiom2}により確かに関係式である.
また変数法則 \ref{valaxiom2}により, A2は自由変数を持たない.
また変形法則 \ref{axiom2}により, A2は仮定を満たす文字$x$, $y$, $z$の取り方に依らずに定まる記号列である.
即ち$u$, $v$, $w$をどの二つも互いに異なる文字とするとき, 
A2は$\forall u(\forall v({\rm Set}_{w}(w = u \vee w = v)))$と一致する.
\end{defi}




\mathstrut%確認済
以後特に断らない限り, 上記のA2は定理であるとする.




\mathstrut
\begin{thm}
\label{sthmaxiom2}%定理4.1%確認済
$a$と$b$を集合とし, $x$をこれらの中に自由変数として現れない文字とする.
このとき関係式$x = a \vee x = b$は$x$について集合を作り得る.
\end{thm}


\noindent{\bf 証明}
~$y$と$z$を互いに異なり, 共に$x$と異なる文字とする.
このとき対公理A2より
\[
  \forall y(\forall z({\rm Set}_{x}(x = y \vee x = z)))
\]
が成り立つから, 推論法則 \ref{dedgallfund2}により
\[
  (a|y, b|z)({\rm Set}_{x}(x = y \vee x = z))
\]
が成り立つ.
ここで$x$が$y$, $z$と異なり, $a$, $b$の中に自由変数として現れないことから, 
一般代入法則 \ref{gsubstfund}, \ref{gsubstsm}によってわかるように, この記号列は
\[
  {\rm Set}_{x}(x = a \vee x = b)
\]
と一致する.
故にこれが成り立つ.
\halmos




\mathstrut%確認済%koko
次に記号列$\{a, b\}$の定義を与える.
後の便宜のため, より一般的な形の記号列を定義しておく.




\mathstrut
\begin{defo}
\label{nset}%変形15%新規%確認済
$\mathscr{T}$を特殊記号として$=$と$\in$を持つ理論とする.
また$n$を自然数とし, $a_{1}, a_{2}, \cdots, a_{n}$を$\mathscr{T}$の記号列とする.
また$x$と$y$を共に$a_{1}, a_{2}, \cdots, a_{n}$のいずれの記号列の中にも
自由変数として現れない文字とする.
このとき
\[
  \{x \mid x = a_{1} \vee x = a_{2} \vee \cdots \vee x = a_{n}\} 
  \equiv \{y \mid y = a_{1} \vee y = a_{2} \vee \cdots \vee y = a_{n}\}
\]
が成り立つ.
\end{defo}


\noindent{\bf 証明}
~$x$と$y$が同じ文字ならば明らかだから, 以下$x$と$y$は異なる文字であるとする.
このとき$y$が$x$と異なり, $a_{1}, a_{2}, \cdots, a_{n}$の中に自由変数として現れないことから, 
変数法則 \ref{valfund}, \ref{valgvee}により, 
$y$は$x = a_{1} \vee x = a_{2} \vee \cdots \vee x = a_{n}$の中に自由変数として現れない.
故に代入法則 \ref{substisettrans}により
\[
  \{x \mid x = a_{1} \vee x = a_{2} \vee \cdots \vee x = a_{n}\} 
  \equiv \{y \mid (y|x)(x = a_{1} \vee x = a_{2} \vee \cdots \vee x = a_{n})\}
\]
が成り立つ.
また$x$が$a_{1}, a_{2}, \cdots, a_{n}$の中に自由変数として現れないことから, 
代入法則 \ref{substfree}, \ref{substfund}, \ref{substgvee}により
\[
  (y|x)(x = a_{1} \vee x = a_{2} \vee \cdots \vee x = a_{n}) 
  \equiv y = a_{1} \vee y = a_{2} \vee \cdots \vee y = a_{n}
\]
が成り立つ.
故に本法則が成り立つ.
\halmos




\mathstrut
\begin{defi}
\label{defnset}%定義2%新規%確認済
$\mathscr{T}$を特殊記号として$=$と$\in$を持つ理論とする.
また$n$を自然数とし, $a_{1}, a_{2}, \cdots, a_{n}$を$\mathscr{T}$の記号列とする.
また$x$と$y$を共に$a_{1}, a_{2}, \cdots, a_{n}$のいずれの記号列の中にも
自由変数として現れない文字とする.
このとき変形法則 \ref{nset}によれば, 
$\{x \mid x = a_{1} \vee x = a_{2} \vee \cdots \vee x = a_{n}\}$と
$\{y \mid y = a_{1} \vee y = a_{2} \vee \cdots \vee y = a_{n}\}$は同じ記号列となる.
$a_{1}, a_{2}, \cdots, a_{n}$に対して定まるこの記号列を, $\{a_{1}, a_{2}, \cdots, a_{n}\}$と書き表す.
\end{defi}




\mathstrut%確認済
以下の変数法則 \ref{valnset}, 一般代入法則 \ref{gsubstnset}, 代入法則 \ref{substnset}, 
構成法則 \ref{formnset}では, $\mathscr{T}$を特殊記号として$=$と$\in$を持つ理論とし, 
これらの法則における``記号列'', ``集合''とは, 
それぞれ$\mathscr{T}$の記号列, $\mathscr{T}$の対象式のこととする.




\mathstrut
\begin{valu}
\label{valnset}%変数25%新規%確認済
$n$を自然数とする.
また$a_{1}, a_{2}, \cdots, a_{n}$を記号列とし, 
$x$をこれらの中に自由変数として現れない文字とする.
このとき$x$は$\{a_{1}, a_{2}, \cdots, a_{n}\}$の中に自由変数として現れない.
\end{valu}


\noindent{\bf 証明}
~このとき定義から$\{a_{1}, a_{2}, \cdots, a_{n}\}$は
$\{x \mid x = a_{1} \vee x = a_{2} \vee \cdots \vee x = a_{n}\}$と同じである.
変数法則 \ref{valiset}によれば, $x$はこの中に自由変数として現れない.
\halmos




\mathstrut
\begin{gsub}
\label{gsubstnset}%一般代入29%新規%確認済
$m$を自然数とし, $a_{1}, \cdots, a_{m}$を記号列とする.
また$n$を自然数とし, $T_{1}, \cdots, T_{n}$を記号列とする.
また$x_{1}, \cdots, x_{n}$を, どの二つも互いに異なる文字とする.
このとき
\[
  (T_{1}|x_{1}, \cdots, T_{n}|x_{n})(\{a_{1}, \cdots, a_{m}\}) 
  \equiv \{(T_{1}|x_{1}, \cdots, T_{n}|x_{n})(a_{1}), \cdots, (T_{1}|x_{1}, \cdots, T_{n}|x_{n})(a_{m})\}
\]
が成り立つ.
\end{gsub}


\noindent{\bf 証明}
~$y$を$x_{1}, \cdots, x_{n}$のいずれとも異なり, 
$a_{1}, \cdots, a_{m}$, $T_{1}, \cdots, T_{n}$のいずれの中にも自由変数として現れない文字とする.
このとき定義から$\{a_{1}, \cdots, a_{m}\}$は
$\{y \mid y = a_{1} \vee \cdots \vee y = a_{m}\}$と同じだから, 
\begin{equation}
\label{gsubstnset1}
  (T_{1}|x_{1}, \cdots, T_{n}|x_{n})(\{a_{1}, \cdots, a_{m}\}) 
  \equiv (T_{1}|x_{1}, \cdots, T_{n}|x_{n})(\{y \mid y = a_{1} \vee \cdots \vee y = a_{m}\})
\end{equation}
である.
また$y$が$x_{1}, \cdots, x_{n}$のいずれとも異なり, 
$T_{1}, \cdots, T_{n}$のいずれの中にも自由変数として現れないことから, 
一般代入法則 \ref{gsubstiset}により
\begin{equation}
\label{gsubstnset2}
  (T_{1}|x_{1}, \cdots, T_{n}|x_{n})(\{y \mid y = a_{1} \vee \cdots \vee y = a_{m}\}) 
  \equiv \{y \mid (T_{1}|x_{1}, \cdots, T_{n}|x_{n})(y = a_{1} \vee \cdots \vee y = a_{m})\}
\end{equation}
が成り立つ.
また$y$が$x_{1}, \cdots, x_{n}$のいずれとも異なることと
一般代入法則 \ref{gsubstfund}, \ref{gsubstgvee}により, 
\begin{multline}
\label{gsubstnset3}
  (T_{1}|x_{1}, \cdots, T_{n}|x_{n})(y = a_{1} \vee \cdots \vee y = a_{m}) \\
  \equiv y = (T_{1}|x_{1}, \cdots, T_{n}|x_{n})(a_{1}) \vee \cdots \vee y = (T_{1}|x_{1}, \cdots, T_{n}|x_{n})(a_{m})
\end{multline}
が成り立つ.
そこで(\ref{gsubstnset1})---(\ref{gsubstnset3})からわかるように, 
$(T_{1}|x_{1}, \cdots, T_{n}|x_{n})(\{a_{1}, \cdots, a_{m}\})$は
\begin{equation}
\label{gsubstnset4}
  \{y \mid y = (T_{1}|x_{1}, \cdots, T_{n}|x_{n})(a_{1}) \vee \cdots \vee y = (T_{1}|x_{1}, \cdots, T_{n}|x_{n})(a_{m})\}
\end{equation}
と一致する.
ここで$y$が$a_{1}, \cdots, a_{m}$, $T_{1}, \cdots, T_{n}$のいずれの中にも
自由変数として現れないことから, 変数法則 \ref{valgsubst}により, 
$y$は$(T_{1}|x_{1}, \cdots, T_{n}|x_{n})(a_{1}), \cdots, (T_{1}|x_{1}, \cdots, T_{n}|x_{n})(a_{m})$の
いずれの中にも自由変数として現れない.
故に定義から, (\ref{gsubstnset4})は
$\{(T_{1}|x_{1}, \cdots, T_{n}|x_{n})(a_{1}), \cdots, (T_{1}|x_{1}, \cdots, T_{n}|x_{n})(a_{m})\}$と同じである.
故に本法則が成り立つ.
\halmos




\mathstrut
\begin{subs}
\label{substnset}%代入34%新規%確認済
$n$を自然数とし, $a_{1}, a_{2}, \cdots, a_{n}$を記号列とする.
また$T$を記号列とし, $x$を文字とする.
このとき
\[
  (T|x)(\{a_{1}, a_{2}, \cdots, a_{n}\}) 
  \equiv \{(T|x)(a_{1}), (T|x)(a_{2}), \cdots, (T|x)(a_{n})\}
\]
が成り立つ.
\end{subs}


\noindent{\bf 証明}
~一般代入法則 \ref{gsubstnset}において, $n$が$1$の場合である.
\halmos




\mathstrut
\begin{form}
\label{formnset}%構成42%新規%確認済
$n$を自然数とし, $a_{1}, a_{2}, \cdots, a_{n}$を集合とする.
このとき$\{a_{1}, a_{2}, \cdots, a_{n}\}$は集合である.
\end{form}


\noindent{\bf 証明}
~$x$を$a_{1}, a_{2}, \cdots, a_{n}$の中に自由変数として現れない文字とするとき, 
定義から$\{a_{1}, a_{2}, \cdots, a_{n}\}$は
$\{x \mid x = a_{1} \vee x = a_{2} \vee \cdots \vee x = a_{n}\}$と同じである.
これが集合となることは, 構成法則 \ref{formfund}, \ref{formgvee}, \ref{formiset}から直ちにわかる.
\halmos




\mathstrut%確認済%koko
$n$を自然数, $a_{1}, a_{2}, \cdots, a_{n}$を集合とするとき, 
上記の構成法則 \ref{formnset}により$\{a_{1}, a_{2}, \cdots, a_{n}\}$は集合である.
これを\textbf{$\bm{a_{1}, a_{2}, \cdots, a_{n}}$のみから成る集合}などという.
特に$a$, $b$を集合とするとき, 集合$\{a, b\}$を
$a$と$b$の\textbf{非順序対} (unordered pair) ともいう.
また$a$を集合とするとき, 集合$\{a\}$を$a$の\textbf{単集合} (singleton) ともいう.

$\{a_{1}, a_{2}, \cdots, a_{n}\}$について成り立つ定理の多くは, 
その証明に次節で導入するschema S7を必要とする (例えば現時点では
$a_{1} \in \{a_{1}, a_{2}, \cdots, a_{n}\}$は示すことができない).
そこで一般の$n$に対する$\{a_{1}, a_{2}, \cdots, a_{n}\}$に関する定理は後でまとめて述べることとし, 
この節の以下の部分では非順序対及び単集合のみを考える.

さて非順序対の定義と定理 \ref{sthmaxiom2}から, 直ちに次の定理を得る.




\mathstrut
\begin{thm}
\label{sthmuopairbasis}%定理4.2%1)と2)は新規%確認済
$a$, $b$, $c$を集合とするとき, 
\begin{equation}
\label{sthmuopairbasis1}
  c \in \{a, b\} \leftrightarrow c = a \vee c = b
\end{equation}
が成り立つ.
またこのことから, 次の1), 2)が成り立つ.

1)
$c \in \{a, b\}$ならば, $c = a \vee c = b$.

2)
$c = a$ならば, $c \in \{a, b\}$.
また$c = b$ならば, $c \in \{a, b\}$.
\end{thm}


\noindent{\bf 証明}
~$x$を$a$, $b$の中に自由変数として現れない文字とする.
このとき$\{a, b\}$は$\{x \mid x = a \vee x = b\}$と同じである.
また定理 \ref{sthmaxiom2}より関係式$x = a \vee x = b$は$x$について集合を作り得る.
故に定理 \ref{sthmisetbasis}より
\[
  c \in \{a, b\} \leftrightarrow (c|x)(x = a \vee x = b)
\]
が成り立つが, $x$が$a$, $b$の中に自由変数として現れないことから, 
代入法則 \ref{substfree}, \ref{substfund}によりこの記号列は
(\ref{sthmuopairbasis1})と一致するから, これが成り立つ.

\noindent
1)
(\ref{sthmuopairbasis1})と推論法則 \ref{dedeqfund}によって明らか.

\noindent
2)
(\ref{sthmuopairbasis1})と推論法則 \ref{dedvee}, \ref{dedeqfund}によって明らか.
\halmos




\mathstrut
\begin{thm}
\label{sthmuopairfund}%定理4.3%確認済
$a$と$b$を集合とするとき, 
\[
  a \in \{a, b\}, ~~
  b \in \{a, b\}
\]
が成り立つ.
\end{thm}


\noindent{\bf 証明}
~Thm \ref{x=x}より$a = a$, $b = b$が共に成り立つから, 
定理 \ref{sthmuopairbasis}より$a \in \{a, b\}$と$b \in \{a, b\}$が共に成り立つ.
\halmos




\mathstrut
\begin{thm}
\label{sthmuopairnotin}%定理4.4%新規%確認済
$a$, $b$, $c$を集合とするとき, 
\begin{equation}
\label{sthmuopairnotin1}
  c \notin \{a, b\} \leftrightarrow c \neq a \wedge c \neq b
\end{equation}
が成り立つ.
またこのことから, 次の1), 2)が成り立つ.

1)
$c \notin \{a, b\}$ならば, $c \neq a$と$c \neq b$が共に成り立つ.

2)
$c \neq a$と$c \neq b$が共に成り立てば, $c \notin \{a, b\}$.
\end{thm}


\noindent{\bf 証明}
~定理 \ref{sthmuopairbasis}より
\[
  c \in \{a, b\} \leftrightarrow c = a \vee c = b
\]
が成り立つから, 推論法則 \ref{dedeqcp}により
\begin{equation}
\label{sthmuopairnotin2}
  c \notin \{a, b\} \leftrightarrow \neg (c = a \vee c = b)
\end{equation}
が成り立つ.
またThm \ref{n1awb1lnavnb}より
\begin{equation}
\label{sthmuopairnotin3}
  \neg (c = a \vee c = b) \leftrightarrow c \neq a \wedge c \neq b
\end{equation}
が成り立つ.
そこで(\ref{sthmuopairnotin2}), (\ref{sthmuopairnotin3})から, 
推論法則 \ref{dedeqtrans}によって(\ref{sthmuopairnotin1})が成り立つ.
1), 2)が成り立つことはこれと推論法則 \ref{dedwedge}, \ref{dedeqfund}によって明らかである.
\halmos




\mathstrut
\begin{thm}
\label{sthmuopairch}%定理4.5%確認済
$a$と$b$を集合とするとき, 
\[
  \{a, b\} = \{b, a\}
\]
が成り立つ.
\end{thm}


\noindent{\bf 証明}
~$x$を$a$, $b$の中に自由変数として現れない, 定数でない文字とする.
このときThm \ref{avblbva}より
\[
  x = a \vee x = b \leftrightarrow x = b \vee x = a
\]
が成り立つから, これと$x$が定数でないことから, 定理 \ref{sthmalleqiset=}より
\[
  \{x \mid x = a \vee x = b\} = \{x \mid x = b \vee x = a\}
\]
が成り立つ.
いま$x$は$a$, $b$の中に自由変数として現れないから, この記号列は$\{a, b\} = \{b, a\}$と同じである.
故にこれが成り立つ.
\halmos




\mathstrut
\begin{thm}
\label{sthmuopairsubset}%定理4.6%1)と2)は新規%確認済
$a$, $b$, $c$を集合とするとき, 
\begin{equation}
\label{sthmuopairsubset1}
  \{a, b\} \subset c \leftrightarrow a \in c \wedge b \in c
\end{equation}
が成り立つ.
またこのことから, 次の1), 2)が成り立つ.

1)
$\{a, b\} \subset c$ならば, $a \in c$と$b \in c$が共に成り立つ.

2)
$a \in c$と$b \in c$が共に成り立てば, $\{a, b\} \subset c$.
\end{thm}


\noindent{\bf 証明}
~定理 \ref{sthmsubsetbasis}より
\begin{align}
  \label{sthmuopairsubset2}
  &\{a, b\} \subset c \to (a \in \{a, b\} \to a \in c), \\
  \mbox{} \notag \\
  \label{sthmuopairsubset3}
  &\{a, b\} \subset c \to (b \in \{a, b\} \to b \in c)
\end{align}
が共に成り立つ.
また定理 \ref{sthmuopairfund}より$a \in \{a, b\}$と$b \in \{a, b\}$が共に成り立つから, 
推論法則 \ref{ded1atb1tbtrue2}により
\begin{align}
  \label{sthmuopairsubset4}
  &(a \in \{a, b\} \to a \in c) \to a \in c, \\
  \mbox{} \notag \\
  \label{sthmuopairsubset5}
  &(b \in \{a, b\} \to b \in c) \to b \in c
\end{align}
が共に成り立つ.
そこで(\ref{sthmuopairsubset2})と(\ref{sthmuopairsubset4}), 
(\ref{sthmuopairsubset3})と(\ref{sthmuopairsubset5})から, それぞれ推論法則 \ref{dedmmp}によって
\[
  \{a, b\} \subset c \to a \in c, ~~
  \{a, b\} \subset c \to b \in c
\]
が成り立つ.
故に推論法則 \ref{dedprewedge}により
\begin{equation}
\label{sthmuopairsubset6}
  \{a, b\} \subset c \to a \in c \wedge b \in c
\end{equation}
が成り立つ.
また$x$を$a$, $b$, $c$の中に自由変数として現れない, 定数でない文字とするとき, 
定理 \ref{sthmuopairbasis}と推論法則 \ref{dedequiv}により
\begin{equation}
\label{sthmuopairsubset7}
  x \in \{a, b\} \to x = a \vee x = b
\end{equation}
が成り立つ.
また定理 \ref{sthm=tineq}と推論法則 \ref{dedpreequiv}により
\[
  x = a \to (a \in c \to x \in c), ~~
  x = b \to (b \in c \to x \in c)
\]
が共に成り立つから, 推論法則 \ref{dedfromaddv}により
\begin{equation}
\label{sthmuopairsubset8}
  x = a \vee x = b \to (a \in c \to x \in c) \vee (b \in c \to x \in c)
\end{equation}
が成り立つ.
またThm \ref{1atc1v1btc1t1awbtc1}より
\begin{equation}
\label{sthmuopairsubset9}
  (a \in c \to x \in c) \vee (b \in c \to x \in c) \to (a \in c \wedge b \in c \to x \in c)
\end{equation}
が成り立つ.
そこで(\ref{sthmuopairsubset7})---(\ref{sthmuopairsubset9})から, 推論法則 \ref{dedmmp}によって
\[
  x \in \{a, b\} \to (a \in c \wedge b \in c \to x \in c)
\]
が成り立つことがわかる.
故に推論法則 \ref{dedch}により
\begin{equation}
\label{sthmuopairsubset10}
  a \in c \wedge b \in c \to (x \in \{a, b\} \to x \in c)
\end{equation}
が成り立つ.
ここで$x$が$a$, $b$, $c$の中に自由変数として現れないことから, 
変数法則 \ref{valfund}, \ref{valwedge}により, 
$x$は$a \in c \wedge b \in c$の中に自由変数として現れない.
また$x$は定数でない.
そこでこれらのことと(\ref{sthmuopairsubset10})が成り立つことから, 
推論法則 \ref{dedalltquansepfreeconst}により
\[
  a \in c \wedge b \in c \to \forall x(x \in \{a, b\} \to x \in c)
\]
が成り立つ.
ここで$x$が$a$, $b$の中に自由変数として現れないことから, 
変数法則 \ref{valnset}により, $x$は$\{a, b\}$の中に自由変数として現れない.
また$x$は$c$の中に自由変数として現れない.
故に上記の記号列は
\begin{equation}
\label{sthmuopairsubset11}
  a \in c \wedge b \in c \to \{a, b\} \subset c
\end{equation}
と同じである.
従ってこれが成り立つ.
そこで(\ref{sthmuopairsubset6}), (\ref{sthmuopairsubset11})から, 
推論法則 \ref{dedequiv}により(\ref{sthmuopairsubset1})が成り立つ.
1), 2)が成り立つことは(\ref{sthmuopairsubset1})と
推論法則 \ref{dedwedge}, \ref{dedeqfund}によって明らかである.
\halmos




\mathstrut
\begin{thm}
\label{sthmuopairnotsubset}%定理4.7%新規%要る?%確認済
$a$, $b$, $c$を集合とするとき, 
\begin{equation}
\label{sthmuopairnotsubset1}
  \{a, b\} \not\subset c \leftrightarrow a \notin c \vee b \notin c
\end{equation}
が成り立つ.
またこのことから, 次の1), 2)が成り立つ.

1)
$\{a, b\} \not\subset c$ならば, $a \notin c \vee b \notin c$.

2)
$a \notin c$ならば, $\{a, b\} \not\subset c$.
また$b \notin c$ならば, $\{a, b\} \not\subset c$.
\end{thm}


\noindent{\bf 証明}
~定理 \ref{sthmuopairsubset}より
\[
  \{a, b\} \subset c \leftrightarrow a \in c \wedge b \in c
\]
が成り立つから, 推論法則 \ref{dedeqcp}により
\begin{equation}
\label{sthmuopairnotsubset2}
  \{a, b\} \not\subset c \leftrightarrow \neg (a \in c \wedge b \in c)
\end{equation}
が成り立つ.
またThm \ref{n1awb1lnavnb}より
\begin{equation}
\label{sthmuopairnotsubset3}
  \neg (a \in c \wedge b \in c) \leftrightarrow a \notin c \vee b \notin c
\end{equation}
が成り立つ.
そこで(\ref{sthmuopairnotsubset2}), (\ref{sthmuopairnotsubset3})から, 
推論法則 \ref{dedeqtrans}によって(\ref{sthmuopairnotsubset1})が成り立つ.

\noindent
1)
(\ref{sthmuopairnotsubset1})と推論法則 \ref{dedeqfund}によって明らか.

\noindent
2)
(\ref{sthmuopairnotsubset1})と推論法則 \ref{dedvee}, \ref{dedeqfund}によって明らか.
\halmos




\mathstrut
\begin{thm}
\label{sthmuopair=}%定理4.8%確認済
\mbox{}

1)
$a$, $b$, $c$を集合とするとき, 
\begin{equation}
\label{sthmuopair=1}
  a = b \leftrightarrow \{a, c\} = \{b, c\}, ~~
  a = b \leftrightarrow \{c, a\} = \{c, b\}
\end{equation}
が成り立つ.
またこのことから, 次の(\ref{sthmuopair=2}), (\ref{sthmuopair=3})が成り立つ.
\begin{align}
  \label{sthmuopair=2}
  &a = b \text{ならば,} ~\{a, c\} = \{b, c\} \text{と} \{c, a\} = \{c, b\} \text{が共に成り立つ.} \\
  \mbox{} \notag \\
  \label{sthmuopair=3}
  &\{a, c\} = \{b, c\} \text{ならば,} ~a = b. ~
  \text{また} \{c, a\} = \{c, b\} \text{ならば,} ~a = b.
\end{align}

2)
$a$, $b$, $c$, $d$を集合とするとき, 
\begin{equation}
\label{sthmuopair=4}
  a = c \wedge b = d \to \{a, b\} = \{c, d\}
\end{equation}
が成り立つ.
またこのことから, 次の(\ref{sthmuopair=5})が成り立つ.
\begin{equation}
\label{sthmuopair=5}
  a = c \text{と} b = d \text{が共に成り立てば,} ~\{a, b\} = \{c, d\}.
\end{equation}
\end{thm}


\noindent{\bf 証明}
~1)
まず(\ref{sthmuopair=1})の前者の記号列が定理であることを示す.
$x$を$c$の中に自由変数として現れない文字とするとき, Thm \ref{T=Ut1TV=UV1}より
\[
  a = b \to (a|x)(\{x, c\}) = (b|x)(\{x, c\})
\]
が成り立つが, 代入法則 \ref{substfree}, \ref{substnset}によればこの記号列は
\begin{equation}
\label{sthmuopair=6}
  a = b \to \{a, c\} = \{b, c\}
\end{equation}
と一致するから, これが成り立つ.
また定理 \ref{sthm=tineq}と推論法則 \ref{dedpreequiv}により
\begin{align}
  \label{sthmuopair=7}
  &\{a, c\} = \{b, c\} \to (a \in \{a, c\} \to a \in \{b, c\}), \\
  \mbox{} \notag \\
  \label{sthmuopair=8}
  &\{a, c\} = \{b, c\} \to (b \in \{b, c\} \to b \in \{a, c\})
\end{align}
が共に成り立つ.
また定理 \ref{sthmuopairfund}より$a \in \{a, c\}$と$b \in \{b, c\}$が共に成り立つから, 
推論法則 \ref{ded1atb1tbtrue2}により
\begin{align}
  \label{sthmuopair=9}
  &(a \in \{a, c\} \to a \in \{b, c\}) \to a \in \{b, c\}, \\
  \mbox{} \notag \\
  \label{sthmuopair=10}
  &(b \in \{b, c\} \to b \in \{a, c\}) \to b \in \{a, c\}
\end{align}
が共に成り立つ.
また定理 \ref{sthmuopairbasis}と推論法則 \ref{dedequiv}により
\begin{align}
  \label{sthmuopair=11}
  &a \in \{b, c\} \to a = b \vee a = c, \\
  \mbox{} \notag \\
  \label{sthmuopair=12}
  &b \in \{a, c\} \to b = a \vee b = c
\end{align}
が共に成り立つ.
またThm \ref{x=yty=x}より
\[
  b = a \to a = b, ~~
  b = c \to c = b
\]
が共に成り立つから, 推論法則 \ref{dedfromaddv}により
\begin{equation}
\label{sthmuopair=13}
  b = a \vee b = c \to a = b \vee c = b
\end{equation}
が成り立つ.
さて以上の(\ref{sthmuopair=7}), (\ref{sthmuopair=9}), (\ref{sthmuopair=11})から, 
推論法則 \ref{dedmmp}によって
\begin{equation}
\label{sthmuopair=14}
  \{a, c\} = \{b, c\} \to a = b \vee a = c
\end{equation}
が成り立つことがわかる.
また(\ref{sthmuopair=8}), (\ref{sthmuopair=10}), (\ref{sthmuopair=12}), (\ref{sthmuopair=13})から, 
同じく推論法則 \ref{dedmmp}によって
\begin{equation}
\label{sthmuopair=15}
  \{a, c\} = \{b, c\} \to a = b \vee c = b
\end{equation}
が成り立つことがわかる.
故に(\ref{sthmuopair=14}), (\ref{sthmuopair=15})から, 推論法則 \ref{dedprewedge}により
\begin{equation}
\label{sthmuopair=16}
  \{a, c\} = \{b, c\} \to (a = b \vee a = c) \wedge (a = b \vee c = b)
\end{equation}
が成り立つ.
またThm \ref{1avb1w1avc1tav1bwc1}より
\begin{equation}
\label{sthmuopair=17}
  (a = b \vee a = c) \wedge (a = b \vee c = b) \to a = b \vee (a = c \wedge c = b)
\end{equation}
が成り立つ.
またThm \ref{x=ywy=ztx=z}より
\[
  a = c \wedge c = b \to a = b
\]
が成り立つから, 推論法則 \ref{dedavbtbtrue1}により
\begin{equation}
\label{sthmuopair=18}
  a = b \vee (a = c \wedge c = b) \to a = b
\end{equation}
が成り立つ.
そこで(\ref{sthmuopair=16})---(\ref{sthmuopair=18})から, 推論法則 \ref{dedmmp}によって
\begin{equation}
\label{sthmuopair=19}
  \{a, c\} = \{b, c\} \to a = b
\end{equation}
が成り立つことがわかる.
故に(\ref{sthmuopair=6}), (\ref{sthmuopair=19})から, 推論法則 \ref{dedequiv}により
\begin{equation}
\label{sthmuopair=20}
  a = b \leftrightarrow \{a, c\} = \{b, c\}
\end{equation}
が成り立つ.

次に(\ref{sthmuopair=1})の後者の記号列が定理であることを示す.
まず今示したように(\ref{sthmuopair=20})が成り立つ.
また定理 \ref{sthmuopairch}より$\{a, c\} = \{c, a\}$と$\{b, c\} = \{c, b\}$が共に成り立つから, 
推論法則 \ref{dedaddeq=}により
\begin{equation}
\label{sthmuopair=21}
  \{a, c\} = \{b, c\} \leftrightarrow \{c, a\} = \{c, b\}
\end{equation}
が成り立つ.
そこで(\ref{sthmuopair=20}), (\ref{sthmuopair=21})から, 
推論法則 \ref{dedeqtrans}によって$a = b \leftrightarrow \{c, a\} = \{c, b\}$が成り立つ.

(\ref{sthmuopair=2}), (\ref{sthmuopair=3})が成り立つことは, 
(\ref{sthmuopair=1})と推論法則 \ref{dedeqfund}から直ちにわかる.

\noindent
2)
$x$と$y$を互いに異なる文字とするとき, Thm \ref{thm=gsubst}より
\[
  a = c \wedge b = d \to (a|x, b|y)(\{x, y\}) = (c|x, d|y)(\{x, y\})
\]
が成り立つが, 一般代入法則 \ref{gsubstnset}によればこの記号列は(\ref{sthmuopair=4})と一致するから, 
(\ref{sthmuopair=4})が成り立つ.
(\ref{sthmuopair=5})が成り立つことは, (\ref{sthmuopair=4})と
推論法則 \ref{dedmp}, \ref{dedwedge}から直ちにわかる.
\halmos




\mathstrut
\begin{thm}
\label{sthmspinuopair}%定理4.9%新規%確認済
$a$と$b$を集合, $R$を関係式とし, $x$を$a$と$b$の中に自由変数として現れない文字とする.
このとき
\begin{align}
  \label{sthmspinuopair1}
  &(\exists x \in \{a, b\})(R) \leftrightarrow (a|x)(R) \vee (b|x)(R), \\
  \mbox{} \notag \\
  \label{sthmspinuopair2}
  &(\forall x \in \{a, b\})(R) \leftrightarrow (a|x)(R) \wedge (b|x)(R)
\end{align}
が共に成り立つ.
またこれらから, 次の1)---4)が成り立つ.

1)
$(\exists x \in \{a, b\})(R)$ならば, $(a|x)(R) \vee (b|x)(R)$.

2)
$(a|x)(R)$ならば, $(\exists x \in \{a, b\})(R)$.
また$(b|x)(R)$ならば, $(\exists x \in \{a, b\})(R)$.

3)
$(\forall x \in \{a, b\})(R)$ならば, $(a|x)(R)$と$(b|x)(R)$が共に成り立つ.

4)
$(a|x)(R)$と$(b|x)(R)$が共に成り立てば, $(\forall x \in \{a, b\})(R)$.
\end{thm}


\noindent{\bf 証明}
~$x$が$a$, $b$の中に自由変数として現れないことから, 
$\{a, b\}$は$\{x \mid x = a \vee x = b\}$と同じである.
また定理 \ref{sthmaxiom2}より, $x = a \vee x = b$は$x$について集合を作り得る.
故に定理 \ref{sthmspiniset}より
\begin{align}
  \label{sthmspinuopair3}
  &(\exists x \in \{a, b\})(R) \leftrightarrow \exists_{x = a \vee x = b}x(R), \\
  \mbox{} \notag \\
  \label{sthmspinuopair4}
  &(\forall x \in \{a, b\})(R) \leftrightarrow \forall_{x = a \vee x = b}x(R)
\end{align}
が共に成り立つ.
またThm \ref{thmspexprev}より
\begin{equation}
\label{sthmspinuopair5}
  \exists_{x = a \vee x = b}x(R) \leftrightarrow \exists_{x = a}x(R) \vee \exists_{x = b}x(R)
\end{equation}
が成り立つ.
またThm \ref{thmspallprev}より
\begin{equation}
\label{sthmspinuopair6}
  \forall_{x = a \vee x = b}x(R) \leftrightarrow \forall_{x = a}x(R) \wedge \forall_{x = b}x(R)
\end{equation}
が成り立つ.
また$x$が$a$, $b$の中に自由変数として現れないことから, 
Thm \ref{thmspquan=}と推論法則 \ref{dedeqch}により
\begin{align*}
  &\exists_{x = a}x(R) \leftrightarrow (a|x)(R), ~~
  \exists_{x = b}x(R) \leftrightarrow (b|x)(R), \\
  \mbox{} \\
  &\forall_{x = a}x(R) \leftrightarrow (a|x)(R), ~~
  \forall_{x = b}x(R) \leftrightarrow (b|x)(R)
\end{align*}
がすべて成り立つ.
故にこのはじめの二つから, 推論法則 \ref{dedaddeqv}により
\begin{equation}
\label{sthmspinuopair7}
  \exists_{x = a}x(R) \vee \exists_{x = b}x(R) \leftrightarrow (a|x)(R) \vee (b|x)(R)
\end{equation}
が成り立ち, あとの二つから, 推論法則 \ref{dedaddeqw}により
\begin{equation}
\label{sthmspinuopair8}
  \forall_{x = a}x(R) \wedge \forall_{x = b}x(R) \leftrightarrow (a|x)(R) \wedge (b|x)(R)
\end{equation}
が成り立つ.
そこで(\ref{sthmspinuopair3}), (\ref{sthmspinuopair5}), (\ref{sthmspinuopair7})から, 
推論法則 \ref{dedeqtrans}によって(\ref{sthmspinuopair1})が成り立つことがわかる.
また(\ref{sthmspinuopair4}), (\ref{sthmspinuopair6}), (\ref{sthmspinuopair8})から, 
同じく推論法則 \ref{dedeqtrans}によって(\ref{sthmspinuopair2})が成り立つことがわかる.

\noindent
1)
(\ref{sthmspinuopair1})と推論法則 \ref{dedeqfund}によって明らか.

\noindent
2)
(\ref{sthmspinuopair1})と推論法則 \ref{dedvee}, \ref{dedeqfund}によって明らか.

\noindent
3), 4)
(\ref{sthmspinuopair2})と推論法則 \ref{dedwedge}, \ref{dedeqfund}によって明らか.
\halmos




\mathstrut
\begin{thm}
\label{sthmsingletonsm}%定理4.10%新規%確認済
$a$を集合とし, $x$を$a$の中に自由変数として現れない文字とする.
このとき関係式$x = a$は$x$について集合を作り得る.
\end{thm}


\noindent{\bf 証明}
~$\tau_{x}(\neg (x = a \vee x = a \leftrightarrow x = a))$を$T$と書けば, $T$は対象式であり, 
Thm \ref{avala}より
\[
  (T|x)(x = a) \vee (T|x)(x = a) \leftrightarrow (T|x)(x = a)
\]
が成り立つ.
ここで代入法則 \ref{substfund}, \ref{substequiv}によれば, この記号列は
\[
  (T|x)(x = a \vee x = a \leftrightarrow x = a)
\]
と一致するから, これが成り立つ.
故に$T$の定義から, 推論法則 \ref{dedallfund}により
\begin{equation}
\label{sthmsingletonsm1}
  \forall x(x = a \vee x = a \leftrightarrow x = a)
\end{equation}
が成り立つ.
ここで$x$が$a$の中に自由変数として現れないことから, 定理 \ref{sthmaxiom2}より
$x = a \vee x = a$は$x$について集合を作り得る.
そこでこれと(\ref{sthmsingletonsm1})が成り立つことから, 
定理 \ref{sthmalleqsm}より$x = a$は$x$について集合を作り得る.
\halmos




\mathstrut
\begin{thm}
\label{sthmaa=a}%定理4.11%新規%確認済
$a$を集合とするとき, 
\[
  \{a, a\} = \{a\}
\]
が成り立つ.
\end{thm}


\noindent{\bf 証明}
~$x$を$a$の中に自由変数として現れない, 定数でない文字とする.
このときThm \ref{avala}より
\[
  x = a \vee x = a \leftrightarrow x = a
\]
が成り立つから, これと$x$が定数でないことから, 定理 \ref{sthmalleqiset=}より
\[
  \{x \mid x = a \vee x = a\} = \{x \mid x = a\}
\]
が成り立つ.
$x$が$a$の中に自由変数として現れないことから, この記号列は$\{a, a\} = \{a\}$と同じである.
故にこれが成り立つ.
\halmos




\mathstrut
\begin{thm}
\label{sthmsingletonbasis}%定理4.12%確認済
$a$と$b$を集合とするとき, 
\begin{equation}
\label{sthmsingletonbasis1}
  b \in \{a\} \leftrightarrow b = a
\end{equation}
が成り立つ.
またこのことから, 次の(\ref{sthmsingletonbasis2})が成り立つ.
\begin{equation}
\label{sthmsingletonbasis2}
  b \in \{a\} \text{ならば,} ~b = a. ~
  \text{また} b = a \text{ならば,} ~b \in \{a\}.
\end{equation}
\end{thm}


\noindent{\bf 証明}
~$x$を$a$の中に自由変数として現れない文字とする.
このとき$\{a\}$は$\{x \mid x = a\}$と同じである.
また定理 \ref{sthmsingletonsm}より関係式$x = a$は$x$について集合を作り得る.
故に定理 \ref{sthmisetbasis}より
\[
  b \in \{a\} \leftrightarrow (b|x)(x = a)
\]
が成り立つが, $x$が$a$の中に自由変数として現れないことから, 
代入法則 \ref{substfree}, \ref{substfund}によりこの記号列は
(\ref{sthmsingletonbasis1})と一致するから, (\ref{sthmsingletonbasis1})が成り立つ.
(\ref{sthmsingletonbasis2})が成り立つことは, 
(\ref{sthmsingletonbasis1})と推論法則 \ref{dedeqfund}によって明らかである.
\halmos




\mathstrut
\begin{thm}
\label{sthmsingletonfund}%定理4.13%確認済
$a$を集合とするとき, 
\[
  a \in \{a\}
\]
が成り立つ.
\end{thm}


\noindent{\bf 証明}
~Thm \ref{x=x}より$a = a$だから, 定理 \ref{sthmsingletonbasis}より$a \in \{a\}$が成り立つ.
\halmos




\mathstrut
\begin{thm}
\label{sthmsingletonsubset}%定理4.14%確認済
$a$と$b$を集合とするとき, 
\begin{equation}
\label{sthmsingletonsubset1}
  \{a\} \subset b \leftrightarrow a \in b
\end{equation}
が成り立つ.
またこのことから, 次の(\ref{sthmsingletonsubset2})が成り立つ.
\begin{equation}
\label{sthmsingletonsubset2}
  \{a\} \subset b \text{ならば,} ~a \in b. ~
  \text{また} a \in b \text{ならば,} ~\{a\} \subset b.
\end{equation}
\end{thm}


\noindent{\bf 証明}
~定理 \ref{sthmaa=a}と推論法則 \ref{ded=ch}により$\{a\} = \{a, a\}$が成り立つから, 
定理 \ref{sthm=tsubseteq}より
\begin{equation}
\label{sthmsingletonsubset3}
  \{a\} \subset b \leftrightarrow \{a, a\} \subset b
\end{equation}
が成り立つ.
また定理 \ref{sthmuopairsubset}より
\begin{equation}
\label{sthmsingletonsubset4}
  \{a, a\} \subset b \leftrightarrow a \in b \wedge a \in b
\end{equation}
が成り立つ.
またThm \ref{awala}より
\begin{equation}
\label{sthmsingletonsubset5}
  a \in b \wedge a \in b \leftrightarrow a \in b
\end{equation}
が成り立つ.
そこで(\ref{sthmsingletonsubset3})---(\ref{sthmsingletonsubset5})から, 
推論法則 \ref{dedeqtrans}によって(\ref{sthmsingletonsubset1})が成り立つことがわかる.
(\ref{sthmsingletonsubset2})が成り立つことは, 
(\ref{sthmsingletonsubset1})と推論法則 \ref{dedeqfund}によって明らかである.
\halmos




\mathstrut
\begin{thm}
\label{sthmsingletonsubsetuopair}%定理4.15%確認済
$a$と$b$を集合とするとき, 
\[
  \{a\} \subset \{a, b\}, ~~
  \{b\} \subset \{a, b\}
\]
が成り立つ.
\end{thm}


\noindent{\bf 証明}
~定理 \ref{sthmuopairfund}より$a \in \{a, b\}$と$b \in \{a, b\}$が共に成り立つから, 
定理 \ref{sthmsingletonsubset}より$\{a\} \subset \{a, b\}$と$\{b\} \subset \{a, b\}$が共に成り立つ.
\halmos




\mathstrut
\begin{thm}
\label{sthmsingleton=}%定理4.16%確認済
$a$と$b$を集合とするとき, 
\begin{equation}
\label{sthmsingleton=1}
  a = b \leftrightarrow \{a\} = \{b\}
\end{equation}
が成り立つ.
またこのことから, 次の(\ref{sthmsingleton=2})が成り立つ.
\begin{equation}
\label{sthmsingleton=2}
  a = b \text{ならば,} ~\{a\} = \{b\}. ~
  \text{また} \{a\} = \{b\} \text{ならば,} ~a = b.
\end{equation}
\end{thm}


\noindent{\bf 証明}
~$x$を文字とするとき, Thm \ref{T=Ut1TV=UV1}より
\[
  a = b \to (a|x)(\{x\}) = (b|x)(\{x\})
\]
が成り立つが, 代入法則 \ref{substnset}によればこの記号列は
\begin{equation}
\label{sthmsingleton=3}
  a = b \to \{a\} = \{b\}
\end{equation}
と一致するから, これが成り立つ.
また定理 \ref{sthm=tineq}と推論法則 \ref{dedpreequiv}により
\begin{equation}
\label{sthmsingleton=4}
  \{a\} = \{b\} \to (a \in \{a\} \to a \in \{b\})
\end{equation}
が成り立つ.
また定理 \ref{sthmsingletonfund}より$a \in \{a\}$が成り立つから, 
推論法則 \ref{ded1atb1tbtrue2}により
\begin{equation}
\label{sthmsingleton=5}
  (a \in \{a\} \to a \in \{b\}) \to a \in \{b\}
\end{equation}
が成り立つ.
また定理 \ref{sthmsingletonbasis}と推論法則 \ref{dedequiv}により
\begin{equation}
\label{sthmsingleton=6}
  a \in \{b\} \to a = b
\end{equation}
が成り立つ.
そこで(\ref{sthmsingleton=4})---(\ref{sthmsingleton=6})から, 推論法則 \ref{dedmmp}によって
\begin{equation}
\label{sthmsingleton=7}
  \{a\} = \{b\} \to a = b
\end{equation}
が成り立つことがわかる.
故に(\ref{sthmsingleton=3}), (\ref{sthmsingleton=7})から, 
推論法則 \ref{dedequiv}により(\ref{sthmsingleton=1})が成り立つ.
(\ref{sthmsingleton=2})が成り立つことは, 
(\ref{sthmsingleton=1})と推論法則 \ref{dedeqfund}によって明らかである.
\halmos




\mathstrut
\begin{thm}
\label{sthmsingleton=subset}%定理4.17%確認済
$a$と$b$を集合とするとき, 
\begin{equation}
\label{sthmsingleton=subset1}
  a = b \leftrightarrow \{a\} \subset \{b\}
\end{equation}
が成り立つ.
またこのことから, 次の(\ref{sthmsingleton=subset2})が成り立つ.
\begin{equation}
\label{sthmsingleton=subset2}
  a = b \text{ならば,} ~\{a\} \subset \{b\}. ~
  \text{また} \{a\} \subset \{b\} \text{ならば,} ~a = b.
\end{equation}
\end{thm}


\noindent{\bf 証明}
~定理 \ref{sthmsingleton=}と推論法則 \ref{dedequiv}により
\begin{equation}
\label{sthmsingleton=subset3}
  a = b \to \{a\} = \{b\}
\end{equation}
が成り立つ.
また定理 \ref{sthm=tsubset}より
\begin{equation}
\label{sthmsingleton=subset4}
  \{a\} = \{b\} \to \{a\} \subset \{b\}
\end{equation}
が成り立つ.
そこで(\ref{sthmsingleton=subset3}), (\ref{sthmsingleton=subset4})から, 推論法則 \ref{dedmmp}によって
\begin{equation}
\label{sthmsingleton=subset5}
  a = b \to \{a\} \subset \{b\}
\end{equation}
が成り立つ.
また定理 \ref{sthmsubsetbasis}より
\begin{equation}
\label{sthmsingleton=subset6}
  \{a\} \subset \{b\} \to (a \in \{a\} \to a \in \{b\})
\end{equation}
が成り立つ.
また定理 \ref{sthmsingletonfund}より$a \in \{a\}$が成り立つから, 
推論法則 \ref{ded1atb1tbtrue2}により
\begin{equation}
\label{sthmsingleton=subset7}
  (a \in \{a\} \to a \in \{b\}) \to a \in \{b\}
\end{equation}
が成り立つ.
また定理 \ref{sthmsingletonbasis}と推論法則 \ref{dedequiv}により
\begin{equation}
\label{sthmsingleton=subset8}
  a \in \{b\} \to a = b
\end{equation}
が成り立つ.
そこで(\ref{sthmsingleton=subset6})---(\ref{sthmsingleton=subset8})から, 推論法則 \ref{dedmmp}によって
\begin{equation}
\label{sthmsingleton=subset9}
  \{a\} \subset \{b\} \to a = b
\end{equation}
が成り立つことがわかる.
故に(\ref{sthmsingleton=subset5}), (\ref{sthmsingleton=subset9})から, 
推論法則 \ref{dedequiv}により(\ref{sthmsingleton=subset1})が成り立つ.
(\ref{sthmsingleton=subset2})が成り立つことは, 
(\ref{sthmsingleton=subset1})と推論法則 \ref{dedeqfund}によって明らかである.
\halmos




\mathstrut
\begin{thm}
\label{sthmsingletonuopairsubset}%定理4.18%新規%確認済
$a$, $b$, $c$を集合とするとき, 
\begin{align}
  \label{sthmsingletonuopairsubset1}
  &\{a\} \subset \{b, c\} \leftrightarrow a = b \vee a = c, \\
  \mbox{} \notag \\
  \label{sthmsingletonuopairsubset2}
  &\{b, c\} \subset \{a\} \leftrightarrow a = b \wedge a = c
\end{align}
が共に成り立つ.
またこれらから, 次の1)---4)が成り立つ.

1)
$\{a\} \subset \{b, c\}$ならば, $a = b \vee a = c$.

2)
$a = b$ならば, $\{a\} \subset \{b, c\}$.
また$a = c$ならば, $\{a\} \subset \{b, c\}$.

3)
$\{b, c\} \subset \{a\}$ならば, $a = b$と$a = c$が共に成り立つ.

4)
$a = b$と$a = c$が共に成り立てば, $\{b, c\} \subset \{a\}$.
\end{thm}


\noindent{\bf 証明}
~定理 \ref{sthmsingletonsubset}より
\begin{equation}
\label{sthmsingletonuopairsubset3}
  \{a\} \subset \{b, c\} \leftrightarrow a \in \{b, c\}
\end{equation}
が成り立つ.
また定理 \ref{sthmuopairbasis}より
\begin{equation}
\label{sthmsingletonuopairsubset4}
  a \in \{b, c\} \leftrightarrow a = b \vee a = c
\end{equation}
が成り立つ.
そこで(\ref{sthmsingletonuopairsubset3}), (\ref{sthmsingletonuopairsubset4})から, 
推論法則 \ref{dedeqtrans}によって(\ref{sthmsingletonuopairsubset1})が成り立つ.
また定理 \ref{sthmuopairsubset}より
\begin{equation}
\label{sthmsingletonuopairsubset5}
  \{b, c\} \subset \{a\} \leftrightarrow b \in \{a\} \wedge c \in \{a\}
\end{equation}
が成り立つ.
また定理 \ref{sthmsingletonbasis}より
\[
  b \in \{a\} \leftrightarrow b = a, ~~
  c \in \{a\} \leftrightarrow c = a
\]
が共に成り立つから, 推論法則 \ref{dedaddeqw}により
\begin{equation}
\label{sthmsingletonuopairsubset6}
  b \in \{a\} \wedge c \in \{a\} \leftrightarrow b = a \wedge c = a
\end{equation}
が成り立つ.
またThm \ref{x=yly=x}より
\[
  b = a \leftrightarrow a = b, ~~
  c = a \leftrightarrow a = c
\]
が共に成り立つから, 推論法則 \ref{dedaddeqw}により
\begin{equation}
\label{sthmsingletonuopairsubset7}
  b = a \wedge c = a \leftrightarrow a = b \wedge a = c
\end{equation}
が成り立つ.
そこで(\ref{sthmsingletonuopairsubset5})---(\ref{sthmsingletonuopairsubset7})から, 
推論法則 \ref{dedeqtrans}によって(\ref{sthmsingletonuopairsubset2})が成り立つことがわかる.

\noindent
1)
(\ref{sthmsingletonuopairsubset1})と推論法則 \ref{dedeqfund}によって明らか.

\noindent
2)
(\ref{sthmsingletonuopairsubset1})と推論法則 \ref{dedvee}, \ref{dedeqfund}によって明らか.

\noindent
3), 4)
(\ref{sthmsingletonuopairsubset2})と推論法則 \ref{dedwedge}, \ref{dedeqfund}によって明らか.
\halmos




\mathstrut
\begin{thm}
\label{sthmsingleton=uopair}%定理4.19%新規%確認済
$a$, $b$, $c$を集合とするとき, 
\begin{equation}
\label{sthmsingleton=uopair1}
  \{a\} = \{b, c\} \leftrightarrow a = b \wedge a = c
\end{equation}
が成り立つ.
またこのことから, 次の1), 2)が成り立つ.

1)
$\{a\} = \{b, c\}$ならば, $a = b$と$a = c$が共に成り立つ.

2)
$a = b$と$a = c$が共に成り立てば, $\{a\} = \{b, c\}$.
\end{thm}


\noindent{\bf 証明}
~定理 \ref{sthmaxiom1}と推論法則 \ref{dedeqch}により
\begin{equation}
\label{sthmsingleton=uopair2}
  \{a\} = \{b, c\} \leftrightarrow \{a\} \subset \{b, c\} \wedge \{b, c\} \subset \{a\}
\end{equation}
が成り立つ.
また定理 \ref{sthmsingletonuopairsubset}より
\[
  \{a\} \subset \{b, c\} \leftrightarrow a = b \vee a = c, ~~
  \{b, c\} \subset \{a\} \leftrightarrow a = b \wedge a = c
\]
が共に成り立つから, 推論法則 \ref{dedaddeqw}により
\begin{equation}
\label{sthmsingleton=uopair3}
  \{a\} \subset \{b, c\} \wedge \{b, c\} \subset \{a\} 
  \leftrightarrow (a = b \vee a = c) \wedge (a = b \wedge a = c)
\end{equation}
が成り立つ.
またThm \ref{awbta}より
\begin{equation}
\label{sthmsingleton=uopair4}
  a = b \wedge a = c \to a = b
\end{equation}
が成り立つ.
またThm \ref{atavb}より
\begin{equation}
\label{sthmsingleton=uopair5}
  a = b \to a = b \vee a = c
\end{equation}
が成り立つ.
そこで(\ref{sthmsingleton=uopair4}), (\ref{sthmsingleton=uopair5})から, 推論法則 \ref{dedmmp}によって
\[
  a = b \wedge a = c \to a = b \vee a = c
\]
が成り立つ.
故に推論法則 \ref{dedawblatrue1}により
\begin{equation}
\label{sthmsingleton=uopair6}
  (a = b \vee a = c) \wedge (a = b \wedge a = c) \leftrightarrow a = b \wedge a = c
\end{equation}
が成り立つ.
そこで(\ref{sthmsingleton=uopair2}), (\ref{sthmsingleton=uopair3}), (\ref{sthmsingleton=uopair6})から, 
推論法則 \ref{dedeqtrans}によって(\ref{sthmsingleton=uopair1})が成り立つことがわかる.
1), 2)が成り立つことは, (\ref{sthmsingleton=uopair1})と
推論法則 \ref{dedwedge}, \ref{dedeqfund}によって明らかである.
\halmos




\mathstrut
\begin{thm}
\label{sthmsingleton=uopairab}%定理4.20%新規%確認済
$a$と$b$を集合とするとき, 
\begin{align}
  \label{sthmsingleton=uopairab1}
  &\{a\} = \{a, b\} \leftrightarrow a = b, \\
  \mbox{} \notag \\
  \label{sthmsingleton=uopairab2}
  &\{b\} = \{a, b\} \leftrightarrow a = b
\end{align}
が共に成り立つ.
またこれらから, 次の1), 2), 3)が成り立つ.

1)
$\{a\} = \{a, b\}$ならば, $a = b$.

2)
$\{b\} = \{a, b\}$ならば, $a = b$.

3)
$a = b$ならば, $\{a\} = \{a, b\}$と$\{b\} = \{a, b\}$が共に成り立つ.
\end{thm}


\noindent{\bf 証明}
~定理 \ref{sthmsingleton=uopair}より
\begin{align}
  \label{sthmsingleton=uopairab3}
  &\{a\} = \{a, b\} \leftrightarrow a = a \wedge a = b, \\
  \mbox{} \notag \\
  \label{sthmsingleton=uopairab4}
  &\{b\} = \{a, b\} \leftrightarrow b = a \wedge b = b
\end{align}
が共に成り立つ.
またThm \ref{x=x}より$a = a$と$b = b$が共に成り立つから, 推論法則 \ref{dedawblatrue2}により
\begin{align}
  \label{sthmsingleton=uopairab5}
  &a = a \wedge a = b \leftrightarrow a = b, \\
  \mbox{} \notag \\
  \label{sthmsingleton=uopairab6}
  &b = a \wedge b = b \leftrightarrow b = a
\end{align}
が共に成り立つ.
またThm \ref{x=yly=x}より
\begin{equation}
\label{sthmsingleton=uopairab7}
  b = a \leftrightarrow a = b
\end{equation}
が成り立つ.
そこで(\ref{sthmsingleton=uopairab3}), (\ref{sthmsingleton=uopairab5})から, 
推論法則 \ref{dedeqtrans}によって(\ref{sthmsingleton=uopairab1})が成り立つ.
また(\ref{sthmsingleton=uopairab4}), (\ref{sthmsingleton=uopairab6}), (\ref{sthmsingleton=uopairab7})から, 
同じく推論法則 \ref{dedeqtrans}によって(\ref{sthmsingleton=uopairab2})が成り立つことがわかる.

\noindent
1)
(\ref{sthmsingleton=uopairab1})と推論法則 \ref{dedeqfund}によって明らか.

\noindent
2)
(\ref{sthmsingleton=uopairab2})と推論法則 \ref{dedeqfund}によって明らか.

\noindent
3)
(\ref{sthmsingleton=uopairab1}), (\ref{sthmsingleton=uopairab2})と推論法則 \ref{dedeqfund}によって明らか.
\halmos




\mathstrut
\begin{thm}
\label{sthmsingletonpsubsetuopair}%定理4.21%新規%確認済
$a$と$b$を集合とするとき, 
\begin{align}
  \label{sthmsingletonpsubsetuopair1}
  &\{a\} \subsetneqq \{a, b\} \leftrightarrow a \neq b, \\
  \mbox{} \notag \\
  \label{sthmsingletonpsubsetuopair2}
  &\{b\} \subsetneqq \{a, b\} \leftrightarrow a \neq b
\end{align}
が共に成り立つ.
またこれらから, 次の1), 2), 3)が成り立つ.

1)
$\{a\} \subsetneqq \{a, b\}$ならば, $a \neq b$.

2)
$\{b\} \subsetneqq \{a, b\}$ならば, $a \neq b$.

3)
$a \neq b$ならば, $\{a\} \subsetneqq \{a, b\}$と$\{b\} \subsetneqq \{a, b\}$が共に成り立つ.
\end{thm}


\noindent{\bf 証明}
~定理 \ref{sthmsingletonsubsetuopair}より$\{a\} \subset \{a, b\}$と$\{b\} \subset \{a, b\}$が
共に成り立つから, 推論法則 \ref{dedawblatrue2}により
\begin{align}
  \label{sthmsingletonpsubsetuopair3}
  &\{a\} \subsetneqq \{a, b\} \leftrightarrow \{a\} \neq \{a, b\}, \\
  \mbox{} \notag \\
  \label{sthmsingletonpsubsetuopair4}
  &\{b\} \subsetneqq \{a, b\} \leftrightarrow \{b\} \neq \{a, b\}
\end{align}
が共に成り立つ.
また定理 \ref{sthmsingleton=uopairab}より
\[
  \{a\} = \{a, b\} \leftrightarrow a = b, ~~
  \{b\} = \{a, b\} \leftrightarrow a = b
\]
が共に成り立つから, 推論法則 \ref{dedeqcp}により
\begin{align}
  \label{sthmsingletonpsubsetuopair5}
  &\{a\} \neq \{a, b\} \leftrightarrow a \neq b, \\
  \mbox{} \notag \\
  \label{sthmsingletonpsubsetuopair6}
  &\{b\} \neq \{a, b\} \leftrightarrow a \neq b
\end{align}
が共に成り立つ.
そこで(\ref{sthmsingletonpsubsetuopair3})と(\ref{sthmsingletonpsubsetuopair5}), 
(\ref{sthmsingletonpsubsetuopair4})と(\ref{sthmsingletonpsubsetuopair6})から, 
それぞれ推論法則 \ref{dedeqtrans}によって
(\ref{sthmsingletonpsubsetuopair1}), (\ref{sthmsingletonpsubsetuopair2})が成り立つ.

\noindent
1)
(\ref{sthmsingletonpsubsetuopair1})と推論法則 \ref{dedeqfund}によって明らか.

\noindent
2)
(\ref{sthmsingletonpsubsetuopair2})と推論法則 \ref{dedeqfund}によって明らか.

\noindent
3)
(\ref{sthmsingletonpsubsetuopair1}), (\ref{sthmsingletonpsubsetuopair2})と
推論法則 \ref{dedeqfund}によって明らか.
\halmos




\mathstrut
\begin{thm}
\label{sthmsubsetsingleton!}%定理4.22%新規%確認済
$a$と$b$を集合とし, $x$を$a$の中に自由変数として現れない文字とする.
このとき
\begin{equation}
\label{sthmsubsetsingleton!1}
  a \subset \{b\} \to \ !x(x \in a)
\end{equation}
が成り立つ.
またこのことから, 次の(\ref{sthmsubsetsingleton!2})が成り立つ.
\begin{equation}
\label{sthmsubsetsingleton!2}
  a \subset \{b\} \text{ならば,} ~!x(x \in a).
\end{equation}
\end{thm}


\noindent{\bf 証明}
~$y$を$x$と異なり, $a$と$b$の中に自由変数として現れない, 定数でない文字とする.
このとき定理 \ref{sthmsubsetbasis}より
\begin{equation}
\label{sthmsubsetsingleton!3}
  a \subset \{b\} \to (y \in a \to y \in \{b\})
\end{equation}
が成り立つ.
また定理 \ref{sthmsingletonbasis}と推論法則 \ref{dedequiv}により
$y \in \{b\} \to y = b$が成り立つから, 推論法則 \ref{dedaddb}により
\begin{equation}
\label{sthmsubsetsingleton!4}
  (y \in a \to y \in \{b\}) \to (y \in a \to y = b)
\end{equation}
が成り立つ.
そこで(\ref{sthmsubsetsingleton!3}), (\ref{sthmsubsetsingleton!4})から, 推論法則 \ref{dedmmp}によって
\begin{equation}
\label{sthmsubsetsingleton!5}
  a \subset \{b\} \to (y \in a \to y = b)
\end{equation}
が成り立つ.
ここで$y$が$a$, $b$の中に自由変数として現れないことから, 
変数法則 \ref{valsubset}, \ref{valnset}により, $y$は$a \subset \{b\}$の中に自由変数として現れない.
また$y$は定数でない.
これらのことと(\ref{sthmsubsetsingleton!5})が成り立つことから, 
推論法則 \ref{dedalltquansepfreeconst}により
\begin{equation}
\label{sthmsubsetsingleton!6}
  a \subset \{b\} \to \forall y(y \in a \to y = b)
\end{equation}
が成り立つ.
また$y$が$b$の中に自由変数として現れないことから, Thm \ref{thmallt!}より
\[
  \forall y(y \in a \to y = b) \to \ !y(y \in a)
\]
が成り立つ.
ここで$x$が$y$と異なり, $a$の中に自由変数として現れないことから, 
変数法則 \ref{valfund}により$x$は$y \in a$の中に自由変数として現れない.
故に代入法則 \ref{subst!trans}により, この記号列は
\[
  \forall y(y \in a \to y = b) \to \ !x((x|y)(y \in a))
\]
と一致する.
また$y$が$a$の中に自由変数として現れないことと代入法則 \ref{substfree}, \ref{substfund}により, 
この記号列は
\begin{equation}
\label{sthmsubsetsingleton!7}
  \forall y(y \in a \to y = b) \to \ !x(x \in a)
\end{equation}
と一致する.
故にこれが成り立つ.
そこで(\ref{sthmsubsetsingleton!6}), (\ref{sthmsubsetsingleton!7})から, 
推論法則 \ref{dedmmp}によって(\ref{sthmsubsetsingleton!1})が成り立つ.
(\ref{sthmsubsetsingleton!2})が成り立つことは, 
(\ref{sthmsubsetsingleton!1})と推論法則 \ref{dedmp}によって明らかである.
\halmos




\mathstrut
\begin{thm}
\label{sthm=singletonex!}%定理4.23%新規%確認済
$a$と$b$を集合とし, $x$を$a$の中に自由変数として現れない文字とする.
このとき
\begin{equation}
\label{sthm=singletonex!1}
  a = \{b\} \to \exists !x(x \in a)
\end{equation}
が成り立つ.
またこのことから, 次の(\ref{sthm=singletonex!2})が成り立つ.
\begin{equation}
\label{sthm=singletonex!2}
  a = \{b\} \text{ならば,} ~\exists !x(x \in a).
\end{equation}
\end{thm}


\noindent{\bf 証明}
~定理 \ref{sthm=tsubset}より
\begin{equation}
\label{sthm=singletonex!3}
  a = \{b\} \to \{b\} \subset a
\end{equation}
が成り立つ.
また定理 \ref{sthmsingletonsubset}と推論法則 \ref{dedequiv}により
\[
  \{b\} \subset a \to b \in a
\]
が成り立つ.
ここで$x$が$a$の中に自由変数として現れないことから, 
代入法則 \ref{substfree}, \ref{substfund}により, この記号列は
\begin{equation}
\label{sthm=singletonex!4}
  \{b\} \subset a \to (b|x)(x \in a)
\end{equation}
と一致する.
故にこれが成り立つ.
またschema S4の適用により
\begin{equation}
\label{sthm=singletonex!5}
  (b|x)(x \in a) \to \exists x(x \in a)
\end{equation}
が成り立つ.
そこで(\ref{sthm=singletonex!3})---(\ref{sthm=singletonex!5})から, 推論法則 \ref{dedmmp}によって
\begin{equation}
\label{sthm=singletonex!6}
  a = \{b\} \to \exists x(x \in a)
\end{equation}
が成り立つことがわかる.
また定理 \ref{sthm=tsubset}より
\begin{equation}
\label{sthm=singletonex!7}
  a = \{b\} \to a \subset \{b\}
\end{equation}
が成り立つ.
また$x$が$a$の中に自由変数として現れないことから, 定理 \ref{sthmsubsetsingleton!}より
\begin{equation}
\label{sthm=singletonex!8}
  a \subset \{b\} \to \ !x(x \in a)
\end{equation}
が成り立つ.
そこで(\ref{sthm=singletonex!7}), (\ref{sthm=singletonex!8})から, 推論法則 \ref{dedmmp}によって
\begin{equation}
\label{sthm=singletonex!9}
  a = \{b\} \to \ !x(x \in a)
\end{equation}
が成り立つ.
故に(\ref{sthm=singletonex!6}), (\ref{sthm=singletonex!9})から, 
推論法則 \ref{dedprewedge}により(\ref{sthm=singletonex!1})が成り立つ.
(\ref{sthm=singletonex!2})が成り立つことは, 
(\ref{sthm=singletonex!1})と推論法則 \ref{dedmp}によって明らかである.
\halmos




\mathstrut
\begin{thm}
\label{sthmex!singleton}%定理4.24%新規%確認済
$a$を集合とし, $x$を$a$の中に自由変数として現れない文字とする.
このとき
\begin{equation}
\label{sthmex!singleton1}
  \exists !x(x \in \{a\})
\end{equation}
が成り立つ.
\end{thm}


\noindent{\bf 証明}
~このとき変数法則 \ref{valnset}により, $x$は$\{a\}$の中に自由変数として現れない.
またThm \ref{x=x}より$\{a\} = \{a\}$が成り立つ.
故に定理 \ref{sthm=singletonex!}より(\ref{sthmex!singleton1})が成り立つ.
\halmos




\mathstrut
\begin{thm}
\label{sthmsma!}%定理4.25%新規%確認済
$a$を集合, $R$を関係式とし, $x$を文字とする.
このとき
\begin{equation}
\label{sthmsma!1}
  {\rm Set}_{x}(R) \wedge \{x \mid R\} \subset \{a\} \to \ !x(R)
\end{equation}
が成り立つ.
またこのことから, 次の(\ref{sthmsma!2})が成り立つ.
\begin{equation}
\label{sthmsma!2}
  R \text{が} x \text{について集合を作り得るとする.} ~
  \text{このとき} \{x \mid R\} \subset \{a\} \text{ならば,} ~!x(R).
\end{equation}
\end{thm}


\noindent{\bf 証明}
~${\rm Set}_{x}(R)$は$\forall x(x \in \{x \mid R\} \leftrightarrow R)$と同じだから, 
Thm \ref{thmquanwedge}より
\begin{equation}
\label{sthmsma!3}
  {\rm Set}_{x}(R) \to \forall x(R \to x \in \{x \mid R\})
\end{equation}
が成り立つ.
またThm \ref{thmallt!sep}より
\begin{equation}
\label{sthmsma!4}
  \forall x(R \to x \in \{x \mid R\}) \to (!x(x \in \{x \mid R\}) \to \ !x(R))
\end{equation}
が成り立つ.
また変数法則 \ref{valiset}により, $x$は$\{x \mid R\}$の中に自由変数として現れないから, 
定理 \ref{sthmsubsetsingleton!}より
\[
  \{x \mid R\} \subset \{a\} \to \ !x(x \in \{x \mid R\})
\]
が成り立つ.
故に推論法則 \ref{dedaddf}により
\begin{equation}
\label{sthmsma!5}
  (!x(x \in \{x \mid R\}) \to \ !x(R)) \to (\{x \mid R\} \subset \{a\} \to \ !x(R))
\end{equation}
が成り立つ.
そこで(\ref{sthmsma!3})---(\ref{sthmsma!5})から, 推論法則 \ref{dedmmp}によって
\[
  {\rm Set}_{x}(R) \to (\{x \mid R\} \subset \{a\} \to \ !x(R))
\]
が成り立つことがわかる.
故に推論法則 \ref{dedtwch}により(\ref{sthmsma!1})が成り立つ.
(\ref{sthmsma!2})が成り立つことは, 
(\ref{sthmsma!1})と推論法則 \ref{dedmp}, \ref{dedwedge}によって明らかである.
\halmos




\mathstrut
\begin{thm}
\label{sthmsmaex!}%定理4.26%新規%確認済
$a$を集合, $R$を関係式とし, $x$を文字とする.
このとき
\begin{equation}
\label{sthmsmaex!1}
  {\rm Set}_{x}(R) \wedge \{x \mid R\} = \{a\} \to \exists !x(R)
\end{equation}
が成り立つ.
またこのことから, 次の(\ref{sthmsmaex!2})が成り立つ.
\begin{equation}
\label{sthmsmaex!2}
  R \text{が} x \text{について集合を作り得るとする.} ~
  \text{このとき} \{x \mid R\} = \{a\} \text{ならば,} ~\exists !x(R).
\end{equation}
\end{thm}


\noindent{\bf 証明}
~${\rm Set}_{x}(R)$は$\forall x(x \in \{x \mid R\} \leftrightarrow R)$と同じだから, 
Thm \ref{thmalleqex!sep}と推論法則 \ref{dedpreequiv}により
\begin{equation}
\label{sthmsmaex!3}
  {\rm Set}_{x}(R) \to (\exists !x(x \in \{x \mid R\}) \to \exists !x(R))
\end{equation}
が成り立つ.
また変数法則 \ref{valiset}により, $x$は$\{x \mid R\}$の中に自由変数として現れないから, 
定理 \ref{sthm=singletonex!}より
\[
  \{x \mid R\} = \{a\} \to \exists !x(x \in \{x \mid R\})
\]
が成り立つ.
故に推論法則 \ref{dedaddf}により
\begin{equation}
\label{sthmsmaex!4}
  (\exists !x(x \in \{x \mid R\}) \to \exists !x(R)) \to (\{x \mid R\} = \{a\} \to \exists !x(R))
\end{equation}
が成り立つ.
そこで(\ref{sthmsmaex!3}), (\ref{sthmsmaex!4})から, 推論法則 \ref{dedmmp}によって
\[
  {\rm Set}_{x}(R) \to (\{x \mid R\} = \{a\} \to \exists !x(R))
\]
が成り立つ.
故に推論法則 \ref{dedtwch}により(\ref{sthmsmaex!1})が成り立つ.
(\ref{sthmsmaex!2})が成り立つことは, 
(\ref{sthmsmaex!1})と推論法則 \ref{dedmp}, \ref{dedwedge}によって明らかである.
\halmos




\mathstrut
\begin{thm}
\label{sthmex!sm}%定理4.27%確認済
$R$を関係式とし, $x$を文字とするとき, 
\begin{equation}
\label{sthmex!sm1}
  \exists !x(R) \leftrightarrow {\rm Set}_{x}(R) \wedge \{x \mid R\} = \{\tau_{x}(R)\}
\end{equation}
が成り立つ.
またこのことから, 次の1), 2)が成り立つ.

1)
$\exists !x(R)$ならば, $R$は$x$について集合を作り得る.
更に$\{x \mid R\} = \{\tau_{x}(R)\}$が成り立つ.

2)
$R$が$x$について集合を作り得るとする.
このとき$\{x \mid R\} = \{\tau_{x}(R)\}$ならば, $\exists !x(R)$.
\end{thm}


\noindent{\bf 証明}
~Thm \ref{thmex!lall}より
\begin{equation}
\label{sthmex!sm2}
  \exists !x(R) \leftrightarrow \forall x(R \leftrightarrow x = \tau_{x}(R))
\end{equation}
が成り立つ.
またThm \ref{thmquanwch}より
\begin{equation}
\label{sthmex!sm3}
  \forall x(R \leftrightarrow x = \tau_{x}(R)) 
  \leftrightarrow \forall x(x = \tau_{x}(R) \leftrightarrow R)
\end{equation}
が成り立つ.
また$y$を$x$と異なり, $R$の中に自由変数として現れない, 定数でない文字とするとき, 
定理 \ref{sthmsingletonbasis}と推論法則 \ref{dedeqch}により
\[
  y = \tau_{x}(R) \leftrightarrow y \in \{\tau_{x}(R)\}
\]
が成り立つから, 推論法則 \ref{dedaddeqeq}により
\[
  (y = \tau_{x}(R) \leftrightarrow (y|x)(R)) 
  \leftrightarrow (y \in \{\tau_{x}(R)\} \leftrightarrow (y|x)(R))
\]
が成り立つ.
故にこれと$y$が定数でないことから, 推論法則 \ref{dedalleqquansepconst}により
\[
  \forall y(y = \tau_{x}(R) \leftrightarrow (y|x)(R)) 
  \leftrightarrow \forall y(y \in \{\tau_{x}(R)\} \leftrightarrow (y|x)(R))
\]
が成り立つ.
ここで変数法則 \ref{valtau}, \ref{valnset}により, 
$x$は$\tau_{x}(R)$及び$\{\tau_{x}(R)\}$の中に自由変数として現れないから, 
代入法則 \ref{substfree}, \ref{substfund}, \ref{substequiv}によってわかるように, 上記の記号列は
\[
  \forall y((y|x)(x = \tau_{x}(R) \leftrightarrow R)) 
  \leftrightarrow \forall y((y|x)(x \in \{\tau_{x}(R)\} \leftrightarrow R))
\]
と一致する.
また$y$が$x$と異なり, $R$の中に自由変数として現れないことから, 
変数法則 \ref{valfund}, \ref{valtau}, \ref{valequiv}, \ref{valnset}によってわかるように, 
$y$は$x = \tau_{x}(R) \leftrightarrow R$及び$x \in \{\tau_{x}(R)\} \leftrightarrow R$の中に
自由変数として現れない.
故に代入法則 \ref{substquantrans}により, 上記の記号列は
\begin{equation}
\label{sthmex!sm4}
  \forall x(x = \tau_{x}(R) \leftrightarrow R) 
  \leftrightarrow \forall x(x \in \{\tau_{x}(R)\} \leftrightarrow R)
\end{equation}
と一致する.
従ってこれが成り立つ.
また上述のように$x$は$\{\tau_{x}(R)\}$の中に自由変数として現れないから, 
定理 \ref{sthmsmbasis&iset=a}より
\begin{equation}
\label{sthmex!sm5}
  \forall x(x \in \{\tau_{x}(R)\} \leftrightarrow R) 
  \leftrightarrow {\rm Set}_{x}(R) \wedge \{x \mid R\} = \{\tau_{x}(R)\}
\end{equation}
が成り立つ.
そこで(\ref{sthmex!sm2})---(\ref{sthmex!sm5})から, 
推論法則 \ref{dedeqtrans}によって(\ref{sthmex!sm1})が成り立つことがわかる.
1), 2)が成り立つことは(\ref{sthmex!sm1})と推論法則 \ref{dedwedge}, \ref{dedeqfund}によって明らかである.
\halmos




\mathstrut
\begin{thm}
\label{sthmspinsingleton}%定理4.28%新規%確認済
$a$を集合, $R$を関係式とし, $x$を$a$の中に自由変数として現れない文字とする.
このとき
\begin{align}
  \label{sthmspinsingleton1}
  &(\exists x \in \{a\})(R) \leftrightarrow (a|x)(R), \\
  \mbox{} \notag \\
  \label{sthmspinsingleton2}
  &(\forall x \in \{a\})(R) \leftrightarrow (a|x)(R), \\
  \mbox{} \notag \\
  \label{sthmspinsingleton3}
  &(!x \in \{a\})(R), \\
  \mbox{} \notag \\
  \label{sthmspinsingleton4}
  &(\exists !x \in \{a\})(R) \leftrightarrow (a|x)(R)
\end{align}
がすべて成り立つ.
またこれらから, 次の1), 2)が成り立つ.

1)
$(\exists x \in \{a\})(R)$, $(\forall x \in \{a\})(R)$, $(\exists !x \in \{a\})(R)$の
いずれかが成り立てば, $(a|x)(R)$.

2)
$(a|x)(R)$ならば, 
$(\exists x \in \{a\})(R)$, $(\forall x \in \{a\})(R)$, $(\exists !x \in \{a\})(R)$がすべて成り立つ.
\end{thm}


\noindent{\bf 証明}
~まず(\ref{sthmspinsingleton1}), (\ref{sthmspinsingleton2})が共に成り立つことを示す.
$x$が$a$の中に自由変数として現れないことから, 変数法則 \ref{valnset}により, 
$x$は$\{a\}$及び$\{a, a\}$の中に自由変数として現れない.
また定理 \ref{sthmaa=a}と推論法則 \ref{ded=ch}により$\{a\} = \{a, a\}$が成り立つ.
そこで定理 \ref{sthmspin=}より
\begin{align}
  \label{sthmspinsingleton5}
  &(\exists x \in \{a\})(R) \leftrightarrow (\exists x \in \{a, a\})(R), \\
  \mbox{} \notag \\
  \label{sthmspinsingleton6}
  &(\forall x \in \{a\})(R) \leftrightarrow (\forall x \in \{a, a\})(R)
\end{align}
が共に成り立つ.
また$x$が$a$の中に自由変数として現れないことから, 定理 \ref{sthmspinuopair}より
\begin{align}
  \label{sthmspinsingleton7}
  &(\exists x \in \{a, a\})(R) \leftrightarrow (a|x)(R) \vee (a|x)(R), \\
  \mbox{} \notag \\
  \label{sthmspinsingleton8}
  &(\forall x \in \{a, a\})(R) \leftrightarrow (a|x)(R) \wedge (a|x)(R)
\end{align}
が共に成り立つ.
またThm \ref{avala}より
\begin{equation}
\label{sthmspinsingleton9}
  (a|x)(R) \vee (a|x)(R) \leftrightarrow (a|x)(R)
\end{equation}
が成り立ち, Thm \ref{awala}より
\begin{equation}
\label{sthmspinsingleton10}
  (a|x)(R) \wedge (a|x)(R) \leftrightarrow (a|x)(R)
\end{equation}
が成り立つ.
そこで(\ref{sthmspinsingleton5}), (\ref{sthmspinsingleton7}), (\ref{sthmspinsingleton9})から, 
推論法則 \ref{dedeqtrans}によって(\ref{sthmspinsingleton1})が成り立つことがわかる.
また(\ref{sthmspinsingleton6}), (\ref{sthmspinsingleton8}), (\ref{sthmspinsingleton10})から, 
同じく推論法則 \ref{dedeqtrans}によって(\ref{sthmspinsingleton2})が成り立つことがわかる.

次に(\ref{sthmspinsingleton3})が成り立つことを示す.
$x$が$a$の中に自由変数として現れないことから, 
定理 \ref{sthmex!singleton}より$\exists !x(x \in \{a\})$が成り立つ.
故に推論法則 \ref{dedwedge}により$!x(x \in \{a\})$が成り立つ.
故に推論法則 \ref{ded!tsp!}により(\ref{sthmspinsingleton3})が成り立つ.

最後に(\ref{sthmspinsingleton4})が成り立つことを示す.
(\ref{sthmspinsingleton3})が成り立つことから, 推論法則 \ref{dedawblatrue2}により
\[
  (\exists !x \in \{a\})(R) \leftrightarrow (\exists x \in \{a\})(R)
\]
が成り立つ.
そこでこれと(\ref{sthmspinsingleton1})から, 推論法則 \ref{dedeqtrans}によって
(\ref{sthmspinsingleton4})が成り立つ.

1), 2)が成り立つことは, 
(\ref{sthmspinsingleton1}), (\ref{sthmspinsingleton2}), (\ref{sthmspinsingleton4})と
推論法則 \ref{dedeqfund}によって明らかである.
\halmos




\mathstrut%確認済%koko
ここで, 以後しばしば用いることになる記号列の省略記法を導入しておく.




\mathstrut
\begin{defo}
\label{element}%変形16%新規%確認済
$\mathscr{T}$を特殊記号として$\in$を持つ理論とし, $a$を$\mathscr{T}$の記号列とする.
また$x$と$y$を共に$a$の中に自由変数として現れない文字とする.
このとき
\[
  \tau_{x}(x \in a) \equiv \tau_{y}(y \in a)
\]
が成り立つ.
\end{defo}


\noindent{\bf 証明}
~$x$と$y$が同じ文字ならば明らかだから, 以下$x$と$y$は異なる文字であるとする.
このとき$y$が$x$と異なり, $a$の中に自由変数として現れないことから, 
変数法則 \ref{valfund}により, $y$は$x \in a$の中に自由変数として現れない.
故に代入法則 \ref{substtautrans}により
\[
  \tau_{x}(x \in a) \equiv \tau_{y}((y|x)(x \in a))
\]
が成り立つ.
また$x$が$a$の中に自由変数として現れないことから, 代入法則 \ref{substfree}, \ref{substfund}により
\[
  (y|x)(x \in a) \equiv y \in a
\]
が成り立つ.
故に本法則が成り立つ.
\halmos




\mathstrut
\begin{defi}
\label{defelm}%定義3%新規%確認済
$\mathscr{T}$を特殊記号として$\in$を持つ理論とし, $a$を$\mathscr{T}$の記号列とする.
また$x$と$y$を共に$a$の中に自由変数として現れない文字とする.
このとき変形法則 \ref{element}によれば, $\tau_{x}(x \in a)$と$\tau_{y}(y \in a)$は同じ記号列となる.
$a$に対して定まるこの記号列を, ${\rm elm}(a)$と書き表す.
\end{defi}




\mathstrut%確認済%koko
以下の変数法則 \ref{valelm}, 一般代入法則 \ref{gsubstelm}, 代入法則 \ref{substelm}, 
構成法則 \ref{formelm}では, $\mathscr{T}$を特殊記号として$\in$を持つ理論とし, 
これらの法則における``記号列'', ``集合''とは, 
それぞれ$\mathscr{T}$の記号列, $\mathscr{T}$の対象式のこととする.




\mathstrut
\begin{valu}
\label{valelm}%変数26%新規%確認済
$a$を記号列とし, $x$を文字とする.
$x$が$a$の中に自由変数として現れなければ, $x$は${\rm elm}(a)$の中に自由変数として現れない.
\end{valu}


\noindent{\bf 証明}
~このとき定義から${\rm elm}(a)$は$\tau_{x}(x \in a)$と同じである.
変数法則 \ref{valtau}によれば, $x$はこの中に自由変数として現れない.
\halmos




\mathstrut
\begin{gsub}
\label{gsubstelm}%一般代入30%新規%確認済
$a$を記号列とする.
また$n$を自然数とし, $T_{1}, T_{2}, \cdots, T_{n}$を記号列とする.
また$x_{1}, x_{2}, \cdots, x_{n}$を, どの二つも互いに異なる文字とする.
このとき
\[
  (T_{1}|x_{1}, T_{2}|x_{2}, \cdots, T_{n}|x_{n})({\rm elm}(a)) 
  \equiv {\rm elm}((T_{1}|x_{1}, T_{2}|x_{2}, \cdots, T_{n}|x_{n})(a))
\]
が成り立つ.
\end{gsub}


\noindent{\bf 証明}
~$y$を$x_{1}, x_{2}, \cdots, x_{n}$のいずれとも異なり, 
$a, T_{1}, T_{2}, \cdots, T_{n}$のいずれの中にも自由変数として現れない文字とする.
このとき定義から${\rm elm}(a)$は$\tau_{y}(y \in a)$と同じだから, 
\begin{equation}
\label{gsubstelm1}
  (T_{1}|x_{1}, T_{2}|x_{2}, \cdots, T_{n}|x_{n})({\rm elm}(a)) 
  \equiv (T_{1}|x_{1}, T_{2}|x_{2}, \cdots, T_{n}|x_{n})(\tau_{y}(y \in a))
\end{equation}
である.
また$y$が$x_{1}, x_{2}, \cdots, x_{n}$のいずれとも異なり, 
$T_{1}, T_{2}, \cdots, T_{n}$のいずれの中にも自由変数として現れないことから, 
一般代入法則 \ref{gsubsttau}により
\begin{equation}
\label{gsubstelm2}
  (T_{1}|x_{1}, T_{2}|x_{2}, \cdots, T_{n}|x_{n})(\tau_{y}(y \in a)) 
  \equiv \tau_{y}((T_{1}|x_{1}, T_{2}|x_{2}, \cdots, T_{n}|x_{n})(y \in a))
\end{equation}
が成り立つ.
また$y$が$x_{1}, x_{2}, \cdots, x_{n}$のいずれとも異なることと一般代入法則 \ref{gsubstfund}により, 
\begin{equation}
\label{gsubstelm3}
  (T_{1}|x_{1}, T_{2}|x_{2}, \cdots, T_{n}|x_{n})(y \in a) 
  \equiv y \in (T_{1}|x_{1}, T_{2}|x_{2}, \cdots, T_{n}|x_{n})(a)
\end{equation}
が成り立つ.
そこで(\ref{gsubstelm1})---(\ref{gsubstelm3})からわかるように, 
$(T_{1}|x_{1}, T_{2}|x_{2}, \cdots, T_{n}|x_{n})({\rm elm}(a))$は
\begin{equation}
\label{gsubstelm4}
  \tau_{y}(y \in (T_{1}|x_{1}, T_{2}|x_{2}, \cdots, T_{n}|x_{n})(a))
\end{equation}
と一致する.
ここで$y$が$a, T_{1}, T_{2}, \cdots, T_{n}$のいずれの中にも自由変数として現れないことから, 
変数法則 \ref{valgsubst}により, 
$y$は$(T_{1}|x_{1}, T_{2}|x_{2}, \cdots, T_{n}|x_{n})(a)$の中に自由変数として現れない.
故に定義から, (\ref{gsubstelm4})は
${\rm elm}((T_{1}|x_{1}, T_{2}|x_{2}, \cdots, T_{n}|x_{n})(a))$と同じである.
故に本法則が成り立つ.
\halmos




\mathstrut
\begin{subs}
\label{substelm}%代入35%新規%確認済
$a$と$T$を記号列とし, $x$を文字とする.
このとき
\[
  (T|x)({\rm elm}(a)) \equiv {\rm elm}((T|x)(a))
\]
が成り立つ.
\end{subs}


\noindent{\bf 証明}
~一般代入法則 \ref{gsubstelm}において, $n$が$1$の場合である.
\halmos




\mathstrut
\begin{form}
\label{formelm}%構成43%新規%確認済
$a$が集合ならば, ${\rm elm}(a)$は集合である.
\end{form}


\noindent{\bf 証明}
~$x$を$a$の中に自由変数として現れない文字とするとき, ${\rm elm}(a)$は$\tau_{x}(x \in a)$である.
$a$が集合ならば, 構成法則 \ref{formfund}によって明らかなように, これは集合である.
\halmos




\mathstrut%註%確認済
{\small
\noindent
\textbf{註.} 
$a$を記号列とし, $x$を$a$の中に自由変数として現れない文字とするとき, 
\begin{equation}
\label{ex&elm}
  \exists x(x \in a) \equiv {\rm elm}(a) \in a
\end{equation}
が成り立つ.

実際$\exists x(x \in a)$は$(\tau_{x}(x \in a)|x)(x \in a)$であるが, 
$x$が$a$の中に自由変数として現れないことから, 
${\rm elm}(a)$の定義と代入法則 \ref{substfree}, \ref{substfund}により, 
これは${\rm elm}(a) \in a$と一致する.

(\ref{ex&elm})は以後特に断りなく引用する.
}




\mathstrut
\begin{thm}
\label{sthmelmbasis}%定理4.29%新規%確認済
$a$と$b$を集合とするとき, 
\begin{equation}
\label{sthmelmbasis1}
  b \in a \to {\rm elm}(a) \in a, ~~
  {\rm elm}(a) \notin a \to b \notin a
\end{equation}
が成り立つ.
またこのことから, 次の1), 2)が成り立つ.

1)
$b \in a$ならば, ${\rm elm}(a) \in a$.

2)
${\rm elm}(a) \notin a$ならば, $b \notin a$.
\end{thm}


\noindent{\bf 証明}
~$x$を$a$の中に自由変数として現れない文字とするとき, schema S4の適用により
\[
  (b|x)(x \in a) \to \exists x(x \in a), 
\]
即ち
\[
  (b|x)(x \in a) \to {\rm elm}(a) \in a
\]
が成り立つが, 代入法則 \ref{substfree}, \ref{substfund}によればこの記号列は
\[
  b \in a \to {\rm elm}(a) \in a
\]
と一致するから, これが成り立つ.
故に推論法則 \ref{dedcp}により
\[
  {\rm elm}(a) \notin a \to b \notin a
\]
も成り立つ.
1), 2)が成り立つことは, (\ref{sthmelmbasis1})と推論法則 \ref{dedmp}によって明らかである.
\halmos




\mathstrut
\begin{thm}
\label{sthmelmnotin}%定理4.30%新規%確認済
$a$を集合とし, $x$を$a$の中に自由変数として現れない文字とする.
このとき
\begin{equation}
\label{sthmelmnotin1}
  {\rm elm}(a) \notin a \leftrightarrow \forall x(x \notin a)
\end{equation}
が成り立つ.
またこのことから, 次の1), 2), 3)が成り立つ.

1)
${\rm elm}(a) \notin a$ならば, $\forall x(x \notin a)$.

2)
$\forall x(x \notin a)$ならば, ${\rm elm}(a) \notin a$.

3)
$x$が定数でなく, $x \notin a$が成り立てば, ${\rm elm}(a) \notin a$.
\end{thm}


\noindent{\bf 証明}
~Thm \ref{thmeaquandm}より
\[
  \neg \exists x(x \in a) \leftrightarrow \forall x(x \notin a)
\]
が成り立つが, $x$が$a$の中に自由変数として現れないことより
この記号列は(\ref{sthmelmnotin1})と同じだから, (\ref{sthmelmnotin1})が成り立つ.

\noindent
1), 2)
(\ref{sthmelmnotin1})と推論法則 \ref{dedeqfund}によって明らか.

\noindent
3)
2)と推論法則 \ref{dedltthmquan}によって明らか.
\halmos




\mathstrut
\begin{thm}
\label{sthmelm=}%定理4.31%新規%確認済
$a$と$b$を集合とするとき, 
\begin{equation}
\label{sthmelm=1}
  a = b \to {\rm elm}(a) = {\rm elm}(b)
\end{equation}
が成り立つ.
またこのことから, 次の(\ref{sthmelm=2})が成り立つ.
\begin{equation}
\label{sthmelm=2}
  a = b \text{ならば,} ~{\rm elm}(a) = {\rm elm}(b).
\end{equation}
\end{thm}


\noindent{\bf 証明}
~$x$を文字とするとき, Thm \ref{T=Ut1TV=UV1}より
\[
  a = b \to (a|x)({\rm elm}(x)) = (b|x)({\rm elm}(x))
\]
が成り立つが, 代入法則 \ref{substelm}によればこの記号列は(\ref{sthmelm=1})と一致するから, 
(\ref{sthmelm=1})が成り立つ.
(\ref{sthmelm=2})が成り立つことは, (\ref{sthmelm=1})と推論法則 \ref{dedmp}によって明らかである.
\halmos




\mathstrut
\begin{thm}
\label{sthmisetelm}%定理4.32%新規%確認済
$R$を関係式とし, $x$を文字とする.
このとき
\begin{equation}
\label{sthmisetelm1}
  {\rm Set}_{x}(R) \to {\rm elm}(\{x \mid R\}) = \tau_{x}(R)
\end{equation}
が成り立つ.
またこのことから, 次の(\ref{sthmisetelm2})が成り立つ.
\begin{equation}
\label{sthmisetelm2}
  R \text{が} x \text{について集合を作り得るならば,} ~{\rm elm}(\{x \mid R\}) = \tau_{x}(R).
\end{equation}
\end{thm}


\noindent{\bf 証明}
~このとき${\rm Set}_{x}(R)$は$\forall x(x \in \{x \mid R\} \leftrightarrow R)$と同じだから, 
schema S6の適用により
\[
  {\rm Set}_{x}(R) \to \tau_{x}(x \in \{x \mid R\}) = \tau_{x}(R)
\]
が成り立つ.
ここで変数法則 \ref{valiset}により, $x$は$\{x \mid R\}$の中に自由変数として現れないから, 
定義よりこの記号列は(\ref{sthmisetelm1})と同じである.
故に(\ref{sthmisetelm1})が成り立つ.
(\ref{sthmisetelm2})が成り立つことは, (\ref{sthmisetelm1})と推論法則 \ref{dedmp}によって明らかである.
\halmos




\mathstrut
\begin{thm}
\label{sthmuopairelm}%定理4.33%新規%確認済
$a$と$b$を集合とするとき, 
\begin{equation}
\label{sthmuopairelm1}
  {\rm elm}(\{a, b\}) = a \vee {\rm elm}(\{a, b\}) = b
\end{equation}
が成り立つ.
\end{thm}


\noindent{\bf 証明}
~定理 \ref{sthmuopairfund}より$a \in \{a, b\}$が成り立つから, 
定理 \ref{sthmelmbasis}より${\rm elm}(\{a, b\}) \in \{a, b\}$が成り立つ.
故に定理 \ref{sthmuopairbasis}より(\ref{sthmuopairelm1})が成り立つ.
\halmos




\mathstrut
\begin{thm}
\label{sthmsingletonelm}%定理4.34%新規%確認済
$a$を集合とするとき, 
\begin{equation}
\label{sthmsingletonelm1}
  {\rm elm}(\{a\}) = a
\end{equation}
が成り立つ.
\end{thm}


\noindent{\bf 証明}
~定理 \ref{sthmsingletonfund}より$a \in \{a\}$が成り立つから, 
定理 \ref{sthmelmbasis}より${\rm elm}(\{a\}) \in \{a\}$が成り立つ.
故に定理 \ref{sthmsingletonbasis}より(\ref{sthmsingletonelm1})が成り立つ.
\halmos




\mathstrut
\begin{thm}
\label{sthm!elm}%定理4.35%新規%確認済
$a$を集合とし, $x$を$a$の中に自由変数として現れない文字とする.
このとき
\begin{equation}
\label{sthm!elm1}
  !x(x \in a) \leftrightarrow a \subset \{{\rm elm}(a)\}
\end{equation}
が成り立つ.
またこのことから, 次の(\ref{sthm!elm2})が成り立つ.
\begin{equation}
\label{sthm!elm2}
  !x(x \in a) \text{ならば,} ~a \subset \{{\rm elm}(a)\}. ~
  \text{また} a \subset \{{\rm elm}(a)\} \text{ならば,} ~!x(x \in a).
\end{equation}
\end{thm}


\noindent{\bf 証明}
~$y$を$x$と異なり, $a$の中に自由変数として現れない, 定数でない文字とする.
このときThm \ref{thm!lall}より
\begin{equation}
\label{sthm!elm3}
  !y(y \in a) \leftrightarrow \forall y(y \in a \to y = \tau_{y}(y \in a))
\end{equation}
が成り立つ.
ここで$x$が$y$と異なり, $a$の中に自由変数として現れないことから, 
変数法則 \ref{valfund}により$x$は$y \in a$の中に自由変数として現れないから, 
代入法則 \ref{subst!trans}により$!y(y \in a)$は$!x((x|y)(y \in a))$と一致する.
また$y$が$a$の中に自由変数として現れないことと代入法則 \ref{substfree}, \ref{substfund}により, 
$!x((x|y)(y \in a))$は$!x(x \in a)$と一致する.
また$y$が$a$の中に自由変数として現れないことから, $\tau_{y}(y \in a)$は${\rm elm}(a)$と同じである.
故にこれらのことから, (\ref{sthm!elm3})は
\begin{equation}
\label{sthm!elm4}
  !x(x \in a) \leftrightarrow \forall y(y \in a \to y = {\rm elm}(a))
\end{equation}
と一致する.
従ってこれが成り立つ.
また定理 \ref{sthmsingletonbasis}と推論法則 \ref{dedeqch}により
\[
  y = {\rm elm}(a) \leftrightarrow y \in \{{\rm elm}(a)\}
\]
が成り立つから, 推論法則 \ref{dedaddeqt}により
\[
  (y \in a \to y = {\rm elm}(a)) \leftrightarrow (y \in a \to y \in \{{\rm elm}(a)\})
\]
が成り立つ.
これと$y$が定数でないことから, 推論法則 \ref{dedalleqquansepconst}により
\[
  \forall y(y \in a \to y = {\rm elm}(a)) \leftrightarrow \forall y(y \in a \to y \in \{{\rm elm}(a)\})
\]
が成り立つ.
ここで$y$が$a$の中に自由変数として現れないことから, 変数法則 \ref{valnset}, \ref{valelm}により, 
$y$は$\{{\rm elm}(a)\}$の中にも自由変数として現れない.
故に上記の記号列は
\begin{equation}
\label{sthm!elm5}
  \forall y(y \in a \to y = {\rm elm}(a)) \leftrightarrow a \subset \{{\rm elm}(a)\}
\end{equation}
と同じである.
従ってこれが成り立つ.
そこで(\ref{sthm!elm4}), (\ref{sthm!elm5})から, 
推論法則 \ref{dedeqtrans}によって(\ref{sthm!elm1})が成り立つ.
(\ref{sthm!elm2})が成り立つことは, (\ref{sthm!elm1})と推論法則 \ref{dedeqfund}によって明らかである.
\halmos




\mathstrut
\begin{thm}
\label{sthmex!elm}%定理4.36%新規%確認済
$a$を集合とし, $x$を$a$の中に自由変数として現れない文字とする.
このとき
\begin{equation}
\label{sthmex!elm1}
  \exists !x(x \in a) \leftrightarrow a = \{{\rm elm}(a)\}
\end{equation}
が成り立つ.
またこのことから, 次の(\ref{sthmex!elm2})が成り立つ.
\begin{equation}
\label{sthmex!elm2}
  \exists !x(x \in a) \text{ならば,} ~a = \{{\rm elm}(a)\}. ~
  \text{また} a = \{{\rm elm}(a)\} \text{ならば,} ~\exists !x(x \in a).
\end{equation}
\end{thm}


\noindent{\bf 証明}
~定理 \ref{sthmex!sm}より
\[
  \exists !x(x \in a) 
  \leftrightarrow {\rm Set}_{x}(x \in a) \wedge \{x \mid x \in a\} = \{\tau_{x}(x \in a)\}
\]
が成り立つ.
ここで$x$が$a$の中に自由変数として現れないことから, この記号列は
\begin{equation}
\label{sthmex!elm3}
  \exists !x(x \in a) 
  \leftrightarrow {\rm Set}_{x}(x \in a) \wedge \{x \mid x \in a\} = \{{\rm elm}(a)\}
\end{equation}
と一致する.
故にこれが成り立つ.
また$x$が$a$の中に自由変数として現れないことから, \S3例1より$x \in a$は$x$について集合を作り得るから, 
推論法則 \ref{dedawblatrue2}により
\begin{equation}
\label{sthmex!elm4}
  {\rm Set}_{x}(x \in a) \wedge \{x \mid x \in a\} = \{{\rm elm}(a)\} 
  \leftrightarrow \{x \mid x \in a\} = \{{\rm elm}(a)\}
\end{equation}
が成り立つ.
また$x$が$a$の中に自由変数として現れないことから, \S3例3より$\{x \mid x \in a\} = a$が成り立つから, 
推論法則 \ref{dedaddeq=}により
\begin{equation}
\label{sthmex!elm5}
  \{x \mid x \in a\} = \{{\rm elm}(a)\} \leftrightarrow a = \{{\rm elm}(a)\}
\end{equation}
が成り立つ.
そこで(\ref{sthmex!elm3})---(\ref{sthmex!elm5})から, 
推論法則 \ref{dedeqtrans}によって(\ref{sthmex!elm1})が成り立つことがわかる.
(\ref{sthmex!elm2})が成り立つことは, (\ref{sthmex!elm1})と推論法則 \ref{dedeqfund}によって明らかである.
\halmos
%ここまで確認



\newpage
\setcounter{defi}{0}
\section{分出と合併のschema}



%確認済%koko
この節では, 集合論のschemaの一つである分出と合併のschemaを導入する.
本節以降で見ていくように, これによって多くの関係式が集合を作り得ることが保証される.




\mathstrut%確認済%koko
$\mathscr{T}$を特殊記号として$\in$を持つ理論とする.
$\mathscr{T}$において, 次の規則S7を考える: 

\mathstrut%確認済
S7. $R$を関係式とし, $x$と$y$を異なる文字とする.
    また$u$と$v$を, 共に$x$及び$y$と異なり, $R$の中に自由変数として現れない文字とする.
    これらから, 記号列
    \begin{equation}
    \label{s7}
      \forall y(\exists u(\forall x(R \to x \in u))) 
      \to \forall v({\rm Set}_{x}(\exists y(y \in v \wedge R)))
    \end{equation}
    を得る.

\mathstrut%確認済
これがschemaの特性a), b) (第I章定義 \ref{defaxiom}) を持つことを確かめる.

まず関係式$R$及び文字$x$, $y$, $u$, $v$に対し, (\ref{s7})が関係式となることは, 
構成法則 \ref{formfund}, \ref{formwedge}, \ref{formquan}, \ref{formsm}によって容易にわかる.
故にS7はschemaの特性a)を持つ.

次にS7がschemaの特性b)を持つことを確かめる.
いま$S$をS7の適用によって得られる記号列とする.
また$T$を対象式とし, $z$を文字とする.
このとき上記の仮定を満たす関係式$R$及び文字$x$, $y$, $u$, $v$があり, 
$S$は(\ref{s7})である.
故にいま$\forall y(\exists u(\forall x(R \to x \in u)))$, 
$\forall v({\rm Set}_{x}(\exists y(y \in v \wedge R)))$をそれぞれ$S_{1}$, $S_{2}$と書けば, 
\[
  S \equiv S_{1} \to S_{2}
\]
である.
従って代入法則 \ref{substfund}により
\begin{equation}
\label{s71}
  (T|z)(S) \equiv (T|z)(S_{1}) \to (T|z)(S_{2})
\end{equation}
が成り立つ.
さていま$p$, $q$, $r$, $s$を, どの二つも互いに異なる文字で, 
いずれも$x$, $y$, $z$, $u$, $v$と異なり, $R$及び$T$の中に自由変数として現れないものとする.
また$(p|x)((q|y)(R))$を$R^{*}$と書く.
このとき構成法則 \ref{formsubstlett}より$R^{*}$は関係式である.
また
\begin{align}
  \label{s72}
  &S_{1} \equiv \forall q(\exists r(\forall p(R^{*} \to p \in r))), \\
  \mbox{} \notag \\
  \label{s73}
  &S_{2} \equiv \forall s({\rm Set}_{p}(\exists q(q \in s \wedge R^{*})))
\end{align}
が共に成り立つ.
これらの証明はやや長いので後に回し, 先にこれらを用いてS7がschemaの特性b)を持つことを証明する.
(\ref{s72}), (\ref{s73})から, 
\begin{align}
  \label{s74}
  &(T|z)(S_{1}) \equiv (T|z)(\forall q(\exists r(\forall p(R^{*} \to p \in r)))), \\
  \mbox{} \notag \\
  \label{s75}
  &(T|z)(S_{2}) \equiv (T|z)(\forall s({\rm Set}_{p}(\exists q(q \in s \wedge R^{*}))))
\end{align}
である.
また$p$, $q$, $r$がいずれも$z$と異なり, $T$の中に自由変数として現れないことから, 
代入法則 \ref{substquan}によってわかるように
\begin{equation}
\label{s76}
  (T|z)(\forall q(\exists r(\forall p(R^{*} \to p \in r)))) 
  \equiv \forall q(\exists r(\forall p((T|z)(R^{*} \to p \in r))))
\end{equation}
が成り立つ.
また$z$が$p$, $r$と異なることと代入法則 \ref{substfund}により
\begin{equation}
\label{s77}
  (T|z)(R^{*} \to p \in r) \equiv (T|z)(R^{*}) \to p \in r
\end{equation}
が成り立つ.
そこで(\ref{s74}), (\ref{s76}), (\ref{s77})から, 
\begin{equation}
\label{s78}
  (T|z)(S_{1}) \equiv \forall q(\exists r(\forall p((T|z)(R^{*}) \to p \in r)))
\end{equation}
が成り立つことがわかる.
また$p$, $q$, $s$がいずれも$z$と異なり, $T$の中に自由変数として現れないことから, 
代入法則 \ref{substquan}, \ref{substsm}によってわかるように
\begin{equation}
\label{s79}
  (T|z)(\forall s({\rm Set}_{p}(\exists q(q \in s \wedge R^{*})))) 
  \equiv \forall s({\rm Set}_{p}(\exists q((T|z)(q \in s \wedge R^{*}))))
\end{equation}
が成り立つ.
また$z$が$q$, $s$と異なることと代入法則 \ref{substwedge}により
\begin{equation}
\label{s710}
  (T|z)(q \in s \wedge R^{*}) \equiv q \in s \wedge (T|z)(R^{*})
\end{equation}
が成り立つ.
そこで(\ref{s75}), (\ref{s79}), (\ref{s710})から, 
\begin{equation}
\label{s711}
  (T|z)(S_{2}) \equiv \forall s({\rm Set}_{p}(\exists q(q \in s \wedge (T|z)(R^{*}))))
\end{equation}
が成り立つことがわかる.
従って(\ref{s71}), (\ref{s78}), (\ref{s711})から, $(T|z)(S)$が
\begin{equation}
\label{s712}
  \forall q(\exists r(\forall p((T|z)(R^{*}) \to p \in r))) 
  \to \forall s({\rm Set}_{p}(\exists q(q \in s \wedge (T|z)(R^{*}))))
\end{equation}
と一致することがわかる.
ここで$R^{*}$が関係式, $T$が対象式であることから, 
構成法則 \ref{formsubst}より$(T|z)(R^{*})$は関係式である.
また$p$と$q$は互いに異なる文字である.
また$r$と$s$は共に$p$, $q$と異なる文字であり, $R$及び$T$の中に自由変数として現れないから, 
変数法則 \ref{valsubst}からわかるように, これらは$(T|z)(R^{*})$の中に自由変数として現れない.
従って(\ref{s712}), 即ち$(T|z)(S)$はS7の直接の適用によっても得られる記号列である.
故にS7はschemaの特性b)を持つ.

これでS7がschemaの特性a), b)を持つことが確かめられた.
故にS7は$\mathscr{T}$のschemaと成り得る.




\mathstrut%確認済
(\ref{s72})の証明: 
$q$は$x$, $u$と異なり, $R$の中に自由変数として現れないから, 
変数法則 \ref{valfund}, \ref{valquan}により, 
$q$は$\exists u(\forall x(R \to x \in u))$の中に自由変数として現れない.
故に代入法則 \ref{substquantrans}により
\begin{equation}
\label{s713}
  S_{1} \equiv \forall q((q|y)(\exists u(\forall x(R \to x \in u))))
\end{equation}
が成り立つ.
また$y$と$q$は共に$x$, $u$と異なるから, 代入法則 \ref{substquan}によってわかるように
\begin{equation}
\label{s714}
  (q|y)(\exists u(\forall x(R \to x \in u))) \equiv \exists u(\forall x((q|y)(R \to x \in u)))
\end{equation}
が成り立つ.
また$y$が$x$, $u$と異なることと代入法則 \ref{substfund}により
\begin{equation}
\label{s715}
  (q|y)(R \to x \in u) \equiv (q|y)(R) \to x \in u
\end{equation}
が成り立つ.
そこで(\ref{s713})---(\ref{s715})から, 
\begin{equation}
\label{s716}
  S_{1} \equiv \forall q(\exists u(\forall x((q|y)(R) \to x \in u)))
\end{equation}
が成り立つことがわかる.
また$r$は$x$, $u$, $q$と異なり, $R$の中に自由変数として現れないから, 
変数法則 \ref{valfund}, \ref{valsubst}, \ref{valquan}により, 
$r$は$\forall x((q|y)(R) \to x \in u)$の中に自由変数として現れない.
故に代入法則 \ref{substquantrans}により
\begin{equation}
\label{s717}
  \exists u(\forall x((q|y)(R) \to x \in u)) \equiv \exists r((r|u)(\forall x((q|y)(R) \to x \in u)))
\end{equation}
が成り立つ.
また$u$と$r$は共に$x$と異なるから, 代入法則 \ref{substquan}により
\begin{equation}
\label{s718}
  (r|u)(\forall x((q|y)(R) \to x \in u)) \equiv \forall x((r|u)((q|y)(R) \to x \in u))
\end{equation}
が成り立つ.
また$u$は$q$と異なり, $R$の中に自由変数として現れないから, 
変数法則 \ref{valsubst}により, $u$は$(q|y)(R)$の中に自由変数として現れない.
このことと$u$が$x$と異なることから, 代入法則 \ref{substfree}, \ref{substfund}により
\begin{equation}
\label{s719}
  (r|u)((q|y)(R) \to x \in u) \equiv (q|y)(R) \to x \in r
\end{equation}
が成り立つ.
そこで(\ref{s717})---(\ref{s719})から, 
\begin{equation}
\label{s720}
  \exists u(\forall x((q|y)(R) \to x \in u)) \equiv \exists r(\forall x((q|y)(R) \to x \in r))
\end{equation}
が成り立つことがわかる.
また$p$は$x$, $q$, $r$と異なり, $R$の中に自由変数として現れないから, 
変数法則 \ref{valfund}, \ref{valsubst}により, 
$p$は$(q|y)(R) \to x \in r$の中に自由変数として現れない.
故に代入法則 \ref{substquantrans}により
\begin{equation}
\label{s721}
  \forall x((q|y)(R) \to x \in r) \equiv \forall p((p|x)((q|y)(R) \to x \in r))
\end{equation}
が成り立つ.
また$x$が$r$と異なることと代入法則 \ref{substfund}により
\begin{equation}
\label{s722}
  (p|x)((q|y)(R) \to x \in r) \equiv R^{*} \to p \in r
\end{equation}
が成り立つ.
そこで(\ref{s721}), (\ref{s722})から, 
\begin{equation}
\label{s723}
  \forall x((q|y)(R) \to x \in r) \equiv \forall p(R^{*} \to p \in r)
\end{equation}
が成り立つことがわかる.
以上の(\ref{s716}), (\ref{s720}), (\ref{s723})から, (\ref{s72})が成り立つことがわかる.

(\ref{s73})の証明: 
$s$は$y$, $v$と異なり, $R$の中に自由変数として現れないから, 
変数法則 \ref{valwedge}, \ref{valquan}, \ref{valsm}により, 
$s$は${\rm Set}_{x}(\exists y(y \in v \wedge R))$の中に自由変数として現れない.
故に代入法則 \ref{substquantrans}により
\begin{equation}
\label{s724}
  S_{2} \equiv \forall s((s|v)({\rm Set}_{x}(\exists y(y \in v \wedge R))))
\end{equation}
が成り立つ.
また$v$と$s$は共に$x$, $y$と異なるから, 代入法則 \ref{substquan}, \ref{substsm}によってわかるように
\begin{equation}
\label{s725}
  (s|v)({\rm Set}_{x}(\exists y(y \in v \wedge R))) 
  \equiv {\rm Set}_{x}(\exists y((s|v)(y \in v \wedge R)))
\end{equation}
が成り立つ.
また$v$が$y$と異なり, $R$の中に自由変数として現れないことから, 
代入法則 \ref{substfree}, \ref{substwedge}により
\begin{equation}
\label{s726}
  (s|v)(y \in v \wedge R) \equiv y \in s \wedge R
\end{equation}
が成り立つ.
そこで(\ref{s724})---(\ref{s726})から, 
\begin{equation}
\label{s727}
  S_{2} \equiv \forall s({\rm Set}_{x}(\exists y(y \in s \wedge R)))
\end{equation}
が成り立つことがわかる.
また$p$は$y$, $s$と異なり, $R$の中に自由変数として現れないから, 
変数法則 \ref{valwedge}, \ref{valquan}により, 
$p$は$\exists y(y \in s \wedge R)$の中に自由変数として現れない.
故に代入法則 \ref{substsmtrans}により
\begin{equation}
\label{s728}
  {\rm Set}_{x}(\exists y(y \in s \wedge R)) \equiv {\rm Set}_{p}((p|x)(\exists y(y \in s \wedge R)))
\end{equation}
が成り立つ.
また$x$と$p$は共に$y$と異なるから, 代入法則 \ref{substquan}により
\begin{equation}
\label{s729}
  (p|x)(\exists y(y \in s \wedge R)) \equiv \exists y((p|x)(y \in s \wedge R))
\end{equation}
が成り立つ.
また$x$が$y$, $s$と異なることと代入法則 \ref{substwedge}により
\begin{equation}
\label{s730}
  (p|x)(y \in s \wedge R) \equiv y \in s \wedge (p|x)(R)
\end{equation}
が成り立つ.
そこで(\ref{s728})---(\ref{s730})から, 
\begin{equation}
\label{s731}
  {\rm Set}_{x}(\exists y(y \in s \wedge R)) \equiv {\rm Set}_{p}(\exists y(y \in s \wedge (p|x)(R)))
\end{equation}
が成り立つことがわかる.
また$q$は$y$, $p$, $s$と異なり, $R$の中に自由変数として現れないから, 
変数法則 \ref{valsubst}, \ref{valwedge}により, 
$q$は$y \in s \wedge (p|x)(R)$の中に自由変数として現れない.
故に代入法則 \ref{substquantrans}により
\begin{equation}
\label{s732}
  \exists y(y \in s \wedge (p|x)(R)) \equiv \exists q((q|y)(y \in s \wedge (p|x)(R)))
\end{equation}
が成り立つ.
また$y$が$s$と異なることと代入法則 \ref{substwedge}により
\begin{equation}
\label{s733}
  (q|y)(y \in s \wedge (p|x)(R)) \equiv q \in s \wedge (q|y)((p|x)(R))
\end{equation}
が成り立つ.
また$x$, $y$, $p$, $q$のどの二つも互いに異なることから, 代入法則 \ref{substsubst}により
\begin{equation}
\label{s734}
  (q|y)((p|x)(R)) \equiv R^{*}
\end{equation}
が成り立つ.
そこで(\ref{s732})---(\ref{s734})から, 
\begin{equation}
\label{s735}
  \exists y(y \in s \wedge (p|x)(R)) \equiv \exists q(q \in s \wedge R^{*})
\end{equation}
が成り立つことがわかる.
以上の(\ref{s727}), (\ref{s731}), (\ref{s735})から, (\ref{s73})が成り立つことがわかる.




\mathstrut
\begin{defi}
\label{defs7}%定義1%確認済
S7は集合論のschemaである.
これを\textbf{分出と合併のschema} (schema of separation and union) という.
\end{defi}




\mathstrut%確認済%koko
以後特に断らない限り, S7はschemaであるとする.




\mathstrut
\begin{thm}
\label{sthms7}%定理5.1%新規%確認済
$R$を関係式とし, $x$と$y$を異なる文字とする.
また$u$と$v$を共に$x$及び$y$と異なり, $R$の中に自由変数として現れない文字とする.
このとき次の1), 2)が成り立つ.

1)
$\forall y(\exists u(\forall x(R \to x \in u)))$ならば, 
$\forall v({\rm Set}_{x}(\exists y(y \in v \wedge R)))$.

2)
$y$が定数でなく, $\exists u(\forall x(R \to x \in u))$が成り立てば, 
$\forall v({\rm Set}_{x}(\exists y(y \in v \wedge R)))$.
\end{thm}


\noindent{\bf 証明}
~このときschema S7の適用により
\[
  \forall y(\exists u(\forall x(R \to x \in u))) 
  \to \forall v({\rm Set}_{x}(\exists y(y \in v \wedge R)))
\]
が成り立つ.
1)が成り立つことはこれと推論法則 \ref{dedmp}によって明らかである.
また2)が成り立つことは1)と推論法則 \ref{dedltthmquan}によって明らかである.
\halmos




\mathstrut%確認済%koko
次の定理 \ref{sthms7a}---\ref{sthms7ab}で, S7を使い易い形にしておく.




\mathstrut
\begin{thm}
\label{sthms7a}%定理5.2%新規%確認済
$a$を集合とし, $R$を関係式とする.
また$x$と$y$を異なる文字とし, $x$は$a$の中に自由変数として現れないとする.
また$v$を$x$, $y$と異なり, $R$の中に自由変数として現れない文字とする.
このとき
\begin{equation}
\label{sthms7a1}
  \forall y(\forall x(R \to x \in a)) \to \forall v({\rm Set}_{x}(\exists y(y \in v \wedge R)))
\end{equation}
が成り立つ.
またこのことから, 次の1)---4)が成り立つ.

1)
$\forall y(\forall x(R \to x \in a))$ならば, 
$\forall v({\rm Set}_{x}(\exists y(y \in v \wedge R)))$.

2)
$y$が定数でなく, $\forall x(R \to x \in a)$が成り立てば, 
$\forall v({\rm Set}_{x}(\exists y(y \in v \wedge R)))$.

3)
$x$が定数でなく, $\forall y(R \to x \in a)$が成り立てば, 
$\forall v({\rm Set}_{x}(\exists y(y \in v \wedge R)))$.

4)
$x$と$y$が共に定数でなく, $R \to x \in a$が成り立てば, 
$\forall v({\rm Set}_{x}(\exists y(y \in v \wedge R)))$.
\end{thm}


\noindent{\bf 証明}
~$u$を$x$, $y$と異なり, $R$の中に自由変数として現れない文字とする.
このとき, 
\[
  (a|u)(\forall x(R \to x \in u)) \to \exists u(\forall x(R \to x \in u))
\]
はschema S4の適用によって得られる記号列である.
ここで$x$と$u$が互いに異なり, $x$, $u$がそれぞれ$a$, $R$の中に自由変数として現れないことから, 
代入法則 \ref{substfree}, \ref{substfund}, \ref{substquan}によってわかるように, この記号列は
\[
  \forall x(R \to x \in a) \to \exists u(\forall x(R \to x \in u))
\]
と一致する.
故にこれはS4の適用によって得られる記号列である.
そこでいま$T$を任意の対象式とするとき, schemaの特性から, 
\[
  (T|y)(\forall x(R \to x \in a) \to \exists u(\forall x(R \to x \in u)))
\]
もS4の適用によって得られる記号列である.
従ってこれが成り立つ.
そこで推論法則 \ref{dedtquanfund2}により
\begin{equation}
\label{sthms7a2}
  \forall y(\forall x(R \to x \in a)) \to \forall y(\exists u(\forall x(R \to x \in u)))
\end{equation}
が成り立つ.
また$R$, $x$, $y$, $u$, $v$に対する仮定から, schema S7の適用により
\begin{equation}
\label{sthms7a3}
  \forall y(\exists u(\forall x(R \to x \in u))) 
  \to \forall v({\rm Set}_{x}(\exists y(y \in v \wedge R)))
\end{equation}
が成り立つ.
そこで(\ref{sthms7a2}), (\ref{sthms7a3})から, 
推論法則 \ref{dedmmp}によって(\ref{sthms7a1})が成り立つ.

\noindent
1)
(\ref{sthms7a1})と推論法則 \ref{dedmp}によって明らか.

\noindent
2)
1)と推論法則 \ref{dedltthmquan}によって明らか.

\noindent
3)
1)と推論法則 \ref{dedltthmquan}, \ref{dedquanch}によって明らか.

\noindent
4)
2)と推論法則 \ref{dedltthmquan}によって明らか.
\halmos




\mathstrut
\begin{thm}
\label{sthms7b}%定理5.3%新規%確認済
$b$を集合とし, $R$を関係式とする.
また$x$と$y$を互いに異なり, 共に$b$の中に自由変数として現れない文字とする.
また$u$を$x$, $y$と異なり, $R$の中に自由変数として現れない文字とする.
このとき
\begin{equation}
\label{sthms7b1}
  \forall y(\exists u(\forall x(R \to x \in u))) \to {\rm Set}_{x}(\exists y(y \in b \wedge R))
\end{equation}
が成り立つ.
またこのことから, 次の1), 2)が成り立つ.

1)
$\forall y(\exists u(\forall x(R \to x \in u)))$ならば, 
$\exists y(y \in b \wedge R)$は$x$について集合を作り得る.

2)
$y$が定数でなく, $\exists u(\forall x(R \to x \in u))$が成り立てば, 
$\exists y(y \in b \wedge R)$は$x$について集合を作り得る.
\end{thm}


\noindent{\bf 証明}
~$v$を$x$, $y$と異なり, $R$の中に自由変数として現れない文字とする.
このときschema S7の適用により
\begin{equation}
\label{sthms7b2}
  \forall y(\exists u(\forall x(R \to x \in u))) 
  \to \forall v({\rm Set}_{x}(\exists y(y \in v \wedge R)))
\end{equation}
が成り立つ.
またThm \ref{thmallfund2}より
\[
  \forall v({\rm Set}_{x}(\exists y(y \in v \wedge R))) 
  \to (b|v)({\rm Set}_{x}(\exists y(y \in v \wedge R)))
\]
が成り立つ.
ここで$x$と$y$が共に$v$と異なり, $b$の中に自由変数として現れないことと, 
$v$が$R$の中に自由変数として現れないことから, 
代入法則 \ref{substfree}, \ref{substwedge}, \ref{substquan}, \ref{substsm}によってわかるように, 
この記号列は
\begin{equation}
\label{sthms7b3}
  \forall v({\rm Set}_{x}(\exists y(y \in v \wedge R))) \to {\rm Set}_{x}(\exists y(y \in b \wedge R))
\end{equation}
と一致する.
故にこれが成り立つ.
そこで(\ref{sthms7b2}), (\ref{sthms7b3})から, 
推論法則 \ref{dedmmp}によって(\ref{sthms7b1})が成り立つ.

\noindent
1)
(\ref{sthms7b1})と推論法則 \ref{dedmp}によって明らか.

\noindent
2)
1)と推論法則 \ref{dedltthmquan}によって明らか.
\halmos




\mathstrut
\begin{thm}
\label{sthms7ab}%定理5.4%新規%確認済
$a$と$b$を集合とし, $R$を関係式とする.
また$x$と$y$を異なる文字とし, $x$は$a$及び$b$の中に自由変数として現れず, 
$y$は$b$の中に自由変数として現れないとする.
このとき
\begin{equation}
\label{sthms7ab1}
  \forall y(\forall x(R \to x \in a)) \to {\rm Set}_{x}(\exists y(y \in b \wedge R))
\end{equation}
が成り立つ.
またこのことから, 次の1)---4)が成り立つ.

1)
$\forall y(\forall x(R \to x \in a))$ならば, 
$\exists y(y \in b \wedge R)$は$x$について集合を作り得る.

2)
$y$が定数でなく, $\forall x(R \to x \in a)$が成り立てば, 
$\exists y(y \in b \wedge R)$は$x$について集合を作り得る.

3)
$x$が定数でなく, $\forall y(R \to x \in a)$が成り立てば, 
$\exists y(y \in b \wedge R)$は$x$について集合を作り得る.

4)
$x$と$y$が共に定数でなく, $R \to x \in a$が成り立てば, 
$\exists y(y \in b \wedge R)$は$x$について集合を作り得る.
\end{thm}


\noindent{\bf 証明}
~$v$を$x$, $y$と異なり, $R$の中に自由変数として現れない文字とする.
このとき定理 \ref{sthms7a}より
\begin{equation}
\label{sthms7ab2}
  \forall y(\forall x(R \to x \in a)) \to \forall v({\rm Set}_{x}(\exists y(y \in v \wedge R)))
\end{equation}
が成り立つ.
またThm \ref{thmallfund2}より
\[
  \forall v({\rm Set}_{x}(\exists y(y \in v \wedge R))) 
  \to (b|v)({\rm Set}_{x}(\exists y(y \in v \wedge R)))
\]
が成り立つ.
ここで$x$と$y$が共に$v$と異なり, $b$の中に自由変数として現れないことと, 
$v$が$R$の中に自由変数として現れないことから, 
代入法則 \ref{substfree}, \ref{substwedge}, \ref{substquan}, \ref{substsm}によってわかるように, 
この記号列は
\begin{equation}
\label{sthms7ab3}
  \forall v({\rm Set}_{x}(\exists y(y \in v \wedge R))) \to {\rm Set}_{x}(\exists y(y \in b \wedge R))
\end{equation}
と一致する.
故にこれが成り立つ.
そこで(\ref{sthms7ab2}), (\ref{sthms7ab3})から, 
推論法則 \ref{dedmmp}によって(\ref{sthms7ab1})が成り立つ.

\noindent
1)
(\ref{sthms7ab1})と推論法則 \ref{dedmp}によって明らか.

\noindent
2)
1)と推論法則 \ref{dedltthmquan}によって明らか.

\noindent
3)
1)と推論法則 \ref{dedltthmquan}, \ref{dedquanch}によって明らか.

\noindent
4)
2)と推論法則 \ref{dedltthmquan}によって明らか.
\halmos




\mathstrut%確認済%koko
次の定理は, S7から得られる重要な結論の一つである.




\mathstrut
\begin{thm}
\label{sthmssetsm}%定理5.5%確認済
$a$を集合, $R$を関係式とし, $x$を$a$の中に自由変数として現れない文字とする.
このとき関係式$x \in a \wedge R$は$x$について集合を作り得る.
\end{thm}


\noindent{\bf 証明}
~$y$と$z$を, 互いに異なり, 共に$x$と異なり, 
$a$及び$R$の中に自由変数として現れない, 定数でない文字とする.
また$(z|x)(R) \wedge y = z$を$S$と書く.
このとき$S$は関係式である.
またThm \ref{awbta}より
\begin{equation}
\label{sthmssetsm1}
  S \to y = z
\end{equation}
が成り立つ.
またThm \ref{x=yty=x}より
\begin{equation}
\label{sthmssetsm2}
  y = z \to z = y
\end{equation}
が成り立つ.
また定理 \ref{sthmsingletonbasis}と推論法則 \ref{dedequiv}により
\begin{equation}
\label{sthmssetsm3}
  z = y \to z \in \{y\}
\end{equation}
が成り立つ.
そこで(\ref{sthmssetsm1})---(\ref{sthmssetsm3})から, 推論法則 \ref{dedmmp}によって
\begin{equation}
\label{sthmssetsm4}
  S \to z \in \{y\}
\end{equation}
が成り立つことがわかる.
さていま$y$と$z$は異なる文字である.
故に変数法則 \ref{valnset}により, $z$は$\{y\}$の中に自由変数として現れない.
また$y$と$z$は共に$a$の中に自由変数として現れない.
また$y$と$z$は共に定数でない.
これらのことと, (\ref{sthmssetsm4})が成り立つことから, 定理 \ref{sthms7ab}より
\begin{equation}
\label{sthmssetsm5}
  {\rm Set}_{z}(\exists y(y \in a \wedge S))
\end{equation}
が成り立つ.
また$S$の定義から, Thm \ref{1awb1wclaw1bwc1}と推論法則 \ref{dedeqch}により
\begin{equation}
\label{sthmssetsm6}
  y \in a \wedge S \leftrightarrow (y \in a \wedge (z|x)(R)) \wedge y = z
\end{equation}
が成り立つ.
またThm \ref{awblbwa}より
\begin{equation}
\label{sthmssetsm7}
  (y \in a \wedge (z|x)(R)) \wedge y = z \leftrightarrow y = z \wedge (y \in a \wedge (z|x)(R))
\end{equation}
が成り立つ.
そこで(\ref{sthmssetsm6}), (\ref{sthmssetsm7})から, 推論法則 \ref{dedeqtrans}によって
\[
  y \in a \wedge S \leftrightarrow y = z \wedge (y \in a \wedge (z|x)(R))
\]
が成り立つ.
故にこれと$y$が定数でないことから, 推論法則 \ref{dedalleqquansepconst}により
\begin{equation}
\label{sthmssetsm8}
  \exists y(y \in a \wedge S) \leftrightarrow \exists y(y = z \wedge (y \in a \wedge (z|x)(R)))
\end{equation}
が成り立つ.
また$y$と$z$が異なることから, Thm \ref{thmspquan=}と推論法則 \ref{dedeqch}により
\[
  \exists y(y = z \wedge (y \in a \wedge (z|x)(R))) \leftrightarrow (z|y)(y \in a \wedge (z|x)(R))
\]
が成り立つ.
ここで$y$が$z$と異なり, $R$の中に自由変数として現れないことから, 
変数法則 \ref{valsubst}により, $y$は$(z|x)(R)$の中に自由変数として現れない.
このことと$y$が$a$の中にも自由変数として現れないことから, 
代入法則 \ref{substfree}, \ref{substfund}, \ref{substwedge}により, 上記の記号列は
\begin{equation}
\label{sthmssetsm9}
  \exists y(y = z \wedge (y \in a \wedge (z|x)(R))) \leftrightarrow z \in a \wedge (z|x)(R)
\end{equation}
と一致する.
故にこれが成り立つ.
そこで(\ref{sthmssetsm8}), (\ref{sthmssetsm9})から, 推論法則 \ref{dedeqtrans}によって
\begin{equation}
\label{sthmssetsm10}
  \exists y(y \in a \wedge S) \leftrightarrow z \in a \wedge (z|x)(R)
\end{equation}
が成り立つ.
(\ref{sthmssetsm5}), (\ref{sthmssetsm10})と, $z$が定数でないことから, 定理 \ref{sthmalleqsm}より
\[
  {\rm Set}_{z}(z \in a \wedge (z|x)(R))
\]
が成り立つ.
ここで$x$が$a$の中に自由変数として現れないことから, 
代入法則 \ref{substfree}, \ref{substfund}, \ref{substwedge}により, この記号列は
\[
  {\rm Set}_{z}((z|x)(x \in a \wedge R))
\]
と一致する.
また$z$は$x$と異なり, $a$及び$R$の中に自由変数として現れないから, 
変数法則 \ref{valfund}, \ref{valwedge}により, $z$は$x \in a \wedge R$の中に自由変数として現れない.
故に代入法則 \ref{substsmtrans}により, 上記の記号列は
\[
  {\rm Set}_{x}(x \in a \wedge R)
\]
と一致する.
従ってこれが成り立つ.
\halmos




\mathstrut
\begin{defi}
\label{defsset}%定義2%新規%確認済
$\mathscr{T}$を特殊記号として$\in$を持つ理論とする.
$\mathscr{T}$の記号列$a$, $R$と文字$x$に対し, 
記号列$\{x \mid x \in a \wedge R\}$を以下$\{x \in a \mid R\}$とも記す.
\end{defi}




\mathstrut%確認済%koko
以下の変数法則 \ref{valsset}, 一般代入法則 \ref{gsubstsset}, 
代入法則 \ref{substssettrans}, \ref{substsset}, 
構成法則 \ref{formsset}では, $\mathscr{T}$を特殊記号として$\in$を持つ理論とし, 
これらの法則における``記号列'', ``集合'', ``関係式''とは, 
それぞれ$\mathscr{T}$の記号列, $\mathscr{T}$の対象式, $\mathscr{T}$の関係式のこととする.




\mathstrut
\begin{valu}
\label{valsset}%変数27%確認済
$a$と$R$を記号列とし, $x$を文字とする.

1)
$x$は$\{x \in a \mid R\}$の中に自由変数として現れない.

2)
$y$を文字とする.
$y$が$a$及び$R$の中に自由変数として現れなければ, 
$y$は$\{x \in a \mid R\}$の中に自由変数として現れない.
\end{valu}


\noindent{\bf 証明}
~1)
$\{x \in a \mid R\}$の定義と変数法則 \ref{valiset}によって明らかである.

\noindent
2)
$y$が$x$と同じ文字ならば1)により明らか.
$y$が$x$と異なる文字ならば, このことと$y$が$a$及び$R$の中に自由変数として現れないことから, 
変数法則 \ref{valfund}, \ref{valwedge}, \ref{valiset}によってわかるように, 
$y$は$\{x \mid x \in a \wedge R\}$, 即ち$\{x \in a \mid R\}$の中に自由変数として現れない.
\halmos




\mathstrut
\begin{gsub}
\label{gsubstsset}%一般代入31%確認済
$a$と$R$を記号列とし, $x$を文字とする.
また$n$を自然数とし, $T_{1}, T_{2}, \cdots, T_{n}$を記号列とする.
また$y_{1}, y_{2}, \cdots, y_{n}$を, どの二つも互いに異なる文字とする.
$x$が$y_{1}, y_{2}, \cdots, y_{n}$のいずれとも異なり, かつ
$T_{1}, T_{2}, \cdots, T_{n}$のいずれの記号列の中にも自由変数として現れなければ, 
\begin{multline*}
  (T_{1}|y_{1}, T_{2}|y_{2}, \cdots, T_{n}|y_{n})(\{x \in a \mid R\}) \\
  \equiv \{x \in (T_{1}|y_{1}, T_{2}|y_{2}, \cdots, T_{n}|y_{n})(a) \mid (T_{1}|y_{1}, T_{2}|y_{2}, \cdots, T_{n}|y_{n})(R)\}
\end{multline*}
が成り立つ.
\end{gsub}


\noindent{\bf 証明}
~定義から$\{x \in a \mid R\}$は$\{x \mid x \in a \wedge R\}$だから, 
\begin{equation}
\label{gsubstsset1}
  (T_{1}|y_{1}, T_{2}|y_{2}, \cdots, T_{n}|y_{n})(\{x \in a \mid R\}) 
  \equiv (T_{1}|y_{1}, T_{2}|y_{2}, \cdots, T_{n}|y_{n})(\{x \mid x \in a \wedge R\})
\end{equation}
である.
また$x$が$y_{1}, y_{2}, \cdots, y_{n}$のいずれとも異なり, かつ
$T_{1}, T_{2}, \cdots, T_{n}$のいずれの記号列の中にも自由変数として現れないことから, 
一般代入法則 \ref{gsubstiset}により
\begin{equation}
\label{gsubstsset2}
  (T_{1}|y_{1}, T_{2}|y_{2}, \cdots, T_{n}|y_{n})(\{x \mid x \in a \wedge R\}) 
  \equiv \{x \mid (T_{1}|y_{1}, T_{2}|y_{2}, \cdots, T_{n}|y_{n})(x \in a \wedge R)\}
\end{equation}
が成り立つ.
また$x$が$y_{1}, y_{2}, \cdots, y_{n}$のいずれとも異なることと
一般代入法則 \ref{gsubstfund}, \ref{gsubstwedge}から, 
\begin{multline}
\label{gsubstsset3}
  (T_{1}|y_{1}, T_{2}|y_{2}, \cdots, T_{n}|y_{n})(x \in a \wedge R) \\
  \equiv x \in (T_{1}|y_{1}, T_{2}|y_{2}, \cdots, T_{n}|y_{n})(a) \wedge (T_{1}|y_{1}, T_{2}|y_{2}, \cdots, T_{n}|y_{n})(R)
\end{multline}
が成り立つ.
以上の(\ref{gsubstsset1})---(\ref{gsubstsset3})から, 
$(T_{1}|y_{1}, T_{2}|y_{2}, \cdots, T_{n}|y_{n})(\{x \in a \mid R\})$が
\[
  \{x \mid x \in (T_{1}|y_{1}, T_{2}|y_{2}, \cdots, T_{n}|y_{n})(a) \wedge (T_{1}|y_{1}, T_{2}|y_{2}, \cdots, T_{n}|y_{n})(R)\}
\]
と一致することがわかる.
これは$\{x \in (T_{1}|y_{1}, T_{2}|y_{2}, \cdots, T_{n}|y_{n})(a) \mid (T_{1}|y_{1}, T_{2}|y_{2}, \cdots, T_{n}|y_{n})(R)\}$と
書き表される記号列である.
\halmos




\mathstrut
\begin{subs}
\label{substssettrans}%代入36%前半を追加%確認済
$a$と$R$を記号列とし, $x$と$y$を文字とする.
$y$が$a$及び$R$の中に自由変数として現れなければ, 
\[
  \{x \in a \mid R\} \equiv \{y \in (y|x)(a) \mid (y|x)(R)\}
\]
が成り立つ.
更に, $x$が$a$の中に自由変数として現れなければ, 
\[
  \{x \in a \mid R\} \equiv \{y \in a \mid (y|x)(R)\}
\]
が成り立つ.
\end{subs}


\noindent{\bf 証明}
~$y$が$x$ならば, 本法則が成り立つことは代入法則 \ref{substsame}によって明らかである.
そこで以下$y$は$x$と異なる文字であるとする.
このとき$y$が$a$及び$R$の中に自由変数として現れないことから, 
変数法則 \ref{valfund}, \ref{valwedge}により, $y$は$x \in a \wedge R$の中に自由変数として現れない.
故に代入法則 \ref{substisettrans}により
\begin{equation}
\label{substssettrans1}
  \{x \mid x \in a \wedge R\} \equiv \{y \mid (y|x)(x \in a \wedge R)\}
\end{equation}
が成り立つ.
また代入法則 \ref{substfund}, \ref{substwedge}により
\begin{equation}
\label{substssettrans2}
  (y|x)(x \in a \wedge R) \equiv y \in (y|x)(a) \wedge (y|x)(R)
\end{equation}
が成り立つ.
そこで(\ref{substssettrans1}), (\ref{substssettrans2})から, 
\[
  \{x \mid x \in a \wedge R\} \equiv \{y \mid y \in (y|x)(a) \wedge (y|x)(R)\}, 
\]
即ち
\[
  \{x \in a \mid R\} \equiv \{y \in (y|x)(a) \mid (y|x)(R)\}
\]
が成り立つ.
特にここで$x$が$a$の中に自由変数として現れなければ, 代入法則 \ref{substfree}により
$(y|x)(a)$は$a$と一致するから, 
\[
  \{x \in a \mid R\} \equiv \{y \in a \mid (y|x)(R)\}
\]
が成り立つ.
\halmos




\mathstrut
\begin{subs}
\label{substsset}%代入37%確認済
$a$, $R$, $T$を記号列とし, $x$と$y$を異なる文字とする.
$x$が$T$の中に自由変数として現れなければ, 
\[
  (T|y)(\{x \in a \mid R\}) \equiv \{x \in (T|y)(a) \mid (T|y)(R)\}
\]
が成り立つ.
\end{subs}


\noindent{\bf 証明}
~一般代入法則 \ref{gsubstsset}において, $n$が$1$の場合である.
\halmos




\mathstrut
\begin{form}
\label{formsset}%構成44%確認済
$a$を集合, $R$を関係式とし, $x$を文字とする.
このとき$\{x \in a \mid R\}$は集合である.
\end{form}


\noindent{\bf 証明}
~構成法則 \ref{formfund}, \ref{formwedge}, \ref{formiset}によって明らかである.
\halmos




\mathstrut
\begin{thm}
\label{sthmssetbasis}%定理5.6%1)と2)は新規%確認済
$a$と$b$を集合, $R$を関係式とし, $x$を$a$の中に自由変数として現れない文字とする.
このとき
\begin{equation}
\label{sthmssetbasis1}
  b \in \{x \in a \mid R\} \leftrightarrow b \in a \wedge (b|x)(R)
\end{equation}
が成り立つ.
またこのことから, 次の1), 2)が成り立つ.

1)
$b \in \{x \in a \mid R\}$ならば, $b \in a$と$(b|x)(R)$が共に成り立つ.

2)
$b \in a$と$(b|x)(R)$が共に成り立てば, $b \in \{x \in a \mid R\}$.
\end{thm}


\noindent{\bf 証明}
~このとき定理 \ref{sthmssetsm}より$x \in a \wedge R$は$x$について集合を作り得るから, 
定理 \ref{sthmisetbasis}より
\[
  b \in \{x \mid x \in a \wedge R\} \leftrightarrow (b|x)(x \in a \wedge R), 
\]
即ち
\[
  b \in \{x \in a \mid R\} \leftrightarrow (b|x)(x \in a \wedge R)
\]
が成り立つ.
ここで$x$が$a$の中に自由変数として現れないことから, 
代入法則 \ref{substfree}, \ref{substfund}, \ref{substwedge}により, 
この記号列は(\ref{sthmssetbasis1})と一致する.
故に(\ref{sthmssetbasis1})が成り立つ.
1), 2)が成り立つことは(\ref{sthmssetbasis1})と
推論法則 \ref{dedwedge}, \ref{dedeqfund}によって明らかである.
\halmos




\mathstrut
\begin{thm}
\label{sthmssetsubseta}%定理5.7%確認済
$a$を集合, $R$を関係式とし, $x$を$a$の中に自由変数として現れない文字とする.
このとき
\begin{equation}
\label{sthmssetsubseta1}
  \{x \in a \mid R\} \subset a
\end{equation}
が成り立つ.
\end{thm}


\noindent{\bf 証明}
~$y$を$a$及び$R$の中に自由変数として現れない, 定数でない文字とする.
このとき変数法則 \ref{valsset}により, $y$は$\{x \in a \mid R\}$の中に自由変数として現れない.
また$x$が$a$の中に自由変数として現れないことから, 
定理 \ref{sthmssetbasis}と推論法則 \ref{dedprewedge}, \ref{dedequiv}によってわかるように, 
\[
  y \in \{x \in a \mid R\} \to y \in a
\]
が成り立つ.
このことと, $y$が定数でなく, 
上述のように$\{x \in a \mid R\}$及び$a$の中に自由変数として現れないことから, 
定理 \ref{sthmsubsetconst}より(\ref{sthmssetsubseta1})が成り立つ.
\halmos




\mathstrut%確認済%koko
定理 \ref{sthmssetsubseta}から直ちに次の定理を得る.




\mathstrut
\begin{thm}
\label{sthmssetsubsetb}%定理5.8%新規%確認済
$a$と$b$を集合, $R$を関係式とし, $x$を$a$の中に自由変数として現れない文字とする.
このとき
\[
  a \subset b \to \{x \in a \mid R\} \subset b, ~~
  b \subset \{x \in a \mid R\} \to b \subset a
\]
が成り立つ.
またこれらから, 次の1), 2)が成り立つ.

1)
$a \subset b$ならば, $\{x \in a \mid R\} \subset b$.

2)
$b \subset \{x \in a \mid R\}$ならば, $b \subset a$.
\end{thm}


\noindent{\bf 証明}
~このとき定理 \ref{sthmssetsubseta}より$\{x \in a \mid R\} \subset a$が成り立つから, 
推論法則 \ref{dedatawbtrue2}により
\begin{align}
  \label{sthmssetsubsetb1}
  &a \subset b \to \{x \in a \mid R\} \subset a \wedge a \subset b, \\
  \mbox{} \notag \\
  \label{sthmssetsubsetb2}
  &b \subset \{x \in a \mid R\} \to b \subset \{x \in a \mid R\} \wedge \{x \in a \mid R\} \subset a
\end{align}
が共に成り立つ.
また定理 \ref{sthmsubsettrans}より
\begin{align}
  \label{sthmssetsubsetb3}
  &\{x \in a \mid R\} \subset a \wedge a \subset b \to \{x \in a \mid R\} \subset b, \\
  \mbox{} \notag \\
  \label{sthmssetsubsetb4}
  &b \subset \{x \in a \mid R\} \wedge \{x \in a \mid R\} \subset a \to b \subset a
\end{align}
が共に成り立つ.
そこで(\ref{sthmssetsubsetb1})と(\ref{sthmssetsubsetb3}), 
(\ref{sthmssetsubsetb2})と(\ref{sthmssetsubsetb4})から, 
それぞれ推論法則 \ref{dedmmp}によって
\[
  a \subset b \to \{x \in a \mid R\} \subset b, ~~
  b \subset \{x \in a \mid R\} \to b \subset a
\]
が成り立つ.
1), 2)が成り立つことはこれらと推論法則 \ref{dedmp}によって明らかである.
\halmos




\mathstrut
\begin{thm}
\label{sthmbsubsetsset}%定理5.9%新規%確認済
$a$と$b$を集合, $R$を関係式とし, $x$を$a$と$b$の中に自由変数として現れない文字とする.
このとき
\begin{equation}
\label{sthmbsubsetsset1}
  b \subset \{x \in a \mid R\} \leftrightarrow b \subset a \wedge (\forall x \in b)(R)
\end{equation}
が成り立つ.
またこのことから, 次の1), 2), 3)が成り立つ.

1)
$b \subset \{x \in a \mid R\}$ならば, $b \subset a$と$(\forall x \in b)(R)$が共に成り立つ.

2)
$b \subset a$と$(\forall x \in b)(R)$が共に成り立てば, $b \subset \{x \in a \mid R\}$.

3)
$x$が定数でなく, $b \subset a$と$x \in b \to R$が共に成り立てば, $b \subset \{x \in a \mid R\}$.
\end{thm}


\noindent{\bf 証明}
~$y$を$x$と異なり, $a$, $b$, $R$の中に自由変数として現れない, 定数でない文字とする.
このとき変数法則 \ref{valsset}により, $y$は$\{x \in a \mid R\}$の中に自由変数として現れない.
また$x$が$a$の中に自由変数として現れないことから, 定理 \ref{sthmssetbasis}より
\[
  y \in \{x \in a \mid R\} \leftrightarrow y \in a \wedge (y|x)(R)
\]
が成り立つから, 推論法則 \ref{dedaddeqt}により
\begin{equation}
\label{sthmbsubsetsset2}
  (y \in b \to y \in \{x \in a \mid R\}) \leftrightarrow (y \in b \to y \in a \wedge (y|x)(R))
\end{equation}
が成り立つ.
またThm \ref{1atbwc1l1atb1w1atc1}より
\begin{equation}
\label{sthmbsubsetsset3}
  (y \in b \to y \in a \wedge (y|x)(R)) 
  \leftrightarrow (y \in b \to y \in a) \wedge (y \in b \to (y|x)(R))
\end{equation}
が成り立つ.
そこで(\ref{sthmbsubsetsset2}), (\ref{sthmbsubsetsset3})から, 推論法則 \ref{dedeqtrans}によって
\[
  (y \in b \to y \in \{x \in a \mid R\}) 
  \leftrightarrow (y \in b \to y \in a) \wedge (y \in b \to (y|x)(R))
\]
が成り立つ.
これと$y$が定数でないことから, 推論法則 \ref{dedalleqquansepconst}により
\[
  \forall y(y \in b \to y \in \{x \in a \mid R\}) 
  \leftrightarrow \forall y((y \in b \to y \in a) \wedge (y \in b \to (y|x)(R)))
\]
が成り立つ.
ここで$y$が$b$の中に自由変数として現れず, 
上述のように$\{x \in a \mid R\}$の中にも自由変数として現れないことから, この記号列は
\begin{equation}
\label{sthmbsubsetsset4}
  b \subset \{x \in a \mid R\} 
  \leftrightarrow \forall y((y \in b \to y \in a) \wedge (y \in b \to (y|x)(R)))
\end{equation}
と同じである.
故にこれが成り立つ.
またThm \ref{thmallw}より
\[
  \forall y((y \in b \to y \in a) \wedge (y \in b \to (y|x)(R))) 
  \leftrightarrow \forall y(y \in b \to y \in a) \wedge \forall y(y \in b \to (y|x)(R))
\]
が成り立つ.
ここで$y$が$a$と$b$の中に自由変数として現れないことから, この記号列は
\begin{equation}
\label{sthmbsubsetsset5}
  \forall y((y \in b \to y \in a) \wedge (y \in b \to (y|x)(R))) 
  \leftrightarrow b \subset a \wedge \forall y(y \in b \to (y|x)(R))
\end{equation}
と同じである.
故にこれが成り立つ.
またThm \ref{thmspallfund}と推論法則 \ref{dedeqch}により
\[
  \forall x(x \in b \to R) \leftrightarrow (\forall x \in b)(R)
\]
が成り立つ.
ここで$y$が$x$と異なり, $b$と$R$の中に自由変数として現れないことから, 
変数法則 \ref{valfund}により, $y$は$x \in b \to R$の中に自由変数として現れない.
故に代入法則 \ref{substquantrans}により, 上記の記号列は
\[
  \forall y((y|x)(x \in b \to R)) \leftrightarrow (\forall x \in b)(R)
\]
と一致する.
また$x$が$b$の中に自由変数として現れないことと代入法則 \ref{substfree}, \ref{substfund}により, 
この記号列は
\[
  \forall y(y \in b \to (y|x)(R)) \leftrightarrow (\forall x \in b)(R)
\]
と一致する.
故にこれが成り立つ.
そこで推論法則 \ref{dedaddeqw}により
\begin{equation}
\label{sthmbsubsetsset6}
  b \subset a \wedge \forall y(y \in b \to (y|x)(R)) 
  \leftrightarrow b \subset a \wedge (\forall x \in b)(R)
\end{equation}
が成り立つ.
以上の(\ref{sthmbsubsetsset4})---(\ref{sthmbsubsetsset6})から, 
推論法則 \ref{dedeqtrans}によって(\ref{sthmbsubsetsset1})が成り立つことがわかる.

\noindent
1), 2)
(\ref{sthmbsubsetsset1})と推論法則 \ref{dedwedge}, \ref{dedeqfund}によって明らか.

\noindent
3)
2)と推論法則 \ref{dedspallfund}によって明らか.
\halmos




\mathstrut
\begin{thm}
\label{sthmsset=a}%定理5.10%1)---4)は新規%確認済
$a$を集合, $R$を関係式とし, $x$を$a$の中に自由変数として現れない文字とする.
このとき
\begin{equation}
\label{sthmsset=a1}
  (\forall x \in a)(R) \leftrightarrow \{x \in a \mid R\} = a
\end{equation}
が成り立つ.
特に
\begin{equation}
\label{sthmsset=a2}
  \forall x(R) \to \{x \in a \mid R\} = a
\end{equation}
が成り立つ.
またこれらから, 次の1)---4)が成り立つ.

1)
$(\forall x \in a)(R)$ならば, $\{x \in a \mid R\} = a$.
また$\{x \in a \mid R\} = a$ならば, $(\forall x \in a)(R)$.

2)
$x$が定数でなく, $x \in a \to R$が成り立てば, $\{x \in a \mid R\} = a$.

3)
$\forall x(R)$ならば, $\{x \in a \mid R\} = a$.

4)
$x$が定数でなく, $R$が成り立てば, $\{x \in a \mid R\} = a$.
\end{thm}


\noindent{\bf 証明}
~まず(\ref{sthmsset=a1})が成り立つことを示す.
定理 \ref{sthmaxiom1}と推論法則 \ref{dedeqch}により
\begin{equation}
\label{sthmsset=a3}
  \{x \in a \mid R\} = a 
  \leftrightarrow \{x \in a \mid R\} \subset a \wedge a \subset \{x \in a \mid R\}
\end{equation}
が成り立つ.
また$x$が$a$の中に自由変数として現れないことから, 
定理 \ref{sthmssetsubseta}より$\{x \in a \mid R\} \subset a$が成り立つから, 
推論法則 \ref{dedawblatrue2}により
\begin{equation}
\label{sthmsset=a4}
  \{x \in a \mid R\} \subset a \wedge a \subset \{x \in a \mid R\} 
  \leftrightarrow a \subset \{x \in a \mid R\}
\end{equation}
が成り立つ.
また$x$が$a$の中に自由変数として現れないことから, 定理 \ref{sthmbsubsetsset}より
\begin{equation}
\label{sthmsset=a5}
  a \subset \{x \in a \mid R\} \leftrightarrow a \subset a \wedge (\forall x \in a)(R)
\end{equation}
が成り立つ.
また定理 \ref{sthmsubsetself}より$a \subset a$が成り立つから, 推論法則 \ref{dedawblatrue2}により
\begin{equation}
\label{sthmsset=a6}
  a \subset a \wedge (\forall x \in a)(R) \leftrightarrow (\forall x \in a)(R)
\end{equation}
が成り立つ.
そこで(\ref{sthmsset=a3})---(\ref{sthmsset=a6})から, 推論法則 \ref{dedeqtrans}によって
\[
  \{x \in a \mid R\} = a \leftrightarrow (\forall x \in a)(R)
\]
が成り立つことがわかる.
故に推論法則 \ref{dedeqch}により(\ref{sthmsset=a1})が成り立つ.

次に(\ref{sthmsset=a2})が成り立つことを示す.
Thm \ref{thmquantspall2}より
\begin{equation}
\label{sthmsset=a7}
  \forall x(R) \to (\forall x \in a)(R)
\end{equation}
が成り立つ.
また(\ref{sthmsset=a1})が成り立つことから, 推論法則 \ref{dedequiv}により
\begin{equation}
\label{sthmsset=a8}
  (\forall x \in a)(R) \to \{x \in a \mid R\} = a
\end{equation}
が成り立つ.
そこで(\ref{sthmsset=a7}), (\ref{sthmsset=a8})から, 
推論法則 \ref{dedmmp}によって(\ref{sthmsset=a2})が成り立つ.

\noindent
1)
(\ref{sthmsset=a1})と推論法則 \ref{dedeqfund}によって明らか.

\noindent
2)
1)と推論法則 \ref{dedspallfund}によって明らか.

\noindent
3)
(\ref{sthmsset=a2})と推論法則 \ref{dedmp}によって明らか.

\noindent
4)
3)と推論法則 \ref{dedltthmquan}によって明らか.
\halmos




\mathstrut
\begin{thm}
\label{sthmsset=arfree}%定理5.11%新規%確認済
$a$を集合, $R$を関係式とし, $x$をこれらの中に自由変数として現れない文字とする.
このとき
\begin{equation}
\label{sthmsset=arfree1}
  R \to \{x \in a \mid R\} = a
\end{equation}
が成り立つ.
またこのことから, 次の(\ref{sthmsset=arfree2})が成り立つ.
\begin{equation}
\label{sthmsset=arfree2}
  R \text{ならば,} ~\{x \in a \mid R\} = a.
\end{equation}
\end{thm}


\noindent{\bf 証明}
~$x$が$R$の中に自由変数として現れないことから, Thm \ref{thmquanfree}と推論法則 \ref{dedequiv}により
\begin{equation}
\label{sthmsset=arfree3}
  R \to \forall x(R)
\end{equation}
が成り立つ.
また$x$が$a$の中に自由変数として現れないことから, 定理 \ref{sthmsset=a}より
\begin{equation}
\label{sthmsset=arfree4}
  \forall x(R) \to \{x \in a \mid R\} = a
\end{equation}
が成り立つ.
そこで(\ref{sthmsset=arfree3}), (\ref{sthmsset=arfree4})から, 
推論法則 \ref{dedmmp}によって(\ref{sthmsset=arfree1})が成り立つ.
(\ref{sthmsset=arfree2})が成り立つことは, 
(\ref{sthmsset=arfree1})と推論法則 \ref{dedmp}によって明らかである.
\halmos




\mathstrut
\begin{thm}
\label{sthmssetsubsetiset}%定理5.12%新規%確認済
$a$を集合, $R$を関係式とし, $x$を$a$の中に自由変数として現れない文字とする.
このとき
\begin{equation}
\label{sthmssetsubsetiset1}
  {\rm Set}_{x}(R) \to \{x \in a \mid R\} \subset \{x \mid R\}
\end{equation}
が成り立つ.
またこのことから, 次の(\ref{sthmssetsubsetiset2})が成り立つ.
\begin{equation}
\label{sthmssetsubsetiset2}
  R \text{が} x \text{について集合を作り得るならば,} ~\{x \in a \mid R\} \subset \{x \mid R\}.
\end{equation}
\end{thm}


\noindent{\bf 証明}
~$y$を$a$及び$R$の中に自由変数として現れない, 定数でない文字とする.
このとき, それぞれ変数法則 \ref{valsm}, \ref{valiset}, \ref{valsset}により, 
$y$は${\rm Set}_{x}(R)$, $\{x \mid R\}$, $\{x \in a \mid R\}$の中に自由変数として現れない.
さて定理 \ref{sthmisetbasis}より
\[
  {\rm Set}_{x}(R) \to (y \in \{x \mid R\} \leftrightarrow (y|x)(R))
\]
が成り立つから, 推論法則 \ref{dedpreequiv}により
\begin{equation}
\label{sthmssetsubsetiset3}
  {\rm Set}_{x}(R) \to ((y|x)(R) \to y \in \{x \mid R\})
\end{equation}
が成り立つ.
また$x$が$a$の中に自由変数として現れないことから, 
定理 \ref{sthmssetbasis}と推論法則 \ref{dedequiv}により
\[
  y \in \{x \in a \mid R\} \to y \in a \wedge (y|x)(R)
\]
が成り立つから, 推論法則 \ref{dedprewedge}により
\[
  y \in \{x \in a \mid R\} \to (y|x)(R)
\]
が成り立つ.
故に推論法則 \ref{dedaddf}により
\begin{equation}
\label{sthmssetsubsetiset4}
  ((y|x)(R) \to y \in \{x \mid R\}) \to (y \in \{x \in a \mid R\} \to y \in \{x \mid R\})
\end{equation}
が成り立つ.
そこで(\ref{sthmssetsubsetiset3}), (\ref{sthmssetsubsetiset4})から, 推論法則 \ref{dedmmp}によって
\[
  {\rm Set}_{x}(R) \to (y \in \{x \in a \mid R\} \to y \in \{x \mid R\})
\]
が成り立つ.
このことと, $y$が定数でなく, 上述のように${\rm Set}_{x}(R)$の中に自由変数として現れないことから, 
推論法則 \ref{dedalltquansepfreeconst}により
\[
  {\rm Set}_{x}(R) \to \forall y(y \in \{x \in a \mid R\} \to y \in \{x \mid R\})
\]
が成り立つ.
ここで上述のように$y$は$\{x \in a \mid R\}$, $\{x \mid R\}$の中に自由変数として現れないから, 
定義よりこの記号列は(\ref{sthmssetsubsetiset1})と同じである.
故に(\ref{sthmssetsubsetiset1})が成り立つ.
(\ref{sthmssetsubsetiset2})が成り立つことは, 
(\ref{sthmssetsubsetiset1})と推論法則 \ref{dedmp}によって明らかである.
\halmos




\mathstrut
\begin{thm}
\label{sthmalltiset=sset}%定理5.13%新規%確認済
$a$を集合, $R$を関係式とし, $x$を$a$の中に自由変数として現れない文字とする.
このとき
\begin{equation}
\label{sthmalltiset=sset1}
  \forall x(R \to x \in a) \leftrightarrow {\rm Set}_{x}(R) \wedge \{x \mid R\} = \{x \in a \mid R\}
\end{equation}
が成り立つ.
またこのことから, 次の1), 2), 3)が成り立つ.

1)
$\forall x(R \to x \in a)$ならば, $R$は$x$について集合を作り得る.
またこのとき$\{x \mid R\} = \{x \in a \mid R\}$が成り立つ.

2)
$x$が定数でなく, $R \to x \in a$が成り立てば, $R$は$x$について集合を作り得る.
またこのとき$\{x \mid R\} = \{x \in a \mid R\}$が成り立つ.

3)
$R$が$x$について集合を作り得るとき, $\{x \mid R\} = \{x \in a \mid R\}$ならば, 
$\forall x(R \to x \in a)$.
\end{thm}


\noindent{\bf 証明}
~$y$を$x$と異なり, $a$及び$R$の中に自由変数として現れない, 定数でない文字とする.
このときThm \ref{1atb1l1awbla1}より
\begin{equation}
\label{sthmalltiset=sset2}
  ((y|x)(R) \to y \in a) \leftrightarrow (y \in a \wedge (y|x)(R) \leftrightarrow (y|x)(R))
\end{equation}
が成り立つ.
また$x$が$a$の中に自由変数として現れないことから, 
定理 \ref{sthmssetbasis}と推論法則 \ref{dedeqch}により
\[
  y \in a \wedge (y|x)(R) \leftrightarrow y \in \{x \in a \mid R\}
\]
が成り立つ.
故に推論法則 \ref{dedaddeqeq}により
\begin{equation}
\label{sthmalltiset=sset3}
  (y \in a \wedge (y|x)(R) \leftrightarrow (y|x)(R)) 
  \leftrightarrow (y \in \{x \in a \mid R\} \leftrightarrow (y|x)(R))
\end{equation}
が成り立つ.
そこで(\ref{sthmalltiset=sset2}), (\ref{sthmalltiset=sset3})から, 推論法則 \ref{dedeqtrans}によって
\[
  ((y|x)(R) \to y \in a) \leftrightarrow (y \in \{x \in a \mid R\} \leftrightarrow (y|x)(R))
\]
が成り立つ.
ここで$x$は$a$の中に自由変数として現れず, 
変数法則 \ref{valsset}により$\{x \in a \mid R\}$の中にも自由変数として現れないから, 
代入法則 \ref{substfree}, \ref{substfund}, \ref{substequiv}により, 上記の記号列は
\[
  (y|x)(R \to x \in a) \leftrightarrow (y|x)(x \in \{x \in a \mid R\} \leftrightarrow R)
\]
と一致する.
故にこれが成り立つ.
このことと$y$が定数でないことから, 推論法則 \ref{dedalleqquansepconst}により
\[
  \forall y((y|x)(R \to x \in a)) 
  \leftrightarrow \forall y((y|x)(x \in \{x \in a \mid R\} \leftrightarrow R))
\]
が成り立つ.
ここで$y$が$x$と異なり, $a$及び$R$の中に自由変数として現れないことから, 
変数法則 \ref{valfund}, \ref{valequiv}, \ref{valsset}により, 
$y$は$R \to x \in a$及び$x \in \{x \in a \mid R\} \leftrightarrow R$の中に自由変数として現れない.
故に代入法則 \ref{substquantrans}により, 上記の記号列は
\begin{equation}
\label{sthmalltiset=sset4}
  \forall x(R \to x \in a) \leftrightarrow \forall x(x \in \{x \in a \mid R\} \leftrightarrow R)
\end{equation}
と一致する.
従ってこれが成り立つ.
また上述のように$x$は$\{x \in a \mid R\}$の中に自由変数として現れないから, 
定理 \ref{sthmsmbasis&iset=a}より
\begin{equation}
\label{sthmalltiset=sset5}
  \forall x(x \in \{x \in a \mid R\} \leftrightarrow R) 
  \leftrightarrow {\rm Set}_{x}(R) \wedge \{x \mid R\} = \{x \in a \mid R\}
\end{equation}
が成り立つ.
そこで(\ref{sthmalltiset=sset4}), (\ref{sthmalltiset=sset5})から, 
推論法則 \ref{dedeqtrans}によって(\ref{sthmalltiset=sset1})が成り立つ.

\noindent
1), 3)
(\ref{sthmalltiset=sset1})と推論法則 \ref{dedwedge}, \ref{dedeqfund}によって明らか.

\noindent
2)
1)と推論法則 \ref{dedltthmquan}によって明らか.
\halmos




\mathstrut
\begin{thm}
\label{sthmalltisetsubseta}%定理5.14%新規%確認済
$a$を集合, $R$を関係式とし, $x$を$a$の中に自由変数として現れない文字とする.
このとき
\begin{equation}
\label{sthmalltisetsubseta1}
  \forall x(R \to x \in a) \leftrightarrow {\rm Set}_{x}(R) \wedge \{x \mid R\} \subset a
\end{equation}
が成り立つ.
またこのことから, 次の1), 2), 3)が成り立つ.

1)
$\forall x(R \to x \in a)$ならば, $R$は$x$について集合を作り得る.
またこのとき$\{x \mid R\} \subset a$が成り立つ.

2)
$x$が定数でなく, $R \to x \in a$が成り立てば, $R$は$x$について集合を作り得る.
またこのとき$\{x \mid R\} \subset a$が成り立つ.

3)
$R$が$x$について集合を作り得るとき, $\{x \mid R\} \subset a$ならば, 
$\forall x(R \to x \in a)$.
\end{thm}


\noindent{\bf 証明}
~$x$が$a$の中に自由変数として現れないことから, 
定理 \ref{sthmalltiset=sset}と推論法則 \ref{dedequiv}により
\[
  \forall x(R \to x \in a) \to {\rm Set}_{x}(R) \wedge \{x \mid R\} = \{x \in a \mid R\}
\]
が成り立つ.
故に推論法則 \ref{dedprewedge}により
\[
  \forall x(R \to x \in a) \to {\rm Set}_{x}(R)
\]
が成り立つ.
故に推論法則 \ref{dedawblatrue1}により
\[
  {\rm Set}_{x}(R) \wedge \forall x(R \to x \in a) \leftrightarrow \forall x(R \to x \in a)
\]
が成り立つ.
故に推論法則 \ref{dedeqch}により
\begin{equation}
\label{sthmalltisetsubseta2}
  \forall x(R \to x \in a) \leftrightarrow {\rm Set}_{x}(R) \wedge \forall x(R \to x \in a)
\end{equation}
が成り立つ.
また$x$が$a$の中に自由変数として現れないことから, 定理 \ref{sthmsmtiset&asubset}より
\[
  {\rm Set}_{x}(R) \to (\forall x(R \to x \in a) \leftrightarrow \{x \mid R\} \subset a)
\]
が成り立つ.
故に推論法則 \ref{dedeq&w}により
\begin{equation}
\label{sthmalltisetsubseta3}
  {\rm Set}_{x}(R) \wedge \forall x(R \to x \in a) 
  \leftrightarrow {\rm Set}_{x}(R) \wedge \{x \mid R\} \subset a
\end{equation}
が成り立つ.
そこで(\ref{sthmalltisetsubseta2}), (\ref{sthmalltisetsubseta3})から, 
推論法則 \ref{dedeqtrans}によって(\ref{sthmalltisetsubseta1})が成り立つ.

\noindent
1), 3)
(\ref{sthmalltisetsubseta1})と推論法則 \ref{dedwedge}, \ref{dedeqfund}によって明らか.

\noindent
2)
1)と推論法則 \ref{dedltthmquan}によって明らか.
\halmos




\mathstrut
\begin{thm}
\label{sthmssetsubset}%定理5.15%確認済
$a$と$b$を集合, $R$を関係式とし, $x$を$a$と$b$の中に自由変数として現れない文字とする.
このとき
\begin{equation}
\label{sthmssetsubset1}
  a \subset b \to \{x \in a \mid R\} \subset \{x \in b \mid R\}
\end{equation}
が成り立つ.
またこのことから, 次の(\ref{sthmssetsubset2})が成り立つ.
\begin{equation}
\label{sthmssetsubset2}
  a \subset b \text{ならば,} ~\{x \in a \mid R\} \subset \{x \in b \mid R\}.
\end{equation}
\end{thm}


\noindent{\bf 証明}
~$y$を$a$, $b$, $R$の中に自由変数として現れない, 定数でない文字とする.
このとき定理 \ref{sthmsubsetbasis}より
\begin{equation}
\label{sthmssetsubset3}
  a \subset b \to (y \in a \to y \in b)
\end{equation}
が成り立つ.
またThm \ref{1atb1t1awctbwc1}より
\begin{equation}
\label{sthmssetsubset4}
  (y \in a \to y \in b) \to (y \in a \wedge (y|x)(R) \to y \in b \wedge (y|x)(R))
\end{equation}
が成り立つ.
また$x$が$a$, $b$の中に自由変数として現れないことから, 定理 \ref{sthmssetbasis}より
\[
  y \in \{x \in a \mid R\} \leftrightarrow y \in a \wedge (y|x)(R), ~~
  y \in \{x \in b \mid R\} \leftrightarrow y \in b \wedge (y|x)(R)
\]
が共に成り立つから, 推論法則 \ref{dedaddeqt}により
\[
  (y \in \{x \in a \mid R\} \to y \in \{x \in b \mid R\}) 
  \leftrightarrow (y \in a \wedge (y|x)(R) \to y \in b \wedge (y|x)(R))
\]
が成り立つ.
故に推論法則 \ref{dedequiv}により
\begin{equation}
\label{sthmssetsubset5}
  (y \in a \wedge (y|x)(R) \to y \in b \wedge (y|x)(R)) 
  \to (y \in \{x \in a \mid R\} \to y \in \{x \in b \mid R\})
\end{equation}
が成り立つ.
そこで(\ref{sthmssetsubset3})---(\ref{sthmssetsubset5})から, 推論法則 \ref{dedmmp}によって
\begin{equation}
\label{sthmssetsubset6}
  a \subset b \to (y \in \{x \in a \mid R\} \to y \in \{x \in b \mid R\})
\end{equation}
が成り立つことがわかる.
ここで$y$が$a$, $b$の中に自由変数として現れないことから, 
変数法則 \ref{valsubset}により$y$は$a \subset b$の中に自由変数として現れない.
また$y$は定数でない.
そこでこれらのことと(\ref{sthmssetsubset6})が成り立つことから, 
推論法則 \ref{dedalltquansepfreeconst}により
\[
  a \subset b \to \forall y(y \in \{x \in a \mid R\} \to y \in \{x \in b \mid R\})
\]
が成り立つ.
ここで$y$が$a$, $b$, $R$の中に自由変数として現れないことから, 
変数法則 \ref{valsset}により$y$は$\{x \in a \mid R\}$, $\{x \in b \mid R\}$の中に自由変数として現れない.
故に上記の記号列は(\ref{sthmssetsubset1})と同じである.
従って(\ref{sthmssetsubset1})が成り立つ.
(\ref{sthmssetsubset2})が成り立つことは, 
(\ref{sthmssetsubset1})と推論法則 \ref{dedmp}によって明らかである.
\halmos




\mathstrut
\begin{thm}
\label{sthmsset=}%定理5.16%確認済
$a$と$b$を集合, $R$を関係式とし, $x$を$a$と$b$の中に自由変数として現れない文字とする.
このとき
\begin{equation}
\label{sthmsset=1}
  a = b \to \{x \in a \mid R\} = \{x \in b \mid R\}
\end{equation}
が成り立つ.
またこのことから, 次の(\ref{sthmsset=2})が成り立つ.
\begin{equation}
\label{sthmsset=2}
  a = b \text{ならば,} ~\{x \in a \mid R\} = \{x \in b \mid R\}.
\end{equation}
\end{thm}


\noindent{\bf 証明}
~$y$を$x$と異なり, $R$の中に自由変数として現れない文字とする.
このときThm \ref{T=Ut1TV=UV1}より
\[
  a = b \to (a|y)(\{x \in y \mid R\}) = (b|y)(\{x \in y \mid R\})
\]
が成り立つ.
ここで$x$と$y$が互いに異なり, $x$が$a$, $b$の中に自由変数として現れず, 
$y$が$R$の中に自由変数として現れないことから, 
代入法則 \ref{substfree}, \ref{substsset}によってわかるように, 
この記号列は(\ref{sthmsset=1})と一致する.
故に(\ref{sthmsset=1})が成り立つ.
(\ref{sthmsset=2})が成り立つことは, (\ref{sthmsset=1})と推論法則 \ref{dedmp}によって明らかである.
\halmos




\mathstrut
\begin{thm}
\label{sthmalltssetsubset}%定理5.17%確認済
$a$を集合, $R$と$S$を関係式とし, $x$を$a$の中に自由変数として現れない文字とする.
このとき
\begin{equation}
\label{sthmalltssetsubset1}
  (\forall x \in a)(R \to S) \leftrightarrow \{x \in a \mid R\} \subset \{x \in a \mid S\}
\end{equation}
が成り立つ.
特に
\begin{equation}
\label{sthmalltssetsubset2}
  \forall x(R \to S) \to \{x \in a \mid R\} \subset \{x \in a \mid S\}
\end{equation}
が成り立つ.
またこれらから, 次の1)---4)が成り立つ.

1)
$(\forall x \in a)(R \to S)$ならば, $\{x \in a \mid R\} \subset \{x \in a \mid S\}$.
また$\{x \in a \mid R\} \subset \{x \in a \mid S\}$ならば, $(\forall x \in a)(R \to S)$.

2)
$x$が定数でなく, $x \in a \to (R \to S)$が成り立てば, $\{x \in a \mid R\} \subset \{x \in a \mid S\}$.

3)
$\forall x(R \to S)$ならば, $\{x \in a \mid R\} \subset \{x \in a \mid S\}$.

4)
$x$が定数でなく, $R \to S$が成り立てば, $\{x \in a \mid R\} \subset \{x \in a \mid S\}$.
\end{thm}


\noindent{\bf 証明}
~まず(\ref{sthmalltssetsubset1})が成り立つことを示す.
Thm \ref{thmspallfund}より
\begin{equation}
\label{sthmalltssetsubset3}
  (\forall x \in a)(R \to S) \leftrightarrow \forall x(x \in a \to (R \to S))
\end{equation}
が成り立つ.
また$y$を$x$と異なり, $a$, $R$, $S$の中に自由変数として現れない, 定数でない文字とするとき, 
Thm \ref{1ct1atb11l1awctbwc1}より
\[
  (y \in a \to ((y|x)(R) \to (y|x)(S))) 
  \leftrightarrow (y \in a \wedge (y|x)(R) \to y \in a \wedge (y|x)(S))
\]
が成り立つ.
ここで$x$が$a$の中に自由変数として現れないことと
代入法則 \ref{substfree}, \ref{substfund}, \ref{substwedge}によれば, この記号列は
\[
  (y|x)(x \in a \to (R \to S)) \leftrightarrow (y|x)(x \in a \wedge R \to x \in a \wedge S)
\]
と一致する.
故にこれが成り立つ.
このことと$y$が定数でないことから, 推論法則 \ref{dedalleqquansepconst}により
\[
  \forall y((y|x)(x \in a \to (R \to S))) 
  \leftrightarrow \forall y((y|x)(x \in a \wedge R \to x \in a \wedge S))
\]
が成り立つ.
ここで$y$が$x$と異なり, $a$, $R$, $S$の中に自由変数として現れないことから, 
変数法則 \ref{valfund}, \ref{valwedge}により, 
$y$は$x \in a \to (R \to S)$, $x \in a \wedge R \to x \in a \wedge S$の中に自由変数として現れない.
故に代入法則 \ref{substquantrans}により, 上記の記号列は
\begin{equation}
\label{sthmalltssetsubset4}
  \forall x(x \in a \to (R \to S)) \leftrightarrow \forall x(x \in a \wedge R \to x \in a \wedge S)
\end{equation}
と一致する.
従ってこれが成り立つ.
また$x$が$a$の中に自由変数として現れないことから, 定理 \ref{sthmssetsm}より
$x \in a \wedge R$と$x \in a \wedge S$は共に$x$について集合を作り得る.
故に定理 \ref{sthmsmtalltisetsubseteq}より, 
\[
  \forall x(x \in a \wedge R \to x \in a \wedge S) 
  \leftrightarrow \{x \mid x \in a \wedge R\} \subset \{x \mid x \in a \wedge S\}, 
\]
即ち
\begin{equation}
\label{sthmalltssetsubset5}
  \forall x(x \in a \wedge R \to x \in a \wedge S) 
  \leftrightarrow \{x \in a \mid R\} \subset \{x \in a \mid S\}
\end{equation}
が成り立つ.
そこで(\ref{sthmalltssetsubset3})---(\ref{sthmalltssetsubset5})から, 
推論法則 \ref{dedeqtrans}によって(\ref{sthmalltssetsubset1})が成り立つことがわかる.

次に(\ref{sthmalltssetsubset2})が成り立つことを示す.
Thm \ref{thmquantspall2}より
\begin{equation}
\label{sthmalltssetsubset6}
  \forall x(R \to S) \to (\forall x \in a)(R \to S)
\end{equation}
が成り立つ.
また(\ref{sthmalltssetsubset1})が成り立つことから, 推論法則 \ref{dedequiv}により
\begin{equation}
\label{sthmalltssetsubset7}
  (\forall x \in a)(R \to S) \to \{x \in a \mid R\} \subset \{x \in a \mid S\}
\end{equation}
が成り立つ.
そこで(\ref{sthmalltssetsubset6}), (\ref{sthmalltssetsubset7})から, 
推論法則 \ref{dedmmp}によって(\ref{sthmalltssetsubset2})が成り立つ.

\noindent
1)
(\ref{sthmalltssetsubset1})と推論法則 \ref{dedeqfund}によって明らか.

\noindent
2)
1)と推論法則 \ref{dedspallfund}によって明らか.

\noindent
3)
(\ref{sthmalltssetsubset2})と推論法則 \ref{dedmp}によって明らか.

\noindent
4)
3)と推論法則 \ref{dedltthmquan}によって明らか.
\halmos




\mathstrut
\begin{thm}
\label{sthmalleqsset=}%定理5.18%一部変更%確認済
$a$を集合, $R$と$S$を関係式とし, $x$を文字とする.
このとき
\begin{equation}
\label{sthmalleqsset=1}
  (\forall x \in a)(R \leftrightarrow S) \to \{x \in a \mid R\} = \{x \in a \mid S\}
\end{equation}
が成り立つ.
特に
\begin{equation}
\label{sthmalleqsset=2}
  \forall x(R \leftrightarrow S) \to \{x \in a \mid R\} = \{x \in a \mid S\}
\end{equation}
が成り立つ.
またこれらから, 次の1)---4)が成り立つ.

1)
$(\forall x \in a)(R \leftrightarrow S)$ならば, $\{x \in a \mid R\} = \{x \in a \mid S\}$.

2)
$x$が定数でなく, $x \in a \to (R \leftrightarrow S)$が成り立てば, 
$\{x \in a \mid R\} = \{x \in a \mid S\}$.

3)
$\forall x(R \leftrightarrow S)$ならば, $\{x \in a \mid R\} = \{x \in a \mid S\}$.

4)
$x$が定数でなく, $R \leftrightarrow S$が成り立てば, $\{x \in a \mid R\} = \{x \in a \mid S\}$.
\end{thm}


\noindent{\bf 証明}
~Thm \ref{thmspallfund}と推論法則 \ref{dedequiv}により
\begin{equation}
\label{sthmalleqsset=3}
  (\forall x \in a)(R \leftrightarrow S) \to \forall x(x \in a \to (R \leftrightarrow S))
\end{equation}
が成り立つ.
また$y$を$x$と異なり, $a$, $R$, $S$の中に自由変数として現れない, 定数でない文字とするとき, 
Thm \ref{1ct1alb11l1awclbwc1}と推論法則 \ref{dedequiv}により
\[
  ((y|x)(x \in a) \to ((y|x)(R) \leftrightarrow (y|x)(S))) 
  \to ((y|x)(x \in a) \wedge (y|x)(R) \leftrightarrow (y|x)(x \in a) \wedge (y|x)(S))
\]
が成り立つが, 代入法則 \ref{substfund}, \ref{substwedge}, \ref{substequiv}によればこの記号列は
\[
  (y|x)(x \in a \to (R \leftrightarrow S)) 
  \to (y|x)(x \in a \wedge R \leftrightarrow x \in a \wedge S)
\]
と一致するから, これが成り立つ.
このことと$y$が定数でないことから, 推論法則 \ref{dedalltquansepconst}により
\[
  \forall y((y|x)(x \in a \to (R \leftrightarrow S))) 
  \to \forall y((y|x)(x \in a \wedge R \leftrightarrow x \in a \wedge S))
\]
が成り立つ.
ここで$y$が$x$と異なり, $a$, $R$, $S$の中に自由変数として現れないことから, 
変数法則 \ref{valfund}, \ref{valwedge}, \ref{valequiv}により, 
$y$は$x \in a \to (R \leftrightarrow S)$, 
$x \in a \wedge R \leftrightarrow x \in a \wedge S$の中に自由変数として現れない.
故に代入法則 \ref{substquantrans}によれば, 上記の記号列は
\begin{equation}
\label{sthmalleqsset=4}
  \forall x(x \in a \to (R \leftrightarrow S)) 
  \to \forall x(x \in a \wedge R \leftrightarrow x \in a \wedge S)
\end{equation}
と一致する.
従ってこれが成り立つ.
また定理 \ref{sthmalleqiset=}より, 
\[
  \forall x(x \in a \wedge R \leftrightarrow x \in a \wedge S) 
  \to \{x \mid x \in a \wedge R\} = \{x \mid x \in a \wedge S\}, 
\]
即ち
\begin{equation}
\label{sthmalleqsset=5}
  \forall x(x \in a \wedge R \leftrightarrow x \in a \wedge S) 
  \to \{x \in a \mid R\} = \{x \in a \mid S\}
\end{equation}
が成り立つ.
そこで(\ref{sthmalleqsset=3})---(\ref{sthmalleqsset=5})から, 
推論法則 \ref{dedmmp}によって(\ref{sthmalleqsset=1})が成り立つことがわかる.
またThm \ref{thmquantspall2}より
\[
  \forall x(R \leftrightarrow S) \to (\forall x \in a)(R \leftrightarrow S)
\]
が成り立つから, これと(\ref{sthmalleqsset=1})から, 
推論法則 \ref{dedmmp}によって(\ref{sthmalleqsset=2})が成り立つ.

\noindent
1)
(\ref{sthmalleqsset=1})と推論法則 \ref{dedmp}によって明らか.

\noindent
2)
1)と推論法則 \ref{dedspallfund}によって明らか.

\noindent
3)
(\ref{sthmalleqsset=2})と推論法則 \ref{dedmp}によって明らか.

\noindent
4)
3)と推論法則 \ref{dedltthmquan}によって明らか.
\halmos




\mathstrut
\begin{thm}
\label{sthmalleqsset=eq}%定理5.19%新規%確認済
$a$を集合, $R$と$S$を関係式とし, $x$を$a$の中に自由変数として現れない文字とする.
このとき
\begin{equation}
\label{sthmalleqsset=eq1}
  (\forall x \in a)(R \leftrightarrow S) \leftrightarrow \{x \in a \mid R\} = \{x \in a \mid S\}
\end{equation}
が成り立つ.
またこのことから, 次の(\ref{sthmalleqsset=eq2})が成り立つ.
\begin{equation}
\label{sthmalleqsset=eq2}
  \{x \in a \mid R\} = \{x \in a \mid S\} \text{ならば,} ~(\forall x \in a)(R \leftrightarrow S).
\end{equation}
\end{thm}


\noindent{\bf 証明}
~Thm \ref{thmspallw}より
\begin{equation}
\label{sthmalleqsset=eq3}
  (\forall x \in a)(R \leftrightarrow S) 
  \leftrightarrow (\forall x \in a)(R \to S) \wedge (\forall x \in a)(S \to R)
\end{equation}
が成り立つ.
また$x$が$a$の中に自由変数として現れないことから, 定理 \ref{sthmalltssetsubset}より
\[
  (\forall x \in a)(R \to S) \leftrightarrow \{x \in a \mid R\} \subset \{x \in a \mid S\}, ~~
  (\forall x \in a)(S \to R) \leftrightarrow \{x \in a \mid S\} \subset \{x \in a \mid R\}
\]
が共に成り立つから, 推論法則 \ref{dedaddeqw}により
\begin{equation}
\label{sthmalleqsset=eq4}
  (\forall x \in a)(R \to S) \wedge (\forall x \in a)(S \to R) 
  \leftrightarrow \{x \in a \mid R\} \subset \{x \in a \mid S\} \wedge \{x \in a \mid S\} \subset \{x \in a \mid R\}
\end{equation}
が成り立つ.
また定理 \ref{sthmaxiom1}より
\begin{equation}
\label{sthmalleqsset=eq5}
  \{x \in a \mid R\} \subset \{x \in a \mid S\} \wedge \{x \in a \mid S\} \subset \{x \in a \mid R\}
  \leftrightarrow \{x \in a \mid R\} = \{x \in a \mid S\}
\end{equation}
が成り立つ.
そこで(\ref{sthmalleqsset=eq3})---(\ref{sthmalleqsset=eq5})から, 
推論法則 \ref{dedeqtrans}によって(\ref{sthmalleqsset=eq1})が成り立つことがわかる.
(\ref{sthmalleqsset=eq2})が成り立つことは, 
(\ref{sthmalleqsset=eq1})と推論法則 \ref{dedeqfund}によって明らかである.
\halmos




\mathstrut
\begin{thm}
\label{sthmspinsset}%定理5.20%新規%確認済
$a$を集合, $R$と$S$を関係式とし, $x$を$a$の中に自由変数として現れない文字とする.
このとき
\begin{align}
  \label{sthmspinsset1}
  &(\exists x \in \{x \in a \mid R\})(S) \leftrightarrow (\exists x \in a)(R \wedge S), \\
  \mbox{} \notag \\
  \label{sthmspinsset2}
  &(\forall x \in \{x \in a \mid R\})(S) \leftrightarrow (\forall x \in a)(R \to S), \\
  \mbox{} \notag \\
  \label{sthmspinsset3}
  &(!x \in \{x \in a \mid R\})(S) \leftrightarrow (!x \in a)(R \wedge S), \\
  \mbox{} \notag \\
  \label{sthmspinsset4}
  &(\exists !x \in \{x \in a \mid R\})(S) \leftrightarrow (\exists !x \in a)(R \wedge S)
\end{align}
がすべて成り立つ.
またこれらから, 次の1)---4)が成り立つ.

1)
$(\exists x \in \{x \in a \mid R\})(S)$ならば, $(\exists x \in a)(R \wedge S)$.
また$(\exists x \in a)(R \wedge S)$ならば, $(\exists x \in \{x \in a \mid R\})(S)$.

2)
$(\forall x \in \{x \in a \mid R\})(S)$ならば, $(\forall x \in a)(R \to S)$.
また$(\forall x \in a)(R \to S)$ならば, $(\forall x \in \{x \in a \mid R\})(S)$.

3)
$(!x \in \{x \in a \mid R\})(S)$ならば, $(!x \in a)(R \wedge S)$.
また$(!x \in a)(R \wedge S)$ならば, $(!x \in \{x \in a \mid R\})(S)$.

4)
$(\exists !x \in \{x \in a \mid R\})(S)$ならば, $(\exists !x \in a)(R \wedge S)$.
また$(\exists !x \in a)(R \wedge S)$ならば, $(\exists !x \in \{x \in a \mid R\})(S)$.
\end{thm}


\noindent{\bf 証明}
~まず(\ref{sthmspinsset1}), (\ref{sthmspinsset3}), (\ref{sthmspinsset4})が成り立つことを示す.
$x$が$a$の中に自由変数として現れないことから, 
定理 \ref{sthmssetsm}より$x \in a \wedge R$は$x$について集合を作り得る.
故に定理 \ref{sthmspiniset}より
\[
  (\exists x \in \{x \mid x \in a \wedge R\})(S) \leftrightarrow \exists_{x \in a \wedge R}x(S), ~~
  (!x \in \{x \mid x \in a \wedge R\})(S) \leftrightarrow \ !_{x \in a \wedge R}x(S), 
\]
即ち
\begin{align}
  \label{sthmspinsset5}
  &(\exists x \in \{x \in a \mid R\})(S) \leftrightarrow \exists x((x \in a \wedge R) \wedge S), \\
  \mbox{} \notag \\
  \label{sthmspinsset6}
  &(!x \in \{x \in a \mid R\})(S) \leftrightarrow \ !x((x \in a \wedge R) \wedge S)
\end{align}
が共に成り立つ.
また$y$を$x$と異なり, $a$, $R$, $S$の中に自由変数として現れない, 定数でない文字とするとき, 
Thm \ref{1awb1wclaw1bwc1}より
\[
  (y \in a \wedge (y|x)(R)) \wedge (y|x)(S) \leftrightarrow y \in a \wedge ((y|x)(R) \wedge (y|x)(S))
\]
が成り立つ.
ここで$x$が$a$の中に自由変数として現れないことから, 
代入法則 \ref{substfree}, \ref{substfund}, \ref{substwedge}, \ref{substequiv}よりこの記号列は
\[
  (y|x)((x \in a \wedge R) \wedge S \leftrightarrow x \in a \wedge (R \wedge S))
\]
と一致する.
故にこれが成り立つ.
このことと$y$が定数でないことから, 推論法則 \ref{dedltthmquan}により
\[
  \forall y((y|x)((x \in a \wedge R) \wedge S \leftrightarrow x \in a \wedge (R \wedge S)))
\]
が成り立つ.
ここで$y$が$x$と異なり, $a$, $R$, $S$の中に自由変数として現れないことから, 
変数法則 \ref{valfund}, \ref{valwedge}, \ref{valequiv}により, $y$は
$(x \in a \wedge R) \wedge S \leftrightarrow x \in a \wedge (R \wedge S)$の中に自由変数として現れない.
故に代入法則 \ref{substquantrans}により, 上記の記号列は
\[
  \forall x((x \in a \wedge R) \wedge S \leftrightarrow x \in a \wedge (R \wedge S))
\]
と一致する.
従ってこれが成り立つ.
そこで推論法則 \ref{dedalleqquansep}, \ref{dedalleq!sep}により, 
\[
  \exists x((x \in a \wedge R) \wedge S) \leftrightarrow \exists x(x \in a \wedge (R \wedge S)), ~~
  !x((x \in a \wedge R) \wedge S) \leftrightarrow \ !x(x \in a \wedge (R \wedge S)), 
\]
即ち
\begin{align}
  \label{sthmspinsset7}
  &\exists x((x \in a \wedge R) \wedge S) \leftrightarrow (\exists x \in a)(R \wedge S), \\
  \mbox{} \notag \\
  \label{sthmspinsset8}
  &!x((x \in a \wedge R) \wedge S) \leftrightarrow (!x \in a)(R \wedge S)
\end{align}
がそれぞれ成り立つ.
以上の(\ref{sthmspinsset5})と(\ref{sthmspinsset7}), (\ref{sthmspinsset6})と(\ref{sthmspinsset8})から, 
それぞれ推論法則 \ref{dedeqtrans}によって(\ref{sthmspinsset1}), (\ref{sthmspinsset3})が成り立つ.
またこれらから, 推論法則 \ref{dedaddeqw}により(\ref{sthmspinsset4})も成り立つ.

次に(\ref{sthmspinsset2})が成り立つことを示す.
いま示したように(\ref{sthmspinsset1})が成り立つが, 
(\ref{sthmspinsset1})における$S$は任意の関係式で良いので, $S$を$\neg S$とした
\begin{equation}
\label{sthmspinsset9}
  (\exists x \in \{x \in a \mid R\})(\neg S) \leftrightarrow (\exists x \in a)(R \wedge \neg S)
\end{equation}
も成り立つ.
またThm \ref{thmspquantweq}と推論法則 \ref{dedeqch}により
\begin{equation}
\label{sthmspinsset10}
  (\exists x \in a)(R \wedge \neg S) \leftrightarrow (\exists x \in a)(\neg (R \to S))
\end{equation}
が成り立つ.
そこで(\ref{sthmspinsset9}), (\ref{sthmspinsset10})から, 推論法則 \ref{dedeqtrans}によって
\[
  (\exists x \in \{x \in a \mid R\})(\neg S) \leftrightarrow (\exists x \in a)(\neg (R \to S))
\]
が成り立つ.
故に推論法則 \ref{dedeqcp}により, 
\[
  \neg (\exists x \in \{x \in a \mid R\})(\neg S) 
  \leftrightarrow \neg (\exists x \in a)(\neg (R \to S)), 
\]
即ち(\ref{sthmspinsset2})が成り立つ.

1)---4)が成り立つことは, (\ref{sthmspinsset1})---(\ref{sthmspinsset4})と
推論法則 \ref{dedeqfund}によって明らかである.
\halmos




\mathstrut
{\small
\noindent
{\bf 例 1.}~%例5.1%新規%確認済
$a$を集合とし, $x$を$a$の中に自由変数として現れない文字とする.
このとき
\[
  a \notin \{x \in a \mid x \notin x\}, ~~
  \{x \in a \mid x \notin x\} \notin a
\]
が共に成り立つ.

実際$\{x \in a \mid x \notin x\}$を$b$と書けば, $b$は集合である.
また$x$が$a$の中に自由変数として現れないことから, 定理 \ref{sthmssetbasis}より
任意の集合$T$に対して
\[
  T \in b \leftrightarrow T \in a \wedge (T|x)(x \notin x)
\]
が成り立つ.
ここで代入法則 \ref{substfund}によれば, この記号列は
\[
  T \in b \leftrightarrow T \in a \wedge T \notin T
\]
と一致するから, これが成り立つ.
従って特に$T$として$a$, $b$をそれぞれ取れば, 
\begin{align}
  \label{ex5.1.1}
  &a \in b \leftrightarrow a \in a \wedge a \notin a, \\
  \mbox{} \notag \\
  \label{ex5.1.2}
  &b \in b \leftrightarrow b \in a \wedge b \notin b
\end{align}
が成り立つことがわかる.
いまThm \ref{n1awna1}より$\neg (a \in a \wedge a \notin a)$が成り立つから, 
これと(\ref{ex5.1.1})から, 推論法則 \ref{dedeqfund}により$a \notin b$が成り立つ.
また(\ref{ex5.1.2})から, 推論法則 \ref{dedequiv}により
\[
  b \in b \to b \in a \wedge b \notin b
\]
が成り立つから, 推論法則 \ref{dedprewedge}により
\[
  b \in b \to b \notin b
\]
が成り立つ.
故に推論法則 \ref{dedatna}により
\begin{equation}
\label{ex5.1.3}
  b \notin b
\end{equation}
が成り立つ.
故に推論法則 \ref{dedatawbtrue2}により
\begin{equation}
\label{ex5.1.4}
  b \in a \to b \in a \wedge b \notin b
\end{equation}
が成り立つ.
また(\ref{ex5.1.2})から, 推論法則 \ref{dedequiv}により
\begin{equation}
\label{ex5.1.5}
  b \in a \wedge b \notin b \to b \in b
\end{equation}
が成り立つ.
そこで(\ref{ex5.1.4}), (\ref{ex5.1.5})から, 推論法則 \ref{dedmmp}によって
\[
  b \in a \to b \in b
\]
が成り立つ.
故に推論法則 \ref{dedcp}により
\[
  b \notin b \to b \notin a
\]
が成り立つ.
そこでこれと(\ref{ex5.1.3})から, 推論法則 \ref{dedmp}によって
$b \notin a$が成り立つ. ------
}




\mathstrut
{\small
\noindent
{\bf 例 2.}~%例5.2%新規%確認済
$x$と$y$を異なる文字とするとき, 
\[
  \neg \exists x(\forall y(x \in y)), ~~
  \neg \exists y(\forall x(x \in y))
\]
が共に成り立つ.

実際$z$を$x$, $y$と異なる定数でない文字とし, $\{x \in z \mid x \notin x\}$を$T$と書く.
このとき$T$は集合であり, $x$が$z$と異なることから, 例1より
\[
  z \notin T, ~~
  T \notin z
\]
が共に成り立つ.
ここで$x$と$y$が共に$z$と異なることと代入法則 \ref{substfund}から, これらの記号列はそれぞれ
\[
  (T|y)(z \notin y), ~~
  (T|x)(x \notin z)
\]
と一致する.
故にこれらが共に成り立つ.
そこで推論法則 \ref{deds4}により
\[
  \exists y(z \notin y), ~~
  \exists x(x \notin z)
\]
が共に成り立つ.
ここで$x$と$y$が異なることと代入法則 \ref{substfund}から, これらの記号列はそれぞれ
\[
  \exists y((z|x)(x \notin y)), ~~
  \exists x((z|y)(x \notin y))
\]
と一致する.
また$x$, $y$, $z$がすべて異なることから, 代入法則 \ref{substquan}により
これらの記号列はそれぞれ
\[
  (z|x)(\exists y(x \notin y)), ~~
  (z|y)(\exists x(x \notin y))
\]
と一致する.
故にこれらが共に成り立つ.
このことと$z$が定数でないことから, 推論法則 \ref{dedltthmquan}により
\[
  \forall z((z|x)(\exists y(x \notin y))), ~~
  \forall z((z|y)(\exists x(x \notin y)))
\]
が共に成り立つ.
ここで$z$が$x$, $y$と異なることから, 変数法則 \ref{valfund}, \ref{valquan}により, 
$z$は$\exists y(x \notin y)$及び$\exists x(x \notin y)$の中に自由変数として現れない.
故に代入法則 \ref{substquantrans}により, 上記の記号列はそれぞれ
\[
  \forall x(\exists y(x \notin y)), ~~
  \forall y(\exists x(x \notin y))
\]
と一致する.
従ってこれらが共に成り立つ.
故に推論法則 \ref{dedquangdm}により, 
\[
  \neg \exists x(\forall y(x \in y)), ~~
  \neg \exists y(\forall x(x \in y))
\]
が共に成り立つ. ------
}




\mathstrut%註%確認済
{\small
\noindent
\textbf{註.} 
上の例2に挙げた二つの定理は, 内容的にはそれぞれ``すべての集合に含まれる集合は存在しない'', 
``すべての集合を含む集合は存在しない''ことを意味している.
このうち前者については, 後で見るように, より強く
\[
  \exists y(\forall x(x \notin y)) ~~~~\text{($x$と$y$は異なる文字)}
\]
が成り立つ.
これは内容的には, ``どんな集合をも含まない集合 (即ち空集合) が存在する''ということである.
}




\mathstrut
\begin{thm}
\label{sthmisetsset}%定理5.21%新規%確認済
$R$と$S$を関係式とし, $x$を文字とする.
このとき
\begin{align}
  \label{sthmisetsset1}
  &{\rm Set}_{x}(R) \to \{x \in \{x \mid R\} \mid S\} = \{x \mid R \wedge S\}, \\
  \mbox{} \notag \\
  \label{sthmisetsset2}
  &{\rm Set}_{x}(R) \to \{x \in \{x \mid R\} \mid S\} = \{x \mid S \wedge R\}
\end{align}
が共に成り立つ.
またこれらから, 次の(\ref{sthmisetsset3})が成り立つ.
\begin{align}
\label{sthmisetsset3}
  &R \text{が} x \text{について集合を作り得るならば,} ~
  \{x \in \{x \mid R\} \mid S\} = \{x \mid R \wedge S\} \text{と} \\
  &\{x \in \{x \mid R\} \mid S\} = \{x \mid S \wedge R\} \text{が共に成り立つ.} \notag
\end{align}
\end{thm}


\noindent{\bf 証明}
~$y$を$x$と異なり, $R$, $S$の中に自由変数として現れない, 定数でない文字とする.
このとき定理 \ref{sthmisetbasis}より
\begin{equation}
\label{sthmisetsset4}
  {\rm Set}_{x}(R) \to (y \in \{x \mid R\} \leftrightarrow (y|x)(R))
\end{equation}
が成り立つ.
またThm \ref{1alb1t1awclbwc1}より
\begin{equation}
\label{sthmisetsset5}
  (y \in \{x \mid R\} \leftrightarrow (y|x)(R)) 
  \to (y \in \{x \mid R\} \wedge (y|x)(S) \leftrightarrow (y|x)(R) \wedge (y|x)(S))
\end{equation}
が成り立つ.
そこで(\ref{sthmisetsset4}), (\ref{sthmisetsset5})から, 推論法則 \ref{dedmmp}によって
\begin{equation}
\label{sthmisetsset6}
  {\rm Set}_{x}(R) \to (y \in \{x \mid R\} \wedge (y|x)(S) \leftrightarrow (y|x)(R) \wedge (y|x)(S))
\end{equation}
が成り立つ.
またThm \ref{awblbwa}より
\[
  (y|x)(R) \wedge (y|x)(S) \leftrightarrow (y|x)(S) \wedge (y|x)(R)
\]
が成り立つから, 推論法則 \ref{dedaddeqeq}により
\[
  (y \in \{x \mid R\} \wedge (y|x)(S) \leftrightarrow (y|x)(R) \wedge (y|x)(S)) 
  \leftrightarrow (y \in \{x \mid R\} \wedge (y|x)(S) \leftrightarrow (y|x)(S) \wedge (y|x)(R))
\]
が成り立つ.
故に推論法則 \ref{dedequiv}により
\begin{multline}
\label{sthmisetsset7}
  (y \in \{x \mid R\} \wedge (y|x)(S) \leftrightarrow (y|x)(R) \wedge (y|x)(S)) \\
  \to (y \in \{x \mid R\} \wedge (y|x)(S) \leftrightarrow (y|x)(S) \wedge (y|x)(R))
\end{multline}
が成り立つ.
そこで(\ref{sthmisetsset6}), (\ref{sthmisetsset7})から, 推論法則 \ref{dedmmp}によって
\begin{equation}
\label{sthmisetsset8}
  {\rm Set}_{x}(R) \to (y \in \{x \mid R\} \wedge (y|x)(S) \leftrightarrow (y|x)(S) \wedge (y|x)(R))
\end{equation}
が成り立つ.
ここで変数法則 \ref{valiset}により, $x$は$\{x \mid R\}$の中に自由変数として現れないから, 
代入法則 \ref{substfree}, \ref{substfund}, \ref{substwedge}, \ref{substequiv}により, 
(\ref{sthmisetsset6}), (\ref{sthmisetsset8})はそれぞれ
\begin{align}
  \label{sthmisetsset9}
  &{\rm Set}_{x}(R) \to (y|x)(x \in \{x \mid R\} \wedge S \leftrightarrow R \wedge S), \\
  \mbox{} \notag \\
  \label{sthmisetsset10}
  &{\rm Set}_{x}(R) \to (y|x)(x \in \{x \mid R\} \wedge S \leftrightarrow S \wedge R)
\end{align}
と一致する.
故にこれらが共に成り立つ.
さていま$y$は$R$の中に自由変数として現れないから, 
変数法則 \ref{valsm}により, $y$は${\rm Set}_{x}(R)$の中に自由変数として現れない.
また$y$は定数でない.
そこでこれらのことと(\ref{sthmisetsset9}), (\ref{sthmisetsset10})が共に成り立つことから, 
推論法則 \ref{dedalltquansepfreeconst}により
\begin{align}
  \label{sthmisetsset11}
  &{\rm Set}_{x}(R) \to \forall y((y|x)(x \in \{x \mid R\} \wedge S \leftrightarrow R \wedge S)), \\
  \mbox{} \notag \\
  \label{sthmisetsset12}
  &{\rm Set}_{x}(R) \to \forall y((y|x)(x \in \{x \mid R\} \wedge S \leftrightarrow S \wedge R))
\end{align}
が共に成り立つ.
ここで$y$が$x$と異なり, $R$, $S$の中に自由変数として現れないことから, 
変数法則 \ref{valfund}, \ref{valwedge}, \ref{valequiv}, \ref{valiset}により, 
$y$は$x \in \{x \mid R\} \wedge S \leftrightarrow R \wedge S$, 
$x \in \{x \mid R\} \wedge S \leftrightarrow S \wedge R$の中に自由変数として現れない.
故に代入法則 \ref{substquantrans}によれば, (\ref{sthmisetsset11}), (\ref{sthmisetsset12})はそれぞれ
\begin{align}
  \label{sthmisetsset13}
  &{\rm Set}_{x}(R) \to \forall x(x \in \{x \mid R\} \wedge S \leftrightarrow R \wedge S), \\
  \mbox{} \notag \\
  \label{sthmisetsset14}
  &{\rm Set}_{x}(R) \to \forall x(x \in \{x \mid R\} \wedge S \leftrightarrow S \wedge R)
\end{align}
と一致する.
従ってこれらが共に成り立つ.
また定理 \ref{sthmalleqiset=}より, 
\begin{align*}
  &\forall x(x \in \{x \mid R\} \wedge S \leftrightarrow R \wedge S) 
  \to \{x \mid x \in \{x \mid R\} \wedge S\} = \{x \mid R \wedge S\}, \\
  \mbox{} \notag \\
  &\forall x(x \in \{x \mid R\} \wedge S \leftrightarrow S \wedge R) 
  \to \{x \mid x \in \{x \mid R\} \wedge S\} = \{x \mid S \wedge R\}, 
\end{align*}
即ち
\begin{align}
  \label{sthmisetsset15}
  &\forall x(x \in \{x \mid R\} \wedge S \leftrightarrow R \wedge S) 
  \to \{x \in \{x \mid R\} \mid S\} = \{x \mid R \wedge S\}, \\
  \mbox{} \notag \\
  \label{sthmisetsset16}
  &\forall x(x \in \{x \mid R\} \wedge S \leftrightarrow S \wedge R) 
  \to \{x \in \{x \mid R\} \mid S\} = \{x \mid S \wedge R\}
\end{align}
が共に成り立つ.
そこで(\ref{sthmisetsset13})と(\ref{sthmisetsset15}), 
(\ref{sthmisetsset14})と(\ref{sthmisetsset16})から, 
それぞれ推論法則 \ref{dedmmp}によって(\ref{sthmisetsset1}), (\ref{sthmisetsset2})が成り立つ.
(\ref{sthmisetsset3})が成り立つことは, 
(\ref{sthmisetsset1}), (\ref{sthmisetsset2})と推論法則 \ref{dedmp}によって明らかである.
\halmos




\mathstrut
\begin{thm}
\label{sthmssetsset}%定理5.22%新規%確認済
$a$を集合, $R$と$S$を関係式とし, $x$を$a$の中に自由変数として現れない文字とする.
このとき
\begin{align}
  \label{sthmssetsset1}
  &\{x \in \{x \in a \mid R\} \mid S\} = \{x \in a \mid R \wedge S\}, \\
  \mbox{} \notag \\
  \label{sthmssetsset2}
  &\{x \in \{x \in a \mid R\} \mid S\} = \{x \in a \mid S \wedge R\}
\end{align}
が共に成り立つ.
\end{thm}


\noindent{\bf 証明}
~まず(\ref{sthmssetsset1})が成り立つことを示す.
$x$が$a$の中に自由変数として現れないことから, 
定理 \ref{sthmssetsm}より$x \in a \wedge R$は$x$について集合を作り得る.
故に定理 \ref{sthmisetsset}より, 
\[
  \{x \in \{x \mid x \in a \wedge R\} \mid S\} = \{x \mid (x \in a \wedge R) \wedge S\}, 
\]
即ち
\begin{equation}
\label{sthmssetsset3}
  \{x \in \{x \in a \mid R\} \mid S\} = \{x \mid (x \in a \wedge R) \wedge S\}
\end{equation}
が成り立つ.
さていま$y$を$x$と異なり, $a$, $R$, $S$の中に自由変数として現れない, 定数でない文字とする.
このときThm \ref{1awb1wclaw1bwc1}より
\[
  ((y|x)(x \in a) \wedge (y|x)(R)) \wedge (y|x)(S) 
  \leftrightarrow (y|x)(x \in a) \wedge ((y|x)(R) \wedge (y|x)(S))
\]
が成り立つが, 代入法則 \ref{substwedge}, \ref{substequiv}によればこの記号列は
\[
  (y|x)((x \in a \wedge R) \wedge S \leftrightarrow x \in a \wedge (R \wedge S))
\]
と一致するから, これが成り立つ.
このことと$y$が定数でないことから, 推論法則 \ref{dedltthmquan}により
\[
  \forall y((y|x)((x \in a \wedge R) \wedge S \leftrightarrow x \in a \wedge (R \wedge S)))
\]
が成り立つ.
ここで$y$が$x$と異なり, $a$, $R$, $S$の中に自由変数として現れないことから, 
変数法則 \ref{valfund}, \ref{valwedge}, \ref{valequiv}により, 
$y$は$(x \in a \wedge R) \wedge S \leftrightarrow x \in a \wedge (R \wedge S)$の中に
自由変数として現れない.
故に代入法則 \ref{substquantrans}によれば, 上記の記号列は
\[
  \forall x((x \in a \wedge R) \wedge S \leftrightarrow x \in a \wedge (R \wedge S))
\]
と一致する.
従ってこれが成り立つ.
故に定理 \ref{sthmalleqiset=}より, 
\[
  \{x \mid (x \in a \wedge R) \wedge S\} = \{x \mid x \in a \wedge (R \wedge S)\}, 
\]
即ち
\begin{equation}
\label{sthmssetsset4}
  \{x \mid (x \in a \wedge R) \wedge S\} = \{x \in a \mid R \wedge S\}
\end{equation}
が成り立つ.
そこで(\ref{sthmssetsset3}), (\ref{sthmssetsset4})から, 
推論法則 \ref{ded=trans}によって(\ref{sthmssetsset1})が成り立つ.

次に(\ref{sthmssetsset2})が成り立つことを示す.
$y$を上と同じ文字とするとき, Thm \ref{awblbwa}より
\[
  (y|x)(R) \wedge (y|x)(S) \leftrightarrow (y|x)(S) \wedge (y|x)(R)
\]
が成り立つが, 代入法則 \ref{substwedge}, \ref{substequiv}によればこの記号列は
\[
  (y|x)(R \wedge S \leftrightarrow S \wedge R)
\]
と一致するから, これが成り立つ.
このことと$y$が定数でないことから, 推論法則 \ref{dedltthmquan}により
\[
  \forall y((y|x)(R \wedge S \leftrightarrow S \wedge R))
\]
が成り立つ.
ここで$y$が$R$, $S$の中に自由変数として現れないことから, 
変数法則 \ref{valwedge}, \ref{valequiv}により, 
$y$は$R \wedge S \leftrightarrow S \wedge R$の中に自由変数として現れない.
故に代入法則 \ref{substquantrans}によれば, 上記の記号列は
\[
  \forall x(R \wedge S \leftrightarrow S \wedge R)
\]
と一致する.
従ってこれが成り立つ.
故に定理 \ref{sthmalleqsset=}より
\begin{equation}
\label{sthmssetsset5}
  \{x \in a \mid R \wedge S\} = \{x \in a \mid S \wedge R\}
\end{equation}
が成り立つ.
また上で示したように(\ref{sthmssetsset1})が成り立つ.
そこで(\ref{sthmssetsset1}), (\ref{sthmssetsset5})から, 
推論法則 \ref{ded=trans}によって(\ref{sthmssetsset2})が成り立つ.
\halmos




\mathstrut
\begin{thm}
\label{sthm!sm}%定理5.23%新規%確認済
$R$を関係式とし, $x$を文字とするとき, 
\begin{equation}
\label{sthm!sm1}
  !x(R) \leftrightarrow {\rm Set}_{x}(R) \wedge \{x \mid R\} \subset \{\tau_{x}(R)\}
\end{equation}
が成り立つ.
またこのことから, 次の1), 2)が成り立つ.

1)
$!x(R)$ならば, $R$は$x$について集合を作り得る.
更に$\{x \mid R\} \subset \{\tau_{x}(R)\}$が成り立つ.

2)
$R$が$x$について集合を作り得るとする.
このとき$\{x \mid R\} \subset \{\tau_{x}(R)\}$ならば, $!x(R)$.
\end{thm}


\noindent{\bf 証明}
~Thm \ref{thm!lall}より
\begin{equation}
\label{sthm!sm2}
  !x(R) \leftrightarrow \forall x(R \to x = \tau_{x}(R))
\end{equation}
が成り立つ.
また$y$を$x$と異なり, $R$の中に自由変数として現れない, 定数でない文字とするとき, 
定理 \ref{sthmsingletonbasis}と推論法則 \ref{dedeqch}により
\[
  y = \tau_{x}(R) \leftrightarrow y \in \{\tau_{x}(R)\}
\]
が成り立つから, 推論法則 \ref{dedaddeqt}により
\[
  ((y|x)(R) \to y = \tau_{x}(R)) \leftrightarrow ((y|x)(R) \to y \in \{\tau_{x}(R)\})
\]
が成り立つ.
故にこれと$y$が定数でないことから, 推論法則 \ref{dedalleqquansepconst}により
\[
  \forall y((y|x)(R) \to y = \tau_{x}(R)) 
  \leftrightarrow \forall y((y|x)(R) \to y \in \{\tau_{x}(R)\})
\]
が成り立つ.
ここで変数法則 \ref{valtau}, \ref{valnset}により, 
$x$は$\tau_{x}(R)$及び$\{\tau_{x}(R)\}$の中に自由変数として現れないから, 
代入法則 \ref{substfree}, \ref{substfund}により, 上記の記号列は
\[
  \forall y((y|x)(R \to x = \tau_{x}(R))) 
  \leftrightarrow \forall y((y|x)(R \to x \in \{\tau_{x}(R)\}))
\]
と一致する.
また$y$が$x$と異なり, $R$の中に自由変数として現れないことから, 
変数法則 \ref{valfund}, \ref{valtau}, \ref{valnset}によってわかるように, 
$y$は$R \to x = \tau_{x}(R)$及び$R \to x \in \{\tau_{x}(R)\}$の中に自由変数として現れない.
故に代入法則 \ref{substquantrans}により, 上記の記号列は
\begin{equation}
\label{sthm!sm3}
  \forall x(R \to x = \tau_{x}(R)) \leftrightarrow \forall x(R \to x \in \{\tau_{x}(R)\})
\end{equation}
と一致する.
従ってこれが成り立つ.
また上述のように$x$は$\{\tau_{x}(R)\}$の中に自由変数として現れないから, 
定理 \ref{sthmalltisetsubseta}より
\begin{equation}
\label{sthm!sm4}
  \forall x(R \to x \in \{\tau_{x}(R)\}) 
  \leftrightarrow {\rm Set}_{x}(R) \wedge \{x \mid R\} \subset \{\tau_{x}(R)\}
\end{equation}
が成り立つ.
そこで(\ref{sthm!sm2})---(\ref{sthm!sm4})から, 
推論法則 \ref{dedeqtrans}によって(\ref{sthm!sm1})が成り立つことがわかる.
1), 2)が成り立つことは(\ref{sthm!sm1})と推論法則 \ref{dedwedge}, \ref{dedeqfund}によって明らかである.
\halmos




\mathstrut
\begin{thm}
\label{sthmsmimp}%定理5.24%sthmsetmakeimproveから変更%1), 2)は新規%確認済
$R$を関係式とし, $x$を文字とする.
また$y$を$x$と異なり, $R$の中に自由変数として現れない文字とする.
このとき
\begin{equation}
\label{sthmsmimp1}
  {\rm Set}_{x}(R) \leftrightarrow \exists y(\forall x(R \to x \in y))
\end{equation}
が成り立つ.
またこのことから, 次の1), 2)が成り立つ.

1)
$R$が$x$について集合を作り得るならば, $\exists y(\forall x(R \to x \in y))$.

2)
$\exists y(\forall x(R \to x \in y))$ならば, $R$は$x$について集合を作り得る.
\end{thm}


\noindent{\bf 証明}
~このとき${\rm Set}_{x}(R)$は$\forall x(x \in \{x \mid R\} \leftrightarrow R)$と同じだから, 
Thm \ref{thmquanwedge}より
\[
  {\rm Set}_{x}(R) \to \forall x(R \to x \in \{x \mid R\})
\]
が成り立つ.
ここで$y$が$x$と異なり, $R$の中に自由変数として現れないことから, 
代入法則 \ref{substfree}, \ref{substfund}により, この記号列は
\[
  {\rm Set}_{x}(R) \to \forall x((\{x \mid R\}|y)(R \to x \in y))
\]
と一致する.
また$x$は$y$と異なり, 変数法則 \ref{valiset}により$\{x \mid R\}$の中に自由変数として現れないから, 
代入法則 \ref{substquan}により, この記号列は
\begin{equation}
\label{sthmsmimp2}
  {\rm Set}_{x}(R) \to (\{x \mid R\}|y)(\forall x(R \to x \in y))
\end{equation}
と一致する.
故にこれが成り立つ.
またschema S4の適用により
\begin{equation}
\label{sthmsmimp3}
  (\{x \mid R\}|y)(\forall x(R \to x \in y)) \to \exists y(\forall x(R \to x \in y))
\end{equation}
が成り立つ.
そこで(\ref{sthmsmimp2}), (\ref{sthmsmimp3})から, 推論法則 \ref{dedmmp}によって
\begin{equation}
\label{sthmsmimp4}
  {\rm Set}_{x}(R) \to \exists y(\forall x(R \to x \in y))
\end{equation}
が成り立つ.
またいま$\tau_{y}(\forall x(R \to x \in y))$を$T$と書けば, $T$は対象式であり, 
変数法則 \ref{valtau}, \ref{valquan}によってわかるように, $x$は$T$の中に自由変数として現れない.
故に定理 \ref{sthmalltiset=sset}と推論法則 \ref{dedprewedge}, \ref{dedequiv}によってわかるように, 
\[
  \forall x(R \to x \in T) \to {\rm Set}_{x}(R)
\]
が成り立つ.
ここで$y$が$x$と異なり, $R$の中に自由変数として現れないことから, 
代入法則 \ref{substfree}, \ref{substfund}により, この記号列は
\[
  \forall x((T|y)(R \to x \in y)) \to {\rm Set}_{x}(R)
\]
と一致する.
また$x$は$y$と異なり, $T$の中に自由変数として現れないから, 
代入法則 \ref{substquan}により, この記号列は
\[
  (T|y)(\forall x(R \to x \in y)) \to {\rm Set}_{x}(R)
\]
と一致する.
また$T$の定義から, この記号列は
\begin{equation}
\label{sthmsmimp5}
  \exists y(\forall x(R \to x \in y)) \to {\rm Set}_{x}(R)
\end{equation}
と同じである.
故にこれが成り立つ.
そこで(\ref{sthmsmimp4}), (\ref{sthmsmimp5})から, 
推論法則 \ref{dedequiv}により(\ref{sthmsmimp1})が成り立つ.
1), 2)が成り立つことは, (\ref{sthmsmimp1})と推論法則 \ref{dedeqfund}によって明らかである.
\halmos




\mathstrut
\begin{thm}
\label{sthmsmfree}%定理5.25%新規%確認済
$R$を関係式とし, $x$を$R$の中に自由変数として現れない文字とする.
このとき
\begin{equation}
\label{sthmsmfree1}
  {\rm Set}_{x}(R) \leftrightarrow \neg R
\end{equation}
が成り立つ.
またこのことから, 次の1), 2)が成り立つ.

1)
$R$が$x$について集合を作り得るならば, $\neg R$.

2)
$\neg R$ならば, $R$は$x$について集合を作り得る.
\end{thm}


\noindent{\bf 証明}
~$y$を$x$と異なり, $R$の中に自由変数として現れない, 定数でない文字とする.
このとき定理 \ref{sthmsmimp}より
\begin{equation}
\label{sthmsmfree2}
  {\rm Set}_{x}(R) \leftrightarrow \exists y(\forall x(R \to x \in y))
\end{equation}
が成り立つ.
また$x$が$R$の中に自由変数として現れないことから, Thm \ref{thmalltallseprfree}より
\[
  \forall x(R \to x \in y) \leftrightarrow (R \to \forall x(x \in y))
\]
が成り立つ.
このことと$y$が定数でないことから, 推論法則 \ref{dedalleqquansepconst}により
\begin{equation}
\label{sthmsmfree3}
  \exists y(\forall x(R \to x \in y)) \leftrightarrow \exists y(R \to \forall x(x \in y))
\end{equation}
が成り立つ.
また$y$が$R$の中に自由変数として現れないことから, Thm \ref{thmextquanseprfree}より
\begin{equation}
\label{sthmsmfree4}
  \exists y(R \to \forall x(x \in y)) \leftrightarrow (R \to \exists y(\forall x(x \in y)))
\end{equation}
が成り立つ.
またThm \ref{1atb1l1nbtna1}より
\begin{equation}
\label{sthmsmfree5}
  (R \to \exists y(\forall x(x \in y))) \leftrightarrow (\neg \exists y(\forall x(x \in y)) \to \neg R)
\end{equation}
が成り立つ.
また$x$と$y$が異なることから, 例2より$\neg \exists y(\forall x(x \in y))$が成り立つから, 
推論法則 \ref{ded1atb1lbtrue}により
\begin{equation}
\label{sthmsmfree6}
  (\neg \exists y(\forall x(x \in y)) \to \neg R) \leftrightarrow \neg R
\end{equation}
が成り立つ.
そこで(\ref{sthmsmfree2})---(\ref{sthmsmfree6})から, 
推論法則 \ref{dedeqtrans}によって(\ref{sthmsmfree1})が成り立つことがわかる.
1), 2)が成り立つことは, (\ref{sthmsmfree1})と推論法則 \ref{dedeqfund}によって明らかである.
\halmos




\mathstrut
\begin{thm}
\label{sthmall&sm}%定理5.26%sthmall&setmakeから変更%確認済
$R$を関係式とし, $x$を文字とするとき, 
\begin{align}
  \label{sthmall&sm1}
  &\forall x(R) \to \neg {\rm Set}_{x}(R), \\
  \mbox{} \notag \\
  \label{sthmall&sm2}
  &\forall x(\neg R) \to {\rm Set}_{x}(R)
\end{align}
が共に成り立つ.
またこれらから, 次の1)---4)が成り立つ.

1)
$\forall x(R)$ならば, $R$は$x$について集合を作り得ない.

2)
$x$が定数でなく, $R$が成り立てば, $R$は$x$について集合を作り得ない.

3)
$\forall x(\neg R)$ならば, $R$は$x$について集合を作り得る.

4)
$x$が定数でなく, $\neg R$が成り立てば, $R$は$x$について集合を作り得る.
\end{thm}


\noindent{\bf 証明}
~まず(\ref{sthmall&sm1})が成り立つことを示す.
$y$を$x$と異なり, $R$の中に自由変数として現れない, 定数でない文字とする.
このとき定理 \ref{sthmsmimp}と推論法則 \ref{dedequiv}により
\begin{equation}
\label{sthmall&sm3}
  {\rm Set}_{x}(R) \to \exists y(\forall x(R \to x \in y))
\end{equation}
が成り立つ.
またThm \ref{thmalltallsep}より
\[
  \forall x(R \to x \in y) \to (\forall x(R) \to \forall x(x \in y))
\]
が成り立つから, このことと$y$が定数でないことから, 推論法則 \ref{dedalltquansepconst}により
\begin{equation}
\label{sthmall&sm4}
  \exists y(\forall x(R \to x \in y)) \to \exists y(\forall x(R) \to \forall x(x \in y))
\end{equation}
が成り立つ.
また$y$が$R$の中に自由変数として現れないことから, 
変数法則 \ref{valquan}により$y$は$\forall x(R)$の中に自由変数として現れないから, 
Thm \ref{thmextquanseprfree}と推論法則 \ref{dedequiv}により
\begin{equation}
\label{sthmall&sm5}
  \exists y(\forall x(R) \to \forall x(x \in y)) \to (\forall x(R) \to \exists y(\forall x(x \in y)))
\end{equation}
が成り立つ.
またThm \ref{1atb1t1nbtna1}より
\begin{equation}
\label{sthmall&sm6}
  (\forall x(R) \to \exists y(\forall x(x \in y))) 
  \to (\neg \exists y(\forall x(x \in y)) \to \neg \forall x(R))
\end{equation}
が成り立つ.
また$x$と$y$が異なることから, 例2より$\neg \exists y(\forall x(x \in y))$が成り立つから, 
推論法則 \ref{ded1atb1tbtrue2}により
\begin{equation}
\label{sthmall&sm7}
  (\neg \exists y(\forall x(x \in y)) \to \neg \forall x(R)) \to \neg \forall x(R)
\end{equation}
が成り立つ.
そこで(\ref{sthmall&sm3})---(\ref{sthmall&sm7})から, 推論法則 \ref{dedmmp}によって
\[
  {\rm Set}_{x}(R) \to \neg \forall x(R)
\]
が成り立つことがわかる.
故に推論法則 \ref{dedcp}により(\ref{sthmall&sm1})が成り立つ.

次に(\ref{sthmall&sm2})が成り立つことを示す.
$y$を上と同じ文字とするとき, Thm \ref{thmquanseptall2}より
\begin{equation}
\label{sthmall&sm8}
  \forall x(\neg R) \to \forall x(R \to x \in y)
\end{equation}
が成り立つ.
また$x$が$y$と異なることから, 定理 \ref{sthmalltiset=sset}と
推論法則 \ref{dedprewedge}, \ref{dedequiv}によってわかるように, 
\begin{equation}
\label{sthmall&sm9}
  \forall x(R \to x \in y) \to {\rm Set}_{x}(R)
\end{equation}
が成り立つ.
そこで(\ref{sthmall&sm8}), (\ref{sthmall&sm9})から, 
推論法則 \ref{dedmmp}によって(\ref{sthmall&sm2})が成り立つ.

\noindent
1)
(\ref{sthmall&sm1})と推論法則 \ref{dedmp}によって明らか.

\noindent
2)
1)と推論法則 \ref{dedltthmquan}によって明らか.

\noindent
3)
(\ref{sthmall&sm2})と推論法則 \ref{dedmp}によって明らか.

\noindent
4)
3)と推論法則 \ref{dedltthmquan}によって明らか.
\halmos




\mathstrut
\begin{thm}
\label{sthmalltsm}%定理5.27%sthmalltsetmakeから変更%確認済
$R$と$S$を関係式とし, $x$を文字とする.
このとき
\begin{equation}
\label{sthmalltsm1}
  \forall x(R \to S) \to ({\rm Set}_{x}(S) \to {\rm Set}_{x}(R))
\end{equation}
が成り立つ.
またこのことから, 次の1)---4)が成り立つ.

1)
$\forall x(R \to S)$ならば, ${\rm Set}_{x}(S) \to {\rm Set}_{x}(R)$.

2)
$x$が定数でなく, $R \to S$が成り立てば, ${\rm Set}_{x}(S) \to {\rm Set}_{x}(R)$.

3)
$\forall x(R \to S)$であり, かつ$S$が$x$について集合を作り得るならば, 
$R$は$x$について集合を作り得る.
またこのとき$\{x \mid R\} \subset \{x \mid S\}$が成り立つ.

4)
$x$が定数でなく, $R \to S$が成り立ち, かつ$S$が$x$について集合を作り得るならば, 
$R$は$x$について集合を作り得る.
またこのとき$\{x \mid R\} \subset \{x \mid S\}$が成り立つ.
\end{thm}


\noindent{\bf 証明}
~$y$を$x$と異なり, $R$及び$S$の中に自由変数として現れない, 定数でない文字とする.
このときThm \ref{thmallpretspquansep}より
\begin{equation}
\label{sthmalltsm2}
  \forall x(R \to S) \to (\forall_{S}x(x \in y) \to \forall_{R}x(x \in y))
\end{equation}
が成り立つ.
またThm \ref{thmspallfund}より
\[
  \forall_{S}x(x \in y) \leftrightarrow \forall x(S \to x \in y), ~~
  \forall_{R}x(x \in y) \leftrightarrow \forall x(R \to x \in y)
\]
が共に成り立つから, 推論法則 \ref{dedaddeqt}により
\[
  (\forall_{S}x(x \in y) \to \forall_{R}x(x \in y)) 
  \leftrightarrow (\forall x(S \to x \in y) \to \forall x(R \to x \in y))
\]
が成り立つ.
故に推論法則 \ref{dedequiv}により
\begin{equation}
\label{sthmalltsm3}
  (\forall_{S}x(x \in y) \to \forall_{R}x(x \in y)) 
  \to (\forall x(S \to x \in y) \to \forall x(R \to x \in y))
\end{equation}
が成り立つ.
そこで(\ref{sthmalltsm2}), (\ref{sthmalltsm3})から, 推論法則 \ref{dedmmp}によって
\begin{equation}
\label{sthmalltsm4}
  \forall x(R \to S) \to (\forall x(S \to x \in y) \to \forall x(R \to x \in y))
\end{equation}
が成り立つ.
ここで$y$が$R$及び$S$の中に自由変数として現れないことから, 
変数法則 \ref{valfund}, \ref{valquan}により, $y$は$\forall x(R \to S)$の中に自由変数として現れない.
また$y$は定数でない.
故にこれらのことと(\ref{sthmalltsm4})から, 推論法則 \ref{dedalltquansepfreeconst}により
\begin{equation}
\label{sthmalltsm5}
  \forall x(R \to S) \to \forall y(\forall x(S \to x \in y) \to \forall x(R \to x \in y))
\end{equation}
が成り立つ.
またThm \ref{thmalltexsep}より
\begin{equation}
\label{sthmalltsm6}
  \forall y(\forall x(S \to x \in y) \to \forall x(R \to x \in y)) 
  \to (\exists y(\forall x(S \to x \in y)) \to \exists y(\forall x(R \to x \in y)))
\end{equation}
が成り立つ.
また$y$が$x$と異なり, $S$及び$R$の中に自由変数として現れないことから, 定理 \ref{sthmsmimp}より
\[
  {\rm Set}_{x}(S) \leftrightarrow \exists y(\forall x(S \to x \in y)), ~~
  {\rm Set}_{x}(R) \leftrightarrow \exists y(\forall x(R \to x \in y))
\]
が共に成り立つ.
故に推論法則 \ref{dedaddeqt}により
\[
  ({\rm Set}_{x}(S) \to {\rm Set}_{x}(R)) 
  \leftrightarrow (\exists y(\forall x(S \to x \in y)) \to \exists y(\forall x(R \to x \in y)))
\]
が成り立つ.
故に推論法則 \ref{dedequiv}により
\begin{equation}
\label{sthmalltsm7}
  (\exists y(\forall x(S \to x \in y)) \to \exists y(\forall x(R \to x \in y))) 
  \to ({\rm Set}_{x}(S) \to {\rm Set}_{x}(R))
\end{equation}
が成り立つ.
そこで(\ref{sthmalltsm5})---(\ref{sthmalltsm7})から, 
推論法則 \ref{dedmmp}によって(\ref{sthmalltsm1})が成り立つことがわかる.

\noindent
1)
(\ref{sthmalltsm1})と推論法則 \ref{dedmp}によって明らか.

\noindent
2)
1)と推論法則 \ref{dedltthmquan}によって明らか.

\noindent
3)
前半は1)と推論法則 \ref{dedmp}, 後半は定理 \ref{sthmsmtalltisetsubseteq}によって明らか.

\noindent
4)
3)と推論法則 \ref{dedltthmquan}によって明らか.
\halmos




\mathstrut
\begin{defo}
\label{oset}%変形17%確認済
$\mathscr{T}$を特殊記号として$=$と$\in$を持つ理論とし, 
$a$と$T$を$\mathscr{T}$の記号列, $x$を文字とする.
また$y$と$z$を共に$x$と異なり, $a$及び$T$の中に自由変数として現れない文字とする.
このとき
\[
  \{y \mid \exists x(x \in a \wedge y = T)\} \equiv \{z \mid \exists x(x \in a \wedge z = T)\}
\]
が成り立つ.
\end{defo}


\noindent{\bf 証明}
~$y$と$z$が同じ文字ならば明らかだから, 以下$y$と$z$は異なる文字であるとする.
このとき$z$が$x$, $y$と異なり, $a$及び$T$の中に自由変数として現れないことから, 
変数法則 \ref{valfund}, \ref{valwedge}, \ref{valquan}により, 
$z$は$\exists x(x \in a \wedge y = T)$の中に自由変数として現れない.
故に代入法則 \ref{substisettrans}により
\[
  \{y \mid \exists x(x \in a \wedge y = T)\} \equiv \{z \mid (z|y)(\exists x(x \in a \wedge y = T))\}
\]
が成り立つ.
また$x$が$y$, $z$と異なり, $y$が$a$及び$T$の中に自由変数として現れないことから, 
代入法則 \ref{substfree}, \ref{substfund}, \ref{substwedge}, \ref{substquan}によってわかるように
\[
  (z|y)(\exists x(x \in a \wedge y = T)) \equiv \exists x(x \in a \wedge z = T)
\]
が成り立つ.
故に本法則が成り立つ.
\halmos




\mathstrut
\begin{defi}
\label{defoset}%定義3%確認済
$\mathscr{T}$を特殊記号として$=$と$\in$を持つ理論とし, 
$a$と$T$を$\mathscr{T}$の記号列, $x$を文字とする.
また$y$と$z$を共に$x$と異なり, $a$及び$T$の中に自由変数として現れない文字とする.
このとき変形法則 \ref{oset}によれば, $\{y \mid \exists x(x \in a \wedge y = T)\}$と
$\{z \mid \exists x(x \in a \wedge z = T)\}$は同じ記号列となる.
$a$, $T$, $x$に対して定まるこの記号列を, $\{T\}_{x \in a}$と書き表す.
混同のおそれがなければ, これを$\{T \mid x \in a\}$とも書き表す.
\end{defi}




\mathstrut%確認済%koko
以下の変数法則 \ref{valoset}, 一般代入法則 \ref{gsubstoset}, 代入法則 \ref{substosettrans}, \ref{substoset}, 
構成法則 \ref{formoset}では, $\mathscr{T}$を特殊記号として$=$と$\in$を持つ理論とし, 
これらの法則における``記号列'', ``集合''とは, 
それぞれ$\mathscr{T}$の記号列, $\mathscr{T}$の対象式のこととする.




\mathstrut
\begin{valu}
\label{valoset}%変数28%確認済
$a$と$T$を記号列とし, $x$を文字とする.

1)
$x$は$\{T\}_{x \in a}$の中に自由変数として現れない.

2)
$y$を文字とする.
$y$が$a$及び$T$の中に自由変数として現れなければ, 
$y$は$\{T\}_{x \in a}$の中に自由変数として現れない.
\end{valu}


\noindent{\bf 証明}
~1)
$z$を$x$と異なり, $a$及び$T$の中に自由変数として現れない文字とすれば, 
定義より$\{T\}_{x \in a}$は$\{z \mid \exists x(x \in a \wedge z = T)\}$と同じである.
変数法則 \ref{valquan}, \ref{valiset}によれば, $x$はこの中に自由変数として現れない.

\noindent
2)
$y$が$x$と同じ文字ならば1)により明らか.
$y$が$x$と異なる文字ならば, このことと$y$が$a$及び$T$の中に自由変数として現れないことから, 
定義より$\{T\}_{x \in a}$は$\{y \mid \exists x(x \in a \wedge y = T)\}$と同じである.
変数法則 \ref{valiset}によれば, $y$はこの中に自由変数として現れない.
\halmos




\mathstrut
\begin{gsub}
\label{gsubstoset}%一般代入32%確認済
$a$と$T$を記号列とし, $x$を文字とする.
また$n$を自然数とし, $U_{1}, U_{2}, \cdots, U_{n}$を記号列とする.
また$y_{1}, y_{2}, \cdots, y_{n}$を, どの二つも互いに異なる文字とする.
$x$が$y_{1}, y_{2}, \cdots, y_{n}$のいずれとも異なり, かつ
$U_{1}, U_{2}, \cdots, U_{n}$のいずれの記号列の中にも自由変数として現れなければ, 
\[
  (U_{1}|y_{1}, U_{2}|y_{2}, \cdots, U_{n}|y_{n})(\{T\}_{x \in a}) 
  \equiv \{(U_{1}|y_{1}, U_{2}|y_{2}, \cdots, U_{n}|y_{n})(T)\}_{x \in (U_{1}|y_{1}, U_{2}|y_{2}, \cdots, U_{n}|y_{n})(a)}
\]
が成り立つ.
\end{gsub}


\noindent{\bf 証明}
~$z$を$x, y_{1}, y_{2}, \cdots, y_{n}$のいずれとも異なり, 
$a, T, U_{1}, U_{2}, \cdots, U_{n}$のいずれの記号列の中にも自由変数として現れない文字とする.
このとき定義から$\{T\}_{x \in a}$は$\{z \mid \exists x(x \in a \wedge z = T)\}$だから, 
\begin{equation}
\label{gsubstoset1}
  (U_{1}|y_{1}, U_{2}|y_{2}, \cdots, U_{n}|y_{n})(\{T\}_{x \in a}) 
  \equiv (U_{1}|y_{1}, U_{2}|y_{2}, \cdots, U_{n}|y_{n})(\{z \mid \exists x(x \in a \wedge z = T)\})
\end{equation}
である.
また$z$が$y_{1}, y_{2}, \cdots, y_{n}$のいずれとも異なり, かつ
$U_{1}, U_{2}, \cdots, U_{n}$のいずれの記号列の中にも自由変数として現れないことから, 
一般代入法則 \ref{gsubstiset}により
\begin{multline}
\label{gsubstoset2}
  (U_{1}|y_{1}, U_{2}|y_{2}, \cdots, U_{n}|y_{n})(\{z \mid \exists x(x \in a \wedge z = T)\}) \\
  \equiv \{z \mid (U_{1}|y_{1}, U_{2}|y_{2}, \cdots, U_{n}|y_{n})(\exists x(x \in a \wedge z = T))\}
\end{multline}
が成り立つ.
また$x$も$y_{1}, y_{2}, \cdots, y_{n}$のいずれとも異なり, かつ
$U_{1}, U_{2}, \cdots, U_{n}$のいずれの記号列の中にも自由変数として現れないから, 
一般代入法則 \ref{gsubstquan}により
\begin{equation}
\label{gsubstoset3}
  (U_{1}|y_{1}, U_{2}|y_{2}, \cdots, U_{n}|y_{n})(\exists x(x \in a \wedge z = T)) 
  \equiv \exists x((U_{1}|y_{1}, U_{2}|y_{2}, \cdots, U_{n}|y_{n})(x \in a \wedge z = T))
\end{equation}
が成り立つ.
また$x$と$z$が共に$y_{1}, y_{2}, \cdots, y_{n}$のいずれとも異なることと
一般代入法則 \ref{gsubstfund}, \ref{gsubstwedge}から, 
\begin{multline}
\label{gsubstoset4}
  (U_{1}|y_{1}, U_{2}|y_{2}, \cdots, U_{n}|y_{n})(x \in a \wedge z = T) \\
  \equiv x \in (U_{1}|y_{1}, U_{2}|y_{2}, \cdots, U_{n}|y_{n})(a) \wedge z = (U_{1}|y_{1}, U_{2}|y_{2}, \cdots, U_{n}|y_{n})(T)
\end{multline}
が成り立つ.
以上の(\ref{gsubstoset1})---(\ref{gsubstoset4})からわかるように, 
$(U_{1}|y_{1}, U_{2}|y_{2}, \cdots, U_{n}|y_{n})(\{T\}_{x \in a})$は
\begin{equation}
\label{gsubstoset5}
  \{z \mid \exists x(x \in (U_{1}|y_{1}, U_{2}|y_{2}, \cdots, U_{n}|y_{n})(a) \wedge z = (U_{1}|y_{1}, U_{2}|y_{2}, \cdots, U_{n}|y_{n})(T))\}
\end{equation}
と一致する.
ここで$z$が$a, T, U_{1}, U_{2}, \cdots, U_{n}$のいずれの記号列の中にも
自由変数として現れないことから, 変数法則 \ref{valgsubst}により, 
$z$は$(U_{1}|y_{1}, U_{2}|y_{2}, \cdots, U_{n}|y_{n})(a)$及び
$(U_{1}|y_{1}, U_{2}|y_{2}, \cdots, U_{n}|y_{n})(T)$の中に自由変数として現れない.
このことと$z$が$x$と異なることから, 定義より(\ref{gsubstoset5})は
\[
  \{(U_{1}|y_{1}, U_{2}|y_{2}, \cdots, U_{n}|y_{n})(T)\}_{x \in (U_{1}|y_{1}, U_{2}|y_{2}, \cdots, U_{n}|y_{n})(a)}
\]
と同じである.
故に本法則が成り立つ.
\halmos




\mathstrut
\begin{subs}
\label{substosettrans}%代入38%確認済
$a$と$T$を記号列とし, $x$と$y$を文字とする.
$y$が$a$及び$T$の中に自由変数として現れなければ, 
\[
  \{T\}_{x \in a} \equiv \{(y|x)(T)\}_{y \in (y|x)(a)}
\]
が成り立つ.
更に, $x$が$a$の中に自由変数として現れなければ, 
\[
  \{T\}_{x \in a} \equiv \{(y|x)(T)\}_{y \in a}
\]
が成り立つ.
\end{subs}


\noindent{\bf 証明}
~$y$が$x$と同じ文字ならば, 代入法則 \ref{substsame}によって本法則が成り立つから, 
以下では$y$は$x$と異なる文字であるとする.
いま$z$を$x$とも$y$とも異なり, $a$及び$T$の中に自由変数として現れない文字とする.
このとき変数法則 \ref{valsubst}により, $z$は$(y|x)(a)$及び$(y|x)(T)$の中に自由変数として現れない.
また定義から
\begin{equation}
\label{substosettrans1}
  \{T\}_{x \in a} \equiv \{z \mid \exists x(x \in a \wedge z = T)\}
\end{equation}
である.
また$y$が$x$, $z$と異なり, $a$及び$T$の中に自由変数として現れないことから, 
変数法則 \ref{valfund}, \ref{valwedge}により, $y$は$x \in a \wedge z = T$の中に自由変数として現れない.
故に代入法則 \ref{substquantrans}により
\begin{equation}
\label{substosettrans2}
  \exists x(x \in a \wedge z = T) \equiv \exists y((y|x)(x \in a \wedge z = T))
\end{equation}
が成り立つ.
また$x$が$z$と異なることと代入法則 \ref{substfund}, \ref{substwedge}により, 
\begin{equation}
\label{substosettrans3}
  (y|x)(x \in a \wedge z = T) \equiv y \in (y|x)(a) \wedge z = (y|x)(T)
\end{equation}
が成り立つ.
以上の(\ref{substosettrans1})---(\ref{substosettrans3})からわかるように, $\{T\}_{x \in a}$は
\begin{equation}
\label{substosettrans4}
  \{z \mid \exists y(y \in (y|x)(a) \wedge z = (y|x)(T))\}
\end{equation}
と同じである.
ここで$z$が$y$と異なり, 上述のように$(y|x)(a)$及び$(y|x)(T)$の中に自由変数として現れないことから, 
定義より(\ref{substosettrans4})は$\{(y|x)(T)\}_{y \in (y|x)(a)}$と同じである.
故に
\[
  \{T\}_{x \in a} \equiv \{(y|x)(T)\}_{y \in (y|x)(a)}
\]
が成り立つ.
特にここで$x$が$a$の中に自由変数として現れなければ, 代入法則 \ref{substfree}により
$(y|x)(a)$は$a$と一致するから, 
\[
  \{T\}_{x \in a} \equiv \{(y|x)(T)\}_{y \in a}
\]
が成り立つ.
\halmos




\mathstrut
\begin{subs}
\label{substoset}%代入39%確認済
$a$, $T$, $U$を記号列とし, $x$と$y$を異なる文字とする.
$x$が$U$の中に自由変数として現れなければ, 
\[
  (U|y)(\{T\}_{x \in a}) \equiv \{(U|y)(T)\}_{x \in (U|y)(a)}
\]
が成り立つ.
\end{subs}


\noindent{\bf 証明}
~一般代入法則 \ref{gsubstoset}において, $n$が$1$の場合である.
\halmos




\mathstrut
\begin{form}
\label{formoset}%構成45%確認済
$a$と$T$を集合とし, $x$を文字とする.
このとき$\{T\}_{x \in a}$は集合である.
\end{form}


\noindent{\bf 証明}
~$y$を$x$と異なり, $a$及び$T$の中に自由変数として現れない文字とするとき, 
定義から$\{T\}_{x \in a}$は$\{y \mid \exists x(x \in a \wedge y = T)\}$である.
これが集合であることは
構成法則 \ref{formfund}, \ref{formwedge}, \ref{formquan}, \ref{formiset}から直ちにわかる.
\halmos




\mathstrut
\begin{thm}
\label{sthmosetsm}%定理5.28%確認済
$a$と$T$を集合とし, $x$を$a$の中に自由変数として現れない文字とする.
また$y$を$x$と異なり, $a$及び$T$の中に自由変数として現れない文字とする.
このとき関係式$\exists x(x \in a \wedge y = T)$は$y$について集合を作り得る.
\end{thm}


\noindent{\bf 証明}
~$z$を$y$と異なり, $T$の中に自由変数として現れない, 定数でない文字とする.
このとき$y$が$z$と異なり, $T$の中に自由変数として現れないことから, 
変数法則 \ref{valsubst}により$y$は$(z|x)(T)$の中に自由変数として現れない.
故に定理 \ref{sthmsingletonsm}より, $y = (z|x)(T)$は$y$について集合を作り得る.
即ち
\[
  \forall y(y \in \{y \mid y = (z|x)(T)\} \leftrightarrow y = (z|x)(T))
\]
が成り立つ.
ここで$y$が$(z|x)(T)$の中に自由変数として現れないことから, この記号列は
\[
  \forall y(y \in \{(z|x)(T)\} \leftrightarrow y = (z|x)(T))
\]
と同じである.
従ってこれが成り立つ.
故に推論法則 \ref{dedquanwedge}により
\[
  \forall y(y = (z|x)(T) \to y \in \{(z|x)(T)\})
\]
が成り立つ.
ここで$y$が$x$, $z$と異なることから, 
代入法則 \ref{substfund}, \ref{substquan}, \ref{substnset}によってわかるように, この記号列は
\[
  (z|x)(\forall y(y = T \to y \in \{T\}))
\]
と一致する.
故にこれが成り立つ.
このことと$z$が定数でないことから, 推論法則 \ref{dedltthmquan}により
\[
  \forall z((z|x)(\forall y(y = T \to y \in \{T\})))
\]
が成り立つ.
ここで$z$が$y$と異なり, $T$の中に自由変数として現れないことから, 
変数法則 \ref{valfund}, \ref{valquan}, \ref{valnset}により
$z$は$\forall y(y = T \to y \in \{T\})$の中に自由変数として現れない.
故に代入法則 \ref{substquantrans}により, 上記の記号列は
\begin{equation}
\label{sthmosetsm1}
  \forall x(\forall y(y = T \to y \in \{T\}))
\end{equation}
と一致する.
従ってこれが成り立つ.
さていま$y$は$T$の中に自由変数として現れないから, 
変数法則 \ref{valnset}により, $y$は$\{T\}$の中に自由変数として現れない.
また$x$と$y$は互いに異なり, 共に$a$の中に自由変数として現れない.
これらのことと(\ref{sthmosetsm1})が成り立つことから, 定理 \ref{sthms7ab}より
$\exists x(x \in a \wedge y = T)$は$y$について集合を作り得る.
\halmos




\mathstrut
\begin{thm}
\label{sthmosetbasis}%定理5.29%確認済
$a$, $b$, $T$を集合とし, $x$を$a$及び$b$の中に自由変数として現れない文字とする.
このとき
\begin{equation}
\label{sthmosetbasis1}
  b \in \{T\}_{x \in a} \leftrightarrow \exists x(x \in a \wedge b = T)
\end{equation}
が成り立つ.
またこのことから, 次の1), 2)が成り立つ.

1)
$b \in \{T\}_{x \in a}$ならば, $\exists x(x \in a \wedge b = T)$.

2)
$\exists x(x \in a \wedge b = T)$ならば, $b \in \{T\}_{x \in a}$.
\end{thm}


\noindent{\bf 証明}
~$y$を$x$と異なり, $a$及び$T$の中に自由変数として現れない文字とする.
このとき定義から$\{T\}_{x \in a}$は$\{y \mid \exists x(x \in a \wedge y = T)\}$と同じである.
また$x$と$y$が互いに異なり, $x$が$a$の中に自由変数として現れず, 
$y$が$a$及び$T$の中に自由変数として現れないことから, 
定理 \ref{sthmosetsm}より$\exists x(x \in a \wedge y = T)$は$y$について集合を作り得る.
そこでこれらのことから, 定理 \ref{sthmisetbasis}より
\[
  b \in \{T\}_{x \in a} \leftrightarrow (b|y)(\exists x(x \in a \wedge y = T))
\]
が成り立つ.
ここで$x$と$y$が互いに異なり, $x$が$b$の中に自由変数として現れず, 
$y$が$a$及び$T$の中に自由変数として現れないことから, 
代入法則 \ref{substfree}, \ref{substfund}, \ref{substwedge}, \ref{substquan}によってわかるように, 
この記号列は(\ref{sthmosetbasis1})と一致する.
故に(\ref{sthmosetbasis1})が成り立つ.
1), 2)が成り立つことは(\ref{sthmosetbasis1})と推論法則 \ref{dedeqfund}によって明らかである.
\halmos




\mathstrut
{\small
\noindent
{\bf 例 3.}~%例5.3%確認済
$a$を集合とし, $x$を$a$の中に自由変数として現れない文字とする.
このとき
\[
  \{x\}_{x \in a} = a
\]
が成り立つ.

実際$y$を$x$と異なり, $a$の中に自由変数として現れない, 定数でない文字とする.
このとき$x$が$y$と異なり, $a$の中に自由変数として現れないことから, 定理 \ref{sthmosetbasis}より
\begin{equation}
\label{ex5.3.1}
  y \in \{x\}_{x \in a} \leftrightarrow \exists x(x \in a \wedge y = x)
\end{equation}
が成り立つ.
またThm \ref{thmquanwch}より, 
\[
  \exists x(x \in a \wedge y = x) \leftrightarrow \exists x(y = x \wedge x \in a), 
\]
即ち
\begin{equation}
\label{ex5.3.2}
  \exists x(x \in a \wedge y = x) \leftrightarrow \exists_{y = x}x(x \in a)
\end{equation}
が成り立つ.
また$z$を$x$, $y$と異なる定数でない文字とするとき, Thm \ref{x=yly=x}より
\[
  y = z \leftrightarrow z = y
\]
が成り立つが, $x$が$y$と異なることと代入法則 \ref{substequiv}によればこの記号列は
\[
  (z|x)(y = x \leftrightarrow x = y)
\]
と一致するから, これが成り立つ.
このことと$z$が定数でないことから, 推論法則 \ref{dedltthmquan}により
\[
  \forall z((z|x)(y = x \leftrightarrow x = y))
\]
が成り立つ.
ここで$z$が$x$, $y$と異なることから, 変数法則 \ref{valequiv}により
$z$は$y = x \leftrightarrow x = y$の中に自由変数として現れない.
故に代入法則 \ref{substquantrans}により, 上記の記号列は
\[
  \forall x(y = x \leftrightarrow x = y)
\]
と一致する.
従ってこれが成り立つ.
そこで推論法則 \ref{dedallpreeqspquansep}により
\begin{equation}
\label{ex5.3.3}
  \exists_{y = x}x(x \in a) \leftrightarrow \exists_{x = y}x(x \in a)
\end{equation}
が成り立つ.
また$x$が$y$と異なることから, Thm \ref{thmspquan=}と推論法則 \ref{dedeqch}により
\[
  \exists_{x = y}x(x \in a) \leftrightarrow (y|x)(x \in a)
\]
が成り立つ.
ここで$x$が$a$の中に自由変数として現れないことから, 
代入法則 \ref{substfree}, \ref{substfund}により, この記号列は
\begin{equation}
\label{ex5.3.4}
  \exists_{x = y}x(x \in a) \leftrightarrow y \in a
\end{equation}
と一致する.
故にこれが成り立つ.
そこで(\ref{ex5.3.1})---(\ref{ex5.3.4})から, 推論法則 \ref{dedeqtrans}によって
\begin{equation}
\label{ex5.3.5}
  y \in \{x\}_{x \in a} \leftrightarrow y \in a
\end{equation}
が成り立つことがわかる.
さていま$y$は$x$と異なり, $a$の中に自由変数として現れないから, 
変数法則 \ref{valoset}により$y$は$\{x\}_{x \in a}$の中にも自由変数として現れない.
また$y$は定数でない.
これらのことと(\ref{ex5.3.5})が成り立つことから, 
定理 \ref{sthmset=}より$\{x\}_{x \in a} = a$が成り立つ. ------
}




\mathstrut
\begin{thm}
\label{sthmosetfund}%定理5.30%確認済
$a$, $T$, $U$を集合とし, $x$を$a$の中に自由変数として現れない文字とする.
このとき
\begin{equation}
\label{sthmosetfund1}
  U \in a \to (U|x)(T) \in \{T\}_{x \in a}
\end{equation}
が成り立つ.
またこのことから, 次の(\ref{sthmosetfund2})が成り立つ.
\begin{equation}
\label{sthmosetfund2}
  U \in a \text{ならば,} ~(U|x)(T) \in \{T\}_{x \in a}.
\end{equation}
\end{thm}


\noindent{\bf 証明}
~Thm \ref{x=x}より$(U|x)(T) = (U|x)(T)$が成り立つから, 推論法則 \ref{dedatawbtrue2}により
\[
  U \in a \to U \in a \wedge (U|x)(T) = (U|x)(T)
\]
が成り立つ.
ここで$y$を$x$と異なり, $a$, $T$, $U$の中に自由変数として現れない, 定数でない文字とすれば, 
変数法則 \ref{valsubst}により$y$は$(U|x)(T)$の中にも自由変数として現れないから, 
代入法則 \ref{substfree}, \ref{substfund}, \ref{substwedge}により上記の記号列は
\[
  U \in a \to ((U|x)(T)|y)(U \in a \wedge y = (U|x)(T))
\]
と一致する.
また$x$が$y$と異なり, $a$の中に自由変数として現れないことから, 
代入法則 \ref{substfree}, \ref{substfund}, \ref{substwedge}によりこの記号列は
\begin{equation}
\label{sthmosetfund3}
  U \in a \to ((U|x)(T)|y)((U|x)(x \in a \wedge y = T))
\end{equation}
と一致する.
故にこれが成り立つ.
またschema S4の適用により
\begin{equation}
\label{sthmosetfund4}
  (U|x)(x \in a \wedge y = T) \to \exists x(x \in a \wedge y = T)
\end{equation}
が成り立つ.
また$x$が$y$と異なり, $a$の中に自由変数として現れないことから, 
定理 \ref{sthmosetbasis}と推論法則 \ref{dedequiv}により
\begin{equation}
\label{sthmosetfund5}
  \exists x(x \in a \wedge y = T) \to y \in \{T\}_{x \in a}
\end{equation}
が成り立つ.
そこで(\ref{sthmosetfund4}), (\ref{sthmosetfund5})から, 推論法則 \ref{dedmmp}によって
\[
  (U|x)(x \in a \wedge y = T) \to y \in \{T\}_{x \in a}
\]
が成り立つ.
このことと$y$が定数でないことから, 推論法則 \ref{dedsubst}により
\[
  ((U|x)(T)|y)((U|x)(x \in a \wedge y = T) \to y \in \{T\}_{x \in a})
\]
が成り立つ.
ここで$y$が$a$, $T$の中に自由変数として現れないことから, 
変数法則 \ref{valoset}により$y$は$\{T\}_{x \in a}$の中に自由変数として現れないから, 
代入法則 \ref{substfree}, \ref{substfund}により上記の記号列は
\begin{equation}
\label{sthmosetfund6}
  ((U|x)(T)|y)((U|x)(x \in a \wedge y = T)) \to (U|x)(T) \in \{T\}_{x \in a}
\end{equation}
と一致する.
故にこれが成り立つ.
そこで(\ref{sthmosetfund3}), (\ref{sthmosetfund6})から, 
推論法則 \ref{dedmmp}によって(\ref{sthmosetfund1})が成り立つ.
(\ref{sthmosetfund2})が成り立つことは, 
(\ref{sthmosetfund1})と推論法則 \ref{dedmp}によって明らかである.
\halmos




\mathstrut
\begin{thm}
\label{sthmosetsubsetb}%定理5.31%確認済
$a$, $b$, $T$を集合とし, $x$を$a$及び$b$の中に自由変数として現れない文字とする.
このとき
\begin{equation}
\label{sthmosetsubsetb1}
  (\forall x \in a)(T \in b) \leftrightarrow \{T\}_{x \in a} \subset b
\end{equation}
が成り立つ.
特に
\begin{equation}
\label{sthmosetsubsetb2}
  \forall x(T \in b) \to \{T\}_{x \in a} \subset b
\end{equation}
が成り立つ.
またこれらから, 次の1)---4)が成り立つ.

1)
$(\forall x \in a)(T \in b)$ならば, $\{T\}_{x \in a} \subset b$.
また$\{T\}_{x \in a} \subset b$ならば, $(\forall x \in a)(T \in b)$.

2)
$x$が定数でなく, $x \in a \to T \in b$が成り立てば, $\{T\}_{x \in a} \subset b$.

3)
$\forall x(T \in b)$ならば, $\{T\}_{x \in a} \subset b$.

4)
$x$が定数でなく, $T \in b$が成り立てば, $\{T\}_{x \in a} \subset b$.
\end{thm}


\noindent{\bf 証明}
~$y$を$x$と異なり, $a$, $b$, $T$の中に自由変数として現れない, 定数でない文字とする.
また$\tau_{x}(x \in a \wedge y = T)$を$U$と書く.
このとき$U$は集合であり, Thm \ref{thmsps4}より
\[
  (\forall x \in a)(T \in b) \to ((U|x)(x \in a) \to (U|x)(T \in b))
\]
が成り立つ.
ここで$x$が$a$, $b$の中に自由変数として現れないことから, 
代入法則 \ref{substfree}, \ref{substfund}によりこの記号列は
\begin{equation}
\label{sthmosetsubsetb3}
  (\forall x \in a)(T \in b) \to (U \in a \to (U|x)(T) \in b)
\end{equation}
と一致する.
故にこれが成り立つ.
またThm \ref{1atb1t1awctbwc1}より
\begin{equation}
\label{sthmosetsubsetb4}
  (U \in a \to (U|x)(T) \in b) 
  \to (U \in a \wedge y = (U|x)(T) \to (U|x)(T) \in b \wedge y = (U|x)(T))
\end{equation}
が成り立つ.
また$x$が$y$と異なり, $a$の中に自由変数として現れないことから, 
定理 \ref{sthmosetbasis}と推論法則 \ref{dedequiv}により
\[
  y \in \{T\}_{x \in a} \to \exists x(x \in a \wedge y = T)
\]
が成り立つ.
ここで$U$の定義から, この記号列は
\[
  y \in \{T\}_{x \in a} \to (U|x)(x \in a \wedge y = T)
\]
と一致する.
また$x$が$y$と異なり, $a$の中に自由変数として現れないことから, 
代入法則 \ref{substfree}, \ref{substfund}, \ref{substwedge}によりこの記号列は
\[
  y \in \{T\}_{x \in a} \to U \in a \wedge y = (U|x)(T)
\]
と一致する.
故にこれが成り立つ.
そこで推論法則 \ref{dedaddf}により
\begin{multline}
\label{sthmosetsubsetb5}
  (U \in a \wedge y = (U|x)(T) \to (U|x)(T) \in b \wedge y = (U|x)(T)) \\
  \to (y \in \{T\}_{x \in a} \to (U|x)(T) \in b \wedge y = (U|x)(T))
\end{multline}
が成り立つ.
また定理 \ref{sthm=tineq}と推論法則 \ref{dedpreequiv}により
\[
  y = (U|x)(T) \to ((U|x)(T) \in b \to y \in b)
\]
が成り立つから, 推論法則 \ref{dedch}, \ref{dedtwch}によってわかるように
\[
  (U|x)(T) \in b \wedge y = (U|x)(T) \to y \in b
\]
が成り立つ.
故に推論法則 \ref{dedaddb}により
\begin{equation}
\label{sthmosetsubsetb6}
  (y \in \{T\}_{x \in a} \to (U|x)(T) \in b \wedge y = (U|x)(T)) 
  \to (y \in \{T\}_{x \in a} \to y \in b)
\end{equation}
が成り立つ.
そこで(\ref{sthmosetsubsetb3})---(\ref{sthmosetsubsetb6})から, 推論法則 \ref{dedmmp}によって
\begin{equation}
\label{sthmosetsubsetb7}
  (\forall x \in a)(T \in b) \to (y \in \{T\}_{x \in a} \to y \in b)
\end{equation}
が成り立つことがわかる.
ここで$y$は$a$, $b$, $T$の中に自由変数として現れないから, 
変数法則 \ref{valfund}, \ref{valspquanin}により
$y$は$(\forall x \in a)(T \in b)$の中に自由変数として現れない.
また$y$は定数でない.
故にこれらのことと(\ref{sthmosetsubsetb7})から, 推論法則 \ref{dedalltquansepfreeconst}により
\[
  (\forall x \in a)(T \in b) \to \forall y(y \in \{T\}_{x \in a} \to y \in b)
\]
が成り立つ.
ここで$y$は$a$, $T$の中に自由変数として現れないから, 
変数法則 \ref{valoset}により$y$は$\{T\}_{x \in a}$の中に自由変数として現れない.
また$y$は$b$の中に自由変数として現れない.
故に上記の記号列は
\begin{equation}
\label{sthmosetsubsetb8}
  (\forall x \in a)(T \in b) \to \{T\}_{x \in a} \subset b
\end{equation}
と同じである.
従ってこれが成り立つ.
また$z$を$x$と異なり, $a$, $b$, $T$の中に自由変数として現れない, 定数でない文字とするとき, 
定理 \ref{sthmsubsetbasis}より
\begin{equation}
\label{sthmosetsubsetb9}
  \{T\}_{x \in a} \subset b \to ((z|x)(T) \in \{T\}_{x \in a} \to (z|x)(T) \in b)
\end{equation}
が成り立つ.
また$x$が$a$の中に自由変数として現れないことから, 定理 \ref{sthmosetfund}より
\[
  z \in a \to (z|x)(T) \in \{T\}_{x \in a}
\]
が成り立つ.
故に推論法則 \ref{dedaddf}により
\[
  ((z|x)(T) \in \{T\}_{x \in a} \to (z|x)(T) \in b) \to (z \in a \to (z|x)(T) \in b)
\]
が成り立つ.
ここで$x$が$a$, $b$の中に自由変数として現れないことから, 
代入法則 \ref{substfree}, \ref{substfund}によりこの記号列は
\begin{equation}
\label{sthmosetsubsetb10}
  ((z|x)(T) \in \{T\}_{x \in a} \to (z|x)(T) \in b) \to (z|x)(x \in a \to T \in b)
\end{equation}
と一致する.
故にこれが成り立つ.
そこで(\ref{sthmosetsubsetb9}), (\ref{sthmosetsubsetb10})から, 推論法則 \ref{dedmmp}によって
\begin{equation}
\label{sthmosetsubsetb11}
  \{T\}_{x \in a} \subset b \to (z|x)(x \in a \to T \in b)
\end{equation}
が成り立つ.
ここで$z$は$a$, $b$, $T$の中に自由変数として現れないから, 
変数法則 \ref{valsubset}, \ref{valoset}により
$z$は$\{T\}_{x \in a} \subset b$の中に自由変数として現れない.
また$z$は定数でない.
故にこれらのことと(\ref{sthmosetsubsetb11})から, 推論法則 \ref{dedalltquansepfreeconst}により
\[
  \{T\}_{x \in a} \subset b \to \forall z((z|x)(x \in a \to T \in b))
\]
が成り立つ.
ここで$z$が$x$と異なり, $a$, $b$, $T$の中に自由変数として現れないことから, 
変数法則 \ref{valfund}により$z$は$x \in a \to T \in b$の中に自由変数として現れない.
故に代入法則 \ref{substquantrans}により, 上記の記号列は
\begin{equation}
\label{sthmosetsubsetb12}
  \{T\}_{x \in a} \subset b \to \forall x(x \in a \to T \in b)
\end{equation}
と一致する.
従ってこれが成り立つ.
またThm \ref{thmspallfund}と推論法則 \ref{dedequiv}により
\begin{equation}
\label{sthmosetsubsetb13}
  \forall x(x \in a \to T \in b) \to (\forall x \in a)(T \in b)
\end{equation}
が成り立つ.
そこで(\ref{sthmosetsubsetb12}), (\ref{sthmosetsubsetb13})から, 推論法則 \ref{dedmmp}によって
\begin{equation}
\label{sthmosetsubsetb14}
  \{T\}_{x \in a} \subset b \to (\forall x \in a)(T \in b)
\end{equation}
が成り立つ.
故に(\ref{sthmosetsubsetb8}), (\ref{sthmosetsubsetb14})から, 
推論法則 \ref{dedequiv}により(\ref{sthmosetsubsetb1})が成り立つ.
またThm \ref{thmquantspall2}より
\[
  \forall x(T \in b) \to (\forall x \in a)(T \in b)
\]
が成り立つから, これと(\ref{sthmosetsubsetb8})から, 
推論法則 \ref{dedmmp}によって(\ref{sthmosetsubsetb2})が成り立つ.

\noindent
1)
(\ref{sthmosetsubsetb1})と推論法則 \ref{dedeqfund}によって明らか.

\noindent
2)
1)と推論法則 \ref{dedspallfund}によって明らか.

\noindent
3)
(\ref{sthmosetsubsetb2})と推論法則 \ref{dedmp}によって明らか.

\noindent
4)
3)と推論法則 \ref{dedltthmquan}によって明らか.
\halmos




\mathstrut
\begin{thm}
\label{sthmosetsubset}%定理5.32%確認済
$a$, $b$, $T$を集合とし, $x$を$a$及び$b$の中に自由変数として現れない文字とする.
このとき
\begin{equation}
\label{sthmosetsubset1}
  a \subset b \to \{T\}_{x \in a} \subset \{T\}_{x \in b}
\end{equation}
が成り立つ.
またこのことから, 次の(\ref{sthmosetsubset2})が成り立つ.
\begin{equation}
\label{sthmosetsubset2}
  a \subset b \text{ならば,} ~\{T\}_{x \in a} \subset \{T\}_{x \in b}.
\end{equation}
\end{thm}


\noindent{\bf 証明}
~$y$を$x$と異なり, $a$, $b$, $T$の中に自由変数として現れない, 定数でない文字とする.
このとき$x$が$a$, $b$の中に自由変数として現れないことから, 定理 \ref{sthmspinsubset}より
\[
  a \subset b \to ((\exists x \in a)(y = T) \to (\exists x \in b)(y = T)), 
\]
即ち
\begin{equation}
\label{sthmosetsubset3}
  a \subset b \to (\exists x(x \in a \wedge y = T) \to \exists x(x \in b \wedge y = T))
\end{equation}
が成り立つ.
ここで$y$が$a$, $b$の中に自由変数として現れないことから, 
変数法則 \ref{valsubset}により$y$は$a \subset b$の中に自由変数として現れない.
また$y$は定数でない.
故にこれらのことと(\ref{sthmosetsubset3})から, 推論法則 \ref{dedalltquansepfreeconst}により
\begin{equation}
\label{sthmosetsubset4}
  a \subset b \to \forall y(\exists x(x \in a \wedge y = T) \to \exists x(x \in b \wedge y = T))
\end{equation}
が成り立つ.
また$x$が$a$, $b$の中に自由変数として現れないことと, 
$y$が$x$と異なり, $a$, $b$, $T$の中に自由変数として現れないことから, 定理 \ref{sthmosetsm}より
$\exists x(x \in a \wedge y = T)$と$\exists x(x \in b \wedge y = T)$は共に$y$について集合を作り得る.
故に定理 \ref{sthmsmtalltisetsubseteq}と推論法則 \ref{dedequiv}により
\[
  \forall y(\exists x(x \in a \wedge y = T) \to \exists x(x \in b \wedge y = T)) 
  \to \{y \mid \exists x(x \in a \wedge y = T)\} \subset \{y \mid \exists x(x \in b \wedge y = T)\}
\]
が成り立つ.
ここで$y$が$x$と異なり, $a$, $b$, $T$の中に自由変数として現れないことから, この記号列は
\begin{equation}
\label{sthmosetsubset5}
  \forall y(\exists x(x \in a \wedge y = T) \to \exists x(x \in b \wedge y = T)) 
  \to \{T\}_{x \in a} \subset \{T\}_{x \in b}
\end{equation}
と同じである.
故にこれが成り立つ.
そこで(\ref{sthmosetsubset4}), (\ref{sthmosetsubset5})から, 
推論法則 \ref{dedmmp}によって(\ref{sthmosetsubset1})が成り立つ.
(\ref{sthmosetsubset2})が成り立つことは, 
(\ref{sthmosetsubset1})と推論法則 \ref{dedmp}によって明らかである.
\halmos




\mathstrut
\begin{thm}
\label{sthmoset=}%定理5.33%確認済
$a$, $b$, $T$を集合とし, $x$を$a$及び$b$の中に自由変数として現れない文字とする.
このとき
\begin{equation}
\label{sthmoset=1}
  a = b \to \{T\}_{x \in a} = \{T\}_{x \in b}
\end{equation}
が成り立つ.
またこのことから, 次の(\ref{sthmoset=2})が成り立つ.
\begin{equation}
\label{sthmoset=2}
  a = b \text{ならば,} ~\{T\}_{x \in a} = \{T\}_{x \in b}.
\end{equation}
\end{thm}


\noindent{\bf 証明}
~$y$を$x$と異なり, $T$の中に自由変数として現れない文字とする.
このときThm \ref{T=Ut1TV=UV1}より
\[
  a = b \to (a|y)(\{T\}_{x \in y}) = (b|y)(\{T\}_{x \in y})
\]
が成り立つ.
ここで$x$が$y$と異なり, $a$及び$b$の中に自由変数として現れないことと, 
$y$が$T$の中に自由変数として現れないことから, 代入法則 \ref{substfree}, \ref{substoset}により
この記号列は(\ref{sthmoset=1})と一致する.
故に(\ref{sthmoset=1})が成り立つ.
(\ref{sthmoset=2})が成り立つことは, (\ref{sthmoset=1})と推論法則 \ref{dedmp}によって明らかである.
\halmos




\mathstrut
\begin{thm}
\label{sthmt=uoset=}%定理5.34%確認済
$a$, $T$, $U$を集合とし, $x$を文字とする.
このとき
\begin{equation}
\label{sthmt=uoset=1}
  (\forall x \in a)(T = U) \to \{T\}_{x \in a} = \{U\}_{x \in a}
\end{equation}
が成り立つ.
特に
\begin{equation}
\label{sthmt=uoset=2}
  \forall x(T = U) \to \{T\}_{x \in a} = \{U\}_{x \in a}
\end{equation}
が成り立つ.
またこれらから, 次の1)---4)が成り立つ.

1)
$(\forall x \in a)(T = U)$ならば, $\{T\}_{x \in a} = \{U\}_{x \in a}$.

2)
$x$が定数でなく, $x \in a \to T = U$が成り立てば, $\{T\}_{x \in a} = \{U\}_{x \in a}$.

3)
$\forall x(T = U)$ならば, $\{T\}_{x \in a} = \{U\}_{x \in a}$.

4)
$x$が定数でなく, $T = U$が成り立てば, $\{T\}_{x \in a} = \{U\}_{x \in a}$.
\end{thm}


\noindent{\bf 証明}
~$y$を$x$と異なり, $a$, $T$, $U$の中に自由変数として現れない, 定数でない文字とする.
また$z$を$y$と異なり, $T$, $U$の中に自由変数として現れない, 定数でない文字とする.
このときThm \ref{x=yt1x=zly=z1}より
\[
  (z|x)(T) = (z|x)(U) \to (y = (z|x)(T) \leftrightarrow y = (z|x)(U))
\]
が成り立つ.
ここで$x$が$y$と異なることと代入法則 \ref{substfund}, \ref{substequiv}によれば, この記号列は
\[
  (z|x)(T = U \to (y = T \leftrightarrow y = U))
\]
と一致する.
故にこれが成り立つ.
このことと$z$が定数でないことから, 推論法則 \ref{dedltthmquan}により
\[
  \forall z((z|x)(T = U \to (y = T \leftrightarrow y = U)))
\]
が成り立つ.
ここで$z$が$y$と異なり, $T$, $U$の中に自由変数として現れないことから, 
変数法則 \ref{valfund}, \ref{valequiv}により
$z$は$T = U \to (y = T \leftrightarrow y = U)$の中に自由変数として現れない.
故に代入法則 \ref{substquantrans}により, 上記の記号列は
\[
  \forall x(T = U \to (y = T \leftrightarrow y = U))
\]
と一致する.
従ってこれが成り立つ.
そこで推論法則 \ref{dedalltspquansep}により
\begin{equation}
\label{sthmt=uoset=3}
  (\forall x \in a)(T = U) \to (\forall x \in a)(y = T \leftrightarrow y = U)
\end{equation}
が成り立つ.
またThm \ref{thmspalleqspexsep}より, 
\[
  (\forall x \in a)(y = T \leftrightarrow y = U) 
  \to ((\exists x \in a)(y = T) \leftrightarrow (\exists x \in a)(y = U)), 
\]
即ち
\begin{equation}
\label{sthmt=uoset=4}
  (\forall x \in a)(y = T \leftrightarrow y = U) 
  \to (\exists x(x \in a \wedge y = T) \leftrightarrow \exists x(x \in a \wedge y = U))
\end{equation}
が成り立つ.
そこで(\ref{sthmt=uoset=3}), (\ref{sthmt=uoset=4})から, 推論法則 \ref{dedmmp}によって
\begin{equation}
\label{sthmt=uoset=5}
  (\forall x \in a)(T = U) 
  \to (\exists x(x \in a \wedge y = T) \leftrightarrow \exists x(x \in a \wedge y = U))
\end{equation}
が成り立つ.
ここで$y$が$a$, $T$, $U$の中に自由変数として現れないことから, 
変数法則 \ref{valfund}, \ref{valspquanin}により, 
$y$は$(\forall x \in a)(T = U)$の中に自由変数として現れない.
また$y$は定数でない.
故にこれらのことと(\ref{sthmt=uoset=5})から, 推論法則 \ref{dedalltquansepfreeconst}により
\begin{equation}
\label{sthmt=uoset=6}
  (\forall x \in a)(T = U) 
  \to \forall y(\exists x(x \in a \wedge y = T) \leftrightarrow \exists x(x \in a \wedge y = U))
\end{equation}
が成り立つ.
また定理 \ref{sthmalleqiset=}より
\[
  \forall y(\exists x(x \in a \wedge y = T) \leftrightarrow \exists x(x \in a \wedge y = U)) 
  \to \{y \mid \exists x(x \in a \wedge y = T)\} = \{y \mid \exists x(x \in a \wedge y = U)\}
\]
が成り立つ.
ここで$y$が$x$と異なり, $a$, $T$, $U$の中に自由変数として現れないことから, この記号列は
\begin{equation}
\label{sthmt=uoset=7}
  \forall y(\exists x(x \in a \wedge y = T) \leftrightarrow \exists x(x \in a \wedge y = U)) 
  \to \{T\}_{x \in a} = \{U\}_{x \in a}
\end{equation}
と同じである.
故にこれが成り立つ.
そこで(\ref{sthmt=uoset=6}), (\ref{sthmt=uoset=7})から, 
推論法則 \ref{dedmmp}によって(\ref{sthmt=uoset=1})が成り立つ.
またThm \ref{thmquantspall2}より
\[
  \forall x(T = U) \to (\forall x \in a)(T = U)
\]
が成り立つから, これと(\ref{sthmt=uoset=1})から, 
推論法則 \ref{dedmmp}によって(\ref{sthmt=uoset=2})が成り立つ.

\noindent
1)
(\ref{sthmt=uoset=1})と推論法則 \ref{dedmp}によって明らか.

\noindent
2)
1)と推論法則 \ref{dedspallfund}によって明らか.

\noindent
3)
(\ref{sthmt=uoset=2})と推論法則 \ref{dedmp}によって明らか.

\noindent
4)
3)と推論法則 \ref{dedltthmquan}によって明らか.
\halmos




\mathstrut
\begin{thm}
\label{sthmspinoset}%定理5.35%新規%確認済
$a$と$T$を集合, $R$を関係式とし, $x$を$a$及び$R$の中に自由変数として現れない文字とする.
また$y$を$x$と異なり, $a$及び$T$の中に自由変数として現れない文字とする.
このとき
\begin{align}
  \label{sthmspinoset1}
  &(\exists y \in \{T\}_{x \in a})(R) \leftrightarrow (\exists x \in a)((T|y)(R)), \\
  \mbox{} \notag \\
  \label{sthmspinoset2}
  &(\forall y \in \{T\}_{x \in a})(R) \leftrightarrow (\forall x \in a)((T|y)(R)), \\
  \mbox{} \notag \\
  \label{sthmspinoset3}
  &(!x \in a)((T|y)(R)) \to (!y \in \{T\}_{x \in a})(R), \\
  \mbox{} \notag \\
  \label{sthmspinoset4}
  &(\exists !x \in a)((T|y)(R)) \to (\exists !y \in \{T\}_{x \in a})(R)
\end{align}
がすべて成り立つ.
またこれらから, 次の1)---4)が成り立つ.

1)
$(\exists y \in \{T\}_{x \in a})(R)$ならば, $(\exists x \in a)((T|y)(R))$.
また$(\exists x \in a)((T|y)(R))$ならば, $(\exists y \in \{T\}_{x \in a})(R)$.

2)
$(\forall y \in \{T\}_{x \in a})(R)$ならば, $(\forall x \in a)((T|y)(R))$.
また$(\forall x \in a)((T|y)(R))$ならば, $(\forall y \in \{T\}_{x \in a})(R)$.

3)
$(!x \in a)((T|y)(R))$ならば, $(!y \in \{T\}_{x \in a})(R)$.

4)
$(\exists !x \in a)((T|y)(R))$ならば, $(\exists !y \in \{T\}_{x \in a})(R)$.
\end{thm}


\noindent{\bf 証明}
~まず(\ref{sthmspinoset1})が成り立つことを示す.
$z$を$y$と異なり, $a$, $T$, $R$の中に自由変数として現れない, 定数でない文字とする.
また$w$を$z$と異なり, $a$, $T$, $R$の中に自由変数として現れない, 定数でない文字とする.
また$(z|x)(T)$, $(w|y)(R)$をそれぞれ$T^{*}$, $R^{*}$と書く.
このとき$T^{*}$, $R^{*}$はそれぞれ集合, 関係式である.
また変数法則 \ref{valsubst}により, $z$, $w$はそれぞれ$R^{*}$, $T^{*}$の中に自由変数として現れない.
さて$z$と$w$が互いに異なり, 共に$a$の中に自由変数として現れないことと, 
$w$が$T^{*}$の中に自由変数として現れないことから, 
定義より$\{T^{*}\}_{z \in a}$は$\{w \mid \exists z(z \in a \wedge w = T^{*})\}$と同じであり, 
定理 \ref{sthmosetsm}より$\exists z(z \in a \wedge w = T^{*})$は$w$について集合を作り得る.
故に定理 \ref{sthmspiniset}より, 
\[
  (\exists w \in \{T^{*}\}_{z \in a})(R^{*}) 
  \leftrightarrow \exists_{\exists z(z \in a \wedge w = T^{*})}w(R^{*}), 
\]
即ち
\begin{equation}
\label{sthmspinoset5}
  (\exists w \in \{T^{*}\}_{z \in a})(R^{*}) 
  \leftrightarrow \exists w((\exists z \in a)(w = T^{*}) \wedge R^{*})
\end{equation}
が成り立つ.
また$z$が$R^{*}$の中に自由変数として現れないことから, 
Thm \ref{thmspexwrfree}と推論法則 \ref{dedeqch}により
\[
  (\exists z \in a)(w = T^{*}) \wedge R^{*} \leftrightarrow (\exists z \in a)(w = T^{*} \wedge R^{*})
\]
が成り立つ.
このことと$w$が定数でないことから, 推論法則 \ref{dedalleqquansepconst}により
\begin{equation}
\label{sthmspinoset6}
  \exists w((\exists z \in a)(w = T^{*}) \wedge R^{*}) 
  \leftrightarrow \exists w((\exists z \in a)(w = T^{*} \wedge R^{*}))
\end{equation}
が成り立つ.
また$w$が$z$と異なり, $a$の中に自由変数として現れないことから, 
変数法則 \ref{valfund}により$w$は$z \in a$の中に自由変数として現れないから, Thm \ref{thmexspexch}より
\begin{equation}
\label{sthmspinoset7}
  \exists w((\exists z \in a)(w = T^{*} \wedge R^{*})) 
  \leftrightarrow (\exists z \in a)(\exists w(w = T^{*} \wedge R^{*}))
\end{equation}
が成り立つ.
また$w$が$T^{*}$の中に自由変数として現れないことから, 
Thm \ref{thmspquan=}と推論法則 \ref{dedeqch}により, 
\[
  \exists_{w = T^{*}}w(R^{*}) \leftrightarrow (T^{*}|w)(R^{*}), 
\]
即ち
\[
  \exists w(w = T^{*} \wedge R^{*}) \leftrightarrow (T^{*}|w)(R^{*})
\]
が成り立つ.
このことと$z$が定数でないことから, 推論法則 \ref{dedalleqspquansepconst}により
\begin{equation}
\label{sthmspinoset8}
  (\exists z \in a)(\exists w(w = T^{*} \wedge R^{*})) 
  \leftrightarrow (\exists z \in a)((T^{*}|w)(R^{*}))
\end{equation}
が成り立つ.
そこで(\ref{sthmspinoset5})---(\ref{sthmspinoset8})から, 推論法則 \ref{dedeqtrans}によって
\begin{equation}
\label{sthmspinoset9}
  (\exists w \in \{T^{*}\}_{z \in a})(R^{*}) \leftrightarrow (\exists z \in a)((T^{*}|w)(R^{*}))
\end{equation}
が成り立つことがわかる.
さていま$z$は$a$, $T$の中に自由変数として現れず, $x$は$a$の中に自由変数として現れないから, 
代入法則 \ref{substosettrans}により, 
\[
  \{(z|x)(T)\}_{z \in a} \equiv \{T\}_{x \in a}, 
\]
即ち
\begin{equation}
\label{sthmspinoset10}
  \{T^{*}\}_{z \in a} \equiv \{T\}_{x \in a}
\end{equation}
が成り立つ.
また$y$と$w$は共に$a$, $T$の中に自由変数として現れないから, 
変数法則 \ref{valoset}により, これらは共に$\{T\}_{x \in a}$の中に自由変数として現れない.
このことと$w$が$R$の中に自由変数として現れないことから, 代入法則 \ref{substspquanintrans}により, 
\[
  (\exists w \in \{T\}_{x \in a})((w|y)(R)) \equiv (\exists y \in \{T\}_{x \in a})(R), 
\]
即ち
\begin{equation}
\label{sthmspinoset11}
  (\exists w \in \{T\}_{x \in a})(R^{*}) \equiv (\exists y \in \{T\}_{x \in a})(R)
\end{equation}
が成り立つ.
また$w$は$R$の中に自由変数として現れないから, 代入法則 \ref{substtrans}により, 
\[
  (T^{*}|w)((w|y)(R)) \equiv (T^{*}|y)(R), 
\]
即ち
\begin{equation}
\label{sthmspinoset12}
  (T^{*}|w)(R^{*}) \equiv ((z|x)(T)|y)(R)
\end{equation}
が成り立つ.
また$y$が$x$, $z$と異なり, $x$が$R$の中に自由変数として現れないことから, 
代入法則 \ref{substfree}, \ref{substsubst}により
\begin{equation}
\label{sthmspinoset13}
  ((z|x)(T)|y)(R) \equiv (z|x)((T|y)(R))
\end{equation}
が成り立つ.
また$z$は$T$, $R$の中に自由変数として現れないから, 
変数法則 \ref{valsubst}により, $z$は$(T|y)(R)$の中に自由変数として現れない.
このことと$x$, $z$が共に$a$の中に自由変数として現れないことから, 
代入法則 \ref{substspquanintrans}により
\begin{equation}
\label{sthmspinoset14}
  (\exists z \in a)((z|x)((T|y)(R))) \equiv (\exists x \in a)((T|y)(R))
\end{equation}
が成り立つ.
以上の(\ref{sthmspinoset10})---(\ref{sthmspinoset14})から, 
(\ref{sthmspinoset9})が(\ref{sthmspinoset1})と一致することがわかる.
故に(\ref{sthmspinoset1})が成り立つ.

次に(\ref{sthmspinoset2})が成り立つことを示す.
いま示したように(\ref{sthmspinoset1})が成り立つが, 
(\ref{sthmspinoset1})における$R$は$x$がその中に自由変数として現れないような任意の関係式で良いので, 
$R$を$\neg R$とした
\[
  (\exists y \in \{T\}_{x \in a})(\neg R) \leftrightarrow (\exists x \in a)((T|y)(\neg R))
\]
も成り立つ ($x$が$R$の中に自由変数として現れないことから, 
変数法則 \ref{valfund}により$x$は$\neg R$の中に自由変数として現れない).
ここで代入法則 \ref{substfund}によれば, この記号列は
\[
  (\exists y \in \{T\}_{x \in a})(\neg R) \leftrightarrow (\exists x \in a)(\neg (T|y)(R))
\]
と一致するから, これが成り立つ.
故に推論法則 \ref{dedeqcp}により, 
\[
  \neg (\exists y \in \{T\}_{x \in a})(\neg R) \leftrightarrow \neg (\exists x \in a)(\neg (T|y)(R)), 
\]
即ち(\ref{sthmspinoset2})が成り立つ.

次に(\ref{sthmspinoset3})が成り立つことを示す.
$\tau_{x}(x \in a \wedge (T|y)(R))$を$U$と書けば, $U$は集合であり, 
変数法則 \ref{valtau}により, $x$は$U$の中に自由変数として現れない.
また$y$が$x$と異なり, $a$, $T$の中に自由変数として現れないことから, 
変数法則 \ref{valfund}, \ref{valsubst}, \ref{valtau}, \ref{valwedge}によってわかるように, 
$y$も$U$の中に自由変数として現れない.
またThm \ref{thmsp!lspall}と推論法則 \ref{dedequiv}により
\begin{equation}
\label{sthmspinoset15}
  (!x \in a)((T|y)(R)) \to (\forall x \in a)((T|y)(R) \to x = U)
\end{equation}
が成り立つ.
さていま$u$を$x$と異なり, $a$, $T$, $R$の中に自由変数として現れない, 定数でない文字とする.
このとき変数法則 \ref{valfund}, \ref{valsubst}, \ref{valtau}, \ref{valwedge}によってわかるように, 
$u$は$U$の中に自由変数として現れない.
またThm \ref{T=Ut1TV=UV1}より
\[
  u = U \to (u|x)(T) = (U|x)(T)
\]
が成り立つから, 推論法則 \ref{dedaddb}により
\[
  ((u|x)((T|y)(R)) \to u = U) \to ((u|x)((T|y)(R)) \to (u|x)(T) = (U|x)(T))
\]
が成り立つ.
ここで$x$が$U$の中に自由変数として現れないことから, 
変数法則 \ref{valsubst}により$x$は$(U|x)(T)$の中にも自由変数として現れないから, 
代入法則 \ref{substfree}, \ref{substfund}により, 上記の記号列は
\[
  (u|x)(((T|y)(R) \to x = U) \to ((T|y)(R) \to T = (U|x)(T)))
\]
と一致する.
故にこれが成り立つ.
このことと$u$が定数でないことから, 推論法則 \ref{dedltthmquan}により
\[
  \forall u((u|x)(((T|y)(R) \to x = U) \to ((T|y)(R) \to T = (U|x)(T))))
\]
が成り立つ.
ここで$u$が$x$と異なり, $T$, $R$, $U$の中に自由変数として現れないことから, 
変数法則 \ref{valfund}, \ref{valsubst}によってわかるように, 
$u$は$((T|y)(R) \to x = U) \to ((T|y)(R) \to T = (U|x)(T))$の中に自由変数として現れない.
故に代入法則 \ref{substquantrans}によれば, 上記の記号列は
\[
  \forall x(((T|y)(R) \to x = U) \to ((T|y)(R) \to T = (U|x)(T)))
\]
と一致する.
従ってこれが成り立つ.
そこで推論法則 \ref{dedalltspquansep}により
\[
  (\forall x \in a)((T|y)(R) \to x = U) \to (\forall x \in a)((T|y)(R) \to T = (U|x)(T))
\]
が成り立つ.
ここで$y$が$U$, $T$の中に自由変数として現れないことから, 
変数法則 \ref{valsubst}により$y$は$(U|x)(T)$の中に自由変数として現れないから, 
代入法則 \ref{substfree}, \ref{substfund}により, 上記の記号列は
\begin{equation}
\label{sthmspinoset16}
  (\forall x \in a)((T|y)(R) \to x = U) \to (\forall x \in a)((T|y)(R \to y = (U|x)(T)))
\end{equation}
と一致する.
故にこれが成り立つ.
また上で示したように(\ref{sthmspinoset2})が成り立つが, 
(\ref{sthmspinoset2})における$R$は$x$がその中に自由変数として現れないような任意の関係式で良いので, 
$R$を$R \to y = (U|x)(T)$とした
\[
  (\forall y \in \{T\}_{x \in a})(R \to y = (U|x)(T)) 
  \leftrightarrow (\forall x \in a)((T|y)(R \to y = (U|x)(T)))
\]
が成り立つ ($x$が$y$と異なり, $R$, $(U|x)(T)$の中に自由変数として現れないことから, 
変数法則 \ref{valfund}により$x$は$R \to y = (U|x)(T)$の中に自由変数として現れない).
故に推論法則 \ref{dedequiv}により
\begin{equation}
\label{sthmspinoset17}
  (\forall x \in a)((T|y)(R \to y = (U|x)(T))) \to (\forall y \in \{T\}_{x \in a})(R \to y = (U|x)(T))
\end{equation}
が成り立つ.
また$y$が$(U|x)(T)$の中に自由変数として現れないことから, Thm \ref{thmspalltsp!}より
\begin{equation}
\label{sthmspinoset18}
  (\forall y \in \{T\}_{x \in a})(R \to y = (U|x)(T)) \to (!y \in \{T\}_{x \in a})(R)
\end{equation}
が成り立つ.
そこで(\ref{sthmspinoset15})---(\ref{sthmspinoset18})から, 
推論法則 \ref{dedmmp}によって(\ref{sthmspinoset3})が成り立つことがわかる.

最後に(\ref{sthmspinoset4})が成り立つことを示す.
上で示したように(\ref{sthmspinoset1})が成り立つから, 推論法則 \ref{dedequiv}により
\begin{equation}
\label{sthmspinoset19}
  (\exists x \in a)((T|y)(R)) \to (\exists y \in \{T\}_{x \in a})(R)
\end{equation}
が成り立つ.
またいま示したように(\ref{sthmspinoset3})が成り立つ.
そこで(\ref{sthmspinoset3}), (\ref{sthmspinoset19})から, 
推論法則 \ref{dedfromaddw}により(\ref{sthmspinoset4})が成り立つ.

\noindent
1)
(\ref{sthmspinoset1})と推論法則 \ref{dedeqfund}によって明らか.

\noindent
2)
(\ref{sthmspinoset2})と推論法則 \ref{dedeqfund}によって明らか.

\noindent
3)
(\ref{sthmspinoset3})と推論法則 \ref{dedmp}によって明らか.

\noindent
4)
(\ref{sthmspinoset4})と推論法則 \ref{dedmp}によって明らか.
\halmos




\mathstrut
\begin{thm}
\label{sthmisetoset}%定理5.36%新規%確認済
$a$と$T$を集合, $R$を関係式とし, $x$を$a$の中に自由変数として現れない文字とする.
このとき
\begin{equation}
\label{sthmisetoset1}
  {\rm Set}_{x}(R) \to (a \in \{T\}_{x \in \{x \mid R\}} \leftrightarrow \exists x(R \wedge a = T))
\end{equation}
が成り立つ.
またこのことから, 次の1), 2)が成り立つ.

1)
$R$が$x$について集合を作り得るならば, 
$a \in \{T\}_{x \in \{x \mid R\}} \leftrightarrow \exists x(R \wedge a = T)$.

2)
$R$が$x$について集合を作り得るとする.
このとき$a \in \{T\}_{x \in \{x \mid R\}}$ならば, $\exists x(R \wedge a = T)$.
またこのとき$\exists x(R \wedge a = T)$ならば, $a \in \{T\}_{x \in \{x \mid R\}}$.
\end{thm}


\noindent{\bf 証明}
~定理 \ref{sthmspiniset}より, 
\[
  {\rm Set}_{x}(R) \to ((\exists x \in \{x \mid R\})(a = T) \leftrightarrow \exists_{R}x(a = T)), 
\]
即ち
\begin{equation}
\label{sthmisetoset2}
  {\rm Set}_{x}(R) 
  \to (\exists x(x \in \{x \mid R\} \wedge a = T) \leftrightarrow \exists x(R \wedge a = T))
\end{equation}
が成り立つ.
また変数法則 \ref{valiset}により$x$は$\{x \mid R\}$の中に自由変数として現れないから, 
このことと$x$が$a$の中に自由変数として現れないことから, 定理 \ref{sthmosetbasis}より
\[
  a \in \{T\}_{x \in \{x \mid R\}} \leftrightarrow \exists x(x \in \{x \mid R\} \wedge a = T)
\]
が成り立つ.
故に推論法則 \ref{dedaddeqeq}により
\[
  (a \in \{T\}_{x \in \{x \mid R\}} \leftrightarrow \exists x(R \wedge a = T)) 
  \leftrightarrow (\exists x(x \in \{x \mid R\} \wedge a = T) \leftrightarrow \exists x(R \wedge a = T))
\]
が成り立つ.
故に推論法則 \ref{dedequiv}により
\begin{equation}
\label{sthmisetoset3}
  (\exists x(x \in \{x \mid R\} \wedge a = T) \leftrightarrow \exists x(R \wedge a = T)) 
  \to (a \in \{T\}_{x \in \{x \mid R\}} \leftrightarrow \exists x(R \wedge a = T))
\end{equation}
が成り立つ.
そこで(\ref{sthmisetoset2}), (\ref{sthmisetoset3})から, 
推論法則 \ref{dedmmp}によって(\ref{sthmisetoset1})が成り立つ.

\noindent
1)
(\ref{sthmisetoset1})と推論法則 \ref{dedmp}によって明らか.

\noindent
2)
1)と推論法則 \ref{dedeqfund}によって明らか.
\halmos




\mathstrut
\begin{thm}
\label{sthmisetosetfund}%定理5.37%新規%確認済
$T$と$U$を集合, $R$を関係式とし, $x$を文字とする.
このとき
\begin{equation}
\label{sthmisetosetfund1}
  {\rm Set}_{x}(R) \to ((U|x)(R) \to (U|x)(T) \in \{T\}_{x \in \{x \mid R\}})
\end{equation}
が成り立つ.
またこのことから, 次の1), 2)が成り立つ.

1)
$R$が$x$について集合を作り得るならば, 
$(U|x)(R) \to (U|x)(T) \in \{T\}_{x \in \{x \mid R\}}$.

2)
$R$が$x$について集合を作り得るとする.
このとき$(U|x)(R)$ならば, $(U|x)(T) \in \{T\}_{x \in \{x \mid R\}}$.
\end{thm}


\noindent{\bf 証明}
~定理 \ref{sthmisetbasis}と推論法則 \ref{dedpreequiv}により
\begin{equation}
\label{sthmisetosetfund2}
  {\rm Set}_{x}(R) \to ((U|x)(R) \to U \in \{x \mid R\})
\end{equation}
が成り立つ.
またThm \ref{1atb1t11btc1t1atc11}より
\begin{multline}
\label{sthmisetosetfund3}
  ((U|x)(R) \to U \in \{x \mid R\}) \\
  \to ((U \in \{x \mid R\} \to (U|x)(T) \in \{T\}_{x \in \{x \mid R\}}) 
  \to ((U|x)(R) \to (U|x)(T) \in \{T\}_{x \in \{x \mid R\}}))
\end{multline}
が成り立つ.
また変数法則 \ref{valiset}により$x$は$\{x \mid R\}$の中に自由変数として現れないから, 
定理 \ref{sthmosetfund}より
\[
  U \in \{x \mid R\} \to (U|x)(T) \in \{T\}_{x \in \{x \mid R\}}
\]
が成り立つ.
故に推論法則 \ref{ded1atb1tbtrue2}により
\begin{multline}
\label{sthmisetosetfund4}
  ((U \in \{x \mid R\} \to (U|x)(T) \in \{T\}_{x \in \{x \mid R\}}) 
  \to ((U|x)(R) \to (U|x)(T) \in \{T\}_{x \in \{x \mid R\}})) \\
  \to ((U|x)(R) \to (U|x)(T) \in \{T\}_{x \in \{x \mid R\}})
\end{multline}
が成り立つ.
そこで(\ref{sthmisetosetfund2})---(\ref{sthmisetosetfund4})から, 
推論法則 \ref{dedmmp}によって(\ref{sthmisetosetfund1})が成り立つことがわかる.

\noindent
1)
(\ref{sthmisetosetfund1})と推論法則 \ref{dedmp}によって明らか.

\noindent
2)
1)と推論法則 \ref{dedmp}によって明らか.
\halmos




\mathstrut
\begin{thm}
\label{sthmuopairoset}%定理5.38%確認済
$a$, $b$, $T$を集合とし, $x$を$a$及び$b$の中に自由変数として現れない文字とする.
このとき
\begin{equation}
\label{sthmuopairoset1}
  \{T\}_{x \in \{a, b\}} = \{(a|x)(T), (b|x)(T)\}
\end{equation}
が成り立つ.
\end{thm}


\noindent{\bf 証明}
~$y$を$x$と異なり, $a$, $b$, $T$の中に自由変数として現れない, 定数でない文字とする.
このとき$x$が$a$, $b$の中に自由変数として現れないことから, 定理 \ref{sthmspinuopair}より, 
\[
  (\exists x \in \{a, b\})(y = T) \leftrightarrow (a|x)(y = T) \vee (b|x)(y = T), 
\]
即ち
\[
  \exists x(x \in \{a, b\} \wedge y = T) \leftrightarrow (a|x)(y = T) \vee (b|x)(y = T)
\]
が成り立つ.
ここで$x$が$y$と異なることと代入法則 \ref{substfund}から, この記号列は
\[
  \exists x(x \in \{a, b\} \wedge y = T) \leftrightarrow y = (a|x)(T) \vee y = (b|x)(T)
\]
と一致する.
故にこれが成り立つ.
このことと$y$が定数でないことから, 定理 \ref{sthmalleqiset=}より
\begin{equation}
\label{sthmuopairoset2}
  \{y \mid \exists x(x \in \{a, b\} \wedge y = T)\} = \{y \mid y = (a|x)(T) \vee y = (b|x)(T)\}
\end{equation}
が成り立つ.
さていま$y$は$a$, $b$の中に自由変数として現れないから, 
変数法則 \ref{valnset}により, $y$は$\{a, b\}$の中に自由変数として現れない.
このことと$y$が$x$と異なり, $T$の中に自由変数として現れないことから, 
\begin{equation}
\label{sthmuopairoset3}
  \{y \mid \exists x(x \in \{a, b\} \wedge y = T)\} \equiv \{T\}_{x \in \{a, b\}}
\end{equation}
が成り立つ.
また$y$が$a$, $b$, $T$の中に自由変数として現れないことから, 
変数法則 \ref{valsubst}により, $y$は$(a|x)(T)$, $(b|x)(T)$の中に自由変数として現れないから, 
\begin{equation}
\label{sthmuopairoset4}
  \{y \mid y = (a|x)(T) \vee y = (b|x)(T)\} \equiv \{(a|x)(T), (b|x)(T)\}
\end{equation}
が成り立つ.
そこで(\ref{sthmuopairoset3}), (\ref{sthmuopairoset4})から, 
(\ref{sthmuopairoset2})が(\ref{sthmuopairoset1})と一致することがわかる.
故に(\ref{sthmuopairoset1})が成り立つ.
\halmos




\mathstrut
\begin{thm}
\label{sthmsingletonoset}%定理5.39%確認済
$a$と$T$を集合とし, $x$を$a$の中に自由変数として現れない文字とする.
このとき
\begin{equation}
\label{sthmsingletonoset1}
  \{T\}_{x \in \{a\}} = \{(a|x)(T)\}
\end{equation}
が成り立つ.
\end{thm}


\noindent{\bf 証明}
~$x$が$a$の中に自由変数として現れないことから, 
変数法則 \ref{valnset}により, $x$は$\{a\}$, $\{a, a\}$の中に自由変数として現れない.
また定理 \ref{sthmaa=a}と推論法則 \ref{ded=ch}により$\{a\} = \{a, a\}$が成り立つ.
故に定理 \ref{sthmoset=}より
\begin{equation}
\label{sthmsingletonoset2}
  \{T\}_{x \in \{a\}} = \{T\}_{x \in \{a, a\}}
\end{equation}
が成り立つ.
また$x$が$a$の中に自由変数として現れないことから, 定理 \ref{sthmuopairoset}より
\begin{equation}
\label{sthmsingletonoset3}
  \{T\}_{x \in \{a, a\}} = \{(a|x)(T), (a|x)(T)\}
\end{equation}
が成り立つ.
また定理 \ref{sthmaa=a}より
\begin{equation}
\label{sthmsingletonoset4}
  \{(a|x)(T), (a|x)(T)\} = \{(a|x)(T)\}
\end{equation}
が成り立つ.
そこで(\ref{sthmsingletonoset2})---(\ref{sthmsingletonoset4})から, 
推論法則 \ref{ded=trans}によって(\ref{sthmsingletonoset1})が成り立つことがわかる.
\halmos




\mathstrut
\begin{thm}
\label{sthmssetoset}%定理5.40%新規%確認済
$a$, $b$, $T$を集合, $R$を関係式とし, $x$を$a$及び$b$の中に自由変数として現れない文字とする.
このとき
\begin{equation}
\label{sthmssetoset1}
  b \in \{T\}_{x \in \{x \in a \mid R\}} \leftrightarrow (\exists x \in a)(R \wedge b = T)
\end{equation}
が成り立つ.
またこのことから, 次の1), 2)が成り立つ.

1)
$b \in \{T\}_{x \in \{x \in a \mid R\}}$ならば, $(\exists x \in a)(R \wedge b = T)$.

2)
$(\exists x \in a)(R \wedge b = T)$ならば, $b \in \{T\}_{x \in \{x \in a \mid R\}}$.
\end{thm}


\noindent{\bf 証明}
~$x$が$a$の中に自由変数として現れないことから, 
定理 \ref{sthmssetsm}より$x \in a \wedge R$は$x$について集合を作り得る.
このことと$x$が$b$の中に自由変数として現れないことから, 定理 \ref{sthmisetoset}より, 
\[
  b \in \{T\}_{x \in \{x \mid x \in a \wedge R\}} 
  \leftrightarrow \exists x((x \in a \wedge R) \wedge b = T), 
\]
即ち
\begin{equation}
\label{sthmssetoset2}
  b \in \{T\}_{x \in \{x \in a \mid R\}} \leftrightarrow \exists x((x \in a \wedge R) \wedge b = T)
\end{equation}
が成り立つ.
さていま$y$を$x$と異なり, $a$, $b$, $T$, $R$の中に自由変数として現れない, 定数でない文字とする.
このときThm \ref{1awb1wclaw1bwc1}より
\[
  (y \in (y|x)(a) \wedge (y|x)(R)) \wedge (y|x)(b) = (y|x)(T) 
  \leftrightarrow y \in (y|x)(a) \wedge ((y|x)(R) \wedge (y|x)(b) = (y|x)(T))
\]
が成り立つが, 代入法則 \ref{substfund}, \ref{substwedge}によればこの記号列は
\[
  (y|x)((x \in a \wedge R) \wedge b = T) \leftrightarrow (y|x)(x \in a \wedge (R \wedge b = T))
\]
と一致するから, これが成り立つ.
このことと$y$が定数でないことから, 推論法則 \ref{dedalleqquansepconst}により
\[
  \exists y((y|x)((x \in a \wedge R) \wedge b = T)) 
  \leftrightarrow \exists y((y|x)(x \in a \wedge (R \wedge b = T)))
\]
が成り立つ.
ここで$y$が$x$と異なり, $a$, $b$, $T$, $R$の中に自由変数として現れないことから, 
変数法則 \ref{valfund}, \ref{valwedge}により, 
$y$は$(x \in a \wedge R) \wedge b = T$, $x \in a \wedge (R \wedge b = T)$の中に自由変数として現れない.
故に代入法則 \ref{substquantrans}によれば, 上記の記号列は
\[
  \exists x((x \in a \wedge R) \wedge b = T) 
  \leftrightarrow \exists x(x \in a \wedge (R \wedge b = T)), 
\]
即ち
\begin{equation}
\label{sthmssetoset3}
  \exists x((x \in a \wedge R) \wedge b = T) \leftrightarrow (\exists x \in a)(R \wedge b = T)
\end{equation}
と一致する.
従ってこれが成り立つ.
そこで(\ref{sthmssetoset2}), (\ref{sthmssetoset3})から, 
推論法則 \ref{dedeqtrans}によって(\ref{sthmssetoset1})が成り立つ.
1), 2)が成り立つことは, (\ref{sthmssetoset1})と推論法則 \ref{dedeqfund}によって明らかである.
\halmos




\mathstrut
\begin{thm}
\label{sthmssetosetfund}%定理5.41%新規%確認済
$a$, $T$, $U$を集合, $R$を関係式とし, $x$を$a$の中に自由変数として現れない文字とする.
このとき
\begin{equation}
\label{sthmssetosetfund1}
  U \in a \wedge (U|x)(R) \to (U|x)(T) \in \{T\}_{x \in \{x \in a \mid R\}}
\end{equation}
が成り立つ.
またこのことから, 次の(\ref{sthmssetosetfund2})が成り立つ.
\begin{equation}
\label{sthmssetosetfund2}
  U \in a \text{と} (U|x)(R) \text{が共に成り立てば,} ~(U|x)(T) \in \{T\}_{x \in \{x \in a \mid R\}}.
\end{equation}
\end{thm}


\noindent{\bf 証明}
~$x$が$a$の中に自由変数として現れないことから, 
定理 \ref{sthmssetsm}より$x \in a \wedge R$は$x$について集合を作り得る.
故に定理 \ref{sthmisetosetfund}より, 
\[
  (U|x)(x \in a \wedge R) \to (U|x)(T) \in \{T\}_{x \in \{x \mid x \in a \wedge R\}}, 
\]
即ち
\[
  (U|x)(x \in a \wedge R) \to (U|x)(T) \in \{T\}_{x \in \{x \in a \mid R\}}
\]
が成り立つ.
ここで$x$が$a$の中に自由変数として現れないことから, 
代入法則 \ref{substfree}, \ref{substfund}, \ref{substwedge}により, 
この記号列は(\ref{sthmssetosetfund1})と一致する.
故に(\ref{sthmssetosetfund1})が成り立つ.
(\ref{sthmssetosetfund2})が成り立つことは, 
(\ref{sthmssetosetfund1})と推論法則 \ref{dedmp}, \ref{dedwedge}によって明らかである.
\halmos




\mathstrut
\begin{thm}
\label{sthmsset&oset}%定理5.42%新規%確認済
$a$と$T$を集合, $R$を関係式とし, $x$を$a$及び$R$の中に自由変数として現れない文字とする.
また$y$を$x$と異なり, $a$及び$T$の中に自由変数として現れない文字とする.
このとき
\begin{equation}
\label{sthmsset&oset1}
  \{y \in \{T\}_{x \in a} \mid R\} = \{T\}_{x \in \{x \in a \mid (T|y)(R)\}}
\end{equation}
が成り立つ.
\end{thm}


\noindent{\bf 証明}
~$z$を$x$と異なり, $a$, $T$, $R$の中に自由変数として現れない, 定数でない文字とする.
このとき変数法則 \ref{valoset}により, $z$は$\{T\}_{x \in a}$の中に自由変数として現れない.
また変数法則 \ref{valsubst}, \ref{valsset}により, 
$z$は$\{x \in a \mid (T|y)(R)\}$の中にも自由変数として現れない.
さて$x$が$z$と異なり, $a$の中に自由変数として現れないことから, 定理 \ref{sthmosetbasis}より, 
\[
  z \in \{T\}_{x \in a} \leftrightarrow \exists x(x \in a \wedge z = T), 
\]
即ち
\[
  z \in \{T\}_{x \in a} \leftrightarrow (\exists x \in a)(z = T)
\]
が成り立つ.
故に推論法則 \ref{dedaddeqw}により
\begin{equation}
\label{sthmsset&oset2}
  z \in \{T\}_{x \in a} \wedge (z|y)(R) \leftrightarrow (\exists x \in a)(z = T) \wedge (z|y)(R)
\end{equation}
が成り立つ.
また$x$が$z$と異なり, $R$の中に自由変数として現れないことから, 
変数法則 \ref{valsubst}により, $x$は$(z|y)(R)$の中に自由変数として現れない.
故にThm \ref{thmspexwrfree}と推論法則 \ref{dedeqch}により
\begin{equation}
\label{sthmsset&oset3}
  (\exists x \in a)(z = T) \wedge (z|y)(R) \leftrightarrow (\exists x \in a)(z = T \wedge (z|y)(R))
\end{equation}
が成り立つ.
またThm \ref{thmspquanwch}より
\begin{equation}
\label{sthmsset&oset4}
  (\exists x \in a)(z = T \wedge (z|y)(R)) \leftrightarrow (\exists x \in a)((z|y)(R) \wedge z = T)
\end{equation}
が成り立つ.
ここで$w$を$x$, $y$, $z$と異なり, $a$, $T$, $R$の中に自由変数として現れない, 定数でない文字とする.
このときThm \ref{thms5eq}より
\[
  z = (w|x)(T) \to ((z|y)(R) \leftrightarrow ((w|x)(T)|y)(R))
\]
が成り立つから, 推論法則 \ref{dedeq&w}により
\[
  (z|y)(R) \wedge z = (w|x)(T) \leftrightarrow ((w|x)(T)|y)(R) \wedge z = (w|x)(T)
\]
が成り立つ.
故に推論法則 \ref{dedaddeqw}により
\begin{equation}
\label{sthmsset&oset5}
  w \in a \wedge ((z|y)(R) \wedge z = (w|x)(T)) 
  \leftrightarrow w \in a \wedge (((w|x)(T)|y)(R) \wedge z = (w|x)(T))
\end{equation}
が成り立つ.
またThm \ref{1awb1wclaw1bwc1}と推論法則 \ref{dedeqch}により
\begin{equation}
\label{sthmsset&oset6}
  w \in a \wedge (((w|x)(T)|y)(R) \wedge z = (w|x)(T)) 
  \leftrightarrow (w \in a \wedge ((w|x)(T)|y)(R)) \wedge z = (w|x)(T)
\end{equation}
が成り立つ.
また$x$が$a$の中に自由変数として現れないことから, 
定理 \ref{sthmssetbasis}と推論法則 \ref{dedeqch}により
\[
  w \in a \wedge (w|x)((T|y)(R)) \leftrightarrow w \in \{x \in a \mid (T|y)(R)\}
\]
が成り立つ.
ここで$y$が$x$, $w$と異なることと, $x$が$R$の中に自由変数として現れないことから, 
代入法則 \ref{substfree}, \ref{substsubst}により, この記号列は
\[
  w \in a \wedge ((w|x)(T)|y)(R) \leftrightarrow w \in \{x \in a \mid (T|y)(R)\}
\]
と一致する.
故にこれが成り立つ.
そこで推論法則 \ref{dedaddeqw}により
\begin{equation}
\label{sthmsset&oset7}
  (w \in a \wedge ((w|x)(T)|y)(R)) \wedge z = (w|x)(T) 
  \leftrightarrow w \in \{x \in a \mid (T|y)(R)\} \wedge z = (w|x)(T)
\end{equation}
が成り立つ.
以上の(\ref{sthmsset&oset5})---(\ref{sthmsset&oset7})から, 推論法則 \ref{dedeqtrans}によって
\[
  w \in a \wedge ((z|y)(R) \wedge z = (w|x)(T)) 
  \leftrightarrow w \in \{x \in a \mid (T|y)(R)\} \wedge z = (w|x)(T)
\]
が成り立つことがわかる.
ここで$x$は$z$と異なり, $a$の中に自由変数として現れず, 
上述のように$(z|y)(R)$の中にも自由変数として現れず, 
変数法則 \ref{valsset}により$\{x \in a \mid (T|y)(R)\}$の中にも自由変数として現れないから, 
代入法則 \ref{substfree}, \ref{substfund}, \ref{substwedge}によってわかるように, 
上記の記号列は
\[
  (w|x)(x \in a \wedge ((z|y)(R) \wedge z = T)) 
  \leftrightarrow (w|x)(x \in \{x \in a \mid (T|y)(R)\} \wedge z = T)
\]
と一致する.
故にこれが成り立つ.
このことと$w$が定数でないことから, 推論法則 \ref{dedalleqquansepconst}により
\[
  \exists w((w|x)(x \in a \wedge ((z|y)(R) \wedge z = T))) 
  \leftrightarrow \exists w((w|x)(x \in \{x \in a \mid (T|y)(R)\} \wedge z = T))
\]
が成り立つ.
ここで$w$が$x$, $z$と異なり, $a$, $T$, $R$の中に自由変数として現れないことから, 
変数法則 \ref{valfund}, \ref{valsubst}, \ref{valwedge}, \ref{valsset}によってわかるように, 
$w$は$x \in a \wedge ((z|y)(R) \wedge z = T)$, 
$x \in \{x \in a \mid (T|y)(R)\} \wedge z = T$の中に自由変数として現れない.
故に代入法則 \ref{substquantrans}によれば, 上記の記号列は, 
\[
  \exists x(x \in a \wedge ((z|y)(R) \wedge z = T)) 
  \leftrightarrow \exists x(x \in \{x \in a \mid (T|y)(R)\} \wedge z = T), 
\]
即ち
\begin{equation}
\label{sthmsset&oset8}
  (\exists x \in a)((z|y)(R) \wedge z = T) 
  \leftrightarrow \exists x(x \in \{x \in a \mid (T|y)(R)\} \wedge z = T)
\end{equation}
と一致する.
従ってこれが成り立つ.
そこで(\ref{sthmsset&oset2}), (\ref{sthmsset&oset3}), 
(\ref{sthmsset&oset4}), (\ref{sthmsset&oset8})から, 推論法則 \ref{dedeqtrans}によって
\[
  z \in \{T\}_{x \in a} \wedge (z|y)(R) 
  \leftrightarrow \exists x(x \in \{x \in a \mid (T|y)(R)\} \wedge z = T)
\]
が成り立つことがわかる.
このことと$z$が定数でないことから, 定理 \ref{sthmalleqiset=}より, 
\[
  \{z \mid z \in \{T\}_{x \in a} \wedge (z|y)(R)\} 
  = \{z \mid \exists x(x \in \{x \in a \mid (T|y)(R)\} \wedge z = T)\}, 
\]
即ち
\begin{equation}
\label{sthmsset&oset9}
  \{z \in \{T\}_{x \in a} \mid (z|y)(R)\} 
  = \{z \mid \exists x(x \in \{x \in a \mid (T|y)(R)\} \wedge z = T)\}
\end{equation}
が成り立つ.
さていま$y$は$a$, $T$の中に自由変数として現れないから, 
変数法則 \ref{valoset}により, $y$は$\{T\}_{x \in a}$の中に自由変数として現れない.
また上述のように, $z$は$\{T\}_{x \in a}$, $R$の中に自由変数として現れない.
故に代入法則 \ref{substssettrans}により, 
\begin{equation}
\label{sthmsset&oset10}
  \{z \in \{T\}_{x \in a} \mid (z|y)(R)\} \equiv \{y \in \{T\}_{x \in a} \mid R\}
\end{equation}
が成り立つ.
またこれも上述のように, $z$は$x$と異なり, 
$\{x \in a \mid (T|y)(R)\}$, $T$の中に自由変数として現れないから, 定義より
\begin{equation}
\label{sthmsset&oset11}
  \{z \mid \exists x(x \in \{x \in a \mid (T|y)(R)\} \wedge z = T)\} 
  \equiv \{T\}_{x \in \{x \in a \mid (T|y)(R)\}}
\end{equation}
である.
そこで(\ref{sthmsset&oset10}), (\ref{sthmsset&oset11})から, 
(\ref{sthmsset&oset9})が(\ref{sthmsset&oset1})と一致することがわかる.
故に(\ref{sthmsset&oset1})が成り立つ.
\halmos




\mathstrut
\begin{thm}
\label{sthmosetoset}%定理5.43%新規%確認済
$a$, $T$, $U$を集合とし, $x$を$a$及び$U$の中に自由変数として現れない文字とする.
また$y$を$x$と異なり, $a$及び$T$の中に自由変数として現れない文字とする.
このとき
\begin{equation}
\label{sthmosetoset1}
  \{U\}_{y \in \{T\}_{x \in a}} = \{(T|y)(U)\}_{x \in a}
\end{equation}
が成り立つ.
\end{thm}


\noindent{\bf 証明}
~$z$を$x$, $y$と異なり, $a$, $T$, $U$の中に自由変数として現れない, 定数でない文字とする.
このとき変数法則 \ref{valoset}により, $z$は$\{T\}_{x \in a}$の中に自由変数として現れない.
また変数法則 \ref{valsubst}により, $z$は$(T|y)(U)$の中にも自由変数として現れない.
さて$x$が$z$と異なり, $U$の中に自由変数として現れないことから, 
変数法則 \ref{valfund}により, $x$は$z = U$の中に自由変数として現れない.
このことと, $x$が$a$の中にも自由変数として現れないこと, 
及び$y$が$x$と異なり, $a$, $T$の中に自由変数として現れないことから, 定理 \ref{sthmspinoset}より, 
\[
  (\exists y \in \{T\}_{x \in a})(z = U) \leftrightarrow (\exists x \in a)((T|y)(z = U)), 
\]
即ち
\[
  \exists y(y \in \{T\}_{x \in a} \wedge z = U) \leftrightarrow \exists x(x \in a \wedge (T|y)(z = U))
\]
が成り立つ.
ここで$y$が$z$と異なることと代入法則 \ref{substfund}によれば, この記号列は
\[
  \exists y(y \in \{T\}_{x \in a} \wedge z = U) \leftrightarrow \exists x(x \in a \wedge z = (T|y)(U))
\]
と一致するから, これが成り立つ.
このことと$z$が定数でないことから, 定理 \ref{sthmalleqiset=}より
\begin{equation}
\label{sthmosetoset2}
  \{z \mid \exists y(y \in \{T\}_{x \in a} \wedge z = U)\} 
  = \{z \mid \exists x(x \in a \wedge z = (T|y)(U))\}
\end{equation}
が成り立つ.
ここで$z$は$x$, $y$と異なり, 
上述のように$\{T\}_{x \in a}$, $U$, $a$, $(T|y)(U)$の中に自由変数として現れないから, 
定義より(\ref{sthmosetoset2})は(\ref{sthmosetoset1})と一致する.
故に(\ref{sthmosetoset1})が成り立つ.
\halmos
%ここまで確認



\newpage
\setcounter{defi}{0}
\section{差集合, 空集合}



%確認済%koko
この節では, 表題の集合を定義し, これらの性質を調べる.




\mathstrut
\begin{defo}
\label{complement}%変形18%確認済
$\mathscr{T}$を特殊記号として$\in$を持つ理論とし, $a$と$b$を$\mathscr{T}$の記号列とする.
また$x$と$y$を共に$a$及び$b$の中に自由変数として現れない文字とする.
このとき
\[
  \{x \in a \mid x \notin b\} \equiv \{y \in a \mid y \notin b\}
\]
が成り立つ.
\end{defo}


\noindent{\bf 証明}
~$x$と$y$が同じ文字ならば明らかだから, 以下$x$と$y$は異なる文字であるとする.
このとき$y$が$x$と異なり, $b$の中に自由変数として現れないことから, 
変数法則 \ref{valfund}により, $y$は$x \notin b$の中に自由変数として現れない.
このことと$x$, $y$が共に$a$の中に自由変数として現れないことから, 代入法則 \ref{substssettrans}により
\[
  \{x \in a \mid x \notin b\} \equiv \{y \in a \mid (y|x)(x \notin b)\}
\]
が成り立つ.
また$x$が$b$の中に自由変数として現れないことから, 代入法則 \ref{substfree}, \ref{substfund}により
\[
  (y|x)(x \notin b) \equiv y \notin b
\]
が成り立つ.
故に本法則が成り立つ.
\halmos




\mathstrut
\begin{defi}
\label{def-}%定義1%確認済
$\mathscr{T}$を特殊記号として$\in$を持つ理論とし, $a$と$b$を$\mathscr{T}$の記号列とする.
また$x$と$y$を共に$a$及び$b$の中に自由変数として現れない文字とする.
このとき変形法則 \ref{complement}によれば, 
$\{x \in a \mid x \notin b\}$と$\{y \in a \mid y \notin b\}$は同じ記号列となる.
$a$と$b$に対して定まるこの記号列を, $(a) - (b)$と書き表す (括弧は適宜省略する).
これを$a$と$b$の (詳しくは, $a$から$b$を引いた) \textbf{集合論的差} (set-theoretic difference) あるいは
単に\textbf{差} (difference) という.
\end{defi}




\mathstrut%確認済%koko
以下の変数法則 \ref{val-}, 一般代入法則 \ref{gsubst-}, 代入法則 \ref{subst-}, 
構成法則 \ref{form-}では, $\mathscr{T}$を特殊記号として$\in$を持つ理論とし, 
これらの法則における``記号列'', ``集合''とは, 
それぞれ$\mathscr{T}$の記号列, $\mathscr{T}$の対象式のこととする.




\mathstrut
\begin{valu}
\label{val-}%変数29%確認済
$a$と$b$を記号列とし, $x$を文字とする.
$x$が$a$及び$b$の中に自由変数として現れなければ, $x$は$a - b$の中に自由変数として現れない.
\end{valu}


\noindent{\bf 証明}
~このとき定義から$a - b$は$\{x \in a \mid x \notin b\}$と同じである.
変数法則 \ref{valsset}によれば, $x$はこの中に自由変数として現れない.
\halmos




\mathstrut
\begin{gsub}
\label{gsubst-}%一般代入33%新規%確認済
$a$と$b$を記号列とする.
また$n$を自然数とし, $T_{1}, T_{2}, \cdots, T_{n}$を記号列とする.
また$x_{1}, x_{2}, \cdots, x_{n}$を, どの二つも互いに異なる文字とする.
このとき
\[
  (T_{1}|x_{1}, T_{2}|x_{2}, \cdots, T_{n}|x_{n})(a - b) 
  \equiv (T_{1}|x_{1}, T_{2}|x_{2}, \cdots, T_{n}|x_{n})(a) - (T_{1}|x_{1}, T_{2}|x_{2}, \cdots, T_{n}|x_{n})(b)
\]
が成り立つ.
\end{gsub}


\noindent{\bf 証明}
~$y$を$x_{1}, x_{2}, \cdots, x_{n}$のいずれとも異なり, 
$a, b, T_{1}, T_{2}, \cdots, T_{n}$のいずれの中にも自由変数として現れない文字とする.
このとき定義から$a - b$は$\{y \in a \mid y \notin b\}$と同じだから, 
\begin{equation}
\label{gsubst-1}
  (T_{1}|x_{1}, T_{2}|x_{2}, \cdots, T_{n}|x_{n})(a - b) 
  \equiv (T_{1}|x_{1}, T_{2}|x_{2}, \cdots, T_{n}|x_{n})(\{y \in a \mid y \notin b\})
\end{equation}
である.
また$y$が$x_{1}, x_{2}, \cdots, x_{n}$のいずれとも異なり, 
$T_{1}, T_{2}, \cdots, T_{n}$のいずれの中にも自由変数として現れないことから, 
一般代入法則 \ref{gsubstsset}により
\begin{multline}
\label{gsubst-2}
  (T_{1}|x_{1}, T_{2}|x_{2}, \cdots, T_{n}|x_{n})(\{y \in a \mid y \notin b\}) \\
  \equiv \{y \in (T_{1}|x_{1}, T_{2}|x_{2}, \cdots, T_{n}|x_{n})(a) \mid (T_{1}|x_{1}, T_{2}|x_{2}, \cdots, T_{n}|x_{n})(y \notin b)\}
\end{multline}
が成り立つ.
また$y$が$x_{1}, x_{2}, \cdots, x_{n}$のいずれとも異なることと一般代入法則 \ref{gsubstfund}により, 
\begin{equation}
\label{gsubst-3}
  (T_{1}|x_{1}, T_{2}|x_{2}, \cdots, T_{n}|x_{n})(y \notin b) 
  \equiv y \notin (T_{1}|x_{1}, T_{2}|x_{2}, \cdots, T_{n}|x_{n})(b)
\end{equation}
が成り立つ.
そこで(\ref{gsubst-1})---(\ref{gsubst-3})からわかるように, 
$(T_{1}|x_{1}, T_{2}|x_{2}, \cdots, T_{n}|x_{n})(a - b)$は
\begin{equation}
\label{gsubst-4}
  \{y \in (T_{1}|x_{1}, T_{2}|x_{2}, \cdots, T_{n}|x_{n})(a) \mid y \notin (T_{1}|x_{1}, T_{2}|x_{2}, \cdots, T_{n}|x_{n})(b)\}
\end{equation}
と一致する.
ここで$y$が$a, b, T_{1}, T_{2}, \cdots, T_{n}$のいずれの中にも自由変数として現れないことから, 
変数法則 \ref{valgsubst}により, 
$y$は$(T_{1}|x_{1}, T_{2}|x_{2}, \cdots, T_{n}|x_{n})(a)$及び
$(T_{1}|x_{1}, T_{2}|x_{2}, \cdots, T_{n}|x_{n})(b)$の中に自由変数として現れない.
故に定義から, (\ref{gsubst-4})は
\[
  (T_{1}|x_{1}, T_{2}|x_{2}, \cdots, T_{n}|x_{n})(a) - (T_{1}|x_{1}, T_{2}|x_{2}, \cdots, T_{n}|x_{n})(b)
\]
と同じである.
故に本法則が成り立つ.
\halmos




\mathstrut
\begin{subs}
\label{subst-}%代入40%確認済
$a$, $b$, $T$を記号列とし, $x$を文字とする.
このとき
\[
  (T|x)(a - b) \equiv (T|x)(a) - (T|x)(b)
\]
が成り立つ.
\end{subs}


\noindent{\bf 証明}
~一般代入法則 \ref{gsubst-}において, $n$が$1$の場合である.
\halmos




\mathstrut
\begin{form}
\label{form-}%構成46%確認済
$a$と$b$が集合ならば, $a - b$は集合である.
\end{form}


\noindent{\bf 証明}
~$x$を$a$, $b$の中に自由変数として現れない文字とするとき, 
$a - b$は$\{x \in a \mid x \notin b\}$である.
$a$と$b$が集合のとき, これが集合となることは, 
構成法則 \ref{formfund}, \ref{formsset}によって直ちにわかる.
\halmos




\mathstrut%確認済%koko
$a$と$b$が集合であるとき, 上記の構成法則 \ref{form-}により, $a - b$は集合である.
これを$a$と$b$の\textbf{差集合} (set difference) ともいう.




\mathstrut
\begin{thm}
\label{sthm-basis}%定理6.1%確認済
$a$, $b$, $c$を集合とするとき, 
\begin{equation}
\label{sthm-basis1}
  c \in a - b \leftrightarrow c \in a \wedge c \notin b
\end{equation}
が成り立つ.
またこのことから, 次の1), 2)が成り立つ.

1)
$c \in a - b$ならば, $c \in a$と$c \notin b$が共に成り立つ.

2)
$c \in a$と$c \notin b$が共に成り立てば, $c \in a - b$.
\end{thm}


\noindent{\bf 証明}
~$x$を$a$及び$b$の中に自由変数として現れない文字とするとき, 定理 \ref{sthmssetbasis}より
\[
  c \in \{x \in a \mid x \notin b\} \leftrightarrow c \in a \wedge (c|x)(x \notin b)
\]
が成り立つが, $a - b$の定義と代入法則 \ref{substfree}, \ref{substfund}によれば
この記号列は(\ref{sthm-basis1})と同じだから, (\ref{sthm-basis1})が成り立つ.
1), 2)が成り立つことは, (\ref{sthm-basis1})と
推論法則 \ref{dedwedge}, \ref{dedeqfund}によって明らかである.
\halmos




\mathstrut
\begin{thm}
\label{sthm-notin}%定理6.2%新規%確認済
$a$, $b$, $c$を集合とするとき, 
\begin{equation}
\label{sthm-notin1}
  c \notin a - b \leftrightarrow c \notin a \vee c \in b
\end{equation}
が成り立つ.
またこのことから, 次の1), 2)が成り立つ.

1)
$c \notin a - b$ならば, $c \notin a \vee c \in b$.

2)
$c \notin a$ならば, $c \notin a - b$.
また$c \in b$ならば, $c \notin a - b$.
\end{thm}


\noindent{\bf 証明}
~定理 \ref{sthm-basis}より, 
\[
  c \in a - b \leftrightarrow c \in a \wedge c \notin b, 
\]
即ち
\[
  c \in a - b \leftrightarrow \neg (c \notin a \vee \neg \neg (c \in b))
\]
が成り立つから, 推論法則 \ref{dedeqcp}により
\begin{equation}
\label{sthm-notin2}
  c \notin a - b \leftrightarrow c \notin a \vee \neg \neg (c \in b)
\end{equation}
が成り立つ.
またThm \ref{nnala}より
\[
  \neg \neg (c \in b) \leftrightarrow c \in b
\]
が成り立つから, 推論法則 \ref{dedaddeqv}により
\begin{equation}
\label{sthm-notin3}
  c \notin a \vee \neg \neg (c \in b) \leftrightarrow c \notin a \vee c \in b
\end{equation}
が成り立つ.
そこで(\ref{sthm-notin2}), (\ref{sthm-notin3})から, 
推論法則 \ref{dedeqtrans}によって(\ref{sthm-notin1})が成り立つ.
1), 2)が成り立つことは, (\ref{sthm-notin1})と
推論法則 \ref{dedvee}, \ref{dedeqfund}によって明らかである.
\halmos




\mathstrut
\begin{thm}
\label{sthma-bsubseta}%定理6.3%確認済
$a$と$b$を集合とするとき, 
\[
  a - b \subset a
\]
が成り立つ.
\end{thm}


\noindent{\bf 証明}
~$x$を$a$及び$b$の中に自由変数として現れない文字とするとき, 定理 \ref{sthmssetsubseta}より
\[
  \{x \in a \mid x \notin b\} \subset a
\]
が成り立つが, $a - b$の定義よりこの記号列は$a - b \subset a$と同じだから, これが成り立つ.
\halmos




\mathstrut
\begin{thm}
\label{sthma-bsubsetc}%定理6.4%新規%確認済
$a$, $b$, $c$を集合とするとき, 
\[
  a \subset c \to a - b \subset c, ~~
  c \subset a - b \to c \subset a
\]
が共に成り立つ.
またこれらから, 次の1), 2)が成り立つ.

1)
$a \subset c$ならば, $a - b \subset c$.

2)
$c \subset a - b$ならば, $c \subset a$.
\end{thm}


\noindent{\bf 証明}
~$x$を$a$及び$b$の中に自由変数として現れない文字とするとき, 定理 \ref{sthmssetsubsetb}より
\[
  a \subset c \to \{x \in a \mid x \notin b\} \subset c, ~~
  c \subset \{x \in a \mid x \notin b\} \to c \subset a
\]
が共に成り立つが, 定義よりこれらの記号列はそれぞれ
\[
  a \subset c \to a - b \subset c, ~~
  c \subset a - b \to c \subset a
\]
と同じだから, これらが共に成り立つ.
1), 2)が成り立つことは, これらと推論法則 \ref{dedmp}によって明らかである.
\halmos




\mathstrut
\begin{thm}
\label{sthma-bsubsetb}%定理6.5%新規%確認済
$a$と$b$を集合とするとき, 
\begin{equation}
\label{sthma-bsubsetb1}
  a - b \subset b \leftrightarrow a \subset b
\end{equation}
が成り立つ.
またこのことから特に, 次の(\ref{sthma-bsubsetb2})が成り立つ.
\begin{equation}
\label{sthma-bsubsetb2}
  a - b \subset b \text{ならば,} ~a \subset b.
\end{equation}
\end{thm}


\noindent{\bf 証明}
~$x$を$a$及び$b$の中に自由変数として現れない, 定数でない文字とする.
このとき定理 \ref{sthm-basis}より
\[
  x \in a - b \leftrightarrow x \in a \wedge x \notin b
\]
が成り立つから, 推論法則 \ref{dedaddeqt}により
\begin{equation}
\label{sthma-bsubsetb3}
  (x \in a - b \to x \in b) \leftrightarrow (x \in a \wedge x \notin b \to x \in b)
\end{equation}
が成り立つ.
またThm \ref{1at1btc11l1awbtc1}と推論法則 \ref{dedeqch}により
\begin{equation}
\label{sthma-bsubsetb4}
  (x \in a \wedge x \notin b \to x \in b) \leftrightarrow (x \in a \to (x \notin b \to x \in b))
\end{equation}
が成り立つ.
またThm \ref{nal1atna1}と推論法則 \ref{dedeqch}により
\[
  (x \notin b \to x \in b) \leftrightarrow x \in b
\]
が成り立つから, 推論法則 \ref{dedaddeqt}により
\begin{equation}
\label{sthma-bsubsetb5}
  (x \in a \to (x \notin b \to x \in b)) \leftrightarrow (x \in a \to x \in b)
\end{equation}
が成り立つ.
そこで(\ref{sthma-bsubsetb3})---(\ref{sthma-bsubsetb5})から, 推論法則 \ref{dedeqtrans}によって
\[
  (x \in a - b \to x \in b) \leftrightarrow (x \in a \to x \in b)
\]
が成り立つことがわかる.
このことと$x$が定数でないことから, 推論法則 \ref{dedalleqquansepconst}により
\[
  \forall x(x \in a - b \to x \in b) \leftrightarrow \forall x(x \in a \to x \in b)
\]
が成り立つ.
ここで$x$が$a$, $b$の中に自由変数として現れないことから, 
変数法則 \ref{val-}により, $x$は$a - b$の中にも自由変数として現れないから, 
上記の記号列は(\ref{sthma-bsubsetb1})と同じである.
故に(\ref{sthma-bsubsetb1})が成り立つ.
(\ref{sthma-bsubsetb2})が成り立つことは, 
(\ref{sthma-bsubsetb1})と推論法則 \ref{dedeqfund}によって明らかである.
\halmos




\mathstrut
\begin{thm}
\label{sthmasubsetbta-bsubsetc}%定理6.6%新規%確認済
$a$, $b$, $c$を集合とするとき, 
\begin{equation}
\label{sthmasubsetbta-bsubsetc1}
  a \subset b \to a - b \subset c
\end{equation}
が成り立つ.
またこのことから, 次の(\ref{sthmasubsetbta-bsubsetc2})が成り立つ.
\begin{equation}
\label{sthmasubsetbta-bsubsetc2}
  a \subset b \text{ならば,} ~a - b \subset c.
\end{equation}
\end{thm}


\noindent{\bf 証明}
~$x$を$a$, $b$, $c$の中に自由変数として現れない, 定数でない文字とする.
このとき変数法則 \ref{val-}により, $x$は$a - b$の中に自由変数として現れない.
またThm \ref{1atb1t1awctbwc1}より
\begin{equation}
\label{sthmasubsetbta-bsubsetc3}
  (x \in a \to x \in b) \to (x \in a \wedge x \notin b \to x \in b \wedge x \notin b)
\end{equation}
が成り立つ.
またThm \ref{n1awna1}より$\neg (x \in b \wedge x \notin b)$が成り立つから, 
推論法則 \ref{dednt}により
\[
  x \in b \wedge x \notin b \to x \in c
\]
が成り立つ.
故に推論法則 \ref{dedaddb}により
\begin{equation}
\label{sthmasubsetbta-bsubsetc4}
  (x \in a \wedge x \notin b \to x \in b \wedge x \notin b) \to (x \in a \wedge x \notin b \to x \in c)
\end{equation}
が成り立つ.
また定理 \ref{sthm-basis}と推論法則 \ref{dedequiv}により
\[
  x \in a - b \to x \in a \wedge x \notin b
\]
が成り立つから, 推論法則 \ref{dedaddf}により
\begin{equation}
\label{sthmasubsetbta-bsubsetc5}
  (x \in a \wedge x \notin b \to x \in c) \to (x \in a - b \to x \in c)
\end{equation}
が成り立つ.
そこで(\ref{sthmasubsetbta-bsubsetc3})---(\ref{sthmasubsetbta-bsubsetc5})から, 
推論法則 \ref{dedmmp}によって
\[
  (x \in a \to x \in b) \to (x \in a - b \to x \in c)
\]
が成り立つことがわかる.
このことと$x$が定数でないことから, 推論法則 \ref{dedalltquansepconst}により
\[
  \forall x(x \in a \to x \in b) \to \forall x(x \in a - b \to x \in c)
\]
が成り立つ.
上述のように$x$は$a$, $b$, $c$, $a - b$のいずれの中にも自由変数として現れないから, 
この記号列は(\ref{sthmasubsetbta-bsubsetc1})と同じである.
故に(\ref{sthmasubsetbta-bsubsetc1})が成り立つ.
(\ref{sthmasubsetbta-bsubsetc2})が成り立つことは, 
(\ref{sthmasubsetbta-bsubsetc1})と推論法則 \ref{dedmp}によって明らかである.
\halmos




\mathstrut
\begin{thm}
\label{sthma-bsubsetb-ceq}%定理6.7%新規%確認済
$a$, $b$, $c$を集合とするとき, 
\begin{align}
  \label{sthma-bsubsetb-ceq1}
  &a - b \subset b - c \leftrightarrow a \subset b, \\
  \mbox{} \notag \\
  \label{sthma-bsubsetb-ceq2}
  &a - b \subset c - a \leftrightarrow a \subset b, \\
  \mbox{} \notag \\
  \label{sthma-bsubsetb-ceq3}
  &a - b \subset b - c \leftrightarrow a - b \subset c - a
\end{align}
がすべて成り立つ.
またこれらから, 次の1), 2)が成り立つ.

1)
$a - b \subset b - c$ならば, 
$a \subset b$と$a - b \subset c - a$が共に成り立つ.

2)
$a - b \subset c - a$ならば, 
$a \subset b$と$a - b \subset b - c$が共に成り立つ.
\end{thm}


\noindent{\bf 証明}
~まず(\ref{sthma-bsubsetb-ceq1})が成り立つことを示す.
定理 \ref{sthma-bsubseta}より$b - c \subset b$が成り立つから, 推論法則 \ref{dedatawbtrue2}により
\begin{equation}
\label{sthma-bsubsetb-ceq4}
  a - b \subset b - c \to a - b \subset b - c \wedge b - c \subset b
\end{equation}
が成り立つ.
また定理 \ref{sthmsubsettrans}より
\begin{equation}
\label{sthma-bsubsetb-ceq5}
  a - b \subset b - c \wedge b - c \subset b \to a - b \subset b
\end{equation}
が成り立つ.
また定理 \ref{sthma-bsubsetb}と推論法則 \ref{dedequiv}により
\begin{equation}
\label{sthma-bsubsetb-ceq6}
  a - b \subset b \to a \subset b
\end{equation}
が成り立つ.
そこで(\ref{sthma-bsubsetb-ceq4})---(\ref{sthma-bsubsetb-ceq6})から, 推論法則 \ref{dedmmp}によって
\begin{equation}
\label{sthma-bsubsetb-ceq7}
  a - b \subset b - c \to a \subset b
\end{equation}
が成り立つことがわかる.
また定理 \ref{sthmasubsetbta-bsubsetc}より
\begin{equation}
\label{sthma-bsubsetb-ceq8}
  a \subset b \to a - b \subset b - c
\end{equation}
が成り立つ.
そこで(\ref{sthma-bsubsetb-ceq7}), (\ref{sthma-bsubsetb-ceq8})から, 
推論法則 \ref{dedequiv}により(\ref{sthma-bsubsetb-ceq1})が成り立つ.

次に(\ref{sthma-bsubsetb-ceq2})が成り立つことを示す.
$x$を$a$, $b$, $c$の中に自由変数として現れない, 定数でない文字とする.
このとき変数法則 \ref{val-}により, $x$は$a - b$, $c - a$の中に自由変数として現れない.
また定理 \ref{sthm-basis}より
\[
  x \in a - b \leftrightarrow x \in a \wedge x \notin b, ~~
  x \in c - a \leftrightarrow x \in c \wedge x \notin a
\]
が共に成り立つから, 推論法則 \ref{dedaddeqt}により
\[
  (x \in a - b \to x \in c - a) 
  \leftrightarrow (x \in a \wedge x \notin b \to x \in c \wedge x \notin a)
\]
が成り立つ.
故に推論法則 \ref{dedequiv}により
\begin{equation}
\label{sthma-bsubsetb-ceq9}
  (x \in a - b \to x \in c - a) \to (x \in a \wedge x \notin b \to x \in c \wedge x \notin a)
\end{equation}
が成り立つ.
またThm \ref{awbta}より
\[
  x \in c \wedge x \notin a \to x \notin a
\]
が成り立つから, 推論法則 \ref{dedaddb}により
\begin{equation}
\label{sthma-bsubsetb-ceq10}
  (x \in a \wedge x \notin b \to x \in c \wedge x \notin a) 
  \to (x \in a \wedge x \notin b \to x \notin a)
\end{equation}
が成り立つ.
またThm \ref{1awbtc1t1at1btc11}より
\begin{equation}
\label{sthma-bsubsetb-ceq11}
  (x \in a \wedge x \notin b \to x \notin a) \to (x \in a \to (x \notin b \to x \notin a))
\end{equation}
が成り立つ.
またschema S3の適用により
\[
  (x \notin b \to x \notin a) \to (x \in a \to x \in b)
\]
が成り立つから, 推論法則 \ref{dedaddb}により
\begin{equation}
\label{sthma-bsubsetb-ceq12}
  (x \in a \to (x \notin b \to x \notin a)) \to (x \in a \to (x \in a \to x \in b))
\end{equation}
が成り立つ.
またThm \ref{1at1atb11t1atb1}より
\begin{equation}
\label{sthma-bsubsetb-ceq13}
  (x \in a \to (x \in a \to x \in b)) \to (x \in a \to x \in b)
\end{equation}
が成り立つ.
そこで(\ref{sthma-bsubsetb-ceq9})---(\ref{sthma-bsubsetb-ceq13})から, 推論法則 \ref{dedmmp}によって
\[
  (x \in a - b \to x \in c - a) \to (x \in a \to x \in b)
\]
が成り立つことがわかる.
このことと$x$が定数でないことから, 推論法則 \ref{dedalltquansepconst}により
\[
  \forall x(x \in a - b \to x \in c - a) \to \forall x(x \in a \to x \in b)
\]
が成り立つ.
上述のように$x$は$a - b$, $c - a$, $a$, $b$のいずれの中にも自由変数として現れないから, 
この記号列は
\begin{equation}
\label{sthma-bsubsetb-ceq14}
  a - b \subset c - a \to a \subset b
\end{equation}
と同じである.
故にこれが成り立つ.
また定理 \ref{sthmasubsetbta-bsubsetc}より
\begin{equation}
\label{sthma-bsubsetb-ceq15}
  a \subset b \to a - b \subset c - a
\end{equation}
が成り立つ.
そこで(\ref{sthma-bsubsetb-ceq14}), (\ref{sthma-bsubsetb-ceq15})から, 
推論法則 \ref{dedequiv}により(\ref{sthma-bsubsetb-ceq2})が成り立つ.

最後に(\ref{sthma-bsubsetb-ceq3})が成り立つことを示す.
(\ref{sthma-bsubsetb-ceq2})から, 推論法則 \ref{dedeqch}により
\[
  a \subset b \leftrightarrow a - b \subset c - a
\]
が成り立つ.
そこでこれと(\ref{sthma-bsubsetb-ceq1})から, 
推論法則 \ref{dedeqtrans}によって(\ref{sthma-bsubsetb-ceq3})が成り立つ.

\noindent
1)
(\ref{sthma-bsubsetb-ceq1}), (\ref{sthma-bsubsetb-ceq3})と
推論法則 \ref{dedeqfund}によって明らかである.

\noindent
2)
(\ref{sthma-bsubsetb-ceq2}), (\ref{sthma-bsubsetb-ceq3})と
推論法則 \ref{dedeqfund}によって明らかである.
\halmos




\mathstrut
\begin{thm}
\label{sthma-bsubsetceq}%定理6.8%新規%確認済
$a$, $b$, $c$を集合とするとき, 
\begin{align}
  \label{sthma-bsubsetceq1}
  &a - b \subset c \leftrightarrow a - c \subset b, \\
  \mbox{} \notag \\
  \label{sthma-bsubsetceq2}
  &a - c \subset b - c \leftrightarrow a - c \subset b, \\
  \mbox{} \notag \\
  \label{sthma-bsubsetceq3}
  &a - c \subset b - c \leftrightarrow a - b \subset c, \\
  \mbox{} \notag \\
  \label{sthma-bsubsetceq4}
  &a - c \subset b - c \leftrightarrow a - b \subset c - b
\end{align}
がすべて成り立つ.
またこれらから, 次の1), 2)が成り立つ.

1)
$a - b \subset c$ならば, 
$a - c \subset b$, $a - c \subset b - c$, $a - b \subset c - b$がすべて成り立つ.

2)
$a - c \subset b - c$ならば, 
$a - c \subset b$, $a - b \subset c$, $a - b \subset c - b$がすべて成り立つ.
\end{thm}


\noindent{\bf 証明}
~$x$を$a$, $b$, $c$の中に自由変数として現れない, 定数でない文字とする.
このとき変数法則 \ref{val-}により, $x$は$a - b$, $a - c$, $b - c$の中に自由変数として現れない.

さてまず(\ref{sthma-bsubsetceq1})が成り立つことを示す.
定理 \ref{sthm-basis}より
\[
  x \in a - b \leftrightarrow x \in a \wedge x \notin b
\]
が成り立つから, 推論法則 \ref{dedaddeqt}により
\begin{equation}
\label{sthma-bsubsetceq5}
  (x \in a - b \to x \in c) \leftrightarrow (x \in a \wedge x \notin b \to x \in c)
\end{equation}
が成り立つ.
またThm \ref{1at1btc11l1awbtc1}と推論法則 \ref{dedeqch}により
\begin{equation}
\label{sthma-bsubsetceq6}
  (x \in a \wedge x \notin b \to x \in c) \leftrightarrow (x \in a \to (x \notin b \to x \in c))
\end{equation}
が成り立つ.
またThm \ref{1atb1l1nbtna1}より
\[
  (x \notin b \to x \in c) \leftrightarrow (x \notin c \to x \in b)
\]
が成り立つから, 推論法則 \ref{dedaddeqt}により
\begin{equation}
\label{sthma-bsubsetceq7}
  (x \in a \to (x \notin b \to x \in c)) \leftrightarrow (x \in a \to (x \notin c \to x \in b))
\end{equation}
が成り立つ.
またThm \ref{1at1btc11l1awbtc1}より
\begin{equation}
\label{sthma-bsubsetceq8}
  (x \in a \to (x \notin c \to x \in b)) \leftrightarrow (x \in a \wedge x \notin c \to x \in b)
\end{equation}
が成り立つ.
また定理 \ref{sthm-basis}と推論法則 \ref{dedeqch}により
\begin{equation}
\label{sthma-bsubsetceq9}
  x \in a \wedge x \notin c \leftrightarrow x \in a - c
\end{equation}
が成り立つから, 推論法則 \ref{dedaddeqt}により
\begin{equation}
\label{sthma-bsubsetceq10}
  (x \in a \wedge x \notin c \to x \in b) \leftrightarrow (x \in a - c \to x \in b)
\end{equation}
が成り立つ.
そこで(\ref{sthma-bsubsetceq5})---(\ref{sthma-bsubsetceq8}), (\ref{sthma-bsubsetceq10})から, 
推論法則 \ref{dedeqtrans}によって
\[
  (x \in a - b \to x \in c) \leftrightarrow (x \in a - c \to x \in b)
\]
が成り立つことがわかる.
このことと$x$が定数でないことから, 推論法則 \ref{dedalleqquansepconst}により
\[
  \forall x(x \in a - b \to x \in c) \leftrightarrow \forall x(x \in a - c \to x \in b)
\]
が成り立つ.
上述のように$x$は$a - b$, $c$, $a - c$, $b$のいずれの中にも自由変数として現れないから, 
この記号列は(\ref{sthma-bsubsetceq1})と同じである.
故に(\ref{sthma-bsubsetceq1})が成り立つ.

次に(\ref{sthma-bsubsetceq2})が成り立つことを示す.
定理 \ref{sthm-basis}より
\[
  x \in a - c \leftrightarrow x \in a \wedge x \notin c, ~~
  x \in b - c \leftrightarrow x \in b \wedge x \notin c
\]
が共に成り立つから, 推論法則 \ref{dedaddeqt}により
\begin{equation}
\label{sthma-bsubsetceq11}
  (x \in a - c \to x \in b - c) 
  \leftrightarrow (x \in a \wedge x \notin c \to x \in b \wedge x \notin c)
\end{equation}
が成り立つ.
またThm \ref{1awctb1l1awctbwc1}と推論法則 \ref{dedeqch}により
\begin{equation}
\label{sthma-bsubsetceq12}
  (x \in a \wedge x \notin c \to x \in b \wedge x \notin c) 
  \leftrightarrow (x \in a \wedge x \notin c \to x \in b)
\end{equation}
が成り立つ.
また(\ref{sthma-bsubsetceq9})が成り立つことから, 推論法則 \ref{dedaddeqt}により
\begin{equation}
\label{sthma-bsubsetceq13}
  (x \in a \wedge x \notin c \to x \in b) \leftrightarrow (x \in a - c \to x \in b)
\end{equation}
が成り立つ.
そこで(\ref{sthma-bsubsetceq11})---(\ref{sthma-bsubsetceq13})から, 推論法則 \ref{dedeqtrans}によって
\[
  (x \in a - c \to x \in b - c) \leftrightarrow (x \in a - c \to x \in b)
\]
が成り立つことがわかる.
このことと$x$が定数でないことから, 推論法則 \ref{dedalleqquansepconst}により
\[
  \forall x(x \in a - c \to x \in b - c) \leftrightarrow \forall x(x \in a - c \to x \in b)
\]
が成り立つ.
上述のように$x$は$a - c$, $b - c$, $b$のいずれの中にも自由変数として現れないから, 
この記号列は(\ref{sthma-bsubsetceq2})と同じである.
故に(\ref{sthma-bsubsetceq2})が成り立つ.

次に(\ref{sthma-bsubsetceq3})が成り立つことを示す.
(\ref{sthma-bsubsetceq1})と推論法則 \ref{dedeqch}により
\[
  a - c \subset b \leftrightarrow a - b \subset c
\]
が成り立つ.
そこでこれと(\ref{sthma-bsubsetceq2})から, 
推論法則 \ref{dedeqtrans}によって(\ref{sthma-bsubsetceq3})が成り立つ.

最後に(\ref{sthma-bsubsetceq4})が成り立つことを示す.
(\ref{sthma-bsubsetceq2})は任意の集合$a$, $b$, $c$に対して成り立つので, 特に$b$と$c$を入れ替えた
\[
  a - b \subset c - b \leftrightarrow a - b \subset c
\]
も成り立つ.
故に推論法則 \ref{dedeqch}により
\[
  a - b \subset c \leftrightarrow a - b \subset c - b
\]
が成り立つ.
そこでこれと(\ref{sthma-bsubsetceq3})から, 
推論法則 \ref{dedeqtrans}によって(\ref{sthma-bsubsetceq4})が成り立つ.

\noindent
1)
(\ref{sthma-bsubsetceq1}), (\ref{sthma-bsubsetceq3}), (\ref{sthma-bsubsetceq4})と
推論法則 \ref{dedeqfund}によって明らかである.

\noindent
2)
(\ref{sthma-bsubsetceq2}), (\ref{sthma-bsubsetceq3}), (\ref{sthma-bsubsetceq4})と
推論法則 \ref{dedeqfund}によって明らかである.
\halmos




\mathstrut
\begin{thm}
\label{sthmc-asubsetc-beq}%定理6.9%新規%確認済
$a$, $b$, $c$を集合とするとき, 
\begin{equation}
\label{sthmc-asubsetc-beq1}
  c - a \subset c - b \leftrightarrow b - a \subset b - c
\end{equation}
が成り立つ.
またこのことから, 次の(\ref{sthmc-asubsetc-beq2})が成り立つ.
\begin{equation}
\label{sthmc-asubsetc-beq2}
  c - a \subset c - b \text{ならば,} ~b - a \subset b - c.
\end{equation}
\end{thm}


\noindent{\bf 証明}
~$x$を$a$, $b$, $c$の中に自由変数として現れない, 定数でない文字とする.
このとき定理 \ref{sthm-basis}より
\[
  x \in c - a \leftrightarrow x \in c \wedge x \notin a, ~~
  x \in c - b \leftrightarrow x \in c \wedge x \notin b
\]
が共に成り立つから, 推論法則 \ref{dedaddeqt}により
\begin{equation}
\label{sthmc-asubsetc-beq3}
  (x \in c - a \to x \in c - b) 
  \leftrightarrow (x \in c \wedge x \notin a \to x \in c \wedge x \notin b)
\end{equation}
が成り立つ.
またThm \ref{1ct1atb11l1awctbwc1}と推論法則 \ref{dedeqch}により
\begin{equation}
\label{sthmc-asubsetc-beq4}
  (x \in c \wedge x \notin a \to x \in c \wedge x \notin b) 
  \leftrightarrow (x \in c \to (x \notin a \to x \notin b))
\end{equation}
が成り立つ.
またThm \ref{1atb1l1nbtna1}と推論法則 \ref{dedeqch}により
\[
  (x \notin a \to x \notin b) \leftrightarrow (x \in b \to x \in a)
\]
が成り立つから, 推論法則 \ref{dedaddeqt}により
\begin{equation}
\label{sthmc-asubsetc-beq5}
  (x \in c \to (x \notin a \to x \notin b)) \leftrightarrow (x \in c \to (x \in b \to x \in a))
\end{equation}
が成り立つ.
またThm \ref{1at1btc11l1bt1atc11}より
\begin{equation}
\label{sthmc-asubsetc-beq6}
  (x \in c \to (x \in b \to x \in a)) \leftrightarrow (x \in b \to (x \in c \to x \in a))
\end{equation}
が成り立つ.
またThm \ref{1atb1l1nbtna1}より
\[
  (x \in c \to x \in a) \leftrightarrow (x \notin a \to x \notin c)
\]
が成り立つから, 推論法則 \ref{dedaddeqt}により
\begin{equation}
\label{sthmc-asubsetc-beq7}
  (x \in b \to (x \in c \to x \in a)) \leftrightarrow (x \in b \to (x \notin a \to x \notin c))
\end{equation}
が成り立つ.
またThm \ref{1ct1atb11l1awctbwc1}より
\begin{equation}
\label{sthmc-asubsetc-beq8}
  (x \in b \to (x \notin a \to x \notin c)) 
  \leftrightarrow (x \in b \wedge x \notin a \to x \in b \wedge x \notin c)
\end{equation}
が成り立つ.
また定理 \ref{sthm-basis}と推論法則 \ref{dedeqch}により
\[
  x \in b \wedge x \notin a \leftrightarrow x \in b - a, ~~
  x \in b \wedge x \notin c \leftrightarrow x \in b - c
\]
が共に成り立つから, 推論法則 \ref{dedaddeqt}により
\begin{equation}
\label{sthmc-asubsetc-beq9}
  (x \in b \wedge x \notin a \to x \in b \wedge x \notin c) 
  \leftrightarrow (x \in b - a \to x \in b - c)
\end{equation}
が成り立つ.
そこで(\ref{sthmc-asubsetc-beq3})---(\ref{sthmc-asubsetc-beq9})から, 
推論法則 \ref{dedeqtrans}によって
\[
  (x \in c - a \to x \in c - b) \leftrightarrow (x \in b - a \to x \in b - c)
\]
が成り立つことがわかる.
このことと$x$が定数でないことから, 推論法則 \ref{dedalleqquansepconst}により
\[
  \forall x(x \in c - a \to x \in c - b) \leftrightarrow \forall x(x \in b - a \to x \in b - c)
\]
が成り立つ.
ここで$x$が$a$, $b$, $c$の中に自由変数として現れないことから, 
変数法則 \ref{val-}により$x$は$c - a$, $c - b$, $b - a$, $b - c$の中に自由変数として現れないから, 
この記号列は(\ref{sthmc-asubsetc-beq1})と同じである.
故に(\ref{sthmc-asubsetc-beq1})が成り立つ.
(\ref{sthmc-asubsetc-beq2})が成り立つことは, 
(\ref{sthmc-asubsetc-beq1})と推論法則 \ref{dedeqfund}によって明らかである.
\halmos




\mathstrut
\begin{thm}
\label{sthm-subset}%定理6.10%確認済
\mbox{}

1)
$a$, $b$, $c$を集合とするとき, 
\begin{align}
  \label{sthm-subset1}
  &a \subset b \to a - c \subset b - c, \\
  \mbox{} \notag \\
  \label{sthm-subset2}
  &a \subset b \to c - b \subset c - a
\end{align}
が共に成り立つ.
またこれらから, 次の(\ref{sthm-subset3})が成り立つ.
\begin{equation}
\label{sthm-subset3}
  a \subset b \text{ならば,} ~
  a - c \subset b - c \text{と} c - b \subset c - a \text{が共に成り立つ.}
\end{equation}

2)
$a$, $b$, $c$, $d$を集合とするとき, 
\begin{equation}
\label{sthm-subset4}
  a \subset b \wedge c \subset d \to a - d \subset b - c
\end{equation}
が成り立つ.
またこのことから, 次の(\ref{sthm-subset5})が成り立つ.
\begin{equation}
\label{sthm-subset5}
  a \subset b \text{と} c \subset d \text{が共に成り立てば,} ~a - d \subset b - c.
\end{equation}
\end{thm}


\noindent{\bf 証明}
~1)
$x$を$a$, $b$, $c$の中に自由変数として現れない, 定数でない文字とする.
このとき定理 \ref{sthmssetsubset}より
\[
  a \subset b \to \{x \in a \mid x \notin c\} \subset \{x \in b \mid x \notin c\}
\]
が成り立つが, 定義よりこの記号列は(\ref{sthm-subset1})と同じだから, (\ref{sthm-subset1})が成り立つ.
またThm \ref{1atb1t1nbtna1}より
\[
  (x \in a \to x \in b) \to (x \notin b \to x \notin a)
\]
が成り立つから, これと$x$が定数でないことから, 推論法則 \ref{dedalltquansepconst}により
\begin{equation}
\label{sthm-subset6}
  \forall x(x \in a \to x \in b) \to \forall x(x \notin b \to x \notin a)
\end{equation}
が成り立つ.
また$x$が$c$の中に自由変数として現れないことから, 定理 \ref{sthmalltssetsubset}より
\begin{equation}
\label{sthm-subset7}
  \forall x(x \notin b \to x \notin a) 
  \to \{x \in c \mid x \notin b\} \subset \{x \in c \mid x \notin a\}
\end{equation}
が成り立つ.
そこで(\ref{sthm-subset6}), (\ref{sthm-subset7})から, 推論法則 \ref{dedmmp}によって
\[
  \forall x(x \in a \to x \in b) \to \{x \in c \mid x \notin b\} \subset \{x \in c \mid x \notin a\}
\]
が成り立つ.
ここで$x$が$a$, $b$, $c$の中に自由変数として現れないことから, 
定義よりこの記号列は(\ref{sthm-subset2})と同じである.
故に(\ref{sthm-subset2})が成り立つ.
(\ref{sthm-subset3})が成り立つことは, 
(\ref{sthm-subset1}), (\ref{sthm-subset2})と推論法則 \ref{dedmp}によって明らかである.

\noindent
2)
1)より
\[
  a \subset b \to a - d \subset b - d, ~~
  c \subset d \to b - d \subset b - c
\]
が共に成り立つから, 推論法則 \ref{dedfromaddw}により
\begin{equation}
\label{sthm-subset8}
  a \subset b \wedge c \subset d \to a - d \subset b - d \wedge b - d \subset b - c
\end{equation}
が成り立つ.
また定理 \ref{sthmsubsettrans}より
\begin{equation}
\label{sthm-subset9}
  a - d \subset b - d \wedge b - d \subset b - c \to a - d \subset b - c
\end{equation}
が成り立つ.
そこで(\ref{sthm-subset8}), (\ref{sthm-subset9})から, 
推論法則 \ref{dedmmp}によって(\ref{sthm-subset4})が成り立つ.
(\ref{sthm-subset5})が成り立つことは, 
(\ref{sthm-subset4})と推論法則 \ref{dedmp}, \ref{dedwedge}によって明らかである.
\halmos




\mathstrut
\begin{thm}
\label{sthm-subseteq}%定理6.11%逆は言えない%確認済
$a$, $b$, $c$を集合とするとき, 
\begin{align}
  \label{sthm-subseteq1}
  &c \subset b \to (a \subset b \leftrightarrow a - c \subset b - c), \\
  \mbox{} \notag \\
  \label{sthm-subseteq2}
  &a \subset c \to (a \subset b \leftrightarrow c - b \subset c - a)
\end{align}
が共に成り立つ.
またこれらから, 次の1)---4)が成り立つ.

1)
$c \subset b$ならば, $a \subset b \leftrightarrow a - c \subset b - c$.

2)
$c \subset b$と$a - c \subset b - c$が共に成り立てば, $a \subset b$.

3)
$a \subset c$ならば, $a \subset b \leftrightarrow c - b \subset c - a$.

4)
$a \subset c$と$c - b \subset c - a$が共に成り立てば, $a \subset b$.
\end{thm}


\noindent{\bf 証明}
~定理 \ref{sthma-bsubsetceq}と推論法則 \ref{dedequiv}により
\[
  a - c \subset b - c \to a - b \subset c
\]
が成り立つから, 推論法則 \ref{dedaddw}により
\begin{equation}
\label{sthm-subseteq3}
  c \subset b \wedge a - c \subset b - c \to c \subset b \wedge a - b \subset c
\end{equation}
が成り立つ.
また定理 \ref{sthmasubsetbta-bsubsetc}より
\begin{equation}
\label{sthm-subseteq4}
  a \subset c \to a - c \subset b
\end{equation}
が成り立つ.
また定理 \ref{sthmc-asubsetc-beq}と推論法則 \ref{dedequiv}により
\begin{equation}
\label{sthm-subseteq5}
  c - b \subset c - a \to a - b \subset a - c
\end{equation}
が成り立つ.
そこで(\ref{sthm-subseteq4}), (\ref{sthm-subseteq5})から, 推論法則 \ref{dedfromaddw}により
\begin{equation}
\label{sthm-subseteq6}
  a \subset c \wedge c - b \subset c - a \to a - c \subset b \wedge a - b \subset a - c
\end{equation}
が成り立つ.
またThm \ref{awbtbwa}より
\begin{align}
  \label{sthm-subseteq7}
  &c \subset b \wedge a - b \subset c \to a - b \subset c \wedge c \subset b, \\
  \mbox{} \notag \\
  \label{sthm-subseteq8}
  &a - c \subset b \wedge a - b \subset a - c \to a - b \subset a - c \wedge a - c \subset b
\end{align}
が共に成り立つ.
また定理 \ref{sthmsubsettrans}より
\begin{align}
  \label{sthm-subseteq9}
  &a - b \subset c \wedge c \subset b \to a - b \subset b, \\
  \mbox{} \notag \\
  \label{sthm-subseteq10}
  &a - b \subset a - c \wedge a - c \subset b \to a - b \subset b
\end{align}
が共に成り立つ.
また定理 \ref{sthma-bsubsetb}と推論法則 \ref{dedequiv}により
\begin{equation}
\label{sthm-subseteq11}
  a - b \subset b \to a \subset b
\end{equation}
が成り立つ.
そこで(\ref{sthm-subseteq3}), (\ref{sthm-subseteq7}), 
(\ref{sthm-subseteq9}), (\ref{sthm-subseteq11})から, 推論法則 \ref{dedmmp}によって
\[
  c \subset b \wedge a - c \subset b - c \to a \subset b
\]
が成り立つことがわかる.
また(\ref{sthm-subseteq6}), (\ref{sthm-subseteq8}), 
(\ref{sthm-subseteq10}), (\ref{sthm-subseteq11})から, 同じく推論法則 \ref{dedmmp}によって
\[
  a \subset c \wedge c - b \subset c - a \to a \subset b
\]
が成り立つこともわかる.
故にこれらから, それぞれ推論法則 \ref{dedtwch}により
\begin{align}
  \label{sthm-subseteq12}
  &c \subset b \to (a - c \subset b - c \to a \subset b), \\
  \mbox{} \notag \\
  \label{sthm-subseteq13}
  &a \subset c \to (c - b \subset c - a \to a \subset b)
\end{align}
が成り立つ.
また定理 \ref{sthm-subset}より
\[
  a \subset b \to a - c \subset b - c, ~~
  a \subset b \to c - b \subset c - a
\]
が共に成り立つから, 推論法則 \ref{dedatawbtrue2}により
\begin{align}
  \label{sthm-subseteq14}
  &(a - c \subset b - c \to a \subset b) \to (a \subset b \leftrightarrow a - c \subset b - c), \\
  \mbox{} \notag \\
  \label{sthm-subseteq15}
  &(c - b \subset c - a \to a \subset b) \to (a \subset b \leftrightarrow c - b \subset c - a)
\end{align}
が共に成り立つ.
そこで(\ref{sthm-subseteq12})と(\ref{sthm-subseteq14}), 
(\ref{sthm-subseteq13})と(\ref{sthm-subseteq15})から, それぞれ推論法則 \ref{dedmmp}によって
(\ref{sthm-subseteq1}), (\ref{sthm-subseteq2})が成り立つ.

\noindent
1)
(\ref{sthm-subseteq1})と推論法則 \ref{dedmp}によって明らか.

\noindent
2)
1)と推論法則 \ref{dedeqfund}によって明らか.

\noindent
3)
(\ref{sthm-subseteq2})と推論法則 \ref{dedmp}によって明らか.

\noindent
4)
3)と推論法則 \ref{dedeqfund}によって明らか.
\halmos




\mathstrut
\begin{thm}
\label{sthm-subsetlast}%定理6.12%新規%確認済
$a$, $b$, $c$を集合とするとき, 
\begin{equation}
\label{sthm-subsetlast1}
  a \subset b \leftrightarrow a - c \subset b - c \wedge c - b \subset c - a
\end{equation}
が成り立つ.
またこのことから, 次の(\ref{sthm-subsetlast2})が成り立つ.
\begin{equation}
\label{sthm-subsetlast2}
  a - c \subset b - c \text{と} c - b \subset c - a \text{が共に成り立てば,} ~a \subset b.
\end{equation}
\end{thm}


\noindent{\bf 証明}
~定理 \ref{sthm-subset}より
\[
  a \subset b \to a - c \subset b - c, ~~
  a \subset b \to c - b \subset c - a
\]
が共に成り立つから, 推論法則 \ref{dedprewedge}により
\begin{equation}
\label{sthm-subsetlast3}
  a \subset b \to a - c \subset b - c \wedge c - b \subset c - a
\end{equation}
が成り立つ.
また定理 \ref{sthma-bsubsetceq}と推論法則 \ref{dedequiv}により
\begin{equation}
\label{sthm-subsetlast4}
  a - c \subset b - c \to a - c \subset b
\end{equation}
が成り立つ.
また定理 \ref{sthmc-asubsetc-beq}と推論法則 \ref{dedequiv}により
\begin{equation}
\label{sthm-subsetlast5}
  c - b \subset c - a \to a - b \subset a - c
\end{equation}
が成り立つ.
そこで(\ref{sthm-subsetlast4}), (\ref{sthm-subsetlast5})から, 推論法則 \ref{dedfromaddw}により
\begin{equation}
\label{sthm-subsetlast6}
  a - c \subset b - c \wedge c - b \subset c - a \to a - c \subset b \wedge a - b \subset a - c
\end{equation}
が成り立つ.
またThm \ref{awbtbwa}より
\begin{equation}
\label{sthm-subsetlast7}
  a - c \subset b \wedge a - b \subset a - c \to a - b \subset a - c \wedge a - c \subset b
\end{equation}
が成り立つ.
また定理 \ref{sthmsubsettrans}より
\begin{equation}
\label{sthm-subsetlast8}
  a - b \subset a - c \wedge a - c \subset b \to a - b \subset b
\end{equation}
が成り立つ.
また定理 \ref{sthma-bsubsetb}と推論法則 \ref{dedequiv}により
\begin{equation}
\label{sthm-subsetlast9}
  a - b \subset b \to a \subset b
\end{equation}
が成り立つ.
そこで(\ref{sthm-subsetlast6})---(\ref{sthm-subsetlast9})から, 推論法則 \ref{dedmmp}によって
\begin{equation}
\label{sthm-subsetlast10}
  a - c \subset b - c \wedge c - b \subset c - a \to a \subset b
\end{equation}
が成り立つことがわかる.
従って(\ref{sthm-subsetlast3}), (\ref{sthm-subsetlast10})から, 
推論法則 \ref{dedequiv}により(\ref{sthm-subsetlast1})が成り立つ.
(\ref{sthm-subsetlast2})が成り立つことは, 
(\ref{sthm-subsetlast1})と推論法則 \ref{dedwedge}, \ref{dedeqfund}によって明らかである.
\halmos




\mathstrut
\begin{thm}
\label{sthma-b=b-ceq}%定理6.13%新規%確認済
$a$, $b$, $c$を集合とするとき, 
\begin{align}
  \label{sthma-b=b-ceq1}
  &a - b = b - c \leftrightarrow a \subset b \wedge b \subset c, \\
  \mbox{} \notag \\
  \label{sthma-b=b-ceq2}
  &a - b = c - a \leftrightarrow a \subset b \wedge c \subset a
\end{align}
が共に成り立つ.
またこれらから, 次の1)---4)が成り立つ.

1)
$a - b = b - c$ならば, $a \subset b$と$b \subset c$が共に成り立つ.

2)
$a \subset b$と$b \subset c$が共に成り立てば, $a - b = b - c$.

3)
$a - b = c - a$ならば, $a \subset b$と$c \subset a$が共に成り立つ.

4)
$a \subset b$と$c \subset a$が共に成り立てば, $a - b = c - a$.
\end{thm}


\noindent{\bf 証明}
~定理 \ref{sthmaxiom1}と推論法則 \ref{dedeqch}により
\begin{align}
  \label{sthma-b=b-ceq3}
  &a - b = b - c \leftrightarrow a - b \subset b - c \wedge b - c \subset a - b, \\
  \mbox{} \notag \\
  \label{sthma-b=b-ceq4}
  &a - b = c - a \leftrightarrow a - b \subset c - a \wedge c - a \subset a - b
\end{align}
が共に成り立つ.
また定理 \ref{sthma-bsubsetb-ceq}より
\begin{align*}
  &a - b \subset b - c \leftrightarrow a \subset b, ~~
  b - c \subset a - b \leftrightarrow b \subset c, \\
  \mbox{} \\
  &a - b \subset c - a \leftrightarrow a \subset b, ~~
  c - a \subset a - b \leftrightarrow c \subset a
\end{align*}
がすべて成り立つから, このはじめの二つ, あとの二つから, それぞれ推論法則 \ref{dedaddeqw}により
\begin{align}
  \label{sthma-b=b-ceq5}
  &a - b \subset b - c \wedge b - c \subset a - b \leftrightarrow a \subset b \wedge b \subset c, \\
  \mbox{} \notag \\
  \label{sthma-b=b-ceq6}
  &a - b \subset c - a \wedge c - a \subset a - b \leftrightarrow a \subset b \wedge c \subset a
\end{align}
が成り立つ.
そこで(\ref{sthma-b=b-ceq3})と(\ref{sthma-b=b-ceq5}), 
(\ref{sthma-b=b-ceq4})と(\ref{sthma-b=b-ceq6})から, 
それぞれ推論法則 \ref{dedeqtrans}によって(\ref{sthma-b=b-ceq1}), (\ref{sthma-b=b-ceq2})が成り立つ.

\noindent
1), 2)
(\ref{sthma-b=b-ceq1})と推論法則 \ref{dedwedge}, \ref{dedeqfund}によって明らか.

\noindent
3), 4)
(\ref{sthma-b=b-ceq2})と推論法則 \ref{dedwedge}, \ref{dedeqfund}によって明らか.
\halmos




\mathstrut
\begin{thm}
\label{sthma-b=b-aeq}%定理6.14%新規%確認済
$a$と$b$を集合とするとき, 
\begin{equation}
\label{sthma-b=b-aeq1}
  a - b = b - a \leftrightarrow a = b
\end{equation}
が成り立つ.
またこのことから, 次の1), 2)が成り立つ.

1)
$a - b = b - a$ならば, $a = b$.

2)
$a = b$ならば, $a - b = b - a$.
\end{thm}


\noindent{\bf 証明}
~定理 \ref{sthma-b=b-ceq}より
\begin{equation}
\label{sthma-b=b-aeq2}
  a - b = b - a \leftrightarrow a \subset b \wedge b \subset a
\end{equation}
が成り立つ.
また定理 \ref{sthmaxiom1}より
\begin{equation}
\label{sthma-b=b-aeq3}
  a \subset b \wedge b \subset a \leftrightarrow a = b
\end{equation}
が成り立つ.
そこで(\ref{sthma-b=b-aeq2}), (\ref{sthma-b=b-aeq3})から, 
推論法則 \ref{dedeqtrans}によって(\ref{sthma-b=b-aeq1})が成り立つ.
1), 2)が成り立つことは, (\ref{sthma-b=b-aeq1})と推論法則 \ref{dedeqfund}によって明らかである.
\halmos




\mathstrut
\begin{thm}
\label{sthma-b=aeqb-a=b}%定理6.15%新規%確認済
$a$と$b$を集合とするとき, 
\begin{equation}
\label{sthma-b=aeqb-a=b1}
  a - b = a \leftrightarrow b - a = b
\end{equation}
が成り立つ.
またこのことから, 次の(\ref{sthma-b=aeqb-a=b2})が成り立つ.
\begin{equation}
\label{sthma-b=aeqb-a=b2}
  a - b = a \text{ならば,} ~b - a = b.
\end{equation}
\end{thm}


\noindent{\bf 証明}
~$x$を$a$及び$b$の中に自由変数として現れない, 定数でない文字とする.
このとき定理 \ref{sthmsset=a}と推論法則 \ref{dedeqch}により
\begin{equation}
\label{sthma-b=aeqb-a=b3}
  \{x \in a \mid x \notin b\} = a \leftrightarrow (\forall x \in a)(x \notin b)
\end{equation}
が成り立つ.
またThm \ref{thmspallfund}より
\begin{equation}
\label{sthma-b=aeqb-a=b4}
  (\forall x \in a)(x \notin b) \leftrightarrow \forall x(x \in a \to x \notin b)
\end{equation}
が成り立つ.
またThm \ref{1atb1l1nbtna1}より
\[
  (x \in a \to x \notin b) \leftrightarrow (x \in b \to x \notin a)
\]
が成り立つから, このことと$x$が定数でないことから, 推論法則 \ref{dedalleqquansepconst}により
\begin{equation}
\label{sthma-b=aeqb-a=b5}
  \forall x(x \in a \to x \notin b) \leftrightarrow \forall x(x \in b \to x \notin a)
\end{equation}
が成り立つ.
またThm \ref{thmspallfund}と推論法則 \ref{dedeqch}により
\begin{equation}
\label{sthma-b=aeqb-a=b6}
  \forall x(x \in b \to x \notin a) \leftrightarrow (\forall x \in b)(x \notin a)
\end{equation}
が成り立つ.
また$x$が$b$の中に自由変数として現れないことから, 定理 \ref{sthmsset=a}より
\begin{equation}
\label{sthma-b=aeqb-a=b7}
  (\forall x \in b)(x \notin a) \leftrightarrow \{x \in b \mid x \notin a\} = b
\end{equation}
が成り立つ.
そこで(\ref{sthma-b=aeqb-a=b3})---(\ref{sthma-b=aeqb-a=b7})から, 推論法則 \ref{dedeqtrans}によって
\[
  \{x \in a \mid x \notin b\} = a \leftrightarrow \{x \in b \mid x \notin a\} = b
\]
が成り立つことがわかる.
ここで$x$が$a$, $b$の中に自由変数として現れないことから, 
定義よりこの記号列は(\ref{sthma-b=aeqb-a=b1})と同じである.
故に(\ref{sthma-b=aeqb-a=b1})が成り立つ.
(\ref{sthma-b=aeqb-a=b2})が成り立つことは, 
(\ref{sthma-b=aeqb-a=b1})と推論法則 \ref{dedeqfund}によって明らかである.
\halmos




\mathstrut
\begin{thm}
\label{sthm-=}%定理6.16%確認済
\mbox{}

1)
$a$, $b$, $c$を集合とするとき, 
\begin{align}
  \label{sthm-=1}
  &a = b \to a - c = b - c, \\
  \mbox{} \notag \\
  \label{sthm-=2}
  &a = b \to c - a = c - b
\end{align}
が共に成り立つ.
またこれらから, 次の(\ref{sthm-=3})が成り立つ.
\begin{equation}
\label{sthm-=3}
  a = b \text{ならば,} ~a - c = b - c \text{と} c - a = c - b \text{が共に成り立つ.}
\end{equation}

2)
$a$, $b$, $c$, $d$を集合とするとき, 
\begin{equation}
\label{sthm-=4}
  a = b \wedge c = d \to a - c = b - d
\end{equation}
が成り立つ.
またこのことから, 次の(\ref{sthm-=5})が成り立つ.
\begin{equation}
\label{sthm-=5}
  a = b \text{と} c = d \text{が共に成り立てば,} ~a - c = b - d.
\end{equation}
\end{thm}


\noindent{\bf 証明}
~1)
$x$を$c$の中に自由変数として現れない文字とするとき, Thm \ref{T=Ut1TV=UV1}より
\[
  a = b \to (a|x)(x - c) = (b|x)(x - c), ~~
  a = b \to (a|x)(c - x) = (b|x)(c - x)
\]
が共に成り立つが, 代入法則 \ref{substfree}, \ref{subst-}によれば
これらはそれぞれ(\ref{sthm-=1}), (\ref{sthm-=2})と一致するから, これらが共に成り立つ.
(\ref{sthm-=3})が成り立つことは, 
(\ref{sthm-=1}), (\ref{sthm-=2})と推論法則 \ref{dedmp}によって明らかである.

\noindent
2)
1)より
\[
  a = b \to a - c = b - c, ~~
  c = d \to b - c = b - d
\]
が共に成り立つから, 推論法則 \ref{dedfromaddw}により
\begin{equation}
\label{sthm-=6}
  a = b \wedge c = d \to a - c = b - c \wedge b - c = b - d
\end{equation}
が成り立つ.
またThm \ref{x=ywy=ztx=z}より
\begin{equation}
\label{sthm-=7}
  a - c = b - c \wedge b - c = b - d \to a - c = b - d
\end{equation}
が成り立つ.
そこで(\ref{sthm-=6}), (\ref{sthm-=7})から, 推論法則 \ref{dedmmp}によって(\ref{sthm-=4})が成り立つ.
(\ref{sthm-=5})が成り立つことは, 
(\ref{sthm-=4})と推論法則 \ref{dedmp}, \ref{dedwedge}によって明らかである.
\halmos




\mathstrut
\begin{thm}
\label{sthm-=eq}%定理6.17%確認済
$a$, $b$, $c$を集合とするとき, 
\begin{align}
  \label{sthm-=eq1}
  &c \subset a \wedge c \subset b \to (a = b \leftrightarrow a - c = b - c), \\
  \mbox{} \notag \\
  \label{sthm-=eq2}
  &a \subset c \wedge b \subset c \to (a = b \leftrightarrow c - a = c - b)
\end{align}
が共に成り立つ.
またこれらから, 次の1)---4)が成り立つ.

1)
$c \subset a$と$c \subset b$が共に成り立てば, $a = b \leftrightarrow a - c = b - c$.

2)
$c \subset a$, $c \subset b$, $a - c = b - c$がすべて成り立てば, $a = b$.

3)
$a \subset c$と$b \subset c$が共に成り立てば, $a = b \leftrightarrow c - a = c - b$.

4)
$a \subset c$, $b \subset c$, $c - a = c - b$がすべて成り立てば, $a = b$.
\end{thm}


\noindent{\bf 証明}
~Thm \ref{awbtbwa}より
\begin{equation}
\label{sthm-=eq3}
  c \subset a \wedge c \subset b \to c \subset b \wedge c \subset a
\end{equation}
が成り立つ.
また定理 \ref{sthm-subseteq}より
\begin{align*}
  &c \subset b \to (a \subset b \leftrightarrow a - c \subset b - c), ~~
  c \subset a \to (b \subset a \leftrightarrow b - c \subset a - c), \\
  \mbox{} \notag \\
  &a \subset c \to (a \subset b \leftrightarrow c - b \subset c - a), ~~
  b \subset c \to (b \subset a \leftrightarrow c - a \subset c - b)
\end{align*}
がすべて成り立つ.
故にこのはじめの二つ, あとの二つから, それぞれ推論法則 \ref{dedfromaddw}により
\begin{align}
  \label{sthm-=eq4}
  &c \subset b \wedge c \subset a 
  \to (a \subset b \leftrightarrow a - c \subset b - c) \wedge (b \subset a \leftrightarrow b - c \subset a - c), \\
  \mbox{} \notag \\
  \label{sthm-=eq5}
  &a \subset c \wedge b \subset c 
  \to (a \subset b \leftrightarrow c - b \subset c - a) \wedge (b \subset a \leftrightarrow c - a \subset c - b)
\end{align}
が成り立つ.
またThm \ref{1alb1w1cld1t1awclbwd1}より
\begin{multline}
\label{sthm-=eq6}
  (a \subset b \leftrightarrow a - c \subset b - c) \wedge (b \subset a \leftrightarrow b - c \subset a - c) \\
  \to (a \subset b \wedge b \subset a \leftrightarrow a - c \subset b - c \wedge b - c \subset a - c), 
\end{multline}
\begin{multline}
\label{sthm-=eq7}
  (a \subset b \leftrightarrow c - b \subset c - a) \wedge (b \subset a \leftrightarrow c - a \subset c - b) \\
  \to (a \subset b \wedge b \subset a \leftrightarrow c - b \subset c - a \wedge c - a \subset c - b)
\end{multline}
が共に成り立つ.
また定理 \ref{sthmaxiom1}より
\begin{align}
  \label{sthm-=eq8}
  &a \subset b \wedge b \subset a \leftrightarrow a = b, \\
  \mbox{} \notag \\
  \label{sthm-=eq9}
  &a - c \subset b - c \wedge b - c \subset a - c \leftrightarrow a - c = b - c, \\
  \mbox{} \notag \\
  \label{sthm-=eq10}
  &c - b \subset c - a \wedge c - a \subset c - b \leftrightarrow c - b = c - a
\end{align}
がすべて成り立つ.
またThm \ref{x=yly=x}より
\begin{equation}
\label{sthm-=eq11}
  c - b = c - a \leftrightarrow c - a = c - b
\end{equation}
が成り立つ.
そこで(\ref{sthm-=eq10}), (\ref{sthm-=eq11})から, 推論法則 \ref{dedeqtrans}によって
\begin{equation}
\label{sthm-=eq12}
  c - b \subset c - a \wedge c - a \subset c - b \leftrightarrow c - a = c - b
\end{equation}
が成り立つ.
故に(\ref{sthm-=eq8})と(\ref{sthm-=eq9}), (\ref{sthm-=eq8})と(\ref{sthm-=eq12})から, 
それぞれ推論法則 \ref{dedaddeqeq}により
\begin{align*}
  &(a \subset b \wedge b \subset a \leftrightarrow a - c \subset b - c \wedge b - c \subset a - c) 
  \leftrightarrow (a = b \leftrightarrow a - c = b - c), \\
  \mbox{} \notag \\
  &(a \subset b \wedge b \subset a \leftrightarrow c - b \subset c - a \wedge c - a \subset c - b) 
  \leftrightarrow (a = b \leftrightarrow c - a = c - b)
\end{align*}
が成り立つ.
故に推論法則 \ref{dedequiv}により
\begin{align}
  \label{sthm-=eq13}
  &(a \subset b \wedge b \subset a \leftrightarrow a - c \subset b - c \wedge b - c \subset a - c) 
  \to (a = b \leftrightarrow a - c = b - c), \\
  \mbox{} \notag \\
  \label{sthm-=eq14}
  &(a \subset b \wedge b \subset a \leftrightarrow c - b \subset c - a \wedge c - a \subset c - b) 
  \to (a = b \leftrightarrow c - a = c - b)
\end{align}
が共に成り立つ.
以上の(\ref{sthm-=eq3}), (\ref{sthm-=eq4}), (\ref{sthm-=eq6}), (\ref{sthm-=eq13})から, 
推論法則 \ref{dedmmp}によって(\ref{sthm-=eq1})が成り立つことがわかる.
また(\ref{sthm-=eq5}), (\ref{sthm-=eq7}), (\ref{sthm-=eq14})から, 
同じく推論法則 \ref{dedmmp}によって(\ref{sthm-=eq2})が成り立つことがわかる.

\noindent
1)
(\ref{sthm-=eq1})と推論法則 \ref{dedmp}, \ref{dedwedge}によって明らか.

\noindent
2)
1)と推論法則 \ref{dedeqfund}によって明らか.

\noindent
3)
(\ref{sthm-=eq2})と推論法則 \ref{dedmp}, \ref{dedwedge}によって明らか.

\noindent
4)
3)と推論法則 \ref{dedeqfund}によって明らか.
\halmos




\mathstrut
\begin{thm}
\label{sthm-=last}%定理6.18%新規%確認済
$a$, $b$, $c$を集合とするとき, 
\begin{equation}
\label{sthm-=last1}
  a = b \leftrightarrow a - c = b - c \wedge c - a = c - b
\end{equation}
が成り立つ.
またこのことから, 次の(\ref{sthm-=last2})が成り立つ.
\begin{equation}
\label{sthm-=last2}
  a - c = b - c \text{と} c - a = c - b \text{が共に成り立てば,} ~a = b.
\end{equation}
\end{thm}


\noindent{\bf 証明}
~定理 \ref{sthm-=}より
\[
  a = b \to a - c = b - c, ~~
  a = b \to c - a = c - b
\]
が共に成り立つから, 推論法則 \ref{dedprewedge}により
\begin{equation}
\label{sthm-=last3}
  a = b \to a - c = b - c \wedge c - a = c - b
\end{equation}
が成り立つ.
また定理 \ref{sthm=tsubset}より
\begin{align*}
  &a - c = b - c \to a - c \subset b - c, ~~
  c - a = c - b \to c - b \subset c - a, \\
  \mbox{} \notag \\
  &a - c = b - c \to b - c \subset a - c, ~~
  c - a = c - b \to c - a \subset c - b
\end{align*}
がすべて成り立つから, このはじめの二つ, あとの二つから, それぞれ推論法則 \ref{dedfromaddw}により
\begin{align}
  \label{sthm-=last4}
  &a - c = b - c \wedge c - a = c - b \to a - c \subset b - c \wedge c - b \subset c - a, \\
  \mbox{} \notag \\
  \label{sthm-=last5}
  &a - c = b - c \wedge c - a = c - b \to b - c \subset a - c \wedge c - a \subset c - b
\end{align}
が成り立つ.
また定理 \ref{sthm-subsetlast}と推論法則 \ref{dedequiv}により
\begin{align}
  \label{sthm-=last6}
  &a - c \subset b - c \wedge c - b \subset c - a \to a \subset b, \\
  \mbox{} \notag \\
  \label{sthm-=last7}
  &b - c \subset a - c \wedge c - a \subset c - b \to b \subset a
\end{align}
が共に成り立つ.
そこで(\ref{sthm-=last4})と(\ref{sthm-=last6}), (\ref{sthm-=last5})と(\ref{sthm-=last7})から, 
それぞれ推論法則 \ref{dedmmp}によって
\begin{align*}
  &a - c = b - c \wedge c - a = c - b \to a \subset b, \\
  \mbox{} \notag \\
  &a - c = b - c \wedge c - a = c - b \to b \subset a
\end{align*}
が成り立つ.
故に推論法則 \ref{dedprewedge}により
\begin{equation}
\label{sthm-=last8}
  a - c = b - c \wedge c - a = c - b \to a \subset b \wedge b \subset a
\end{equation}
が成り立つ.
また定理 \ref{sthmaxiom1}と推論法則 \ref{dedequiv}により
\begin{equation}
\label{sthm-=last9}
  a \subset b \wedge b \subset a \to a = b
\end{equation}
が成り立つ.
そこで(\ref{sthm-=last8}), (\ref{sthm-=last9})から, 推論法則 \ref{dedmmp}によって
\begin{equation}
\label{sthm-=last10}
  a - c = b - c \wedge c - a = c - b \to a = b
\end{equation}
が成り立つ.
故に(\ref{sthm-=last3}), (\ref{sthm-=last10})から, 
推論法則 \ref{dedequiv}により(\ref{sthm-=last1})が成り立つ.
(\ref{sthm-=last2})が成り立つことは, 
(\ref{sthm-=last1})と推論法則 \ref{dedwedge}, \ref{dedeqfund}によって明らかである.
\halmos




\mathstrut
\begin{thm}
\label{sthmspin-}%定理6.19%新規%要る?一応述べておく%確認済
$a$と$b$を集合, $R$を関係式とし, $x$を$a$及び$b$の中に自由変数として現れない文字とする.
このとき
\begin{align}
  \label{sthmspin-1}
  &(\exists x \in a - b)(R) \leftrightarrow \{x \in a \mid R\} \not\subset b, \\
  \mbox{} \notag \\
  \label{sthmspin-2}
  &(\forall x \in a - b)(R) \leftrightarrow \{x \in a \mid \neg R\} \subset b
\end{align}
が共に成り立つ.
またこれらから, 次の1), 2)が成り立つ.

1)
$(\exists x \in a - b)(R)$ならば, $\{x \in a \mid R\} \not\subset b$.
また$\{x \in a \mid R\} \not\subset b$ならば, $(\exists x \in a - b)(R)$.

2)
$(\forall x \in a - b)(R)$ならば, $\{x \in a \mid \neg R\} \subset b$.
また$\{x \in a \mid \neg R\} \subset b$ならば, $(\forall x \in a - b)(R)$.
\end{thm}


\noindent{\bf 証明}
~まず(\ref{sthmspin-1})が成り立つことを示す.
$x$が$a$の中に自由変数として現れないことから, 定理 \ref{sthmspinsset}より
\[
  (\exists x \in \{x \in a \mid x \notin b\})(R) \leftrightarrow (\exists x \in a)(x \notin b \wedge R)
\]
が成り立つ.
ここで$x$は$b$の中にも自由変数として現れないから, 定義よりこの記号列は
\begin{equation}
\label{sthmspin-3}
  (\exists x \in a - b)(R) \leftrightarrow (\exists x \in a)(x \notin b \wedge R)
\end{equation}
と同じである.
故にこれが成り立つ.
またThm \ref{thmspquanwch}より
\begin{equation}
\label{sthmspin-4}
  (\exists x \in a)(x \notin b \wedge R) \leftrightarrow (\exists x \in a)(R \wedge x \notin b)
\end{equation}
が成り立つ.
また$x$が$a$の中に自由変数として現れないことから, 
定理 \ref{sthmspinsset}と推論法則 \ref{dedeqch}により
\begin{equation}
\label{sthmspin-5}
  (\exists x \in a)(R \wedge x \notin b) 
  \leftrightarrow (\exists x \in \{x \in a \mid R\})(x \notin b)
\end{equation}
が成り立つ.
また変数法則 \ref{valsset}により, $x$は$\{x \in a \mid R\}$の中に自由変数として現れないから, 
このことと$x$が$b$の中に自由変数として現れないことから, 
定理 \ref{sthmnotsubset}と推論法則 \ref{dedeqch}により, 
\[
  \exists x(x \in \{x \in a \mid R\} \wedge x \notin b) 
  \leftrightarrow \{x \in a \mid R\} \not\subset b, 
\]
即ち
\begin{equation}
\label{sthmspin-6}
  (\exists x \in \{x \in a \mid R\})(x \notin b) \leftrightarrow \{x \in a \mid R\} \not\subset b
\end{equation}
が成り立つ.
そこで(\ref{sthmspin-3})---(\ref{sthmspin-6})から, 
推論法則 \ref{dedeqtrans}によって(\ref{sthmspin-1})が成り立つことがわかる.

次に(\ref{sthmspin-2})が成り立つことを示す.
(\ref{sthmspin-1})は任意の関係式$R$に対して成り立つので, $R$を$\neg R$に置き換えた
\[
  (\exists x \in a - b)(\neg R) \leftrightarrow \{x \in a \mid \neg R\} \not\subset b
\]
も成り立つ.
故に推論法則 \ref{dedeqcp}により, 
\[
  \neg (\exists x \in a - b)(\neg R) \leftrightarrow \{x \in a \mid \neg R\} \subset b, 
\]
即ち(\ref{sthmspin-2})が成り立つ.

\noindent
1)
(\ref{sthmspin-1})と推論法則 \ref{dedeqfund}によって明らか.

\noindent
2)
(\ref{sthmspin-2})と推論法則 \ref{dedeqfund}によって明らか.
\halmos




\mathstrut
\begin{thm}
\label{sthmab--}%定理6.20%新規%確認済
$a$と$b$を集合とするとき, 
\begin{align}
  \label{sthmab--1}
  &(a - b) - b = a - b, \\
  \mbox{} \notag \\
  \label{sthmab--2}
  &a - (b - a) = a
\end{align}
が共に成り立つ.
\end{thm}


\noindent{\bf 証明}
~$x$を$a$及び$b$の中に自由変数として現れない, 定数でない文字とする.
このとき変数法則 \ref{val-}により, $x$は$a - b$, $b - a$の中に自由変数として現れない.

さてまず(\ref{sthmab--1})が成り立つことを示す.
定理 \ref{sthm-basis}と推論法則 \ref{dedequiv}により
\[
  x \in a - b \to x \in a \wedge x \notin b
\]
が成り立つから, 推論法則 \ref{dedprewedge}により
\[
  x \in a - b \to x \notin b
\]
が成り立つ.
このことと, $x$が定数でなく, 上述のように$a - b$の中に自由変数として現れないことから, 
定理 \ref{sthmsset=a}より
\[
  \{x \in a - b \mid x \notin b\} = a - b
\]
が成り立つ.
ここで$x$は$b$の中にも自由変数として現れないから, 定義よりこの記号列は(\ref{sthmab--1})と同じである.
故に(\ref{sthmab--1})が成り立つ.

次に(\ref{sthmab--2})が成り立つことを示す.
定理 \ref{sthm-notin}と推論法則 \ref{dedequiv}により
\[
  x \notin b \vee x \in a \to x \notin b - a
\]
が成り立つから, 推論法則 \ref{deddil}により
\[
  x \in a \to x \notin b - a
\]
が成り立つ.
このことと, $x$が定数でなく, $a$の中に自由変数として現れないことから, 定理 \ref{sthmsset=a}より
\[
  \{x \in a \mid x \notin b - a\} = a
\]
が成り立つ.
ここで上述のように$x$は$b - a$の中にも自由変数として現れないから, 
定義よりこの記号列は(\ref{sthmab--2})と同じである.
故に(\ref{sthmab--2})が成り立つ.
\halmos




\mathstrut
\begin{thm}
\label{sthmab--eqsubset}%定理6.21%新規%確認済
$a$と$b$を集合とするとき, 
\begin{align}
  \label{sthmab--eqsubset1}
  &a - (a - b) = a \leftrightarrow a \subset b, \\
  \mbox{} \notag \\
  \label{sthmab--eqsubset2}
  &a - (a - b) = b \leftrightarrow b \subset a
\end{align}
が共に成り立つ.
またこれらから, 次の1)---4)が成り立つ.

1)
$a - (a - b) = a$ならば, $a \subset b$.

2)
$a \subset b$ならば, $a - (a - b) = a$.

3)
$a - (a - b) = b$ならば, $b \subset a$.

4)
$b \subset a$ならば, $a - (a - b) = b$.
\end{thm}


\noindent{\bf 証明}
~$x$を$a$及び$b$の中に自由変数として現れない, 定数でない文字とする.
このとき変数法則 \ref{val-}により, $x$は$a - b$の中に自由変数として現れない.
さて$x$が$a$の中に自由変数として現れないことから, 
定理 \ref{sthmssetsm}より$x \in a \wedge x \notin a - b$は$x$について集合を作り得るから, 
推論法則 \ref{dedeqch}, \ref{dedawblatrue2}により
\begin{align}
  \label{sthmab--eqsubset3}
  &a - (a - b) = a \leftrightarrow {\rm Set}_{x}(x \in a \wedge x \notin a - b) \wedge a - (a - b) = a, \\
  \mbox{} \notag \\
  \label{sthmab--eqsubset4}
  &a - (a - b) = b \leftrightarrow {\rm Set}_{x}(x \in a \wedge x \notin a - b) \wedge a - (a - b) = b
\end{align}
が共に成り立つ.
また$x$が$a$, $b$の中に自由変数として現れないことから, 
定理 \ref{sthmsmbasis&iset=a}と推論法則 \ref{dedeqch}により
\begin{align*}
  &{\rm Set}_{x}(x \in a \wedge x \notin a - b) \wedge \{x \mid x \in a \wedge x \notin a - b\} = a 
  \leftrightarrow \forall x(x \in a \leftrightarrow x \in a \wedge x \notin a - b), \\
  \mbox{} \notag \\
  &{\rm Set}_{x}(x \in a \wedge x \notin a - b) \wedge \{x \mid x \in a \wedge x \notin a - b\} = b 
  \leftrightarrow \forall x(x \in b \leftrightarrow x \in a \wedge x \notin a - b)
\end{align*}
が共に成り立つ.
ここで$x$が$a$の中に自由変数として現れず, 上述のように$a - b$の中にも自由変数として現れないことから, 
定義よりこれらの記号列はそれぞれ
\begin{align}
  \label{sthmab--eqsubset5}
  &{\rm Set}_{x}(x \in a \wedge x \notin a - b) \wedge a - (a - b) = a 
  \leftrightarrow \forall x(x \in a \leftrightarrow x \in a \wedge x \notin a - b), \\
  \mbox{} \notag \\
  \label{sthmab--eqsubset6}
  &{\rm Set}_{x}(x \in a \wedge x \notin a - b) \wedge a - (a - b) = b 
  \leftrightarrow \forall x(x \in b \leftrightarrow x \in a \wedge x \notin a - b)
\end{align}
と一致する.
故にこれらが共に成り立つ.
また定理 \ref{sthm-notin}より
\[
  x \notin a - b \leftrightarrow x \notin a \vee x \in b
\]
が成り立つから, 推論法則 \ref{dedaddeqw}により
\begin{equation}
\label{sthmab--eqsubset7}
  x \in a \wedge x \notin a - b \leftrightarrow x \in a \wedge (x \notin a \vee x \in b)
\end{equation}
が成り立つ.
またThm \ref{aw1bvc1l1awb1v1awc1}より
\begin{equation}
\label{sthmab--eqsubset8}
  x \in a \wedge (x \notin a \vee x \in b) 
  \leftrightarrow (x \in a \wedge x \notin a) \vee (x \in a \wedge x \in b)
\end{equation}
が成り立つ.
またThm \ref{n1awna1}より$\neg (x \in a \wedge x \notin a)$が成り立つから, 
推論法則 \ref{dedavblbtrue2}により
\begin{equation}
\label{sthmab--eqsubset9}
  (x \in a \wedge x \notin a) \vee (x \in a \wedge x \in b) \leftrightarrow x \in a \wedge x \in b
\end{equation}
が成り立つ.
そこで(\ref{sthmab--eqsubset7})---(\ref{sthmab--eqsubset9})から, 推論法則 \ref{dedeqtrans}によって
\[
  x \in a \wedge x \notin a - b \leftrightarrow x \in a \wedge x \in b
\]
が成り立つことがわかる.
故に推論法則 \ref{dedaddeqeq}により
\begin{align}
  \label{sthmab--eqsubset10}
  &(x \in a \leftrightarrow x \in a \wedge x \notin a - b) 
  \leftrightarrow (x \in a \leftrightarrow x \in a \wedge x \in b), \\
  \mbox{} \notag \\
  \label{sthmab--eqsubset11}
  &(x \in b \leftrightarrow x \in a \wedge x \notin a - b) 
  \leftrightarrow (x \in b \leftrightarrow x \in a \wedge x \in b)
\end{align}
が共に成り立つ.
またThm \ref{1alb1l1bla1}より
\begin{align}
  \label{sthmab--eqsubset12}
  &(x \in a \leftrightarrow x \in a \wedge x \in b) 
  \leftrightarrow (x \in a \wedge x \in b \leftrightarrow x \in a), \\
  \mbox{} \notag \\
  \label{sthmab--eqsubset13}
  &(x \in b \leftrightarrow x \in a \wedge x \in b) 
  \leftrightarrow (x \in a \wedge x \in b \leftrightarrow x \in b)
\end{align}
が共に成り立つ.
またThm \ref{1atb1l1awbla1}と推論法則 \ref{dedeqch}により
\begin{align}
  \label{sthmab--eqsubset14}
  &(x \in a \wedge x \in b \leftrightarrow x \in a) \leftrightarrow (x \in a \to x \in b), \\
  \mbox{} \notag \\
  \label{sthmab--eqsubset15}
  &(x \in a \wedge x \in b \leftrightarrow x \in b) \leftrightarrow (x \in b \to x \in a)
\end{align}
が共に成り立つ.
そこで(\ref{sthmab--eqsubset10}), (\ref{sthmab--eqsubset12}), (\ref{sthmab--eqsubset14})から, 
推論法則 \ref{dedeqtrans}によって
\[
  (x \in a \leftrightarrow x \in a \wedge x \notin a - b) \leftrightarrow (x \in a \to x \in b)
\]
が成り立つことがわかる.
また(\ref{sthmab--eqsubset11}), (\ref{sthmab--eqsubset13}), (\ref{sthmab--eqsubset15})から, 
同じく推論法則 \ref{dedeqtrans}によって
\[
  (x \in b \leftrightarrow x \in a \wedge x \notin a - b) \leftrightarrow (x \in b \to x \in a)
\]
が成り立つことがわかる.
そこでこれらのことと, $x$が定数でないことから, 推論法則 \ref{dedalleqquansepconst}により
\begin{align*}
  &\forall x(x \in a \leftrightarrow x \in a \wedge x \notin a - b) 
  \leftrightarrow \forall x(x \in a \to x \in b), \\
  \mbox{} \notag \\
  &\forall x(x \in b \leftrightarrow x \in a \wedge x \notin a - b) 
  \leftrightarrow \forall x(x \in b \to x \in a)
\end{align*}
が共に成り立つ.
ここで$x$が$a$, $b$の中に自由変数として現れないことから, 定義よりこれらの記号列はそれぞれ
\begin{align}
  \label{sthmab--eqsubset16}
  &\forall x(x \in a \leftrightarrow x \in a \wedge x \notin a - b) \leftrightarrow a \subset b, \\
  \mbox{} \notag \\
  \label{sthmab--eqsubset17}
  &\forall x(x \in b \leftrightarrow x \in a \wedge x \notin a - b) \leftrightarrow b \subset a
\end{align}
と一致する.
故にこれらが共に成り立つ.
そこで(\ref{sthmab--eqsubset3}), (\ref{sthmab--eqsubset5}), (\ref{sthmab--eqsubset16})から, 
推論法則 \ref{dedeqtrans}によって(\ref{sthmab--eqsubset1})が成り立つことがわかる.
また(\ref{sthmab--eqsubset4}), (\ref{sthmab--eqsubset6}), (\ref{sthmab--eqsubset17})から, 
同じく推論法則 \ref{dedeqtrans}によって(\ref{sthmab--eqsubset2})が成り立つことがわかる.

\noindent
1), 2)
(\ref{sthmab--eqsubset1})と推論法則 \ref{dedeqfund}によって明らか.

\noindent
3), 4)
(\ref{sthmab--eqsubset2})と推論法則 \ref{dedeqfund}によって明らか.
\halmos




\mathstrut
\begin{thm}
\label{sthm-ch}%定理6.22%新規%確認済
$a$, $b$, $c$を集合とするとき, 
\begin{equation}
\label{sthm-ch1}
  (a - b) - c = (a - c) - b
\end{equation}
が成り立つ.
\end{thm}


\noindent{\bf 証明}
~$x$を$a$, $b$, $c$の中に自由変数として現れない文字とする.
このとき変数法則 \ref{val-}により, $x$は$a - b$, $a - c$の中に自由変数として現れない.
さて$x$が$a$の中に自由変数として現れないことから, 定理 \ref{sthmssetsset}より
\[
  \{x \in \{x \in a \mid x \notin b\} \mid x \notin c\} 
  = \{x \in a \mid x \notin c \wedge x \notin b\}
\]
が成り立ち, 定理 \ref{sthmssetsset}と推論法則 \ref{ded=ch}により
\[
  \{x \in a \mid x \notin c \wedge x \notin b\} 
  = \{x \in \{x \in a \mid x \notin c\} \mid x \notin b\}
\]
が成り立つ.
故にこれらから, 推論法則 \ref{ded=trans}によって
\[
  \{x \in \{x \in a \mid x \notin b\} \mid x \notin c\} 
  = \{x \in \{x \in a \mid x \notin c\} \mid x \notin b\}
\]
が成り立つ.
ここで$x$が$a$, $b$, $c$の中に自由変数として現れないことから, 定義よりこの記号列は
\[
  \{x \in a - b \mid x \notin c\} = \{x \in a - c \mid x \notin b\}
\]
と同じである.
また上述のように$x$は$a - b$, $a - c$の中にも自由変数として現れないから, 
定義よりこの記号列は(\ref{sthm-ch1})と同じである.
故に(\ref{sthm-ch1})が成り立つ.
\halmos




\mathstrut
\begin{thm}
\label{sthma-iset}%定理6.23%新規%確認済
$a$を集合, $R$を関係式とし, $x$を$a$の中に自由変数として現れない文字とする.
このとき
\begin{equation}
\label{sthma-iset1}
  {\rm Set}_{x}(R) \to a - \{x \mid R\} = \{x \in a \mid \neg R\}
\end{equation}
が成り立つ.
またこのことから, 次の(\ref{sthma-iset2})が成り立つ.
\begin{equation}
\label{sthma-iset2}
  R \text{が} x \text{について集合を作り得るならば,} ~a - \{x \mid R\} = \{x \in a \mid \neg R\}.
\end{equation}
\end{thm}


\noindent{\bf 証明}
~$y$を$x$と異なり, $R$の中に自由変数として現れない, 定数でない文字とする.
このときThm \ref{1alb1l1nalnb1}と推論法則 \ref{dedequiv}により
\[
  (y \in \{x \mid R\} \leftrightarrow (y|x)(R)) 
  \to (y \notin \{x \mid R\} \leftrightarrow \neg (y|x)(R))
\]
が成り立つ.
ここで変数法則 \ref{valiset}により, $x$は$\{x \mid R\}$の中に自由変数として現れないから, 
代入法則 \ref{substfree}, \ref{substfund}, \ref{substequiv}によれば, この記号列は
\[
  (y|x)(x \in \{x \mid R\} \leftrightarrow R) \to (y|x)(x \notin \{x \mid R\} \leftrightarrow \neg R)
\]
と一致する.
故にこれが成り立つ.
このことと$y$が定数でないことから, 推論法則 \ref{dedalltquansepconst}により
\[
  \forall y((y|x)(x \in \{x \mid R\} \leftrightarrow R)) 
  \to \forall y((y|x)(x \notin \{x \mid R\} \leftrightarrow \neg R))
\]
が成り立つ.
ここで$y$が$x$と異なり, $R$の中に自由変数として現れないことから, 
変数法則 \ref{valfund}, \ref{valequiv}, \ref{valiset}により, 
$y$は$x \in \{x \mid R\} \leftrightarrow R$, 
$x \notin \{x \mid R\} \leftrightarrow \neg R$の中に自由変数として現れない.
故に代入法則 \ref{substquantrans}によれば, 上記の記号列は
\[
  \forall x(x \in \{x \mid R\} \leftrightarrow R) 
  \to \forall x(x \notin \{x \mid R\} \leftrightarrow \neg R)
\]
と一致する.
また定義からこの記号列は
\begin{equation}
\label{sthma-iset3}
  {\rm Set}_{x}(R) \to \forall x(x \notin \{x \mid R\} \leftrightarrow \neg R)
\end{equation}
と同じである.
従ってこれが成り立つ.
また定理 \ref{sthmalleqsset=}より
\[
  \forall x(x \notin \{x \mid R\} \leftrightarrow \neg R) 
  \to \{x \in a \mid x \notin \{x \mid R\}\} = \{x \in a \mid \neg R\}
\]
が成り立つ.
ここで$x$は$a$の中に自由変数として現れず, 
上述のように$\{x \mid R\}$の中にも自由変数として現れないから, 定義よりこの記号列は
\begin{equation}
\label{sthma-iset4}
  \forall x(x \notin \{x \mid R\} \leftrightarrow \neg R) 
  \to a - \{x \mid R\} = \{x \in a \mid \neg R\}
\end{equation}
と同じである.
故にこれが成り立つ.
そこで(\ref{sthma-iset3}), (\ref{sthma-iset4})から, 推論法則 \ref{dedmmp}によって
(\ref{sthma-iset1})が成り立つ.
(\ref{sthma-iset2})が成り立つことは, (\ref{sthma-iset1})と推論法則 \ref{dedmp}によって明らかである.
\halmos




\mathstrut
\begin{thm}
\label{sthmiset-}%定理6.24%新規%確認済
$R$と$S$を関係式とし, $x$を文字とする.
このとき
\begin{equation}
\label{sthmiset-1}
  {\rm Set}_{x}(R) \wedge {\rm Set}_{x}(S) \to \{x \mid R\} - \{x \mid S\} = \{x \mid R \wedge \neg S\}
\end{equation}
が成り立つ.
またこのことから, 次の(\ref{sthmiset-2})が成り立つ.
\begin{equation}
\label{sthmiset-2}
  R \text{と} S \text{が共に} x \text{について集合を作り得るならば,} ~
  \{x \mid R\} - \{x \mid S\} = \{x \mid R \wedge \neg S\}.
\end{equation}
\end{thm}


\noindent{\bf 証明}
~Thm \ref{awbtbwa}より
\begin{equation}
\label{sthmiset-3}
  {\rm Set}_{x}(R) \wedge {\rm Set}_{x}(S) \to {\rm Set}_{x}(S) \wedge {\rm Set}_{x}(R)
\end{equation}
が成り立つ.
また変数法則 \ref{valiset}により, $x$は$\{x \mid R\}$の中に自由変数として現れないから, 
定理 \ref{sthma-iset}より
\begin{equation}
\label{sthmiset-4}
  {\rm Set}_{x}(S) \to \{x \mid R\} - \{x \mid S\} = \{x \in \{x \mid R\} \mid \neg S\}
\end{equation}
が成り立つ.
また定理 \ref{sthmisetsset}より
\begin{equation}
\label{sthmiset-5}
  {\rm Set}_{x}(R) \to \{x \in \{x \mid R\} \mid \neg S\} = \{x \mid R \wedge \neg S\}
\end{equation}
が成り立つ.
そこで(\ref{sthmiset-4}), (\ref{sthmiset-5})から, 推論法則 \ref{dedfromaddw}により
\begin{multline}
\label{sthmiset-6}
  {\rm Set}_{x}(S) \wedge {\rm Set}_{x}(R) \\
  \to \{x \mid R\} - \{x \mid S\} = \{x \in \{x \mid R\} \mid \neg S\} 
  \wedge \{x \in \{x \mid R\} \mid \neg S\} = \{x \mid R \wedge \neg S\}
\end{multline}
が成り立つ.
またThm \ref{x=ywy=ztx=z}より
\begin{multline}
\label{sthmiset-7}
  \{x \mid R\} - \{x \mid S\} = \{x \in \{x \mid R\} \mid \neg S\} 
  \wedge \{x \in \{x \mid R\} \mid \neg S\} = \{x \mid R \wedge \neg S\} \\
  \to \{x \mid R\} - \{x \mid S\} = \{x \mid R \wedge \neg S\}
\end{multline}
が成り立つ.
そこで(\ref{sthmiset-3}), (\ref{sthmiset-6}), (\ref{sthmiset-7})から, 
推論法則 \ref{dedmmp}によって(\ref{sthmiset-1})が成り立つ.
(\ref{sthmiset-2})が成り立つことは, 
(\ref{sthmiset-1})と推論法則 \ref{dedmp}, \ref{dedwedge}によって明らかである.
\halmos




\mathstrut
\begin{thm}
\label{sthma-sset}%定理6.25%新規%確認済
$a$を集合, $R$を関係式とし, $x$を$a$の中に自由変数として現れない文字とする.
このとき
\begin{equation}
\label{sthma-sset1}
  a - \{x \in a \mid R\} = \{x \in a \mid \neg R\}
\end{equation}
が成り立つ.
\end{thm}


\noindent{\bf 証明}
~$y$を$a$, $R$の中に自由変数として現れない, 定数でない文字とする.
このとき変数法則 \ref{valsset}により, $y$は$\{x \in a \mid R\}$の中に自由変数として現れない.
また$x$が$a$の中に自由変数として現れないことから, 定理 \ref{sthmssetbasis}より
\[
  y \in \{x \in a \mid R\} \leftrightarrow y \in a \wedge (y|x)(R)
\]
が成り立つから, 推論法則 \ref{dedeqcp}により
\begin{equation}
\label{sthma-sset2}
  y \notin \{x \in a \mid R\} \leftrightarrow \neg (y \in a \wedge (y|x)(R))
\end{equation}
が成り立つ.
またThm \ref{n1awb1lnavnb}より
\begin{equation}
\label{sthma-sset3}
  \neg (y \in a \wedge (y|x)(R)) \leftrightarrow y \notin a \vee \neg (y|x)(R)
\end{equation}
が成り立つ.
そこで(\ref{sthma-sset2}), (\ref{sthma-sset3})から, 推論法則 \ref{dedeqtrans}によって
\[
  y \notin \{x \in a \mid R\} \leftrightarrow y \notin a \vee \neg (y|x)(R)
\]
が成り立つ.
故に推論法則 \ref{dedaddeqw}により
\begin{equation}
\label{sthma-sset4}
  y \in a \wedge y \notin \{x \in a \mid R\} 
  \leftrightarrow y \in a \wedge (y \notin a \vee \neg (y|x)(R))
\end{equation}
が成り立つ.
またThm \ref{aw1bvc1l1awb1v1awc1}より
\begin{equation}
\label{sthma-sset5}
  y \in a \wedge (y \notin a \vee \neg (y|x)(R)) 
  \leftrightarrow (y \in a \wedge y \notin a) \vee (y \in a \wedge \neg (y|x)(R))
\end{equation}
が成り立つ.
またThm \ref{n1awna1}より$\neg (y \in a \wedge y \notin a)$が成り立つから, 
推論法則 \ref{dedavblbtrue2}により
\begin{equation}
\label{sthma-sset6}
  (y \in a \wedge y \notin a) \vee (y \in a \wedge \neg (y|x)(R)) 
  \leftrightarrow y \in a \wedge \neg (y|x)(R)
\end{equation}
が成り立つ.
そこで(\ref{sthma-sset4})---(\ref{sthma-sset6})から, 推論法則 \ref{dedeqtrans}によって
\[
  y \in a \wedge y \notin \{x \in a \mid R\} \leftrightarrow y \in a \wedge \neg (y|x)(R)
\]
が成り立つことがわかる.
このことと$y$が定数でないことから, 定理 \ref{sthmalleqiset=}より
\[
  \{y \mid y \in a \wedge y \notin \{x \in a \mid R\}\} = \{y \mid y \in a \wedge \neg (y|x)(R)\}, 
\]
即ち
\begin{equation}
\label{sthma-sset7}
  \{y \in a \mid y \notin \{x \in a \mid R\}\} = \{y \in a \mid \neg (y|x)(R)\}
\end{equation}
が成り立つ.
ここで$y$が$a$の中に自由変数として現れず, 
上述のように$\{x \in a \mid R\}$の中にも自由変数として現れないことから, 
定義より$\{y \in a \mid y \notin \{x \in a \mid R\}\}$は$a - \{x \in a \mid R\}$と同じである.
また$y$が$R$の中に自由変数として現れないことから, 変数法則 \ref{valfund}により, 
$y$は$\neg R$の中に自由変数として現れない.
このことと$x$, $y$が共に$a$の中に自由変数として現れないことから, 
代入法則 \ref{substfund}, \ref{substssettrans}により, 
$\{y \in a \mid \neg (y|x)(R)\}$は$\{x \in a \mid \neg R\}$と一致する.
以上のことからわかるように, (\ref{sthma-sset7})は(\ref{sthma-sset1})と一致する.
従って(\ref{sthma-sset1})が成り立つ.
\halmos




\mathstrut
\begin{thm}
\label{sthma-iset=a-sset}%定理6.26%新規%要る?%確認済
$a$を集合, $R$を関係式とし, $x$を$a$の中に自由変数として現れない文字とする.
このとき
\begin{equation}
\label{sthma-iset=a-sset1}
  {\rm Set}_{x}(R) \to a - \{x \mid R\} = a - \{x \in a \mid R\}
\end{equation}
が成り立つ.
またこのことから, 次の(\ref{sthma-iset=a-sset2})が成り立つ.
\begin{equation}
\label{sthma-iset=a-sset2}
  R \text{が} x \text{について集合を作り得るならば,} ~a - \{x \mid R\} = a - \{x \in a \mid R\}.
\end{equation}
\end{thm}


\noindent{\bf 証明}
~$x$が$a$の中に自由変数として現れないことから, 定理 \ref{sthma-iset}より
\begin{equation}
\label{sthma-iset=a-sset3}
  {\rm Set}_{x}(R) \to a - \{x \mid R\} = \{x \in a \mid \neg R\}
\end{equation}
が成り立つ.
同じく$x$が$a$の中に自由変数として現れないことから, 定理 \ref{sthma-sset}より
\[
  a - \{x \in a \mid R\} = \{x \in a \mid \neg R\}
\]
が成り立つ.
故に推論法則 \ref{dedaddeq=}により
\[
  a - \{x \mid R\} = a - \{x \in a \mid R\} \leftrightarrow a - \{x \mid R\} = \{x \in a \mid \neg R\}
\]
が成り立つ.
故に推論法則 \ref{dedequiv}により
\begin{equation}
\label{sthma-iset=a-sset4}
  a - \{x \mid R\} = \{x \in a \mid \neg R\} \to a - \{x \mid R\} = a - \{x \in a \mid R\}
\end{equation}
が成り立つ.
そこで(\ref{sthma-iset=a-sset3}), (\ref{sthma-iset=a-sset4})から, 
推論法則 \ref{dedmmp}によって(\ref{sthma-iset=a-sset1})が成り立つ.
(\ref{sthma-iset=a-sset2})が成り立つことは, 
(\ref{sthma-iset=a-sset1})と推論法則 \ref{dedmp}によって明らかである.
\halmos




\mathstrut
\begin{thm}
\label{sthmiset-sset}%定理6.27%新規%確認済
$a$を集合, $R$を関係式とし, $x$を$a$の中に自由変数として現れない文字とする.
このとき
\begin{equation}
\label{sthmiset-sset1}
  {\rm Set}_{x}(R) \to \{x \mid R\} - \{x \in a \mid R\} = \{x \mid R\} - a
\end{equation}
が成り立つ.
またこのことから, 次の(\ref{sthmiset-sset2})が成り立つ.
\begin{equation}
\label{sthmiset-sset2}
  R \text{が} x \text{について集合を作り得るならば,} ~\{x \mid R\} - \{x \in a \mid R\} = \{x \mid R\} - a.
\end{equation}
\end{thm}


\noindent{\bf 証明}
~$y$を$x$と異なり, $a$及び$R$の中に自由変数として現れない, 定数でない文字とする.
このとき$x$が$a$の中に自由変数として現れないことから, 定理 \ref{sthmssetbasis}より, 
\[
  y \in \{x \in a \mid R\} \leftrightarrow y \in a \wedge (y|x)(R), 
\]
即ち
\[
  y \in \{x \in a \mid R\} \leftrightarrow \neg (y \notin a \vee \neg (y|x)(R))
\]
が成り立つ.
故に推論法則 \ref{dedeqcp}により
\[
  y \notin \{x \in a \mid R\} \leftrightarrow y \notin a \vee \neg (y|x)(R)
\]
が成り立つ.
故に推論法則 \ref{dedaddeqw}により
\begin{equation}
\label{sthmiset-sset3}
  (y|x)(R) \wedge y \notin \{x \in a \mid R\} 
  \leftrightarrow (y|x)(R) \wedge (y \notin a \vee \neg (y|x)(R))
\end{equation}
が成り立つ.
またThm \ref{aw1bvc1l1awb1v1awc1}より
\begin{equation}
\label{sthmiset-sset4}
  (y|x)(R) \wedge (y \notin a \vee \neg (y|x)(R)) 
  \leftrightarrow ((y|x)(R) \wedge y \notin a) \vee ((y|x)(R) \wedge \neg (y|x)(R))
\end{equation}
が成り立つ.
またThm \ref{n1awna1}より
\[
  \neg ((y|x)(R) \wedge \neg (y|x)(R))
\]
が成り立つから, 推論法則 \ref{dedavblbtrue2}により
\begin{equation}
\label{sthmiset-sset5}
  ((y|x)(R) \wedge y \notin a) \vee ((y|x)(R) \wedge \neg (y|x)(R)) 
  \leftrightarrow (y|x)(R) \wedge y \notin a
\end{equation}
が成り立つ.
そこで(\ref{sthmiset-sset3})---(\ref{sthmiset-sset5})から, 推論法則 \ref{dedeqtrans}によって
\[
  (y|x)(R) \wedge y \notin \{x \in a \mid R\} \leftrightarrow (y|x)(R) \wedge y \notin a
\]
が成り立つことがわかる.
故に推論法則 \ref{dedeq&w}により
\[
  (y|x)(R) \to (y \notin \{x \in a \mid R\} \leftrightarrow y \notin a)
\]
が成り立つ.
ここで$x$は$a$の中に自由変数として現れず, 
変数法則 \ref{valsset}により$\{x \in a \mid R\}$の中にも自由変数として現れないから, 
代入法則 \ref{substfree}, \ref{substfund}, \ref{substequiv}によれば, この記号列は
\[
  (y|x)(R) \to (y|x)(x \notin \{x \in a \mid R\} \leftrightarrow x \notin a)
\]
と一致する.
故にこれが成り立つ.
このことと$y$が定数でないことから, 推論法則 \ref{dedspallfund}により
\[
  \forall_{(y|x)(R)}y((y|x)(x \notin \{x \in a \mid R\} \leftrightarrow x \notin a))
\]
が成り立つ.
ここで$y$が$x$と異なり, $a$, $R$の中に自由変数として現れないことから, 
変数法則 \ref{valfund}, \ref{valequiv}, \ref{valsset}により, 
$y$は$x \notin \{x \in a \mid R\} \leftrightarrow x \notin a$の中にも自由変数として現れない.
故に代入法則 \ref{substspquantrans}により, 上記の記号列は
\begin{equation}
\label{sthmiset-sset8}
  \forall_{R}x(x \notin \{x \in a \mid R\} \leftrightarrow x \notin a)
\end{equation}
と一致する.
従ってこれが成り立つ.
また定理 \ref{sthmspiniset}より
\[
  {\rm Set}_{x}(R) 
  \to ((\forall x \in \{x \mid R\})(x \notin \{x \in a \mid R\} \leftrightarrow x \notin a) 
  \leftrightarrow \forall_{R}x(x \notin \{x \in a \mid R\} \leftrightarrow x \notin a))
\]
が成り立つから, 推論法則 \ref{dedpreequiv}により
\[
  {\rm Set}_{x}(R) 
  \to (\forall_{R}x(x \notin \{x \in a \mid R\} \leftrightarrow x \notin a) 
  \to (\forall x \in \{x \mid R\})(x \notin \{x \in a \mid R\} \leftrightarrow x \notin a))
\]
が成り立つ.
故に推論法則 \ref{dedch}により
\begin{equation}
\label{sthmiset-sset9}
  \forall_{R}x(x \notin \{x \in a \mid R\} \leftrightarrow x \notin a) 
  \to ({\rm Set}_{x}(R) \to (\forall x \in \{x \mid R\})(x \notin \{x \in a \mid R\} \leftrightarrow x \notin a))
\end{equation}
が成り立つ.
そこで(\ref{sthmiset-sset8}), (\ref{sthmiset-sset9})から, 推論法則 \ref{dedmp}によって
\begin{equation}
\label{sthmiset-sset10}
  {\rm Set}_{x}(R) 
  \to (\forall x \in \{x \mid R\})(x \notin \{x \in a \mid R\} \leftrightarrow x \notin a)
\end{equation}
が成り立つ.
また定理 \ref{sthmalleqsset=}より
\[
  (\forall x \in \{x \mid R\})(x \notin \{x \in a \mid R\} \leftrightarrow x \notin a) 
  \to \{x \in \{x \mid R\} \mid x \notin \{x \in a \mid R\}\} = \{x \in \{x \mid R\} \mid x \notin a\}
\]
が成り立つ.
ここで$x$は$a$の中に自由変数として現れず, 
上述のように$\{x \in a \mid R\}$の中にも自由変数として現れない.
また変数法則 \ref{valiset}により, $x$は$\{x \mid R\}$の中にも自由変数として現れない.
故に定義から, 上記の記号列は
\begin{equation}
\label{sthmiset-sset11}
  (\forall x \in \{x \mid R\})(x \notin \{x \in a \mid R\} \leftrightarrow x \notin a) 
  \to \{x \mid R\} - \{x \in a \mid R\} = \{x \mid R\} - a
\end{equation}
と同じである.
従ってこれが成り立つ.
そこで(\ref{sthmiset-sset10}), (\ref{sthmiset-sset11})から, 
推論法則 \ref{dedmmp}によって(\ref{sthmiset-sset1})が成り立つ.
(\ref{sthmiset-sset2})が成り立つことは, 
(\ref{sthmiset-sset1})と推論法則 \ref{dedmp}によって明らかである.
\halmos




\mathstrut
\begin{thm}
\label{sthmsset-iset}%定理6.28%新規%確認済
$a$を集合, $R$と$S$を関係式とし, $x$を$a$の中に自由変数として現れない文字とする.
このとき
\begin{equation}
\label{sthmsset-iset1}
  {\rm Set}_{x}(S) \to \{x \in a \mid R\} - \{x \mid S\} = \{x \in a \mid R \wedge \neg S\}
\end{equation}
が成り立つ.
またこのことから, 次の(\ref{sthmsset-iset2})が成り立つ.
\begin{equation}
\label{sthmsset-iset2}
  S \text{が} x \text{について集合を作り得るならば,} ~\{x \in a \mid R\} - \{x \mid S\} = \{x \in a \mid R \wedge \neg S\}.
\end{equation}
\end{thm}


\noindent{\bf 証明}
~変数法則 \ref{valsset}により, $x$は$\{x \in a \mid R\}$の中に自由変数として現れないから, 
定理 \ref{sthma-iset}より
\begin{equation}
\label{sthmsset-iset3}
  {\rm Set}_{x}(S) \to \{x \in a \mid R\} - \{x \mid S\} = \{x \in \{x \in a \mid R\} \mid \neg S\}
\end{equation}
が成り立つ.
また$x$が$a$の中に自由変数として現れないことから, 定理 \ref{sthmssetsset}より
\[
  \{x \in \{x \in a \mid R\} \mid \neg S\} = \{x \in a \mid R \wedge \neg S\}
\]
が成り立つ.
故に推論法則 \ref{dedaddeq=}により
\[
  \{x \in a \mid R\} - \{x \mid S\} = \{x \in \{x \in a \mid R\} \mid \neg S\} 
  \leftrightarrow \{x \in a \mid R\} - \{x \mid S\} = \{x \in a \mid R \wedge \neg S\}
\]
が成り立つ.
故に推論法則 \ref{dedequiv}により
\begin{equation}
\label{sthmsset-iset4}
  \{x \in a \mid R\} - \{x \mid S\} = \{x \in \{x \in a \mid R\} \mid \neg S\} 
  \to \{x \in a \mid R\} - \{x \mid S\} = \{x \in a \mid R \wedge \neg S\}
\end{equation}
が成り立つ.
そこで(\ref{sthmsset-iset3}), (\ref{sthmsset-iset4})から, 
推論法則 \ref{dedmmp}によって(\ref{sthmsset-iset1})が成り立つ.
(\ref{sthmsset-iset2})が成り立つことは, 
(\ref{sthmsset-iset1})と推論法則 \ref{dedmp}によって明らかである.
\halmos




\mathstrut
\begin{thm}
\label{sthmsset-}%定理6.29%三つ目は新規%確認済
$a$と$b$を集合, $R$を関係式とし, $x$を$a$及び$b$の中に自由変数として現れない文字とする.
このとき
\begin{align}
  \label{sthmsset-1}
  &\{x \in a \mid R\} - b = \{x \in a - b \mid R\}, \\
  \mbox{} \notag \\
  \label{sthmsset-2}
  &\{x \in a \mid R\} - \{x \in b \mid R\} = \{x \in a - b \mid R\}, \\
  \mbox{} \notag \\
  \label{sthmsset-3}
  &\{x \in a \mid R\} - b = \{x \in a \mid R\} - \{x \in b \mid R\}
\end{align}
がすべて成り立つ.
\end{thm}


\noindent{\bf 証明}
~まず(\ref{sthmsset-1})が成り立つことを示す.
$x$が$a$の中に自由変数として現れないことから, 定理 \ref{sthmssetsset}より
\[
  \{x \in \{x \in a \mid R\} \mid x \notin b\} = \{x \in a \mid x \notin b \wedge R\}
\]
が成り立つ.
ここで$x$は$b$の中に自由変数として現れず, 
変数法則 \ref{valsset}により$\{x \in a \mid R\}$の中にも自由変数として現れないから, 
定義よりこの記号列は
\begin{equation}
\label{sthmsset-4}
  \{x \in a \mid R\} - b = \{x \in a \mid x \notin b \wedge R\}
\end{equation}
と同じである.
故にこれが成り立つ.
また$x$が$a$の中に自由変数として現れないことから, 定理 \ref{sthmssetsset}と推論法則 \ref{ded=ch}により
\[
  \{x \in a \mid x \notin b \wedge R\} = \{x \in \{x \in a \mid x \notin b\} \mid R\}
\]
が成り立つ.
ここで$x$は$b$の中にも自由変数として現れないから, 定義よりこの記号列は
\begin{equation}
\label{sthmsset-5}
  \{x \in a \mid x \notin b \wedge R\} = \{x \in a - b \mid R\}
\end{equation}
と同じである.
故にこれが成り立つ.
そこで(\ref{sthmsset-4}), (\ref{sthmsset-5})から, 
推論法則 \ref{ded=trans}によって(\ref{sthmsset-1})が成り立つ.

次に(\ref{sthmsset-2})が成り立つことを示す.
$x$が$b$の中に自由変数として現れないことから, 
定理 \ref{sthmssetsm}より$x \in b \wedge R$は$x$について集合を作り得る.
このことと$x$が$a$の中に自由変数として現れないことから, 定理 \ref{sthmsset-iset}より, 
\[
  \{x \in a \mid R\} - \{x \mid x \in b \wedge R\} = \{x \in a \mid R \wedge \neg (x \in b \wedge R)\}, 
\]
即ち
\begin{equation}
\label{sthmsset-6}
  \{x \in a \mid R\} - \{x \in b \mid R\} = \{x \in a \mid R \wedge \neg (x \in b \wedge R)\}
\end{equation}
が成り立つ.
さていま$y$を$x$と異なり, $b$及び$R$の中に自由変数として現れない, 定数でない文字とする.
このときThm \ref{n1awb1lnavnb}より
\[
  \neg (y \in b \wedge (y|x)(R)) \leftrightarrow y \notin b \vee \neg (y|x)(R)
\]
が成り立つから, 推論法則 \ref{dedaddeqw}により
\begin{equation}
\label{sthmsset-7}
  (y|x)(R) \wedge \neg (y \in b \wedge (y|x)(R)) 
  \leftrightarrow (y|x)(R) \wedge (y \notin b \vee \neg (y|x)(R))
\end{equation}
が成り立つ.
またThm \ref{aw1bvc1l1awb1v1awc1}より
\begin{equation}
\label{sthmsset-8}
  (y|x)(R) \wedge (y \notin b \vee \neg (y|x)(R)) 
  \leftrightarrow ((y|x)(R) \wedge y \notin b) \vee ((y|x)(R) \wedge \neg (y|x)(R))
\end{equation}
が成り立つ.
またThm \ref{n1awna1}より
\[
  \neg ((y|x)(R) \wedge \neg (y|x)(R))
\]
が成り立つから, 推論法則 \ref{dedavblbtrue2}により
\begin{equation}
\label{sthmsset-9}
  ((y|x)(R) \wedge y \notin b) \vee ((y|x)(R) \wedge \neg (y|x)(R)) 
  \leftrightarrow (y|x)(R) \wedge y \notin b
\end{equation}
が成り立つ.
そこで(\ref{sthmsset-7})---(\ref{sthmsset-9})から, 推論法則 \ref{dedeqtrans}によって
\[
  (y|x)(R) \wedge \neg (y \in b \wedge (y|x)(R)) \leftrightarrow (y|x)(R) \wedge y \notin b
\]
が成り立つことがわかる.
ここで$x$が$b$の中に自由変数として現れないことから, 
代入法則 \ref{substfree}, \ref{substfund}, \ref{substwedge}, \ref{substequiv}により, この記号列は
\[
  (y|x)(R \wedge \neg (x \in b \wedge R) \leftrightarrow R \wedge x \notin b)
\]
と一致する.
故にこれが成り立つ.
このことと$y$が定数でないことから, 推論法則 \ref{dedltthmquan}により
\[
  \forall y((y|x)(R \wedge \neg (x \in b \wedge R) \leftrightarrow R \wedge x \notin b))
\]
が成り立つ.
ここで$y$が$x$と異なり, $b$, $R$の中に自由変数として現れないことから, 
変数法則 \ref{valfund}, \ref{valwedge}, \ref{valequiv}により, 
$y$は$R \wedge \neg (x \in b \wedge R) \leftrightarrow R \wedge x \notin b$の中に
自由変数として現れない.
故に代入法則 \ref{substquantrans}により, 上記の記号列は
\[
  \forall x(R \wedge \neg (x \in b \wedge R) \leftrightarrow R \wedge x \notin b)
\]
と一致する.
従ってこれが成り立つ.
そこで定理 \ref{sthmalleqsset=}より
\begin{equation}
\label{sthmsset-10}
  \{x \in a \mid R \wedge \neg (x \in b \wedge R)\} = \{x \in a \mid R \wedge x \notin b\}
\end{equation}
が成り立つ.
また$x$が$a$の中に自由変数として現れないことから, 定理 \ref{sthmssetsset}と推論法則 \ref{ded=ch}により
\[
  \{x \in a \mid R \wedge x \notin b\} = \{x \in \{x \in a \mid x \notin b\} \mid R\}
\]
が成り立つ.
ここで$x$は$b$の中にも自由変数として現れないから, 定義よりこの記号列は
\begin{equation}
\label{sthmsset-11}
  \{x \in a \mid R \wedge x \notin b\} = \{x \in a - b \mid R\}
\end{equation}
と同じである.
故にこれが成り立つ.
そこで(\ref{sthmsset-6}), (\ref{sthmsset-10}), (\ref{sthmsset-11})から, 
推論法則 \ref{ded=trans}によって(\ref{sthmsset-2})が成り立つことがわかる.

最後に(\ref{sthmsset-3})が成り立つことを示す.
(\ref{sthmsset-2})から, 推論法則 \ref{ded=ch}により
\[
  \{x \in a - b \mid R\} = \{x \in a \mid R\} - \{x \in b \mid R\}
\]
が成り立つ.
そこでこれと(\ref{sthmsset-1})から, 推論法則 \ref{ded=trans}によって(\ref{sthmsset-3})が成り立つ.
\halmos




\mathstrut
\begin{thm}
\label{sthmsset-rs}%定理6.30%sthmrwnssetsepから変更%確認済
$a$を集合, $R$と$S$を関係式とし, $x$を$a$の中に自由変数として現れない文字とする.
このとき
\begin{equation}
\label{sthmsset-rs1}
  \{x \in a \mid R\} - \{x \in a \mid S\} = \{x \in a \mid R \wedge \neg S\}
\end{equation}
が成り立つ.
\end{thm}


\noindent{\bf 証明}
~変数法則 \ref{valsset}により, $x$は$\{x \in a \mid S\}$の中に自由変数として現れない.
このことと$x$が$a$の中にも自由変数として現れないことから, 定理 \ref{sthmsset-}より
\begin{equation}
\label{sthmsset-rs2}
  \{x \in a \mid R\} - \{x \in a \mid S\} = \{x \in a - \{x \in a \mid S\} \mid R\}
\end{equation}
が成り立つ.
また$x$が$a$の中に自由変数として現れないことから, 定理 \ref{sthma-sset}より
\begin{equation}
\label{sthmsset-rs3}
  a - \{x \in a \mid S\} = \{x \in a \mid \neg S\}
\end{equation}
が成り立つ.
ここで上述のように$x$は$\{x \in a \mid S\}$の中にも自由変数として現れないから, 
変数法則 \ref{val-}により, $x$は$a - \{x \in a \mid S\}$の中に自由変数として現れない.
また変数法則 \ref{valsset}により, $x$は$\{x \in a \mid \neg S\}$の中にも自由変数として現れない.
これらのことと(\ref{sthmsset-rs3})から, 定理 \ref{sthmsset=}より
\begin{equation}
\label{sthmsset-rs4}
  \{x \in a - \{x \in a \mid S\} \mid R\} = \{x \in \{x \in a \mid \neg S\} \mid R\}
\end{equation}
が成り立つ.
また$x$が$a$の中に自由変数として現れないことから, 定理 \ref{sthmssetsset}より
\begin{equation}
\label{sthmsset-rs5}
  \{x \in \{x \in a \mid \neg S\} \mid R\} = \{x \in a \mid R \wedge \neg S\}
\end{equation}
が成り立つ.
そこで(\ref{sthmsset-rs2}), (\ref{sthmsset-rs4}), (\ref{sthmsset-rs5})から, 
推論法則 \ref{ded=trans}によって(\ref{sthmsset-rs1})が成り立つことがわかる.
\halmos




\mathstrut
\begin{thm}
\label{sthmsset-iset=sset-sset}%定理6.31%新規%要る?%確認済
$a$を集合, $R$と$S$を関係式とし, $x$を$a$の中に自由変数として現れない文字とする.
このとき
\begin{equation}
\label{sthmsset-iset=sset-sset1}
  {\rm Set}_{x}(S) \to \{x \in a \mid R\} - \{x \mid S\} = \{x \in a \mid R\} - \{x \in a \mid S\}
\end{equation}
が成り立つ.
またこのことから, 次の(\ref{sthmsset-iset=sset-sset2})が成り立つ.
\begin{equation}
\label{sthmsset-iset=sset-sset2}
  S \text{が} x \text{について集合を作り得るならば,} ~\{x \in a \mid R\} - \{x \mid S\} = \{x \in a \mid R\} - \{x \in a \mid S\}.
\end{equation}
\end{thm}


\noindent{\bf 証明}
~$x$が$a$の中に自由変数として現れないことから, 定理 \ref{sthmsset-iset}より
\begin{equation}
\label{sthmsset-iset=sset-sset3}
  {\rm Set}_{x}(S) \to \{x \in a \mid R\} - \{x \mid S\} = \{x \in a \mid R \wedge \neg S\}
\end{equation}
が成り立つ.
同じく$x$が$a$の中に自由変数として現れないことから, 定理 \ref{sthmsset-rs}より
\[
  \{x \in a \mid R\} - \{x \in a \mid S\} = \{x \in a \mid R \wedge \neg S\}
\]
が成り立つ.
故に推論法則 \ref{dedaddeq=}により
\[
  \{x \in a \mid R\} - \{x \mid S\} = \{x \in a \mid R\} - \{x \in a \mid S\} 
  \leftrightarrow \{x \in a \mid R\} - \{x \mid S\} = \{x \in a \mid R \wedge \neg S\}
\]
が成り立つ.
故に推論法則 \ref{dedequiv}により
\begin{equation}
\label{sthmsset-iset=sset-sset4}
  \{x \in a \mid R\} - \{x \mid S\} = \{x \in a \mid R \wedge \neg S\} 
  \to \{x \in a \mid R\} - \{x \mid S\} = \{x \in a \mid R\} - \{x \in a \mid S\}
\end{equation}
が成り立つ.
そこで(\ref{sthmsset-iset=sset-sset3}), (\ref{sthmsset-iset=sset-sset4})から, 
推論法則 \ref{dedmmp}によって(\ref{sthmsset-iset=sset-sset1})が成り立つ.
(\ref{sthmsset-iset=sset-sset2})が成り立つことは, 
(\ref{sthmsset-iset=sset-sset1})と推論法則 \ref{dedmp}によって明らかである.
\halmos




\mathstrut
\begin{thm}
\label{sthma-bsubsetsset}%定理6.32%新規%確認済
$a$と$b$を集合, $R$を関係式とし, $x$を$b$の中に自由変数として現れない文字とする.
このとき
\begin{equation}
\label{sthma-bsubsetsset1}
  a - b \subset \{x \in b \mid R\} \leftrightarrow a \subset b
\end{equation}
が成り立つ.
またこのことから特に, 次の(\ref{sthma-bsubsetsset2})が成り立つ.
\begin{equation}
\label{sthma-bsubsetsset2}
  a - b \subset \{x \in b \mid R\} \text{ならば,} ~a \subset b.
\end{equation}
\end{thm}


\noindent{\bf 証明}
~$x$が$b$の中に自由変数として現れないことから, 定理 \ref{sthmssetsubseta}より
\[
  \{x \in b \mid R\} \subset b
\]
が成り立つ.
故に定理 \ref{sthm-subseteq}より
\begin{equation}
\label{sthma-bsubsetsset3}
  a \subset b \leftrightarrow a - \{x \in b \mid R\} \subset b - \{x \in b \mid R\}
\end{equation}
が成り立つ.
また定理 \ref{sthma-bsubsetceq}より
\begin{equation}
\label{sthma-bsubsetsset4}
  a - \{x \in b \mid R\} \subset b - \{x \in b \mid R\} 
  \leftrightarrow a - b \subset \{x \in b \mid R\}
\end{equation}
が成り立つ.
そこで(\ref{sthma-bsubsetsset3}), (\ref{sthma-bsubsetsset4})から, 推論法則 \ref{dedeqtrans}によって
\[
  a \subset b \leftrightarrow a - b \subset \{x \in b \mid R\}
\]
が成り立つ.
故に推論法則 \ref{dedeqch}により(\ref{sthma-bsubsetsset1})が成り立つ.
(\ref{sthma-bsubsetsset2})が成り立つことは, 
(\ref{sthma-bsubsetsset1})と推論法則 \ref{dedeqfund}によって明らかである.
\halmos




\mathstrut
\begin{thm}
\label{sthma-uopair}%定理6.33%新規%確認済
$a$, $b$, $c$を集合とし, $x$をこれらの中に自由変数として現れない文字とする.
このとき
\begin{equation}
\label{sthma-uopair1}
  a - \{b, c\} = \{x \in a \mid x \neq b \wedge x \neq c\}
\end{equation}
が成り立つ.
\end{thm}


\noindent{\bf 証明}
~$y$を$x$と異なり, $b$及び$c$の中に自由変数として現れない, 定数でない文字とする.
このとき定理 \ref{sthmuopairbasis}より
\[
  y \in \{b, c\} \leftrightarrow y = b \vee y = c
\]
が成り立つから, 推論法則 \ref{dedeqcp}により
\begin{equation}
\label{sthma-uopair2}
  y \notin \{b, c\} \leftrightarrow \neg (y = b \vee y = c)
\end{equation}
が成り立つ.
またThm \ref{n1awb1lnavnb}より
\begin{equation}
\label{sthma-uopair3}
  \neg (y = b \vee y = c) \leftrightarrow y \neq b \wedge y \neq c
\end{equation}
が成り立つ.
そこで(\ref{sthma-uopair2}), (\ref{sthma-uopair3})から, 推論法則 \ref{dedeqtrans}によって
\[
  y \notin \{b, c\} \leftrightarrow y \neq b \wedge y \neq c
\]
が成り立つ.
ここで$x$が$b$, $c$の中に自由変数として現れないことから, 
変数法則 \ref{valnset}により, $x$は$\{b, c\}$の中にも自由変数として現れない.
故に代入法則 \ref{substfree}, \ref{substfund}, \ref{substwedge}, \ref{substequiv}によれば, 
この記号列は
\[
  (y|x)(x \notin \{b, c\} \leftrightarrow x \neq b \wedge x \neq c)
\]
と一致する.
従ってこれが成り立つ.
このことと$y$が定数でないことから, 推論法則 \ref{dedltthmquan}により
\[
  \forall y((y|x)(x \notin \{b, c\} \leftrightarrow x \neq b \wedge x \neq c))
\]
が成り立つ.
ここで$y$が$x$と異なり, $b$, $c$の中に自由変数として現れないことから, 
変数法則 \ref{valfund}, \ref{valwedge}, \ref{valequiv}, \ref{valnset}により, 
$y$は$x \notin \{b, c\} \leftrightarrow x \neq b \wedge x \neq c$の中に自由変数として現れない.
故に代入法則 \ref{substquantrans}によれば, 上記の記号列は
\[
  \forall x(x \notin \{b, c\} \leftrightarrow x \neq b \wedge x \neq c)
\]
と一致する.
従ってこれが成り立つ.
そこで定理 \ref{sthmalleqsset=}より
\[
  \{x \in a \mid x \notin \{b, c\}\} = \{x \in a \mid x \neq b \wedge x \neq c\}
\]
が成り立つ.
ここで$x$は$a$の中に自由変数として現れず, 上述のように$\{b, c\}$の中にも自由変数として現れないから, 
定義よりこの記号列は(\ref{sthma-uopair1})と同じである.
故に(\ref{sthma-uopair1})が成り立つ.
\halmos




\mathstrut
\begin{thm}
\label{sthma-singleton}%定理6.34%新規%確認済
$a$と$b$を集合とし, $x$をこれらの中に自由変数として現れない文字とする.
このとき
\begin{equation}
\label{sthma-singleton1}
  a - \{b\} = \{x \in a \mid x \neq b\}
\end{equation}
が成り立つ.
\end{thm}


\noindent{\bf 証明}
~$x$が$b$の中に自由変数として現れないことから, 
定理 \ref{sthmsingletonsm}より$x = b$は$x$について集合を作り得る.
このことと$x$が$a$の中に自由変数として現れないことから, 定理 \ref{sthma-iset}より
\[
  a - \{x \mid x = b\} = \{x \in a \mid x \neq b\}
\]
が成り立つ.
ここで$x$が$b$の中に自由変数として現れないことから, 
定義よりこの記号列は(\ref{sthma-singleton1})と同じである.
故に(\ref{sthma-singleton1})が成り立つ.
\halmos




\mathstrut
\begin{thm}
\label{sthma--singleton=a-uopair}%定理6.35%新規%確認済
$a$, $b$, $c$を集合とするとき, 
\begin{equation}
\label{sthma--singleton=a-uopair1}
  (a - \{b\}) - \{c\} = a - \{b, c\}
\end{equation}
が成り立つ.
\end{thm}


\noindent{\bf 証明}
~$x$を$a$, $b$, $c$の中に自由変数として現れない文字とする.
このとき定理 \ref{sthma-singleton}より
\[
  a - \{b\} = \{x \in a \mid x \neq b\}
\]
が成り立つから, 定理 \ref{sthm-=}より
\begin{equation}
\label{sthma--singleton=a-uopair2}
  (a - \{b\}) - \{c\} = \{x \in a \mid x \neq b\} - \{c\}
\end{equation}
が成り立つ.
また変数法則 \ref{valsset}により, $x$は$\{x \in a \mid x \neq b\}$の中に自由変数として現れないから, 
このことと$x$が$c$の中にも自由変数として現れないことから, 定理 \ref{sthma-singleton}より
\begin{equation}
\label{sthma--singleton=a-uopair3}
  \{x \in a \mid x \neq b\} - \{c\} = \{x \in \{x \in a \mid x \neq b\} \mid x \neq c\}
\end{equation}
が成り立つ.
また$x$が$a$の中に自由変数として現れないことから, 定理 \ref{sthmssetsset}より
\begin{equation}
\label{sthma--singleton=a-uopair4}
  \{x \in \{x \in a \mid x \neq b\} \mid x \neq c\} = \{x \in a \mid x \neq b \wedge x \neq c\}
\end{equation}
が成り立つ.
また$x$が$a$, $b$, $c$の中に自由変数として現れないことから, 
定理 \ref{sthma-uopair}と推論法則 \ref{ded=ch}により
\begin{equation}
\label{sthma--singleton=a-uopair5}
  \{x \in a \mid x \neq b \wedge x \neq c\} = a - \{b, c\}
\end{equation}
が成り立つ.
そこで(\ref{sthma--singleton=a-uopair2})---(\ref{sthma--singleton=a-uopair5})から, 
推論法則 \ref{ded=trans}によって(\ref{sthma--singleton=a-uopair1})が成り立つことがわかる.
\halmos




\mathstrut
\begin{thm}
\label{sthmu-s=s-s}%定理6.36%新規%確認済
$a$と$b$を集合とするとき, 
\begin{align}
  \label{sthmu-s=s-s1}
  &\{a, b\} - \{a\} = \{b\} - \{a\}, \\
  \mbox{} \notag \\
  \label{sthmu-s=s-s2}
  &\{a, b\} - \{b\} = \{a\} - \{b\}
\end{align}
が共に成り立つ.
\end{thm}


\noindent{\bf 証明}
~$x$を$a$及び$b$の中に自由変数として現れない, 定数でない文字とする.
このとき定理 \ref{sthmuopairbasis}より
\begin{equation}
\label{sthmu-s=s-s3}
  x \in \{a, b\} \leftrightarrow x = a \vee x = b
\end{equation}
が成り立つ.
また定理 \ref{sthmsingletonbasis}と推論法則 \ref{dedeqch}により
\[
  x = a \leftrightarrow x \in \{a\}, ~~
  x = b \leftrightarrow x \in \{b\}
\]
が共に成り立つから, 推論法則 \ref{dedaddeqv}により
\begin{equation}
\label{sthmu-s=s-s4}
  x = a \vee x = b \leftrightarrow x \in \{a\} \vee x \in \{b\}
\end{equation}
が成り立つ.
そこで(\ref{sthmu-s=s-s3}), (\ref{sthmu-s=s-s4})から, 推論法則 \ref{dedeqtrans}によって
\[
  x \in \{a, b\} \leftrightarrow x \in \{a\} \vee x \in \{b\}
\]
が成り立つ.
故に推論法則 \ref{dedaddeqw}により
\begin{align}
  \label{sthmu-s=s-s5}
  &x \in \{a, b\} \wedge x \notin \{a\} 
  \leftrightarrow (x \in \{a\} \vee x \in \{b\}) \wedge x \notin \{a\}, \\
  \mbox{} \notag \\
  \label{sthmu-s=s-s6}
  &x \in \{a, b\} \wedge x \notin \{b\} 
  \leftrightarrow (x \in \{a\} \vee x \in \{b\}) \wedge x \notin \{b\}
\end{align}
が共に成り立つ.
またThm \ref{aw1bvc1l1awb1v1awc1}より
\begin{align}
  \label{sthmu-s=s-s7}
  &(x \in \{a\} \vee x \in \{b\}) \wedge x \notin \{a\} 
  \leftrightarrow (x \in \{a\} \wedge x \notin \{a\}) \vee (x \in \{b\} \wedge x \notin \{a\}), \\
  \mbox{} \notag \\
  \label{sthmu-s=s-s8}
  &(x \in \{a\} \vee x \in \{b\}) \wedge x \notin \{b\} 
  \leftrightarrow (x \in \{a\} \wedge x \notin \{b\}) \vee (x \in \{b\} \wedge x \notin \{b\})
\end{align}
が共に成り立つ.
またThm \ref{n1awna1}より
\[
  \neg (x \in \{a\} \wedge x \notin \{a\}), ~~
  \neg (x \in \{b\} \wedge x \notin \{b\})
\]
が共に成り立つから, 推論法則 \ref{dedavblbtrue2}により
\begin{align}
  \label{sthmu-s=s-s9}
  &(x \in \{a\} \wedge x \notin \{a\}) \vee (x \in \{b\} \wedge x \notin \{a\}) 
  \leftrightarrow x \in \{b\} \wedge x \notin \{a\}, \\
  \mbox{} \notag \\
  \label{sthmu-s=s-s10}
  &(x \in \{a\} \wedge x \notin \{b\}) \vee (x \in \{b\} \wedge x \notin \{b\}) 
  \leftrightarrow x \in \{a\} \wedge x \notin \{b\}
\end{align}
が共に成り立つ.
そこで(\ref{sthmu-s=s-s5}), (\ref{sthmu-s=s-s7}), (\ref{sthmu-s=s-s9})から, 
推論法則 \ref{dedeqtrans}によって
\[
  x \in \{a, b\} \wedge x \notin \{a\} \leftrightarrow x \in \{b\} \wedge x \notin \{a\}
\]
が成り立つことがわかる.
また(\ref{sthmu-s=s-s6}), (\ref{sthmu-s=s-s8}), (\ref{sthmu-s=s-s10})から, 
同じく推論法則 \ref{dedeqtrans}によって
\[
  x \in \{a, b\} \wedge x \notin \{b\} \leftrightarrow x \in \{a\} \wedge x \notin \{b\}
\]
が成り立つことがわかる.
これらのことと$x$が定数でないことから, 定理 \ref{sthmalleqiset=}より
\begin{align*}
  &\{x \mid x \in \{a, b\} \wedge x \notin \{a\}\} = \{x \mid x \in \{b\} \wedge x \notin \{a\}\}, \\
  \mbox{} \notag \\
  &\{x \mid x \in \{a, b\} \wedge x \notin \{b\}\} = \{x \mid x \in \{a\} \wedge x \notin \{b\}\}
\end{align*}
が共に成り立つ.
ここで$x$が$a$, $b$の中に自由変数として現れないことから, 変数法則 \ref{valnset}により
$x$は$\{a, b\}$, $\{a\}$, $\{b\}$のいずれの記号列の中にも自由変数として現れないから, 
定義より上記の記号列はそれぞれ(\ref{sthmu-s=s-s1}), (\ref{sthmu-s=s-s2})と同じである.
故に(\ref{sthmu-s=s-s1})と(\ref{sthmu-s=s-s2})が共に成り立つ.
\halmos




\mathstrut
\begin{thm}
\label{sthmuopair-c}%定理6.37%新規%確認済
$a$, $b$, $c$を集合とするとき, 
\begin{align}
  \label{sthmuopair-c1}
  &\{a, b\} - c = \{a, b\} \leftrightarrow a \notin c \wedge b \notin c, \\
  \mbox{} \notag \\
  \label{sthmuopair-c2}
  &c - \{a, b\} = c \leftrightarrow a \notin c \wedge b \notin c
\end{align}
が共に成り立つ.
またこれらから, 次の1), 2), 3)が成り立つ.

1)
$\{a, b\} - c = \{a, b\}$ならば, $a \notin c$と$b \notin c$が共に成り立つ.

2)
$c - \{a, b\} = c$ならば, $a \notin c$と$b \notin c$が共に成り立つ.

3)
$a \notin c$と$b \notin c$が共に成り立てば, 
$\{a, b\} - c = \{a, b\}$と$c - \{a, b\} = c$が共に成り立つ.
\end{thm}


\noindent{\bf 証明}
~$x$を$a$, $b$, $c$の中に自由変数として現れない文字とする.
このとき変数法則 \ref{valnset}により, $x$は$\{a, b\}$の中に自由変数として現れないから, 
定理 \ref{sthmsset=a}と推論法則 \ref{dedeqch}により
\[
  \{x \in \{a, b\} \mid x \notin c\} = \{a, b\} \leftrightarrow (\forall x \in \{a, b\})(x \notin c)
\]
が成り立つ.
ここで$x$が$c$の中にも自由変数として現れないことから, 定義よりこの記号列は
\begin{equation}
\label{sthmuopair-c3}
  \{a, b\} - c = \{a, b\} \leftrightarrow (\forall x \in \{a, b\})(x \notin c)
\end{equation}
と同じである.
故にこれが成り立つ.
また$x$が$a$, $b$の中に自由変数として現れないことから, 定理 \ref{sthmspinuopair}より
\[
  (\forall x \in \{a, b\})(x \notin c) \leftrightarrow (a|x)(x \notin c) \wedge (b|x)(x \notin c)
\]
が成り立つ.
ここで$x$が$c$の中に自由変数として現れないことから, 
代入法則 \ref{substfree}, \ref{substfund}によりこの記号列は
\begin{equation}
\label{sthmuopair-c4}
  (\forall x \in \{a, b\})(x \notin c) \leftrightarrow a \notin c \wedge b \notin c
\end{equation}
と一致する.
故にこれが成り立つ.
そこで(\ref{sthmuopair-c3}), (\ref{sthmuopair-c4})から, 
推論法則 \ref{dedeqtrans}によって(\ref{sthmuopair-c1})が成り立つ.
また定理 \ref{sthma-b=aeqb-a=b}より
\[
  c - \{a, b\} = c \leftrightarrow \{a, b\} - c = \{a, b\}
\]
が成り立つから, これと(\ref{sthmuopair-c1})から, 
推論法則 \ref{dedeqtrans}によって(\ref{sthmuopair-c2})が成り立つ.
1), 2), 3)が成り立つことは, (\ref{sthmuopair-c1}), (\ref{sthmuopair-c2})と
推論法則 \ref{dedwedge}, \ref{dedeqfund}によって明らかである.
\halmos




\mathstrut
\begin{thm}
\label{sthmsingleton-b}%定理6.38%新規%確認済
$a$と$b$を集合とするとき, 
\begin{align}
  \label{sthmsingleton-b1}
  &\{a\} - b = \{a\} \leftrightarrow a \notin b, \\
  \mbox{} \notag \\
  \label{sthmsingleton-b2}
  &b - \{a\} = b \leftrightarrow a \notin b
\end{align}
が共に成り立つ.
またこれらから, 次の1), 2)が成り立つ.

1)
$\{a\} - b = \{a\}$ならば, $a \notin b$.
また$b - \{a\} = b$ならば, $a \notin b$.

2)
$a \notin b$ならば, $\{a\} - b = \{a\}$と$b - \{a\} = b$が共に成り立つ.
\end{thm}


\noindent{\bf 証明}
~$x$を$a$及び$b$の中に自由変数として現れない文字とする.
このとき変数法則 \ref{valnset}により, $x$は$\{a\}$の中に自由変数として現れないから, 
定理 \ref{sthmsset=a}と推論法則 \ref{dedeqch}により
\[
  \{x \in \{a\} \mid x \notin b\} = \{a\} \leftrightarrow (\forall x \in \{a\})(x \notin b)
\]
が成り立つ.
ここで$x$が$b$の中にも自由変数として現れないことから, 定義よりこの記号列は
\begin{equation}
\label{sthmsingleton-b3}
  \{a\} - b = \{a\} \leftrightarrow (\forall x \in \{a\})(x \notin b)
\end{equation}
と同じである.
故にこれが成り立つ.
また$x$が$a$の中に自由変数として現れないことから, 定理 \ref{sthmspinsingleton}より
\[
  (\forall x \in \{a\})(x \notin b) \leftrightarrow (a|x)(x \notin b)
\]
が成り立つ.
ここで$x$が$b$の中に自由変数として現れないことから, 
代入法則 \ref{substfree}, \ref{substfund}によりこの記号列は
\begin{equation}
\label{sthmsingleton-b4}
  (\forall x \in \{a\})(x \notin b) \leftrightarrow a \notin b
\end{equation}
と一致する.
故にこれが成り立つ.
そこで(\ref{sthmsingleton-b3}), (\ref{sthmsingleton-b4})から, 
推論法則 \ref{dedeqtrans}によって(\ref{sthmsingleton-b1})が成り立つ.
また定理 \ref{sthma-b=aeqb-a=b}より
\[
  b - \{a\} = b \leftrightarrow \{a\} - b = \{a\}
\]
が成り立つから, これと(\ref{sthmsingleton-b1})から, 
推論法則 \ref{dedeqtrans}によって(\ref{sthmsingleton-b2})が成り立つ.
1), 2)が成り立つことは, (\ref{sthmsingleton-b1}), (\ref{sthmsingleton-b2})と
推論法則 \ref{dedeqfund}によって明らかである.
\halmos




\mathstrut
\begin{thm}
\label{sthmuopairsingleton-}%定理6.39%新規%要る?%確認済
$a$, $b$, $c$を集合とするとき, 
\begin{align}
  \label{sthmuopairsingleton-1}
  &\{a, b\} - \{c\} = \{a, b\} \leftrightarrow a \neq c \wedge b \neq c, \\
  \mbox{} \notag \\
  \label{sthmuopairsingleton-2}
  &\{c\} - \{a, b\} = \{c\} \leftrightarrow a \neq c \wedge b \neq c
\end{align}
が共に成り立つ.
またこれらから, 次の1), 2), 3)が成り立つ.

1)
$\{a, b\} - \{c\} = \{a, b\}$ならば, $a \neq c$と$b \neq c$が共に成り立つ.

2)
$\{c\} - \{a, b\} = \{c\}$ならば, $a \neq c$と$b \neq c$が共に成り立つ.

3)
$a \neq c$と$b \neq c$が共に成り立てば, 
$\{a, b\} - \{c\} = \{a, b\}$と$\{c\} - \{a, b\} = \{c\}$が共に成り立つ.
\end{thm}


\noindent{\bf 証明}
~定理 \ref{sthmuopair-c}より
\begin{equation}
\label{sthmuopairsingleton-3}
  \{a, b\} - \{c\} = \{a, b\} \leftrightarrow a \notin \{c\} \wedge b \notin \{c\}
\end{equation}
が成り立つ.
また定理 \ref{sthmsingletonbasis}より
\[
  a \in \{c\} \leftrightarrow a = c, ~~
  b \in \{c\} \leftrightarrow b = c
\]
が共に成り立つから, 推論法則 \ref{dedeqcp}により
\[
  a \notin \{c\} \leftrightarrow a \neq c, ~~
  b \notin \{c\} \leftrightarrow b \neq c
\]
が共に成り立つ.
故に推論法則 \ref{dedaddeqw}により
\begin{equation}
\label{sthmuopairsingleton-4}
  a \notin \{c\} \wedge b \notin \{c\} \leftrightarrow a \neq c \wedge b \neq c
\end{equation}
が成り立つ.
そこで(\ref{sthmuopairsingleton-3}), (\ref{sthmuopairsingleton-4})から, 
推論法則 \ref{dedeqtrans}によって(\ref{sthmuopairsingleton-1})が成り立つ.
また定理 \ref{sthma-b=aeqb-a=b}より
\[
  \{c\} - \{a, b\} = \{c\} \leftrightarrow \{a, b\} - \{c\} = \{a, b\}
\]
が成り立つから, これと(\ref{sthmuopairsingleton-1})から, 
推論法則 \ref{dedeqtrans}によって(\ref{sthmuopairsingleton-2})が成り立つ.
1), 2), 3)が成り立つことは, (\ref{sthmuopairsingleton-1}), (\ref{sthmuopairsingleton-2})と
推論法則 \ref{dedwedge}, \ref{dedeqfund}によって明らかである.
\halmos




\mathstrut
\begin{thm}
\label{sthmsingleton-}%定理6.40%確認済
$a$と$b$を集合とするとき, 
\begin{equation}
\label{sthmsingleton-1}
  \{a\} - \{b\} = \{a\} \leftrightarrow a \neq b
\end{equation}
が成り立つ.
またこのことから, 次の1), 2)が成り立つ.

1)
$\{a\} - \{b\} = \{a\}$ならば, $a \neq b$.

2)
$a \neq b$ならば, $\{a\} - \{b\} = \{a\}$.
\end{thm}


\noindent{\bf 証明}
~定理 \ref{sthmsingleton-b}より
\begin{equation}
\label{sthmsingleton-2}
  \{a\} - \{b\} = \{a\} \leftrightarrow a \notin \{b\}
\end{equation}
が成り立つ.
また定理 \ref{sthmsingletonbasis}より
\[
  a \in \{b\} \leftrightarrow a = b
\]
が成り立つから, 推論法則 \ref{dedeqcp}により
\begin{equation}
\label{sthmsingleton-3}
  a \notin \{b\} \leftrightarrow a \neq b
\end{equation}
が成り立つ.
そこで(\ref{sthmsingleton-2}), (\ref{sthmsingleton-3})から, 
推論法則 \ref{dedeqtrans}によって(\ref{sthmsingleton-1})が成り立つ.
1), 2)が成り立つことは, (\ref{sthmsingleton-1})と推論法則 \ref{dedeqfund}によって明らかである.
\halmos




\mathstrut
\begin{thm}
\label{sthmu-s=seq}%定理6.41%新規%確認済
$a$と$b$を集合とするとき, 
\begin{align}
  \label{sthmu-s=seq1}
  &\{a, b\} - \{a\} = \{b\} \leftrightarrow a \neq b, \\
  \mbox{} \notag \\
  \label{sthmu-s=seq2}
  &\{a, b\} - \{b\} = \{a\} \leftrightarrow a \neq b
\end{align}
が共に成り立つ.
またこれらから, 次の1), 2), 3)が成り立つ.

1)
$\{a, b\} - \{a\} = \{b\}$ならば, $a \neq b$.

2)
$\{a, b\} - \{b\} = \{a\}$ならば, $a \neq b$.

3)
$a \neq b$ならば, $\{a, b\} - \{a\} = \{b\}$と$\{a, b\} - \{b\} = \{a\}$が共に成り立つ.
\end{thm}


\noindent{\bf 証明}
~定理 \ref{sthmu-s=s-s}より
\[
  \{a, b\} - \{a\} = \{b\} - \{a\}, ~~
  \{a, b\} - \{b\} = \{a\} - \{b\}
\]
が共に成り立つから, 推論法則 \ref{dedaddeq=}により
\begin{align}
  \label{sthmu-s=seq3}
  &\{a, b\} - \{a\} = \{b\} \leftrightarrow \{b\} - \{a\} = \{b\}, \\
  \mbox{} \notag \\
  \label{sthmu-s=seq4}
  &\{a, b\} - \{b\} = \{a\} \leftrightarrow \{a\} - \{b\} = \{a\}
\end{align}
が共に成り立つ.
また定理 \ref{sthmsingleton-}より
\begin{align}
  \label{sthmu-s=seq5}
  &\{b\} - \{a\} = \{b\} \leftrightarrow b \neq a, \\
  \mbox{} \notag \\
  \label{sthmu-s=seq6}
  &\{a\} - \{b\} = \{a\} \leftrightarrow a \neq b
\end{align}
が共に成り立つ.
またThm \ref{xn=ylyn=x}より
\begin{equation}
\label{sthmu-s=seq7}
  b \neq a \leftrightarrow a \neq b
\end{equation}
が成り立つ.
そこで(\ref{sthmu-s=seq3}), (\ref{sthmu-s=seq5}), (\ref{sthmu-s=seq7})から, 
推論法則 \ref{dedeqtrans}によって(\ref{sthmu-s=seq1})が成り立つことがわかる.
また(\ref{sthmu-s=seq4}), (\ref{sthmu-s=seq6})から, 
同じく推論法則 \ref{dedeqtrans}によって(\ref{sthmu-s=seq2})が成り立つ.

\noindent
1)
(\ref{sthmu-s=seq1})と推論法則 \ref{dedeqfund}によって明らか.

\noindent
2)
(\ref{sthmu-s=seq2})と推論法則 \ref{dedeqfund}によって明らか.

\noindent
3)
(\ref{sthmu-s=seq1}), (\ref{sthmu-s=seq2})と推論法則 \ref{dedeqfund}によって明らか.
\halmos




\mathstrut
\begin{thm}
\label{sthmoset-}%定理6.42%確認済
$a$, $b$, $T$を集合とし, $x$を$a$及び$b$の中に自由変数として現れない文字とする.
このとき
\begin{equation}
\label{sthmoset-1}
  \{T\}_{x \in a} - \{T\}_{x \in b} \subset \{T\}_{x \in a - b}
\end{equation}
が成り立つ.
\end{thm}


\noindent{\bf 証明}
~$y$を$x$と異なり, $a$, $b$, $T$の中に自由変数として現れない文字とする.
このとき$y$が$x$と異なるということから, 変数法則 \ref{valfund}, \ref{valoset}により, 
$x$は$y \notin \{T\}_{x \in b}$の中に自由変数として現れない.
このことと, $x$が$a$の中に自由変数として現れないこと, 
及び$y$が$x$と異なり, $a$, $T$の中に自由変数として現れないことから, 定理 \ref{sthmsset&oset}より
\[
  \{y \in \{T\}_{x \in a} \mid y \notin \{T\}_{x \in b}\} 
  = \{T\}_{x \in \{x \in a \mid (T|y)(y \notin \{T\}_{x \in b})\}}
\]
が成り立つ.
ここで$y$が$a$, $b$, $T$の中に自由変数として現れないことから, 変数法則 \ref{valoset}により, 
$y$は$\{T\}_{x \in a}$, $\{T\}_{x \in b}$の中に自由変数として現れないから, 
定義と代入法則 \ref{substfree}, \ref{substfund}によれば, この記号列は
\begin{equation}
\label{sthmoset-2}
  \{T\}_{x \in a} - \{T\}_{x \in b} = \{T\}_{x \in \{x \in a \mid T \notin \{T\}_{x \in b}\}}
\end{equation}
と一致する.
故にこれが成り立つ.
さていま$z$を$x$と異なり, $b$及び$T$の中に自由変数として現れない, 定数でない文字とする.
このとき$x$が$b$の中に自由変数として現れないことから, 定理 \ref{sthmosetfund}より
\[
  z \in b \to (z|x)(T) \in \{T\}_{x \in b}
\]
が成り立つ.
故に推論法則 \ref{dedcp}により
\[
  (z|x)(T) \notin \{T\}_{x \in b} \to z \notin b
\]
が成り立つ.
ここで$x$は$b$の中に自由変数として現れず, 
変数法則 \ref{valoset}により$\{T\}_{x \in b}$の中にも自由変数として現れないから, 
代入法則 \ref{substfree}, \ref{substfund}によれば, この記号列は
\[
  (z|x)(T \notin \{T\}_{x \in b} \to x \notin b)
\]
と一致する.
故にこれが成り立つ.
このことと$z$が定数でないことから, 推論法則 \ref{dedltthmquan}により
\[
  \forall z((z|x)(T \notin \{T\}_{x \in b} \to x \notin b))
\]
が成り立つ.
ここで$z$が$x$と異なり, $b$, $T$の中に自由変数として現れないことから, 
変数法則 \ref{valfund}, \ref{valoset}により, 
$z$は$T \notin \{T\}_{x \in b} \to x \notin b$の中に自由変数として現れないから, 
代入法則 \ref{substquantrans}により, 上記の記号列は
\[
  \forall x(T \notin \{T\}_{x \in b} \to x \notin b)
\]
と一致する.
故にこれが成り立つ.
このことと$x$が$a$の中に自由変数として現れないことから, 定理 \ref{sthmalltssetsubset}より
\[
  \{x \in a \mid T \notin \{T\}_{x \in b}\} \subset \{x \in a \mid x \notin b\}
\]
が成り立つ.
ここで$x$は$b$の中にも自由変数として現れないから, 定義よりこの記号列は
\begin{equation}
\label{sthmoset-3}
  \{x \in a \mid T \notin \{T\}_{x \in b}\} \subset a - b
\end{equation}
と同じである.
故にこれが成り立つ.
さていま変数法則 \ref{valsset}により, 
$x$は$\{x \in a \mid T \notin \{T\}_{x \in b}\}$の中に自由変数として現れない.
また$x$が$a$, $b$の中に自由変数として現れないことから, 
変数法則 \ref{val-}により, $x$は$a - b$の中に自由変数として現れない.
これらのことと(\ref{sthmoset-3})から, 定理 \ref{sthmosetsubset}より
\begin{equation}
\label{sthmoset-4}
  \{T\}_{x \in \{x \in a \mid T \notin \{T\}_{x \in b}\}} \subset \{T\}_{x \in a - b}
\end{equation}
が成り立つ.
そこで(\ref{sthmoset-2}), (\ref{sthmoset-4})から, 
定理 \ref{sthm=tsubseteq}より(\ref{sthmoset-1})が成り立つ.
\halmos




\mathstrut%確認済%koko
次に空集合を定義し, その性質を調べる.
まず以下の四つの定理を証明する.




\mathstrut
\begin{thm}
\label{sthmallnotineqallsubset}%定理6.43%sthmemptyallsubsetから変更%1)----4)を追加%確認済
$a$を集合とし, $x$と$y$を共に$a$の中に自由変数として現れない文字とする.
このとき
\begin{equation}
\label{sthmallnotineqallsubset1}
  \forall x(x \notin a) \leftrightarrow \forall y(a \subset y)
\end{equation}
が成り立つ.
またこのことから, 次の1)---4)が成り立つ.

1)
$\forall x(x \notin a)$ならば, $\forall y(a \subset y)$.

2)
$x$が定数でなく, $x \notin a$が成り立てば, $\forall y(a \subset y)$.

3)
$\forall y(a \subset y)$ならば, $\forall x(x \notin a)$.

4)
$y$が定数でなく, $a \subset y$が成り立てば, $\forall x(x \notin a)$.
\end{thm}


\noindent{\bf 証明}
~$z$を$x$, $y$と異なり, $a$の中に自由変数として現れない, 定数でない文字とする.
このとき, それぞれ変数法則 \ref{valfund}, \ref{valsubset}により, 
$z$は$x \notin a$, $a \subset y$の中に自由変数として現れない.
故に変数法則 \ref{valquan}により, 
$z$は$\forall x(x \notin a)$, $\forall y(a \subset y)$の中にも自由変数として現れない.
さてThm \ref{thmquanseptall2}より
\[
  \forall x(x \notin a) \to \forall x(x \in a \to x \in z)
\]
が成り立つ.
ここで$x$が$z$と異なり, $a$の中に自由変数として現れないことから, 定義よりこの記号列は
\[
  \forall x(x \notin a) \to a \subset z
\]
と一致する.
また$y$が$a$の中に自由変数として現れないことから, 
代入法則 \ref{substfree}, \ref{substsubset}により, この記号列は
\[
  \forall x(x \notin a) \to (z|y)(a \subset y)
\]
と一致する.
故にこれが成り立つ.
このことと, $z$が定数でなく, 上述のように$\forall x(x \notin a)$の中に自由変数として現れないことから, 
推論法則 \ref{dedalltquansepfreeconst}により
\[
  \forall x(x \notin a) \to \forall z((z|y)(a \subset y))
\]
が成り立つ.
ここで上述のように$z$は$a \subset y$の中に自由変数として現れないから, 
代入法則 \ref{substquantrans}によれば, この記号列は
\begin{equation}
\label{sthmallnotineqallsubset2}
  \forall x(x \notin a) \to \forall y(a \subset y)
\end{equation}
と一致する.
故にこれが成り立つ.
またThm \ref{thmallfund2}より
\[
  \forall y(a \subset y) \to (a - a|y)(a \subset y)
\]
が成り立つ.
ここで$y$が$a$の中に自由変数として現れないことから, 
代入法則 \ref{substfree}, \ref{substsubset}により, この記号列は
\begin{equation}
\label{sthmallnotineqallsubset3}
  \forall y(a \subset y) \to a \subset a - a
\end{equation}
と一致する.
故にこれが成り立つ.
また定理 \ref{sthmsubsetbasis}より
\begin{equation}
\label{sthmallnotineqallsubset4}
  a \subset a - a \to (z \in a \to z \in a - a)
\end{equation}
が成り立つ.
またThm \ref{1atb1t1nbtna1}より
\begin{equation}
\label{sthmallnotineqallsubset5}
  (z \in a \to z \in a - a) \to (z \notin a - a \to z \notin a)
\end{equation}
が成り立つ.
また定理 \ref{sthm-basis}より
\[
  z \in a - a \leftrightarrow z \in a \wedge z \notin a
\]
が成り立つ.
このことと, Thm \ref{n1awna1}より$\neg (z \in a \wedge z \notin a)$が成り立つことから, 
推論法則 \ref{dedeqfund}により$z \notin a - a$が成り立つ.
故に推論法則 \ref{ded1atb1tbtrue2}により
\begin{equation}
\label{sthmallnotineqallsubset6}
  (z \notin a - a \to z \notin a) \to z \notin a
\end{equation}
が成り立つ.
そこで(\ref{sthmallnotineqallsubset3})---(\ref{sthmallnotineqallsubset6})から, 推論法則 \ref{dedmmp}によって
\[
  \forall y(a \subset y) \to z \notin a
\]
が成り立つことがわかる.
ここで$x$が$a$の中に自由変数として現れないことから, 
代入法則 \ref{substfree}, \ref{substfund}により, この記号列は
\[
  \forall y(a \subset y) \to (z|x)(x \notin a)
\]
と一致する.
故にこれが成り立つ.
このことと, $z$が定数でなく, 上述のように$\forall y(a \subset y)$の中に自由変数として現れないことから, 
推論法則 \ref{dedalltquansepfreeconst}により
\[
  \forall y(a \subset y) \to \forall z((z|x)(x \notin a))
\]
が成り立つ.
ここで上述のように$z$は$x \notin a$の中に自由変数として現れないから, 
代入法則 \ref{substquantrans}によれば, この記号列は
\begin{equation}
\label{sthmallnotineqallsubset7}
  \forall y(a \subset y) \to \forall x(x \notin a)
\end{equation}
と一致する.
故にこれが成り立つ.
そこで(\ref{sthmallnotineqallsubset2}), (\ref{sthmallnotineqallsubset7})から, 
推論法則 \ref{dedequiv}により(\ref{sthmallnotineqallsubset1})が成り立つ.

\noindent
1), 3)
(\ref{sthmallnotineqallsubset1})と推論法則 \ref{dedeqfund}によって明らか.

\noindent
2)
1)と推論法則 \ref{dedltthmquan}によって明らか.

\noindent
4)
3)と推論法則 \ref{dedltthmquan}によって明らか.
\halmos




\mathstrut
\begin{thm}
\label{sthmallnotintsubset}%定理6.44%新規%確認済
$a$と$b$を集合とし, $x$を$a$の中に自由変数として現れない文字とする.
このとき
\begin{equation}
\label{sthmallnotintsubset1}
  \forall x(x \notin a) \to a \subset b
\end{equation}
が成り立つ.
またこのことから, 次の1), 2)が成り立つ.

1)
$\forall x(x \notin a)$ならば, $a \subset b$.

2)
$x$が定数でなく, $x \notin a$が成り立てば, $a \subset b$.
\end{thm}


\noindent{\bf 証明}
~$x$が$a$の中に自由変数として現れないことから, 
定理 \ref{sthmallnotineqallsubset}と推論法則 \ref{dedequiv}により
\begin{equation}
\label{sthmallnotintsubset2}
  \forall x(x \notin a) \to \forall x(a \subset x)
\end{equation}
が成り立つ.
またThm \ref{thmallfund2}より
\[
  \forall x(a \subset x) \to (b|x)(a \subset x)
\]
が成り立つが, $x$が$a$の中に自由変数として現れないことから, 
代入法則 \ref{substfree}, \ref{substsubset}によればこの記号列は
\begin{equation}
\label{sthmallnotintsubset3}
  \forall x(a \subset x) \to a \subset b
\end{equation}
と一致するから, これが成り立つ.
そこで(\ref{sthmallnotintsubset2}), (\ref{sthmallnotintsubset3})から, 
推論法則 \ref{dedmmp}によって(\ref{sthmallnotintsubset1})が成り立つ.

\noindent
1)
(\ref{sthmallnotintsubset1})と推論法則 \ref{dedmp}によって明らか.

\noindent
2)
1)と推論法則 \ref{dedltthmquan}によって明らか.
\halmos




\mathstrut
\begin{thm}
\label{sthmallsubsettnotin}%定理6.45%新規%確認済
$a$と$b$を集合とし, $x$を$a$の中に自由変数として現れない文字とする.
このとき
\begin{equation}
\label{sthmallsubsettnotin1}
  \forall x(a \subset x) \to b \notin a
\end{equation}
が成り立つ.
またこのことから, 次の1), 2)が成り立つ.

1)
$\forall x(a \subset x)$ならば, $b \notin a$.

2)
$x$が定数でなく, $a \subset x$が成り立てば, $b \notin a$.
\end{thm}


\noindent{\bf 証明}
~$x$が$a$の中に自由変数として現れないことから, 
定理 \ref{sthmallnotineqallsubset}と推論法則 \ref{dedequiv}により
\begin{equation}
\label{sthmallsubsettnotin2}
  \forall x(a \subset x) \to \forall x(x \notin a)
\end{equation}
が成り立つ.
またThm \ref{thmallfund2}より
\[
  \forall x(x \notin a) \to (b|x)(x \notin a)
\]
が成り立つが, $x$が$a$の中に自由変数として現れないことから, 
代入法則 \ref{substfree}, \ref{substfund}によればこの記号列は
\begin{equation}
\label{sthmallsubsettnotin3}
  \forall x(x \notin a) \to b \notin a
\end{equation}
と一致するから, これが成り立つ.
そこで(\ref{sthmallsubsettnotin2}), (\ref{sthmallsubsettnotin3})から, 
推論法則 \ref{dedmmp}によって(\ref{sthmallsubsettnotin1})が成り立つ.

\noindent
1)
(\ref{sthmallsubsettnotin1})と推論法則 \ref{dedmp}によって明らか.

\noindent
2)
1)と推論法則 \ref{dedltthmquan}によって明らか.
\halmos




\mathstrut
\begin{thm}
\label{sthmemptyex!}%定理6.46%確認済
$x$と$y$を異なる文字とするとき, 
\begin{equation}
\label{sthmemptyex!1}
  \exists !y(\forall x(x \notin y))
\end{equation}
が成り立つ.
\end{thm}


\noindent{\bf 証明}
~$z$を$x$, $y$と異なる定数でない文字とする.
このとき定理 \ref{sthm-basis}より
\[
  z \in y - y \leftrightarrow z \in y \wedge z \notin y
\]
が成り立つ.
このことと, Thm \ref{n1awna1}より$\neg (z \in y \wedge z \notin y)$が成り立つことから, 
推論法則 \ref{dedeqfund}により
\[
  z \notin y - y
\]
が成り立つ.
ここで$x$が$y$と異なることから, 変数法則 \ref{val-}により$x$は$y - y$の中に自由変数として現れないから, 
代入法則 \ref{substfree}, \ref{substfund}により, この記号列は
\[
  (z|x)(x \notin y - y)
\]
と一致する.
故にこれが成り立つ.
このことと$z$が定数でないことから, 推論法則 \ref{dedltthmquan}により
\[
  \forall z((z|x)(x \notin y - y))
\]
が成り立つ.
ここで$z$が$x$, $y$と異なることから, 変数法則 \ref{valfund}, \ref{val-}により, 
$z$は$x \notin y - y$の中に自由変数として現れないから, 
代入法則 \ref{substquantrans}によればこの記号列は
\[
  \forall x(x \notin y - y)
\]
と一致する.
また$x$が$y$と異なり, 上述のように$y - y$の中に自由変数として現れないことから, 
代入法則 \ref{substquan}により, この記号列は
\[
  (y - y|y)(\forall x(x \notin y))
\]
と一致する.
故にこれが成り立つ.
そこで推論法則 \ref{deds4}により
\begin{equation}
\label{sthmemptyex!2}
  \exists y(\forall x(x \notin y))
\end{equation}
が成り立つ.
さていま$u$, $v$を, 互いに異なり, 共に$x$, $y$と異なる, 定数でない文字とする.
このとき変数法則 \ref{valquan}により, 
$u$と$v$は共に$\forall x(x \notin y)$の中に自由変数として現れない.
また$x$が$u$とも$v$とも異なることから, 定理 \ref{sthmallnotintsubset}より
\[
  \forall x(x \notin u) \to u \subset v, ~~
  \forall x(x \notin v) \to v \subset u
\]
が共に成り立つ.
ここで$x$は$y$とも異なるから, 代入法則 \ref{substquan}により, これらの記号列はそれぞれ
\[
  (u|y)(\forall x(x \notin y)) \to u \subset v, ~~
  (v|y)(\forall x(x \notin y)) \to v \subset u
\]
と一致する.
故にこれらが共に成り立つ.
そこで推論法則 \ref{dedfromaddw}により
\begin{equation}
\label{sthmemptyex!3}
  (u|y)(\forall x(x \notin y)) \wedge (v|y)(\forall x(x \notin y)) \to u \subset v \wedge v \subset u
\end{equation}
が成り立つ.
また定理 \ref{sthmaxiom1}と推論法則 \ref{dedequiv}により
\begin{equation}
\label{sthmemptyex!4}
  u \subset v \wedge v \subset u \to u = v
\end{equation}
が成り立つ.
そこで(\ref{sthmemptyex!3}), (\ref{sthmemptyex!4})から, 推論法則 \ref{dedmmp}によって
\[
  (u|y)(\forall x(x \notin y)) \wedge (v|y)(\forall x(x \notin y)) \to u = v
\]
が成り立つ.
このことと, $u$, $v$が互いに異なり, 共に$y$と異なり, 共に定数でなく, 
上述のように共に$\forall x(x \notin y)$の中に自由変数として現れないことから, 
推論法則 \ref{ded!const}により
\begin{equation}
\label{sthmemptyex!5}
  !y(\forall x(x \notin y))
\end{equation}
が成り立つ.
そこで(\ref{sthmemptyex!2}), (\ref{sthmemptyex!5})から, 
推論法則 \ref{dedwedge}により(\ref{sthmemptyex!1})が成り立つ.
\halmos




\mathstrut
\begin{defo}
\label{emptyset}%変形19%確認済
$\mathscr{T}$を特殊記号として$\in$を持つ理論とする.
また$x$と$y$, $z$と$w$をそれぞれ互いに異なる文字とする.
このとき
\[
  \tau_{y}(\forall x(x \notin y)) \equiv \tau_{w}(\forall z(z \notin w))
\]
が成り立つ.
\end{defo}


\noindent{\bf 証明}
~$u$と$v$を, 互いに異なり, 共に$x$, $y$, $z$, $w$のいずれとも異なる文字とする.
このとき$u$は$x \notin y$の中に自由変数として現れないから, 代入法則 \ref{substquantrans}により
\begin{equation}
\label{emptyset1}
  \forall x(x \notin y) \equiv \forall u((u|x)(x \notin y))
\end{equation}
が成り立つ.
また$x$と$y$が異なることから, 
\begin{equation}
\label{emptyset2}
  (u|x)(x \notin y) \equiv u \notin y
\end{equation}
が成り立つ.
また$u$と$v$が異なることから, 
\begin{equation}
\label{emptyset3}
  u \notin y \equiv (y|v)(u \notin v)
\end{equation}
が成り立つ.
また$u$が$y$, $v$と異なることから, 代入法則 \ref{substquan}により
\begin{equation}
\label{emptyset4}
  \forall u((y|v)(u \notin v)) \equiv (y|v)(\forall u(u \notin v))
\end{equation}
が成り立つ.
また$y$が$u$, $v$と異なることから, 変数法則 \ref{valquan}により, 
$y$は$\forall u(u \notin v)$の中に自由変数として現れないから, 
代入法則 \ref{substtautrans}により
\begin{equation}
\label{emptyset5}
  \tau_{y}((y|v)(\forall u(u \notin v))) \equiv \tau_{v}(\forall u(u \notin v))
\end{equation}
が成り立つ.
そこで(\ref{emptyset1})---(\ref{emptyset5})から, 
\[
  \tau_{y}(\forall x(x \notin y)) \equiv \tau_{v}(\forall u(u \notin v))
\]
が成り立つことがわかる.
以上の議論と全く同様にして, 
\[
  \tau_{w}(\forall z(z \notin w)) \equiv \tau_{v}(\forall u(u \notin v))
\]
も成り立つ.
従って, $\tau_{y}(\forall x(x \notin y))$と$\tau_{w}(\forall z(z \notin w))$は同一の記号列である.
\halmos




\mathstrut
\begin{defi}
\label{defempty}%定義2%確認済
$\mathscr{T}$を特殊記号として$\in$を持つ理論とする.
また$x$と$y$, $z$と$w$をそれぞれ互いに異なる文字とする.
このとき変形法則 \ref{emptyset}によれば, 
$\tau_{y}(\forall x(x \notin y))$と$\tau_{w}(\forall z(z \notin w))$は同じ記号列となる.
以下この記号列を$\phi$と書き表す.
\end{defi}




\mathstrut%註%確認済
{\small
\noindent
\textbf{註.} 
$\phi$を省略記法を用いずに書けば, 
$\tau \neg \neg \neg\! \in\! \tau \neg \neg\! \in\! \Box \Box' \Box$となる.
}




\mathstrut%確認済%koko
以下の変数法則 \ref{valempty}, 構成法則 \ref{formempty}では, 
$\mathscr{T}$を特殊記号として$\in$を持つ理論とする.
また構成法則 \ref{formempty}における``集合''とは, $\mathscr{T}$の対象式のこととする.




\mathstrut
\begin{valu}
\label{valempty}%変数30%内容を変更した%確認済
$\phi$は自由変数を持たない.
\end{valu}


\noindent{\bf 証明}
~$x$を任意の文字とする.
このとき$y$を$x$と異なる文字とすれば, 定義より$\phi$は$\tau_{x}(\forall y(y \notin x))$と同じである.
変数法則 \ref{valtau}により, $x$はこの中に自由変数として現れない.
\halmos




\mathstrut
\begin{form}
\label{formempty}%構成47%内容を変更した%確認済
$\phi$は集合である.
\end{form}


\noindent{\bf 証明}
~$x$と$y$を異なる文字とするとき, 定義より$\phi$は$\tau_{y}(\forall x(x \notin y))$と同じである.
これが集合となることは, 構成法則 \ref{formfund}, \ref{formquan}によって直ちにわかる.
\halmos




\mathstrut%確認済%koko
上記の構成法則 \ref{formempty}により, $\in$を特殊記号に持つ任意の理論において, $\phi$はその集合である.
以下$\phi$を\textbf{空集合} (empty set) と呼ぶ.




\mathstrut%確認済%koko
$\mathscr{T}$を特殊記号として$=$と$\in$を持つ理論とし, $a$を$\mathscr{T}$の集合とする.
$a = \phi$が$\mathscr{T}$の定理であるとき, 
($\mathscr{T}$において) \textbf{${\bm a}$は空である}という.
また$a \neq \phi$が$\mathscr{T}$の定理であるとき, 
($\mathscr{T}$において) \textbf{${\bm a}$は空でない}という (この後者の表現は
``$a = \phi$は ($\mathscr{T}$の) 定理でない''という意味ではないから注意).




\mathstrut
\begin{thm}
\label{sthmemptyeqallnotin}%定理6.47%新規%確認済
$a$を集合とし, $x$を$a$の中に自由変数として現れない文字とする.
このとき
\begin{equation}
\label{sthmemptyeqallnotin1}
  a = \phi \leftrightarrow \forall x(x \notin a)
\end{equation}
が成り立つ.
またこのことから, 次の1), 2), 3)が成り立つ.

1)
$a$が空ならば, $\forall x(x \notin a)$.

2)
$\forall x(x \notin a)$ならば, $a$は空である.

3)
$x$が定数でなく, $x \notin a$が成り立てば, $a$は空である.
\end{thm}


\noindent{\bf 証明}
~$y$を$x$と異なる文字とするとき, 
定理 \ref{sthmemptyex!}より$\exists !y(\forall x(x \notin y))$が成り立つ.
故に推論法則 \ref{dedex!tTtau}により
\[
  (a|y)(\forall x(x \notin y)) \leftrightarrow a = \tau_{y}(\forall x(x \notin y))
\]
が成り立つ.
ここで$x$と$y$が異なることから, 定義よりこの記号列は
\[
  (a|y)(\forall x(x \notin y)) \leftrightarrow a = \phi
\]
と同じである.
また$x$が$y$と異なり, $a$の中に自由変数として現れないことから, 
代入法則 \ref{substquan}により, この記号列は
\[
  \forall x(x \notin a) \leftrightarrow a = \phi
\]
と一致する.
故にこれが成り立つ.
そこで推論法則 \ref{dedeqch}により(\ref{sthmemptyeqallnotin1})が成り立つ.

\noindent
1), 2)
(\ref{sthmemptyeqallnotin1})と推論法則 \ref{dedeqfund}によって明らか.

\noindent
3)
2)と推論法則 \ref{dedltthmquan}によって明らか.
\halmos




\mathstrut
\begin{thm}
\label{sthmallnotinempty}%定理6.48%新規%確認済
$x$を文字とするとき, 
\[
  \forall x(x \notin \phi)
\]
が成り立つ.
\end{thm}


\noindent{\bf 証明}
~変数法則 \ref{valempty}により, $x$は$\phi$の中に自由変数として現れない.
またThm \ref{x=x}より$\phi$は空である.
故に定理 \ref{sthmemptyeqallnotin}より$\forall x(x \notin \phi)$が成り立つ.
\halmos




\mathstrut
\begin{thm}
\label{sthmemptytnotin}%定理6.49%新規%確認済
$a$と$b$を集合とするとき, 
\begin{equation}
\label{sthmemptytnotin1}
  a = \phi \to b \notin a
\end{equation}
が成り立つ.
またこのことから, 次の(\ref{sthmemptytnotin2})が成り立つ.
\begin{equation}
\label{sthmemptytnotin2}
  a \text{が空ならば,} ~b \notin a.
\end{equation}
\end{thm}


\noindent{\bf 証明}
~$x$を$a$の中に自由変数として現れない文字とするとき, 
定理 \ref{sthmemptyeqallnotin}と推論法則 \ref{dedequiv}により
\begin{equation}
\label{sthmemptytnotin3}
  a = \phi \to \forall x(x \notin a)
\end{equation}
が成り立つ.
またThm \ref{thmallfund2}より
\[
  \forall x(x \notin a) \to (b|x)(x \notin a)
\]
が成り立つが, $x$が$a$の中に自由変数として現れないことから, 
代入法則 \ref{substfree}, \ref{substfund}によりこの記号列は
\begin{equation}
\label{sthmemptytnotin4}
  \forall x(x \notin a) \to b \notin a
\end{equation}
と一致するから, これが成り立つ.
そこで(\ref{sthmemptytnotin3}), (\ref{sthmemptytnotin4})から, 
推論法則 \ref{dedmmp}によって(\ref{sthmemptytnotin1})が成り立つ.
(\ref{sthmemptytnotin2})が成り立つことは, 
(\ref{sthmemptytnotin1})と推論法則 \ref{dedmp}によって明らかである.
\halmos




\mathstrut
\begin{thm}
\label{sthmnotinempty}%定理6.50%新規%確認済
$a$を集合とするとき, 
\[
  a \notin \phi
\]
が成り立つ.
\end{thm}


\noindent{\bf 証明}
~Thm \ref{x=x}より$\phi$は空だから, 定理 \ref{sthmemptytnotin}より$a \notin \phi$が成り立つ.
\halmos




\mathstrut
\begin{thm}
\label{sthmemptyeqallsubset}%定理6.51%新規%確認済
$a$を集合とし, $x$を$a$の中に自由変数として現れない文字とする.
このとき
\begin{equation}
\label{sthmemptyeqallsubset1}
  a = \phi \leftrightarrow \forall x(a \subset x)
\end{equation}
が成り立つ.
またこのことから, 次の1), 2), 3)が成り立つ.

1)
$a$が空ならば, $\forall x(a \subset x)$.

2)
$\forall x(a \subset x)$ならば, $a$は空である.

3)
$x$が定数でなく, $a \subset x$が成り立てば, $a$は空である.
\end{thm}


\noindent{\bf 証明}
~$x$が$a$の中に自由変数として現れないことから, 定理 \ref{sthmemptyeqallnotin}より
\begin{equation}
\label{sthmemptyeqallsubset2}
  a = \phi \leftrightarrow \forall x(x \notin a)
\end{equation}
が成り立つ.
同じく$x$が$a$の中に自由変数として現れないことから, 定理 \ref{sthmallnotineqallsubset}より
\begin{equation}
\label{sthmemptyeqallsubset3}
  \forall x(x \notin a) \leftrightarrow \forall x(a \subset x)
\end{equation}
が成り立つ.
そこで(\ref{sthmemptyeqallsubset2}), (\ref{sthmemptyeqallsubset3})から, 
推論法則 \ref{dedeqtrans}によって(\ref{sthmemptyeqallsubset1})が成り立つ.

\noindent
1), 2)
(\ref{sthmemptyeqallsubset1})と推論法則 \ref{dedeqfund}によって明らか.

\noindent
3)
2)と推論法則 \ref{dedltthmquan}によって明らか.
\halmos




\mathstrut
\begin{thm}
\label{sthmallemptysubset}%定理6.52%新規%確認済
$x$を文字とするとき, 
\[
  \forall x(\phi \subset x)
\]
が成り立つ.
\end{thm}


\noindent{\bf 証明}
~変数法則 \ref{valempty}により, $x$は$\phi$の中に自由変数として現れない.
またThm \ref{x=x}より$\phi$は空である.
故に定理 \ref{sthmemptyeqallsubset}より$\forall x(\phi \subset x)$が成り立つ.
\halmos




\mathstrut
\begin{thm}
\label{sthmemptytsubset}%定理6.53%新規%確認済
$a$と$b$を集合とするとき, 
\begin{equation}
\label{sthmemptytsubset1}
  a = \phi \to a \subset b
\end{equation}
が成り立つ.
またこのことから, 次の(\ref{sthmemptytsubset2})が成り立つ.
\begin{equation}
\label{sthmemptytsubset2}
  a \text{が空ならば,} ~a \subset b.
\end{equation}
\end{thm}


\noindent{\bf 証明}
~$x$を$a$の中に自由変数として現れない文字とするとき, 
定理 \ref{sthmemptyeqallsubset}と推論法則 \ref{dedequiv}により
\begin{equation}
\label{sthmemptytsubset3}
  a = \phi \to \forall x(a \subset x)
\end{equation}
が成り立つ.
またThm \ref{thmallfund2}より
\[
  \forall x(a \subset x) \to (b|x)(a \subset x)
\]
が成り立つが, $x$が$a$の中に自由変数として現れないことから, 
代入法則 \ref{substfree}, \ref{substsubset}によりこの記号列は
\begin{equation}
\label{sthmemptytsubset4}
  \forall x(a \subset x) \to a \subset b
\end{equation}
と一致するから, これが成り立つ.
そこで(\ref{sthmemptytsubset3}), (\ref{sthmemptytsubset4})から, 
推論法則 \ref{dedmmp}によって(\ref{sthmemptytsubset1})が成り立つ.
(\ref{sthmemptytsubset2})が成り立つことは, 
(\ref{sthmemptytsubset1})と推論法則 \ref{dedmp}によって明らかである.
\halmos




\mathstrut
\begin{thm}
\label{sthmemptysubset}%定理6.54%新規%確認済
$a$を集合とするとき, 
\[
  \phi \subset a
\]
が成り立つ.
\end{thm}


\noindent{\bf 証明}
~Thm \ref{x=x}より$\phi$は空だから, 定理 \ref{sthmemptytsubset}より$\phi \subset a$が成り立つ.
\halmos




\mathstrut
\begin{thm}
\label{sthmemptysubset=eq}%定理6.55%確認済
$a$を集合とするとき, 
\begin{equation}
\label{sthmemptysubset=eq1}
  a \subset \phi \leftrightarrow a = \phi
\end{equation}
が成り立つ.
またこのことから特に, 次の(\ref{sthmemptysubset=eq2})が成り立つ.
\begin{equation}
\label{sthmemptysubset=eq2}
  a \subset \phi ~\text{ならば,} ~a \text{は空である.}
\end{equation}
\end{thm}


\noindent{\bf 証明}
~定理 \ref{sthmemptysubset}より$\phi \subset a$が成り立つから, 推論法則 \ref{dedawblatrue2}により
\[
  a \subset \phi \wedge \phi \subset a \leftrightarrow a \subset \phi
\]
が成り立つ.
故に推論法則 \ref{dedeqch}により
\begin{equation}
\label{sthmemptysubset=eq3}
  a \subset \phi \leftrightarrow a \subset \phi \wedge \phi \subset a
\end{equation}
が成り立つ.
また定理 \ref{sthmaxiom1}より
\begin{equation}
\label{sthmemptysubset=eq4}
  a \subset \phi \wedge \phi \subset a \leftrightarrow a = \phi
\end{equation}
が成り立つ.
そこで(\ref{sthmemptysubset=eq3}), (\ref{sthmemptysubset=eq4})から, 
推論法則 \ref{dedeqtrans}によって(\ref{sthmemptysubset=eq1})が成り立つ.
(\ref{sthmemptysubset=eq2})が成り立つことは, 
(\ref{sthmemptysubset=eq1})と推論法則 \ref{dedeqfund}によって明らかである.
\halmos




\mathstrut
\begin{thm}
\label{sthmnotemptyeqexin}%定理6.56%新規%確認済
$a$を集合とし, $x$を$a$の中に自由変数として現れない文字とする.
このとき
\begin{equation}
\label{sthmnotemptyeqexin1}
  a \neq \phi \leftrightarrow \exists x(x \in a)
\end{equation}
が成り立つ.
またこのことから, 次の1), 2)が成り立つ.

1)
$a$が空でなければ, $\exists x(x \in a)$.

2)
$\exists x(x \in a)$ならば, $a$は空でない.
\end{thm}


\noindent{\bf 証明}
~$x$が$a$の中に自由変数として現れないことから, 定理 \ref{sthmemptyeqallnotin}より
\begin{equation}
\label{sthmnotemptyeqexin2}
  a = \phi \leftrightarrow \forall x(x \notin a)
\end{equation}
が成り立つ.
またThm \ref{thmeaquandm}と推論法則 \ref{dedeqch}により
\begin{equation}
\label{sthmnotemptyeqexin3}
  \forall x(x \notin a) \leftrightarrow \neg \exists x(x \in a)
\end{equation}
が成り立つ.
そこで(\ref{sthmnotemptyeqexin2}), (\ref{sthmnotemptyeqexin3})から, 推論法則 \ref{dedeqtrans}によって
\[
  a = \phi \leftrightarrow \neg \exists x(x \in a)
\]
が成り立つ.
故に推論法則 \ref{dedeqcp}により(\ref{sthmnotemptyeqexin1})が成り立つ.
1), 2)が成り立つことは, (\ref{sthmnotemptyeqexin1})と推論法則 \ref{dedeqfund}によって明らかである.
\halmos




\mathstrut
\begin{thm}
\label{sthmintnotempty}%定理6.57%新規%確認済
$a$と$b$を集合とするとき, 
\begin{equation}
\label{sthmintnotempty1}
  b \in a \to a \neq \phi
\end{equation}
が成り立つ.
またこのことから, 次の(\ref{sthmintnotempty2})が成り立つ.
\begin{equation}
\label{sthmintnotempty2}
  b \in a \text{ならば,} ~a \text{は空でない.}
\end{equation}
\end{thm}


\noindent{\bf 証明}
~定理 \ref{sthmemptytnotin}より
\[
  a = \phi \to b \notin a
\]
が成り立つから, 推論法則 \ref{dedcp}により(\ref{sthmintnotempty1})が成り立つ.
(\ref{sthmintnotempty2})が成り立つことは, 
(\ref{sthmintnotempty1})と推論法則 \ref{dedmp}によって明らかである.
\halmos




\mathstrut
\begin{thm}
\label{sthmnotemptyeqpsubset}%定理6.58%新規%確認済
$a$を集合とするとき, 
\begin{equation}
\label{sthmnotemptyeqpsubset1}
  a \neq \phi \leftrightarrow \phi \subsetneqq a
\end{equation}
が成り立つ.
またこのことから, 次の1), 2)が成り立つ.

1)
$a$が空でなければ, $\phi \subsetneqq a$.

2)
$\phi \subsetneqq a$ならば, $a$は空でない.
\end{thm}


\noindent{\bf 証明}
~Thm \ref{xn=ylyn=x}より
\begin{equation}
\label{sthmnotemptyeqpsubset2}
  a \neq \phi \leftrightarrow \phi \neq a
\end{equation}
が成り立つ.
また定理 \ref{sthmemptysubset}より$\phi \subset a$が成り立つから, 
推論法則 \ref{dedeqch}, \ref{dedawblatrue2}により
\begin{equation}
\label{sthmnotemptyeqpsubset3}
  \phi \neq a \leftrightarrow \phi \subsetneqq a
\end{equation}
が成り立つ.
そこで(\ref{sthmnotemptyeqpsubset2}), (\ref{sthmnotemptyeqpsubset3})から, 
推論法則 \ref{dedeqtrans}によって(\ref{sthmnotemptyeqpsubset1})が成り立つ.
1), 2)が成り立つことは, (\ref{sthmnotemptyeqpsubset1})と推論法則 \ref{dedeqfund}によって明らかである.
\halmos




\mathstrut
\begin{thm}
\label{sthmpsubsettnotempty}%定理6.59%新規%確認済
$a$と$b$を集合とするとき, 
\begin{equation}
\label{sthmpsubsettnotempty1}
  a \subsetneqq b \to b \neq \phi
\end{equation}
が成り立つ.
またこのことから, 次の(\ref{sthmpsubsettnotempty2})が成り立つ.
\begin{equation}
\label{sthmpsubsettnotempty2}
  a \subsetneqq b \text{ならば,} ~b \text{は空でない.}
\end{equation}
\end{thm}


\noindent{\bf 証明}
~定理 \ref{sthmemptysubset}より$\phi \subset a$が成り立つから, 推論法則 \ref{dedatawbtrue2}により
\begin{equation}
\label{sthmpsubsettnotempty3}
  a \subsetneqq b \to \phi \subset a \wedge a \subsetneqq b
\end{equation}
が成り立つ.
また定理 \ref{sthmpsubsettrans}より
\begin{equation}
\label{sthmpsubsettnotempty4}
  \phi \subset a \wedge a \subsetneqq b \to \phi \subsetneqq b
\end{equation}
が成り立つ.
また定理 \ref{sthmnotemptyeqpsubset}と推論法則 \ref{dedequiv}により
\begin{equation}
\label{sthmpsubsettnotempty5}
  \phi \subsetneqq b \to b \neq \phi
\end{equation}
が成り立つ.
そこで(\ref{sthmpsubsettnotempty3})---(\ref{sthmpsubsettnotempty5})から, 
推論法則 \ref{dedmmp}によって(\ref{sthmpsubsettnotempty1})が成り立つことがわかる.
(\ref{sthmpsubsettnotempty2})が成り立つことは, 
(\ref{sthmpsubsettnotempty1})と推論法則 \ref{dedmp}によって明らかである.
\halmos




\mathstrut
\begin{thm}
\label{sthmnotemptyeqexpsubset}%定理6.60%新規%確認済
$a$を集合とし, $x$を$a$の中に自由変数として現れない文字とする.
このとき
\begin{equation}
\label{sthmnotemptyeqexpsubset1}
  a \neq \phi \leftrightarrow \exists x(x \subsetneqq a)
\end{equation}
が成り立つ.
またこのことから, 次の1), 2)が成り立つ.

1)
$a$が空でなければ, $\exists x(x \subsetneqq a)$.

2)
$\exists x(x \subsetneqq a)$ならば, $a$は空でない.
\end{thm}


\noindent{\bf 証明}
~定理 \ref{sthmnotemptyeqpsubset}と推論法則 \ref{dedequiv}により
\[
  a \neq \phi \to \phi \subsetneqq a
\]
が成り立つ.
ここで$x$が$a$の中に自由変数として現れないことから, 
代入法則 \ref{substfree}, \ref{substfund}, \ref{substwedge}, \ref{substsubset}によってわかるように, 
この記号列は
\begin{equation}
\label{sthmnotemptyeqexpsubset2}
  a \neq \phi \to (\phi|x)(x \subsetneqq a)
\end{equation}
と一致する.
故にこれが成り立つ.
またschema S4の適用により
\begin{equation}
\label{sthmnotemptyeqexpsubset3}
  (\phi|x)(x \subsetneqq a) \to \exists x(x \subsetneqq a)
\end{equation}
が成り立つ.
そこで(\ref{sthmnotemptyeqexpsubset2}), (\ref{sthmnotemptyeqexpsubset3})から, 
推論法則 \ref{dedmmp}によって
\begin{equation}
\label{sthmnotemptyeqexpsubset4}
  a \neq \phi \to \exists x(x \subsetneqq a)
\end{equation}
が成り立つ.
また定理 \ref{sthmpsubsettnotempty}より
\[
  \tau_{x}(x \subsetneqq a) \subsetneqq a \to a \neq \phi
\]
が成り立つ.
ここで$x$が$a$の中に自由変数として現れないことから, 
代入法則 \ref{substfree}, \ref{substfund}, \ref{substwedge}, \ref{substsubset}によってわかるように, 
この記号列は
\[
  (\tau_{x}(x \subsetneqq a)|x)(x \subsetneqq a) \to a \neq \phi, 
\]
即ち
\begin{equation}
\label{sthmnotemptyeqexpsubset5}
  \exists x(x \subsetneqq a) \to a \neq \phi
\end{equation}
と一致する.
故にこれが成り立つ.
そこで(\ref{sthmnotemptyeqexpsubset4}), (\ref{sthmnotemptyeqexpsubset5})から, 
推論法則 \ref{dedequiv}によって(\ref{sthmnotemptyeqexpsubset1})が成り立つ.
1), 2)が成り立つことは, (\ref{sthmnotemptyeqexpsubset1})と推論法則 \ref{dedeqfund}によって明らかである.
\halmos




\mathstrut
\begin{thm}
\label{sthmelm&empty}%定理6.61%新規%確認済
$a$を集合とするとき, 
\begin{align}
  \label{sthmelm&empty1}
  &a = \phi \leftrightarrow {\rm elm}(a) \notin a, \\
  \mbox{} \notag \\
  \label{sthmelm&empty2}
  &a \neq \phi \leftrightarrow {\rm elm}(a) \in a
\end{align}
が共に成り立つ.
またこれらから, 次の1), 2)が成り立つ.

1)
${\rm elm}(a) \notin a$ならば, $a$は空である.

2)
$a$が空でなければ, ${\rm elm}(a) \in a$.
\end{thm}


\noindent{\bf 証明}
~$x$を$a$の中に自由変数として現れない文字とするとき, 定理 \ref{sthmnotemptyeqexin}より
\[
  a \neq \phi \leftrightarrow \exists x(x \in a)
\]
が成り立つが, この記号列は(\ref{sthmelm&empty2})と同じだから, (\ref{sthmelm&empty2})が成り立つ.
故に推論法則 \ref{dedeqcp}により(\ref{sthmelm&empty1})も成り立つ.

\noindent
1)
(\ref{sthmelm&empty1})と推論法則 \ref{dedeqfund}によって明らか.

\noindent
2)
(\ref{sthmelm&empty2})と推論法則 \ref{dedeqfund}によって明らか.
\halmos




\mathstrut
\begin{thm}
\label{sthmemptytspin}%定理6.62%新規%確認済
$a$を集合, $R$を関係式とし, $x$を$a$の中に自由変数として現れない文字とする.
このとき
\begin{align}
  \label{sthmemptytspin1}
  &a = \phi \to \neg (\exists x \in a)(R), \\
  \mbox{} \notag \\
  \label{sthmemptytspin2}
  &a = \phi \to (\forall x \in a)(R)
\end{align}
が共に成り立つ.
またこれらから, 次の(\ref{sthmemptytspin3})が成り立つ.
\begin{equation}
\label{sthmemptytspin3}
  a \text{が空ならば,} ~\neg (\exists x \in a)(R) \text{と} (\forall x \in a)(R) \text{が共に成り立つ.}
\end{equation}
\end{thm}


\noindent{\bf 証明}
~$x$が$a$の中に自由変数として現れないことから, 
定理 \ref{sthmemptyeqallnotin}と推論法則 \ref{dedequiv}により
\begin{equation}
\label{sthmemptytspin4}
  a = \phi \to \forall x(x \notin a)
\end{equation}
が成り立つ.
またThm \ref{thmspexneg}より
\begin{equation}
\label{sthmemptytspin5}
  \forall x(x \notin a) \to \neg (\exists x \in a)(R)
\end{equation}
が成り立つ.
またThm \ref{thmquantspall2}より
\begin{equation}
\label{sthmemptytspin6}
  \forall x(x \notin a) \to (\forall x \in a)(R)
\end{equation}
が成り立つ.
そこで(\ref{sthmemptytspin4})と(\ref{sthmemptytspin5}), 
(\ref{sthmemptytspin4})と(\ref{sthmemptytspin6})から, 
それぞれ推論法則 \ref{dedmmp}によって(\ref{sthmemptytspin1}), (\ref{sthmemptytspin2})が成り立つ.
(\ref{sthmemptytspin3})が成り立つことは, 
(\ref{sthmemptytspin1}), (\ref{sthmemptytspin2})と推論法則 \ref{dedmp}によって明らかである.
\halmos




\mathstrut
\begin{thm}
\label{sthmspinempty}%定理6.63%新規%確認済
$R$を関係式とし, $x$を文字とするとき, 
\[
  \neg (\exists x \in \phi)(R), ~~
  (\forall x \in \phi)(R)
\]
が共に成り立つ.
\end{thm}


\noindent{\bf 証明}
~変数法則 \ref{valempty}により, $x$は$\phi$の中に自由変数として現れない.
またThm \ref{x=x}より$\phi$は空である.
故に定理 \ref{sthmemptytspin}より, 
$\neg (\exists x \in \phi)(R)$と$(\forall x \in \phi)(R)$が共に成り立つ.
\halmos




\mathstrut
\begin{thm}
\label{sthmisetempty}%定理6.64%新規%確認済
$R$を関係式とし, $x$を文字とするとき, 
\begin{align}
  \label{sthmisetempty1}
  &\neg \exists x(R) \leftrightarrow {\rm Set}_{x}(R) \wedge \{x \mid R\} = \phi, \\
  \mbox{} \notag \\
  \label{sthmisetempty2}
  &\forall x(\neg R) \leftrightarrow {\rm Set}_{x}(R) \wedge \{x \mid R\} = \phi
\end{align}
が共に成り立つ.
またこれらから, 次の1)---4)が成り立つ.

1)
$\neg \exists x(R)$ならば, $R$は$x$について集合を作り得る.
またこのとき$\{x \mid R\}$は空である.

2)
$\forall x(\neg R)$ならば, $R$は$x$について集合を作り得る.
またこのとき$\{x \mid R\}$は空である.

3)
$x$が定数でなく, $\neg R$が成り立てば, $R$は$x$について集合を作り得る.
またこのとき$\{x \mid R\}$は空である.

4)
$R$が$x$について集合を作り得るとする.
このとき$\{x \mid R\}$が空ならば, $\neg \exists x(R)$と$\forall x(\neg R)$が共に成り立つ.
\end{thm}


\noindent{\bf 証明}
~$y$を$x$と異なり, $R$の中に自由変数として現れない, 定数でない文字とする.
このときThm \ref{1alb1l1nalnb1}より
\begin{equation}
\label{sthmisetempty3}
  (y \in \phi \leftrightarrow (y|x)(R)) \leftrightarrow (y \notin \phi \leftrightarrow \neg (y|x)(R))
\end{equation}
が成り立つ.
また定理 \ref{sthmnotinempty}より$y \notin \phi$が成り立つから, 推論法則 \ref{ded1alb1lbtrue}により
\begin{equation}
\label{sthmisetempty4}
  (y \notin \phi \leftrightarrow \neg (y|x)(R)) \leftrightarrow \neg (y|x)(R)
\end{equation}
が成り立つ.
そこで(\ref{sthmisetempty3}), (\ref{sthmisetempty4})から, 推論法則 \ref{dedeqtrans}によって
\[
  (y \in \phi \leftrightarrow (y|x)(R)) \leftrightarrow \neg (y|x)(R)
\]
が成り立つ.
故に推論法則 \ref{dedeqch}により
\[
  \neg (y|x)(R) \leftrightarrow (y \in \phi \leftrightarrow (y|x)(R))
\]
が成り立つ.
ここで変数法則 \ref{valempty}により, $x$は$\phi$の中に自由変数として現れないから, 
代入法則 \ref{substfree}, \ref{substfund}, \ref{substequiv}により, この記号列は
\[
  (y|x)(\neg R \leftrightarrow (x \in \phi \leftrightarrow R))
\]
と一致する.
故にこれが成り立つ.
このことと$y$が定数でないことから, 推論法則 \ref{dedltthmquan}により
\[
  \forall y((y|x)(\neg R \leftrightarrow (x \in \phi \leftrightarrow R)))
\]
が成り立つ.
ここで$y$が$x$と異なり, $R$の中に自由変数として現れないことから, 
変数法則 \ref{valfund}, \ref{valequiv}, \ref{valempty}により, 
$y$は$\neg R \leftrightarrow (x \in \phi \leftrightarrow R)$の中に自由変数として現れない.
故に代入法則 \ref{substquantrans}により, 上記の記号列は
\[
  \forall x(\neg R \leftrightarrow (x \in \phi \leftrightarrow R))
\]
と一致する.
従ってこれが成り立つ.
そこで推論法則 \ref{dedalleqquansep}により
\begin{equation}
\label{sthmisetempty5}
  \forall x(\neg R) \leftrightarrow \forall x(x \in \phi \leftrightarrow R)
\end{equation}
が成り立つ.
また上述のように$x$は$\phi$の中に自由変数として現れないから, 定理 \ref{sthmsmbasis&iset=a}より
\begin{equation}
\label{sthmisetempty6}
  \forall x(x \in \phi \leftrightarrow R) \leftrightarrow {\rm Set}_{x}(R) \wedge \{x \mid R\} = \phi
\end{equation}
が成り立つ.
そこで(\ref{sthmisetempty5}), (\ref{sthmisetempty6})から, 
推論法則 \ref{dedeqtrans}によって(\ref{sthmisetempty2})が成り立つ.
またThm \ref{thmeaquandm}より
\[
  \neg \exists x(R) \leftrightarrow \forall x(\neg R)
\]
が成り立つから, これと(\ref{sthmisetempty2})から, 
推論法則 \ref{dedeqtrans}によって(\ref{sthmisetempty1})が成り立つ.

\noindent
1)
(\ref{sthmisetempty1})と推論法則 \ref{dedwedge}, \ref{dedeqfund}によって明らか.

\noindent
2)
(\ref{sthmisetempty2})と推論法則 \ref{dedwedge}, \ref{dedeqfund}によって明らか.

\noindent
3)
2)と推論法則 \ref{dedltthmquan}によって明らか.

\noindent
4)
(\ref{sthmisetempty1}), (\ref{sthmisetempty2})と
推論法則 \ref{dedwedge}, \ref{dedeqfund}によって明らか.
\halmos




\mathstrut
\begin{thm}
\label{sthmisetfree}%定理6.65%新規%$\neg R \leftrightarrow \{x \mid R\} = \phi$は言えない?%確認済
$R$を関係式とし, $x$を$R$の中に自由変数として現れない文字とする.
このとき
\begin{equation}
\label{sthmisetfree1}
  \neg R \leftrightarrow {\rm Set}_{x}(R) \wedge \{x \mid R\} = \phi
\end{equation}
が成り立つ.
またこのことから特に, 次の(\ref{sthmisetfree2})が成り立つ.
\begin{equation}
\label{sthmisetfree2}
  \neg R \text{ならば,} ~R \text{は} x \text{について集合を作り得る.} ~
  \text{またこのとき} \{x \mid R\} \text{は空である.}
\end{equation}
\end{thm}


\noindent{\bf 証明}
~定理 \ref{sthmisetempty}より
\[
  \neg \exists x(R) \leftrightarrow {\rm Set}_{x}(R) \wedge \{x \mid R\} = \phi
\]
が成り立つが, $x$が$R$の中に自由変数として現れないことから, 変形法則 \ref{quanfree}により
この記号列は(\ref{sthmisetfree1})と一致するから, (\ref{sthmisetfree1})が成り立つ.
(\ref{sthmisetfree2})が成り立つことは, (\ref{sthmisetfree1})と
推論法則 \ref{dedwedge}, \ref{dedeqfund}によって明らかである.
\halmos




\mathstrut
\begin{thm}
\label{sthmsunotempty}%定理6.66%新規%確認済
\mbox{}

1)
$a$を集合とするとき, $\{a\}$は空でない.

2)
$a$と$b$を集合とするとき, $\{a, b\}$は空でない.
\end{thm}


\noindent{\bf 証明}
~1)
定理 \ref{sthmsingletonfund}より$a \in \{a\}$が成り立つから, 
定理 \ref{sthmintnotempty}より$\{a\}$は空でない.

\noindent
2)
定理 \ref{sthmuopairfund}より$a \in \{a, b\}$が成り立つから, 
定理 \ref{sthmintnotempty}より$\{a, b\}$は空でない.
\halmos




\mathstrut
\begin{thm}
\label{sthmexxxn=y}%定理6.67%確認済
$x$と$y$を異なる文字とするとき, 
\[
  \exists x(\exists y(x \neq y))
\]
が成り立つ.
\end{thm}


\noindent{\bf 証明}
~定理 \ref{sthmsunotempty}より
\[
  \{\phi\} \neq \phi
\]
が成り立つが, $x$と$y$が異なる文字であることから, この記号列は
\[
  (\{\phi\}|x, \phi|y)(x \neq y)
\]
と一致するから, これが成り立つ.
故に推論法則 \ref{dedgs4}により$\exists x(\exists y(x \neq y))$が成り立つ.
\halmos




\mathstrut
\begin{thm}
\label{sthmsubsetofsingleton}%定理6.68%確認済
$a$と$b$を集合とするとき, 
\begin{equation}
\label{sthmsubsetofsingleton1}
  b \subset \{a\} \leftrightarrow b = \phi \vee b = \{a\}
\end{equation}
が成り立つ.
またこのことから特に, 次の(\ref{sthmsubsetofsingleton2})が成り立つ.
\begin{equation}
\label{sthmsubsetofsingleton2}
  b \subset \{a\} \text{ならば,} ~b = \phi \vee b = \{a\}.
\end{equation}
\end{thm}


\noindent{\bf 証明}
~$x$を$a$, $b$の中に自由変数として現れない, 定数でない文字とする.
このとき定理 \ref{sthmsingletonbasis}と推論法則 \ref{dedequiv}により
\[
  x \in \{a\} \to x = a
\]
が成り立つから, 推論法則 \ref{dedaddb}により
\begin{equation}
\label{sthmsubsetofsingleton3}
  (x \in b \to x \in \{a\}) \to (x \in b \to x = a)
\end{equation}
が成り立つ.
また定理 \ref{sthm=tineq}より
\[
  x = a \to (x \in b \leftrightarrow a \in b)
\]
が成り立つから, 推論法則 \ref{dedpreequiv}により
\[
  x = a \to (x \in b \to a \in b)
\]
が成り立つ.
故に推論法則 \ref{dedch}により
\[
  x \in b \to (x = a \to a \in b)
\]
が成り立つ.
故に推論法則 \ref{deds2}により
\begin{equation}
\label{sthmsubsetofsingleton4}
  (x \in b \to x = a) \to (x \in b \to a \in b)
\end{equation}
が成り立つ.
そこで(\ref{sthmsubsetofsingleton3}), (\ref{sthmsubsetofsingleton4})から, 
推論法則 \ref{dedmmp}によって
\[
  (x \in b \to x \in \{a\}) \to (x \in b \to a \in b)
\]
が成り立つ.
このことと$x$が定数でないことから, 推論法則 \ref{dedalltquansepconst}により
\[
  \forall x(x \in b \to x \in \{a\}) \to \forall x(x \in b \to a \in b)
\]
が成り立つ.
ここで$x$が$a$の中に自由変数として現れないことから, 
変数法則 \ref{valnset}により$x$は$\{a\}$の中に自由変数として現れないから, 
このことと$x$が$b$の中にも自由変数として現れないことから, 定義よりこの記号列は
\begin{equation}
\label{sthmsubsetofsingleton5}
  b \subset \{a\} \to \forall x(x \in b \to a \in b)
\end{equation}
と同じである.
故にこれが成り立つ.
また$x$が$a$, $b$の中に自由変数として現れないことから, 
変数法則 \ref{valfund}により$x$は$a \in b$の中に自由変数として現れないから, 
Thm \ref{thmalltexsepsfree}と推論法則 \ref{dedequiv}により
\begin{equation}
\label{sthmsubsetofsingleton6}
  \forall x(x \in b \to a \in b) \to (\exists x(x \in b) \to a \in b)
\end{equation}
が成り立つ.
また$x$が$b$の中に自由変数として現れないことから, 定理 \ref{sthmnotemptyeqexin}より
\begin{equation}
\label{sthmsubsetofsingleton7}
  b \neq \phi \leftrightarrow \exists x(x \in b)
\end{equation}
が成り立つ.
また定理 \ref{sthmsingletonsubset}より
\begin{equation}
\label{sthmsubsetofsingleton8}
  \{a\} \subset b \leftrightarrow a \in b
\end{equation}
が成り立つ.
そこで(\ref{sthmsubsetofsingleton7}), (\ref{sthmsubsetofsingleton8})から, 
推論法則 \ref{dedaddeqt}により
\[
  (b \neq \phi \to \{a\} \subset b) \leftrightarrow (\exists x(x \in b) \to a \in b)
\]
が成り立つ.
故に推論法則 \ref{dedequiv}により
\begin{equation}
\label{sthmsubsetofsingleton9}
  (\exists x(x \in b) \to a \in b) \to (b \neq \phi \to \{a\} \subset b)
\end{equation}
が成り立つ.
そこで(\ref{sthmsubsetofsingleton5}), (\ref{sthmsubsetofsingleton6}), (\ref{sthmsubsetofsingleton9})から, 
推論法則 \ref{dedmmp}によって
\[
  b \subset \{a\} \to (b \neq \phi \to \{a\} \subset b)
\]
が成り立つことがわかる.
故に推論法則 \ref{dedt&w}により
\begin{equation}
\label{sthmsubsetofsingleton10}
  b \subset \{a\} \wedge b \neq \phi \to b \subset \{a\} \wedge \{a\} \subset b
\end{equation}
が成り立つ.
また定理 \ref{sthmaxiom1}と推論法則 \ref{dedequiv}により
\begin{equation}
\label{sthmsubsetofsingleton11}
  b \subset \{a\} \wedge \{a\} \subset b \to b = \{a\}
\end{equation}
が成り立つ.
そこで(\ref{sthmsubsetofsingleton10}), (\ref{sthmsubsetofsingleton11})から, 
推論法則 \ref{dedmmp}によって
\[
  b \subset \{a\} \wedge b \neq \phi \to b = \{a\}
\]
が成り立つ.
故に推論法則 \ref{dedtwch}により, 
\[
  b \subset \{a\} \to (b \neq \phi \to b = \{a\}), 
\]
即ち
\begin{equation}
\label{sthmsubsetofsingleton12}
  b \subset \{a\} \to b = \phi \vee b = \{a\}
\end{equation}
が成り立つ.
また定理 \ref{sthmemptytsubset}より
\[
  b = \phi \to b \subset \{a\}
\]
が成り立ち, 定理 \ref{sthm=tsubset}より
\[
  b = \{a\} \to b \subset \{a\}
\]
が成り立つから, 推論法則 \ref{deddil}により
\begin{equation}
\label{sthmsubsetofsingleton13}
  b = \phi \vee b = \{a\} \to b \subset \{a\}
\end{equation}
が成り立つ.
そこで(\ref{sthmsubsetofsingleton12}), (\ref{sthmsubsetofsingleton13})から, 
推論法則 \ref{dedequiv}により(\ref{sthmsubsetofsingleton1})が成り立つ.
(\ref{sthmsubsetofsingleton2})が成り立つことは, 
(\ref{sthmsubsetofsingleton1})と推論法則 \ref{dedeqfund}によって明らかである.
\halmos




\mathstrut
\begin{thm}
\label{sthmsubsetofuopair}%定理6.69%新規%確認済
$a$, $b$, $c$を集合とするとき, 
\begin{equation}
\label{sthmsubsetofuopair1}
  c \subset \{a, b\} \leftrightarrow c = \phi \vee c = \{a\} \vee c = \{b\} \vee c = \{a, b\}
\end{equation}
が成り立つ.
またこのことから特に, 次の(\ref{sthmsubsetofuopair2})が成り立つ.
\begin{equation}
\label{sthmsubsetofuopair2}
  c \subset \{a, b\} \text{ならば,} ~c = \phi \vee c = \{a\} \vee c = \{b\} \vee c = \{a, b\}.
\end{equation}
\end{thm}


\noindent{\bf 証明}
~まず
\begin{align}
  \label{sthmsubsetofuopair3}
  &c \subset \{a, b\} \wedge a \notin c \to c \subset \{b\}, \\
  \mbox{} \notag \\
  \label{sthmsubsetofuopair4}
  &c \subset \{a, b\} \wedge b \notin c \to c \subset \{a\}
\end{align}
が共に成り立つことを示す.

(\ref{sthmsubsetofuopair3})の証明: 
$x$を$a$, $b$, $c$の中に自由変数として現れない, 定数でない文字とする.
このとき定理 \ref{sthmuopairbasis}と推論法則 \ref{dedequiv}により
\[
  x \in \{a, b\} \to x = a \vee x = b
\]
が成り立つから, 推論法則 \ref{dedaddb}により
\begin{equation}
\label{sthmsubsetofuopair5}
  (x \in c \to x \in \{a, b\}) \to (x \in c \to x = a \vee x = b)
\end{equation}
が成り立つ.
またThm \ref{1atbvc1t1atb1vc}より
\begin{equation}
\label{sthmsubsetofuopair6}
  (x \in c \to x = a \vee x = b) \to (x \in c \to x = a) \vee x = b
\end{equation}
が成り立つ.
また定理 \ref{sthm=tineq}より
\[
  x = a \to (x \in c \leftrightarrow a \in c)
\]
が成り立つから, 推論法則 \ref{dedpreequiv}により
\[
  x = a \to (x \in c \to a \in c)
\]
が成り立つ.
故に推論法則 \ref{dedch}により
\[
  x \in c \to (x = a \to a \in c)
\]
が成り立つ.
故に推論法則 \ref{deds2}により
\begin{equation}
\label{sthmsubsetofuopair7}
  (x \in c \to x = a) \to (x \in c \to a \in c)
\end{equation}
が成り立つ.
また定理 \ref{sthmsingletonbasis}と推論法則 \ref{dedequiv}により
\begin{equation}
\label{sthmsubsetofuopair8}
  x = b \to x \in \{b\}
\end{equation}
が成り立つ.
そこで(\ref{sthmsubsetofuopair7}), (\ref{sthmsubsetofuopair8})から, 推論法則 \ref{dedfromaddv}により
\begin{equation}
\label{sthmsubsetofuopair9}
  (x \in c \to x = a) \vee x = b \to (x \in c \to a \in c) \vee x \in \{b\}
\end{equation}
が成り立つ.
またThm \ref{1atb1vct1atbvc1}より, 
\[
  (x \in c \to a \in c) \vee x \in \{b\} \to (x \in c \to a \in c \vee x \in \{b\}), 
\]
即ち
\begin{equation}
\label{sthmsubsetofuopair10}
  (x \in c \to a \in c) \vee x \in \{b\} \to (x \in c \to (a \notin c \to x \in \{b\}))
\end{equation}
が成り立つ.
またThm \ref{1at1btc11t1bt1atc11}より
\begin{equation}
\label{sthmsubsetofuopair11}
  (x \in c \to (a \notin c \to x \in \{b\})) \to (a \notin c \to (x \in c \to x \in \{b\}))
\end{equation}
が成り立つ.
そこで(\ref{sthmsubsetofuopair5}), (\ref{sthmsubsetofuopair6}), 
(\ref{sthmsubsetofuopair9})---(\ref{sthmsubsetofuopair11})から, 推論法則 \ref{dedmmp}によって
\[
  (x \in c \to x \in \{a, b\}) \to (a \notin c \to (x \in c \to x \in \{b\}))
\]
が成り立つことがわかる.
このことと$x$が定数でないことから, 推論法則 \ref{dedalltquansepconst}により
\[
  \forall x(x \in c \to x \in \{a, b\}) \to \forall x(a \notin c \to (x \in c \to x \in \{b\}))
\]
が成り立つ.
ここで$x$が$a$, $b$の中に自由変数として現れないことから, 
変数法則 \ref{valnset}により$x$は$\{a, b\}$の中に自由変数として現れないから, 
このことと$x$が$c$の中にも自由変数として現れないことから, 定義よりこの記号列は
\begin{equation}
\label{sthmsubsetofuopair12}
  c \subset \{a, b\} \to \forall x(a \notin c \to (x \in c \to x \in \{b\}))
\end{equation}
と同じである.
故にこれが成り立つ.
また$x$が$a$, $c$の中に自由変数として現れないことから, 
変数法則 \ref{valfund}により$x$は$a \notin c$の中に自由変数として現れないから, 
Thm \ref{thmalltallseprfree}と推論法則 \ref{dedequiv}により
\[
  \forall x(a \notin c \to (x \in c \to x \in \{b\})) 
  \to (a \notin c \to \forall x(x \in c \to x \in \{b\}))
\]
が成り立つ.
ここで$x$が$b$の中に自由変数として現れないことから, 
変数法則 \ref{valnset}により$x$は$\{b\}$の中に自由変数として現れないから, 
このことと$x$が$c$の中にも自由変数として現れないことから, 定義よりこの記号列は
\begin{equation}
\label{sthmsubsetofuopair13}
  \forall x(a \notin c \to (x \in c \to x \in \{b\})) \to (a \notin c \to c \subset \{b\})
\end{equation}
と同じである.
故にこれが成り立つ.
そこで(\ref{sthmsubsetofuopair12}), (\ref{sthmsubsetofuopair13})から, 推論法則 \ref{dedmmp}によって
\[
  c \subset \{a, b\} \to (a \notin c \to c \subset \{b\})
\]
が成り立つ.
故に推論法則 \ref{dedtwch}により(\ref{sthmsubsetofuopair3})が成り立つ.

(\ref{sthmsubsetofuopair4})の証明: 
定理 \ref{sthmuopairch}より
\[
  \{a, b\} = \{b, a\}
\]
が成り立つから, 定理 \ref{sthm=tsubseteq}より
\[
  c \subset \{a, b\} \leftrightarrow c \subset \{b, a\}
\]
が成り立つ.
故に推論法則 \ref{dedequiv}により
\[
  c \subset \{a, b\} \to c \subset \{b, a\}
\]
が成り立つ.
故に推論法則 \ref{dedaddw}により
\begin{equation}
\label{sthmsubsetofuopair14}
  c \subset \{a, b\} \wedge b \notin c \to c \subset \{b, a\} \wedge b \notin c
\end{equation}
が成り立つ.
また上で示したように(\ref{sthmsubsetofuopair3})が成り立つが, 
(\ref{sthmsubsetofuopair3})における$a$と$b$は任意の集合で良いから, 
$a$と$b$を入れ替えた
\begin{equation}
\label{sthmsubsetofuopair15}
  c \subset \{b, a\} \wedge b \notin c \to c \subset \{a\}
\end{equation}
も成り立つ.
そこで(\ref{sthmsubsetofuopair14}), (\ref{sthmsubsetofuopair15})から, 
推論法則 \ref{dedmmp}によって(\ref{sthmsubsetofuopair4})が成り立つ.

さて(\ref{sthmsubsetofuopair1})が成り立つことを示す.
定理 \ref{sthmaxiom1}と推論法則 \ref{dedequiv}により
\[
  c \subset \{a, b\} \wedge \{a, b\} \subset c \to c = \{a, b\}
\]
が成り立つから, 推論法則 \ref{dedtwch}により
\begin{equation}
\label{sthmsubsetofuopair16}
  c \subset \{a, b\} \to (\{a, b\} \subset c \to c = \{a, b\})
\end{equation}
が成り立つ.
またThm \ref{1atb1t1nbtna1}より
\begin{equation}
\label{sthmsubsetofuopair17}
  (\{a, b\} \subset c \to c = \{a, b\}) \to (c \neq \{a, b\} \to \{a, b\} \not\subset c)
\end{equation}
が成り立つ.
そこで(\ref{sthmsubsetofuopair16}), (\ref{sthmsubsetofuopair17})から, 推論法則 \ref{dedmmp}によって
\[
  c \subset \{a, b\} \to (c \neq \{a, b\} \to \{a, b\} \not\subset c)
\]
が成り立つ.
故に推論法則 \ref{dedt&w}により
\begin{equation}
\label{sthmsubsetofuopair18}
  c \subset \{a, b\} \wedge c \neq \{a, b\} \to c \subset \{a, b\} \wedge \{a, b\} \not\subset c
\end{equation}
が成り立つ.
また定理 \ref{sthmuopairsubset}と推論法則 \ref{dedequiv}により, 
\[
  a \in c \wedge b \in c \to \{a, b\} \subset c, 
\]
即ち
\[
  \neg (a \notin c \vee b \notin c) \to \{a, b\} \subset c
\]
が成り立つから, 推論法則 \ref{dedcp}により
\begin{equation}
\label{sthmsubsetofuopair19}
  \{a, b\} \not\subset c \to a \notin c \vee b \notin c
\end{equation}
が成り立つ.
またThm \ref{avbtbva}より
\begin{equation}
\label{sthmsubsetofuopair20}
  a \notin c \vee b \notin c \to b \notin c \vee a \notin c
\end{equation}
が成り立つ.
そこで(\ref{sthmsubsetofuopair19}), (\ref{sthmsubsetofuopair20})から, 推論法則 \ref{dedmmp}によって
\[
  \{a, b\} \not\subset c \to b \notin c \vee a \notin c
\]
が成り立つ.
故に推論法則 \ref{dedaddw}により
\begin{equation}
\label{sthmsubsetofuopair21}
  c \subset \{a, b\} \wedge \{a, b\} \not\subset c 
  \to c \subset \{a, b\} \wedge (b \notin c \vee a \notin c)
\end{equation}
が成り立つ.
またThm \ref{aw1bvc1t1awb1v1awc1}より
\begin{equation}
\label{sthmsubsetofuopair22}
  c \subset \{a, b\} \wedge (b \notin c \vee a \notin c) 
  \to (c \subset \{a, b\} \wedge b \notin c) \vee (c \subset \{a, b\} \wedge a \notin c)
\end{equation}
が成り立つ.
また定理 \ref{sthmsubsetofsingleton}と推論法則 \ref{dedequiv}により
\[
  c \subset \{a\} \to c = \phi \vee c = \{a\}, ~~
  c \subset \{b\} \to c = \phi \vee c = \{b\}
\]
が共に成り立つから, この前者と(\ref{sthmsubsetofuopair4}), 後者と(\ref{sthmsubsetofuopair3})から, 
それぞれ推論法則 \ref{dedmmp}によって
\[
  c \subset \{a, b\} \wedge b \notin c \to c = \phi \vee c = \{a\}, ~~
  c \subset \{a, b\} \wedge a \notin c \to c = \phi \vee c = \{b\}
\]
が成り立つ.
故に推論法則 \ref{dedfromaddv}により
\begin{equation}
\label{sthmsubsetofuopair23}
  (c \subset \{a, b\} \wedge b \notin c) \vee (c \subset \{a, b\} \wedge a \notin c) 
  \to (c = \phi \vee c = \{a\}) \vee (c = \phi \vee c = \{b\})
\end{equation}
が成り立つ.
またThm \ref{1avb1v1avc1tav1bvc1}より
\begin{equation}
\label{sthmsubsetofuopair24}
  (c = \phi \vee c = \{a\}) \vee (c = \phi \vee c = \{b\}) \to c = \phi \vee (c = \{a\} \vee c = \{b\})
\end{equation}
が成り立つ.
またThm \ref{av1bvc1t1avb1vc}より
\begin{equation}
\label{sthmsubsetofuopair25}
  c = \phi \vee (c = \{a\} \vee c = \{b\}) \to c = \phi \vee c = \{a\} \vee c = \{b\}
\end{equation}
が成り立つ.
そこで(\ref{sthmsubsetofuopair18}), (\ref{sthmsubsetofuopair21})---(\ref{sthmsubsetofuopair25})から, 
推論法則 \ref{dedmmp}によって
\[
  c \subset \{a, b\} \wedge c \neq \{a, b\} \to c = \phi \vee c = \{a\} \vee c = \{b\}
\]
が成り立つことがわかる.
故に推論法則 \ref{dedtwch}により, 
\[
  c \subset \{a, b\} \to (c \neq \{a, b\} \to c = \phi \vee c = \{a\} \vee c = \{b\}), 
\]
即ち
\begin{equation}
\label{sthmsubsetofuopair26}
  c \subset \{a, b\} \to c = \{a, b\} \vee (c = \phi \vee c = \{a\} \vee c = \{b\})
\end{equation}
が成り立つ.
またThm \ref{avbtbva}より
\begin{equation}
\label{sthmsubsetofuopair27}
  c = \{a, b\} \vee (c = \phi \vee c = \{a\} \vee c = \{b\}) 
  \to c = \phi \vee c = \{a\} \vee c = \{b\} \vee c = \{a, b\}
\end{equation}
が成り立つ.
そこで(\ref{sthmsubsetofuopair26}), (\ref{sthmsubsetofuopair27})から, 推論法則 \ref{dedmmp}によって
\begin{equation}
\label{sthmsubsetofuopair28}
  c \subset \{a, b\} \to c = \phi \vee c = \{a\} \vee c = \{b\} \vee c = \{a, b\}
\end{equation}
が成り立つ.
また定理 \ref{sthmemptytsubset}より
\begin{equation}
\label{sthmsubsetofuopair29}
  c = \phi \to c \subset \{a, b\}
\end{equation}
が成り立つ.
また定理 \ref{sthm=tsubseteq}より
\begin{align}
  \label{sthmsubsetofuopair30}
  &c = \{a\} \to (c \subset \{a, b\} \leftrightarrow \{a\} \subset \{a, b\}), \\
  \mbox{} \notag \\
  \label{sthmsubsetofuopair31}
  &c = \{b\} \to (c \subset \{a, b\} \leftrightarrow \{b\} \subset \{a, b\})
\end{align}
が共に成り立つ.
また定理 \ref{sthmsingletonsubsetuopair}より
\[
  \{a\} \subset \{a, b\}, ~~
  \{b\} \subset \{a, b\}
\]
が共に成り立つから, 推論法則 \ref{ded1alb1tbtrue2}により
\begin{align}
  \label{sthmsubsetofuopair32}
  &(c \subset \{a, b\} \leftrightarrow \{a\} \subset \{a, b\}) \to c \subset \{a, b\}, \\
  \mbox{} \notag \\
  \label{sthmsubsetofuopair33}
  &(c \subset \{a, b\} \leftrightarrow \{b\} \subset \{a, b\}) \to c \subset \{a, b\}
\end{align}
が共に成り立つ.
そこで(\ref{sthmsubsetofuopair30})と(\ref{sthmsubsetofuopair32}), 
(\ref{sthmsubsetofuopair31})と(\ref{sthmsubsetofuopair33})から, それぞれ推論法則 \ref{dedmmp}によって
\begin{align}
  \label{sthmsubsetofuopair34}
  &c = \{a\} \to c \subset \{a, b\}, \\
  \mbox{} \notag \\
  \label{sthmsubsetofuopair35}
  &c = \{b\} \to c \subset \{a, b\}
\end{align}
が成り立つ.
また定理 \ref{sthm=tsubset}より
\begin{equation}
\label{sthmsubsetofuopair36}
  c = \{a, b\} \to c \subset \{a, b\}
\end{equation}
が成り立つ.
そこで(\ref{sthmsubsetofuopair29}), (\ref{sthmsubsetofuopair34})---(\ref{sthmsubsetofuopair36})から, 
推論法則 \ref{dedgdil}により
\begin{equation}
\label{sthmsubsetofuopair37}
  c = \phi \vee c = \{a\} \vee c = \{b\} \vee c = \{a, b\} \to c \subset \{a, b\}
\end{equation}
が成り立つ.
故に(\ref{sthmsubsetofuopair28}), (\ref{sthmsubsetofuopair37})から, 
推論法則 \ref{dedequiv}により(\ref{sthmsubsetofuopair1})が成り立つ.
(\ref{sthmsubsetofuopair2})が成り立つことは, 
(\ref{sthmsubsetofuopair1})と推論法則 \ref{dedeqfund}によって明らかである.
\halmos




\mathstrut
\begin{thm}
\label{sthmssetempty}%定理6.70%新規%確認済
$a$を集合, $R$を関係式とし, $x$を$a$の中に自由変数として現れない文字とする.
このとき
\begin{align}
  \label{sthmssetempty1}
  &\neg (\exists x \in a)(R) \leftrightarrow \{x \in a \mid R\} = \phi, \\
  \mbox{} \notag \\
  \label{sthmssetempty2}
  &(\forall x \in a)(\neg R) \leftrightarrow \{x \in a \mid R\} = \phi
\end{align}
が共に成り立つ.
またこれらから, 次の1)---4)が成り立つ.

1)
$\neg (\exists x \in a)(R)$ならば, $\{x \in a \mid R\}$は空である.

2)
$(\forall x \in a)(\neg R)$ならば, $\{x \in a \mid R\}$は空である.

3)
$x$が定数でなく, $x \in a \to \neg R$が成り立てば, $\{x \in a \mid R\}$は空である.

4)
$\{x \in a \mid R\}$が空ならば, $\neg (\exists x \in a)(R)$と$(\forall x \in a)(\neg R)$が共に成り立つ.
\end{thm}


\noindent{\bf 証明}
~定理 \ref{sthmisetempty}より, 
\[
  \neg \exists x(x \in a \wedge R) 
  \leftrightarrow {\rm Set}_{x}(x \in a \wedge R) \wedge \{x \mid x \in a \wedge R\} = \phi, 
\]
即ち
\begin{equation}
\label{sthmssetempty3}
  \neg (\exists x \in a)(R) 
  \leftrightarrow {\rm Set}_{x}(x \in a \wedge R) \wedge \{x \in a \mid R\} = \phi
\end{equation}
が成り立つ.
また$x$が$a$の中に自由変数として現れないことから, 
定理 \ref{sthmssetsm}より$x \in a \wedge R$は$x$について集合を作り得るから, 
推論法則 \ref{dedawblatrue2}により
\begin{equation}
\label{sthmssetempty4}
  {\rm Set}_{x}(x \in a \wedge R) \wedge \{x \in a \mid R\} = \phi 
  \leftrightarrow \{x \in a \mid R\} = \phi
\end{equation}
が成り立つ.
そこで(\ref{sthmssetempty3}), (\ref{sthmssetempty4})から, 
推論法則 \ref{dedeqtrans}によって(\ref{sthmssetempty1})が成り立つ.
またThm \ref{thmeaspquandm}と推論法則 \ref{dedeqch}により
\[
  (\forall x \in a)(\neg R) \leftrightarrow \neg (\exists x \in a)(R)
\]
が成り立つから, これと(\ref{sthmssetempty1})から, 
推論法則 \ref{dedeqtrans}によって(\ref{sthmssetempty2})が成り立つ.

\noindent
1)
(\ref{sthmssetempty1})と推論法則 \ref{dedeqfund}によって明らか.

\noindent
2)
(\ref{sthmssetempty2})と推論法則 \ref{dedeqfund}によって明らか.

\noindent
3)
2)と推論法則 \ref{dedspallfund}によって明らか.

\noindent
4)
(\ref{sthmssetempty1}), (\ref{sthmssetempty2})と推論法則 \ref{dedeqfund}によって明らか.
\halmos




\mathstrut
\begin{thm}
\label{sthmssetemptyt}%定理6.71%新規%確認済
$a$を集合, $R$を関係式とし, $x$を$a$の中に自由変数として現れない文字とする.
このとき
\begin{align}
  \label{sthmssetemptyt1}
  &\neg \exists x(R) \to \{x \in a \mid R\} = \phi, \\
  \mbox{} \notag \\
  \label{sthmssetemptyt2}
  &\forall x(\neg R) \to \{x \in a \mid R\} = \phi
\end{align}
が共に成り立つ.
またこれらから, 次の1), 2), 3)が成り立つ.

1)
$\neg \exists x(R)$ならば, $\{x \in a \mid R\}$は空である.

2)
$\forall x(\neg R)$ならば, $\{x \in a \mid R\}$は空である.

3)
$x$が定数でなく, $\neg R$が成り立てば, $\{x \in a \mid R\}$は空である.
\end{thm}


\noindent{\bf 証明}
~Thm \ref{thmspexneg}より
\begin{align}
  \label{sthmssetemptyt3}
  &\neg \exists x(R) \to \neg (\exists x \in a)(R), \\
  \mbox{} \notag \\
  \label{sthmssetemptyt4}
  &\forall x(\neg R) \to \neg (\exists x \in a)(R)
\end{align}
が共に成り立つ.
また$x$が$a$の中に自由変数として現れないことから, 
定理 \ref{sthmssetempty}と推論法則 \ref{dedequiv}により
\begin{equation}
\label{sthmssetemptyt5}
  \neg (\exists x \in a)(R) \to \{x \in a \mid R\} = \phi
\end{equation}
が成り立つ.
そこで(\ref{sthmssetemptyt3})と(\ref{sthmssetemptyt5}), 
(\ref{sthmssetemptyt4})と(\ref{sthmssetemptyt5})から, それぞれ推論法則 \ref{dedmmp}によって
(\ref{sthmssetemptyt1}), (\ref{sthmssetemptyt2})が成り立つ.

\noindent
1)
(\ref{sthmssetemptyt1})と推論法則 \ref{dedmp}によって明らか.

\noindent
2)
(\ref{sthmssetemptyt2})と推論法則 \ref{dedmp}によって明らか.

\noindent
3)
2)と推論法則 \ref{dedltthmquan}によって明らか.
\halmos




\mathstrut
\begin{thm}
\label{sthmaemptytssetempty}%定理6.72%新規%確認済
$a$を集合, $R$を関係式とし, $x$を$a$の中に自由変数として現れない文字とする.
このとき
\begin{equation}
\label{sthmaemptytssetempty1}
  a = \phi \to \{x \in a \mid R\} = \phi
\end{equation}
が成り立つ.
またこのことから, 次の(\ref{sthmaemptytssetempty2})が成り立つ.
\begin{equation}
\label{sthmaemptytssetempty2}
  a \text{が空ならば,} ~\{x \in a \mid R\} \text{は空である.}
\end{equation}
\end{thm}


\noindent{\bf 証明}
~$x$が$a$の中に自由変数として現れないことから, 定理 \ref{sthmemptytspin}より
\begin{equation}
\label{sthmaemptytssetempty3}
  a = \phi \to \neg (\exists x \in a)(R)
\end{equation}
が成り立つ.
同じく$x$が$a$の中に自由変数として現れないことから, 
定理 \ref{sthmssetempty}と推論法則 \ref{dedequiv}により
\begin{equation}
\label{sthmaemptytssetempty4}
  \neg (\exists x \in a)(R) \to \{x \in a \mid R\} = \phi
\end{equation}
が成り立つ.
そこで(\ref{sthmaemptytssetempty3}), (\ref{sthmaemptytssetempty4})から, 
推論法則 \ref{dedmmp}によって(\ref{sthmaemptytssetempty1})が成り立つ.
(\ref{sthmaemptytssetempty2})が成り立つことは, 
(\ref{sthmaemptytssetempty1})と推論法則 \ref{dedmp}によって明らかである.
\halmos




\mathstrut
\begin{thm}
\label{sthmemptyssetempty}%定理6.73%新規%確認済
$R$を関係式とし, $x$を文字とするとき, $\{x \in \phi \mid R\}$は空である.
\end{thm}


\noindent{\bf 証明}
~変数法則 \ref{valempty}より$x$は$\phi$の中に自由変数として現れない.
またThm \ref{x=x}より$\phi$は空である.
故に定理 \ref{sthmaemptytssetempty}より, $\{x \in \phi \mid R\}$は空である.
\halmos




\mathstrut
\begin{thm}
\label{sthmsset=arfreeeq}%定理6.74%新規%確認済
$a$を集合, $R$を関係式とし, $x$をこれらの中に自由変数として現れない文字とする.
このとき
\begin{equation}
\label{sthmsset=arfreeeq1}
  a = \phi \vee R \leftrightarrow \{x \in a \mid R\} = a
\end{equation}
が成り立つ.
またこのことから, 次の1), 2)が成り立つ.

1)
$a = \phi \vee R$ならば, $\{x \in a \mid R\} = a$.

2)
$\{x \in a \mid R\} = a$ならば, $a = \phi \vee R$.
\end{thm}


\noindent{\bf 証明}
~$x$が$a$の中に自由変数として現れないことから, 定理 \ref{sthmnotemptyeqexin}より
\[
  a \neq \phi \leftrightarrow \exists x(x \in a)
\]
が成り立つ.
故に推論法則 \ref{dedaddeqt}により, 
\[
  (a \neq \phi \to R) \leftrightarrow (\exists x(x \in a) \to R), 
\]
即ち
\begin{equation}
\label{sthmsset=arfreeeq2}
  a = \phi \vee R \leftrightarrow (\exists x(x \in a) \to R)
\end{equation}
が成り立つ.
また$x$が$R$の中に自由変数として現れないことから, 
Thm \ref{thmspquanrfree}と推論法則 \ref{dedeqch}により
\begin{equation}
\label{sthmsset=arfreeeq3}
  (\exists x(x \in a) \to R) \leftrightarrow (\forall x \in a)(R)
\end{equation}
が成り立つ.
また$x$が$a$の中に自由変数として現れないことから, 定理 \ref{sthmsset=a}より
\begin{equation}
\label{sthmsset=arfreeeq4}
  (\forall x \in a)(R) \leftrightarrow \{x \in a \mid R\} = a
\end{equation}
が成り立つ.
そこで(\ref{sthmsset=arfreeeq2})---(\ref{sthmsset=arfreeeq4})から, 
推論法則 \ref{dedeqtrans}によって(\ref{sthmsset=arfreeeq1})が成り立つことがわかる.
1), 2)が成り立つことは, (\ref{sthmsset=arfreeeq1})と推論法則 \ref{dedeqfund}によって明らかである.
\halmos




\mathstrut
\begin{thm}
\label{sthmssetemptyrfree}%定理6.75%新規%確認済
$a$を集合, $R$を関係式とし, $x$をこれらの中に自由変数として現れない文字とする.
このとき
\begin{equation}
\label{sthmssetemptyrfree1}
  a = \phi \vee \neg R \leftrightarrow \{x \in a \mid R\} = \phi
\end{equation}
が成り立つ.
特に, 
\begin{equation}
\label{sthmssetemptyrfree2}
  \neg R \to \{x \in a \mid R\} = \phi
\end{equation}
が成り立つ.
またこれらから, 次の1), 2)が成り立つ.

1)
$\neg R$ならば, $\{x \in a \mid R\}$は空である.

2)
$\{x \in a \mid R\}$が空ならば, $a = \phi \vee \neg R$.
\end{thm}


\noindent{\bf 証明}
~$x$が$a$の中に自由変数として現れないことから, 定理 \ref{sthmnotemptyeqexin}より
\[
  a \neq \phi \leftrightarrow \exists x(x \in a)
\]
が成り立つ.
故に推論法則 \ref{dedaddeqt}により, 
\[
  (a \neq \phi \to \neg R) \leftrightarrow (\exists x(x \in a) \to \neg R), 
\]
即ち
\begin{equation}
\label{sthmssetemptyrfree3}
  a = \phi \vee \neg R \leftrightarrow (\exists x(x \in a) \to \neg R)
\end{equation}
が成り立つ.
また$x$が$R$の中に自由変数として現れないことから, 
変数法則 \ref{valfund}により$x$は$\neg R$の中に自由変数として現れないから, 
Thm \ref{thmspquanrfree}と推論法則 \ref{dedeqch}により
\begin{equation}
\label{sthmssetemptyrfree4}
  (\exists x(x \in a) \to \neg R) \leftrightarrow (\forall x \in a)(\neg R)
\end{equation}
が成り立つ.
また$x$が$a$の中に自由変数として現れないことから, 定理 \ref{sthmssetempty}より
\begin{equation}
\label{sthmssetemptyrfree5}
  (\forall x \in a)(\neg R) \leftrightarrow \{x \in a \mid R\} = \phi
\end{equation}
が成り立つ.
そこで(\ref{sthmssetemptyrfree3})---(\ref{sthmssetemptyrfree5})から, 
推論法則 \ref{dedeqtrans}によって(\ref{sthmssetemptyrfree1})が成り立つことがわかる.
また(\ref{sthmssetemptyrfree1})と推論法則 \ref{dedequiv}により
\[
  a = \phi \vee \neg R \to \{x \in a \mid R\} = \phi
\]
が成り立つから, 推論法則 \ref{deddil}により(\ref{sthmssetemptyrfree2})が成り立つ.

\noindent
1)
(\ref{sthmssetemptyrfree2})と推論法則 \ref{dedmp}によって明らか.

\noindent
2)
(\ref{sthmssetemptyrfree1})と推論法則 \ref{dedeqfund}によって明らか.
\halmos




\mathstrut
\begin{thm}
\label{sthmosetempty}%定理6.76%新規%確認済
$a$と$T$を集合とし, $x$を$a$の中に自由変数として現れない文字とする.
このとき
\begin{equation}
\label{sthmosetempty1}
  \{T\}_{x \in a} = \phi \leftrightarrow a = \phi
\end{equation}
が成り立つ.
またこのことから, 次の1), 2)が成り立つ.

1)
$\{T\}_{x \in a}$が空ならば, $a$は空である.

2)
$a$が空ならば, $\{T\}_{x \in a}$は空である.
\end{thm}


\noindent{\bf 証明}
~$y$を$x$と異なり, $a$, $T$の中に自由変数として現れない, 定数でない文字とする.
このとき変数法則 \ref{valoset}により, $y$は$\{T\}_{x \in a}$の中に自由変数として現れないから, 
定理 \ref{sthmnotemptyeqexin}より
\begin{equation}
\label{sthmosetempty2}
  \{T\}_{x \in a} \neq \phi \leftrightarrow \exists y(y \in \{T\}_{x \in a})
\end{equation}
が成り立つ.
また$x$が$y$と異なり, $a$の中に自由変数として現れないことから, 定理 \ref{sthmosetbasis}より
\[
  y \in \{T\}_{x \in a} \leftrightarrow \exists x(x \in a \wedge y = T)
\]
が成り立つ.
このことと$y$が定数でないことから, 推論法則 \ref{dedalleqquansepconst}により
\begin{equation}
\label{sthmosetempty3}
  \exists y(y \in \{T\}_{x \in a}) \leftrightarrow \exists y(\exists x(x \in a \wedge y = T))
\end{equation}
が成り立つ.
また$y$が$x$と異なり, $a$の中に自由変数として現れないことから, 
変数法則 \ref{valfund}により$y$は$x \in a$の中に自由変数として現れないから, Thm \ref{thmexspexch}より, 
\[
  \exists y((\exists x \in a)(y = T)) \leftrightarrow (\exists x \in a)(\exists y(y = T)), 
\]
即ち
\begin{equation}
\label{sthmosetempty4}
  \exists y(\exists x(x \in a \wedge y = T)) \leftrightarrow \exists x(x \in a \wedge \exists y(y = T))
\end{equation}
が成り立つ.
さていま$z$を$x$, $y$と異なり, $a$, $T$の中に自由変数として現れない, 定数でない文字とする.
このとき$y$が$z$と異なり, $T$の中に自由変数として現れないことから, 
変数法則 \ref{valsubst}により$y$は$(z|x)(T)$の中に自由変数として現れないから, Thm \ref{thmex=}より
\[
  \exists y(y = (z|x)(T))
\]
が成り立つ.
故に推論法則 \ref{dedawblatrue2}により
\[
  (z|x)(x \in a) \wedge \exists y(y = (z|x)(T)) \leftrightarrow (z|x)(x \in a)
\]
が成り立つ.
ここで$y$が$x$, $z$と異なることから, 
代入法則 \ref{substfund}, \ref{substwedge}, \ref{substequiv}, \ref{substquan}によってわかるように, 
この記号列は
\[
  (z|x)(x \in a \wedge \exists y(y = T) \leftrightarrow x \in a)
\]
と一致する.
故にこれが成り立つ.
このことと$z$が定数でないことから, 推論法則 \ref{dedltthmquan}により
\[
  \forall z((z|x)(x \in a \wedge \exists y(y = T) \leftrightarrow x \in a))
\]
が成り立つ.
ここで$z$が$x$, $y$と異なり, $a$, $T$の中に自由変数として現れないことから, 
変数法則 \ref{valfund}, \ref{valwedge}, \ref{valequiv}, \ref{valquan}によってわかるように, 
$z$は$x \in a \wedge \exists y(y = T) \leftrightarrow x \in a$の中に自由変数として現れない.
故に代入法則 \ref{substquantrans}により, 上記の記号列は
\[
  \forall x(x \in a \wedge \exists y(y = T) \leftrightarrow x \in a)
\]
と一致する.
従ってこれが成り立つ.
そこで推論法則 \ref{dedalleqquansep}により
\begin{equation}
\label{sthmosetempty5}
  \exists x(x \in a \wedge \exists y(y = T)) \leftrightarrow \exists x(x \in a)
\end{equation}
が成り立つ.
また$x$が$a$の中に自由変数として現れないことから, 
定理 \ref{sthmnotemptyeqexin}と推論法則 \ref{dedeqch}により
\begin{equation}
\label{sthmosetempty6}
  \exists x(x \in a) \leftrightarrow a \neq \phi
\end{equation}
が成り立つ.
そこで(\ref{sthmosetempty2})---(\ref{sthmosetempty6})から, 推論法則 \ref{dedeqtrans}によって
\[
  \{T\}_{x \in a} \neq \phi \leftrightarrow a \neq \phi
\]
が成り立つことがわかる.
故に推論法則 \ref{dedeqcp}により(\ref{sthmosetempty1})が成り立つ.
1), 2)が成り立つことは, (\ref{sthmosetempty1})と推論法則 \ref{dedeqfund}によって明らかである.
\halmos




\mathstrut
\begin{thm}
\label{sthmemptyosetempty}%定理6.77%新規%確認済
$T$を集合とし, $x$を文字とするとき, $\{T\}_{x \in \phi}$は空である.
\end{thm}


\noindent{\bf 証明}
~変数法則 \ref{valempty}より$x$は$\phi$の中に自由変数として現れない.
またThm \ref{x=x}より$\phi$は空である.
故に定理 \ref{sthmosetempty}より, $\{T\}_{x \in \phi}$は空である.
\halmos




\mathstrut
\begin{thm}
\label{sthmoset=singleton}%定理6.78%新規%確認済
$a$, $b$, $T$を集合とし, $x$を$a$及び$b$の中に自由変数として現れない文字とする.
このとき
\begin{equation}
\label{sthmoset=singleton1}
  a \neq \phi \wedge (\forall x \in a)(T = b) \leftrightarrow \{T\}_{x \in a} = \{b\}
\end{equation}
が成り立つ.
またこのことから, 次の1), 2), 3)が成り立つ.

1)
$a$が空でなく, $(\forall x \in a)(T = b)$が成り立てば, $\{T\}_{x \in a} = \{b\}$.

2)
$a$は空でないとする.
また$x$が定数でなく, $x \in a \to T = b$が成り立つとする.
このとき$\{T\}_{x \in a} = \{b\}$.

3)
$\{T\}_{x \in a} = \{b\}$ならば, $a$は空でなく, $(\forall x \in a)(T = b)$が成り立つ.
\end{thm}


\noindent{\bf 証明}
~$y$を$b$, $T$の中に自由変数として現れない, 定数でない文字とする.
このとき定理 \ref{sthmsingletonbasis}と推論法則 \ref{dedeqch}により
\[
  (y|x)(T) = b \leftrightarrow (y|x)(T) \in \{b\}
\]
が成り立つ.
ここで$x$が$b$の中に自由変数として現れないことから, 
変数法則 \ref{valnset}により$x$は$\{b\}$の中にも自由変数として現れないから, 
代入法則 \ref{substfree}, \ref{substfund}, \ref{substequiv}により, この記号列は
\[
  (y|x)(T = b \leftrightarrow T \in \{b\})
\]
と一致する.
故にこれが成り立つ.
このことと$y$が定数でないことから, 推論法則 \ref{dedltthmquan}により
\[
  \forall y((y|x)(T = b \leftrightarrow T \in \{b\}))
\]
が成り立つ.
ここで$y$が$b$, $T$の中に自由変数として現れないことから, 
変数法則 \ref{valfund}, \ref{valequiv}, \ref{valnset}によってわかるように, 
$y$は$T = b \leftrightarrow T \in \{b\}$の中に自由変数として現れない.
故に代入法則 \ref{substquantrans}により, 上記の記号列は
\[
  \forall x(T = b \leftrightarrow T \in \{b\})
\]
と一致する.
従ってこれが成り立つ.
そこで推論法則 \ref{dedalleqspquansep}により
\begin{equation}
\label{sthmoset=singleton2}
  (\forall x \in a)(T = b) \leftrightarrow (\forall x \in a)(T \in \{b\})
\end{equation}
が成り立つ.
また$x$が$a$の中に自由変数として現れず, 上述のように$\{b\}$の中にも自由変数として現れないことから, 
定理 \ref{sthmosetsubsetb}より
\begin{equation}
\label{sthmoset=singleton3}
  (\forall x \in a)(T \in \{b\}) \leftrightarrow \{T\}_{x \in a} \subset \{b\}
\end{equation}
が成り立つ.
また定理 \ref{sthmsubsetofsingleton}より
\begin{equation}
\label{sthmoset=singleton4}
  \{T\}_{x \in a} \subset \{b\} \leftrightarrow \{T\}_{x \in a} = \phi \vee \{T\}_{x \in a} = \{b\}
\end{equation}
が成り立つ.
また$x$が$a$の中に自由変数として現れないことから, 定理 \ref{sthmosetempty}より
\begin{equation}
\label{sthmoset=singleton5}
  \{T\}_{x \in a} = \phi \leftrightarrow a = \phi
\end{equation}
が成り立つ.
故に推論法則 \ref{dedaddeqv}により
\begin{equation}
\label{sthmoset=singleton6}
  \{T\}_{x \in a} = \phi \vee \{T\}_{x \in a} = \{b\} 
  \leftrightarrow a = \phi \vee \{T\}_{x \in a} = \{b\}
\end{equation}
が成り立つ.
そこで(\ref{sthmoset=singleton2})---(\ref{sthmoset=singleton4}), (\ref{sthmoset=singleton6})から, 
推論法則 \ref{dedeqtrans}によって
\begin{equation}
\label{sthmoset=singleton7}
  (\forall x \in a)(T = b) \leftrightarrow a = \phi \vee \{T\}_{x \in a} = \{b\}
\end{equation}
が成り立つことがわかる.
故に推論法則 \ref{dedequiv}により, 
\[
  (\forall x \in a)(T = b) \to a = \phi \vee \{T\}_{x \in a} = \{b\}, 
\]
即ち
\[
  (\forall x \in a)(T = b) \to (a \neq \phi \to \{T\}_{x \in a} = \{b\})
\]
が成り立つ.
故に推論法則 \ref{dedch}により
\[
  a \neq \phi \to ((\forall x \in a)(T = b) \to \{T\}_{x \in a} = \{b\})
\]
が成り立つ.
故に推論法則 \ref{dedtwch}により
\begin{equation}
\label{sthmoset=singleton8}
  a \neq \phi \wedge (\forall x \in a)(T = b) \to \{T\}_{x \in a} = \{b\}
\end{equation}
が成り立つ.
また定理 \ref{sthm=tineq}より
\begin{equation}
\label{sthmoset=singleton9}
  \{T\}_{x \in a} = \{b\} \to (b \in \{T\}_{x \in a} \leftrightarrow b \in \{b\})
\end{equation}
が成り立つ.
また定理 \ref{sthmsingletonfund}より$b \in \{b\}$が成り立つから, 推論法則 \ref{ded1alb1tbtrue2}により
\begin{equation}
\label{sthmoset=singleton10}
  (b \in \{T\}_{x \in a} \leftrightarrow b \in \{b\}) \to b \in \{T\}_{x \in a}
\end{equation}
が成り立つ.
また定理 \ref{sthmintnotempty}より
\begin{equation}
\label{sthmoset=singleton11}
  b \in \{T\}_{x \in a} \to \{T\}_{x \in a} \neq \phi
\end{equation}
が成り立つ.
また(\ref{sthmoset=singleton5})から, 推論法則 \ref{dedeqcp}により
\[
  \{T\}_{x \in a} \neq \phi \leftrightarrow a \neq \phi
\]
が成り立つから, 推論法則 \ref{dedequiv}により
\begin{equation}
\label{sthmoset=singleton12}
  \{T\}_{x \in a} \neq \phi \to a \neq \phi
\end{equation}
が成り立つ.
そこで(\ref{sthmoset=singleton9})---(\ref{sthmoset=singleton12})から, 推論法則 \ref{dedmmp}によって
\begin{equation}
\label{sthmoset=singleton13}
  \{T\}_{x \in a} = \{b\} \to a \neq \phi
\end{equation}
が成り立つことがわかる.
また(\ref{sthmoset=singleton7})から, 推論法則 \ref{dedequiv}により
\[
  a = \phi \vee \{T\}_{x \in a} = \{b\} \to (\forall x \in a)(T = b)
\]
が成り立つから, 推論法則 \ref{deddil}により
\begin{equation}
\label{sthmoset=singleton14}
  \{T\}_{x \in a} = \{b\} \to (\forall x \in a)(T = b)
\end{equation}
が成り立つ.
そこで(\ref{sthmoset=singleton13}), (\ref{sthmoset=singleton14})から, 推論法則 \ref{dedprewedge}により
\begin{equation}
\label{sthmoset=singleton15}
  \{T\}_{x \in a} = \{b\} \to a \neq \phi \wedge (\forall x \in a)(T = b)
\end{equation}
が成り立つ.
故に(\ref{sthmoset=singleton8}), (\ref{sthmoset=singleton15})から, 
推論法則 \ref{dedequiv}により(\ref{sthmoset=singleton1})が成り立つ.

\noindent
1), 3)
(\ref{sthmoset=singleton1})と推論法則 \ref{dedwedge}, \ref{dedeqfund}によって明らか.

\noindent
2)
1)と推論法則 \ref{dedspallfund}によって明らか.
\halmos




\mathstrut
\begin{thm}
\label{sthmoset=singletont}%定理6.79%新規%確認済
$a$, $b$, $T$を集合とし, $x$を$a$及び$b$の中に自由変数として現れない文字とする.
このとき
\begin{equation}
\label{sthmoset=singletont1}
  a \neq \phi \wedge \forall x(T = b) \to \{T\}_{x \in a} = \{b\}
\end{equation}
が成り立つ.
またこのことから, 次の1), 2)が成り立つ.

1)
$a$が空でなく, $\forall x(T = b)$が成り立てば, $\{T\}_{x \in a} = \{b\}$.

2)
$a$は空でないとする.
また$x$が定数でなく, $T = b$が成り立つとする.
このとき$\{T\}_{x \in a} = \{b\}$.
\end{thm}


\noindent{\bf 証明}
~Thm \ref{thmquantspall2}より
\[
  \forall x(T = b) \to (\forall x \in a)(T = b)
\]
が成り立つから, 推論法則 \ref{dedaddw}により
\begin{equation}
\label{sthmoset=singletont2}
  a \neq \phi \wedge \forall x(T = b) \to a \neq \phi \wedge (\forall x \in a)(T = b)
\end{equation}
が成り立つ.
また$x$が$a$, $b$の中に自由変数として現れないことから, 
定理 \ref{sthmoset=singleton}と推論法則 \ref{dedequiv}により
\begin{equation}
\label{sthmoset=singletont3}
  a \neq \phi \wedge (\forall x \in a)(T = b) \to \{T\}_{x \in a} = \{b\}
\end{equation}
が成り立つ.
そこで(\ref{sthmoset=singletont2}), (\ref{sthmoset=singletont3})から, 
推論法則 \ref{dedmmp}によって(\ref{sthmoset=singletont1})が成り立つ.

\noindent
1)
(\ref{sthmoset=singletont1})と推論法則 \ref{dedmp}, \ref{dedwedge}によって明らか.

\noindent
2)
1)と推論法則 \ref{dedltthmquan}によって明らか.
\halmos




\mathstrut
\begin{thm}
\label{sthmoset=singletontfree}%定理6.80%新規%確認済
$a$と$T$を集合とし, $x$をこれらの中に自由変数として現れない文字とする.
このとき
\begin{equation}
\label{sthmoset=singletontfree1}
  a \neq \phi \leftrightarrow \{T\}_{x \in a} = \{T\}
\end{equation}
が成り立つ.
またこのことから特に, 次の(\ref{sthmoset=singletontfree2})が成り立つ.
\begin{equation}
\label{sthmoset=singletontfree2}
  a \text{が空でなければ,} ~\{T\}_{x \in a} = \{T\}.
\end{equation}
\end{thm}


\noindent{\bf 証明}
~Thm \ref{x=x}より$T = T$が成り立つが, $x$が$T$の中に自由変数として現れないことから, 
変数法則 \ref{valfund}により$x$はこの記号列の中に自由変数として現れないから, 
推論法則 \ref{dedspquanrfree2}により$(\forall x \in a)(T = T)$が成り立つ.
故に推論法則 \ref{dedeqch}, \ref{dedawblatrue2}により
\begin{equation}
\label{sthmoset=singletontfree3}
  a \neq \phi \leftrightarrow a \neq \phi \wedge (\forall x \in a)(T = T)
\end{equation}
が成り立つ.
また$x$が$a$, $T$の中に自由変数として現れないことから, 定理 \ref{sthmoset=singleton}より
\begin{equation}
\label{sthmoset=singletontfree4}
  a \neq \phi \wedge (\forall x \in a)(T = T) \leftrightarrow \{T\}_{x \in a} = \{T\}
\end{equation}
が成り立つ.
そこで(\ref{sthmoset=singletontfree3}), (\ref{sthmoset=singletontfree4})から, 
推論法則 \ref{dedeqtrans}によって(\ref{sthmoset=singletontfree1})が成り立つ.
(\ref{sthmoset=singletontfree2})が成り立つことは, 
(\ref{sthmoset=singletontfree1})と推論法則 \ref{dedeqfund}によって明らかである.
\halmos




\mathstrut
\begin{thm}
\label{sthm-empty}%定理6.81%新規%確認済
$a$と$b$を集合とするとき, 
\begin{equation}
\label{sthm-empty1}
  a - b = \phi \leftrightarrow a \subset b
\end{equation}
が成り立つ.
またこのことから, 次の1), 2)が成り立つ.

1)
$a - b$が空ならば, $a \subset b$.

2)
$a \subset b$ならば, $a - b$は空である.
\end{thm}


\noindent{\bf 証明}
~$x$を$a$, $b$の中に自由変数として現れない文字とする.
このとき定理 \ref{sthmssetempty}と推論法則 \ref{dedeqch}により, 
\[
  \{x \in a \mid x \notin b\} = \phi \leftrightarrow \neg (\exists x \in a)(x \notin b), 
\]
即ち
\begin{equation}
\label{sthm-empty2}
  \{x \in a \mid x \notin b\} = \phi \leftrightarrow (\forall x \in a)(x \in b)
\end{equation}
が成り立つ.
またThm \ref{thmspallfund}より
\begin{equation}
\label{sthm-empty3}
  (\forall x \in a)(x \in b) \leftrightarrow \forall x(x \in a \to x \in b)
\end{equation}
が成り立つ.
そこで(\ref{sthm-empty2}), (\ref{sthm-empty3})から, 推論法則 \ref{dedeqtrans}によって
\[
  \{x \in a \mid x \notin b\} = \phi \leftrightarrow \forall x(x \in a \to x \in b)
\]
が成り立つ.
ここで$x$が$a$, $b$の中に自由変数として現れないことから, 
定義よりこの記号列は(\ref{sthm-empty1})と同じである.
故に(\ref{sthm-empty1})が成り立つ.
1), 2)が成り立つことは, (\ref{sthm-empty1})と推論法則 \ref{dedeqfund}によって明らかである.
\halmos




\mathstrut
\begin{thm}
\label{sthma-empty}%定理6.82%確認済
$a$を集合とするとき, 
\begin{align}
  \label{sthma-empty1}
  &a - a = \phi, \\
  \mbox{} \notag \\
  \label{sthma-empty2}
  &a - \phi = a, \\
  \mbox{} \notag \\
  \label{sthma-empty3}
  &\phi - a = \phi
\end{align}
がすべて成り立つ.
\end{thm}


\noindent{\bf 証明}
~定理 \ref{sthmsubsetself}より$a \subset a$が成り立つから, 
定理 \ref{sthm-empty}より(\ref{sthma-empty1})が成り立つ.
また定理 \ref{sthmemptysubset}より$\phi \subset a$が成り立つから, 
定理 \ref{sthm-empty}より(\ref{sthma-empty3})が成り立つ.
また(\ref{sthma-empty3})から, 定理 \ref{sthma-b=aeqb-a=b}より(\ref{sthma-empty2})が成り立つ.
\halmos




\mathstrut
\begin{thm}
\label{sthmbsubseta-b}%定理6.83%新規%確認済
$a$と$b$を集合とするとき, 
\begin{equation}
\label{sthmbsubseta-b1}
  b \subset a - b \leftrightarrow b = \phi
\end{equation}
が成り立つ.
またこのことから特に, 次の(\ref{sthmbsubseta-b2})が成り立つ.
\begin{equation}
\label{sthmbsubseta-b2}
  b \subset a - b \text{ならば,} ~b \text{は空である.}
\end{equation}
\end{thm}


\noindent{\bf 証明}
~定理 \ref{sthma-empty}と推論法則 \ref{ded=ch}により$b = b - \phi$が成り立つから, 
定理 \ref{sthm=tsubseteq}より
\begin{equation}
\label{sthmbsubseta-b3}
  b \subset a - b \leftrightarrow b - \phi \subset a - b
\end{equation}
が成り立つ.
また定理 \ref{sthma-bsubsetb-ceq}より
\begin{equation}
\label{sthmbsubseta-b4}
  b - \phi \subset a - b \leftrightarrow b \subset \phi
\end{equation}
が成り立つ.
また定理 \ref{sthmemptysubset=eq}より
\begin{equation}
\label{sthmbsubseta-b5}
  b \subset \phi \leftrightarrow b = \phi
\end{equation}
が成り立つ.
そこで(\ref{sthmbsubseta-b3})---(\ref{sthmbsubseta-b5})から, 
推論法則 \ref{dedeqtrans}によって(\ref{sthmbsubseta-b1})が成り立つことがわかる.
(\ref{sthmbsubseta-b2})が成り立つことは, 
(\ref{sthmbsubseta-b1})と推論法則 \ref{dedeqfund}によって明らかである.
\halmos




\mathstrut
\begin{thm}
\label{sthma-b=b}%定理6.84%新規%確認済
$a$と$b$を集合とするとき, 
\begin{equation}
\label{sthma-b=b1}
  a - b = b \leftrightarrow a = \phi \wedge b = \phi
\end{equation}
が成り立つ.
またこのことから, 次の1), 2)が成り立つ.

1)
$a - b = b$ならば, $a$と$b$は共に空である.

2)
$a$と$b$が共に空ならば, $a - b = b$.
\end{thm}


\noindent{\bf 証明}
~定理 \ref{sthmaxiom1}と推論法則 \ref{dedeqch}により
\begin{equation}
\label{sthma-b=b2}
  a - b = b \leftrightarrow a - b \subset b \wedge b \subset a - b
\end{equation}
が成り立つ.
また定理 \ref{sthma-bsubsetb}より
\begin{equation}
\label{sthma-b=b3}
  a - b \subset b \leftrightarrow a \subset b
\end{equation}
が成り立つ.
また定理 \ref{sthmbsubseta-b}より
\begin{equation}
\label{sthma-b=b4}
  b \subset a - b \leftrightarrow b = \phi
\end{equation}
が成り立つ.
そこで(\ref{sthma-b=b3}), (\ref{sthma-b=b4})から, 推論法則 \ref{dedaddeqw}により
\begin{equation}
\label{sthma-b=b5}
  a - b \subset b \wedge b \subset a - b \leftrightarrow a \subset b \wedge b = \phi
\end{equation}
が成り立つ.
また定理 \ref{sthm=tsubseteq}より
\[
  b = \phi \to (a \subset b \leftrightarrow a \subset \phi)
\]
が成り立つから, 推論法則 \ref{dedeq&w}により
\begin{equation}
\label{sthma-b=b6}
  a \subset b \wedge b = \phi \leftrightarrow a \subset \phi \wedge b = \phi
\end{equation}
が成り立つ.
また定理 \ref{sthmemptysubset=eq}より
\[
  a \subset \phi \leftrightarrow a = \phi
\]
が成り立つから, 推論法則 \ref{dedaddeqw}により
\begin{equation}
\label{sthma-b=b7}
  a \subset \phi \wedge b = \phi \leftrightarrow a = \phi \wedge b = \phi
\end{equation}
が成り立つ.
そこで(\ref{sthma-b=b2}), (\ref{sthma-b=b5})---(\ref{sthma-b=b7})から, 
推論法則 \ref{dedeqtrans}によって(\ref{sthma-b=b1})が成り立つことがわかる.
1), 2)が成り立つことは, (\ref{sthma-b=b1})と
推論法則 \ref{dedwedge}, \ref{dedeqfund}によって明らかである.
\halmos




\mathstrut
\begin{thm}
\label{sthma-bpsubsetb}%定理6.85%新規%要る?%確認済
$a$と$b$を集合とするとき, 
\begin{equation}
\label{sthma-bpsubsetb1}
  a - b \subsetneqq b \leftrightarrow a \subset b \wedge b \neq \phi
\end{equation}
が成り立つ.
またこのことから, 次の1), 2)が成り立つ.

1)
$a - b \subsetneqq b$ならば, $b$は空でなく, $a \subset b$が成り立つ.

2)
$b$が空でなく, $a \subset b$が成り立てば, $a - b \subsetneqq b$.
\end{thm}


\noindent{\bf 証明}
~定理 \ref{sthmpsubset&axiom1}より
\begin{equation}
\label{sthma-bpsubsetb2}
  a - b \subsetneqq b \leftrightarrow a - b \subset b \wedge b \not\subset a - b
\end{equation}
が成り立つ.
また定理 \ref{sthma-bsubsetb}より
\begin{equation}
\label{sthma-bpsubsetb3}
  a - b \subset b \leftrightarrow a \subset b
\end{equation}
が成り立つ.
また定理 \ref{sthmbsubseta-b}より
\[
  b \subset a - b \leftrightarrow b = \phi
\]
が成り立つから, 推論法則 \ref{dedeqcp}により
\begin{equation}
\label{sthma-bpsubsetb4}
  b \not\subset a - b \leftrightarrow b \neq \phi
\end{equation}
が成り立つ.
そこで(\ref{sthma-bpsubsetb3}), (\ref{sthma-bpsubsetb4})から, 推論法則 \ref{dedaddeqw}により
\begin{equation}
\label{sthma-bpsubsetb5}
  a - b \subset b \wedge b \not\subset a - b \leftrightarrow a \subset b \wedge b \neq \phi
\end{equation}
が成り立つ.
故に(\ref{sthma-bpsubsetb2}), (\ref{sthma-bpsubsetb5})から, 
推論法則 \ref{dedeqtrans}によって(\ref{sthma-bpsubsetb1})が成り立つ.
1), 2)が成り立つことは, (\ref{sthma-bpsubsetb1})と
推論法則 \ref{dedwedge}, \ref{dedeqfund}によって明らかである.
\halmos




\mathstrut
\begin{thm}
\label{sthmbpsubseta-b}%定理6.86%新規%要る?%確認済
$a$と$b$を集合とするとき, 
\begin{equation}
\label{sthmbpsubseta-b1}
  b \subsetneqq a - b \leftrightarrow a \neq \phi \wedge b = \phi
\end{equation}
が成り立つ.
またこのことから, 次の1), 2)が成り立つ.

1)
$b \subsetneqq a - b$ならば, $a$は空でなく, $b$は空である.

2)
$a$が空でなく, $b$が空ならば, $b \subsetneqq a - b$.
\end{thm}


\noindent{\bf 証明}
~定理 \ref{sthmbsubseta-b}より
\begin{equation}
\label{sthmbpsubseta-b2}
  b \subset a - b \leftrightarrow b = \phi
\end{equation}
が成り立つ.
またThm \ref{x=yly=x}より
\begin{equation}
\label{sthmbpsubseta-b3}
  b = a - b \leftrightarrow a - b = b
\end{equation}
が成り立つ.
また定理 \ref{sthma-b=b}より
\begin{equation}
\label{sthmbpsubseta-b4}
  a - b = b \leftrightarrow a = \phi \wedge b = \phi
\end{equation}
が成り立つ.
そこで(\ref{sthmbpsubseta-b3}), (\ref{sthmbpsubseta-b4})から, 推論法則 \ref{dedeqtrans}によって, 
\[
  b = a - b \leftrightarrow a = \phi \wedge b = \phi, 
\]
即ち
\[
  b = a - b \leftrightarrow \neg (a \neq \phi \vee b \neq \phi)
\]
が成り立つ.
故に推論法則 \ref{dedeqcp}により
\begin{equation}
\label{sthmbpsubseta-b5}
  b \neq a - b \leftrightarrow a \neq \phi \vee b \neq \phi
\end{equation}
が成り立つ.
そこで(\ref{sthmbpsubseta-b2}), (\ref{sthmbpsubseta-b5})から, 推論法則 \ref{dedaddeqw}により
\begin{equation}
\label{sthmbpsubseta-b6}
  b \subsetneqq a - b \leftrightarrow b = \phi \wedge (a \neq \phi \vee b \neq \phi)
\end{equation}
が成り立つ.
またThm \ref{aw1bvc1l1awb1v1awc1}より
\begin{equation}
\label{sthmbpsubseta-b7}
  b = \phi \wedge (a \neq \phi \vee b \neq \phi) 
  \leftrightarrow (b = \phi \wedge a \neq \phi) \vee (b = \phi \wedge b \neq \phi)
\end{equation}
が成り立つ.
またThm \ref{n1awna1}より$\neg (b = \phi \wedge b \neq \phi)$が成り立つから, 
推論法則 \ref{dedavblbtrue2}により
\begin{equation}
\label{sthmbpsubseta-b8}
  (b = \phi \wedge a \neq \phi) \vee (b = \phi \wedge b \neq \phi) 
  \leftrightarrow b = \phi \wedge a \neq \phi
\end{equation}
が成り立つ.
またThm \ref{awblbwa}より
\begin{equation}
\label{sthmbpsubseta-b9}
  b = \phi \wedge a \neq \phi \leftrightarrow a \neq \phi \wedge b = \phi
\end{equation}
が成り立つ.
そこで(\ref{sthmbpsubseta-b6})---(\ref{sthmbpsubseta-b9})から, 
推論法則 \ref{dedeqtrans}によって(\ref{sthmbpsubseta-b1})が成り立つことがわかる.
1), 2)が成り立つことは, (\ref{sthmbpsubseta-b1})と
推論法則 \ref{dedwedge}, \ref{dedeqfund}によって明らかである.
\halmos




\mathstrut
\begin{thm}
\label{sthmsingleton-empty}%定理6.87%新規%要る?%確認済
$a$と$b$を集合とするとき, 
\begin{equation}
\label{sthmsingleton-empty1}
  \{a\} - \{b\} = \phi \leftrightarrow a = b
\end{equation}
が成り立つ.
またこのことから, 次の1), 2)が成り立つ.

1)
$\{a\} - \{b\}$が空ならば, $a = b$.

2)
$a = b$ならば, $\{a\} - \{b\}$は空である.
\end{thm}


\noindent{\bf 証明}
~定理 \ref{sthm-empty}より
\begin{equation}
\label{sthmsingleton-empty2}
  \{a\} - \{b\} = \phi \leftrightarrow \{a\} \subset \{b\}
\end{equation}
が成り立つ.
また定理 \ref{sthmsingleton=subset}と推論法則 \ref{dedeqch}により
\begin{equation}
\label{sthmsingleton-empty3}
  \{a\} \subset \{b\} \leftrightarrow a = b
\end{equation}
が成り立つ.
そこで(\ref{sthmsingleton-empty2}), (\ref{sthmsingleton-empty3})から, 
推論法則 \ref{dedeqtrans}によって(\ref{sthmsingleton-empty1})が成り立つ.
1), 2)が成り立つことは, (\ref{sthmsingleton-empty1})と推論法則 \ref{dedeqfund}によって明らかである.
\halmos
%ここまで確認



\newpage
\setcounter{defi}{0}
\section{和集合と共通部分}



%確認済%koko
この節では, 表題の集合を定義し, それらの性質を述べる.




\mathstrut
\begin{defo}
\label{cup}%変形20%確認済
$\mathscr{T}$を特殊記号として$\in$を持つ理論とし, $a$と$b$を$\mathscr{T}$の記号列とする.
また$x$と$y$を共に$a$及び$b$の中に自由変数として現れない文字とする.
このとき
\[
  \{x \mid x \in a \vee x \in b\} \equiv \{y \mid y \in a \vee y \in b\}
\]
が成り立つ.
\end{defo}


\noindent{\bf 証明}
~$x$と$y$が同じ文字ならば明らかだから, 以下$x$と$y$は異なる文字であるとする.
このとき$y$が$x$と異なり, $a$, $b$の中に自由変数として現れないことから, 
変数法則 \ref{valfund}により$y$は$x \in a \vee x \in b$の中に自由変数として現れないから, 
代入法則 \ref{substisettrans}により
\[
  \{x \mid x \in a \vee x \in b\} \equiv \{y \mid (y|x)(x \in a \vee x \in b)\}
\]
が成り立つ.
また$x$が$a$, $b$の中に自由変数として現れないことから, 代入法則 \ref{substfree}, \ref{substfund}により
\[
  (y|x)(x \in a \vee x \in b) \equiv y \in a \vee y \in b
\]
が成り立つ.
故に本法則が成り立つ.
\halmos




\mathstrut
\begin{defi}
\label{defcup}%定義1%確認済
$\mathscr{T}$を特殊記号として$\in$を持つ理論とし, $a$と$b$を$\mathscr{T}$の記号列とする.
また$x$と$y$を共に$a$及び$b$の中に自由変数として現れない文字とする.
このとき変形法則 \ref{cup}によれば, 
$\{x \mid x \in a \vee x \in b\}$と$\{y \mid y \in a \vee y \in b\}$は同じ記号列となる.
$a$と$b$に対して定まるこの記号列を, $(a) \cup (b)$と書き表す (括弧は適宜省略する).
これを$a$と$b$の\textbf{集合論的和} (set-theoretic sum) あるいは単に\textbf{和} (sum), 
または\textbf{合併} (union), \textbf{結び} (join) などという.
\end{defi}




\mathstrut%確認済%koko
以下の変数法則 \ref{valcup}, 一般代入法則 \ref{gsubstcup}, 代入法則 \ref{substcup}, 
構成法則 \ref{formcup}では, $\mathscr{T}$を特殊記号として$\in$を持つ理論とし, 
これらの法則における``記号列'', ``集合''とは, 
それぞれ$\mathscr{T}$の記号列, $\mathscr{T}$の対象式のこととする.




\mathstrut
\begin{valu}
\label{valcup}%変数31%確認済
$a$と$b$を記号列とし, $x$を文字とする.
$x$が$a$及び$b$の中に自由変数として現れなければ, $x$は$a \cup b$の中に自由変数として現れない.
\end{valu}


\noindent{\bf 証明}
~このとき定義から$a \cup b$は$\{x \mid x \in a \vee x \in b\}$と同じである.
変数法則 \ref{valiset}によれば, $x$はこの中に自由変数として現れない.
\halmos




\mathstrut
\begin{gsub}
\label{gsubstcup}%一般代入34%新規%確認済
$a$と$b$を記号列とする.
また$n$を自然数とし, $T_{1}, T_{2}, \cdots, T_{n}$を記号列とする.
また$x_{1}, x_{2}, \cdots, x_{n}$を, どの二つも互いに異なる文字とする.
このとき
\[
  (T_{1}|x_{1}, T_{2}|x_{2}, \cdots, T_{n}|x_{n})(a \cup b) 
  \equiv (T_{1}|x_{1}, T_{2}|x_{2}, \cdots, T_{n}|x_{n})(a) \cup (T_{1}|x_{1}, T_{2}|x_{2}, \cdots, T_{n}|x_{n})(b)
\]
が成り立つ.
\end{gsub}


\noindent{\bf 証明}
~$y$を$x_{1}, x_{2}, \cdots, x_{n}$のいずれとも異なり, 
$a, b, T_{1}, T_{2}, \cdots, T_{n}$のいずれの中にも自由変数として現れない文字とする.
このとき定義から$a \cup b$は$\{y \mid y \in a \vee y \in b\}$と同じだから, 
\begin{equation}
\label{gsubstcup1}
  (T_{1}|x_{1}, T_{2}|x_{2}, \cdots, T_{n}|x_{n})(a \cup b) 
  \equiv (T_{1}|x_{1}, T_{2}|x_{2}, \cdots, T_{n}|x_{n})(\{y \mid y \in a \vee y \in b\})
\end{equation}
である.
また$y$が$x_{1}, x_{2}, \cdots, x_{n}$のいずれとも異なり, 
$T_{1}, T_{2}, \cdots, T_{n}$のいずれの中にも自由変数として現れないことから, 
一般代入法則 \ref{gsubstiset}により
\begin{equation}
\label{gsubstcup2}
  (T_{1}|x_{1}, T_{2}|x_{2}, \cdots, T_{n}|x_{n})(\{y \mid y \in a \vee y \in b\}) 
  \equiv \{y \mid (T_{1}|x_{1}, T_{2}|x_{2}, \cdots, T_{n}|x_{n})(y \in a \vee y \in b)\}
\end{equation}
が成り立つ.
また$y$が$x_{1}, x_{2}, \cdots, x_{n}$のいずれとも異なることと一般代入法則 \ref{gsubstfund}により, 
\begin{multline}
\label{gsubstcup3}
  (T_{1}|x_{1}, T_{2}|x_{2}, \cdots, T_{n}|x_{n})(y \in a \vee y \in b) \\
  \equiv y \in (T_{1}|x_{1}, T_{2}|x_{2}, \cdots, T_{n}|x_{n})(a) \vee y \in (T_{1}|x_{1}, T_{2}|x_{2}, \cdots, T_{n}|x_{n})(b)
\end{multline}
が成り立つ.
そこで(\ref{gsubstcup1})---(\ref{gsubstcup3})からわかるように, 
$(T_{1}|x_{1}, T_{2}|x_{2}, \cdots, T_{n}|x_{n})(a \cup b)$は
\begin{equation}
\label{gsubstcup4}
  \{y \mid y \in (T_{1}|x_{1}, T_{2}|x_{2}, \cdots, T_{n}|x_{n})(a) \vee y \in (T_{1}|x_{1}, T_{2}|x_{2}, \cdots, T_{n}|x_{n})(b)\}
\end{equation}
と一致する.
ここで$y$が$a, b, T_{1}, T_{2}, \cdots, T_{n}$のいずれの中にも自由変数として現れないことから, 
変数法則 \ref{valgsubst}により, 
$y$は$(T_{1}|x_{1}, T_{2}|x_{2}, \cdots, T_{n}|x_{n})(a)$及び
$(T_{1}|x_{1}, T_{2}|x_{2}, \cdots, T_{n}|x_{n})(b)$の中に自由変数として現れない.
故に定義から, (\ref{gsubstcup4})は
$(T_{1}|x_{1}, T_{2}|x_{2}, \cdots, T_{n}|x_{n})(a) \cup (T_{1}|x_{1}, T_{2}|x_{2}, \cdots, T_{n}|x_{n})(b)$と同じである.
故に本法則が成り立つ.
\halmos




\mathstrut
\begin{subs}
\label{substcup}%代入41%確認済
$a$, $b$, $T$を記号列とし, $x$を文字とする.
このとき
\[
  (T|x)(a \cup b) \equiv (T|x)(a) \cup (T|x)(b)
\]
が成り立つ.
\end{subs}


\noindent{\bf 証明}
~一般代入法則 \ref{gsubstcup}において, $n$が$1$の場合である.
\halmos




\mathstrut
\begin{form}
\label{formcup}%構成48%確認済
$a$と$b$が集合ならば, $a \cup b$は集合である.
\end{form}


\noindent{\bf 証明}
~$x$を$a$, $b$の中に自由変数として現れない文字とするとき, 
定義より$a \cup b$は$\{x \mid x \in a \vee x \in b\}$である.
$a$と$b$が集合のとき, これが集合となることは, 
構成法則 \ref{formfund}, \ref{formiset}によって直ちにわかる.
\halmos




\mathstrut%確認済%koko
$a$と$b$が集合であるとき, 上記の構成法則 \ref{formcup}により, $a \cup b$は集合である.
そこでこのとき$a \cup b$を$a$と$b$の\textbf{和集合}または\textbf{合併集合}ともいう.




\mathstrut
\begin{thm}
\label{sthmcupsm}%定理7.1%確認済
$a$と$b$を集合とし, $x$をこれらの中に自由変数として現れない文字とする.
このとき関係式$x \in a \vee x \in b$は$x$について集合を作り得る.
\end{thm}


\noindent{\bf 証明}
~$y$を$a$, $b$の中に自由変数として現れない, 定数でない文字とする.
また$z$を$x$, $y$と異なり, $a$, $b$の中に自由変数として現れない, 定数でない文字とする.
このとき変数法則 \ref{valnset}により, $y$と$z$は共に$\{a, b\}$の中に自由変数として現れない.
さていまThm \ref{ata}より
\[
  z \in y \to z \in y
\]
が成り立つから, このことと$y$と$z$が互いに異なり, 共に定数でなく, 
上述のように共に$\{a, b\}$の中に自由変数として現れないことから, 定理 \ref{sthms7ab}より
\begin{equation}
\label{sthmcupsm1}
  {\rm Set}_{z}(\exists y(y \in \{a, b\} \wedge z \in y))
\end{equation}
が成り立つ.
また$y$が$a$, $b$の中に自由変数として現れないことから, 定理 \ref{sthmspinuopair}より, 
\[
  (\exists y \in \{a, b\})(z \in y) \leftrightarrow (a|y)(z \in y) \vee (b|y)(z \in y), 
\]
即ち
\[
  \exists y(y \in \{a, b\} \wedge z \in y) \leftrightarrow (a|y)(z \in y) \vee (b|y)(z \in y)
\]
が成り立つ.
ここで$y$と$z$が互いに異なることから, この記号列は
\[
  \exists y(y \in \{a, b\} \wedge z \in y) \leftrightarrow z \in a \vee z \in b
\]
と一致する.
また$x$が$a$, $b$の中に自由変数として現れないことから, 
代入法則 \ref{substfree}, \ref{substfund}により, この記号列は
\begin{equation}
\label{sthmcupsm2}
  \exists y(y \in \{a, b\} \wedge z \in y) \leftrightarrow (z|x)(x \in a \vee x \in b)
\end{equation}
と一致する.
故にこれが成り立つ.
そこで(\ref{sthmcupsm1}), (\ref{sthmcupsm2})と, $z$が定数でないことから, 定理 \ref{sthmalleqsm}より
\[
  {\rm Set}_{z}((z|x)(x \in a \vee x \in b))
\]
が成り立つ.
ここで$z$が$x$と異なり, $a$, $b$の中に自由変数として現れないことから, 
変数法則 \ref{valfund}により$z$は$x \in a \vee x \in b$の中に自由変数として現れないから, 
代入法則 \ref{substsmtrans}によればこの記号列は
\[
  {\rm Set}_{x}(x \in a \vee x \in b)
\]
と一致する.
故にこれが成り立つ.
\halmos




\mathstrut
\begin{thm}
\label{sthmcupbasis}%定理7.2%確認済
$a$, $b$, $c$を集合とするとき, 
\begin{equation}
\label{sthmcupbasis1}
  c \in a \cup b \leftrightarrow c \in a \vee c \in b
\end{equation}
が成り立つ.
またこのことから, 次の1), 2)が成り立つ.

1)
$c \in a \cup b$ならば, $c \in a \vee c \in b$.

2)
$c \in a$ならば, $c \in a \cup b$.
また$c \in b$ならば, $c \in a \cup b$.
\end{thm}


\noindent{\bf 証明}
~$x$を$a$, $b$の中に自由変数として現れない文字とするとき, 
定理 \ref{sthmcupsm}より$x \in a \vee x \in b$は$x$について集合を作り得るから, 
定理 \ref{sthmisetbasis}より
\[
  c \in \{x \mid x \in a \vee x \in b\} \leftrightarrow (c|x)(x \in a \vee x \in b)
\]
が成り立つ.
ここで$x$が$a$, $b$の中に自由変数として現れないことから, 
$a \cup b$の定義と代入法則 \ref{substfree}, \ref{substfund}によれば, 
この記号列は(\ref{sthmcupbasis1})と一致する.
故に(\ref{sthmcupbasis1})が成り立つ.

\noindent
1)
(\ref{sthmcupbasis1})と推論法則 \ref{dedeqfund}によって明らか.

\noindent
2)
(\ref{sthmcupbasis1})と推論法則 \ref{dedvee}, \ref{dedeqfund}によって明らか.
\halmos




\mathstrut
\begin{thm}
\label{sthmcupnotin}%定理7.3%新規%確認済
$a$, $b$, $c$を集合とするとき, 
\begin{equation}
\label{sthmcupnotin1}
  c \notin a \cup b \leftrightarrow c \notin a \wedge c \notin b
\end{equation}
が成り立つ.
またこのことから, 次の1), 2)が成り立つ.

1)
$c \notin a \cup b$ならば, $c \notin a$と$c \notin b$が共に成り立つ.

2)
$c \notin a$と$c \notin b$が共に成り立てば, $c \notin a \cup b$.
\end{thm}


\noindent{\bf 証明}
~定理 \ref{sthmcupbasis}より
\[
  c \in a \cup b \leftrightarrow c \in a \vee c \in b
\]
が成り立つから, 推論法則 \ref{dedeqcp}により
\begin{equation}
\label{sthmcupnotin2}
  c \notin a \cup b \leftrightarrow \neg (c \in a \vee c \in b)
\end{equation}
が成り立つ.
またThm \ref{n1awb1lnavnb}より
\begin{equation}
\label{sthmcupnotin3}
  \neg (c \in a \vee c \in b) \leftrightarrow c \notin a \wedge c \notin b
\end{equation}
が成り立つ.
そこで(\ref{sthmcupnotin2}), (\ref{sthmcupnotin3})から, 
推論法則 \ref{dedeqtrans}によって(\ref{sthmcupnotin1})が成り立つ.
1), 2)が成り立つことは, (\ref{sthmcupnotin1})と
推論法則 \ref{dedwedge}, \ref{dedeqfund}によって明らかである.
\halmos




\mathstrut
\begin{thm}
\label{sthmsubsetcup}%定理7.4%確認済
$a$と$b$を集合とするとき, 
\[
  a \subset a \cup b, ~~
  b \subset a \cup b
\]
が共に成り立つ.
\end{thm}


\noindent{\bf 証明}
~$x$を$a$, $b$の中に自由変数として現れない, 定数でない文字とする.
このとき変数法則 \ref{valcup}により, $x$は$a \cup b$の中に自由変数として現れない.
また定理 \ref{sthmcupbasis}と推論法則 \ref{dedequiv}により
\[
  x \in a \vee x \in b \to x \in a \cup b
\]
が成り立つから, 推論法則 \ref{deddil}により
\[
  x \in a \to x \in a \cup b, ~~
  x \in b \to x \in a \cup b
\]
が共に成り立つ.
このことと, $x$が定数でなく, 上述のように$a$, $b$, $a \cup b$のいずれの記号列の中にも
自由変数として現れないことから, 
定理 \ref{sthmsubsetconst}より$a \subset a \cup b$と$b \subset a \cup b$が共に成り立つ.
\halmos




\mathstrut
\begin{thm}
\label{sthmcsubsetacupb}%定理7.5%新規%確認済
$a$, $b$, $c$を集合とするとき, 
\begin{align}
  \label{sthmcsubsetacupb1}
  &c \subset a \to c \subset a \cup b, \\
  \mbox{} \notag \\
  \label{sthmcsubsetacupb2}
  &c \subset b \to c \subset a \cup b
\end{align}
が共に成り立つ.
またこれらから, 次の1), 2)が成り立つ.

1)
$c \subset a$ならば, $c \subset a \cup b$.

2)
$c \subset b$ならば, $c \subset a \cup b$.
\end{thm}


\noindent{\bf 証明}
~定理 \ref{sthmsubsetcup}より
\[
  a \subset a \cup b, ~~
  b \subset a \cup b
\]
が共に成り立つから, 推論法則 \ref{dedatawbtrue2}により
\begin{align}
  \label{sthmcsubsetacupb3}
  &c \subset a \to c \subset a \wedge a \subset a \cup b, \\
  \mbox{} \notag \\
  \label{sthmcsubsetacupb4}
  &c \subset b \to c \subset b \wedge b \subset a \cup b
\end{align}
が共に成り立つ.
また定理 \ref{sthmsubsettrans}より
\begin{align}
  \label{sthmcsubsetacupb5}
  &c \subset a \wedge a \subset a \cup b \to c \subset a \cup b, \\
  \mbox{} \notag \\
  \label{sthmcsubsetacupb6}
  &c \subset b \wedge b \subset a \cup b \to c \subset a \cup b
\end{align}
が共に成り立つ.
そこで(\ref{sthmcsubsetacupb3})と(\ref{sthmcsubsetacupb5}), 
(\ref{sthmcsubsetacupb4})と(\ref{sthmcsubsetacupb6})から, 
それぞれ推論法則 \ref{dedmmp}によって(\ref{sthmcsubsetacupb1}), (\ref{sthmcsubsetacupb2})が成り立つ.

\noindent
1)
(\ref{sthmcsubsetacupb1})と推論法則 \ref{dedmp}によって明らか.

\noindent
2)
(\ref{sthmcsubsetacupb2})と推論法則 \ref{dedmp}によって明らか.
\halmos




\mathstrut
\begin{thm}
\label{sthmacupbsubsetc}%定理7.6%確認済
$a$, $b$, $c$を集合とするとき, 
\begin{equation}
\label{sthmacupbsubsetc1}
  a \cup b \subset c \leftrightarrow a \subset c \wedge b \subset c
\end{equation}
が成り立つ.
またこのことから, 次の1), 2)が成り立つ.

1)
$a \cup b \subset c$ならば, $a \subset c$と$b \subset c$が共に成り立つ.

2)
$a \subset c$と$b \subset c$が共に成り立てば, $a \cup b \subset c$.
\end{thm}


\noindent{\bf 証明}
~$x$を$a$, $b$, $c$の中に自由変数として現れない, 定数でない文字とする.
このとき$x$が$a$, $b$の中に自由変数として現れないことから, 
定理 \ref{sthmcupsm}より$x \in a \vee x \in b$は$x$について集合を作り得るから, 
このことと$x$が$c$の中に自由変数として現れないことから, 
定理 \ref{sthmsmtiset&asubset}と推論法則 \ref{dedeqch}により
\[
  \{x \mid x \in a \vee x \in b\} \subset c \leftrightarrow \forall x(x \in a \vee x \in b \to x \in c)
\]
が成り立つ.
ここで$x$が$a$, $b$の中に自由変数として現れないことから, 定義よりこの記号列は
\begin{equation}
\label{sthmacupbsubsetc2}
  a \cup b \subset c \leftrightarrow \forall x(x \in a \vee x \in b \to x \in c)
\end{equation}
と同じである.
故にこれが成り立つ.
またThm \ref{1avbtc1l1atc1w1btc1}より
\[
  (x \in a \vee x \in b \to x \in c) \leftrightarrow (x \in a \to x \in c) \wedge (x \in b \to x \in c)
\]
が成り立つから, このことと$x$が定数でないことから, 推論法則 \ref{dedalleqquansepconst}により
\begin{equation}
\label{sthmacupbsubsetc3}
  \forall x(x \in a \vee x \in b \to x \in c) 
  \leftrightarrow \forall x((x \in a \to x \in c) \wedge (x \in b \to x \in c))
\end{equation}
が成り立つ.
またThm \ref{thmallw}より
\[
  \forall x((x \in a \to x \in c) \wedge (x \in b \to x \in c)) 
  \leftrightarrow \forall x(x \in a \to x \in c) \wedge \forall x(x \in b \to x \in c)
\]
が成り立つ.
ここで$x$が$a$, $b$, $c$の中に自由変数として現れないことから, 定義よりこの記号列は
\begin{equation}
\label{sthmacupbsubsetc4}
  \forall x((x \in a \to x \in c) \wedge (x \in b \to x \in c)) 
  \leftrightarrow a \subset c \wedge b \subset c
\end{equation}
と同じである.
故にこれが成り立つ.
そこで(\ref{sthmacupbsubsetc2})---(\ref{sthmacupbsubsetc4})から, 
推論法則 \ref{dedeqtrans}によって(\ref{sthmacupbsubsetc1})が成り立つことがわかる.
1), 2)が成り立つことは, (\ref{sthmacupbsubsetc1})と
推論法則 \ref{dedwedge}, \ref{dedeqfund}によって明らかである.
\halmos




\mathstrut
\begin{thm}
\label{sthmacupbsubseta}%定理7.7%新規%要る?%確認済
$a$と$b$を集合とするとき, 
\begin{align}
  \label{sthmacupbsubseta1}
  &a \cup b \subset a \leftrightarrow b \subset a, \\
  \mbox{} \notag \\
  \label{sthmacupbsubseta2}
  &a \cup b \subset b \leftrightarrow a \subset b
\end{align}
が共に成り立つ.
またこれらから, 特に次の1), 2)が成り立つ.

1)
$b \subset a$ならば, $a \cup b \subset a$.

2)
$a \subset b$ならば, $a \cup b \subset b$.
\end{thm}


\noindent{\bf 証明}
~定理 \ref{sthmacupbsubsetc}より
\begin{align}
  \label{sthmacupbsubseta3}
  &a \cup b \subset a \leftrightarrow a \subset a \wedge b \subset a, \\
  \mbox{} \notag \\
  \label{sthmacupbsubseta4}
  &a \cup b \subset b \leftrightarrow a \subset b \wedge b \subset b
\end{align}
が共に成り立つ.
また定理 \ref{sthmsubsetself}より$a \subset a$と$b \subset b$が共に成り立つから, 
推論法則 \ref{dedawblatrue2}により
\begin{align}
  \label{sthmacupbsubseta5}
  &a \subset a \wedge b \subset a \leftrightarrow b \subset a, \\
  \mbox{} \notag \\
  \label{sthmacupbsubseta6}
  &a \subset b \wedge b \subset b \leftrightarrow a \subset b
\end{align}
が共に成り立つ.
そこで(\ref{sthmacupbsubseta3})と(\ref{sthmacupbsubseta5}), 
(\ref{sthmacupbsubseta4})と(\ref{sthmacupbsubseta6})から, それぞれ推論法則 \ref{dedeqtrans}によって
(\ref{sthmacupbsubseta1}), (\ref{sthmacupbsubseta2})が成り立つ.

\noindent
1)
(\ref{sthmacupbsubseta1})と推論法則 \ref{dedeqfund}によって明らか.

\noindent
2)
(\ref{sthmacupbsubseta2})と推論法則 \ref{dedeqfund}によって明らか.
\halmos




\mathstrut
\begin{thm}
\label{sthmasubsetbcupceq}%定理7.8%新規%確認済
$a$, $b$, $c$を集合とするとき, 
\begin{align}
  \label{sthmasubsetbcupceq1}
  &a \subset b \cup c \leftrightarrow a \cup c \subset b \cup c, \\
  \mbox{} \notag \\
  \label{sthmasubsetbcupceq2}
  &a \subset b \cup c \leftrightarrow b \cup a \subset b \cup c, \\
  \mbox{} \notag \\
  \label{sthmasubsetbcupceq3}
  &a \cup c \subset b \cup c \leftrightarrow b \cup a \subset b \cup c
\end{align}
がすべて成り立つ.
またこれらから, 次の1), 2), 3)が成り立つ.

1)
$a \subset b \cup c$ならば, 
$a \cup c \subset b \cup c$と$b \cup a \subset b \cup c$が共に成り立つ.

2)
$a \cup c \subset b \cup c$ならば, 
$a \subset b \cup c$と$b \cup a \subset b \cup c$が共に成り立つ.

3)
$b \cup a \subset b \cup c$ならば, 
$a \subset b \cup c$と$a \cup c \subset b \cup c$が共に成り立つ.
\end{thm}


\noindent{\bf 証明}
~定理 \ref{sthmacupbsubsetc}より
\begin{align}
  \label{sthmasubsetbcupceq4}
  &a \cup c \subset b \cup c \leftrightarrow a \subset b \cup c \wedge c \subset b \cup c, \\
  \mbox{} \notag \\
  \label{sthmasubsetbcupceq5}
  &b \cup a \subset b \cup c \leftrightarrow b \subset b \cup c \wedge a \subset b \cup c
\end{align}
が共に成り立つ.
また定理 \ref{sthmsubsetcup}より
\[
  c \subset b \cup c, ~~
  b \subset b \cup c
\]
が共に成り立つから, 推論法則 \ref{dedawblatrue2}により
\begin{align}
  \label{sthmasubsetbcupceq6}
  &a \subset b \cup c \wedge c \subset b \cup c \leftrightarrow a \subset b \cup c, \\
  \mbox{} \notag \\
  \label{sthmasubsetbcupceq7}
  &b \subset b \cup c \wedge a \subset b \cup c \leftrightarrow a \subset b \cup c
\end{align}
が共に成り立つ.
そこで(\ref{sthmasubsetbcupceq4})と(\ref{sthmasubsetbcupceq6}), 
(\ref{sthmasubsetbcupceq5})と(\ref{sthmasubsetbcupceq7})から, それぞれ推論法則 \ref{dedeqtrans}によって
\begin{align}
  \label{sthmasubsetbcupceq8}
  &a \cup c \subset b \cup c \leftrightarrow a \subset b \cup c, \\
  \mbox{} \notag \\
  \label{sthmasubsetbcupceq9}
  &b \cup a \subset b \cup c \leftrightarrow a \subset b \cup c
\end{align}
が成り立つ.
故に(\ref{sthmasubsetbcupceq8}), (\ref{sthmasubsetbcupceq9})から, それぞれ推論法則 \ref{dedeqch}により
(\ref{sthmasubsetbcupceq1}), (\ref{sthmasubsetbcupceq2})が成り立つ.
また(\ref{sthmasubsetbcupceq2}), (\ref{sthmasubsetbcupceq8})から, 
推論法則 \ref{dedeqtrans}によって(\ref{sthmasubsetbcupceq3})が成り立つ.

\noindent
1)
(\ref{sthmasubsetbcupceq1}), (\ref{sthmasubsetbcupceq2})と推論法則 \ref{dedeqfund}によって明らか.

\noindent
2)
(\ref{sthmasubsetbcupceq1}), (\ref{sthmasubsetbcupceq3})と推論法則 \ref{dedeqfund}によって明らか.

\noindent
3)
(\ref{sthmasubsetbcupceq2}), (\ref{sthmasubsetbcupceq3})と推論法則 \ref{dedeqfund}によって明らか.
\halmos




\mathstrut
\begin{thm}
\label{sthmcupsubset}%定理7.9%確認済
\mbox{}

1)
$a$, $b$, $c$を集合とするとき, 
\begin{align}
  \label{sthmcupsubset1}
  &a \subset b \to a \cup c \subset b \cup c, \\
  \mbox{} \notag \\
  \label{sthmcupsubset2}
  &a \subset b \to c \cup a \subset c \cup b
\end{align}
が共に成り立つ.
またこれらから, 次の(\ref{sthmcupsubset3})が成り立つ.
\begin{equation}
\label{sthmcupsubset3}
  a \subset b \text{ならば,} ~
  a \cup c \subset b \cup c \text{と} c \cup a \subset c \cup b \text{が共に成り立つ.}
\end{equation}

2)
$a$, $b$, $c$, $d$を集合とするとき, 
\begin{equation}
\label{sthmcupsubset4}
  a \subset b \wedge c \subset d \to a \cup c \subset b \cup d
\end{equation}
が成り立つ.
またこのことから, 次の(\ref{sthmcupsubset5})が成り立つ.
\begin{equation}
\label{sthmcupsubset5}
  a \subset b \text{と} c \subset d \text{が共に成り立てば,} ~a \cup c \subset b \cup d.
\end{equation}
\end{thm}


\noindent{\bf 証明}
~1)
定理 \ref{sthmasubsetbcupceq}より
\[
  a \subset b \cup c \leftrightarrow a \cup c \subset b \cup c, ~~
  a \subset c \cup b \leftrightarrow c \cup a \subset c \cup b
\]
が共に成り立つから, 推論法則 \ref{dedaddeqt}により
\begin{align}
  \label{sthmcupsubset6}
  &(a \subset b \to a \subset b \cup c) \leftrightarrow (a \subset b \to a \cup c \subset b \cup c), \\
  \mbox{} \notag \\
  \label{sthmcupsubset7}
  &(a \subset b \to a \subset c \cup b) \leftrightarrow (a \subset b \to c \cup a \subset c \cup b)
\end{align}
が共に成り立つ.
ここで定理 \ref{sthmcsubsetacupb}より
\[
  a \subset b \to a \subset b \cup c, ~~
  a \subset b \to a \subset c \cup b
\]
が共に成り立つから, この前者と(\ref{sthmcupsubset6}), 後者と(\ref{sthmcupsubset7})から, 
それぞれ推論法則 \ref{dedeqfund}により(\ref{sthmcupsubset1}), (\ref{sthmcupsubset2})が成り立つ.
(\ref{sthmcupsubset3})が成り立つことは, 
(\ref{sthmcupsubset1}), (\ref{sthmcupsubset2})と推論法則 \ref{dedmp}によって明らかである.

\noindent
2)
1)より
\[
  a \subset b \to a \cup c \subset b \cup c, ~~
  c \subset d \to b \cup c \subset b \cup d
\]
が共に成り立つから, 推論法則 \ref{dedfromaddw}により
\begin{equation}
\label{sthmcupsubset8}
  a \subset b \wedge c \subset d \to a \cup c \subset b \cup c \wedge b \cup c \subset b \cup d
\end{equation}
が成り立つ.
また定理 \ref{sthmsubsettrans}より
\begin{equation}
\label{sthmcupsubset9}
  a \cup c \subset b \cup c \wedge b \cup c \subset b \cup d \to a \cup c \subset b \cup d
\end{equation}
が成り立つ.
そこで(\ref{sthmcupsubset8}), (\ref{sthmcupsubset9})から, 
推論法則 \ref{dedmmp}によって(\ref{sthmcupsubset4})が成り立つ.
(\ref{sthmcupsubset5})が成り立つことは, 
(\ref{sthmcupsubset4})と推論法則 \ref{dedmp}, \ref{dedwedge}によって明らかである.
\halmos




\mathstrut
\begin{thm}
\label{sthmcupsubseteq}%定理7.10%新規%逆は言えない%要る?%確認済
$a$, $b$, $c$を集合とするとき, 
\begin{align}
  \label{sthmcupsubseteq1}
  &c \subset b \to (a \subset b \leftrightarrow a \cup c \subset b \cup c), \\
  \mbox{} \notag \\
  \label{sthmcupsubseteq2}
  &c \subset b \to (a \subset b \leftrightarrow c \cup a \subset c \cup b)
\end{align}
が共に成り立つ.
またこれらから, 次の1), 2), 3)が成り立つ.

1)
$c \subset b$ならば, $a \subset b \leftrightarrow a \cup c \subset b \cup c$と
$a \subset b \leftrightarrow c \cup a \subset c \cup b$が共に成り立つ.

2)
$c \subset b$と$a \cup c \subset b \cup c$が共に成り立てば, $a \subset b$.

3)
$c \subset b$と$c \cup a \subset c \cup b$が共に成り立てば, $a \subset b$.
\end{thm}


\noindent{\bf 証明}
~Thm \ref{awbtbwa}より
\begin{align}
  \label{sthmcupsubseteq3}
  &c \subset b \wedge a \cup c \subset b \cup c \to a \cup c \subset b \cup c \wedge c \subset b, \\
  \mbox{} \notag \\
  \label{sthmcupsubseteq4}
  &c \subset b \wedge c \cup a \subset c \cup b \to c \cup a \subset c \cup b \wedge c \subset b
\end{align}
が共に成り立つ.
また定理 \ref{sthmasubsetbcupceq}と推論法則 \ref{dedequiv}により
\begin{align}
  \label{sthmcupsubseteq5}
  &a \cup c \subset b \cup c \to a \subset b \cup c, \\
  \mbox{} \notag \\
  \label{sthmcupsubseteq6}
  &c \cup a \subset c \cup b \to a \subset c \cup b
\end{align}
が共に成り立つ.
また定理 \ref{sthmacupbsubseta}と推論法則 \ref{dedequiv}により
\begin{align}
  \label{sthmcupsubseteq7}
  &c \subset b \to b \cup c \subset b, \\
  \mbox{} \notag \\
  \label{sthmcupsubseteq8}
  &c \subset b \to c \cup b \subset b
\end{align}
が共に成り立つ.
そこで(\ref{sthmcupsubseteq5})と(\ref{sthmcupsubseteq7}), 
(\ref{sthmcupsubseteq6})と(\ref{sthmcupsubseteq8})から, それぞれ推論法則 \ref{dedfromaddw}により
\begin{align}
  \label{sthmcupsubseteq9}
  &a \cup c \subset b \cup c \wedge c \subset b \to a \subset b \cup c \wedge b \cup c \subset b, \\
  \mbox{} \notag \\
  \label{sthmcupsubseteq10}
  &c \cup a \subset c \cup b \wedge c \subset b \to a \subset c \cup b \wedge c \cup b \subset b
\end{align}
が成り立つ.
また定理 \ref{sthmsubsettrans}より
\begin{align}
  \label{sthmcupsubseteq11}
  &a \subset b \cup c \wedge b \cup c \subset b \to a \subset b, \\
  \mbox{} \notag \\
  \label{sthmcupsubseteq12}
  &a \subset c \cup b \wedge c \cup b \subset b \to a \subset b
\end{align}
が共に成り立つ.
そこで(\ref{sthmcupsubseteq3}), (\ref{sthmcupsubseteq9}), (\ref{sthmcupsubseteq11})から, 
推論法則 \ref{dedmmp}によって
\[
  c \subset b \wedge a \cup c \subset b \cup c \to a \subset b
\]
が成り立つことがわかる.
また(\ref{sthmcupsubseteq4}), (\ref{sthmcupsubseteq10}), (\ref{sthmcupsubseteq12})から, 
同じく推論法則 \ref{dedmmp}によって
\[
  c \subset b \wedge c \cup a \subset c \cup b \to a \subset b
\]
が成り立つことがわかる.
故にこれらから, 推論法則 \ref{dedtwch}により
\begin{align}
  \label{sthmcupsubseteq13}
  &c \subset b \to (a \cup c \subset b \cup c \to a \subset b), \\
  \mbox{} \notag \\
  \label{sthmcupsubseteq14}
  &c \subset b \to (c \cup a \subset c \cup b \to a \subset b)
\end{align}
が共に成り立つ.
また定理 \ref{sthmcupsubset}より
\[
  a \subset b \to a \cup c \subset b \cup c, ~~
  a \subset b \to c \cup a \subset c \cup b
\]
が共に成り立つから, 推論法則 \ref{deds1}により
\begin{align}
  \label{sthmcupsubseteq15}
  &c \subset b \to (a \subset b \to a \cup c \subset b \cup c), \\
  \mbox{} \notag \\
  \label{sthmcupsubseteq16}
  &c \subset b \to (a \subset b \to c \cup a \subset c \cup b)
\end{align}
が共に成り立つ.
そこで(\ref{sthmcupsubseteq13})と(\ref{sthmcupsubseteq15}), 
(\ref{sthmcupsubseteq14})と(\ref{sthmcupsubseteq16})から, 
それぞれ推論法則 \ref{dedpreequiv}により(\ref{sthmcupsubseteq1}), (\ref{sthmcupsubseteq2})が成り立つ.

\noindent
1)
(\ref{sthmcupsubseteq1}), (\ref{sthmcupsubseteq2})と推論法則 \ref{dedmp}によって明らか.

\noindent
2), 3)
1)と推論法則 \ref{dedeqfund}によって明らか.
\halmos




\mathstrut
\begin{thm}
\label{sthmacupb=a}%定理7.11%新規%確認済
$a$と$b$を集合とするとき, 
\begin{align}
  \label{sthmacupb=a1}
  &a \cup b = a \leftrightarrow b \subset a, \\
  \mbox{} \notag \\
  \label{sthmacupb=a2}
  &a \cup b = b \leftrightarrow a \subset b
\end{align}
が共に成り立つ.
またこれらから, 次の1)---4)が成り立つ.

1)
$a \cup b = a$ならば, $b \subset a$.

2)
$b \subset a$ならば, $a \cup b = a$.

3)
$a \cup b = b$ならば, $a \subset b$.

4)
$a \subset b$ならば, $a \cup b = b$.
\end{thm}


\noindent{\bf 証明}
~定理 \ref{sthmaxiom1}と推論法則 \ref{dedeqch}により
\begin{align}
  \label{sthmacupb=a3}
  &a \cup b = a \leftrightarrow a \cup b \subset a \wedge a \subset a \cup b, \\
  \mbox{} \notag \\
  \label{sthmacupb=a4}
  &a \cup b = b \leftrightarrow a \cup b \subset b \wedge b \subset a \cup b
\end{align}
が共に成り立つ.
また定理 \ref{sthmsubsetcup}より
\[
  a \subset a \cup b, ~~
  b \subset a \cup b
\]
が共に成り立つから, 推論法則 \ref{dedawblatrue2}により
\begin{align}
  \label{sthmacupb=a5}
  &a \cup b \subset a \wedge a \subset a \cup b \leftrightarrow a \cup b \subset a, \\
  \mbox{} \notag \\
  \label{sthmacupb=a6}
  &a \cup b \subset b \wedge b \subset a \cup b \leftrightarrow a \cup b \subset b
\end{align}
が共に成り立つ.
また定理 \ref{sthmacupbsubseta}より
\begin{align}
  \label{sthmacupb=a7}
  &a \cup b \subset a \leftrightarrow b \subset a, \\
  \mbox{} \notag \\
  \label{sthmacupb=a8}
  &a \cup b \subset b \leftrightarrow a \subset b
\end{align}
が共に成り立つ.
そこで(\ref{sthmacupb=a3}), (\ref{sthmacupb=a5}), (\ref{sthmacupb=a7})から, 
推論法則 \ref{dedeqtrans}によって(\ref{sthmacupb=a1})が成り立つことがわかる.
また(\ref{sthmacupb=a4}), (\ref{sthmacupb=a6}), (\ref{sthmacupb=a8})から, 
同じく推論法則 \ref{dedeqtrans}によって(\ref{sthmacupb=a2})が成り立つことがわかる.

\noindent
1), 2)
(\ref{sthmacupb=a1})と推論法則 \ref{dedeqfund}によって明らか.

\noindent
3), 4)
(\ref{sthmacupb=a2})と推論法則 \ref{dedeqfund}によって明らか.
\halmos




\mathstrut
\begin{thm}
\label{sthmcuppsubset}%定理7.12%新規%確認済
$a$と$b$を集合とするとき, 
\begin{align}
  \label{sthmcuppsubset1}
  &a \subsetneqq a \cup b \leftrightarrow b \not\subset a, \\
  \mbox{} \notag \\
  \label{sthmcuppsubset2}
  &b \subsetneqq a \cup b \leftrightarrow a \not\subset b
\end{align}
が共に成り立つ.
またこれらから, 次の1)---4)が成り立つ.

1)
$a \subsetneqq a \cup b$ならば, $b \not\subset a$.

2)
$b \not\subset a$ならば, $a \subsetneqq a \cup b$.

3)
$b \subsetneqq a \cup b$ならば, $a \not\subset b$.

4)
$a \not\subset b$ならば, $b \subsetneqq a \cup b$.
\end{thm}


\noindent{\bf 証明}
~定理 \ref{sthmpsubset&axiom1}より
\begin{align}
  \label{sthmcuppsubset3}
  &a \subsetneqq a \cup b \leftrightarrow a \subset a \cup b \wedge a \cup b \not\subset a, \\
  \mbox{} \notag \\
  \label{sthmcuppsubset4}
  &b \subsetneqq a \cup b \leftrightarrow b \subset a \cup b \wedge a \cup b \not\subset b
\end{align}
が共に成り立つ.
また定理 \ref{sthmsubsetcup}より
\[
  a \subset a \cup b, ~~
  b \subset a \cup b
\]
が共に成り立つから, 推論法則 \ref{dedawblatrue2}により
\begin{align}
  \label{sthmcuppsubset5}
  &a \subset a \cup b \wedge a \cup b \not\subset a \leftrightarrow a \cup b \not\subset a, \\
  \mbox{} \notag \\
  \label{sthmcuppsubset6}
  &b \subset a \cup b \wedge a \cup b \not\subset b \leftrightarrow a \cup b \not\subset b
\end{align}
が共に成り立つ.
また定理 \ref{sthmacupbsubseta}より
\[
  a \cup b \subset a \leftrightarrow b \subset a, ~~
  a \cup b \subset b \leftrightarrow a \subset b
\]
が共に成り立つから, 推論法則 \ref{dedeqcp}により
\begin{align}
  \label{sthmcuppsubset7}
  &a \cup b \not\subset a \leftrightarrow b \not\subset a, \\
  \mbox{} \notag \\
  \label{sthmcuppsubset8}
  &a \cup b \not\subset b \leftrightarrow a \not\subset b
\end{align}
が共に成り立つ.
そこで(\ref{sthmcuppsubset3}), (\ref{sthmcuppsubset5}), (\ref{sthmcuppsubset7})から, 
推論法則 \ref{dedeqtrans}によって(\ref{sthmcuppsubset1})が成り立つことがわかる.
また(\ref{sthmcuppsubset4}), (\ref{sthmcuppsubset6}), (\ref{sthmcuppsubset8})から, 
同じく推論法則 \ref{dedeqtrans}によって(\ref{sthmcuppsubset2})が成り立つことがわかる.

\noindent
1), 2)
(\ref{sthmcuppsubset1})と推論法則 \ref{dedeqfund}によって明らか.

\noindent
3), 4)
(\ref{sthmcuppsubset2})と推論法則 \ref{dedeqfund}によって明らか.
\halmos




\mathstrut
\begin{thm}
\label{sthmcup=}%定理7.13%確認済
\mbox{}

1)
$a$, $b$, $c$を集合とするとき, 
\begin{align}
  \label{sthmcup=1}
  &a = b \to a \cup c = b \cup c, \\
  \mbox{} \notag \\
  \label{sthmcup=2}
  &a = b \to c \cup a = c \cup b
\end{align}
が共に成り立つ.
またこれらから, 次の(\ref{sthmcup=3})が成り立つ.
\begin{equation}
\label{sthmcup=3}
  a = b \text{ならば,} ~a \cup c = b \cup c \text{と} c \cup a = c \cup b \text{が共に成り立つ.}
\end{equation}

2)
$a$, $b$, $c$, $d$を集合とするとき, 
\begin{equation}
\label{sthmcup=4}
  a = b \wedge c = d \to a \cup c = b \cup d
\end{equation}
が成り立つ.
またこのことから, 次の(\ref{sthmcup=5})が成り立つ.
\begin{equation}
\label{sthmcup=5}
  a = b \text{と} c = d \text{が共に成り立てば,} ~a \cup c = b \cup d.
\end{equation}
\end{thm}


\noindent{\bf 証明}
~1)
$x$を$c$の中に自由変数として現れない文字とするとき, Thm \ref{T=Ut1TV=UV1}より
\[
  a = b \to (a|x)(x \cup c) = (b|x)(x \cup c), ~~
  a = b \to (a|x)(c \cup x) = (b|x)(c \cup x)
\]
が共に成り立つが, 代入法則 \ref{substfree}, \ref{substcup}によれば
これらはそれぞれ(\ref{sthmcup=1}), (\ref{sthmcup=2})と一致するから, これらが共に成り立つ.
(\ref{sthmcup=3})が成り立つことは, 
(\ref{sthmcup=1}), (\ref{sthmcup=2})と推論法則 \ref{dedmp}によって明らかである.

\noindent
2)
1)より
\[
  a = b \to a \cup c = b \cup c, ~~
  c = d \to b \cup c = b \cup d
\]
が共に成り立つから, 推論法則 \ref{dedfromaddw}により
\begin{equation}
\label{sthmcup=6}
  a = b \wedge c = d \to a \cup c = b \cup c \wedge b \cup c = b \cup d
\end{equation}
が成り立つ.
またThm \ref{x=ywy=ztx=z}より
\begin{equation}
\label{sthmcup=7}
  a \cup c = b \cup c \wedge b \cup c = b \cup d \to a \cup c = b \cup d
\end{equation}
が成り立つ.
そこで(\ref{sthmcup=6}), (\ref{sthmcup=7})から, 
推論法則 \ref{dedmmp}によって(\ref{sthmcup=4})が成り立つ.
(\ref{sthmcup=5})が成り立つことは, 
(\ref{sthmcup=4})と推論法則 \ref{dedmp}, \ref{dedwedge}によって明らかである.
\halmos




\mathstrut
\begin{thm}
\label{sthmcup=eq}%定理7.14%新規%要る?%確認済
$a$, $b$, $c$を集合とするとき, 
\begin{align}
  \label{sthmcup=eq1}
  &c \subset a \wedge c \subset b \to (a = b \leftrightarrow a \cup c = b \cup c), \\
  \mbox{} \notag \\
  \label{sthmcup=eq2}
  &c \subset a \wedge c \subset b \to (a = b \leftrightarrow c \cup a = c \cup b)
\end{align}
が共に成り立つ.
またこれらから, 次の1), 2), 3)が成り立つ.

1)
$c \subset a$と$c \subset b$が共に成り立てば, 
$a = b \leftrightarrow a \cup c = b \cup c$と$a = b \leftrightarrow c \cup a = c \cup b$が共に成り立つ.

2)
$c \subset a$, $c \subset b$, $a \cup c = b \cup c$がすべて成り立てば, $a = b$.

3)
$c \subset a$, $c \subset b$, $c \cup a = c \cup b$がすべて成り立てば, $a = b$.
\end{thm}


\noindent{\bf 証明}
~定理 \ref{sthmacupb=a}と推論法則 \ref{dedequiv}により
\begin{align*}
  &c \subset a \to a \cup c = a, ~~
  c \subset b \to b \cup c = b, \\
  \mbox{} \notag \\
  &c \subset a \to c \cup a = a, ~~
  c \subset b \to c \cup b = b
\end{align*}
がすべて成り立つから, このはじめの二つ, あとの二つから, それぞれ推論法則 \ref{dedfromaddw}により
\begin{align}
  \label{sthmcup=eq3}
  &c \subset a \wedge c \subset b \to a \cup c = a \wedge b \cup c = b, \\
  \mbox{} \notag \\
  \label{sthmcup=eq4}
  &c \subset a \wedge c \subset b \to c \cup a = a \wedge c \cup b = b
\end{align}
が成り立つ.
またThm \ref{x=ywz=ut1x=zly=u1}より
\begin{align}
  \label{sthmcup=eq5}
  &a \cup c = a \wedge b \cup c = b \to (a \cup c = b \cup c \leftrightarrow a = b), \\
  \mbox{} \notag \\
  \label{sthmcup=eq6}
  &c \cup a = a \wedge c \cup b = b \to (c \cup a = c \cup b \leftrightarrow a = b)
\end{align}
が共に成り立つ.
またThm \ref{1alb1t1bla1}より
\begin{align}
  \label{sthmcup=eq7}
  &(a \cup c = b \cup c \leftrightarrow a = b) \to (a = b \leftrightarrow a \cup c = b \cup c), \\
  \mbox{} \notag \\
  \label{sthmcup=eq8}
  &(c \cup a = c \cup b \leftrightarrow a = b) \to (a = b \leftrightarrow c \cup a = c \cup b)
\end{align}
が共に成り立つ.
そこで(\ref{sthmcup=eq3}), (\ref{sthmcup=eq5}), (\ref{sthmcup=eq7})から, 
推論法則 \ref{dedmmp}によって(\ref{sthmcup=eq1})が成り立つことがわかる.
また(\ref{sthmcup=eq4}), (\ref{sthmcup=eq6}), (\ref{sthmcup=eq8})から, 
同じく推論法則 \ref{dedmmp}によって(\ref{sthmcup=eq2})が成り立つことがわかる.

\noindent
1)
(\ref{sthmcup=eq1}), (\ref{sthmcup=eq2})と推論法則 \ref{dedmp}, \ref{dedwedge}によって明らか.

\noindent
2), 3)
1)と推論法則 \ref{dedeqfund}によって明らか.
\halmos




\mathstrut
\begin{thm}
\label{sthmspincup}%定理7.15%新規%確認済
$a$と$b$を集合, $R$を関係式とし, $x$を$a$及び$b$の中に自由変数として現れない文字とする.
このとき
\begin{align}
  \label{sthmspincup1}
  &(\exists x \in a \cup b)(R) \leftrightarrow (\exists x \in a)(R) \vee (\exists x \in b)(R), \\
  \mbox{} \notag \\
  \label{sthmspincup2}
  &(\forall x \in a \cup b)(R) \leftrightarrow (\forall x \in a)(R) \wedge (\forall x \in b)(R), \\
  \mbox{} \notag \\
  \label{sthmspincup3}
  &(!x \in a \cup b)(R) \to (!x \in a)(R) \wedge (!x \in b)(R), \\
  \mbox{} \notag \\
  \label{sthmspincup4}
  &(\exists !x \in a \cup b)(R) \to (\exists !x \in a)(R) \vee (\exists !x \in b)(R)
\end{align}
がすべて成り立つ.
またこれらから, 次の1)---6)が成り立つ.

1)
$(\exists x \in a \cup b)(R)$ならば, $(\exists x \in a)(R) \vee (\exists x \in b)(R)$.

2)
$(\exists x \in a)(R)$ならば, $(\exists x \in a \cup b)(R)$.
また$(\exists x \in b)(R)$ならば, $(\exists x \in a \cup b)(R)$.

3)
$(\forall x \in a \cup b)(R)$ならば, $(\forall x \in a)(R)$と$(\forall x \in b)(R)$が共に成り立つ.

4)
$(\forall x \in a)(R)$と$(\forall x \in b)(R)$が共に成り立てば, $(\forall x \in a \cup b)(R)$.

5)
$(!x \in a \cup b)(R)$ならば, $(!x \in a)(R)$と$(!x \in b)(R)$が共に成り立つ.

6)
$(\exists !x \in a \cup b)(R)$ならば, $(\exists !x \in a)(R) \vee (\exists !x \in b)(R)$.
\end{thm}


\noindent{\bf 証明}
~$x$が$a$, $b$の中に自由変数として現れないことから, 
定理 \ref{sthmcupsm}より$x \in a \vee x \in b$は$x$について集合を作り得る.
故に定理 \ref{sthmspiniset}より
\begin{align*}
  &(\exists x \in \{x \mid x \in a \vee x \in b\})(R) 
  \leftrightarrow \exists_{x \in a \vee x \in b}x(R), \\
  \mbox{} \notag \\
  &(\forall x \in \{x \mid x \in a \vee x \in b\})(R) 
  \leftrightarrow \forall_{x \in a \vee x \in b}x(R)
\end{align*}
が共に成り立つが, 定義よりこれらの記号列はそれぞれ
\begin{align}
  \label{sthmspincup5}
  &(\exists x \in a \cup b)(R) \leftrightarrow \exists_{x \in a \vee x \in b}x(R), \\
  \mbox{} \notag \\
  \label{sthmspincup6}
  &(\forall x \in a \cup b)(R) \leftrightarrow \forall_{x \in a \vee x \in b}x(R)
\end{align}
と同じだから, これらが共に成り立つ.
またThm \ref{thmspexprev}より
\begin{equation}
\label{sthmspincup7}
  \exists_{x \in a \vee x \in b}x(R) \leftrightarrow (\exists x \in a)(R) \vee (\exists x \in b)(R)
\end{equation}
が成り立つ.
またThm \ref{thmspallprev}より
\begin{equation}
\label{sthmspincup8}
  \forall_{x \in a \vee x \in b}x(R) \leftrightarrow (\forall x \in a)(R) \wedge (\forall x \in b)(R)
\end{equation}
が成り立つ.
そこで(\ref{sthmspincup5})と(\ref{sthmspincup7}), (\ref{sthmspincup6})と(\ref{sthmspincup8})から, 
それぞれ推論法則 \ref{dedeqtrans}によって(\ref{sthmspincup1}), (\ref{sthmspincup2})が成り立つ.
また$x$が$a$, $b$の中に自由変数として現れないことから, 
変数法則 \ref{valcup}により, $x$は$a \cup b$の中にも自由変数として現れない.
このことと, 定理 \ref{sthmsubsetcup}より
\[
  a \subset a \cup b, ~~
  b \subset a \cup b
\]
が共に成り立つことから, 定理 \ref{sthmspinsubset}より
\begin{align}
  \label{sthmspincup9}
  &(!x \in a \cup b)(R) \to (!x \in a)(R), \\
  \mbox{} \notag \\
  \label{sthmspincup10}
  &(!x \in a \cup b)(R) \to (!x \in b)(R)
\end{align}
が共に成り立つ.
故にこれらから, 推論法則 \ref{dedprewedge}により(\ref{sthmspincup3})が成り立つ.
また(\ref{sthmspincup1})から, 推論法則 \ref{dedequiv}により
\[
  (\exists x \in a \cup b)(R) \to (\exists x \in a)(R) \vee (\exists x \in b)(R)
\]
が成り立つから, 推論法則 \ref{dedaddw}により
\begin{equation}
\label{sthmspincup11}
  (\exists !x \in a \cup b)(R) 
  \to ((\exists x \in a)(R) \vee (\exists x \in b)(R)) \wedge (!x \in a \cup b)(R)
\end{equation}
が成り立つ.
またThm \ref{aw1bvc1t1awb1v1awc1}より
\begin{multline}
\label{sthmspincup12}
  ((\exists x \in a)(R) \vee (\exists x \in b)(R)) \wedge (!x \in a \cup b)(R) \\
  \to ((\exists x \in a)(R) \wedge (!x \in a \cup b)(R)) \vee ((\exists x \in b)(R) \wedge (!x \in a \cup b)(R))
\end{multline}
が成り立つ.
また(\ref{sthmspincup9}), (\ref{sthmspincup10})から, それぞれ推論法則 \ref{dedaddw}により
\begin{align*}
  &(\exists x \in a)(R) \wedge (!x \in a \cup b)(R) \to (\exists !x \in a)(R), \\
  \mbox{} \notag \\
  &(\exists x \in b)(R) \wedge (!x \in a \cup b)(R) \to (\exists !x \in b)(R)
\end{align*}
が成り立つから, 推論法則 \ref{dedfromaddv}により
\begin{multline}
\label{sthmspincup13}
  ((\exists x \in a)(R) \wedge (!x \in a \cup b)(R)) \vee ((\exists x \in b)(R) \wedge (!x \in a \cup b)(R)) \\
  \to (\exists !x \in a)(R) \vee (\exists !x \in b)(R)
\end{multline}
が成り立つ.
そこで(\ref{sthmspincup11})---(\ref{sthmspincup13})から, 
推論法則 \ref{dedmmp}によって(\ref{sthmspincup4})が成り立つことがわかる.

\noindent
1)
(\ref{sthmspincup1})と推論法則 \ref{dedeqfund}によって明らか.

\noindent
2)
(\ref{sthmspincup1})と推論法則 \ref{dedvee}, \ref{dedeqfund}によって明らか.

\noindent
3), 4)
(\ref{sthmspincup2})と推論法則 \ref{dedwedge}, \ref{dedeqfund}によって明らか.

\noindent
5)
(\ref{sthmspincup3})と推論法則 \ref{dedmp}, \ref{dedwedge}によって明らか.

\noindent
6)
(\ref{sthmspincup4})と推論法則 \ref{dedmp}によって明らか.
\halmos




\mathstrut
\begin{thm}
\label{sthmcupidempotent}%定理7.16%確認済
$a$を集合とするとき, 
\[
  a \cup a = a
\]
が成り立つ.
\end{thm}


\noindent{\bf 証明}
~定理 \ref{sthmsubsetself}より$a \subset a$が成り立つから, 
定理 \ref{sthmacupb=a}より$a \cup a = a$が成り立つ.
\halmos




\mathstrut
\begin{thm}
\label{sthmcupch}%定理7.17%確認済
$a$と$b$を集合とするとき, 
\[
  a \cup b = b \cup a
\]
が成り立つ.
\end{thm}


\noindent{\bf 証明}
~$x$を$a$, $b$の中に自由変数として現れない, 定数でない文字とする.
このときThm \ref{avblbva}より
\[
  x \in a \vee x \in b \leftrightarrow x \in b \vee x \in a
\]
が成り立つから, このことと$x$が定数でないことから, 定理 \ref{sthmalleqiset=}より
\[
  \{x \mid x \in a \vee x \in b\} = \{x \mid x \in b \vee x \in a\}
\]
が成り立つ.
ここで$x$が$a$, $b$の中に自由変数として現れないことから, 
定義よりこの記号列は$a \cup b = b \cup a$と同じである.
故にこれが成り立つ.
\halmos




\mathstrut
\begin{thm}
\label{sthmcupcomb}%定理7.18%確認済
$a$, $b$, $c$を集合とするとき, 
\[
  (a \cup b) \cup c = a \cup (b \cup c)
\]
が成り立つ.
\end{thm}


\noindent{\bf 証明}
~$x$を$a$, $b$, $c$の中に自由変数として現れない, 定数でない文字とする.
このとき変数法則 \ref{valcup}により, $x$は$a \cup b$, $b \cup c$の中に自由変数として現れない.
また定理 \ref{sthmcupbasis}より
\[
  x \in a \cup b \leftrightarrow x \in a \vee x \in b
\]
が成り立つから, 推論法則 \ref{dedaddeqv}により
\begin{equation}
\label{sthmcupcomb1}
  x \in a \cup b \vee x \in c \leftrightarrow (x \in a \vee x \in b) \vee x \in c
\end{equation}
が成り立つ.
またThm \ref{1avb1vclav1bvc1}より
\begin{equation}
\label{sthmcupcomb2}
  (x \in a \vee x \in b) \vee x \in c \leftrightarrow x \in a \vee (x \in b \vee x \in c)
\end{equation}
が成り立つ.
また定理 \ref{sthmcupbasis}と推論法則 \ref{dedeqch}により
\[
  x \in b \vee x \in c \leftrightarrow x \in b \cup c
\]
が成り立つから, 推論法則 \ref{dedaddeqv}により
\begin{equation}
\label{sthmcupcomb3}
  x \in a \vee (x \in b \vee x \in c) \leftrightarrow x \in a \vee x \in b \cup c
\end{equation}
が成り立つ.
そこで(\ref{sthmcupcomb1})---(\ref{sthmcupcomb3})から, 推論法則 \ref{dedeqtrans}によって
\[
  x \in a \cup b \vee x \in c \leftrightarrow x \in a \vee x \in b \cup c
\]
が成り立つことがわかる.
このことと$x$が定数でないことから, 定理 \ref{sthmalleqiset=}より
\[
  \{x \mid x \in a \cup b \vee x \in c\} = \{x \mid x \in a \vee x \in b \cup c\}
\]
が成り立つ.
ここで上述のように, $x$は$a \cup b$, $c$, $a$, $b \cup c$のいずれの中にも自由変数として現れないから, 
定義よりこの記号列は$(a \cup b) \cup c = a \cup (b \cup c)$と同じである.
故にこれが成り立つ.
\halmos




\mathstrut
\begin{thm}
\label{sthmcupdist}%定理7.19%確認済
$a$, $b$, $c$を集合とするとき, 
\[
  a \cup (b \cup c) = (a \cup b) \cup (a \cup c), ~~
  (a \cup b) \cup c = (a \cup c) \cup (b \cup c)
\]
が共に成り立つ.
\end{thm}


\noindent{\bf 証明}
~$x$を$a$, $b$, $c$の中に自由変数として現れない, 定数でない文字とする.
このとき変数法則 \ref{valcup}により, 
$x$は$a \cup b$, $a \cup c$, $b \cup c$の中に自由変数として現れない.
また定理 \ref{sthmcupbasis}より
\[
  x \in b \cup c \leftrightarrow x \in b \vee x \in c, ~~
  x \in a \cup b \leftrightarrow x \in a \vee x \in b
\]
が共に成り立つから, 推論法則 \ref{dedaddeqv}により
\begin{align}
  \label{sthmcupdist1}
  &x \in a \vee x \in b \cup c \leftrightarrow x \in a \vee (x \in b \vee x \in c), \\
  \mbox{} \notag \\
  \label{sthmcupdist2}
  &x \in a \cup b \vee x \in c \leftrightarrow (x \in a \vee x \in b) \vee x \in c
\end{align}
が共に成り立つ.
またThm \ref{av1bvc1l1avb1v1avc1}より
\begin{align}
  \label{sthmcupdist3}
  &x \in a \vee (x \in b \vee x \in c) 
  \leftrightarrow (x \in a \vee x \in b) \vee (x \in a \vee x \in c), \\
  \mbox{} \notag \\
  \label{sthmcupdist4}
  &(x \in a \vee x \in b) \vee x \in c 
  \leftrightarrow (x \in a \vee x \in c) \vee (x \in b \vee x \in c)
\end{align}
が共に成り立つ.
また定理 \ref{sthmcupbasis}と推論法則 \ref{dedeqch}により
\begin{align}
  \label{sthmcupdist5}
  &x \in a \vee x \in b \leftrightarrow x \in a \cup b, \\
  \mbox{} \notag \\
  \label{sthmcupdist6}
  &x \in a \vee x \in c \leftrightarrow x \in a \cup c, \\
  \mbox{} \notag \\
  \label{sthmcupdist7}
  &x \in b \vee x \in c \leftrightarrow x \in b \cup c
\end{align}
がすべて成り立つ.
故に(\ref{sthmcupdist5})と(\ref{sthmcupdist6}), (\ref{sthmcupdist6})と(\ref{sthmcupdist7})から, 
それぞれ推論法則 \ref{dedaddeqv}により
\begin{align}
  \label{sthmcupdist8}
  &(x \in a \vee x \in b) \vee (x \in a \vee x \in c) 
  \leftrightarrow x \in a \cup b \vee x \in a \cup c, \\
  \mbox{} \notag \\
  \label{sthmcupdist9}
  &(x \in a \vee x \in c) \vee (x \in b \vee x \in c) 
  \leftrightarrow x \in a \cup c \vee x \in b \cup c
\end{align}
が成り立つ.
そこで(\ref{sthmcupdist1}), (\ref{sthmcupdist3}), (\ref{sthmcupdist8})から, 
推論法則 \ref{dedeqtrans}によって
\[
  x \in a \vee x \in b \cup c \leftrightarrow x \in a \cup b \vee x \in a \cup c
\]
が成り立つことがわかる.
また(\ref{sthmcupdist2}), (\ref{sthmcupdist4}), (\ref{sthmcupdist9})から, 
同じく推論法則 \ref{dedeqtrans}によって
\[
  x \in a \cup b \vee x \in c \leftrightarrow x \in a \cup c \vee x \in b \cup c
\]
が成り立つことがわかる.
これらのことと, $x$が定数でないことから, 定理 \ref{sthmalleqiset=}より
\begin{align*}
  &\{x \mid x \in a \vee x \in b \cup c\} = \{x \mid x \in a \cup b \vee x \in a \cup c\}, \\
  \mbox{} \notag \\
  &\{x \mid x \in a \cup b \vee x \in c\} = \{x \mid x \in a \cup c \vee x \in b \cup c\}
\end{align*}
が共に成り立つ.
ここで上述のように, $x$は$a$, $c$, $a \cup b$, $a \cup c$, $b \cup c$のいずれの中にも
自由変数として現れないから, 定義よりこれらの記号列はそれぞれ
$a \cup (b \cup c) = (a \cup b) \cup (a \cup c)$, 
$(a \cup b) \cup c = (a \cup c) \cup (b \cup c)$と同じである.
故にこれらが共に成り立つ.
\halmos




\mathstrut
\begin{thm}
\label{sthmvsm}%定理7.20%確認済
$R$と$S$を関係式とし, $x$を文字とする.
このとき
\begin{equation}
\label{sthmvsm1}
  {\rm Set}_{x}(R \vee S) \leftrightarrow {\rm Set}_{x}(R) \wedge {\rm Set}_{x}(S)
\end{equation}
が成り立つ.
またこのことから, 次の1), 2)が成り立つ.

1)
$R \vee S$が$x$について集合を作り得るならば, $R$と$S$は共に$x$について集合を作り得る.

2)
$R$と$S$が共に$x$について集合を作り得るならば, $R \vee S$は$x$について集合を作り得る.
\end{thm}


\noindent{\bf 証明}
~$y$を$x$と異なり, $R$, $S$の中に自由変数として現れない, 定数でない文字とする.
このときThm \ref{atavb}より
\[
  (y|x)(R) \to (y|x)(R) \vee (y|x)(S), ~~
  (y|x)(S) \to (y|x)(R) \vee (y|x)(S)
\]
が共に成り立つが, 代入法則 \ref{substfund}によればこれらの記号列はそれぞれ
\[
  (y|x)(R \to R \vee S), ~~
  (y|x)(S \to R \vee S)
\]
と一致するから, これらが共に成り立つ.
このことと$y$が定数でないことから, 推論法則 \ref{dedltthmquan}により
\[
  \forall y((y|x)(R \to R \vee S)), ~~
  \forall y((y|x)(S \to R \vee S))
\]
が共に成り立つ.
ここで$y$が$R$, $S$の中に自由変数として現れないことから, 変数法則 \ref{valfund}により
$y$は$R \to R \vee S$, $S \to R \vee S$の中に自由変数として現れないから, 
代入法則 \ref{substquantrans}によれば上記の記号列はそれぞれ
\[
  \forall x(R \to R \vee S), ~~
  \forall x(S \to R \vee S)
\]
と一致する.
故にこれらが共に成り立つ.
そこで定理 \ref{sthmalltsm}より
\[
  {\rm Set}_{x}(R \vee S) \to {\rm Set}_{x}(R), ~~
  {\rm Set}_{x}(R \vee S) \to {\rm Set}_{x}(S)
\]
が共に成り立つ.
故に推論法則 \ref{dedprewedge}により
\begin{equation}
\label{sthmvsm2}
  {\rm Set}_{x}(R \vee S) \to {\rm Set}_{x}(R) \wedge {\rm Set}_{x}(S)
\end{equation}
が成り立つ.
また定理 \ref{sthmisetbasis}と推論法則 \ref{dedpreequiv}により
\[
  {\rm Set}_{x}(R) \to ((y|x)(R) \to y \in \{x \mid R\}), ~~
  {\rm Set}_{x}(S) \to ((y|x)(S) \to y \in \{x \mid S\})
\]
が共に成り立つから, 推論法則 \ref{dedfromaddw}により
\begin{equation}
\label{sthmvsm3}
  {\rm Set}_{x}(R) \wedge {\rm Set}_{x}(S) 
  \to ((y|x)(R) \to y \in \{x \mid R\}) \wedge ((y|x)(S) \to y \in \{x \mid S\})
\end{equation}
が成り立つ.
またThm \ref{1atb1w1ctd1t1avctbvd1}より
\begin{multline}
\label{sthmvsm4}
  ((y|x)(R) \to y \in \{x \mid R\}) \wedge ((y|x)(S) \to y \in \{x \mid S\}) \\
  \to ((y|x)(R) \vee (y|x)(S) \to y \in \{x \mid R\} \vee y \in \{x \mid S\})
\end{multline}
が成り立つ.
また定理 \ref{sthmcupbasis}と推論法則 \ref{dedequiv}により
\[
  y \in \{x \mid R\} \vee y \in \{x \mid S\} \to y \in \{x \mid R\} \cup \{x \mid S\}
\]
が成り立つから, 推論法則 \ref{dedaddb}により
\[
  ((y|x)(R) \vee (y|x)(S) \to y \in \{x \mid R\} \vee y \in \{x \mid S\}) 
  \to ((y|x)(R) \vee (y|x)(S) \to y \in \{x \mid R\} \cup \{x \mid S\})
\]
が成り立つ.
ここで変数法則 \ref{valiset}, \ref{valcup}により, 
$x$は$\{x \mid R\} \cup \{x \mid S\}$の中に自由変数として現れないから, 
代入法則 \ref{substfree}, \ref{substfund}によれば上記の記号列は
\begin{equation}
\label{sthmvsm5}
  ((y|x)(R) \vee (y|x)(S) \to y \in \{x \mid R\} \vee y \in \{x \mid S\}) 
  \to (y|x)(R \vee S \to x \in \{x \mid R\} \cup \{x \mid S\})
\end{equation}
と一致する.
故にこれが成り立つ.
そこで(\ref{sthmvsm3})---(\ref{sthmvsm5})から, 推論法則 \ref{dedmmp}によって
\begin{equation}
\label{sthmvsm6}
  {\rm Set}_{x}(R) \wedge {\rm Set}_{x}(S) \to (y|x)(R \vee S \to x \in \{x \mid R\} \cup \{x \mid S\})
\end{equation}
が成り立つことがわかる.
ここで$y$が$R$, $S$の中に自由変数として現れないことから, 変数法則 \ref{valwedge}, \ref{valsm}により, 
$y$は${\rm Set}_{x}(R) \wedge {\rm Set}_{x}(S)$の中に自由変数として現れない.
また$y$は定数でない.
これらのことと, (\ref{sthmvsm6})が成り立つことから, 推論法則 \ref{dedalltquansepfreeconst}により
\[
  {\rm Set}_{x}(R) \wedge {\rm Set}_{x}(S) 
  \to \forall y((y|x)(R \vee S \to x \in \{x \mid R\} \cup \{x \mid S\}))
\]
が成り立つ.
ここで$y$が$x$と異なり, $R$, $S$の中に自由変数として現れないことから, 
変数法則 \ref{valfund}, \ref{valiset}, \ref{valcup}により, 
$y$は$R \vee S \to x \in \{x \mid R\} \cup \{x \mid S\}$の中に自由変数として現れないから, 
代入法則 \ref{substquantrans}によれば上記の記号列は
\begin{equation}
\label{sthmvsm7}
  {\rm Set}_{x}(R) \wedge {\rm Set}_{x}(S) 
  \to \forall x(R \vee S \to x \in \{x \mid R\} \cup \{x \mid S\})
\end{equation}
と一致する.
故にこれが成り立つ.
また上述のように$x$は$\{x \mid R\} \cup \{x \mid S\}$の中に自由変数として現れないから, 
定理 \ref{sthmalltisetsubseta}と推論法則 \ref{dedequiv}により
\[
  \forall x(R \vee S \to x \in \{x \mid R\} \cup \{x \mid S\}) 
  \to {\rm Set}_{x}(R \vee S) \wedge \{x \mid R \vee S\} \subset \{x \mid R\} \cup \{x \mid S\}
\]
が成り立つ.
故に推論法則 \ref{dedprewedge}により
\begin{equation}
\label{sthmvsm8}
  \forall x(R \vee S \to x \in \{x \mid R\} \cup \{x \mid S\}) \to {\rm Set}_{x}(R \vee S)
\end{equation}
が成り立つ.
そこで(\ref{sthmvsm7}), (\ref{sthmvsm8})から, 推論法則 \ref{dedmmp}によって
\begin{equation}
\label{sthmvsm9}
  {\rm Set}_{x}(R) \wedge {\rm Set}_{x}(S) \to {\rm Set}_{x}(R \vee S)
\end{equation}
が成り立つ.
故に(\ref{sthmvsm2}), (\ref{sthmvsm9})から, 推論法則 \ref{dedequiv}により(\ref{sthmvsm1})が成り立つ.
1), 2)が成り立つことは, (\ref{sthmvsm1})と推論法則 \ref{dedwedge}, \ref{dedeqfund}によって明らかである.
\halmos




\mathstrut
\begin{thm}
\label{sthmisetcup}%定理7.21%確認済
$R$と$S$を関係式とし, $x$を文字とする.
このとき
\begin{align}
  \label{sthmisetcup1}
  &{\rm Set}_{x}(R) \wedge {\rm Set}_{x}(S) \to \{x \mid R\} \cup \{x \mid S\} = \{x \mid R \vee S\}, \\
  \mbox{} \notag \\
  \label{sthmisetcup2}
  &{\rm Set}_{x}(R \vee S) \to \{x \mid R\} \cup \{x \mid S\} = \{x \mid R \vee S\}
\end{align}
が共に成り立つ.
またこれらから, 次の1), 2)が成り立つ.

1)
$R$と$S$が共に$x$について集合を作り得るならば, $\{x \mid R\} \cup \{x \mid S\} = \{x \mid R \vee S\}$.

2)
$R \vee S$が$x$について集合を作り得るならば, $\{x \mid R\} \cup \{x \mid S\} = \{x \mid R \vee S\}$.
\end{thm}


\noindent{\bf 証明}
~$y$を$x$と異なり, $R$, $S$の中に自由変数として現れない, 定数でない文字とする.
このとき定理 \ref{sthmisetbasis}より
\[
  {\rm Set}_{x}(R) \to (y \in \{x \mid R\} \leftrightarrow (y|x)(R)), ~~
  {\rm Set}_{x}(S) \to (y \in \{x \mid S\} \leftrightarrow (y|x)(S))
\]
が共に成り立つから, 推論法則 \ref{dedfromaddw}により
\begin{equation}
\label{sthmisetcup3}
  {\rm Set}_{x}(R) \wedge {\rm Set}_{x}(S) 
  \to (y \in \{x \mid R\} \leftrightarrow (y|x)(R)) \wedge (y \in \{x \mid S\} \leftrightarrow (y|x)(S))
\end{equation}
が成り立つ.
またThm \ref{1alb1w1cld1t1avclbvd1}より
\begin{multline*}
  (y \in \{x \mid R\} \leftrightarrow (y|x)(R)) \wedge (y \in \{x \mid S\} \leftrightarrow (y|x)(S)) \\
  \to (y \in \{x \mid R\} \vee y \in \{x \mid S\} \leftrightarrow (y|x)(R) \vee (y|x)(S))
\end{multline*}
が成り立つ.
ここで変数法則 \ref{valiset}により, $x$は$\{x \mid R\}$, $\{x \mid S\}$の中に自由変数として現れないから, 
代入法則 \ref{substfree}, \ref{substfund}, \ref{substequiv}によれば上記の記号列は
\begin{multline}
\label{sthmisetcup4}
  (y \in \{x \mid R\} \leftrightarrow (y|x)(R)) \wedge (y \in \{x \mid S\} \leftrightarrow (y|x)(S)) \\
  \to (y|x)(x \in \{x \mid R\} \vee x \in \{x \mid S\} \leftrightarrow R \vee S)
\end{multline}
と一致する.
故にこれが成り立つ.
そこで(\ref{sthmisetcup3}), (\ref{sthmisetcup4})から, 推論法則 \ref{dedmmp}によって
\begin{equation}
\label{sthmisetcup5}
  {\rm Set}_{x}(R) \wedge {\rm Set}_{x}(S) 
  \to (y|x)(x \in \{x \mid R\} \vee x \in \{x \mid S\} \leftrightarrow R \vee S)
\end{equation}
が成り立つ.
ここで$y$が$R$, $S$の中に自由変数として現れないことから, 変数法則 \ref{valwedge}, \ref{valsm}により, 
$y$は${\rm Set}_{x}(R) \wedge {\rm Set}_{x}(S)$の中に自由変数として現れない.
また$y$は定数でない.
これらのことと, (\ref{sthmisetcup5})が成り立つことから, 推論法則 \ref{dedalltquansepfreeconst}により
\[
  {\rm Set}_{x}(R) \wedge {\rm Set}_{x}(S) 
  \to \forall y((y|x)(x \in \{x \mid R\} \vee x \in \{x \mid S\} \leftrightarrow R \vee S))
\]
が成り立つ.
ここで$y$が$x$と異なり, $R$, $S$の中に自由変数として現れないことから, 
変数法則 \ref{valfund}, \ref{valequiv}, \ref{valiset}により, 
$y$は$x \in \{x \mid R\} \vee x \in \{x \mid S\} \leftrightarrow R \vee S$の中に
自由変数として現れないから, 代入法則 \ref{substquantrans}によれば上記の記号列は
\begin{equation}
\label{sthmisetcup6}
  {\rm Set}_{x}(R) \wedge {\rm Set}_{x}(S) 
  \to \forall x(x \in \{x \mid R\} \vee x \in \{x \mid S\} \leftrightarrow R \vee S)
\end{equation}
と一致する.
故にこれが成り立つ.
また定理 \ref{sthmalleqiset=}より
\[
  \forall x(x \in \{x \mid R\} \vee x \in \{x \mid S\} \leftrightarrow R \vee S) 
  \to \{x \mid x \in \{x \mid R\} \vee x \in \{x \mid S\}\} = \{x \mid R \vee S\}
\]
が成り立つが, 上述のように$x$は$\{x \mid R\}$, $\{x \mid S\}$の中に自由変数として現れないから, 
定義よりこの記号列は
\begin{equation}
\label{sthmisetcup7}
  \forall x(x \in \{x \mid R\} \vee x \in \{x \mid S\} \leftrightarrow R \vee S) 
  \to \{x \mid R\} \cup \{x \mid S\} = \{x \mid R \vee S\}
\end{equation}
と一致する.
故にこれが成り立つ.
そこで(\ref{sthmisetcup6}), (\ref{sthmisetcup7})から, 
推論法則 \ref{dedmmp}によって(\ref{sthmisetcup1})が成り立つ.
また定理 \ref{sthmvsm}と推論法則 \ref{dedequiv}により
\[
  {\rm Set}_{x}(R \vee S) \to {\rm Set}_{x}(R) \wedge {\rm Set}_{x}(S)
\]
が成り立つから, これと(\ref{sthmisetcup1})から, 
推論法則 \ref{dedmmp}によって(\ref{sthmisetcup2})が成り立つ.

\noindent
1)
(\ref{sthmisetcup1})と推論法則 \ref{dedmp}, \ref{dedwedge}によって明らか.

\noindent
2)
(\ref{sthmisetcup2})と推論法則 \ref{dedmp}によって明らか.
\halmos




\mathstrut
\begin{thm}
\label{sthmnsm}%定理7.22%新規%確認済
$R$を関係式とし, $x$を文字とするとき, 
\begin{equation}
\label{sthmnsm1}
  \neg ({\rm Set}_{x}(R) \wedge {\rm Set}_{x}(\neg R))
\end{equation}
が成り立つ.
\end{thm}


\noindent{\bf 証明}
~Thm \ref{allx1rtr1}より, $\forall x(\neg R \to \neg R)$, 
即ち$\forall x(R \vee \neg R)$が成り立つから, 定理 \ref{sthmall&sm}より
\begin{equation}
\label{sthmnsm2}
  \neg {\rm Set}_{x}(R \vee \neg R)
\end{equation}
が成り立つ.
また定理 \ref{sthmvsm}より
\begin{equation}
\label{sthmnsm3}
  {\rm Set}_{x}(R \vee \neg R) \leftrightarrow {\rm Set}_{x}(R) \wedge {\rm Set}_{x}(\neg R)
\end{equation}
が成り立つ.
そこで(\ref{sthmnsm2}), (\ref{sthmnsm3})から, 
推論法則 \ref{dedeqfund}により(\ref{sthmnsm1})が成り立つ.
\halmos




\mathstrut
\begin{thm}
\label{sthmssetcup}%定理7.23%確認済
$a$と$b$を集合, $R$を関係式とし, $x$を$a$及び$b$の中に自由変数として現れない文字とする.
このとき
\begin{equation}
\label{sthmssetcup1}
  \{x \in a \mid R\} \cup \{x \in b \mid R\} = \{x \in a \cup b \mid R\}
\end{equation}
が成り立つ.
\end{thm}


\noindent{\bf 証明}
~$y$を$a$, $b$, $R$の中に自由変数として現れない, 定数でない文字とする.
このとき変数法則 \ref{valsset}により, 
$y$は$\{x \in a \mid R\}$, $\{x \in b \mid R\}$の中に自由変数として現れない.
また変数法則 \ref{valcup}により, $y$は$a \cup b$の中にも自由変数として現れない.
さて$x$が$a$, $b$の中に自由変数として現れないことから, 定理 \ref{sthmssetbasis}より
\[
  y \in \{x \in a \mid R\} \leftrightarrow y \in a \wedge (y|x)(R), ~~
  y \in \{x \in b \mid R\} \leftrightarrow y \in b \wedge (y|x)(R)
\]
が共に成り立つから, 推論法則 \ref{dedaddeqv}により
\begin{equation}
\label{sthmssetcup2}
  y \in \{x \in a \mid R\} \vee y \in \{x \in b \mid R\} 
  \leftrightarrow (y \in a \wedge (y|x)(R)) \vee (y \in b \wedge (y|x)(R))
\end{equation}
が成り立つ.
またThm \ref{aw1bvc1l1awb1v1awc1}と推論法則 \ref{dedeqch}により
\begin{equation}
\label{sthmssetcup3}
  (y \in a \wedge (y|x)(R)) \vee (y \in b \wedge (y|x)(R)) 
  \leftrightarrow (y \in a \vee y \in b) \wedge (y|x)(R)
\end{equation}
が成り立つ.
また定理 \ref{sthmcupbasis}と推論法則 \ref{dedeqch}により
\[
  y \in a \vee y \in b \leftrightarrow y \in a \cup b
\]
が成り立つから, 推論法則 \ref{dedaddeqw}により
\begin{equation}
\label{sthmssetcup4}
  (y \in a \vee y \in b) \wedge (y|x)(R) \leftrightarrow y \in a \cup b \wedge (y|x)(R)
\end{equation}
が成り立つ.
そこで(\ref{sthmssetcup2})---(\ref{sthmssetcup4})から, 推論法則 \ref{dedeqtrans}によって
\[
  y \in \{x \in a \mid R\} \vee y \in \{x \in b \mid R\} \leftrightarrow y \in a \cup b \wedge (y|x)(R)
\]
が成り立つことがわかる.
このことと$y$が定数でないことから, 定理 \ref{sthmalleqiset=}より, 
\[
  \{y \mid y \in \{x \in a \mid R\} \vee y \in \{x \in b \mid R\}\} = \{y \mid y \in a \cup b \wedge (y|x)(R)\}, 
\]
即ち
\begin{equation}
\label{sthmssetcup5}
  \{y \mid y \in \{x \in a \mid R\} \vee y \in \{x \in b \mid R\}\} = \{y \in a \cup b \mid (y|x)(R)\}
\end{equation}
が成り立つ.
ここで$x$が$a$, $b$の中に自由変数として現れないことから, 
変数法則 \ref{valcup}により$x$は$a \cup b$の中に自由変数として現れない.
また上述のように, $y$は$\{x \in a \mid R\}$, $\{x \in b \mid R\}$, 
$a \cup b$, $R$のいずれの中にも自由変数として現れない.
故に定義と代入法則 \ref{substssettrans}によれば, (\ref{sthmssetcup5})は(\ref{sthmssetcup1})と一致する.
従って(\ref{sthmssetcup1})が成り立つ.
\halmos




\mathstrut
\begin{thm}
\label{sthmssetcuprs}%定理7.24%確認済
$a$を集合, $R$と$S$を関係式とし, $x$を$a$の中に自由変数として現れない文字とする.
このとき
\begin{equation}
\label{sthmssetcuprs1}
  \{x \in a \mid R\} \cup \{x \in a \mid S\} = \{x \in a \mid R \vee S\}
\end{equation}
が成り立つ.
\end{thm}


\noindent{\bf 証明}
~$y$を$a$, $R$, $S$の中に自由変数として現れない, 定数でない文字とする.
このとき変数法則 \ref{valfund}により, $y$は$R \vee S$の中に自由変数として現れない.
また変数法則 \ref{valsset}により, 
$y$は$\{x \in a \mid R\}$, $\{x \in a \mid S\}$の中にも自由変数として現れない.
さて$x$が$a$の中に自由変数として現れないことから, 定理 \ref{sthmssetbasis}より
\[
  y \in \{x \in a \mid R\} \leftrightarrow y \in a \wedge (y|x)(R), ~~
  y \in \{x \in a \mid S\} \leftrightarrow y \in a \wedge (y|x)(S)
\]
が共に成り立つから, 推論法則 \ref{dedaddeqv}により
\begin{equation}
\label{sthmssetcuprs2}
  y \in \{x \in a \mid R\} \vee y \in \{x \in a \mid S\} 
  \leftrightarrow (y \in a \wedge (y|x)(R)) \vee (y \in a \wedge (y|x)(S))
\end{equation}
が成り立つ.
またThm \ref{aw1bvc1l1awb1v1awc1}と推論法則 \ref{dedeqch}により
\[
  (y \in a \wedge (y|x)(R)) \vee (y \in a \wedge (y|x)(S)) 
  \leftrightarrow y \in a \wedge ((y|x)(R) \vee (y|x)(S))
\]
が成り立つが, 代入法則 \ref{substfund}によればこの記号列は
\begin{equation}
\label{sthmssetcuprs3}
  (y \in a \wedge (y|x)(R)) \vee (y \in a \wedge (y|x)(S)) 
  \leftrightarrow y \in a \wedge (y|x)(R \vee S)
\end{equation}
と一致するから, これが成り立つ.
そこで(\ref{sthmssetcuprs2}), (\ref{sthmssetcuprs3})から, 推論法則 \ref{dedeqtrans}によって
\[
  y \in \{x \in a \mid R\} \vee y \in \{x \in a \mid S\} \leftrightarrow y \in a \wedge (y|x)(R \vee S)
\]
が成り立つ.
このことと$y$が定数でないことから, 定理 \ref{sthmalleqiset=}より, 
\[
  \{y \mid y \in \{x \in a \mid R\} \vee y \in \{x \in a \mid S\}\} = \{y \mid y \in a \wedge (y|x)(R \vee S)\}, 
\]
即ち
\[
  \{y \mid y \in \{x \in a \mid R\} \vee y \in \{x \in a \mid S\}\} = \{y \in a \mid (y|x)(R \vee S)\}
\]
が成り立つ.
ここで$x$が$a$の中に自由変数として現れないことと, 
上述のように$y$が$\{x \in a \mid R\}$, $\{x \in a \mid S\}$, $a$, $R \vee S$のいずれの中にも
自由変数として現れないことから, 定義と代入法則 \ref{substssettrans}により, 
上記の記号列は(\ref{sthmssetcuprs1})と一致する.
故に(\ref{sthmssetcuprs1})が成り立つ.
\halmos




\mathstrut
\begin{thm}
\label{sthmsingletoncup}%定理7.25%確認済
$a$と$b$を集合とするとき, 
\[
  \{a\} \cup \{b\} = \{a, b\}
\]
が成り立つ.
\end{thm}


\noindent{\bf 証明}
~$x$を$a$, $b$の中に自由変数として現れない文字とするとき, 
定理 \ref{sthmsingletonsm}より$x = a$と$x = b$は共に$x$について集合を作り得るから, 
定理 \ref{sthmisetcup}より
\[
  \{x \mid x = a\} \cup \{x \mid x = b\} = \{x \mid x = a \vee x = b\}
\]
が成り立つ.
ここで$x$が$a$, $b$の中に自由変数として現れないことから, 
定義よりこの記号列は$\{a\} \cup \{b\} = \{a, b\}$と同じである.
故にこれが成り立つ.
\halmos




\mathstrut
\begin{thm}
\label{sthmosetcup}%定理7.26%確認済
$a$, $b$, $T$を集合とし, $x$を$a$及び$b$の中に自由変数として現れない文字とする.
このとき
\begin{equation}
\label{sthmosetcup1}
  \{T\}_{x \in a \cup b} = \{T\}_{x \in a} \cup \{T\}_{x \in b}
\end{equation}
が成り立つ.
\end{thm}


\noindent{\bf 証明}
~$y$を$x$と異なり, $a$, $b$, $T$の中に自由変数として現れない, 定数でない文字とする.
このとき変数法則 \ref{valoset}により, 
$y$は$\{T\}_{x \in a}$, $\{T\}_{x \in b}$の中に自由変数として現れない.
また変数法則 \ref{valcup}により, $y$は$a \cup b$の中にも自由変数として現れない.
さて$x$が$a$, $b$の中に自由変数として現れないことから, 定理 \ref{sthmspincup}より, 
\[
  (\exists x \in a \cup b)(y = T) 
  \leftrightarrow (\exists x \in a)(y = T) \vee (\exists x \in b)(y = T), 
\]
即ち
\begin{equation}
\label{sthmosetcup2}
  \exists x(x \in a \cup b \wedge y = T) 
  \leftrightarrow \exists x(x \in a \wedge y = T) \vee \exists x(x \in b \wedge y = T)
\end{equation}
が成り立つ.
また$x$が$y$と異なり, $a$, $b$の中に自由変数として現れないことから, 
定理 \ref{sthmosetbasis}と推論法則 \ref{dedeqch}により
\[
  \exists x(x \in a \wedge y = T) \leftrightarrow y \in \{T\}_{x \in a}, ~~
  \exists x(x \in b \wedge y = T) \leftrightarrow y \in \{T\}_{x \in b}
\]
が共に成り立つ.
故に推論法則 \ref{dedaddeqv}により
\begin{equation}
\label{sthmosetcup3}
  \exists x(x \in a \wedge y = T) \vee \exists x(x \in b \wedge y = T) 
  \leftrightarrow y \in \{T\}_{x \in a} \vee y \in \{T\}_{x \in b}
\end{equation}
が成り立つ.
そこで(\ref{sthmosetcup2}), (\ref{sthmosetcup3})から, 推論法則 \ref{dedeqtrans}によって
\[
  \exists x(x \in a \cup b \wedge y = T) 
  \leftrightarrow y \in \{T\}_{x \in a} \vee y \in \{T\}_{x \in b}
\]
が成り立つ.
このことと$y$が定数でないことから, 定理 \ref{sthmalleqiset=}より
\[
  \{y \mid \exists x(x \in a \cup b \wedge y = T)\} 
  = \{y \mid y \in \{T\}_{x \in a} \vee y \in \{T\}_{x \in b}\}
\]
が成り立つ.
ここで$y$が$x$と異なり, 上述のように$a \cup b$, $T$, $\{T\}_{x \in a}$, $\{T\}_{x \in b}$の
いずれの中にも自由変数として現れないことから, 定義より上記の記号列は(\ref{sthmosetcup1})と同じである.
故に(\ref{sthmosetcup1})が成り立つ.
\halmos




\mathstrut
\begin{thm}
\label{sthm-cupsubset}%定理7.27%新規%確認済
$a$, $b$, $c$を集合とするとき, 
\begin{align}
  \label{sthm-cupsubset1}
  &a - b \subset c \leftrightarrow a \subset b \cup c, \\
  \mbox{} \notag \\
  \label{sthm-cupsubset2}
  &a - c \subset b \leftrightarrow a \subset b \cup c, \\
  \mbox{} \notag \\
  \label{sthm-cupsubset3}
  &a - c \subset b - c \leftrightarrow a \subset b \cup c, \\
  \mbox{} \notag \\
  \label{sthm-cupsubset4}
  &a - b \subset c - b \leftrightarrow a \subset b \cup c
\end{align}
がすべて成り立つ.
またこれらから, 次の1), 2)が成り立つ.

1)
$a - b \subset c$, $a - c \subset b$, $a - c \subset b - c$, $a - b \subset c - b$の
いずれかが成り立てば, $a \subset b \cup c$.

2)
$a \subset b \cup c$ならば, 
$a - b \subset c$, $a - c \subset b$, $a - c \subset b - c$, $a - b \subset c - b$がすべて成り立つ.
\end{thm}


\noindent{\bf 証明}
~$x$を$a$, $b$, $c$の中に自由変数として現れない, 定数でない文字とする.
このとき変数法則 \ref{val-}により, $x$は$a - b$の中に自由変数として現れない.
また変数法則 \ref{valcup}により, $x$は$b \cup c$の中にも自由変数として現れない.
さて定理 \ref{sthm-basis}より
\[
  x \in a - b \leftrightarrow x \in a \wedge x \notin b
\]
が成り立つから, 推論法則 \ref{dedaddeqt}により
\begin{equation}
\label{sthm-cupsubset5}
  (x \in a - b \to x \in c) \leftrightarrow (x \in a \wedge x \notin b \to x \in c)
\end{equation}
が成り立つ.
またThm \ref{1at1btc11l1awbtc1}と推論法則 \ref{dedeqch}により, 
\[
  (x \in a \wedge x \notin b \to x \in c) \leftrightarrow (x \in a \to (x \notin b \to x \in c)), 
\]
即ち
\begin{equation}
\label{sthm-cupsubset6}
  (x \in a \wedge x \notin b \to x \in c) \leftrightarrow (x \in a \to x \in b \vee x \in c)
\end{equation}
が成り立つ.
また定理 \ref{sthmcupbasis}と推論法則 \ref{dedeqch}により
\[
  x \in b \vee x \in c \leftrightarrow x \in b \cup c
\]
が成り立つから, 推論法則 \ref{dedaddeqt}により
\begin{equation}
\label{sthm-cupsubset7}
  (x \in a \to x \in b \vee x \in c) \leftrightarrow (x \in a \to x \in b \cup c)
\end{equation}
が成り立つ.
そこで(\ref{sthm-cupsubset5})---(\ref{sthm-cupsubset7})から, 推論法則 \ref{dedeqtrans}によって
\[
  (x \in a - b \to x \in c) \leftrightarrow (x \in a \to x \in b \cup c)
\]
が成り立つことがわかる.
このことと$x$が定数でないことから, 推論法則 \ref{dedalleqquansepconst}により
\[
  \forall x(x \in a - b \to x \in c) \leftrightarrow \forall x(x \in a \to x \in b \cup c)
\]
が成り立つ.
ここで上述のように, $x$は$a - b$, $c$, $a$, $b \cup c$のいずれの中にも自由変数として現れないから, 
定義より上記の記号列は(\ref{sthm-cupsubset1})と同じである.
故に(\ref{sthm-cupsubset1})が成り立つ.
また定理 \ref{sthma-bsubsetceq}より
\begin{align}
  \label{sthm-cupsubset8}
  &a - c \subset b \leftrightarrow a - b \subset c, \\
  \mbox{} \notag \\
  \label{sthm-cupsubset9}
  &a - c \subset b - c \leftrightarrow a - b \subset c, \\
  \mbox{} \notag \\
  \label{sthm-cupsubset10}
  &a - b \subset c - b \leftrightarrow a - b \subset c
\end{align}
がすべて成り立つ.
そこで(\ref{sthm-cupsubset1})と(\ref{sthm-cupsubset8}), 
(\ref{sthm-cupsubset1})と(\ref{sthm-cupsubset9}), 
(\ref{sthm-cupsubset1})と(\ref{sthm-cupsubset10})から, それぞれ推論法則 \ref{dedeqtrans}によって
(\ref{sthm-cupsubset2}), (\ref{sthm-cupsubset3}), (\ref{sthm-cupsubset4})が成り立つ.
1), 2)が成り立つことは, (\ref{sthm-cupsubset1})---(\ref{sthm-cupsubset4})と
推論法則 \ref{dedeqfund}によって明らかである.
\halmos




\mathstrut
\begin{thm}
\label{sthm-cupcupsubset}%定理7.28%新規%確認済
$a$, $b$, $c$を集合とするとき, 
\begin{align}
  \label{sthm-cupcupsubset1}
  &a - b \subset c \leftrightarrow a \cup c \subset b \cup c, \\
  \mbox{} \notag \\
  \label{sthm-cupcupsubset2}
  &a - c \subset b \leftrightarrow a \cup c \subset b \cup c, \\
  \mbox{} \notag \\
  \label{sthm-cupcupsubset3}
  &a - c \subset b - c \leftrightarrow a \cup c \subset b \cup c, \\
  \mbox{} \notag \\
  \label{sthm-cupcupsubset4}
  &a - b \subset c - b \leftrightarrow a \cup c \subset b \cup c
\end{align}
がすべて成り立つ.
またこれらから, 次の1), 2)が成り立つ.

1)
$a - b \subset c$, $a - c \subset b$, $a - c \subset b - c$, $a - b \subset c - b$の
いずれかが成り立てば, $a \cup c \subset b \cup c$.

2)
$a \cup c \subset b \cup c$ならば, 
$a - b \subset c$, $a - c \subset b$, $a - c \subset b - c$, $a - b \subset c - b$がすべて成り立つ.
\end{thm}


\noindent{\bf 証明}
~定理 \ref{sthm-cupsubset}より
\begin{align}
  \label{sthm-cupcupsubset5}
  &a - b \subset c \leftrightarrow a \subset b \cup c, \\
  \mbox{} \notag \\
  \label{sthm-cupcupsubset6}
  &a - c \subset b \leftrightarrow a \subset b \cup c, \\
  \mbox{} \notag \\
  \label{sthm-cupcupsubset7}
  &a - c \subset b - c \leftrightarrow a \subset b \cup c, \\
  \mbox{} \notag \\
  \label{sthm-cupcupsubset8}
  &a - b \subset c - b \leftrightarrow a \subset b \cup c
\end{align}
がすべて成り立つ.
また定理 \ref{sthmasubsetbcupceq}より
\begin{equation}
\label{sthm-cupcupsubset9}
  a \subset b \cup c \leftrightarrow a \cup c \subset b \cup c
\end{equation}
が成り立つ.
そこで(\ref{sthm-cupcupsubset5})と(\ref{sthm-cupcupsubset9}), 
(\ref{sthm-cupcupsubset6})と(\ref{sthm-cupcupsubset9}), 
(\ref{sthm-cupcupsubset7})と(\ref{sthm-cupcupsubset9}), 
(\ref{sthm-cupcupsubset8})と(\ref{sthm-cupcupsubset9})から, 
それぞれ推論法則 \ref{dedeqtrans}によって(\ref{sthm-cupcupsubset1}), (\ref{sthm-cupcupsubset2}), 
(\ref{sthm-cupcupsubset3}), (\ref{sthm-cupcupsubset4})が成り立つ.
1), 2)が成り立つことは, (\ref{sthm-cupcupsubset1})---(\ref{sthm-cupcupsubset4})と
推論法則 \ref{dedeqfund}によって明らかである.
\halmos




\mathstrut
\begin{thm}
\label{sthm-cup=}%定理7.29%新規%確認済
$a$, $b$, $c$を集合とするとき, 
\begin{align}
  \label{sthm-cup=1}
  &a - c = b - c \leftrightarrow a \cup c = b \cup c, \\
  \mbox{} \notag \\
  \label{sthm-cup=2}
  &a - c = b - c \leftrightarrow c \cup a = c \cup b
\end{align}
が共に成り立つ.
またこれらから, 次の1), 2), 3)が成り立つ.

1)
$a - c = b - c$ならば, $a \cup c = b \cup c$と$c \cup a = c \cup b$が共に成り立つ.

2)
$a \cup c = b \cup c$ならば, $a - c = b - c$.

3)
$c \cup a = c \cup b$ならば, $a - c = b - c$.
\end{thm}


\noindent{\bf 証明}
~$x$を$a$, $b$, $c$の中に自由変数として現れない, 定数でない文字とする.
このとき定理 \ref{sthmssetsm}より, 
$x \in a \wedge x \notin c$と$x \in b \wedge x \notin c$は共に$x$について集合を作り得る.
故に定理 \ref{sthmsmtalleqiset=eq}より
\[
  \forall x(x \in a \wedge x \notin c \leftrightarrow x \in b \wedge x \notin c) 
  \leftrightarrow \{x \mid x \in a \wedge x \notin c\} = \{x \mid x \in b \wedge x \notin c\}
\]
が成り立つが, 定義よりこの記号列は
\[
  \forall x(x \in a \wedge x \notin c \leftrightarrow x \in b \wedge x \notin c) 
  \leftrightarrow a - c = b - c
\]
と同じだから, これが成り立つ.
故に推論法則 \ref{dedeqch}により
\begin{equation}
\label{sthm-cup=3}
  a - c = b - c 
  \leftrightarrow \forall x(x \in a \wedge x \notin c \leftrightarrow x \in b \wedge x \notin c)
\end{equation}
が成り立つ.
またThm \ref{1ct1alb11l1awclbwc1}と推論法則 \ref{dedeqch}により, 
\[
  (x \in a \wedge x \notin c \leftrightarrow x \in b \wedge x \notin c) 
  \leftrightarrow (x \notin c \to (x \in a \leftrightarrow x \in b)), 
\]
即ち
\begin{equation}
\label{sthm-cup=4}
  (x \in a \wedge x \notin c \leftrightarrow x \in b \wedge x \notin c) 
  \leftrightarrow x \in c \vee (x \in a \leftrightarrow x \in b)
\end{equation}
が成り立つ.
またThm \ref{1alb1vcl1avclbvc1}より
\begin{align}
  \label{sthm-cup=5}
  &x \in c \vee (x \in a \leftrightarrow x \in b) 
  \leftrightarrow (x \in a \vee x \in c \leftrightarrow x \in b \vee x \in c), \\
  \mbox{} \notag \\
  \label{sthm-cup=6}
  &x \in c \vee (x \in a \leftrightarrow x \in b) 
  \leftrightarrow (x \in c \vee x \in a \leftrightarrow x \in c \vee x \in b)
\end{align}
が共に成り立つ.
そこで(\ref{sthm-cup=4})と(\ref{sthm-cup=5}), (\ref{sthm-cup=4})と(\ref{sthm-cup=6})から, 
それぞれ推論法則 \ref{dedeqtrans}によって
\begin{align*}
  &(x \in a \wedge x \notin c \leftrightarrow x \in b \wedge x \notin c) 
  \leftrightarrow (x \in a \vee x \in c \leftrightarrow x \in b \vee x \in c), \\
  \mbox{} \notag \\
  &(x \in a \wedge x \notin c \leftrightarrow x \in b \wedge x \notin c) 
  \leftrightarrow (x \in c \vee x \in a \leftrightarrow x \in c \vee x \in b)
\end{align*}
が成り立つ.
これらのことと$x$が定数でないことから, 推論法則 \ref{dedalleqquansepconst}により
\begin{align}
  \label{sthm-cup=7}
  &\forall x(x \in a \wedge x \notin c \leftrightarrow x \in b \wedge x \notin c) 
  \leftrightarrow \forall x(x \in a \vee x \in c \leftrightarrow x \in b \vee x \in c), \\
  \mbox{} \notag \\
  \label{sthm-cup=8}
  &\forall x(x \in a \wedge x \notin c \leftrightarrow x \in b \wedge x \notin c) 
  \leftrightarrow \forall x(x \in c \vee x \in a \leftrightarrow x \in c \vee x \in b)
\end{align}
が共に成り立つ.
また$x$が$a$, $b$, $c$の中に自由変数として現れないことから, 定理 \ref{sthmcupsm}より
$x \in a \vee x \in c$, $x \in b \vee x \in c$, $x \in c \vee x \in a$, $x \in c \vee x \in b$は
いずれも$x$について集合を作り得る.
故に定理 \ref{sthmsmtalleqiset=eq}より
\begin{align*}
  &\forall x(x \in a \vee x \in c \leftrightarrow x \in b \vee x \in c) 
  \leftrightarrow \{x \mid x \in a \vee x \in c\} = \{x \mid x \in b \vee x \in c\}, \\
  \mbox{} \notag \\
  &\forall x(x \in c \vee x \in a \leftrightarrow x \in c \vee x \in b) 
  \leftrightarrow \{x \mid x \in c \vee x \in a\} = \{x \mid x \in c \vee x \in b\}
\end{align*}
が共に成り立つが, 定義よりこれらの記号列はそれぞれ
\begin{align}
  \label{sthm-cup=9}
  &\forall x(x \in a \vee x \in c \leftrightarrow x \in b \vee x \in c) 
  \leftrightarrow a \cup c = b \cup c, \\
  \mbox{} \notag \\
  \label{sthm-cup=10}
  &\forall x(x \in c \vee x \in a \leftrightarrow x \in c \vee x \in b) 
  \leftrightarrow c \cup a = c \cup b
\end{align}
と同じだから, これらが共に成り立つ.
そこで(\ref{sthm-cup=3}), (\ref{sthm-cup=7}), (\ref{sthm-cup=9})から, 
推論法則 \ref{dedeqtrans}によって(\ref{sthm-cup=1})が成り立つことがわかる.
また(\ref{sthm-cup=3}), (\ref{sthm-cup=8}), (\ref{sthm-cup=10})から, 
同じく推論法則 \ref{dedeqtrans}によって(\ref{sthm-cup=2})が成り立つことがわかる.

\noindent
1)
(\ref{sthm-cup=1}), (\ref{sthm-cup=2})と推論法則 \ref{dedeqfund}によって明らか.

\noindent
2)
(\ref{sthm-cup=1})と推論法則 \ref{dedeqfund}によって明らか.

\noindent
3)
(\ref{sthm-cup=2})と推論法則 \ref{dedeqfund}によって明らか.
\halmos




\mathstrut
\begin{thm}
\label{sthmcup-cup=-cup}%定理7.30%新規%確認済
$a$, $b$, $c$を集合とするとき, 
\begin{align}
  \label{sthmcup-cup=-cup1}
  &a \cup c - b \cup c = a - b \cup c, \\
  \mbox{} \notag \\
  \label{sthmcup-cup=-cup2}
  &c \cup a - c \cup b = a - c \cup b
\end{align}
が共に成り立つ.
\end{thm}


\noindent{\bf 証明}
~定理 \ref{sthmcupdist}と推論法則 \ref{ded=ch}により
\begin{equation}
\label{sthmcup-cup=-cup3}
  (a \cup c) \cup (b \cup c) = (a \cup b) \cup c
\end{equation}
が成り立つ.
また定理 \ref{sthmcupcomb}より
\begin{equation}
\label{sthmcup-cup=-cup4}
  (a \cup b) \cup c = a \cup (b \cup c)
\end{equation}
が成り立つ.
そこで(\ref{sthmcup-cup=-cup3}), (\ref{sthmcup-cup=-cup4})から, 推論法則 \ref{ded=trans}によって
\[
  (a \cup c) \cup (b \cup c) = a \cup (b \cup c)
\]
が成り立つ.
故に定理 \ref{sthm-cup=}より(\ref{sthmcup-cup=-cup1})が成り立つ.
また定理 \ref{sthmcupch}より
\[
  c \cup a = a \cup c, ~~
  c \cup b = b \cup c
\]
が共に成り立つから, 定理 \ref{sthm-=}より
\begin{equation}
\label{sthmcup-cup=-cup5}
  c \cup a - c \cup b = a \cup c - b \cup c
\end{equation}
が成り立つ.
同じく定理 \ref{sthmcupch}より
\[
  b \cup c = c \cup b
\]
が成り立つから, 定理 \ref{sthm-=}より
\begin{equation}
\label{sthmcup-cup=-cup6}
  a - b \cup c = a - c \cup b
\end{equation}
が成り立つ.
そこで(\ref{sthmcup-cup=-cup5}), (\ref{sthmcup-cup=-cup1}), (\ref{sthmcup-cup=-cup6})から, 
推論法則 \ref{ded=trans}によって(\ref{sthmcup-cup=-cup2})が成り立つことがわかる.
\halmos




\mathstrut
\begin{thm}
\label{sthm--=-cup}%定理7.31%確認済
$a$, $b$, $c$を集合とするとき, 
\begin{align}
  \label{sthm--=-cup1}
  &(a - b) - c = a - b \cup c, \\
  \mbox{} \notag \\
  \label{sthm--=-cup2}
  &(a - c) - b = a - b \cup c
\end{align}
が共に成り立つ.
\end{thm}


\noindent{\bf 証明}
~$x$を$a$, $b$, $c$の中に自由変数として現れない, 定数でない文字とする.
このとき変数法則 \ref{val-}により, $x$は$a - b$の中に自由変数として現れない.
また変数法則 \ref{valcup}により, $x$は$b \cup c$の中にも自由変数として現れない.
さて定理 \ref{sthm-basis}より
\[
  x \in a - b \leftrightarrow x \in a \wedge x \notin b
\]
が成り立つから, 推論法則 \ref{dedaddeqw}により
\begin{equation}
\label{sthm--=-cup3}
  x \in a - b \wedge x \notin c \leftrightarrow (x \in a \wedge x \notin b) \wedge x \notin c
\end{equation}
が成り立つ.
またThm \ref{1awb1wclaw1bwc1}より
\begin{equation}
\label{sthm--=-cup4}
  (x \in a \wedge x \notin b) \wedge x \notin c 
  \leftrightarrow x \in a \wedge (x \notin b \wedge x \notin c)
\end{equation}
が成り立つ.
また定理 \ref{sthmcupnotin}と推論法則 \ref{dedeqch}により
\[
  x \notin b \wedge x \notin c \leftrightarrow x \notin b \cup c
\]
が成り立つから, 推論法則 \ref{dedaddeqw}により
\begin{equation}
\label{sthm--=-cup5}
  x \in a \wedge (x \notin b \wedge x \notin c) \leftrightarrow x \in a \wedge x \notin b \cup c
\end{equation}
が成り立つ.
そこで(\ref{sthm--=-cup3})---(\ref{sthm--=-cup5})から, 推論法則 \ref{dedeqtrans}によって
\[
  x \in a - b \wedge x \notin c \leftrightarrow x \in a \wedge x \notin b \cup c
\]
が成り立つことがわかる.
このことと$x$が定数でないことから, 定理 \ref{sthmalleqiset=}より
\[
  \{x \mid x \in a - b \wedge x \notin c\} = \{x \mid x \in a \wedge x \notin b \cup c\}
\]
が成り立つ.
ここで上述のように, $x$は$a - b$, $c$, $a$, $b \cup c$のいずれの中にも自由変数として現れないから, 
定義より上記の記号列は(\ref{sthm--=-cup1})と同じである.
故に(\ref{sthm--=-cup1})が成り立つ.
また定理 \ref{sthm-ch}より
\[
  (a - c) - b = (a - b) - c
\]
が成り立つから, これと(\ref{sthm--=-cup1})から, 
推論法則 \ref{ded=trans}によって(\ref{sthm--=-cup2})が成り立つ.
\halmos




\mathstrut
\begin{thm}
\label{sthm-cupdist}%定理7.32%確認済
$a$, $b$, $c$を集合とするとき, 
\begin{equation}
\label{sthm-cupdist1}
  a \cup b - c = (a - c) \cup (b - c)
\end{equation}
が成り立つ.
\end{thm}


\noindent{\bf 証明}
~$x$を$a$, $b$, $c$の中に自由変数として現れない, 定数でない文字とする.
このとき変数法則 \ref{val-}により, $x$は$a - c$, $b - c$の中に自由変数として現れない.
また変数法則 \ref{valcup}により, $x$は$a \cup b$の中にも自由変数として現れない.
さて定理 \ref{sthmcupbasis}より
\[
  x \in a \cup b \leftrightarrow x \in a \vee x \in b
\]
が成り立つから, 推論法則 \ref{dedaddeqw}により
\begin{equation}
\label{sthm-cupdist2}
  x \in a \cup b \wedge x \notin c \leftrightarrow (x \in a \vee x \in b) \wedge x \notin c
\end{equation}
が成り立つ.
またThm \ref{aw1bvc1l1awb1v1awc1}より
\begin{equation}
\label{sthm-cupdist3}
  (x \in a \vee x \in b) \wedge x \notin c 
  \leftrightarrow (x \in a \wedge x \notin c) \vee (x \in b \wedge x \notin c)
\end{equation}
が成り立つ.
また定理 \ref{sthm-basis}と推論法則 \ref{dedeqch}により
\[
  x \in a \wedge x \notin c \leftrightarrow x \in a - c, ~~
  x \in b \wedge x \notin c \leftrightarrow x \in b - c
\]
が共に成り立つから, 推論法則 \ref{dedaddeqv}により
\begin{equation}
\label{sthm-cupdist4}
  (x \in a \wedge x \notin c) \vee (x \in b \wedge x \notin c) 
  \leftrightarrow x \in a - c \vee x \in b - c
\end{equation}
が成り立つ.
そこで(\ref{sthm-cupdist2})---(\ref{sthm-cupdist4})から, 推論法則 \ref{dedeqtrans}によって
\[
  x \in a \cup b \wedge x \notin c \leftrightarrow x \in a - c \vee x \in b - c
\]
が成り立つことがわかる.
このことと$x$が定数でないことから, 定理 \ref{sthmalleqiset=}より
\[
  \{x \mid x \in a \cup b \wedge x \notin c\} = \{x \mid x \in a - c \vee x \in b - c\}
\]
が成り立つ.
ここで上述のように, $x$は$a \cup b$, $c$, $a - c$, $b - c$のいずれの中にも自由変数として現れないから, 
定義より上記の記号列は(\ref{sthm-cupdist1})と同じである.
故に(\ref{sthm-cupdist1})が成り立つ.
\halmos




\mathstrut
\begin{thm}
\label{sthm-cupdistab}%定理7.33%確認済
$a$と$b$を集合とするとき, 
\begin{align}
  \label{sthm-cupdistab1}
  &a \cup b - a = b - a, \\
  \mbox{} \notag \\
  \label{sthm-cupdistab2}
  &a \cup b - b = a - b
\end{align}
が共に成り立つ.
\end{thm}


\noindent{\bf 証明}
~$x$を$a$, $b$の中に自由変数として現れない, 定数でない文字とする.
このとき定理 \ref{sthmcupbasis}より
\[
  x \in a \cup b \leftrightarrow x \in a \vee x \in b
\]
が成り立つから, 推論法則 \ref{dedaddeqw}により
\begin{align}
  \label{sthm-cupdistab3}
  &x \in a \cup b \wedge x \notin a \leftrightarrow (x \in a \vee x \in b) \wedge x \notin a, \\
  \mbox{} \notag \\
  \label{sthm-cupdistab4}
  &x \in a \cup b \wedge x \notin b \leftrightarrow (x \in a \vee x \in b) \wedge x \notin b
\end{align}
が共に成り立つ.
またThm \ref{aw1bvc1l1awb1v1awc1}より
\begin{align}
  \label{sthm-cupdistab5}
  &(x \in a \vee x \in b) \wedge x \notin a 
  \leftrightarrow (x \in a \wedge x \notin a) \vee (x \in b \wedge x \notin a), \\
  \mbox{} \notag \\
  \label{sthm-cupdistab6}
  &(x \in a \vee x \in b) \wedge x \notin b 
  \leftrightarrow (x \in a \wedge x \notin b) \vee (x \in b \wedge x \notin b)
\end{align}
が共に成り立つ.
またThm \ref{n1awna1}より
\[
  \neg (x \in a \wedge x \notin a), ~~
  \neg (x \in b \wedge x \notin b)
\]
が共に成り立つから, 推論法則 \ref{dedavblbtrue2}により
\begin{align}
  \label{sthm-cupdistab7}
  &(x \in a \wedge x \notin a) \vee (x \in b \wedge x \notin a) 
  \leftrightarrow x \in b \wedge x \notin a, \\
  \mbox{} \notag \\
  \label{sthm-cupdistab8}
  &(x \in a \wedge x \notin b) \vee (x \in b \wedge x \notin b) 
  \leftrightarrow x \in a \wedge x \notin b
\end{align}
が共に成り立つ.
そこで(\ref{sthm-cupdistab3}), (\ref{sthm-cupdistab5}), (\ref{sthm-cupdistab7})から, 
推論法則 \ref{dedeqtrans}によって
\[
  x \in a \cup b \wedge x \notin a \leftrightarrow x \in b \wedge x \notin a
\]
が成り立つことがわかる.
また(\ref{sthm-cupdistab4}), (\ref{sthm-cupdistab6}), (\ref{sthm-cupdistab8})から, 
同じく推論法則 \ref{dedeqtrans}によって
\[
  x \in a \cup b \wedge x \notin b \leftrightarrow x \in a \wedge x \notin b
\]
が成り立つことがわかる.
これらのことと$x$が定数でないことから, 定理 \ref{sthmalleqiset=}より
\begin{align*}
  &\{x \mid x \in a \cup b \wedge x \notin a\} = \{x \mid x \in b \wedge x \notin a\}, \\
  \mbox{} \notag \\
  &\{x \mid x \in a \cup b \wedge x \notin b\} = \{x \mid x \in a \wedge x \notin b\}
\end{align*}
が共に成り立つ.
ここで$x$が$a$, $b$の中に自由変数として現れないことから, 
変数法則 \ref{valcup}により, $x$は$a \cup b$の中にも自由変数として現れないから, 
定義より上記の記号列はそれぞれ(\ref{sthm-cupdistab1}), (\ref{sthm-cupdistab2})と同じである.
故に(\ref{sthm-cupdistab1})と(\ref{sthm-cupdistab2})が共に成り立つ.
\halmos




\mathstrut
\begin{thm}
\label{sthm-&cup}%定理7.34%確認済
$a$, $b$, $c$を集合とするとき, 
\begin{align}
  \label{sthm-&cup1}
  &a \cup (b - c) = a \cup b - (c - a), \\
  \mbox{} \notag \\
  \label{sthm-&cup2}
  &(a - b) \cup c = a \cup c - (b - c)
\end{align}
が共に成り立つ.
\end{thm}


\noindent{\bf 証明}
~$x$を$a$, $b$, $c$の中に自由変数として現れない, 定数でない文字とする.
このとき変数法則 \ref{val-}により, $x$は$b - c$, $c - a$, $a - b$の中に自由変数として現れない.
また変数法則 \ref{valcup}により, $x$は$a \cup b$, $a \cup c$の中にも自由変数として現れない.
さて定理 \ref{sthm-basis}より
\[
  x \in b - c \leftrightarrow x \in b \wedge x \notin c, ~~
  x \in a - b \leftrightarrow x \in a \wedge x \notin b
\]
が共に成り立つから, 推論法則 \ref{dedaddeqv}により
\begin{align}
  \label{sthm-&cup3}
  &x \in a \vee x \in b - c \leftrightarrow x \in a \vee (x \in b \wedge x \notin c), \\
  \mbox{} \notag \\
  \label{sthm-&cup4}
  &x \in a - b \vee x \in c \leftrightarrow (x \in a \wedge x \notin b) \vee x \in c
\end{align}
が共に成り立つ.
またThm \ref{aw1bvc1l1awb1v1awc1}より
\begin{align}
  \label{sthm-&cup5}
  &x \in a \vee (x \in b \wedge x \notin c) 
  \leftrightarrow (x \in a \vee x \in b) \wedge (x \in a \vee x \notin c), \\
  \mbox{} \notag \\
  \label{sthm-&cup6}
  &(x \in a \wedge x \notin b) \vee x \in c 
  \leftrightarrow (x \in a \vee x \in c) \wedge (x \notin b \vee x \in c)
\end{align}
が共に成り立つ.
また定理 \ref{sthmcupbasis}と推論法則 \ref{dedeqch}により
\begin{align}
  \label{sthm-&cup7}
  &x \in a \vee x \in b \leftrightarrow x \in a \cup b, \\
  \mbox{} \notag \\
  \label{sthm-&cup8}
  &x \in a \vee x \in c \leftrightarrow x \in a \cup c
\end{align}
が共に成り立つ.
またThm \ref{avblbva}より
\begin{equation}
\label{sthm-&cup9}
  x \in a \vee x \notin c \leftrightarrow x \notin c \vee x \in a
\end{equation}
が成り立つ.
また定理 \ref{sthm-notin}と推論法則 \ref{dedeqch}により
\begin{align}
  \label{sthm-&cup10}
  &x \notin c \vee x \in a \leftrightarrow x \notin c - a, \\
  \mbox{} \notag \\
  \label{sthm-&cup11}
  &x \notin b \vee x \in c \leftrightarrow x \notin b - c
\end{align}
が共に成り立つ.
そこで(\ref{sthm-&cup9}), (\ref{sthm-&cup10})から, 推論法則 \ref{dedeqtrans}によって
\begin{equation}
\label{sthm-&cup12}
  x \in a \vee x \notin c \leftrightarrow x \notin c - a
\end{equation}
が成り立つ.
故に(\ref{sthm-&cup7})と(\ref{sthm-&cup12}), (\ref{sthm-&cup8})と(\ref{sthm-&cup11})から, 
それぞれ推論法則 \ref{dedaddeqw}により
\begin{align}
  \label{sthm-&cup13}
  &(x \in a \vee x \in b) \wedge (x \in a \vee x \notin c) 
  \leftrightarrow x \in a \cup b \wedge x \notin c - a, \\
  \mbox{} \notag \\
  \label{sthm-&cup14}
  &(x \in a \vee x \in c) \wedge (x \notin b \vee x \in c) 
  \leftrightarrow x \in a \cup c \wedge x \notin b - c
\end{align}
が成り立つ.
そこで(\ref{sthm-&cup3}), (\ref{sthm-&cup5}), (\ref{sthm-&cup13})から, 
推論法則 \ref{dedeqtrans}によって
\[
  x \in a \vee x \in b - c \leftrightarrow x \in a \cup b \wedge x \notin c - a
\]
が成り立つことがわかる.
また(\ref{sthm-&cup4}), (\ref{sthm-&cup6}), (\ref{sthm-&cup14})から, 
同じく推論法則 \ref{dedeqtrans}によって
\[
  x \in a - b \vee x \in c \leftrightarrow x \in a \cup c \wedge x \notin b - c
\]
が成り立つことがわかる.
これらのことと$x$が定数でないことから, 定理 \ref{sthmalleqiset=}より
\begin{align*}
  &\{x \mid x \in a \vee x \in b - c\} = \{x \mid x \in a \cup b \wedge x \notin c - a\}, \\
  \mbox{} \notag \\
  &\{x \mid x \in a - b \vee x \in c\} = \{x \mid x \in a \cup c \wedge x \notin b - c\}
\end{align*}
が共に成り立つ.
ここで上述のように, $x$は$a$, $b - c$, $a \cup b$, $c - a$, $a - b$, $c$, $a \cup c$の
いずれの中にも自由変数として現れないから, 定義より上記の記号列はそれぞれ
(\ref{sthm-&cup1}), (\ref{sthm-&cup2})と同じである.
故に(\ref{sthm-&cup1})と(\ref{sthm-&cup2})が共に成り立つ.
\halmos




\mathstrut
\begin{thm}
\label{sthm-&cupab}%定理7.35%新規%確認済
$a$と$b$を集合とするとき, 
\begin{align}
  \label{sthm-&cupab1}
  &a \cup (b - a) = a \cup b, \\
  \mbox{} \notag \\
  \label{sthm-&cupab2}
  &(a - b) \cup b = a \cup b
\end{align}
が共に成り立つ.
\end{thm}


\noindent{\bf 証明}
~$x$を$a$, $b$の中に自由変数として現れない, 定数でない文字とする.
このとき定理 \ref{sthm-basis}より
\[
  x \in b - a \leftrightarrow x \in b \wedge x \notin a, ~~
  x \in a - b \leftrightarrow x \in a \wedge x \notin b
\]
が共に成り立つから, 推論法則 \ref{dedaddeqv}により
\begin{align}
  \label{sthm-&cupab3}
  &x \in a \vee x \in b - a \leftrightarrow x \in a \vee (x \in b \wedge x \notin a), \\
  \mbox{} \notag \\
  \label{sthm-&cupab4}
  &x \in a - b \vee x \in b \leftrightarrow (x \in a \wedge x \notin b) \vee x \in b
\end{align}
が共に成り立つ.
またThm \ref{aw1bvc1l1awb1v1awc1}より
\begin{align}
  \label{sthm-&cupab5}
  &x \in a \vee (x \in b \wedge x \notin a) 
  \leftrightarrow (x \in a \vee x \in b) \wedge (x \in a \vee x \notin a), \\
  \mbox{} \notag \\
  \label{sthm-&cupab6}
  &(x \in a \wedge x \notin b) \vee x \in b 
  \leftrightarrow (x \in a \vee x \in b) \wedge (x \notin b \vee x \in b)
\end{align}
が共に成り立つ.
またThm \ref{avna}より
\[
  x \in a \vee x \notin a
\]
が成り立つ.
またThm \ref{nnata}より, $\neg \neg (x \in b) \to x \in b$, 即ち
\[
  x \notin b \vee x \in b
\]
が成り立つ.
故にこれらから, それぞれ推論法則 \ref{dedawblatrue2}により
\begin{align}
  \label{sthm-&cupab7}
  &(x \in a \vee x \in b) \wedge (x \in a \vee x \notin a) \leftrightarrow x \in a \vee x \in b, \\
  \mbox{} \notag \\
  \label{sthm-&cupab8}
  &(x \in a \vee x \in b) \wedge (x \notin b \vee x \in b) \leftrightarrow x \in a \vee x \in b
\end{align}
が成り立つ.
そこで(\ref{sthm-&cupab3}), (\ref{sthm-&cupab5}), (\ref{sthm-&cupab7})から, 
推論法則 \ref{dedeqtrans}によって
\[
  x \in a \vee x \in b - a \leftrightarrow x \in a \vee x \in b
\]
が成り立つことがわかる.
また(\ref{sthm-&cupab4}), (\ref{sthm-&cupab6}), (\ref{sthm-&cupab8})から, 
同じく推論法則 \ref{dedeqtrans}によって
\[
  x \in a - b \vee x \in b \leftrightarrow x \in a \vee x \in b
\]
が成り立つことがわかる.
これらのことと$x$が定数でないことから, 定理 \ref{sthmalleqiset=}より
\begin{align*}
  &\{x \mid x \in a \vee x \in b - a\} = \{x \mid x \in a \vee x \in b\}, \\
  \mbox{} \notag \\
  &\{x \mid x \in a - b \vee x \in b\} = \{x \mid x \in a \vee x \in b\}
\end{align*}
が共に成り立つ.
ここで$x$が$a$, $b$の中に自由変数として現れないことから, 
変数法則 \ref{val-}により, $x$は$b - a$, $a - b$の中にも自由変数として現れないから, 
定義より上記の記号列はそれぞれ(\ref{sthm-&cupab1}), (\ref{sthm-&cupab2})と同じである.
故に(\ref{sthm-&cupab1})と(\ref{sthm-&cupab2})が共に成り立つ.
\halmos




\mathstrut
\begin{thm}
\label{sthm-&cupabeq}%定理7.36%新規%要る?%確認済
$a$と$b$を集合とするとき, 
\begin{align}
  \label{sthm-&cupabeq1}
  &a \cup (b - a) = b \leftrightarrow a \subset b, \\
  \mbox{} \notag \\
  \label{sthm-&cupabeq2}
  &(a - b) \cup b = a \leftrightarrow b \subset a
\end{align}
が共に成り立つ.
またこれらから, 次の1)---4)が成り立つ.

1)
$a \cup (b - a) = b$ならば, $a \subset b$.

2)
$a \subset b$ならば, $a \cup (b - a) = b$.

3)
$(a - b) \cup b = a$ならば, $b \subset a$.

4)
$b \subset a$ならば, $(a - b) \cup b = a$.
\end{thm}


\noindent{\bf 証明}
~定理 \ref{sthm-&cupab}より
\[
  a \cup (b - a) = a \cup b, ~~
  (a - b) \cup b = a \cup b
\]
が共に成り立つから, 推論法則 \ref{dedaddeq=}により
\begin{align}
  \label{sthm-&cupabeq3}
  &a \cup (b - a) = b \leftrightarrow a \cup b = b, \\
  \mbox{} \notag \\
  \label{sthm-&cupabeq4}
  &(a - b) \cup b = a \leftrightarrow a \cup b = a
\end{align}
が共に成り立つ.
また定理 \ref{sthmacupb=a}より
\begin{align}
  \label{sthm-&cupabeq5}
  &a \cup b = b \leftrightarrow a \subset b, \\
  \mbox{} \notag \\
  \label{sthm-&cupabeq6}
  &a \cup b = a \leftrightarrow b \subset a
\end{align}
が共に成り立つ.
そこで(\ref{sthm-&cupabeq3})と(\ref{sthm-&cupabeq5}), 
(\ref{sthm-&cupabeq4})と(\ref{sthm-&cupabeq6})から, 
それぞれ推論法則 \ref{dedeqtrans}によって(\ref{sthm-&cupabeq1}), (\ref{sthm-&cupabeq2})が成り立つ.

\noindent
1), 2)
(\ref{sthm-&cupabeq1})と推論法則 \ref{dedeqfund}によって明らか.

\noindent
3), 4)
(\ref{sthm-&cupabeq2})と推論法則 \ref{dedeqfund}によって明らか.
\halmos




\mathstrut
\begin{thm}
\label{sthma-bsset}%定理7.37%新規%確認済
$a$と$b$を集合, $R$を関係式とし, $x$を$a$及び$b$の中に自由変数として現れない文字とする.
このとき
\begin{equation}
\label{sthma-bsset1}
  a - \{x \in b \mid R\} = (a - b) \cup \{x \in a \mid \neg R\}
\end{equation}
が成り立つ.
\end{thm}


\noindent{\bf 証明}
~$y$を$a$, $b$, $R$の中に自由変数として現れない, 定数でない文字とする.
このとき変数法則 \ref{valsset}により, $y$は$\{x \in b \mid R\}$の中に自由変数として現れない.
また変数法則 \ref{valfund}, \ref{valsset}により, 
$y$は$\{x \in a \mid \neg R\}$の中にも自由変数として現れない.
また変数法則 \ref{val-}により, $y$は$a - b$の中にも自由変数として現れない.
さて$x$が$b$の中に自由変数として現れないことから, 定理 \ref{sthmssetbasis}より
\[
  y \in \{x \in b \mid R\} \leftrightarrow y \in b \wedge (y|x)(R), 
\]
即ち
\[
  y \in \{x \in b \mid R\} \leftrightarrow \neg (y \notin b \vee \neg (y|x)(R))
\]
が成り立つから, 推論法則 \ref{dedeqcp}により
\[
  y \notin \{x \in b \mid R\} \leftrightarrow y \notin b \vee \neg (y|x)(R)
\]
が成り立つ.
故に推論法則 \ref{dedaddeqw}により
\begin{equation}
\label{sthma-bsset2}
  y \in a \wedge y \notin \{x \in b \mid R\} 
  \leftrightarrow y \in a \wedge (y \notin b \vee \neg (y|x)(R))
\end{equation}
が成り立つ.
またThm \ref{aw1bvc1l1awb1v1awc1}より
\begin{equation}
\label{sthma-bsset3}
  y \in a \wedge (y \notin b \vee \neg (y|x)(R)) 
  \leftrightarrow (y \in a \wedge y \notin b) \vee (y \in a \wedge \neg (y|x)(R))
\end{equation}
が成り立つ.
また定理 \ref{sthm-basis}と推論法則 \ref{dedeqch}により
\begin{equation}
\label{sthma-bsset4}
  y \in a \wedge y \notin b \leftrightarrow y \in a - b
\end{equation}
が成り立つ.
また$x$が$a$の中に自由変数として現れないことから, 
定理 \ref{sthmssetbasis}と推論法則 \ref{dedeqch}により
\[
  y \in a \wedge (y|x)(\neg R) \leftrightarrow y \in \{x \in a \mid \neg R\}
\]
が成り立つが, 代入法則 \ref{substfund}によればこの記号列は
\begin{equation}
\label{sthma-bsset5}
  y \in a \wedge \neg (y|x)(R) \leftrightarrow y \in \{x \in a \mid \neg R\}
\end{equation}
と一致するから, これが成り立つ.
故に(\ref{sthma-bsset4}), (\ref{sthma-bsset5})から, 推論法則 \ref{dedaddeqv}により
\begin{equation}
\label{sthma-bsset6}
  (y \in a \wedge y \notin b) \vee (y \in a \wedge \neg (y|x)(R)) 
  \leftrightarrow y \in a - b \vee y \in \{x \in a \mid \neg R\}
\end{equation}
が成り立つ.
そこで(\ref{sthma-bsset2}), (\ref{sthma-bsset3}), (\ref{sthma-bsset6})から, 
推論法則 \ref{dedeqtrans}によって
\[
  y \in a \wedge y \notin \{x \in b \mid R\} 
  \leftrightarrow y \in a - b \vee y \in \{x \in a \mid \neg R\}
\]
が成り立つことがわかる.
このことと$y$が定数でないことから, 定理 \ref{sthmalleqiset=}より
\[
  \{y \mid y \in a \wedge y \notin \{x \in b \mid R\}\} 
  = \{y \mid y \in a - b \vee y \in \{x \in a \mid \neg R\}\}
\]
が成り立つ.
ここで上述のように, $y$は$a$, $\{x \in b \mid R\}$, $a - b$, $\{x \in a \mid \neg R\}$の
いずれの中にも自由変数として現れないから, 定義より上記の記号列は(\ref{sthma-bsset1})と同じである.
故に(\ref{sthma-bsset1})が成り立つ.
\halmos




\mathstrut
\begin{thm}
\label{sthmiset-sset&cup}%定理7.38%新規%確認済
$a$を集合, $R$と$S$を関係式とし, $x$を$a$の中に自由変数として現れない文字とする.
このとき
\begin{equation}
\label{sthmiset-sset&cup1}
  {\rm Set}_{x}(R) 
  \to \{x \mid R\} - \{x \in a \mid S\} = (\{x \mid R\} - a) \cup \{x \mid R \wedge \neg S\}
\end{equation}
が成り立つ.
またこのことから, 次の(\ref{sthmiset-sset&cup2})が成り立つ.
\begin{equation}
\label{sthmiset-sset&cup2}
  R \text{が} x \text{について集合を作り得るならば,} ~
  \{x \mid R\} - \{x \in a \mid S\} = (\{x \mid R\} - a) \cup \{x \mid R \wedge \neg S\}.
\end{equation}
\end{thm}


\noindent{\bf 証明}
~定理 \ref{sthmisetsset}より
\begin{equation}
\label{sthmiset-sset&cup3}
  {\rm Set}_{x}(R) \to \{x \in \{x \mid R\} \mid \neg S\} = \{x \mid R \wedge \neg S\}
\end{equation}
が成り立つ.
また定理 \ref{sthmcup=}より
\begin{multline}
\label{sthmiset-sset&cup4}
  \{x \in \{x \mid R\} \mid \neg S\} = \{x \mid R \wedge \neg S\} \\
  \to (\{x \mid R\} - a) \cup \{x \in \{x \mid R\} \mid \neg S\} = (\{x \mid R\} - a) \cup \{x \mid R \wedge \neg S\}
\end{multline}
が成り立つ.
また変数法則 \ref{valiset}により, $x$は$\{x \mid R\}$の中に自由変数として現れないから, 
このことと$x$が$a$の中に自由変数として現れないことから, 定理 \ref{sthma-bsset}より
\[
  \{x \mid R\} - \{x \in a \mid S\} = (\{x \mid R\} - a) \cup \{x \in \{x \mid R\} \mid \neg S\}
\]
が成り立つ.
故に推論法則 \ref{dedaddeq=}により
\begin{multline*}
  \{x \mid R\} - \{x \in a \mid S\} = (\{x \mid R\} - a) \cup \{x \mid R \wedge \neg S\} \\
  \leftrightarrow (\{x \mid R\} - a) \cup \{x \in \{x \mid R\} \mid \neg S\} = (\{x \mid R\} - a) \cup \{x \mid R \wedge \neg S\}
\end{multline*}
が成り立つ.
故に推論法則 \ref{dedequiv}により
\begin{multline}
\label{sthmiset-sset&cup5}
  (\{x \mid R\} - a) \cup \{x \in \{x \mid R\} \mid \neg S\} = (\{x \mid R\} - a) \cup \{x \mid R \wedge \neg S\} \\
  \to \{x \mid R\} - \{x \in a \mid S\} = (\{x \mid R\} - a) \cup \{x \mid R \wedge \neg S\}
\end{multline}
が成り立つ.
そこで(\ref{sthmiset-sset&cup3})---(\ref{sthmiset-sset&cup5})から, 
推論法則 \ref{dedmmp}によって(\ref{sthmiset-sset&cup1})が成り立つことがわかる.
(\ref{sthmiset-sset&cup2})が成り立つことは, 
(\ref{sthmiset-sset&cup1})と推論法則 \ref{dedmp}によって明らかである.
\halmos




\mathstrut
\begin{thm}
\label{sthmsset-sset&cup}%定理7.39%新規%確認済
$a$と$b$を集合, $R$と$S$を関係式とし, $x$を$a$及び$b$の中に自由変数として現れない文字とする.
このとき
\begin{equation}
\label{sthmsset-sset&cup1}
  \{x \in a \mid R\} - \{x \in b \mid S\} 
  = \{x \in a - b \mid R\} \cup \{x \in a \mid R \wedge \neg S\}
\end{equation}
が成り立つ.
\end{thm}


\noindent{\bf 証明}
~変数法則 \ref{valsset}により, $x$は$\{x \in a \mid R\}$の中に自由変数として現れないから, 
このことと$x$が$b$の中に自由変数として現れないことから, 定理 \ref{sthma-bsset}より
\begin{equation}
\label{sthmsset-sset&cup2}
  \{x \in a \mid R\} - \{x \in b \mid S\} 
  = (\{x \in a \mid R\} - b) \cup \{x \in \{x \in a \mid R\} \mid \neg S\}
\end{equation}
が成り立つ.
また$x$が$a$, $b$の中に自由変数として現れないことから, 定理 \ref{sthmsset-}より
\begin{equation}
\label{sthmsset-sset&cup3}
  \{x \in a \mid R\} - b = \{x \in a - b \mid R\}
\end{equation}
が成り立つ.
また$x$が$a$の中に自由変数として現れないことから, 定理 \ref{sthmssetsset}より
\begin{equation}
\label{sthmsset-sset&cup4}
  \{x \in \{x \in a \mid R\} \mid \neg S\} = \{x \in a \mid R \wedge \neg S\}
\end{equation}
が成り立つ.
故に(\ref{sthmsset-sset&cup3}), (\ref{sthmsset-sset&cup4})から, 定理 \ref{sthmcup=}より
\begin{equation}
\label{sthmsset-sset&cup5}
  (\{x \in a \mid R\} - b) \cup \{x \in \{x \in a \mid R\} \mid \neg S\} 
  = \{x \in a - b \mid R\} \cup \{x \in a \mid R \wedge \neg S\}
\end{equation}
が成り立つ.
そこで(\ref{sthmsset-sset&cup2}), (\ref{sthmsset-sset&cup5})から, 
推論法則 \ref{ded=trans}によって(\ref{sthmsset-sset&cup1})が成り立つ.
\halmos




\mathstrut
\begin{thm}
\label{sthmcupempty}%定理7.40%新規%確認済
$a$と$b$を集合とするとき, 
\begin{equation}
\label{sthmcupempty1}
  a \cup b = \phi \leftrightarrow a = \phi \wedge b = \phi
\end{equation}
が成り立つ.
またこのことから, 次の1), 2)が成り立つ.

1)
$a \cup b$が空ならば, $a$と$b$は共に空である.

2)
$a$と$b$が共に空ならば, $a \cup b$は空である.
\end{thm}


\noindent{\bf 証明}
~定理 \ref{sthmemptysubset=eq}と推論法則 \ref{dedeqch}により
\begin{equation}
\label{sthmcupempty2}
  a \cup b = \phi \leftrightarrow a \cup b \subset \phi
\end{equation}
が成り立つ.
また定理 \ref{sthmacupbsubsetc}より
\begin{equation}
\label{sthmcupempty3}
  a \cup b \subset \phi \leftrightarrow a \subset \phi \wedge b \subset \phi
\end{equation}
が成り立つ.
また定理 \ref{sthmemptysubset=eq}より
\[
  a \subset \phi \leftrightarrow a = \phi, ~~
  b \subset \phi \leftrightarrow b = \phi
\]
が共に成り立つから, 推論法則 \ref{dedaddeqw}により
\begin{equation}
\label{sthmcupempty4}
  a \subset \phi \wedge b \subset \phi \leftrightarrow a = \phi \wedge b = \phi
\end{equation}
が成り立つ.
そこで(\ref{sthmcupempty2})---(\ref{sthmcupempty4})から, 
推論法則 \ref{dedeqtrans}によって(\ref{sthmcupempty1})が成り立つことがわかる.
1), 2)が成り立つことは, (\ref{sthmcupempty1})と
推論法則 \ref{dedwedge}, \ref{dedeqfund}によって明らかである.
\halmos




\mathstrut
\begin{thm}
\label{sthmacupempty}%定理7.41%確認済
$a$を集合とするとき, 
\[
  a \cup \phi = a, ~~
  \phi \cup a = a
\]
が共に成り立つ.
\end{thm}


\noindent{\bf 証明}
~定理 \ref{sthmemptysubset}より$\phi \subset a$が成り立つから, 
定理 \ref{sthmacupb=a}より$a \cup \phi = a$と$\phi \cup a = a$が共に成り立つ.
\halmos
%ここまで確認



\mathstrut
\begin{defo}
\label{cap}%変形21%確認済
$\mathscr{T}$を特殊記号として$\in$を持つ理論とし, $a$と$b$を$\mathscr{T}$の記号列とする.
また$x$と$y$を共に$a$及び$b$の中に自由変数として現れない文字とする.
このとき
\[
  \{x \in a \mid x \in b\} \equiv \{y \in a \mid y \in b\}
\]
が成り立つ.
\end{defo}


\noindent{\bf 証明}
~$x$と$y$が同じ文字ならば明らかだから, 以下$x$と$y$は異なる文字であるとする.
このとき$y$が$x$と異なり, $b$の中に自由変数として現れないことから, 
変数法則 \ref{valfund}により, $y$は$x \in b$の中に自由変数として現れない.
このことと$x$, $y$が共に$a$の中に自由変数として現れないことから, 代入法則 \ref{substssettrans}により
\[
  \{x \in a \mid x \in b\} \equiv \{y \in a \mid (y|x)(x \in b)\}
\]
が成り立つ.
また$x$が$b$の中に自由変数として現れないことから, 代入法則 \ref{substfree}, \ref{substfund}により
\[
  (y|x)(x \in b) \equiv y \in b
\]
が成り立つ.
故に本法則が成り立つ.
\halmos




\mathstrut
\begin{defi}
\label{defcap}%定義2%確認済
$\mathscr{T}$を特殊記号として$\in$を持つ理論とし, $a$と$b$を$\mathscr{T}$の記号列とする.
また$x$と$y$を共に$a$及び$b$の中に自由変数として現れない文字とする.
このとき変形法則 \ref{cap}によれば, 
$\{x \in a \mid x \in b\}$と$\{y \in a \mid y \in b\}$は同じ記号列となる.
$a$と$b$に対して定まるこの記号列を, $(a) \cap (b)$と書き表す (括弧は適宜省略する).
これを$a$と$b$の\textbf{共通部分} (intersection), 
または\textbf{共通部}, \textbf{共部分}, \textbf{交叉}, \textbf{交わり} (meet) などという.
\end{defi}




\mathstrut%確認済%koko
以下の変数法則 \ref{valcap}, 一般代入法則 \ref{gsubstcap}, 代入法則 \ref{substcap}, 
構成法則 \ref{formcap}では, $\mathscr{T}$を特殊記号として$\in$を持つ理論とし, 
これらの法則における``記号列'', ``集合''とは, 
それぞれ$\mathscr{T}$の記号列, $\mathscr{T}$の対象式のこととする.




\mathstrut
\begin{valu}
\label{valcap}%変数32%確認済
$a$と$b$を記号列とし, $x$を文字とする.
$x$が$a$及び$b$の中に自由変数として現れなければ, $x$は$a \cap b$の中に自由変数として現れない.
\end{valu}


\noindent{\bf 証明}
~このとき定義から$a \cap b$は$\{x \in a \mid x \in b\}$と同じである.
変数法則 \ref{valsset}によれば, $x$はこの中に自由変数として現れない.
\halmos




\mathstrut
\begin{gsub}
\label{gsubstcap}%一般代入35%新規%確認済
$a$と$b$を記号列とする.
また$n$を自然数とし, $T_{1}, T_{2}, \cdots, T_{n}$を記号列とする.
また$x_{1}, x_{2}, \cdots, x_{n}$を, どの二つも互いに異なる文字とする.
このとき
\[
  (T_{1}|x_{1}, T_{2}|x_{2}, \cdots, T_{n}|x_{n})(a \cap b) 
  \equiv (T_{1}|x_{1}, T_{2}|x_{2}, \cdots, T_{n}|x_{n})(a) \cap (T_{1}|x_{1}, T_{2}|x_{2}, \cdots, T_{n}|x_{n})(b)
\]
が成り立つ.
\end{gsub}


\noindent{\bf 証明}
~$y$を$x_{1}, x_{2}, \cdots, x_{n}$のいずれとも異なり, 
$a, b, T_{1}, T_{2}, \cdots, T_{n}$のいずれの中にも自由変数として現れない文字とする.
このとき定義から$a \cap b$は$\{y \in a \mid y \in b\}$と同じだから, 
\begin{equation}
\label{gsubstcap1}
  (T_{1}|x_{1}, T_{2}|x_{2}, \cdots, T_{n}|x_{n})(a \cap b) 
  \equiv (T_{1}|x_{1}, T_{2}|x_{2}, \cdots, T_{n}|x_{n})(\{y \in a \mid y \in b\})
\end{equation}
である.
また$y$が$x_{1}, x_{2}, \cdots, x_{n}$のいずれとも異なり, 
$T_{1}, T_{2}, \cdots, T_{n}$のいずれの中にも自由変数として現れないことから, 
一般代入法則 \ref{gsubstsset}により
\begin{multline}
\label{gsubstcap2}
  (T_{1}|x_{1}, T_{2}|x_{2}, \cdots, T_{n}|x_{n})(\{y \in a \mid y \in b\}) \\
  \equiv \{y \in (T_{1}|x_{1}, T_{2}|x_{2}, \cdots, T_{n}|x_{n})(a) \mid (T_{1}|x_{1}, T_{2}|x_{2}, \cdots, T_{n}|x_{n})(y \in b)\}
\end{multline}
が成り立つ.
また$y$が$x_{1}, x_{2}, \cdots, x_{n}$のいずれとも異なることと一般代入法則 \ref{gsubstfund}により, 
\begin{equation}
\label{gsubstcap3}
  (T_{1}|x_{1}, T_{2}|x_{2}, \cdots, T_{n}|x_{n})(y \in b) 
  \equiv y \in (T_{1}|x_{1}, T_{2}|x_{2}, \cdots, T_{n}|x_{n})(b)
\end{equation}
が成り立つ.
そこで(\ref{gsubstcap1})---(\ref{gsubstcap3})からわかるように, 
$(T_{1}|x_{1}, T_{2}|x_{2}, \cdots, T_{n}|x_{n})(a \cap b)$は
\begin{equation}
\label{gsubstcap4}
  \{y \in (T_{1}|x_{1}, T_{2}|x_{2}, \cdots, T_{n}|x_{n})(a) \mid y \in (T_{1}|x_{1}, T_{2}|x_{2}, \cdots, T_{n}|x_{n})(b)\}
\end{equation}
と一致する.
ここで$y$が$a, b, T_{1}, T_{2}, \cdots, T_{n}$のいずれの中にも自由変数として現れないことから, 
変数法則 \ref{valgsubst}により, $y$は$(T_{1}|x_{1}, T_{2}|x_{2}, \cdots, T_{n}|x_{n})(a)$及び
$(T_{1}|x_{1}, T_{2}|x_{2}, \cdots, T_{n}|x_{n})(b)$の中に自由変数として現れない.
故に定義から, (\ref{gsubstcap4})は
\[
  (T_{1}|x_{1}, T_{2}|x_{2}, \cdots, T_{n}|x_{n})(a) \cap (T_{1}|x_{1}, T_{2}|x_{2}, \cdots, T_{n}|x_{n})(b)
\]
と同じである.
故に本法則が成り立つ.
\halmos




\mathstrut
\begin{subs}
\label{substcap}%代入42%確認済
$a$, $b$, $T$を記号列とし, $x$を文字とする.
このとき
\[
  (T|x)(a \cap b) \equiv (T|x)(a) \cap (T|x)(b)
\]
が成り立つ.
\end{subs}


\noindent{\bf 証明}
~一般代入法則 \ref{gsubstcap}において, $n$が$1$の場合である.
\halmos




\mathstrut
\begin{form}
\label{formcap}%構成49%確認済
$a$と$b$が集合ならば, $a \cap b$は集合である.
\end{form}


\noindent{\bf 証明}
~$x$を$a$, $b$の中に自由変数として現れない文字とするとき, 
定義より$a \cap b$は$\{x \in a \mid x \in b\}$である.
$a$と$b$が集合のとき, これが集合となることは, 
構成法則 \ref{formfund}, \ref{formsset}によって直ちにわかる.
\halmos




\mathstrut
\begin{thm}
\label{sthmcapbasis}%定理7.42%確認済
$a$, $b$, $c$を集合とするとき, 
\begin{equation}
\label{sthmcapbasis1}
  c \in a \cap b \leftrightarrow c \in a \wedge c \in b
\end{equation}
が成り立つ.
またこのことから, 次の1), 2)が成り立つ.

1)
$c \in a \cap b$ならば, $c \in a$と$c \in b$が共に成り立つ.

2)
$c \in a$と$c \in b$が共に成り立てば, $c \in a \cap b$.
\end{thm}


\noindent{\bf 証明}
~$x$を$a$, $b$の中に自由変数として現れない文字とするとき, 定理 \ref{sthmssetbasis}より
\[
  c \in \{x \in a \mid x \in b\} \leftrightarrow c \in a \wedge (c|x)(x \in b)
\]
が成り立つが, 定義と代入法則 \ref{substfree}, \ref{substfund}によれば
この記号列は(\ref{sthmcapbasis1})と一致するから, (\ref{sthmcapbasis1})が成り立つ.
1), 2)が成り立つことは, (\ref{sthmcapbasis1})と
推論法則 \ref{dedwedge}, \ref{dedeqfund}によって明らかである.
\halmos




\mathstrut
\begin{thm}
\label{sthmcapnotin}%定理7.43%新規%確認済
$a$, $b$, $c$を集合とするとき, 
\begin{equation}
\label{sthmcapnotin1}
  c \notin a \cap b \leftrightarrow c \notin a \vee c \notin b
\end{equation}
が成り立つ.
またこのことから, 次の1), 2)が成り立つ.

1)
$c \notin a \cap b$ならば, $c \notin a \vee c \notin b$.

2)
$c \notin a$ならば, $c \notin a \cap b$.
また$c \notin b$ならば, $c \notin a \cap b$.
\end{thm}


\noindent{\bf 証明}
~定理 \ref{sthmcapbasis}より, 
\[
  c \in a \cap b \leftrightarrow c \in a \wedge c \in b, 
\]
即ち
\[
  c \in a \cap b \leftrightarrow \neg (c \notin a \vee c \notin b)
\]
が成り立つから, 推論法則 \ref{dedeqcp}により(\ref{sthmcapnotin1})が成り立つ.

\noindent
1)
(\ref{sthmcapnotin1})と推論法則 \ref{dedeqfund}によって明らか.

\noindent
2)
(\ref{sthmcapnotin1})と推論法則 \ref{dedvee}, \ref{dedeqfund}によって明らか.
\halmos




\mathstrut
\begin{thm}
\label{sthmsubsetcap}%定理7.44%確認済
$a$と$b$を集合とするとき, 
\[
  a \cap b \subset a, ~~
  a \cap b \subset b
\]
が共に成り立つ.
\end{thm}


\noindent{\bf 証明}
~$x$を$a$, $b$の中に自由変数として現れない, 定数でない文字とする.
このとき変数法則 \ref{valcap}により, $x$は$a \cap b$の中に自由変数として現れない.
また定理 \ref{sthmcapbasis}と推論法則 \ref{dedequiv}により
\[
  x \in a \cap b \to x \in a \wedge x \in b
\]
が成り立つから, 推論法則 \ref{dedprewedge}により
\[
  x \in a \cap b \to x \in a, ~~
  x \in a \cap b \to x \in b
\]
が共に成り立つ.
このことと, $x$が定数でなく, 上述のように$a$, $b$, $a \cap b$のいずれの記号列の中にも
自由変数として現れないことから, 
定理 \ref{sthmsubsetconst}より$a \cap b \subset a$と$a \cap b \subset b$が共に成り立つ.
\halmos




\mathstrut
\begin{thm}
\label{sthmacapbsubsetc}%定理7.45%新規%確認済
$a$, $b$, $c$を集合とするとき, 
\begin{align}
  \label{sthmacapbsubsetc1}
  &a \subset c \to a \cap b \subset c, \\
  \mbox{} \notag \\
  \label{sthmacapbsubsetc2}
  &b \subset c \to a \cap b \subset c
\end{align}
が共に成り立つ.
またこれらから, 次の1), 2)が成り立つ.

1)
$a \subset c$ならば, $a \cap b \subset c$.

2)
$b \subset c$ならば, $a \cap b \subset c$.
\end{thm}


\noindent{\bf 証明}
~定理 \ref{sthmsubsetcap}より
\[
  a \cap b \subset a, ~~
  a \cap b \subset b
\]
が共に成り立つから, 推論法則 \ref{dedatawbtrue2}により
\begin{align}
  \label{sthmacapbsubsetc3}
  &a \subset c \to a \cap b \subset a \wedge a \subset c, \\
  \mbox{} \notag \\
  \label{sthmacapbsubsetc4}
  &b \subset c \to a \cap b \subset b \wedge b \subset c
\end{align}
が共に成り立つ.
また定理 \ref{sthmsubsettrans}より
\begin{align}
  \label{sthmacapbsubsetc5}
  &a \cap b \subset a \wedge a \subset c \to a \cap b \subset c, \\
  \mbox{} \notag \\
  \label{sthmacapbsubsetc6}
  &a \cap b \subset b \wedge b \subset c \to a \cap b \subset c
\end{align}
が共に成り立つ.
そこで(\ref{sthmacapbsubsetc3})と(\ref{sthmacapbsubsetc5}), 
(\ref{sthmacapbsubsetc4})と(\ref{sthmacapbsubsetc6})から, 
それぞれ推論法則 \ref{dedmmp}によって(\ref{sthmacapbsubsetc1}), (\ref{sthmacapbsubsetc2})が成り立つ.

\noindent
1)
(\ref{sthmacapbsubsetc1})と推論法則 \ref{dedmp}によって明らか.

\noindent
2)
(\ref{sthmacapbsubsetc2})と推論法則 \ref{dedmp}によって明らか.
\halmos




\mathstrut
\begin{thm}
\label{sthmcsubsetacapb}%定理7.46%確認済
$a$, $b$, $c$を集合とするとき, 
\begin{equation}
\label{sthmcsubsetacapb1}
  c \subset a \cap b \leftrightarrow c \subset a \wedge c \subset b
\end{equation}
が成り立つ.
またこのことから, 次の1), 2)が成り立つ.

1)
$c \subset a \cap b$ならば, $c \subset a$と$c \subset b$が共に成り立つ.

2)
$c \subset a$と$c \subset b$が共に成り立てば, $c \subset a \cap b$.
\end{thm}


\noindent{\bf 証明}
~$x$を$a$, $b$, $c$の中に自由変数として現れない, 定数でない文字とする.
このとき変数法則 \ref{valcap}により, $x$は$a \cap b$の中に自由変数として現れない.
さて定理 \ref{sthmcapbasis}より
\[
  x \in a \cap b \leftrightarrow x \in a \wedge x \in b
\]
が成り立つから, 推論法則 \ref{dedaddeqt}により
\begin{equation}
\label{sthmcsubsetacapb2}
  (x \in c \to x \in a \cap b) \leftrightarrow (x \in c \to x \in a \wedge x \in b)
\end{equation}
が成り立つ.
またThm \ref{1atbwc1l1atb1w1atc1}より
\begin{equation}
\label{sthmcsubsetacapb3}
  (x \in c \to x \in a \wedge x \in b) 
  \leftrightarrow (x \in c \to x \in a) \wedge (x \in c \to x \in b)
\end{equation}
が成り立つ.
そこで(\ref{sthmcsubsetacapb2}), (\ref{sthmcsubsetacapb3})から, 推論法則 \ref{dedeqtrans}によって
\[
  (x \in c \to x \in a \cap b) \leftrightarrow (x \in c \to x \in a) \wedge (x \in c \to x \in b)
\]
が成り立つ.
このことと$x$が定数でないことから, 推論法則 \ref{dedalleqquansepconst}により
\[
  \forall x(x \in c \to x \in a \cap b) 
  \leftrightarrow \forall x((x \in c \to x \in a) \wedge (x \in c \to x \in b))
\]
が成り立つ.
ここで上述のように, $x$は$c$, $a \cap b$の中に自由変数として現れないから, 定義より上記の記号列は
\begin{equation}
\label{sthmcsubsetacapb4}
  c \subset a \cap b \leftrightarrow \forall x((x \in c \to x \in a) \wedge (x \in c \to x \in b))
\end{equation}
と同じである.
故にこれが成り立つ.
またThm \ref{thmallw}より
\[
  \forall x((x \in c \to x \in a) \wedge (x \in c \to x \in b)) 
  \leftrightarrow \forall x(x \in c \to x \in a) \wedge \forall x(x \in c \to x \in b)
\]
が成り立つが, $x$が$a$, $b$, $c$の中に自由変数として現れないことから, 定義よりこの記号列は
\begin{equation}
\label{sthmcsubsetacapb5}
  \forall x((x \in c \to x \in a) \wedge (x \in c \to x \in b)) 
  \leftrightarrow c \subset a \wedge c \subset b
\end{equation}
と同じだから, これが成り立つ.
そこで(\ref{sthmcsubsetacapb4}), (\ref{sthmcsubsetacapb5})から, 
推論法則 \ref{dedeqtrans}によって(\ref{sthmcsubsetacapb1})が成り立つ.
1), 2)が成り立つことは, (\ref{sthmcsubsetacapb1})と
推論法則 \ref{dedwedge}, \ref{dedeqfund}によって明らかである.
\halmos




\mathstrut
\begin{thm}
\label{sthmasubsetacapb}%定理7.47%新規%要る?%確認済
$a$と$b$を集合とするとき, 
\begin{align}
  \label{sthmasubsetacapb1}
  &a \subset a \cap b \leftrightarrow a \subset b, \\
  \mbox{} \notag \\
  \label{sthmasubsetacapb2}
  &b \subset a \cap b \leftrightarrow b \subset a
\end{align}
が共に成り立つ.
またこれらから, 特に次の1), 2)が成り立つ.

1)
$a \subset b$ならば, $a \subset a \cap b$.

2)
$b \subset a$ならば, $b \subset a \cap b$.
\end{thm}


\noindent{\bf 証明}
~定理 \ref{sthmcsubsetacapb}より
\begin{align}
  \label{sthmasubsetacapb3}
  &a \subset a \cap b \leftrightarrow a \subset a \wedge a \subset b, \\
  \mbox{} \notag \\
  \label{sthmasubsetacapb4}
  &b \subset a \cap b \leftrightarrow b \subset a \wedge b \subset b
\end{align}
が共に成り立つ.
また定理 \ref{sthmsubsetself}より$a \subset a$と$b \subset b$が共に成り立つから, 
推論法則 \ref{dedawblatrue2}により
\begin{align}
  \label{sthmasubsetacapb5}
  &a \subset a \wedge a \subset b \leftrightarrow a \subset b, \\
  \mbox{} \notag \\
  \label{sthmasubsetacapb6}
  &b \subset a \wedge b \subset b \leftrightarrow b \subset a
\end{align}
が共に成り立つ.
そこで(\ref{sthmasubsetacapb3})と(\ref{sthmasubsetacapb5}), 
(\ref{sthmasubsetacapb4})と(\ref{sthmasubsetacapb6})から, それぞれ推論法則 \ref{dedeqtrans}によって
(\ref{sthmasubsetacapb1}), (\ref{sthmasubsetacapb2})が成り立つ.

\noindent
1)
(\ref{sthmasubsetacapb1})と推論法則 \ref{dedeqfund}によって明らか.

\noindent
2)
(\ref{sthmasubsetacapb2})と推論法則 \ref{dedeqfund}によって明らか.
\halmos




\mathstrut
\begin{thm}
\label{sthmacapbsubsetceq}%定理7.48%新規%確認済
$a$, $b$, $c$を集合とするとき, 
\begin{align}
  \label{sthmacapbsubsetceq1}
  &a \cap b \subset c \leftrightarrow a \cap b \subset a \cap c, \\
  \mbox{} \notag \\
  \label{sthmacapbsubsetceq2}
  &a \cap b \subset c \leftrightarrow a \cap b \subset c \cap b, \\
  \mbox{} \notag \\
  \label{sthmacapbsubsetceq3}
  &a \cap b \subset a \cap c \leftrightarrow a \cap b \subset c \cap b
\end{align}
がすべて成り立つ.
またこれらから, 次の1), 2), 3)が成り立つ.

1)
$a \cap b \subset c$ならば, 
$a \cap b \subset a \cap c$と$a \cap b \subset c \cap b$が共に成り立つ.

2)
$a \cap b \subset a \cap c$ならば, 
$a \cap b \subset c$と$a \cap b \subset c \cap b$が共に成り立つ.

3)
$a \cap b \subset c \cap b$ならば, 
$a \cap b \subset c$と$a \cap b \subset a \cap c$が共に成り立つ.
\end{thm}


\noindent{\bf 証明}
~定理 \ref{sthmcsubsetacapb}より
\begin{align}
  \label{sthmacapbsubsetceq4}
  &a \cap b \subset a \cap c \leftrightarrow a \cap b \subset a \wedge a \cap b \subset c, \\
  \mbox{} \notag \\
  \label{sthmacapbsubsetceq5}
  &a \cap b \subset c \cap b \leftrightarrow a \cap b \subset c \wedge a \cap b \subset b
\end{align}
が共に成り立つ.
また定理 \ref{sthmsubsetcap}より
\[
  a \cap b \subset a, ~~
  a \cap b \subset b
\]
が共に成り立つから, 推論法則 \ref{dedawblatrue2}により
\begin{align}
  \label{sthmacapbsubsetceq6}
  &a \cap b \subset a \wedge a \cap b \subset c \leftrightarrow a \cap b \subset c, \\
  \mbox{} \notag \\
  \label{sthmacapbsubsetceq7}
  &a \cap b \subset c \wedge a \cap b \subset b \leftrightarrow a \cap b \subset c
\end{align}
が共に成り立つ.
そこで(\ref{sthmacapbsubsetceq4})と(\ref{sthmacapbsubsetceq6}), 
(\ref{sthmacapbsubsetceq5})と(\ref{sthmacapbsubsetceq7})から, それぞれ推論法則 \ref{dedeqtrans}によって
\begin{align}
  \label{sthmacapbsubsetceq8}
  &a \cap b \subset a \cap c \leftrightarrow a \cap b \subset c, \\
  \mbox{} \notag \\
  \label{sthmacapbsubsetceq9}
  &a \cap b \subset c \cap b \leftrightarrow a \cap b \subset c
\end{align}
が成り立つ.
故に(\ref{sthmacapbsubsetceq8}), (\ref{sthmacapbsubsetceq9})から, それぞれ推論法則 \ref{dedeqch}により
(\ref{sthmacapbsubsetceq1}), (\ref{sthmacapbsubsetceq2})が成り立つ.
また(\ref{sthmacapbsubsetceq2}), (\ref{sthmacapbsubsetceq8})から, 
推論法則 \ref{dedeqtrans}によって(\ref{sthmacapbsubsetceq3})が成り立つ.

\noindent
1)
(\ref{sthmacapbsubsetceq1}), (\ref{sthmacapbsubsetceq2})と推論法則 \ref{dedeqfund}によって明らか.

\noindent
2)
(\ref{sthmacapbsubsetceq1}), (\ref{sthmacapbsubsetceq3})と推論法則 \ref{dedeqfund}によって明らか.

\noindent
3)
(\ref{sthmacapbsubsetceq2}), (\ref{sthmacapbsubsetceq3})と推論法則 \ref{dedeqfund}によって明らか.
\halmos




\mathstrut
\begin{thm}
\label{sthmcapsubset}%定理7.49%確認済
\mbox{}

1)
$a$, $b$, $c$を集合とするとき, 
\begin{align}
  \label{sthmcapsubset1}
  &a \subset b \to a \cap c \subset b \cap c, \\
  \mbox{} \notag \\
  \label{sthmcapsubset2}
  &a \subset b \to c \cap a \subset c \cap b
\end{align}
が共に成り立つ.
またこれらから, 次の(\ref{sthmcapsubset3})が成り立つ.
\begin{equation}
\label{sthmcapsubset3}
  a \subset b \text{ならば,} ~
  a \cap c \subset b \cap c \text{と} c \cap a \subset c \cap b \text{が共に成り立つ.}
\end{equation}

2)
$a$, $b$, $c$, $d$を集合とするとき, 
\begin{equation}
\label{sthmcapsubset4}
  a \subset b \wedge c \subset d \to a \cap c \subset b \cap d
\end{equation}
が成り立つ.
またこのことから, 次の(\ref{sthmcapsubset5})が成り立つ.
\begin{equation}
\label{sthmcapsubset5}
  a \subset b \text{と} c \subset d \text{が共に成り立てば,} ~a \cap c \subset b \cap d.
\end{equation}
\end{thm}


\noindent{\bf 証明}
~1)
定理 \ref{sthmacapbsubsetceq}より
\[
  a \cap c \subset b \leftrightarrow a \cap c \subset b \cap c, ~~
  c \cap a \subset b \leftrightarrow c \cap a \subset c \cap b
\]
が共に成り立つから, 推論法則 \ref{dedaddeqt}により
\begin{align}
  \label{sthmcapsubset6}
  &(a \subset b \to a \cap c \subset b) \leftrightarrow (a \subset b \to a \cap c \subset b \cap c), \\
  \mbox{} \notag \\
  \label{sthmcapsubset7}
  &(a \subset b \to c \cap a \subset b) \leftrightarrow (a \subset b \to c \cap a \subset c \cap b)
\end{align}
が共に成り立つ.
ここで定理 \ref{sthmacapbsubsetc}より
\[
  a \subset b \to a \cap c \subset b, ~~
  a \subset b \to c \cap a \subset b
\]
が共に成り立つから, この前者と(\ref{sthmcapsubset6}), 後者と(\ref{sthmcapsubset7})から, 
それぞれ推論法則 \ref{dedeqfund}により(\ref{sthmcapsubset1}), (\ref{sthmcapsubset2})が成り立つ.
(\ref{sthmcapsubset3})が成り立つことは, 
(\ref{sthmcapsubset1}), (\ref{sthmcapsubset2})と推論法則 \ref{dedmp}によって明らかである.

\noindent
2)
1)より
\[
  a \subset b \to a \cap c \subset b \cap c, ~~
  c \subset d \to b \cap c \subset b \cap d
\]
が共に成り立つから, 推論法則 \ref{dedfromaddw}により
\begin{equation}
\label{sthmcapsubset8}
  a \subset b \wedge c \subset d \to a \cap c \subset b \cap c \wedge b \cap c \subset b \cap d
\end{equation}
が成り立つ.
また定理 \ref{sthmsubsettrans}より
\begin{equation}
\label{sthmcapsubset9}
  a \cap c \subset b \cap c \wedge b \cap c \subset b \cap d \to a \cap c \subset b \cap d
\end{equation}
が成り立つ.
そこで(\ref{sthmcapsubset8}), (\ref{sthmcapsubset9})から, 
推論法則 \ref{dedmmp}によって(\ref{sthmcapsubset4})が成り立つ.
(\ref{sthmcapsubset5})が成り立つことは, 
(\ref{sthmcapsubset4})と推論法則 \ref{dedmp}, \ref{dedwedge}によって明らかである.
\halmos




\mathstrut
\begin{thm}
\label{sthmcapsubseteq}%定理7.50%新規%逆は言えない%要る?%確認済
$a$, $b$, $c$を集合とするとき, 
\begin{align}
  \label{sthmcapsubseteq1}
  &a \subset c \to (a \subset b \leftrightarrow a \cap c \subset b \cap c), \\
  \mbox{} \notag \\
  \label{sthmcapsubseteq2}
  &a \subset c \to (a \subset b \leftrightarrow c \cap a \subset c \cap b)
\end{align}
が共に成り立つ.
またこれらから, 次の1), 2), 3)が成り立つ.

1)
$a \subset c$ならば, $a \subset b \leftrightarrow a \cap c \subset b \cap c$と
$a \subset b \leftrightarrow c \cap a \subset c \cap b$が共に成り立つ.

2)
$a \subset c$と$a \cap c \subset b \cap c$が共に成り立てば, $a \subset b$.

3)
$a \subset c$と$c \cap a \subset c \cap b$が共に成り立てば, $a \subset b$.
\end{thm}


\noindent{\bf 証明}
~定理 \ref{sthmasubsetacapb}と推論法則 \ref{dedequiv}により
\begin{align}
  \label{sthmcapsubseteq3}
  &a \subset c \to a \subset a \cap c, \\
  \mbox{} \notag \\
  \label{sthmcapsubseteq4}
  &a \subset c \to a \subset c \cap a
\end{align}
が共に成り立つ.
また定理 \ref{sthmacapbsubsetceq}と推論法則 \ref{dedequiv}により
\begin{align}
  \label{sthmcapsubseteq5}
  &a \cap c \subset b \cap c \to a \cap c \subset b, \\
  \mbox{} \notag \\
  \label{sthmcapsubseteq6}
  &c \cap a \subset c \cap b \to c \cap a \subset b
\end{align}
が共に成り立つ.
そこで(\ref{sthmcapsubseteq3})と(\ref{sthmcapsubseteq5}), 
(\ref{sthmcapsubseteq4})と(\ref{sthmcapsubseteq6})から, それぞれ推論法則 \ref{dedfromaddw}により
\begin{align}
  \label{sthmcapsubseteq7}
  &a \subset c \wedge a \cap c \subset b \cap c \to a \subset a \cap c \wedge a \cap c \subset b, \\
  \mbox{} \notag \\
  \label{sthmcapsubseteq8}
  &a \subset c \wedge c \cap a \subset c \cap b \to a \subset c \cap a \wedge c \cap a \subset b
\end{align}
が成り立つ.
また定理 \ref{sthmsubsettrans}より
\begin{align}
  \label{sthmcapsubseteq9}
  &a \subset a \cap c \wedge a \cap c \subset b \to a \subset b, \\
  \mbox{} \notag \\
  \label{sthmcapsubseteq10}
  &a \subset c \cap a \wedge c \cap a \subset b \to a \subset b
\end{align}
が共に成り立つ.
そこで(\ref{sthmcapsubseteq7})と(\ref{sthmcapsubseteq9}), 
(\ref{sthmcapsubseteq8})と(\ref{sthmcapsubseteq10})から, それぞれ推論法則 \ref{dedmmp}によって
\[
  a \subset c \wedge a \cap c \subset b \cap c \to a \subset b, ~~
  a \subset c \wedge c \cap a \subset c \cap b \to a \subset b
\]
が成り立つ.
故にこれらから, 推論法則 \ref{dedtwch}により
\begin{align}
  \label{sthmcapsubseteq11}
  &a \subset c \to (a \cap c \subset b \cap c \to a \subset b), \\
  \mbox{} \notag \\
  \label{sthmcapsubseteq12}
  &a \subset c \to (c \cap a \subset c \cap b \to a \subset b)
\end{align}
が共に成り立つ.
また定理 \ref{sthmcapsubset}より
\[
  a \subset b \to a \cap c \subset b \cap c, ~~
  a \subset b \to c \cap a \subset c \cap b
\]
が共に成り立つから, 推論法則 \ref{deds1}により
\begin{align}
  \label{sthmcapsubseteq13}
  &a \subset c \to (a \subset b \to a \cap c \subset b \cap c), \\
  \mbox{} \notag \\
  \label{sthmcapsubseteq14}
  &a \subset c \to (a \subset b \to c \cap a \subset c \cap b)
\end{align}
が共に成り立つ.
そこで(\ref{sthmcapsubseteq11})と(\ref{sthmcapsubseteq13}), 
(\ref{sthmcapsubseteq12})と(\ref{sthmcapsubseteq14})から, 
それぞれ推論法則 \ref{dedpreequiv}により(\ref{sthmcapsubseteq1}), (\ref{sthmcapsubseteq2})が成り立つ.

\noindent
1)
(\ref{sthmcapsubseteq1}), (\ref{sthmcapsubseteq2})と推論法則 \ref{dedmp}によって明らか.

\noindent
2), 3)
1)と推論法則 \ref{dedeqfund}によって明らか.
\halmos




\mathstrut
\begin{thm}
\label{sthmacapb=a}%定理7.51%新規%確認済
$a$と$b$を集合とするとき, 
\begin{align}
  \label{sthmacapb=a1}
  &a \cap b = a \leftrightarrow a \subset b, \\
  \mbox{} \notag \\
  \label{sthmacapb=a2}
  &a \cap b = b \leftrightarrow b \subset a
\end{align}
が共に成り立つ.
またこれらから, 次の1)---4)が成り立つ.

1)
$a \cap b = a$ならば, $a \subset b$.

2)
$a \subset b$ならば, $a \cap b = a$.

3)
$a \cap b = b$ならば, $b \subset a$.

4)
$b \subset a$ならば, $a \cap b = b$.
\end{thm}


\noindent{\bf 証明}
~定理 \ref{sthmaxiom1}と推論法則 \ref{dedeqch}により
\begin{align}
  \label{sthmacapb=a3}
  &a \cap b = a \leftrightarrow a \cap b \subset a \wedge a \subset a \cap b, \\
  \mbox{} \notag \\
  \label{sthmacapb=a4}
  &a \cap b = b \leftrightarrow a \cap b \subset b \wedge b \subset a \cap b
\end{align}
が共に成り立つ.
また定理 \ref{sthmsubsetcap}より
\[
  a \cap b \subset a, ~~
  a \cap b \subset b
\]
が共に成り立つから, 推論法則 \ref{dedawblatrue2}により
\begin{align}
  \label{sthmacapb=a5}
  &a \cap b \subset a \wedge a \subset a \cap b \leftrightarrow a \subset a \cap b, \\
  \mbox{} \notag \\
  \label{sthmacapb=a6}
  &a \cap b \subset b \wedge b \subset a \cap b \leftrightarrow b \subset a \cap b
\end{align}
が共に成り立つ.
また定理 \ref{sthmasubsetacapb}より
\begin{align}
  \label{sthmacapb=a7}
  &a \subset a \cap b \leftrightarrow a \subset b, \\
  \mbox{} \notag \\
  \label{sthmacapb=a8}
  &b \subset a \cap b \leftrightarrow b \subset a
\end{align}
が共に成り立つ.
そこで(\ref{sthmacapb=a3}), (\ref{sthmacapb=a5}), (\ref{sthmacapb=a7})から, 
推論法則 \ref{dedeqtrans}によって(\ref{sthmacapb=a1})が成り立つことがわかる.
また(\ref{sthmacapb=a4}), (\ref{sthmacapb=a6}), (\ref{sthmacapb=a8})から, 
同じく推論法則 \ref{dedeqtrans}によって(\ref{sthmacapb=a2})が成り立つことがわかる.

\noindent
1), 2)
(\ref{sthmacapb=a1})と推論法則 \ref{dedeqfund}によって明らか.

\noindent
3), 4)
(\ref{sthmacapb=a2})と推論法則 \ref{dedeqfund}によって明らか.
\halmos




\mathstrut
\begin{thm}
\label{sthmcappsubset}%定理7.52%新規%確認済
$a$と$b$を集合とするとき, 
\begin{align}
  \label{sthmcappsubset1}
  &a \cap b \subsetneqq a \leftrightarrow a \not\subset b, \\
  \mbox{} \notag \\
  \label{sthmcappsubset2}
  &a \cap b \subsetneqq b \leftrightarrow b \not\subset a
\end{align}
が共に成り立つ.
またこれらから, 次の1)---4)が成り立つ.

1)
$a \cap b \subsetneqq a$ならば, $a \not\subset b$.

2)
$a \not\subset b$ならば, $a \cap b \subsetneqq a$.

3)
$a \cap b \subsetneqq b$ならば, $b \not\subset a$.

4)
$b \not\subset a$ならば, $a \cap b \subsetneqq b$.
\end{thm}


\noindent{\bf 証明}
~定理 \ref{sthmsubsetcap}より
\[
  a \cap b \subset a, ~~
  a \cap b \subset b
\]
が共に成り立つから, 推論法則 \ref{dedawblatrue2}により
\begin{align}
  \label{sthmcappsubset3}
  &a \cap b \subsetneqq a \leftrightarrow a \cap b \neq a, \\
  \mbox{} \notag \\
  \label{sthmcappsubset4}
  &a \cap b \subsetneqq b \leftrightarrow a \cap b \neq b
\end{align}
が共に成り立つ.
また定理 \ref{sthmacapb=a}より
\[
  a \cap b = a \leftrightarrow a \subset b, ~~
  a \cap b = b \leftrightarrow b \subset a
\]
が共に成り立つから, 推論法則 \ref{dedeqcp}により
\begin{align}
  \label{sthmcappsubset5}
  &a \cap b \neq a \leftrightarrow a \not\subset b, \\
  \mbox{} \notag \\
  \label{sthmcappsubset6}
  &a \cap b \neq b \leftrightarrow b \not\subset a
\end{align}
が共に成り立つ.
そこで(\ref{sthmcappsubset3})と(\ref{sthmcappsubset5}), 
(\ref{sthmcappsubset4})と(\ref{sthmcappsubset6})から, 
それぞれ推論法則 \ref{dedeqtrans}によって(\ref{sthmcappsubset1}), (\ref{sthmcappsubset2})が成り立つ.

\noindent
1), 2)
(\ref{sthmcappsubset1})と推論法則 \ref{dedeqfund}によって明らか.

\noindent
3), 4)
(\ref{sthmcappsubset2})と推論法則 \ref{dedeqfund}によって明らか.
\halmos




\mathstrut
\begin{thm}
\label{sthmcap=}%定理7.53%確認済
\mbox{}

1)
$a$, $b$, $c$を集合とするとき, 
\begin{align}
  \label{sthmcap=1}
  &a = b \to a \cap c = b \cap c, \\
  \mbox{} \notag \\
  \label{sthmcap=2}
  &a = b \to c \cap a = c \cap b
\end{align}
が共に成り立つ.
またこれらから, 次の(\ref{sthmcap=3})が成り立つ.
\begin{equation}
\label{sthmcap=3}
  a = b \text{ならば,} ~a \cap c = b \cap c \text{と} c \cap a = c \cap b \text{が共に成り立つ.}
\end{equation}

2)
$a$, $b$, $c$, $d$を集合とするとき, 
\begin{equation}
\label{sthmcap=4}
  a = b \wedge c = d \to a \cap c = b \cap d
\end{equation}
が成り立つ.
またこのことから, 次の(\ref{sthmcap=5})が成り立つ.
\begin{equation}
\label{sthmcap=5}
  a = b \text{と} c = d \text{が共に成り立てば,} ~a \cap c = b \cap d.
\end{equation}
\end{thm}


\noindent{\bf 証明}
~1)
$x$を$c$の中に自由変数として現れない文字とするとき, Thm \ref{T=Ut1TV=UV1}より
\[
  a = b \to (a|x)(x \cap c) = (b|x)(x \cap c), ~~
  a = b \to (a|x)(c \cap x) = (b|x)(c \cap x)
\]
が共に成り立つが, 代入法則 \ref{substfree}, \ref{substcap}によれば
これらはそれぞれ(\ref{sthmcap=1}), (\ref{sthmcap=2})と一致するから, これらが共に成り立つ.
(\ref{sthmcap=3})が成り立つことは, 
(\ref{sthmcap=1}), (\ref{sthmcap=2})と推論法則 \ref{dedmp}によって明らかである.

\noindent
2)
1)より
\[
  a = b \to a \cap c = b \cap c, ~~
  c = d \to b \cap c = b \cap d
\]
が共に成り立つから, 推論法則 \ref{dedfromaddw}により
\begin{equation}
\label{sthmcap=6}
  a = b \wedge c = d \to a \cap c = b \cap c \wedge b \cap c = b \cap d
\end{equation}
が成り立つ.
またThm \ref{x=ywy=ztx=z}より
\begin{equation}
\label{sthmcap=7}
  a \cap c = b \cap c \wedge b \cap c = b \cap d \to a \cap c = b \cap d
\end{equation}
が成り立つ.
そこで(\ref{sthmcap=6}), (\ref{sthmcap=7})から, 
推論法則 \ref{dedmmp}によって(\ref{sthmcap=4})が成り立つ.
(\ref{sthmcap=5})が成り立つことは, 
(\ref{sthmcap=4})と推論法則 \ref{dedmp}, \ref{dedwedge}によって明らかである.
\halmos




\mathstrut
\begin{thm}
\label{sthmcap=eq}%定理7.54%新規%要る?%確認済
$a$, $b$, $c$を集合とするとき, 
\begin{align}
  \label{sthmcap=eq1}
  &a \subset c \wedge b \subset c \to (a = b \leftrightarrow a \cap c = b \cap c), \\
  \mbox{} \notag \\
  \label{sthmcap=eq2}
  &a \subset c \wedge b \subset c \to (a = b \leftrightarrow c \cap a = c \cap b)
\end{align}
が共に成り立つ.
またこれらから, 次の1), 2), 3)が成り立つ.

1)
$a \subset c$と$b \subset c$が共に成り立てば, 
$a = b \leftrightarrow a \cap c = b \cap c$と$a = b \leftrightarrow c \cap a = c \cap b$が共に成り立つ.

2)
$a \subset c$, $b \subset c$, $a \cap c = b \cap c$がすべて成り立てば, $a = b$.

3)
$a \subset c$, $b \subset c$, $c \cap a = c \cap b$がすべて成り立てば, $a = b$.
\end{thm}


\noindent{\bf 証明}
~定理 \ref{sthmacapb=a}と推論法則 \ref{dedequiv}により
\begin{align*}
  &a \subset c \to a \cap c = a, ~~
  b \subset c \to b \cap c = b, \\
  \mbox{} \notag \\
  &a \subset c \to c \cap a = a, ~~
  b \subset c \to c \cap b = b
\end{align*}
がすべて成り立つから, このはじめの二つ, あとの二つから, それぞれ推論法則 \ref{dedfromaddw}により
\begin{align}
  \label{sthmcap=eq3}
  &a \subset c \wedge b \subset c \to a \cap c = a \wedge b \cap c = b, \\
  \mbox{} \notag \\
  \label{sthmcap=eq4}
  &a \subset c \wedge b \subset c \to c \cap a = a \wedge c \cap b = b
\end{align}
が成り立つ.
またThm \ref{x=ywz=ut1x=zly=u1}より
\begin{align}
  \label{sthmcap=eq5}
  &a \cap c = a \wedge b \cap c = b \to (a \cap c = b \cap c \leftrightarrow a = b), \\
  \mbox{} \notag \\
  \label{sthmcap=eq6}
  &c \cap a = a \wedge c \cap b = b \to (c \cap a = c \cap b \leftrightarrow a = b)
\end{align}
が共に成り立つ.
またThm \ref{1alb1t1bla1}より
\begin{align}
  \label{sthmcap=eq7}
  &(a \cap c = b \cap c \leftrightarrow a = b) \to (a = b \leftrightarrow a \cap c = b \cap c), \\
  \mbox{} \notag \\
  \label{sthmcap=eq8}
  &(c \cap a = c \cap b \leftrightarrow a = b) \to (a = b \leftrightarrow c \cap a = c \cap b)
\end{align}
が共に成り立つ.
そこで(\ref{sthmcap=eq3}), (\ref{sthmcap=eq5}), (\ref{sthmcap=eq7})から, 
推論法則 \ref{dedmmp}によって(\ref{sthmcap=eq1})が成り立つことがわかる.
また(\ref{sthmcap=eq4}), (\ref{sthmcap=eq6}), (\ref{sthmcap=eq8})から, 
同じく推論法則 \ref{dedmmp}によって(\ref{sthmcap=eq2})が成り立つことがわかる.

\noindent
1)
(\ref{sthmcap=eq1}), (\ref{sthmcap=eq2})と推論法則 \ref{dedmp}, \ref{dedwedge}によって明らか.

\noindent
2), 3)
1)と推論法則 \ref{dedeqfund}によって明らか.
\halmos




\mathstrut
\begin{thm}
\label{sthmspincap}%定理7.55%新規%確認済
$a$と$b$を集合, $R$を関係式とし, $x$を$a$及び$b$の中に自由変数として現れない文字とする.
このとき
\begin{align}
  \label{sthmspincap1}
  &(\exists x \in a \cap b)(R) \to (\exists x \in a)(R) \wedge (\exists x \in b)(R), \\
  \mbox{} \notag \\
  \label{sthmspincap2}
  &(\forall x \in a)(R) \vee (\forall x \in b)(R) \to (\forall x \in a \cap b)(R), \\
  \mbox{} \notag \\
  \label{sthmspincap3}
  &(!x \in a)(R) \vee (!x \in b)(R) \to (!x \in a \cap b)(R)
\end{align}
がすべて成り立つ.
またこれらから, 次の1), 2), 3)が成り立つ.

1)
$(\exists x \in a \cap b)(R)$ならば, $(\exists x \in a)(R)$と$(\exists x \in b)(R)$が共に成り立つ.

2)
$(\forall x \in a)(R)$ならば, $(\forall x \in a \cap b)(R)$.
また$(\forall x \in b)(R)$ならば, $(\forall x \in a \cap b)(R)$.

3)
$(!x \in a)(R)$ならば, $(!x \in a \cap b)(R)$.
また$(!x \in b)(R)$ならば, $(!x \in a \cap b)(R)$.

\end{thm}


\noindent{\bf 証明}
~$x$が$a$, $b$の中に自由変数として現れないことから, 
変数法則 \ref{valcap}により, $x$は$a \cap b$の中にも自由変数として現れない.
このことと, 定理 \ref{sthmsubsetcap}より
\[
  a \cap b \subset a, ~~
  a \cap b \subset b
\]
が共に成り立つことから, 定理 \ref{sthmspinsubset}より
\begin{align}
  \label{sthmspincap4}
  &(\exists x \in a \cap b)(R) \to (\exists x \in a)(R), \\
  \mbox{} \notag \\
  \label{sthmspincap5}
  &(\exists x \in a \cap b)(R) \to (\exists x \in b)(R), \\
  \mbox{} \notag \\
  \label{sthmspincap6}
  &(\forall x \in a)(R) \to (\forall x \in a \cap b)(R), \\
  \mbox{} \notag \\
  \label{sthmspincap7}
  &(\forall x \in b)(R) \to (\forall x \in a \cap b)(R), \\
  \mbox{} \notag \\
  \label{sthmspincap8}
  &(!x \in a)(R) \to (!x \in a \cap b)(R), \\
  \mbox{} \notag \\
  \label{sthmspincap9}
  &(!x \in b)(R) \to (!x \in a \cap b)(R)
\end{align}
がすべて成り立つ.
そこで(\ref{sthmspincap4}), (\ref{sthmspincap5})から, 
推論法則 \ref{dedprewedge}により(\ref{sthmspincap1})が成り立つ.
また(\ref{sthmspincap6})と(\ref{sthmspincap7}), (\ref{sthmspincap8})と(\ref{sthmspincap9})から, 
それぞれ推論法則 \ref{deddil}により(\ref{sthmspincap2}), (\ref{sthmspincap3})が成り立つ.

\noindent
1)
(\ref{sthmspincap1})と推論法則 \ref{dedmp}, \ref{dedwedge}によって明らか.

\noindent
2)
(\ref{sthmspincap2})と推論法則 \ref{dedmp}, \ref{dedvee}によって明らか.

\noindent
3)
(\ref{sthmspincap3})と推論法則 \ref{dedmp}, \ref{dedvee}によって明らか.
\halmos




\mathstrut
\begin{thm}
\label{sthmcapidempotent}%定理7.56%確認済
$a$を集合とするとき, 
\[
  a \cap a = a
\]
が成り立つ.
\end{thm}


\noindent{\bf 証明}
~定理 \ref{sthmsubsetself}より$a \subset a$が成り立つから, 
定理 \ref{sthmacapb=a}より$a \cap a = a$が成り立つ.
\halmos




\mathstrut
\begin{thm}
\label{sthmcapch}%定理7.57%確認済
$a$と$b$を集合とするとき, 
\[
  a \cap b = b \cap a
\]
が成り立つ.
\end{thm}


\noindent{\bf 証明}
~$x$を$a$, $b$の中に自由変数として現れない, 定数でない文字とする.
このときThm \ref{awblbwa}より
\[
  x \in a \wedge x \in b \leftrightarrow x \in b \wedge x \in a
\]
が成り立つから, このことと$x$が定数でないことから, 定理 \ref{sthmalleqiset=}より
\[
  \{x \mid x \in a \wedge x \in b\} = \{x \mid x \in b \wedge x \in a\}
\]
が成り立つ.
ここで$x$が$a$, $b$の中に自由変数として現れないことから, 
定義よりこの記号列は$a \cap b = b \cap a$と同じである.
故にこれが成り立つ.
\halmos




\mathstrut
\begin{thm}
\label{sthmcapcomb}%定理7.58%確認済
$a$, $b$, $c$を集合とするとき, 
\[
  (a \cap b) \cap c = a \cap (b \cap c)
\]
が成り立つ.
\end{thm}


\noindent{\bf 証明}
~$x$を$a$, $b$, $c$の中に自由変数として現れない, 定数でない文字とする.
このとき変数法則 \ref{valcap}により, $x$は$a \cap b$, $b \cap c$の中に自由変数として現れない.
また定理 \ref{sthmcapbasis}より
\[
  x \in a \cap b \leftrightarrow x \in a \wedge x \in b
\]
が成り立つから, 推論法則 \ref{dedaddeqw}により
\begin{equation}
\label{sthmcapcomb1}
  x \in a \cap b \wedge x \in c \leftrightarrow (x \in a \wedge x \in b) \wedge x \in c
\end{equation}
が成り立つ.
またThm \ref{1awb1wclaw1bwc1}より
\begin{equation}
\label{sthmcapcomb2}
  (x \in a \wedge x \in b) \wedge x \in c \leftrightarrow x \in a \wedge (x \in b \wedge x \in c)
\end{equation}
が成り立つ.
また定理 \ref{sthmcapbasis}と推論法則 \ref{dedeqch}により
\[
  x \in b \wedge x \in c \leftrightarrow x \in b \cap c
\]
が成り立つから, 推論法則 \ref{dedaddeqw}により
\begin{equation}
\label{sthmcapcomb3}
  x \in a \wedge (x \in b \wedge x \in c) \leftrightarrow x \in a \wedge x \in b \cap c
\end{equation}
が成り立つ.
そこで(\ref{sthmcapcomb1})---(\ref{sthmcapcomb3})から, 推論法則 \ref{dedeqtrans}によって
\[
  x \in a \cap b \wedge x \in c \leftrightarrow x \in a \wedge x \in b \cap c
\]
が成り立つことがわかる.
このことと$x$が定数でないことから, 定理 \ref{sthmalleqiset=}より
\[
  \{x \mid x \in a \cap b \wedge x \in c\} = \{x \mid x \in a \wedge x \in b \cap c\}
\]
が成り立つ.
ここで上述のように, $x$は$a \cap b$, $c$, $a$, $b \cap c$のいずれの中にも自由変数として現れないから, 
定義よりこの記号列は$(a \cap b) \cap c = a \cap (b \cap c)$と同じである.
故にこれが成り立つ.
\halmos




\mathstrut
\begin{thm}
\label{sthmcapdist}%定理7.59%確認済
$a$, $b$, $c$を集合とするとき, 
\[
  a \cap (b \cap c) = (a \cap b) \cap (a \cap c), ~~
  (a \cap b) \cap c = (a \cap c) \cap (b \cap c)
\]
が共に成り立つ.
\end{thm}


\noindent{\bf 証明}
~$x$を$a$, $b$, $c$の中に自由変数として現れない, 定数でない文字とする.
このとき変数法則 \ref{valcap}により, 
$x$は$b \cap c$, $a \cap b$, $a \cap c$の中に自由変数として現れない.
また定理 \ref{sthmcapbasis}より
\[
  x \in b \cap c \leftrightarrow x \in b \wedge x \in c, ~~
  x \in a \cap b \leftrightarrow x \in a \wedge x \in b
\]
が共に成り立つから, 推論法則 \ref{dedaddeqw}により
\begin{align}
  \label{sthmcapdist1}
  &x \in a \wedge x \in b \cap c \leftrightarrow x \in a \wedge (x \in b \wedge x \in c), \\
  \mbox{} \notag \\
  \label{sthmcapdist2}
  &x \in a \cap b \wedge x \in c \leftrightarrow (x \in a \wedge x \in b) \wedge x \in c
\end{align}
が共に成り立つ.
またThm \ref{aw1bwc1l1awb1w1awc1}より
\begin{align}
  \label{sthmcapdist3}
  &x \in a \wedge (x \in b \wedge x \in c) 
  \leftrightarrow (x \in a \wedge x \in b) \wedge (x \in a \wedge x \in c), \\
  \mbox{} \notag \\
  \label{sthmcapdist4}
  &(x \in a \wedge x \in b) \wedge x \in c 
  \leftrightarrow (x \in a \wedge x \in c) \wedge (x \in b \wedge x \in c)
\end{align}
が共に成り立つ.
また定理 \ref{sthmcapbasis}と推論法則 \ref{dedeqch}により
\begin{align}
  \label{sthmcapdist5}
  &x \in a \wedge x \in b \leftrightarrow x \in a \cap b, \\
  \mbox{} \notag \\
  \label{sthmcapdist6}
  &x \in a \wedge x \in c \leftrightarrow x \in a \cap c, \\
  \mbox{} \notag \\
  \label{sthmcapdist7}
  &x \in b \wedge x \in c \leftrightarrow x \in b \cap c
\end{align}
がすべて成り立つ.
故に(\ref{sthmcapdist5})と(\ref{sthmcapdist6}), (\ref{sthmcapdist6})と(\ref{sthmcapdist7})から, 
それぞれ推論法則 \ref{dedaddeqw}により
\begin{align}
  \label{sthmcapdist8}
  &(x \in a \wedge x \in b) \wedge (x \in a \wedge x \in c) 
  \leftrightarrow x \in a \cap b \wedge x \in a \cap c, \\
  \mbox{} \notag \\
  \label{sthmcapdist9}
  &(x \in a \wedge x \in c) \wedge (x \in b \wedge x \in c) 
  \leftrightarrow x \in a \cap c \wedge x \in b \cap c
\end{align}
が成り立つ.
そこで(\ref{sthmcapdist1}), (\ref{sthmcapdist3}), (\ref{sthmcapdist8})から, 
推論法則 \ref{dedeqtrans}によって
\[
  x \in a \wedge x \in b \cap c \leftrightarrow x \in a \cap b \wedge x \in a \cap c
\]
が成り立つことがわかる.
また(\ref{sthmcapdist2}), (\ref{sthmcapdist4}), (\ref{sthmcapdist9})から, 
同じく推論法則 \ref{dedeqtrans}によって
\[
  x \in a \cap b \wedge x \in c \leftrightarrow x \in a \cap c \wedge x \in b \cap c
\]
が成り立つことがわかる.
これらのことと, $x$が定数でないことから, 定理 \ref{sthmalleqiset=}より
\begin{align*}
  &\{x \mid x \in a \wedge x \in b \cap c\} = \{x \mid x \in a \cap b \wedge x \in a \cap c\}, \\
  \mbox{} \notag \\
  &\{x \mid x \in a \cap b \wedge x \in c\} = \{x \mid x \in a \cap c \wedge x \in b \cap c\}
\end{align*}
が共に成り立つ.
ここで上述のように, $x$は$a$, $b \cap c$, $a \cap b$, $a \cap c$, $c$のいずれの中にも
自由変数として現れないから, 定義よりこれらの記号列はそれぞれ
$a \cap (b \cap c) = (a \cap b) \cap (a \cap c)$, 
$(a \cap b) \cap c = (a \cap c) \cap (b \cap c)$と同じである.
故にこれらが共に成り立つ.
\halmos




\mathstrut
\begin{thm}
\label{sthmacapiset}%定理7.60%確認済
$a$を集合, $R$を関係式とし, $x$を$a$の中に自由変数として現れない文字とする.
このとき
\begin{align}
  \label{sthmacapiset1}
  &{\rm Set}_{x}(R) \to a \cap \{x \mid R\} = \{x \in a \mid R\}, \\
  \mbox{} \notag \\
  \label{sthmacapiset2}
  &{\rm Set}_{x}(R) \to \{x \mid R\} \cap a = \{x \in a \mid R\}
\end{align}
が共に成り立つ.
またこれらから, 次の(\ref{sthmacapiset3})が成り立つ.
\begin{align}
\label{sthmacapiset3}
  &R \text{が} x \text{について集合を作り得るならば,} ~
  a \cap \{x \mid R\} = \{x \in a \mid R\} \text{と} \\
  &\{x \mid R\} \cap a = \{x \in a \mid R\} \text{が共に成り立つ.} \notag
\end{align}
\end{thm}


\noindent{\bf 証明}
~定理 \ref{sthmalleqsset=}より, 
\[
  \forall x(x \in \{x \mid R\} \leftrightarrow R) 
  \to \{x \in a \mid x \in \{x \mid R\}\} = \{x \in a \mid R\}, 
\]
即ち
\[
  {\rm Set}_{x}(R) \to \{x \in a \mid x \in \{x \mid R\}\} = \{x \in a \mid R\}
\]
が成り立つ.
ここで変数法則 \ref{valiset}により, $x$は$\{x \mid R\}$の中に自由変数として現れないから, 
このことと$x$が$a$の中に自由変数として現れないことから, 
定義より上記の記号列は(\ref{sthmacapiset1})と同じである.
故に(\ref{sthmacapiset1})が成り立つ.
また定理 \ref{sthmcapch}より
\[
  a \cap \{x \mid R\} = \{x \mid R\} \cap a
\]
が成り立つから, 推論法則 \ref{dedaddeq=}により
\[
  a \cap \{x \mid R\} = \{x \in a \mid R\} \leftrightarrow \{x \mid R\} \cap a = \{x \in a \mid R\}
\]
が成り立つ.
故に推論法則 \ref{dedequiv}により
\[
  a \cap \{x \mid R\} = \{x \in a \mid R\} \to \{x \mid R\} \cap a = \{x \in a \mid R\}
\]
が成り立つ.
そこでこれと(\ref{sthmacapiset1})から, 推論法則 \ref{dedmmp}によって(\ref{sthmacapiset2})が成り立つ.
(\ref{sthmacapiset3})が成り立つことは, 
(\ref{sthmacapiset1}), (\ref{sthmacapiset2})と推論法則 \ref{dedmp}によって明らかである.
\halmos




\mathstrut
\begin{thm}
\label{sthmwsm}%定理7.61%確認済
$R$と$S$を関係式とし, $x$を文字とする.
このとき
\begin{equation}
\label{sthmwsm1}
  {\rm Set}_{x}(R) \vee {\rm Set}_{x}(S) \to {\rm Set}_{x}(R \wedge S)
\end{equation}
が成り立つ.
またこのことから, 次の1), 2)が成り立つ.

1)
$R$が$x$について集合を作り得るならば, $R \wedge S$は$x$について集合を作り得る.

2)
$S$が$x$について集合を作り得るならば, $R \wedge S$は$x$について集合を作り得る.
\end{thm}


\noindent{\bf 証明}
~$y$を$R$, $S$の中に自由変数として現れない, 定数でない文字とする.
このときThm \ref{awbta}より
\[
  (y|x)(R) \wedge (y|x)(S) \to (y|x)(R), ~~
  (y|x)(R) \wedge (y|x)(S) \to (y|x)(S)
\]
が共に成り立つが, 代入法則 \ref{substfund}, \ref{substwedge}によればこれらの記号列はそれぞれ
\[
  (y|x)(R \wedge S \to R), ~~
  (y|x)(R \wedge S \to S)
\]
と一致するから, これらが共に成り立つ.
このことと$y$が定数でないことから, 推論法則 \ref{dedltthmquan}により
\[
  \forall y((y|x)(R \wedge S \to R)), ~~
  \forall y((y|x)(R \wedge S \to S))
\]
が共に成り立つ.
ここで$y$が$R$, $S$の中に自由変数として現れないことから, 変数法則 \ref{valfund}, \ref{valwedge}により, 
$y$は$R \wedge S \to R$, $R \wedge S \to S$の中に自由変数として現れないから, 
代入法則 \ref{substquantrans}によれば上記の記号列はそれぞれ
\[
  \forall x(R \wedge S \to R), ~~
  \forall x(R \wedge S \to S)
\]
と一致する.
故にこれらが共に成り立つ.
そこで定理 \ref{sthmalltsm}より
\[
  {\rm Set}_{x}(R) \to {\rm Set}_{x}(R \wedge S), ~~
  {\rm Set}_{x}(S) \to {\rm Set}_{x}(R \wedge S)
\]
が共に成り立つ.
故に推論法則 \ref{deddil}により(\ref{sthmwsm1})が成り立つ.
1), 2)が成り立つことは, (\ref{sthmwsm1})と推論法則 \ref{dedmp}, \ref{dedvee}によって明らかである.
\halmos




\mathstrut
\begin{thm}
\label{sthmisetcap}%定理7.62%確認済
$R$と$S$を関係式とし, $x$を文字とする.
このとき
\begin{equation}
\label{sthmisetcap1}
  {\rm Set}_{x}(R) \wedge {\rm Set}_{x}(S) \to \{x \mid R\} \cap \{x \mid S\} = \{x \mid R \wedge S\}
\end{equation}
が成り立つ.
またこのことから, 次の(\ref{sthmisetcap2})が成り立つ.
\begin{equation}
\label{sthmisetcap2}
  R \text{と} S \text{が} \text{共に} x \text{について集合を作り得るならば,} ~
  \{x \mid R\} \cap \{x \mid S\} = \{x \mid R \wedge S\}.
\end{equation}
\end{thm}


\noindent{\bf 証明}
~変数法則 \ref{valiset}により, $x$は$\{x \mid S\}$の中に自由変数として現れないから, 
定理 \ref{sthmacapiset}より
\begin{equation}
\label{sthmisetcap3}
  {\rm Set}_{x}(R) \to \{x \mid R\} \cap \{x \mid S\} = \{x \in \{x \mid S\} \mid R\}
\end{equation}
が成り立つ.
また定理 \ref{sthmisetsset}より
\begin{equation}
\label{sthmisetcap4}
  {\rm Set}_{x}(S) \to \{x \in \{x \mid S\} \mid R\} = \{x \mid R \wedge S\}
\end{equation}
が成り立つ.
そこで(\ref{sthmisetcap3}), (\ref{sthmisetcap4})から, 推論法則 \ref{dedfromaddw}により
\begin{equation}
\label{sthmisetcap5}
  {\rm Set}_{x}(R) \wedge {\rm Set}_{x}(S) 
  \to \{x \mid R\} \cap \{x \mid S\} = \{x \in \{x \mid S\} \mid R\} \wedge \{x \in \{x \mid S\} \mid R\} = \{x \mid R \wedge S\}
\end{equation}
が成り立つ.
またThm \ref{x=ywy=ztx=z}より
\begin{multline}
\label{sthmisetcap6}
  \{x \mid R\} \cap \{x \mid S\} = \{x \in \{x \mid S\} \mid R\} \wedge \{x \in \{x \mid S\} \mid R\} = \{x \mid R \wedge S\} \\
  \to \{x \mid R\} \cap \{x \mid S\} = \{x \mid R \wedge S\}
\end{multline}
が成り立つ.
そこで(\ref{sthmisetcap5}), (\ref{sthmisetcap6})から, 
推論法則 \ref{dedmmp}によって(\ref{sthmisetcap1})が成り立つ.
(\ref{sthmisetcap2})が成り立つことは, 
(\ref{sthmisetcap1})と推論法則 \ref{dedmp}, \ref{dedwedge}によって明らかである.
\halmos




\mathstrut
\begin{thm}
\label{sthmisetcapsset}%定理7.63%新規%確認済
$a$を集合, $R$と$S$を関係式とし, $x$を$a$の中に自由変数として現れない文字とする.
このとき
\begin{align}
  \label{sthmisetcapsset1}
  &{\rm Set}_{x}(R) \to \{x \mid R\} \cap \{x \in a \mid S\} = \{x \in a \mid R \wedge S\}, \\
  \mbox{} \notag \\
  \label{sthmisetcapsset2}
  &{\rm Set}_{x}(R) \to \{x \in a \mid S\} \cap \{x \mid R\} = \{x \in a \mid S \wedge R\}
\end{align}
が共に成り立つ.
またこれらから, 次の(\ref{sthmisetcapsset3})が成り立つ.
\begin{align}
  \label{sthmisetcapsset3}
  &R \text{が} x \text{について集合を作り得るならば,} ~
  \{x \mid R\} \cap \{x \in a \mid S\} = \{x \in a \mid R \wedge S\} \text{と} \\
  &\{x \in a \mid S\} \cap \{x \mid R\} = \{x \in a \mid S \wedge R\} \text{が共に成り立つ.} \notag
\end{align}
\end{thm}


\noindent{\bf 証明}
~変数法則 \ref{valsset}により, $x$は$\{x \in a \mid S\}$の中に自由変数として現れないから, 
定理 \ref{sthmacapiset}より
\begin{align}
  \label{sthmisetcapsset4}
  &{\rm Set}_{x}(R) \to \{x \mid R\} \cap \{x \in a \mid S\} = \{x \in \{x \in a \mid S\} \mid R\}, \\
  \mbox{} \notag \\
  \label{sthmisetcapsset5}
  &{\rm Set}_{x}(R) \to \{x \in a \mid S\} \cap \{x \mid R\} = \{x \in \{x \in a \mid S\} \mid R\}
\end{align}
が共に成り立つ.
また$x$が$a$の中に自由変数として現れないことから, 定理 \ref{sthmssetsset}より
\[
  \{x \in \{x \in a \mid S\} \mid R\} = \{x \in a \mid R \wedge S\}, ~~
  \{x \in \{x \in a \mid S\} \mid R\} = \{x \in a \mid S \wedge R\}
\]
が共に成り立つ.
故に推論法則 \ref{dedaddeq=}により
\begin{align*}
  &\{x \mid R\} \cap \{x \in a \mid S\} = \{x \in \{x \in a \mid S\} \mid R\} 
  \leftrightarrow \{x \mid R\} \cap \{x \in a \mid S\} = \{x \in a \mid R \wedge S\}, \\
  \mbox{} \notag \\
  &\{x \in a \mid S\} \cap \{x \mid R\} = \{x \in \{x \in a \mid S\} \mid R\} 
  \leftrightarrow \{x \in a \mid S\} \cap \{x \mid R\} = \{x \in a \mid S \wedge R\}
\end{align*}
が共に成り立つ.
故に推論法則 \ref{dedequiv}により
\begin{align}
  \label{sthmisetcapsset6}
  &\{x \mid R\} \cap \{x \in a \mid S\} = \{x \in \{x \in a \mid S\} \mid R\} 
  \to \{x \mid R\} \cap \{x \in a \mid S\} = \{x \in a \mid R \wedge S\}, \\
  \mbox{} \notag \\
  \label{sthmisetcapsset7}
  &\{x \in a \mid S\} \cap \{x \mid R\} = \{x \in \{x \in a \mid S\} \mid R\} 
  \to \{x \in a \mid S\} \cap \{x \mid R\} = \{x \in a \mid S \wedge R\}
\end{align}
が共に成り立つ.
そこで(\ref{sthmisetcapsset4})と(\ref{sthmisetcapsset6}), 
(\ref{sthmisetcapsset5})と(\ref{sthmisetcapsset7})から, 
それぞれ推論法則 \ref{dedmmp}によって(\ref{sthmisetcapsset1}), (\ref{sthmisetcapsset2})が成り立つ.
(\ref{sthmisetcapsset3})が成り立つことは, 
(\ref{sthmisetcapsset1}), (\ref{sthmisetcapsset2})と推論法則 \ref{dedmp}によって明らかである.
\halmos




\mathstrut
\begin{thm}
\label{sthmssetcap}%定理7.64%確認済
$a$と$b$を集合, $R$を関係式とし, $x$を$a$及び$b$の中に自由変数として現れない文字とする.
このとき
\begin{align}
  \label{sthmssetcap1}
  &a \cap \{x \in b \mid R\} = \{x \in a \cap b \mid R\}, \\
  \mbox{} \notag \\
  \label{sthmssetcap2}
  &\{x \in a \mid R\} \cap b = \{x \in a \cap b \mid R\}, \\
  \mbox{} \notag \\
  \label{sthmssetcap3}
  &\{x \in a \mid R\} \cap \{x \in b \mid R\} = \{x \in a \cap b \mid R\}
\end{align}
がすべて成り立つ.
\end{thm}


\noindent{\bf 証明}
~$y$を$a$, $b$, $R$の中に自由変数として現れない, 定数でない文字とする.
このとき変数法則 \ref{valsset}により, 
$y$は$\{x \in a \mid R\}$, $\{x \in b \mid R\}$の中に自由変数として現れない.
また変数法則 \ref{valcap}により, $y$は$a \cap b$の中にも自由変数として現れない.
また$x$が$a$, $b$の中に自由変数として現れないことから, 
変数法則 \ref{valcap}により, $x$も$a \cap b$の中に自由変数として現れない.
さて$x$が$a$, $b$の中に自由変数として現れないことから, 定理 \ref{sthmssetbasis}より
\[
  y \in \{x \in a \mid R\} \leftrightarrow y \in a \wedge (y|x)(R), ~~
  y \in \{x \in b \mid R\} \leftrightarrow y \in b \wedge (y|x)(R)
\]
が共に成り立つ.
故に推論法則 \ref{dedaddeqw}により, 
\begin{align}
  \label{sthmssetcap4}
  &y \in a \wedge y \in \{x \in b \mid R\} \leftrightarrow y \in a \wedge (y \in b \wedge (y|x)(R)), \\
  \mbox{} \notag \\
  \label{sthmssetcap5}
  &y \in \{x \in a \mid R\} \wedge y \in b \leftrightarrow (y \in a \wedge (y|x)(R)) \wedge y \in b, \\
  \mbox{} \notag \\
  \label{sthmssetcap6}
  &y \in \{x \in a \mid R\} \wedge y \in \{x \in b \mid R\} 
  \leftrightarrow (y \in a \wedge (y|x)(R)) \wedge (y \in b \wedge (y|x)(R))
\end{align}
がすべて成り立つ.
またThm \ref{1awb1wclaw1bwc1}と推論法則 \ref{dedeqch}により
\begin{equation}
\label{sthmssetcap7}
  y \in a \wedge (y \in b \wedge (y|x)(R)) \leftrightarrow (y \in a \wedge y \in b) \wedge (y|x)(R)
\end{equation}
が成り立つ.
またThm \ref{thmgwcheq}より
\begin{equation}
\label{sthmssetcap8}
  (y \in a \wedge (y|x)(R)) \wedge y \in b \leftrightarrow (y \in a \wedge y \in b) \wedge (y|x)(R)
\end{equation}
が成り立つ.
またThm \ref{aw1bwc1l1awb1w1awc1}と推論法則 \ref{dedeqch}により
\begin{equation}
\label{sthmssetcap9}
  (y \in a \wedge (y|x)(R)) \wedge (y \in b \wedge (y|x)(R)) 
  \leftrightarrow (y \in a \wedge y \in b) \wedge (y|x)(R)
\end{equation}
が成り立つ.
また定理 \ref{sthmcapbasis}と推論法則 \ref{dedeqch}により
\[
  y \in a \wedge y \in b \leftrightarrow y \in a \cap b
\]
が成り立つから, 推論法則 \ref{dedaddeqw}により
\begin{equation}
\label{sthmssetcap10}
  (y \in a \wedge y \in b) \wedge (y|x)(R) \leftrightarrow y \in a \cap b \wedge (y|x)(R)
\end{equation}
が成り立つ.
そこで(\ref{sthmssetcap4}), (\ref{sthmssetcap7}), (\ref{sthmssetcap10})から, 
推論法則 \ref{dedeqtrans}によって
\[
  y \in a \wedge y \in \{x \in b \mid R\} \leftrightarrow y \in a \cap b \wedge (y|x)(R)
\]
が成り立つことがわかる.
また(\ref{sthmssetcap5}), (\ref{sthmssetcap8}), (\ref{sthmssetcap10})から, 
同じく推論法則 \ref{dedeqtrans}によって
\[
  y \in \{x \in a \mid R\} \wedge y \in b \leftrightarrow y \in a \cap b \wedge (y|x)(R)
\]
が成り立つことがわかる.
また(\ref{sthmssetcap6}), (\ref{sthmssetcap9}), (\ref{sthmssetcap10})から, 
同じく推論法則 \ref{dedeqtrans}によって
\[
  y \in \{x \in a \mid R\} \wedge y \in \{x \in b \mid R\} 
  \leftrightarrow y \in a \cap b \wedge (y|x)(R)
\]
が成り立つことがわかる.
これらのことと, $y$が定数でないことから, 定理 \ref{sthmalleqiset=}より
\begin{align*}
  &\{y \mid y \in a \wedge y \in \{x \in b \mid R\}\} = \{y \mid y \in a \cap b \wedge (y|x)(R)\}, \\
  \mbox{} \notag \\
  &\{y \mid y \in \{x \in a \mid R\} \wedge y \in b\} = \{y \mid y \in a \cap b \wedge (y|x)(R)\}, \\
  \mbox{} \notag \\
  &\{y \mid y \in \{x \in a \mid R\} \wedge y \in \{x \in b \mid R\}\} 
  = \{y \mid y \in a \cap b \wedge (y|x)(R)\}, 
\end{align*}
即ち
\begin{align*}
  &\{y \in a \mid y \in \{x \in b \mid R\}\} = \{y \in a \cap b \mid (y|x)(R)\}, \\
  \mbox{} \notag \\
  &\{y \in \{x \in a \mid R\} \mid y \in b\} = \{y \in a \cap b \mid (y|x)(R)\}, \\
  \mbox{} \notag \\
  &\{y \in \{x \in a \mid R\} \mid y \in \{x \in b \mid R\}\} = \{y \in a \cap b \mid (y|x)(R)\}
\end{align*}
がすべて成り立つ.
ここで上述のように, $y$が$a$, $b$, $R$, $\{x \in a \mid R\}$, $\{x \in b \mid R\}$, $a \cap b$の
いずれの中にも自由変数として現れないことと, $x$が$a \cap b$の中に自由変数として現れないことから, 
定義と代入法則 \ref{substssettrans}により, 
上記の記号列はそれぞれ(\ref{sthmssetcap1}), (\ref{sthmssetcap2}), (\ref{sthmssetcap3})と一致する.
故に(\ref{sthmssetcap1}), (\ref{sthmssetcap2}), (\ref{sthmssetcap3})がすべて成り立つ.
\halmos




\mathstrut
\begin{thm}
\label{sthmssetcaprs}%定理7.65%確認済
$a$を集合, $R$と$S$を関係式とし, $x$を$a$の中に自由変数として現れない文字とする.
このとき
\begin{equation}
\label{sthmssetcaprs1}
  \{x \in a \mid R\} \cap \{x \in a \mid S\} = \{x \in a \mid R \wedge S\}
\end{equation}
が成り立つ.
\end{thm}


\noindent{\bf 証明}
~$y$を$a$, $R$, $S$の中に自由変数として現れない, 定数でない文字とする.
このとき変数法則 \ref{valwedge}により, $y$は$R \wedge S$の中に自由変数として現れない.
また変数法則 \ref{valsset}により, 
$y$は$\{x \in a \mid R\}$, $\{x \in a \mid S\}$の中にも自由変数として現れない.
さて$x$が$a$の中に自由変数として現れないことから, 定理 \ref{sthmssetbasis}より
\[
  y \in \{x \in a \mid R\} \leftrightarrow y \in a \wedge (y|x)(R), ~~
  y \in \{x \in a \mid S\} \leftrightarrow y \in a \wedge (y|x)(S)
\]
が共に成り立つから, 推論法則 \ref{dedaddeqw}により
\begin{equation}
\label{sthmssetcaprs2}
  y \in \{x \in a \mid R\} \wedge y \in \{x \in a \mid S\} 
  \leftrightarrow (y \in a \wedge (y|x)(R)) \wedge (y \in a \wedge (y|x)(S))
\end{equation}
が成り立つ.
またThm \ref{aw1bwc1l1awb1w1awc1}と推論法則 \ref{dedeqch}により
\[
  (y \in a \wedge (y|x)(R)) \wedge (y \in a \wedge (y|x)(S)) 
  \leftrightarrow y \in a \wedge ((y|x)(R) \wedge (y|x)(S))
\]
が成り立つが, 代入法則 \ref{substwedge}によればこの記号列は
\begin{equation}
\label{sthmssetcaprs3}
  (y \in a \wedge (y|x)(R)) \wedge (y \in a \wedge (y|x)(S)) 
  \leftrightarrow y \in a \wedge (y|x)(R \wedge S)
\end{equation}
と一致するから, これが成り立つ.
そこで(\ref{sthmssetcaprs2}), (\ref{sthmssetcaprs3})から, 推論法則 \ref{dedeqtrans}によって
\[
  y \in \{x \in a \mid R\} \wedge y \in \{x \in a \mid S\} 
  \leftrightarrow y \in a \wedge (y|x)(R \wedge S)
\]
が成り立つ.
このことと$y$が定数でないことから, 定理 \ref{sthmalleqiset=}より
\[
  \{y \mid y \in \{x \in a \mid R\} \wedge y \in \{x \in a \mid S\}\} 
  = \{y \mid y \in a \wedge (y|x)(R \wedge S)\}, 
\]
即ち
\[
  \{y \in \{x \in a \mid R\} \mid y \in \{x \in a \mid S\}\} = \{y \in a \mid (y|x)(R \wedge S)\}
\]
が成り立つ.
ここで上述のように, $y$が$\{x \in a \mid R\}$, $\{x \in a \mid S\}$, $a$, $R \wedge S$の
いずれの中にも自由変数として現れないことと, $x$が$a$の中に自由変数として現れないことから, 
定義と代入法則 \ref{substssettrans}により, 上記の記号列は(\ref{sthmssetcaprs1})と一致する.
故に(\ref{sthmssetcaprs1})が成り立つ.
\halmos




\mathstrut
\begin{thm}
\label{sthmssetcapabrs}%定理7.66%新規%確認済
$a$と$b$を集合, $R$と$S$を関係式とし, $x$を$a$及び$b$の中に自由変数として現れない文字とする.
このとき
\begin{equation}
\label{sthmssetcapabrs1}
  \{x \in a \mid R\} \cap \{x \in b \mid S\} = \{x \in a \cap b \mid R \wedge S\}
\end{equation}
が成り立つ.
\end{thm}


\noindent{\bf 証明}
~変数法則 \ref{valsset}により, $x$は$\{x \in a \mid R\}$の中に自由変数として現れないから, 
このことと$x$が$b$の中に自由変数として現れないことから, 定理 \ref{sthmssetcap}より
\begin{equation}
\label{sthmssetcapabrs2}
  \{x \in a \mid R\} \cap \{x \in b \mid S\} = \{x \in \{x \in a \mid R\} \cap b \mid S\}
\end{equation}
が成り立つ.
また$x$が$a$, $b$の中に自由変数として現れないことから, 定理 \ref{sthmssetcap}より
\begin{equation}
\label{sthmssetcapabrs3}
  \{x \in a \mid R\} \cap b = \{x \in a \cap b \mid R\}
\end{equation}
が成り立つ.
ここで上述のように, $x$は$\{x \in a \mid R\}$, $b$の中に自由変数として現れないから, 
変数法則 \ref{valcap}により, $x$は$\{x \in a \mid R\} \cap b$の中に自由変数として現れない.
また変数法則 \ref{valsset}により, $x$は$\{x \in a \cap b \mid R\}$の中にも自由変数として現れない.
そこでこれらのことと(\ref{sthmssetcapabrs3})から, 定理 \ref{sthmsset=}より
\begin{equation}
\label{sthmssetcapabrs4}
  \{x \in \{x \in a \mid R\} \cap b \mid S\} = \{x \in \{x \in a \cap b \mid R\} \mid S\}
\end{equation}
が成り立つ.
また$x$が$a$, $b$の中に自由変数として現れないことから, 変数法則 \ref{valcap}により, 
$x$は$a \cap b$の中に自由変数として現れないから, 定理 \ref{sthmssetsset}より
\begin{equation}
\label{sthmssetcapabrs5}
  \{x \in \{x \in a \cap b \mid R\} \mid S\} = \{x \in a \cap b \mid R \wedge S\}
\end{equation}
が成り立つ.
そこで(\ref{sthmssetcapabrs2}), (\ref{sthmssetcapabrs4}), (\ref{sthmssetcapabrs5})から, 
推論法則 \ref{ded=trans}によって(\ref{sthmssetcapabrs1})が成り立つことがわかる.
\halmos
%ここまで確認





























































\newpage
\mathstrut
\noindent
[\textbf{2}] \textbf{共通部分}




%旧cap跡地




%旧defcap跡地




%旧valcap跡地




%旧substcap跡地




%旧formcap跡地



\mathstrut
\begin{thm}
\label{sthmcapelement}%定理
$a$, $b$, $c$を集合とするとき, 
\[
  c \in a \cap b \leftrightarrow c \in a \wedge c \in b
\]
が成り立つ.
\end{thm}


\noindent{\bf 証明}
~$x$を$a$及び$b$の中に自由変数として現れない文字とすれば, 
定義から$a \cap b$は$\{x \in a|x \in b\}$であるから, 
定理 \ref{sthmssetbasis}より
\[
  c \in a \cap b \leftrightarrow c \in a \wedge (c|x)(x \in b)
\]
が成り立つ.
ここで$x$が$b$の中に自由変数として現れないことから, 
代入法則 \ref{substfree}, \ref{substfund}により, 上記の記号列は
\[
  c \in a \cap b \leftrightarrow c \in a \wedge c \in b
\]
と一致する.
よってこれが定理となる.
\halmos




\mathstrut
\begin{thm}
\label{sthmcap}%定理
$a$と$b$を集合とするとき, 
\[
  a \cap b \subset a, ~~
  a \cap b \subset b
\]
が成り立つ.
\end{thm}


\noindent{\bf 証明}
~$x$を, $a$及び$b$の中に自由変数として現れない, 定数でない文字とする.
このとき変数法則 \ref{valcap}により, $x$は$a \cap b$の中に自由変数として現れない.
また定理 \ref{sthmcapelement}と推論法則 \ref{dedequiv}により, 
\[
  x \in a \cap b \to x \in a \wedge x \in b
\]
が成り立つ.
またThm \ref{awbta}より
\[
  x \in a \wedge x \in b \to x \in a, ~~
  x \in a \wedge x \in b \to x \in b
\]
が共に成り立つ.
そこで推論法則 \ref{dedmmp}により, 
\[
\tag{$*$}
  x \in a \cap b \to x \in a, ~~
  x \in a \cap b \to x \in b
\]
が共に成り立つ.
いま$x$は定数でなく, 上述のように$a$, $b$, $a \cap b$のいずれの記号列の中にも
自由変数として現れないから, ($*$)から, 定理 \ref{sthmsubsetconst}によって
$a \cap b \subset a$と$a \cap b \subset b$が共に成り立つことがわかる.
\halmos




\mathstrut
\begin{thm}
\label{sthmcapdil}%定理
$a$, $b$, $c$を集合とするとき, 
\[
  c \subset a \wedge c \subset b \leftrightarrow c \subset a \cap b
\]
が成り立つ.
またこのことから, 次の($*$)が成り立つ:

($*$) ~~$c \subset a$と$c \subset b$が共に成り立つとき, 
        $c \subset a \cap b$が成り立つ.
        逆に$c \subset a \cap b$が成り立つとき, 
        $c \subset a$と$c \subset b$が共に成り立つ.
\end{thm}


\noindent{\bf 証明}
~まず$c \subset a \wedge c \subset b \leftrightarrow c \subset a \cap b$が
成り立つことを示す.
推論法則 \ref{dedequiv}があるから, 
\begin{align*}
  \tag{1}
  &c \subset a \wedge c \subset b \to c \subset a \cap b, \\
  \mbox{} \\
  \tag{2}
  &c \subset a \cap b \to c \subset a \wedge c \subset b
\end{align*}
が共に成り立つことを示せば良い.

(1)の証明: 
$x$を$a$, $b$, $c$の中に自由変数として現れない文字とする.
また$\tau_{x}(\neg (x \in c \to x \in a \cap b))$を$T$と書く.
$T$は集合であり, 定理 \ref{sthmsubsetbasis}より
\[
  c \subset a \to (T \in c \to T \in a), ~~
  c \subset b \to (T \in c \to T \in b)
\]
が共に成り立つから, 推論法則 \ref{dedfromaddw}により
\[
\tag{3}
  c \subset a \wedge c \subset b \to (T \in c \to T \in a) \wedge (T \in c \to T \in b)
\]
が成り立つ.
またThm \ref{1atb1w1atc1t1atbwc1}より
\[
\tag{4}
  (T \in c \to T \in a) \wedge (T \in c \to T \in b) \to (T \in c \to T \in a \wedge T \in b)
\]
が成り立つ.
また定理 \ref{sthmcapelement}と推論法則 \ref{dedequiv}により
$T \in a \wedge T \in b \to T \in a \cap b$が
成り立つから, 推論法則 \ref{dedaddb}により
\[
  (T \in c \to T \in a \wedge T \in b) \to (T \in c \to T \in a \cap b)
\]
が成り立つ.
いま$x$は$a$及び$b$の中に自由変数として現れないから, 
変数法則 \ref{valcap}により, $x$は$a \cap b$の中にも自由変数として現れない.
また$x$は$c$の中にも自由変数として現れない.
そこで代入法則 \ref{substfree}, \ref{substfund}により, 上記の記号列は
\[
\tag{5}
  (T \in c \to T \in a \wedge T \in b) \to (T|x)(x \in c \to x \in a \cap b)
\]
と一致し, これが定理となる.
また$T$の定義から, Thm \ref{thmallfund1}と推論法則 \ref{dedequiv}により
\[
  (T|x)(x \in c \to x \in a \cap b) \to \forall x(x \in c \to x \in a \cap b)
\]
が成り立つ.
上述のように$x$は$c$及び$a \cap b$の中に自由変数として現れないから, 
定義からこの記号列は
\[
\tag{6}
  (T|x)(x \in c \to x \in a \cap b) \to c \subset a \cap b
\]
と同じである.
よってこれが定理となる.
そこで(3)---(6)から, 推論法則 \ref{dedmmp}によって
(1)が成り立つことがわかる.

(2)の証明: 
定理 \ref{sthmcap}より$a \cap b \subset a$と$a \cap b \subset b$が
共に成り立つから, 推論法則 \ref{dedatawbtrue2}により, 
\[
  c \subset a \cap b \to c \subset a \cap b \wedge a \cap b \subset a, ~~
  c \subset a \cap b \to c \subset a \cap b \wedge a \cap b \subset b
\]
が共に成り立つ.
また定理 \ref{sthmsubsettrans}より
\[
  c \subset a \cap b \wedge a \cap b \subset a \to c \subset a, ~~
  c \subset a \cap b \wedge a \cap b \subset b \to c \subset b
\]
が共に成り立つ.
そこでこれらから, 推論法則 \ref{dedmmp}によって
\[
  c \subset a \cap b \to c \subset a, ~~
  c \subset a \cap b \to c \subset b
\]
が共に成り立つから, 推論法則 \ref{dedprewedge}によって(2)が成り立つ.

さていま$c \subset a$と$c \subset b$が共に成り立つとする.
このとき推論法則 \ref{dedwedge}により$c \subset a \wedge c \subset b$が成り立つ.
また上で示したように, (1), 即ち$c \subset a \wedge c \subset b \to c \subset a \cap b$が成り立つ.
そこで推論法則 \ref{dedmp}により, $c \subset a \cap b$が成り立つ.
逆に$c \subset a \cap b$が成り立つとき, 上で示したように, (2), 即ち
$c \subset a \cap b \to c \subset a \wedge c \subset b$が成り立つから, 
推論法則 \ref{dedmp}により$c \subset a \wedge c \subset b$が成り立つ.
そこで推論法則 \ref{dedwedge}により, $c \subset a$と$c \subset b$が
共に成り立つ.
これで($*$)が示された.
\halmos




%旧sthmcapsubset跡地




%旧sthmcap=跡地




%旧sthmcapidempotent跡地




%旧sthmcapch跡地




%旧sthmcapcomb跡地




%旧sthmcapdist跡地




\mathstrut
\begin{thm}
\label{sthmcapsubset=}%定理
$a$と$b$を集合とするとき, 
\[
  a \subset b \leftrightarrow a \cap b = a
\]
が成り立つ.
\end{thm}


\noindent{\bf 証明}
~推論法則 \ref{dedequiv}があるから, 
\[
  a \subset b \to a \cap b = a, ~~
  a \cap b = a \to a \subset b
\]
が共に成り立つことを示せば良い.

まず前者が成り立つことを示す.
定理 \ref{sthmcapsubset}より
\[
\tag{1}
  a \subset b \to a \cap a \subset a \cap b
\]
が成り立つ.
また定理 \ref{sthmcapidempotent}より$a \cap a = a$が成り立つから, 
推論法則 \ref{dedatawbtrue2}により
\[
\tag{2}
  a \cap a \subset a \cap b \to a \cap a = a \wedge a \cap a \subset a \cap b
\]
が成り立つ.
また定理 \ref{sthm=&subset}より
\[
\tag{3}
  a \cap a = a \wedge a \cap a \subset a \cap b \to a \subset a \cap b
\]
が成り立つ.
そこで(1), (2), (3)から, 推論法則 \ref{dedmmp}によって
\[
\tag{4}
  a \subset b \to a \subset a \cap b
\]
が成り立つ.
また定理 \ref{sthmcap}より$a \cap b \subset a$が成り立つから, 
推論法則 \ref{deds1}により
\[
\tag{5}
  a \subset b \to a \cap b \subset a
\]
が成り立つ.
そこで(4), (5)から, 推論法則 \ref{dedprewedge}によって
\[
\tag{6}
  a \subset b \to a \cap b \subset a \wedge a \subset a \cap b
\]
が成り立つ.
また定理 \ref{sthmaxiom1}と推論法則 \ref{dedequiv}により, 
\[
\tag{7}
  a \cap b \subset a \wedge a \subset a \cap b \to a \cap b = a
\]
が成り立つ.
そこで(6), (7)から, 推論法則 \ref{dedmmp}によって
$a \subset b \to a \cap b = a$が成り立つ.

次に後者が成り立つことを示す.
定理 \ref{sthmcap}より$a \cap b \subset b$が成り立つから, 
推論法則 \ref{dedatawbtrue2}により
\[
  a \cap b = a \to a \cap b = a \wedge a \cap b \subset b
\]
が成り立つ.
また定理 \ref{sthm=&subset}より
\[
  a \cap b = a \wedge a \cap b \subset b \to a \subset b
\]
が成り立つ.
そこでこれらから, 推論法則 \ref{dedmmp}によって
$a \cap b = a \to a \subset b$が成り立つ.
\halmos




\mathstrut
\begin{thm}
\label{sthmsetabs}%定理
$a$と$b$を集合とするとき, 
\[
  (a \cup b) \cap a = a, ~~
  (a \cap b) \cup a = a
\]
が成り立つ.
\end{thm}


\noindent{\bf 証明}
~まず前者から示す.
定理 \ref{sthmcap}より
\[
\tag{1}
  (a \cup b) \cap a \subset a
\]
が成り立つ.
また定理 \ref{sthmsubsetcup}より$a \subset a \cup b$が成り立ち, 
定理 \ref{sthmsubsetself}より$a \subset a$が成り立つから, 
定理 \ref{sthmcapdil}により
\[
\tag{2}
  a \subset (a \cup b) \cap a
\]
が成り立つ.
そこで(1), (2)から, 定理 \ref{sthmaxiom1}によって
$(a \cup b) \cap a = a$が成り立つ.

次に後者が成り立つことを示す.
定理 \ref{sthmcap}より$a \cap b \subset a$が成り立ち, 
定理 \ref{sthmsubsetself}より$a \subset a$が成り立つから, 
定理 \ref{sthmacupbsubsetc}により
\[
\tag{3}
  (a \cap b) \cup a \subset a
\]
が成り立つ.
また定理 \ref{sthmsubsetcup}より
\[
\tag{4}
  a \subset (a \cap b) \cup a
\]
が成り立つ.
そこで(3), (4)から, 定理 \ref{sthmaxiom1}によって
$(a \cap b) \cup a = a$が成り立つ.
\halmos




\mathstrut
\begin{thm}
\label{sthmsetdist}%定理
$a$, $b$, $c$を集合とするとき, 
\begin{align*}
  a \cap (b \cup c) = (a \cap b) \cup (a \cap c)&, ~~(a \cup b) \cap c = (a \cap c) \cup (b \cap c), \\
  \mbox{}& \\
  a \cup (b \cap c) = (a \cup b) \cap (a \cup c)&, ~~(a \cap b) \cup c = (a \cup c) \cap (b \cup c)
\end{align*}
が成り立つ.
\end{thm}


\noindent{\bf 証明}
~$x$を$a$, $b$, $c$の中に自由変数として現れない, 定数でない文字とする.
このとき変数法則 \ref{valcup}, \ref{valcap}からわかるように, 
$x$は$a \cap (b \cup c)$, $(a \cap b) \cup (a \cap c)$, $(a \cup b) \cap c$, $(a \cap c) \cup (b \cap c)$, 
$a \cup (b \cap c)$, $(a \cup b) \cap (a \cup c)$, $(a \cap b) \cup c$, $(a \cup c) \cap (b \cup c)$の
いずれの記号列の中にも自由変数として現れない.

さてまずはじめの二つが成り立つことを示す.
定理 \ref{sthmcapelement}より
\begin{align*}
  \tag{1} x \in a \cap (b \cup c) &\leftrightarrow x \in a \wedge x \in b \cup c, \\
  \mbox{}& \\
  \tag{2} x \in (a \cup b) \cap c &\leftrightarrow x \in a \cup b \wedge x \in c
\end{align*}
が成り立つ.
また定理 \ref{sthmcupbasis}より
$x \in b \cup c \leftrightarrow x \in b \vee x \in c$と
$x \in a \cup b \leftrightarrow x \in a \vee x \in b$が成り立つから, 
推論法則 \ref{dedaddeqw}により
\begin{align*}
  \tag{3} x \in a \wedge x \in b \cup c &\leftrightarrow x \in a \wedge (x \in b \vee x \in c), \\
  \mbox{}& \\
  \tag{4} x \in a \cup b \wedge x \in c &\leftrightarrow (x \in a \vee x \in b) \wedge x \in c
\end{align*}
が成り立つ.
またThm \ref{aw1bvc1l1awb1v1awc1}より
\begin{align*}
  \tag{5} x \in a \wedge (x \in b \vee x \in c) &\leftrightarrow (x \in a \wedge x \in b) \vee (x \in a \wedge x \in c), \\
  \mbox{}& \\
  \tag{6} (x \in a \vee x \in b) \wedge x \in c &\leftrightarrow (x \in a \wedge x \in c) \vee (x \in b \wedge x \in c)
\end{align*}
が成り立つ.
また定理 \ref{sthmcapelement}と推論法則 \ref{dedeqch}により
\[
  x \in a \wedge x \in b \leftrightarrow x \in a \cap b, ~~
  x \in a \wedge x \in c \leftrightarrow x \in a \cap c, ~~
  x \in b \wedge x \in c \leftrightarrow x \in b \cap c
\]
が成り立つから, 推論法則 \ref{dedaddeqv}により
\begin{align*}
  \tag{7} (x \in a \wedge x \in b) \vee (x \in a \wedge x \in c) &\leftrightarrow x \in a \cap b \vee x \in a \cap c, \\
  \mbox{}& \\
  \tag{8} (x \in a \wedge x \in c) \vee (x \in b \wedge x \in c) &\leftrightarrow x \in a \cap c \vee x \in b \cap c
\end{align*}
が成り立つ.
また定理 \ref{sthmcupbasis}と推論法則 \ref{dedeqch}により
\begin{align*}
  \tag{9} x \in a \cap b \vee x \in a \cap c &\leftrightarrow x \in (a \cap b) \cup (a \cap c), \\
  \mbox{}& \\
  \tag{10} x \in a \cap c \vee x \in b \cap c &\leftrightarrow x \in (a \cap c) \cup (b \cap c)
\end{align*}
が成り立つ.
そこで, (1), (3), (5), (7), (9)から推論法則 \ref{dedeqtrans}によって
\[
\tag{11}
  x \in a \cap (b \cup c) \leftrightarrow x \in (a \cap b) \cup (a \cap c)
\]
が成り立つことがわかり, 
(2), (4), (6), (8), (10)から同じく推論法則 \ref{dedeqtrans}によって
\[
\tag{12}
  x \in (a \cup b) \cap c \leftrightarrow x \in (a \cap c) \cup (b \cap c)
\]
が成り立つことがわかる.
いま$x$は定数でなく, 上述のように$a \cap (b \cup c)$, $(a \cap b) \cup (a \cap c)$, 
$(a \cup b) \cap c$, $(a \cap c) \cup (b \cap c)$のいずれの記号列の中にも自由変数として現れないから, 
(11)と(12)から, 定理 \ref{sthmset=}によってそれぞれ
$a \cap (b \cup c) = (a \cap b) \cup (a \cap c)$, $(a \cup b) \cap c = (a \cap c) \cup (b \cap c)$が
成り立つ.
これではじめの二つが成り立つことが示された.

次に後の二つが成り立つことを示す.
定理 \ref{sthmcupbasis}より
\begin{align*}
  \tag{13} x \in a \cup (b \cap c) &\leftrightarrow x \in a \vee x \in b \cap c, \\
  \mbox{}& \\
  \tag{14} x \in (a \cap b) \cup c &\leftrightarrow x \in a \cap b \vee x \in c
\end{align*}
が成り立つ.
また定理 \ref{sthmcapelement}より
$x \in b \cap c \leftrightarrow x \in b \wedge x \in c$と
$x \in a \cap b \leftrightarrow x \in a \wedge x \in b$が成り立つから, 
推論法則 \ref{dedaddeqv}により
\begin{align*}
  \tag{15} x \in a \vee x \in b \cap c &\leftrightarrow x \in a \vee (x \in b \wedge x \in c), \\
  \mbox{}& \\
  \tag{16} x \in a \cap b \vee x \in c &\leftrightarrow (x \in a \wedge x \in b) \vee x \in c
\end{align*}
が成り立つ.
またThm \ref{aw1bvc1l1awb1v1awc1}より
\begin{align*}
  \tag{17} x \in a \vee (x \in b \wedge x \in c) &\leftrightarrow (x \in a \vee x \in b) \wedge (x \in a \vee x \in c), \\
  \mbox{}& \\
  \tag{18} (x \in a \wedge x \in b) \vee x \in c &\leftrightarrow (x \in a \vee x \in c) \wedge (x \in b \vee x \in c)
\end{align*}
が成り立つ.
また定理 \ref{sthmcupbasis}と推論法則 \ref{dedeqch}により
\[
  x \in a \vee x \in b \leftrightarrow x \in a \cup b, ~~
  x \in a \vee x \in c \leftrightarrow x \in a \cup c, ~~
  x \in b \vee x \in c \leftrightarrow x \in b \cup c
\]
が成り立つから, 推論法則 \ref{dedaddeqw}により
\begin{align*}
  \tag{19} (x \in a \vee x \in b) \wedge (x \in a \vee x \in c) &\leftrightarrow x \in a \cup b \wedge x \in a \cup c, \\
  \mbox{}& \\
  \tag{20} (x \in a \vee x \in c) \wedge (x \in b \vee x \in c) &\leftrightarrow x \in a \cup c \wedge x \in b \cup c
\end{align*}
が成り立つ.
また定理 \ref{sthmcapelement}と推論法則 \ref{dedeqch}により
\begin{align*}
  \tag{21} x \in a \cup b \wedge x \in a \cup c &\leftrightarrow x \in (a \cup b) \cap (a \cup c), \\
  \mbox{}& \\
  \tag{22} x \in a \cup c \wedge x \in b \cup c &\leftrightarrow x \in (a \cup c) \cap (b \cup c)
\end{align*}
が成り立つ.
そこで, (13), (15), (17), (19), (21)から推論法則 \ref{dedeqtrans}によって
\[
\tag{23}
  x \in a \cup (b \cap c) \leftrightarrow x \in (a \cup b) \cap (a \cup c)
\]
が成り立つことがわかり, 
(14), (16), (18), (20), (22)から同じく推論法則 \ref{dedeqtrans}によって
\[
\tag{24}
  x \in (a \cap b) \cup c \leftrightarrow x \in (a \cup c) \cap (b \cup c)
\]
が成り立つことがわかる.
いま$x$は定数でなく, はじめに述べたように$a \cup (b \cap c)$, $(a \cup b) \cap (a \cup c)$, 
$(a \cap b) \cup c$, $(a \cup c) \cap (b \cup c)$のいずれの記号列の中にも自由変数として現れないから, 
(23)と(24)から, 定理 \ref{sthmset=}によってそれぞれ
$a \cup (b \cap c) = (a \cup b) \cap (a \cup c)$, $(a \cap b) \cup c = (a \cup c) \cap (b \cup c)$が
成り立つ.
これで後の二つが成り立つことが示された.
\halmos




\mathstrut
\begin{thm}
\label{sthmwsetmake}%定理
$R$と$S$を関係式とし, $x$を文字とする.

1) このとき
   \[
     {\rm Set}_{x}(R) \vee {\rm Set}_{x}(S) \to {\rm Set}_{x}(R \wedge S)
   \]
   が成り立つ.

2) $R$が$x$について集合を作り得るならば, 
   $R \wedge S$は$x$について集合を作り得る.
   また$S$が$x$について集合を作り得るならば, 
   $R \wedge S$は$x$について集合を作り得る.

3) $R$, $S$, $R \wedge S$がすべて$x$について集合を作り得るとき, 
   \[
     \{x|R \wedge S\} = \{x|R\} \cap \{x|S\}
   \]
   が成り立つ.
\end{thm}


\noindent{\bf 証明}
~1)
$\tau_{x}(\neg (R \wedge S \to R))$, $\tau_{x}(\neg (R \wedge S \to S))$をそれぞれ
$T$, $U$と書けば, これらは集合であり, Thm \ref{awbta}より
\[
  (T|x)(R) \wedge (T|x)(S) \to (T|x)(R), ~~
  (U|x)(R) \wedge (U|x)(S) \to (U|x)(S)
\]
が共に成り立つが, 代入法則 \ref{substfund}, \ref{substwedge}によりこれらはそれぞれ
\[
  (T|x)(R \wedge S \to R), ~~
  (U|x)(R \wedge S \to S)
\]
と一致するから, これらが定理となる.
そこで$T$と$U$の定義から, 推論法則 \ref{dedallfund}により
\[
  \forall x(R \wedge S \to R), ~~
  \forall x(R \wedge S \to S)
\]
が共に成り立つ.
また定理 \ref{sthmalltsm}より
\[
  \forall x(R \wedge S \to R) \to ({\rm Set}_{x}(R) \to {\rm Set}_{x}(R \wedge S)), ~~
  \forall x(R \wedge S \to S) \to ({\rm Set}_{x}(S) \to {\rm Set}_{x}(R \wedge S))
\]
が共に成り立つ.
そこで推論法則 \ref{dedmp}により, 
\begin{align*}
  \tag{1}
  {\rm Set}_{x}(R) &\to {\rm Set}_{x}(R \wedge S), \\
  \mbox{} \\
  \tag{2}
  {\rm Set}_{x}(S) &\to {\rm Set}_{x}(R \wedge S)
\end{align*}
が共に成り立ち, これらから, 推論法則 \ref{deddil}によって
${\rm Set}_{x}(R) \vee {\rm Set}_{x}(S) \to {\rm Set}_{x}(R \wedge S)$が成り立つ.

\noindent
2)
いま上で示したように, (1)と(2)が共に成り立つ.
そこで${\rm Set}_{x}(R)$が成り立つとき, これと(1)から, 
推論法則 \ref{dedmp}によって${\rm Set}_{x}(R \wedge S)$が成り立つ.
また${\rm Set}_{x}(S)$が成り立つとき, これと(2)から, 
推論法則 \ref{dedmp}によって${\rm Set}_{x}(R \wedge S)$が成り立つ.

\noindent
3)
いま$R$, $S$, $R \wedge S$がすべて$x$について集合を作り得るとする.
また$y$を, $R$及び$S$の中に自由変数として現れない, 定数でない文字とする.
このとき定理 \ref{sthmisetbasis}より
\[
\tag{3}
  y \in \{x|R \wedge S\} \leftrightarrow (y|x)(R \wedge S)
\]
が成り立ち, 定理 \ref{sthmisetbasis}と推論法則 \ref{dedeqch}により, 
\[
  (y|x)(R) \leftrightarrow y \in \{x|R\}, ~~
  (y|x)(S) \leftrightarrow y \in \{x|S\}
\]
が共に成り立つ.
この後者の二つの定理から, 推論法則 \ref{dedaddeqw}によって
\[
  (y|x)(R) \wedge (y|x)(S) \leftrightarrow y \in \{x|R\} \wedge y \in \{x|S\}
\]
が成り立つが, 代入法則 \ref{substwedge}により, この記号列は
\[
\tag{4}
  (y|x)(R \wedge S) \leftrightarrow y \in \{x|R\} \wedge y \in \{x|S\}
\]
と一致するから, これが定理となる.
また定理 \ref{sthmcapelement}と推論法則 \ref{dedeqch}により, 
\[
\tag{5}
  y \in \{x|R\} \wedge y \in \{x|S\} \leftrightarrow y \in \{x|R\} \cap \{x|S\}
\]
が成り立つ.
そこで(3)---(5)から, 推論法則 \ref{dedeqtrans}によって
\[
\tag{6}
  y \in \{x|R \wedge S\} \leftrightarrow y \in \{x|R\} \cap \{x|S\}
\]
が成り立つ.
いま$y$は$R$及び$S$の中に自由変数として現れない文字だから, 
変数法則 \ref{valwedge}により, $y$は$R \wedge S$の中にも自由変数として現れない.
そこで変数法則 \ref{valiset}により, $y$は$\{x|R\}$, $\{x|S\}$, $\{x|R \wedge S\}$のいずれの記号列の中にも
自由変数として現れない.
そこで変数法則 \ref{valcap}により, $y$は$\{x|R\} \cap \{x|S\}$の中にも自由変数として現れない.
また$y$は定数でない.
そこでこれらのことと, (6)が成り立つことから, 定理 \ref{sthmset=}によって
$\{x|R \wedge S\} = \{x|R\} \cap \{x|S\}$が成り立つことがわかる.
\halmos




\mathstrut
\begin{thm}
\label{sthmcapsingleton}%定理
$a$と$b$を集合とするとき, 
\[
  a = b \leftrightarrow \{a\} \cap \{b\} = \{a\}, ~~
  a = b \leftrightarrow \{a\} \cap \{b\} = \{b\}
\]
が成り立つ.
\end{thm}


\noindent{\bf 証明}
~まず前者が成り立つことを示す.
定理 \ref{sthmsingletonbasis}と推論法則 \ref{dedeqch}により
\[
\tag{1}
  a = b \leftrightarrow a \in \{b\}
\]
が成り立つ.
また定理 \ref{sthmsingletonfund}より
$a \in \{a\}$が成り立つから, 推論法則 \ref{dedawblatrue2}により
\[
  a \in \{a\} \wedge a \in \{b\} \leftrightarrow a \in \{b\}
\]
が成り立つ.
そこで推論法則 \ref{dedeqch}により, 
\[
\tag{2}
  a \in \{b\} \leftrightarrow a \in \{a\} \wedge a \in \{b\}
\]
が成り立つ.
また定理 \ref{sthmcapelement}と推論法則 \ref{dedeqch}により, 
\[
\tag{3}
  a \in \{a\} \wedge a \in \{b\} \leftrightarrow a \in \{a\} \cap \{b\}
\]
が成り立つ.
また定理 \ref{sthmsingletonsubset}と推論法則 \ref{dedeqch}により, 
\[
\tag{4}
  a \in \{a\} \cap \{b\} \leftrightarrow \{a\} \subset \{a\} \cap \{b\}
\]
が成り立つ.
また定理 \ref{sthmcap}より
$\{a\} \cap \{b\} \subset \{a\}$が成り立つから, 
推論法則 \ref{dedawblatrue2}により
\[
  \{a\} \cap \{b\} \subset \{a\} \wedge \{a\} \subset \{a\} \cap \{b\} \leftrightarrow \{a\} \subset \{a\} \cap \{b\}
\]
が成り立つ.
そこで推論法則 \ref{dedeqch}により, 
\[
\tag{5}
  \{a\} \subset \{a\} \cap \{b\} \leftrightarrow \{a\} \cap \{b\} \subset \{a\} \wedge \{a\} \subset \{a\} \cap \{b\}
\]
が成り立つ.
また定理 \ref{sthmaxiom1}より
\[
\tag{6}
  \{a\} \cap \{b\} \subset \{a\} \wedge \{a\} \subset \{a\} \cap \{b\} \leftrightarrow \{a\} \cap \{b\} = \{a\}
\]
が成り立つ.
そこで(1)---(6)から, 推論法則 \ref{dedeqtrans}によって
$a = b \leftrightarrow \{a\} \cap \{b\} = \{a\}$が成り立つことがわかる.

次に後者が成り立つことを示す.
Thm \ref{x=yly=x}より
\[
\tag{7}
  a = b \leftrightarrow b = a
\]
が成り立つ.
またいま示したことから, 
\[
\tag{8}
  b = a \leftrightarrow \{b\} \cap \{a\} = \{b\}
\]
が成り立つ.
また定理 \ref{sthmcapch}より
$\{b\} \cap \{a\} = \{a\} \cap \{b\}$が成り立つから, 
推論法則 \ref{dedaddeq=}により
\[
\tag{9}
  \{b\} \cap \{a\} = \{b\} \leftrightarrow \{a\} \cap \{b\} = \{b\}
\]
が成り立つ.
そこで(7), (8), (9)から, 推論法則 \ref{dedeqtrans}によって
$a = b \leftrightarrow \{a\} \cap \{b\} = \{b\}$が成り立つことがわかる.
\halmos




\mathstrut
\begin{thm}
\label{sthmsetsepcap}%定理
$a$を集合, $R$を関係式とし, $x$を$a$の中に自由変数として現れない文字とする.
$R$が$x$について集合を作り得るならば, 
\[
  \{x \in a|R\} = a \cap \{x|R\}
\]
が成り立つ.
\end{thm}


\noindent{\bf 証明}
~定義から$\{x \in a|R\}$は$\{x|x \in a \wedge R\}$であり, 
$x$が$a$の中に自由変数として現れないという仮定から, 
定理 \ref{sthmssetsm}より$x \in a \wedge R$は$x$について集合を作り得る.
また同じ仮定から, \S 2例1より$x \in a$は
$x$について集合を作り得る.
また仮定より$R$は$x$について集合を作り得る.
そこで定理 \ref{sthmwsetmake}により, 
\[
  \{x \in a|R\} = \{x|x \in a\} \cap \{x|R\}
\]
が成り立つ.
またいま\S 2例3より$\{x|x \in a\} = a$が成り立つから, 
定理 \ref{sthmcap=}により
\[
  \{x|x \in a\} \cap \{x|R\} = a \cap \{x|R\}
\]
が成り立つ.
そこでこれらから, 推論法則 \ref{ded=trans}によって
$\{x \in a|R\} = a \cap \{x|R\}$が成り立つ.
\halmos




\mathstrut
\begin{thm}
\label{sthmrwssetsep}%定理
$a$を集合, $R$と$S$を関係式とし, $x$を$a$の中に自由変数として現れない文字とする.
このとき
\[
  \{x \in a|R \wedge S\} = \{x \in a|R\} \cap \{x \in a|S\}
\]
が成り立つ.
\end{thm}


\noindent{\bf 証明}
~$y$を$a$, $R$, $S$のいずれの記号列の中にも自由変数として現れない, 定数でない文字とする.
このとき, $x$が$a$の中に自由変数として現れないことから, 
定理 \ref{sthmssetbasis}より
\[
  y \in \{x \in a|R \wedge S\} \leftrightarrow y \in a \wedge (y|x)(R \wedge S)
\]
が成り立つが, 代入法則 \ref{substwedge}によりこの記号列は
\[
\tag{1}
  y \in \{x \in a|R \wedge S\} \leftrightarrow y \in a \wedge ((y|x)(R) \wedge (y|x)(S))
\]
と一致するから, これが定理となる.
またThm \ref{aw1bwc1l1awb1w1awc1}より
\[
\tag{2}
  y \in a \wedge ((y|x)(R) \wedge (y|x)(S)) \leftrightarrow (y \in a \wedge (y|x)(R)) \wedge (y \in a \wedge (y|x)(S))
\]
が成り立つ.
また上記と同様に, $x$が$a$の中に自由変数として現れないことから, 
定理 \ref{sthmssetbasis}と推論法則 \ref{dedeqch}により, 
\[
  y \in a \wedge (y|x)(R) \leftrightarrow y \in \{x \in a|R\}, ~~
  y \in a \wedge (y|x)(S) \leftrightarrow y \in \{x \in a|S\}
\]
が共に成り立つ.
そこで推論法則 \ref{dedaddeqw}により, 
\[
\tag{3}
  (y \in a \wedge (y|x)(R)) \wedge (y \in a \wedge (y|x)(S)) \leftrightarrow y \in \{x \in a|R\} \wedge y \in \{x \in a|S\}
\]
が成り立つ.
また定理 \ref{sthmcapelement}と推論法則 \ref{dedeqch}により, 
\[
\tag{4}
  y \in \{x \in a|R\} \wedge y \in \{x \in a|S\} \leftrightarrow y \in \{x \in a|R\} \cap \{x \in a|S\}
\]
が成り立つ.
そこで(1)---(4)から, 推論法則 \ref{dedeqtrans}によって
\[
\tag{5}
  y \in \{x \in a|R \wedge S\} \leftrightarrow y \in \{x \in a|R\} \cap \{x \in a|S\}
\]
が成り立つことがわかる.
いま$y$は$R$及び$S$の中に自由変数として現れないから, 変数法則 \ref{valwedge}により, 
$y$は$R \wedge S$の中にも自由変数として現れない.
また$y$は$a$の中にも自由変数として現れない.
そこで変数法則 \ref{valsset}により, $y$は
$\{x \in a|R\}$, $\{x \in a|S\}$, $\{x \in a|R \wedge S\}$のいずれの記号列の中にも自由変数として現れない.
そこで変数法則 \ref{valcap}により, $y$は$\{x \in a|R\} \cap \{x \in a|S\}$の中にも自由変数として現れない.
また$y$は定数でない.
そこでこれらのことと, (5)が成り立つことから, 定理 \ref{sthmset=}によって
$\{x \in a|R \wedge S\} = \{x \in a|R\} \cap \{x \in a|S\}$が成り立つことがわかる.
\halmos




\mathstrut
\begin{thm}
\label{sthmacapbsetsep}%定理
$a$と$b$を集合, $R$を関係式とし, $x$を$a$及び$b$の中に自由変数として現れない文字とする.
このとき
\begin{align*}
  \{x \in a \cap b|R\} &= a \cap \{x \in b|R\}, \\
  \mbox{} \\
  \{x \in a \cap b|R\} &= \{x \in a|R\} \cap b, \\
  \mbox{} \\
  \{x \in a \cap b|R\} &= \{x \in a|R\} \cap \{x \in b|R\}
\end{align*}
が成り立つ.
\end{thm}


\noindent{\bf 証明}
~$y$を$a$, $b$, $R$のいずれの記号列の中にも自由変数として現れない, 定数でない文字とする.
このとき変数法則 \ref{valsset}, \ref{valcap}からわかるように, $y$は
$\{x \in a \cap b|R\}$, $a \cap \{x \in b|R\}$, $\{x \in a|R\} \cap b$, $\{x \in a|R\} \cap \{x \in b|R\}$の
いずれの記号列の中にも自由変数として現れない.
また$x$が$a$及び$b$の中に自由変数として現れないことから, 変数法則 \ref{valcap}により
$x$は$a \cap b$の中に自由変数として現れないから, 
定理 \ref{sthmssetbasis}より
\[
\tag{1}
  y \in \{x \in a \cap b|R\} \leftrightarrow y \in a \cap b \wedge (y|x)(R)
\]
が成り立つ.
また定理 \ref{sthmcapelement}より
$y \in a \cap b \leftrightarrow y \in a \wedge y \in b$が成り立つから, 
推論法則 \ref{dedaddeqw}により, 
\[
\tag{2}
  y \in a \cap b \wedge (y|x)(R) \leftrightarrow (y \in a \wedge y \in b) \wedge (y|x)(R)
\]
が成り立つ.
またThm \ref{1awb1wclaw1bwc1}より
\[
\tag{3}
  (y \in a \wedge y \in b) \wedge (y|x)(R) \leftrightarrow y \in a \wedge (y \in b \wedge (y|x)(R))
\]
が成り立つ.
またThm \ref{awblbwa}より$y \in b \wedge (y|x)(R) \leftrightarrow (y|x)(R) \wedge y \in b$が成り立つから, 
推論法則 \ref{dedaddeqw}により
\[
\tag{4}
  y \in a \wedge (y \in b \wedge (y|x)(R)) \leftrightarrow y \in a \wedge ((y|x)(R) \wedge y \in b)
\]
が成り立つ.
またThm \ref{1awb1wclaw1bwc1}と推論法則 \ref{dedeqch}により
\[
\tag{5}
  y \in a \wedge ((y|x)(R) \wedge y \in b) \leftrightarrow (y \in a \wedge (y|x)(R)) \wedge y \in b
\]
が成り立つ.
またThm \ref{aw1bwc1l1awb1w1awc1}より
\[
\tag{6}
  (y \in a \wedge y \in b) \wedge (y|x)(R) \leftrightarrow (y \in a \wedge (y|x)(R)) \wedge (y \in b \wedge (y|x)(R))
\]
が成り立つ.
また$x$が$a$及び$b$の中に自由変数として現れないことから, 
定理 \ref{sthmssetbasis}と推論法則 \ref{dedeqch}により, 
\[
  y \in a \wedge (y|x)(R) \leftrightarrow y \in \{x \in a|R\}, ~~
  y \in b \wedge (y|x)(R) \leftrightarrow y \in \{x \in b|R\}
\]
が共に成り立つ.
そこで推論法則 \ref{dedaddeqw}により, 
\begin{align*}
  \tag{7}
  y \in a \wedge (y \in b \wedge (y|x)(R)) &\leftrightarrow y \in a \wedge y \in \{x \in b|R\}, \\
  \mbox{} \\
  \tag{8}
  (y \in a \wedge (y|x)(R)) \wedge y \in b &\leftrightarrow y \in \{x \in a|R\} \wedge y \in b, \\
  \mbox{} \\
  \tag{9}
  (y \in a \wedge (y|x)(R)) \wedge (y \in b \wedge (y|x)(R)) &\leftrightarrow y \in \{x \in a|R\} \wedge y \in \{x \in b|R\}
\end{align*}
が成り立つ.
また定理 \ref{sthmcapelement}と推論法則 \ref{dedeqch}により, 
\begin{align*}
  \tag{10}
  y \in a \wedge y \in \{x \in b|R\} &\leftrightarrow y \in a \cap \{x \in b|R\}, \\
  \mbox{} \\
  \tag{11}
  y \in \{x \in a|R\} \wedge y \in b &\leftrightarrow y \in \{x \in a|R\} \cap b, \\
  \mbox{} \\
  \tag{12}
  y \in \{x \in a|R\} \wedge y \in \{x \in b|R\} &\leftrightarrow y \in \{x \in a|R\} \cap \{x \in b|R\}
\end{align*}
が成り立つ.
そこで(1), (2), (3), (7), (10)から, 推論法則 \ref{dedeqtrans}によって
\[
\tag{13}
  y \in \{x \in a \cap b|R\} \leftrightarrow y \in a \cap \{x \in b|R\}
\]
が成り立ち, 
(1)---(5), (8), (11)から, 同じく推論法則 \ref{dedeqtrans}によって
\[
\tag{14}
  y \in \{x \in a \cap b|R\} \leftrightarrow y \in \{x \in a|R\} \cap b
\]
が成り立ち, 
(1), (2), (6), (9), (12)から, やはり推論法則 \ref{dedeqtrans}によって
\[
\tag{15}
  y \in \{x \in a \cap b|R\} \leftrightarrow y \in \{x \in a|R\} \cap \{x \in b|R\}
\]
が成り立つことがわかる.
いま$y$は定数でなく, はじめに述べたように
$\{x \in a \cap b|R\}$, $a \cap \{x \in b|R\}$, $\{x \in a|R\} \cap b$, $\{x \in a|R\} \cap \{x \in b|R\}$の
いずれの記号列の中にも自由変数として現れないから, (13), (14), (15)から, 定理 \ref{sthmset=}によって
$\{x \in a \cap b|R\} = a \cap \{x \in b|R\}$, $\{x \in a \cap b|R\} = \{x \in a|R\} \cap b$, 
$\{x \in a \cap b|R\} = \{x \in a|R\} \cap \{x \in b|R\}$がすべて成り立つ.
\halmos




\mathstrut
\begin{thm}
\label{sthmcapobjectset}%定理
$a$, $b$, $T$を集合とし, $x$を$a$及び$b$の中に自由変数として現れない文字とする.
このとき
\[
  \{T|x \in a \cap b\} \subset \{T|x \in a\} \cap \{T|x \in b\}
\]
が成り立つ.
\end{thm}


\noindent{\bf 証明}
~仮定と変数法則 \ref{valcap}により, $x$は$a$, $b$, $a \cap b$のいずれの記号列の中にも
自由変数として現れない.
このことと, 定理 \ref{sthmcap}より
\[
  a \cap b \subset a, ~~
  a \cap b \subset b
\]
が共に成り立つことから, 
定理 \ref{sthmosetsubset}によって
\[
  \{T|x \in a \cap b\} \subset \{T|x \in a\}, ~~
  \{T|x \in a \cap b\} \subset \{T|x \in b\}
\]
が共に成り立つことがわかる.
そこでこれらから, 定理 \ref{sthmcapdil}によって
$\{T|x \in a \cap b\} \subset \{T|x \in a\} \cap \{T|x \in b\}$が成り立つ.
\halmos




\mathstrut
{\small
\noindent
{\bf 註.}~
上記の定理 \ref{sthmcapobjectset}において, 逆の包含関係は必ずしも成り立たない
(後の例1参照).
}
%[2]確認済



\mathstrut
\noindent
[\textbf{3}] \textbf{差集合}




%kokoko









\mathstrut
\begin{thm}
\label{sthm-abs}%定理
$a$と$b$を集合とするとき, 
\[
  a - b = (a \cup b) - b, ~~
  a - b = a - (a \cap b)
\]
が成り立つ.
\end{thm}


\noindent{\bf 証明}
~$x$を$a$及び$b$の中に自由変数として現れない, 定数でない文字とする.
このとき変数法則 \ref{valcup}, \ref{valcap}, \ref{val-}からわかるように, 
$x$は$a - b$, $(a \cup b) - b$, $a - (a \cap b)$のいずれの記号列の中にも
自由変数として現れない.

さてまず前者が成り立つことを示す.
定理 \ref{sthm-basis}より
\[
\tag{1}
  x \in a - b \leftrightarrow x \in a \wedge x \notin b
\]
が成り立つ.
またThm \ref{n1awna1}より$\neg (x \in b \wedge x \notin b)$が成り立つから, 
推論法則 \ref{dedavblbtrue2}, \ref{dedeqch}により
\[
\tag{2}
  x \in a \wedge x \notin b \leftrightarrow 
  (x \in a \wedge x \notin b) \vee (x \in b \wedge x \notin b)
\]
が成り立つ.
またThm \ref{aw1bvc1l1awb1v1awc1}と推論法則 \ref{dedeqch}により
\[
\tag{3}
  (x \in a \wedge x \notin b) \vee (x \in b \wedge x \notin b) \leftrightarrow 
  (x \in a \vee x \in b) \wedge x \notin b
\]
が成り立つ.
また定理 \ref{sthmcupbasis}と推論法則 \ref{dedeqch}により
$x \in a \vee x \in b \leftrightarrow x \in a \cup b$が成り立つから, 
推論法則 \ref{dedaddeqw}により
\[
\tag{4}
  (x \in a \vee x \in b) \wedge x \notin b \leftrightarrow x \in a \cup b \wedge x \notin b
\]
が成り立つ.
また定理 \ref{sthm-basis}と推論法則 \ref{dedeqch}により
\[
\tag{5}
  x \in a \cup b \wedge x \notin b \leftrightarrow x \in (a \cup b) - b
\]
が成り立つ.
以上の(1)---(5)から, 推論法則 \ref{dedeqtrans}によって
\[
\tag{6}
  x \in a - b \leftrightarrow x \in (a \cup b) - b
\]
が成り立つことがわかる.
いま$x$は定数でなく, 上述のように$a - b$及び$(a \cup b) - b$の中に
自由変数として現れないから, 
(6)から, 定理 \ref{sthmset=}によって$a - b = (a \cup b) - b$が成り立つ.

次に後者が成り立つことを示す.
Thm \ref{n1awna1}より$\neg (x \in a \wedge x \notin a)$が成り立つから, 
推論法則 \ref{dedavblbtrue2}, \ref{dedeqch}により
\[
\tag{7}
  x \in a \wedge x \notin b \leftrightarrow 
  (x \in a \wedge x \notin a) \vee (x \in a \wedge x \notin b)
\]
が成り立つ.
またThm \ref{aw1bvc1l1awb1v1awc1}と推論法則 \ref{dedeqch}により
\[
\tag{8}
  (x \in a \wedge x \notin a) \vee (x \in a \wedge x \notin b) \leftrightarrow 
  x \in a \wedge (x \notin a \vee x \notin b)
\]
が成り立つ.
またThm \ref{n1awb1lnavnb}と推論法則 \ref{dedeqch}により
\[
\tag{9}
  x \notin a \vee x \notin b \leftrightarrow \neg (x \in a \wedge x \in b)
\]
が成り立つ.
また定理 \ref{sthmcapelement}と推論法則 \ref{dedeqch}により
$x \in a \wedge x \in b \leftrightarrow x \in a \cap b$が成り立つから, 
推論法則 \ref{dedeqcp}により
\[
\tag{10}
  \neg (x \in a \wedge x \in b) \leftrightarrow x \notin a \cap b
\]
が成り立つ.
そこで(9), (10)から, 推論法則 \ref{dedeqtrans}によって
$x \notin a \vee x \notin b \leftrightarrow x \notin a \cap b$が成り立ち, 
これから推論法則 \ref{dedaddeqw}によって
\[
\tag{11}
  x \in a \wedge (x \notin a \vee x \notin b) \leftrightarrow x \in a \wedge x \notin a \cap b
\]
が成り立つ.
また定理 \ref{sthm-basis}と推論法則 \ref{dedeqch}により
\[
\tag{12}
  x \in a \wedge x \notin a \cap b \leftrightarrow x \in a - (a \cap b)
\]
が成り立つ.
以上の(1), (7), (8), (11), (12)から, 推論法則 \ref{dedeqtrans}によって
\[
\tag{13}
  x \in a - b \leftrightarrow x \in a - (a \cap b)
\]
が成り立つことがわかる.
いま$x$は定数でなく, はじめに述べたように$a - b$及び$a - (a \cap b)$の中に
自由変数として現れないから, 
(13)から, 定理 \ref{sthmset=}によって$a - b = a - (a \cap b)$が成り立つ.
\halmos




\mathstrut
\begin{thm}
\label{sthmsubsetfrom-abs}%定理
$a$, $b$, $c$を集合とするとき, 
\[
  a \cup b \subset a \cup c \leftrightarrow b - a \subset c - a, ~~
  a \cap b \subset a \cap c \leftrightarrow a - c \subset a - b
\]
が成り立つ.
\end{thm}


\noindent{\bf 証明}
~まず前者が成り立つことを示す.
定理 \ref{sthmcupch}より
\[
  a \cup b = b \cup a, ~~
  a \cup c = c \cup a
\]
が共に成り立つから, 定理 \ref{sthm=tsubseteq}により, 
\begin{align*}
  \tag{1}
  a \cup b \subset a \cup c &\leftrightarrow b \cup a \subset a \cup c, \\
  \mbox{} \\
  \tag{2}
  b \cup a \subset a \cup c &\leftrightarrow b \cup a \subset c \cup a
\end{align*}
が共に成り立つ.
また定理 \ref{sthmsubsetcup}より$a \subset c \cup a$が成り立つから, 
定理 \ref{sthm-subseteq}により, 
\[
\tag{3}
  b \cup a \subset c \cup a \leftrightarrow (b \cup a) - a \subset (c \cup a) - a
\]
が成り立つ.
また定理 \ref{sthm-abs}より
\[
  b - a = (b \cup a) - a, ~~
  c - a = (c \cup a) - a
\]
が共に成り立つから, 推論法則 \ref{ded=ch}により
\[
  (b \cup a) - a = b - a, ~~
  (c \cup a) - a = c - a
\]
が共に成り立つ.
そこで定理 \ref{sthm=tsubseteq}により, 
\begin{align*}
  \tag{4}
  (b \cup a) - a \subset (c \cup a) - a &\leftrightarrow b - a \subset (c \cup a) - a, \\
  \mbox{} \\
  \tag{5}
  b - a \subset (c \cup a) - a &\leftrightarrow b - a \subset c - a
\end{align*}
が共に成り立つ.
そこで(1)---(5)から, 推論法則 \ref{dedeqtrans}によって
$a \cup b \subset a \cup c \leftrightarrow b - a \subset c - a$が成り立つことがわかる.

次に後者が成り立つことを示す.
定理 \ref{sthmcap}より$a \cap b \subset a$が成り立つから, 
定理 \ref{sthm-subseteq}により, 
\[
\tag{6}
  a \cap b \subset a \cap c \leftrightarrow a - (a \cap c) \subset a - (a \cap b)
\]
が成り立つ.
また定理 \ref{sthm-abs}より
\[
  a - c = a - (a \cap c), ~~
  a - b = a - (a \cap b)
\]
が共に成り立つから, 推論法則 \ref{ded=ch}により
\[
  a - (a \cap c) = a - c, ~~
  a - (a \cap b) = a - b
\]
が共に成り立つ.
そこで定理 \ref{sthm=tsubseteq}により, 
\begin{align*}
  \tag{7}
  a - (a \cap c) \subset a - (a \cap b) &\leftrightarrow a - c \subset a - (a \cap b), \\
  \mbox{} \\
  \tag{8}
  a - c \subset a - (a \cap b) &\leftrightarrow a - c \subset a - b
\end{align*}
が共に成り立つ.
そこで(6), (7), (8)から, 推論法則 \ref{dedeqtrans}によって
$a \cap b \subset a \cap c \leftrightarrow a - c \subset a - b$が成り立つことがわかる.
\halmos




\mathstrut
\begin{thm}
\label{sthm=from-abs}%定理
$a$, $b$, $c$を集合とするとき, 
\[
  a \cup b = a \cup c \leftrightarrow b - a = c - a, ~~
  a \cap b = a \cap c \leftrightarrow a - b = a - c
\]
が成り立つ.
\end{thm}


\noindent{\bf 証明}
~まず前者が成り立つことを示す.
定理 \ref{sthmsubsetcup}より$a \subset a \cup b$と$a \subset a \cup c$が共に成り立つから, 
定理 \ref{sthm-=eq}により
\[
\tag{1}
  a \cup b = a \cup c \leftrightarrow (a \cup b) - a = (a \cup c) - a
\]
が成り立つ.
また定理 \ref{sthmcupch}より$a \cup b = b \cup a$と$a \cup c = c \cup a$が共に成り立つから, 
定理 \ref{sthm-=}により
\[
  (a \cup b) - a = (b \cup a) - a, ~~
  (a \cup c) - a = (c \cup a) - a
\]
が共に成り立つ.
そこで推論法則 \ref{dedaddeq=}により, 
\[
\tag{2}
  (a \cup b) - a = (a \cup c) - a \leftrightarrow (b \cup a) - a = (c \cup a) - a
\]
が成り立つ.
また定理 \ref{sthm-abs}より$b - a = (b \cup a) - a$と$c - a = (c \cup a) - a$が共に成り立つから, 
推論法則 \ref{ded=ch}により
\[
  (b \cup a) - a = b - a, ~~
  (c \cup a) - a = c - a
\]
が共に成り立つ.
そこで推論法則 \ref{dedaddeq=}により, 
\[
\tag{3}
  (b \cup a) - a = (c \cup a) - a \leftrightarrow b - a = c - a
\]
が成り立つ.
そこで(1), (2), (3)から, 推論法則 \ref{dedeqtrans}によって
$a \cup b = a \cup c \leftrightarrow b - a = c - a$が成り立つことがわかる.

次に後者が成り立つことを示す.
定理 \ref{sthmcap}より$a \cap b \subset a$と$a \cap c \subset a$が共に成り立つから, 
定理 \ref{sthm-=eq}により
\[
\tag{4}
  a \cap b = a \cap c \leftrightarrow a - (a \cap b) = a - (a \cap c)
\]
が成り立つ.
また定理 \ref{sthm-abs}より$a - b = a - (a \cap b)$と$a - c = a - (a \cap c)$が共に成り立つから, 
推論法則 \ref{ded=ch}により
\[
  a - (a \cap b) = a - b, ~~
  a - (a \cap c) = a - c
\]
が共に成り立つ.
そこで推論法則 \ref{dedaddeq=}により, 
\[
\tag{5}
  a - (a \cap b) = a - (a \cap c) \leftrightarrow a - b = a - c
\]
が成り立つ.
そこで(4), (5)から, 推論法則 \ref{dedeqtrans}によって
$a \cap b = a \cap c \leftrightarrow a - b = a - c$が成り立つ.
\halmos




\mathstrut
\begin{thm}
\label{sthm-dm}%定理
$a$, $b$, $c$を集合とするとき, 
\[
  a - (b \cup c) = (a - b) \cap (a - c), ~~
  a - (b \cap c) = (a - b) \cup (a - c)
\]
が成り立つ.
\end{thm}


\noindent{\bf 証明}
~$x$を, $a$, $b$, $c$の中に自由変数として現れない, 定数でない文字とする.
このとき変数法則 \ref{valcup}, \ref{valcap}, \ref{val-}からわかるように, 
$x$は$a - (b \cup c)$, $(a - b) \cap (a - c)$, $a - (b \cap c)$, $(a - b) \cup (a - c)$の
いずれの記号列の中にも自由変数として現れない.

さてまず前者が成り立つことを示す.
定理 \ref{sthm-basis}より
\[
\tag{1}
  x \in a - (b \cup c) \leftrightarrow x \in a \wedge x \notin b \cup c
\]
が成り立つ.
また定理 \ref{sthmcupbasis}より
$x \in b \cup c \leftrightarrow x \in b \vee x \in c$が成り立つから, 
推論法則 \ref{dedeqcp}により
\[
\tag{2}
  x \notin b \cup c \leftrightarrow \neg (x \in b \vee x \in c)
\]
が成り立つ.
またThm \ref{n1awb1lnavnb}より
\[
\tag{3}
  \neg (x \in b \vee x \in c) \leftrightarrow x \notin b \wedge x \notin c
\]
が成り立つ.
そこで(2), (3)から, 推論法則 \ref{dedeqtrans}によって
\[
  x \notin b \cup c \leftrightarrow x \notin b \wedge x \notin c
\]
が成り立ち, これから推論法則 \ref{dedaddeqw}によって
\[
\tag{4}
  x \in a \wedge x \notin b \cup c \leftrightarrow x \in a \wedge (x \notin b \wedge x \notin c)
\]
が成り立つ.
またThm \ref{aw1bwc1l1awb1w1awc1}より
\[
\tag{5}
  x \in a \wedge (x \notin b \wedge x \notin c) \leftrightarrow 
  (x \in a \wedge x \notin b) \wedge (x \in a \wedge x \notin c)
\]
が成り立つ.
また定理 \ref{sthm-basis}と推論法則 \ref{dedeqch}により
\[
  x \in a \wedge x \notin b \leftrightarrow x \in a - b, ~~
  x \in a \wedge x \notin c \leftrightarrow x \in a - c
\]
が共に成り立つから, 推論法則 \ref{dedaddeqw}により
\[
\tag{6}
  (x \in a \wedge x \notin b) \wedge (x \in a \wedge x \notin c) \leftrightarrow 
  x \in a - b \wedge x \in a - c
\]
が成り立つ.
また定理 \ref{sthmcapelement}と推論法則 \ref{dedeqch}により
\[
\tag{7}
  x \in a - b \wedge x \in a - c \leftrightarrow x \in (a - b) \cap (a - c)
\]
が成り立つ.
以上の(1), (4), (5), (6), (7)から, 推論法則 \ref{dedeqtrans}によって
\[
\tag{8}
  x \in a - (b \cup c) \leftrightarrow x \in (a - b) \cap (a - c)
\]
が成り立つことがわかる.
いま$x$は定数でなく, 上述のように$a - (b \cup c)$及び$(a - b) \cap (a - c)$の中に
自由変数として現れないから, 
(8)から, 定理 \ref{sthmset=}によって$a - (b \cup c) = (a - b) \cap (a - c)$が成り立つ.

次に後者が成り立つことを示す.
定理 \ref{sthm-basis}より
\[
\tag{9}
  x \in a - (b \cap c) \leftrightarrow x \in a \wedge x \notin b \cap c
\]
が成り立つ.
また定理 \ref{sthmcapelement}より
$x \in b \cap c \leftrightarrow x \in b \wedge x \in c$が成り立つから, 
推論法則 \ref{dedeqcp}により
\[
\tag{10}
  x \notin b \cap c \leftrightarrow \neg (x \in b \wedge x \in c)
\]
が成り立つ.
またThm \ref{n1awb1lnavnb}より
\[
\tag{11}
  \neg (x \in b \wedge x \in c) \leftrightarrow x \notin b \vee x \notin c
\]
が成り立つ.
そこで(10), (11)から, 推論法則 \ref{dedeqtrans}により
\[
  x \notin b \cap c \leftrightarrow x \notin b \vee x \notin c
\]
が成り立ち, これから推論法則 \ref{dedaddeqw}によって
\[
\tag{12}
  x \in a \wedge x \notin b \cap c \leftrightarrow x \in a \wedge (x \notin b \vee x \notin c)
\]
が成り立つ.
またThm \ref{aw1bvc1l1awb1v1awc1}より
\[
\tag{13}
  x \in a \wedge (x \notin b \vee x \notin c) \leftrightarrow 
  (x \in a \wedge x \notin b) \vee (x \in a \wedge x \notin c)
\]
が成り立つ.
また定理 \ref{sthm-basis}と推論法則 \ref{dedeqch}により
\[
  x \in a \wedge x \notin b \leftrightarrow x \in a - b, ~~
  x \in a \wedge x \notin c \leftrightarrow x \in a - c
\]
が共に成り立つから, 推論法則 \ref{dedaddeqv}により
\[
\tag{14}
  (x \in a \wedge x \notin b) \vee (x \in a \wedge x \notin c) \leftrightarrow 
  x \in a - b \vee x \in a - c
\]
が成り立つ.
また定理 \ref{sthmcupbasis}と推論法則 \ref{dedeqch}により
\[
\tag{15}
  x \in a - b \vee x \in a - c \leftrightarrow x \in (a - b) \cup (a - c)
\]
が成り立つ.
以上の(9), (12), (13), (14), (15)から, 推論法則 \ref{dedeqtrans}によって
\[
\tag{16}
  x \in a - (b \cap c) \leftrightarrow x \in (a - b) \cup (a - c)
\]
が成り立つことがわかる.
いま$x$は定数でなく, はじめに述べたように$a - (b \cap c)$及び$(a - b) \cup (a - c)$の中に
自由変数として現れないから, 
(16)から, 定理 \ref{sthmset=}によって$a - (b \cap c) = (a - b) \cup (a - c)$が成り立つ.
\halmos




\mathstrut
\begin{thm}
\label{sthm-dist}%定理
$a$, $b$, $c$を集合とするとき, 
\[
  (a \cup b) - c = (a - c) \cup (b - c), ~~
  (a \cap b) - c = (a - c) \cap (b - c)
\]
が成り立つ.
\end{thm}


\noindent{\bf 証明}
~$x$を, $a$, $b$, $c$の中に自由変数として現れない, 定数でない文字とする.
このとき変数法則 \ref{valcup}, \ref{valcap}, \ref{val-}からわかるように, 
$x$は$(a \cup b) - c$, $(a - c) \cup (b - c)$, $(a \cap b) - c$, $(a - c) \cap (b - c)$の
いずれの記号列の中にも自由変数として現れない.

さてまず前者が成り立つことを示す.
定理 \ref{sthm-basis}より
\[
\tag{1}
  x \in (a \cup b) - c \leftrightarrow x \in a \cup b \wedge x \notin c
\]
が成り立つ.
また定理 \ref{sthmcupbasis}より
$x \in a \cup b \leftrightarrow x \in a \vee x \in b$が成り立つから, 
推論法則 \ref{dedaddeqw}により
\[
\tag{2}
  x \in a \cup b \wedge x \notin c \leftrightarrow (x \in a \vee x \in b) \wedge x \notin c
\]
が成り立つ.
またThm \ref{aw1bvc1l1awb1v1awc1}より
\[
\tag{3}
  (x \in a \vee x \in b) \wedge x \notin c \leftrightarrow 
  (x \in a \wedge x \notin c) \vee (x \in b \wedge x \notin c)
\]
が成り立つ.
また定理 \ref{sthm-basis}と推論法則 \ref{dedeqch}により
\[
  x \in a \wedge x \notin c \leftrightarrow x \in a - c, ~~
  x \in b \wedge x \notin c \leftrightarrow x \in b - c
\]
が共に成り立つから, 推論法則 \ref{dedaddeqv}により
\[
\tag{4}
  (x \in a \wedge x \notin c) \vee (x \in b \wedge x \notin c) \leftrightarrow 
  x \in a - c \vee x \in b - c
\]
が成り立つ.
また定理 \ref{sthmcupbasis}と推論法則 \ref{dedeqch}により
\[
\tag{5}
  x \in a - c \vee x \in b - c \leftrightarrow x \in (a - c) \cup (b - c)
\]
が成り立つ.
以上の(1)---(5)から, 推論法則 \ref{dedeqtrans}によって
\[
\tag{6}
  x \in (a \cup b) - c \leftrightarrow x \in (a - c) \cup (b - c)
\]
が成り立つことがわかる.
いま$x$は定数でなく, はじめに述べたように$(a \cup b) - c$及び$(a - c) \cup (b - c)$の中に
自由変数として現れないから, 
(6)から, 定理 \ref{sthmset=}によって$(a \cup b) - c = (a - c) \cup (b - c)$が成り立つ.

次に後者が成り立つことを示す.
定理 \ref{sthm-basis}より
\[
\tag{7}
  x \in (a \cap b) - c \leftrightarrow x \in a \cap b \wedge x \notin c
\]
が成り立つ.
また定理 \ref{sthmcapelement}より
$x \in a \cap b \leftrightarrow x \in a \wedge x \in b$が成り立つから, 
推論法則 \ref{dedaddeqw}により
\[
\tag{8}
  x \in a \cap b \wedge x \notin c \leftrightarrow 
  (x \in a \wedge x \in b) \wedge x \notin c
\]
が成り立つ.
またThm \ref{aw1bwc1l1awb1w1awc1}より
\[
\tag{9}
  (x \in a \wedge x \in b) \wedge x \notin c \leftrightarrow 
  (x \in a \wedge x \notin c) \wedge (x \in b \wedge x \notin c)
\]
が成り立つ.
また定理 \ref{sthm-basis}と推論法則 \ref{dedeqch}により
\[
  x \in a \wedge x \notin c \leftrightarrow x \in a - c, ~~
  x \in b \wedge x \notin c \leftrightarrow x \in b - c
\]
が共に成り立つから, 推論法則 \ref{dedaddeqw}により
\[
\tag{10}
  (x \in a \wedge x \notin c) \wedge (x \in b \wedge x \notin c) \leftrightarrow 
  x \in a - c \wedge x \in b - c
\]
が成り立つ.
また定理 \ref{sthmcapelement}と推論法則 \ref{dedeqch}により
\[
\tag{11}
  x \in a - c \wedge x \in b - c \leftrightarrow x \in (a - c) \cap (b - c)
\]
が成り立つ.
以上の(7)---(11)から, 推論法則 \ref{dedeqtrans}によって
\[
\tag{12}
  x \in (a \cap b) - c \leftrightarrow x \in (a - c) \cap (b - c)
\]
が成り立つことがわかる.
いま$x$は定数でなく, はじめに述べたように$(a \cap b) - c$及び$(a - c) \cap (b - c)$の中に
自由変数として現れないから, 
(12)から, 定理 \ref{sthmset=}によって$(a \cap b) - c = (a - c) \cap (b - c)$が成り立つ.
\halmos




\mathstrut
\begin{thm}
\label{sthmcup-}%定理
$a$, $b$, $c$を集合とするとき, 
\[
  (a - b) \cup c = (a \cup c) - (b - c), ~~
  a \cup (b - c) = (a \cup b) - (c - a)
\]
が成り立つ.
\end{thm}


\noindent{\bf 証明}
~まず前者が成り立つことを示す.
$x$を$a$, $b$, $c$のいずれの記号列の中にも自由変数として現れない, 
定数でない文字とする.
このとき変数法則 \ref{valcup}, \ref{val-}からわかるように, $x$は
$(a - b) \cup c$及び$(a \cup c) - (b - c)$の中に自由変数として現れない.
そして定理 \ref{sthmcupbasis}より
\[
\tag{1}
  x \in (a - b) \cup c \leftrightarrow x \in a - b \vee x \in c
\]
が成り立つ.
また定理 \ref{sthm-basis}より
$x \in a - b \leftrightarrow x \in a \wedge x \notin b$が成り立つから, 
推論法則 \ref{dedaddeqv}により
\[
\tag{2}
  x \in a - b \vee x \in c \leftrightarrow (x \in a \wedge x \notin b) \vee x \in c
\]
が成り立つ.
またThm \ref{aw1bvc1l1awb1v1awc1}より
\[
\tag{3}
  (x \in a \wedge x \notin b) \vee x \in c \leftrightarrow (x \in a \vee x \in c) \wedge (x \notin b \vee x \in c)
\]
が成り立つ.
また定理 \ref{sthmcupbasis}と推論法則 \ref{dedeqch}により
\[
\tag{4}
  x \in a \vee x \in c \leftrightarrow x \in a \cup c
\]
が成り立つ.
またThm \ref{nnala}と推論法則 \ref{dedeqch}により, 
$x \in c \leftrightarrow \neg \neg x \in c$, 即ち
$x \in c \leftrightarrow \neg x \notin c$が成り立つから, 
推論法則 \ref{dedaddeqv}により
\[
\tag{5}
  x \notin b \vee x \in c \leftrightarrow x \notin b \vee \neg x \notin c
\]
が成り立つ.
またThm \ref{n1awb1lnavnb}と推論法則 \ref{dedeqch}により
\[
\tag{6}
  x \notin b \vee \neg x \notin c \leftrightarrow \neg (x \in b \wedge x \notin c)
\]
が成り立つ.
また定理 \ref{sthm-basis}と推論法則 \ref{dedeqch}により
$x \in b \wedge x \notin c \leftrightarrow x \in b - c$が成り立つから, 
推論法則 \ref{dedeqcp}により
\[
\tag{7}
  \neg (x \in b \wedge x \notin c) \leftrightarrow x \notin b - c
\]
が成り立つ.
そこで(5), (6), (7)から, 推論法則 \ref{dedeqtrans}によって
\[
\tag{8}
  x \notin b \vee x \in c \leftrightarrow x \notin b - c
\]
が成り立つ.
そこで(4), (8)から, 推論法則 \ref{dedaddeqw}によって
\[
\tag{9}
  (x \in a \vee x \in c) \wedge (x \notin b \vee x \in c) \leftrightarrow x \in a \cup c \wedge x \notin b - c
\]
が成り立つ.
また定理 \ref{sthm-basis}と推論法則 \ref{dedeqch}により
\[
\tag{10}
  x \in a \cup c \wedge x \notin b - c \leftrightarrow x \in (a \cup c) - (b - c)
\]
が成り立つ.
そこで(1), (2), (3), (9), (10)から, 推論法則 \ref{dedeqtrans}によって
\[
\tag{11}
  x \in (a - b) \cup c \leftrightarrow x \in (a \cup c) - (b - c)
\]
が成り立つことがわかる.
いま$x$は定数でなく, 上述のように$(a - b) \cup c$及び$(a \cup c) - (b - c)$の中に
自由変数として現れないから, (11)から, 定理 \ref{sthmset=}によって
$(a - b) \cup c = (a \cup c) - (b - c)$が成り立つ.

次に後者が成り立つことを示す.
まず定理 \ref{sthmcupch}より
\[
\tag{12}
  a \cup (b - c) = (b - c) \cup a
\]
が成り立つ.
またいま上に示したことから, 
\[
\tag{13}
  (b - c) \cup a = (b \cup a) - (c - a)
\]
が成り立つ.
また定理 \ref{sthmcupch}より
$b \cup a = a \cup b$が成り立つから, 定理 \ref{sthm-=}により
\[
\tag{14}
  (b \cup a) - (c - a) = (a \cup b) - (c - a)
\]
が成り立つ.
そこで(12), (13), (14)から, 推論法則 \ref{ded=trans}によって
$a \cup (b - c) = (a \cup b) - (c - a)$が成り立つことがわかる.
\halmos




\mathstrut
\begin{thm}
\label{sthmcap-}%定理
$a$, $b$, $c$を集合とするとき, 
\begin{align*}
  (a - b) \cap c &= (a \cap c) - b, ~~(a - b) \cap c = (a \cap c) -(b \cap c), \\
  \mbox{} \\
  a \cap (b - c) &= (a \cap b) - c, ~~a \cap (b - c) = (a \cap b) - (a \cap c)
\end{align*}
が成り立つ.
\end{thm}


\noindent{\bf 証明}
~まず第一のものが定理となることを示す.
$x$を$a$, $b$, $c$のいずれの記号列の中にも自由変数として現れない, 定数でない文字とする.
このとき変数法則 \ref{valcap}, \ref{val-}からわかるように, $x$は
$(a - b) \cap c$及び$(a \cap c) - b$の中に自由変数として現れない.
また定理 \ref{sthmcapelement}より
\[
\tag{1}
  x \in (a - b) \cap c \leftrightarrow x \in a - b \wedge x \in c
\]
が成り立つ.
また定理 \ref{sthm-basis}より
$x \in a - b \leftrightarrow x \in a \wedge x \notin b$が成り立つから, 
推論法則 \ref{dedaddeqw}により
\[
\tag{2}
  x \in a - b \wedge x \in c \leftrightarrow (x \in a \wedge x \notin b) \wedge x \in c
\]
が成り立つ.
またThm \ref{1awb1wclaw1bwc1}より
\[
\tag{3}
  (x \in a \wedge x \notin b) \wedge x \in c \leftrightarrow x \in a \wedge (x \notin b \wedge x \in c)
\]
が成り立つ.
またThm \ref{awblbwa}より
$x \notin b \wedge x \in c \leftrightarrow x \in c \wedge x \notin b$が成り立つから, 
推論法則 \ref{dedaddeqw}により
\[
\tag{4}
  x \in a \wedge (x \notin b \wedge x \in c) \leftrightarrow x \in a \wedge (x \in c \wedge x \notin b)
\]
が成り立つ.
またThm \ref{1awb1wclaw1bwc1}と推論法則 \ref{dedeqch}により
\[
\tag{5}
  x \in a \wedge (x \in c \wedge x \notin b) \leftrightarrow (x \in a \wedge x \in c) \wedge x \notin b
\]
が成り立つ.
また定理 \ref{sthmcapelement}と推論法則 \ref{dedeqch}により
\[
\tag{6}
  x \in a \wedge x \in c \leftrightarrow x \in a \cap c
\]
が成り立つから, 推論法則 \ref{dedaddeqw}により
\[
\tag{7}
  (x \in a \wedge x \in c) \wedge x \notin b \leftrightarrow x \in a \cap c \wedge x \notin b
\]
が成り立つ.
また定理 \ref{sthm-basis}と推論法則 \ref{dedeqch}により
\[
\tag{8}
  x \in a \cap c \wedge x \notin b \leftrightarrow x \in (a \cap c) - b
\]
が成り立つ.
そこで(1)---(5), (7), (8)から, 推論法則 \ref{dedeqtrans}によって
\[
\tag{9}
  x \in (a - b) \cap c \leftrightarrow x \in (a \cap c) - b
\]
が成り立つことがわかる.
いま$x$は定数でなく, 上述のように$(a - b) \cap c$及び$(a \cap c) - b$の中に自由変数として現れないから, 
(9)から, 定理 \ref{sthmset=}によって
\[
\tag{10}
  (a - b) \cap c = (a \cap c) - b
\]
が成り立つ.

次に第二のものが定理となることを示す.
定理 \ref{sthm=from-abs}と推論法則 \ref{dedequiv}により, 
\[
\tag{11}
  (a \cap c) \cap b = (a \cap c) \cap (b \cap c) \to (a \cap c) - b = (a \cap c) - (b \cap c)
\]
が成り立つ.
また定理 \ref{sthmcapcomb}より
\[
\tag{12}
  (a \cap c) \cap b = a \cap (c \cap b)
\]
が成り立つ.
また定理 \ref{sthmcapch}より$c \cap b = b \cap c$が成り立つから, 
定理 \ref{sthmcap=}により, 
\[
\tag{13}
  a \cap (c \cap b) = a \cap (b \cap c)
\]
が成り立つ.
また定理 \ref{sthmcapcomb}より$(a \cap b) \cap c = a \cap (b \cap c)$が成り立つから, 
推論法則 \ref{ded=ch}により, 
\[
\tag{14}
  a \cap (b \cap c) = (a \cap b) \cap c
\]
が成り立つ.
また定理 \ref{sthmcapdist}より
\[
\tag{15}
  (a \cap b) \cap c = (a \cap c) \cap (b \cap c)
\]
が成り立つ.
そこで(12)---(15)から, 推論法則 \ref{ded=trans}によって
$(a \cap c) \cap b = (a \cap c) \cap (b \cap c)$が成り立ち, 
これと(11)から, 推論法則 \ref{dedmp}によって
\[
\tag{16}
  (a \cap c) - b = (a \cap c) - (b \cap c)
\]
が成り立つ.
そこで(10), (16)から, 推論法則 \ref{ded=trans}によって
$(a - b) \cap c = (a \cap c) - (b \cap c)$が成り立つ.

最後に第三及び第四のものが定理となることを示す.
まず定理 \ref{sthmcapch}より
\[
\tag{17}
  a \cap (b - c) = (b - c) \cap a
\]
が成り立つ.
またいま示したことから, 
\begin{align*}
  \tag{18}
  (b - c) \cap a &= (b \cap a) - c, \\
  \mbox{} \\
  \tag{19}
  (b - c) \cap a &= (b \cap a) - (c \cap a)
\end{align*}
が共に成り立つ.
また定理 \ref{sthmcapch}より
\[
  b \cap a = a \cap b, ~~
  c \cap a = a \cap c
\]
が共に成り立つから, 定理 \ref{sthm-=}により
\begin{align*}
  \tag{20}
  (b \cap a) - c &= (a \cap b) - c, \\
  \mbox{} \\
  \tag{21}
  (b \cap a) - (c \cap a) &= (a \cap b) - (a \cap c)
\end{align*}
が共に成り立つ.
そこで(17), (18), (20)から, 推論法則 \ref{ded=trans}によって
$a \cap (b - c) = (a \cap b) - c$が成り立ち, 
(17), (19), (21)から, 同じく推論法則 \ref{ded=trans}によって
$a \cap (b - c) = (a \cap b) - (a \cap c)$が成り立つことがわかる.
\halmos




\mathstrut
\begin{thm}
\label{sthm--}%定理
$a$, $b$, $c$を集合とするとき, 
\[
  (a - b) - c = a - (b \cup c), ~~
  a - (b - c) = (a - b) \cup (a \cap c)
\]
が成り立つ.
\end{thm}


\noindent{\bf 証明}
~$x$を$a$, $b$, $c$のいずれの記号列の中にも自由変数として現れない, 定数でない文字とする.
このとき変数法則 \ref{valcup}, \ref{valcap}, \ref{val-}からわかるように, 
$x$は$(a - b) - c$, $a - (b \cup c)$, $a - (b - c)$, $(a - b) \cup (a \cap c)$の
いずれの記号列の中にも自由変数として現れない.

さてまず前者が成り立つことを示す.
定理 \ref{sthm-basis}より
\[
\tag{1}
  x \in (a - b) - c \leftrightarrow x \in a - b \wedge x \notin c
\]
が成り立つ.
同じく定理 \ref{sthm-basis}より
$x \in a - b \leftrightarrow x \in a \wedge x \notin b$が成り立つから, 
推論法則 \ref{dedaddeqw}により
\[
\tag{2}
  x \in a - b \wedge x \notin c \leftrightarrow (x \in a \wedge x \notin b) \wedge x \notin c
\]
が成り立つ.
またThm \ref{1awb1wclaw1bwc1}より
\[
\tag{3}
  (x \in a \wedge x \notin b) \wedge x \notin c \leftrightarrow x \in a \wedge (x \notin b \wedge x \notin c)
\]
が成り立つ.
またThm \ref{n1awb1lnavnb}と推論法則 \ref{dedeqch}により
\[
\tag{4}
  x \notin b \wedge x \notin c \leftrightarrow \neg (x \in b \vee x \in c)
\]
が成り立つ.
また定理 \ref{sthmcupbasis}と推論法則 \ref{dedeqch}により
$x \in b \vee x \in c \leftrightarrow x \in b \cup c$が成り立つから, 
推論法則 \ref{dedeqcp}により
\[
\tag{5}
  \neg (x \in b \vee x \in c) \leftrightarrow x \notin b \cup c
\]
が成り立つ.
そこで(4), (5)から, 推論法則 \ref{dedeqtrans}によって
$x \notin b \wedge x \notin c \leftrightarrow x \notin b \cup c$が成り立ち, 
これから推論法則 \ref{dedaddeqw}によって
\[
\tag{6}
  x \in a \wedge (x \notin b \wedge x \notin c) \leftrightarrow x \in a \wedge x \notin b \cup c
\]
が成り立つ.
また定理 \ref{sthm-basis}と推論法則 \ref{dedeqch}により
\[
\tag{7}
  x \in a \wedge x \notin b \cup c \leftrightarrow x \in a - (b \cup c)
\]
が成り立つ.
以上の(1), (2), (3), (6), (7)から, 推論法則 \ref{dedeqtrans}によって
\[
\tag{8}
  x \in (a - b) - c \leftrightarrow x \in a - (b \cup c)
\]
が成り立つことがわかる.
いま$x$は定数でなく, はじめに述べたように$(a - b) - c$及び$a - (b \cup c)$の中に
自由変数として現れないから, 
(8)から, 定理 \ref{sthmset=}によって$(a - b) - c = a - (b \cup c)$が成り立つ.

次に後者が成り立つことを示す.
定理 \ref{sthm-basis}より
\[
\tag{9}
  x \in a - (b - c) \leftrightarrow x \in a \wedge x \notin b - c
\]
が成り立つ.
同じく定理 \ref{sthm-basis}より
$x \in b - c \leftrightarrow x \in b \wedge x \notin c$が成り立つから, 
推論法則 \ref{dedeqcp}により
\[
\tag{10}
  x \notin b - c \leftrightarrow \neg (x \in b \wedge x \notin c)
\]
が成り立つ.
またThm \ref{n1awb1lnavnb}より
\[
\tag{11}
  \neg (x \in b \wedge x \notin c) \leftrightarrow x \notin b \vee \neg \neg x \in c
\]
が成り立つ.
またThm \ref{nnala}より
$\neg \neg x \in c \leftrightarrow x \in c$が成り立つから, 
推論法則 \ref{dedaddeqv}により
\[
\tag{12}
  x \notin b \vee \neg \neg x \in c \leftrightarrow x \notin b \vee x \in c
\]
が成り立つ.
そこで(10), (11), (12)から, 推論法則 \ref{dedeqtrans}によって
$x \notin b - c \leftrightarrow x \notin b \vee x \in c$が成り立ち, 
これから推論法則 \ref{dedaddeqw}によって
\[
\tag{13}
  x \in a \wedge x \notin b - c \leftrightarrow x \in a \wedge (x \notin b \vee x \in c)
\]
が成り立つ.
またThm \ref{aw1bvc1l1awb1v1awc1}より
\[
\tag{14}
  x \in a \wedge (x \notin b \vee x \in c) \leftrightarrow 
  (x \in a \wedge x \notin b) \vee (x \in a \wedge x \in c)
\]
が成り立つ.
また定理 \ref{sthm-basis}と推論法則 \ref{dedeqch}により
$x \in a \wedge x \notin b \leftrightarrow x \in a - b$が成り立ち, 
定理 \ref{sthmcapelement}と推論法則 \ref{dedeqch}により
$x \in a \wedge x \in c \leftrightarrow x \in a \cap c$が成り立つから, 
これらから, 推論法則 \ref{dedaddeqv}によって
\[
\tag{15}
  (x \in a \wedge x \notin b) \vee (x \in a \wedge x \in c) \leftrightarrow 
  x \in a - b \vee x \in a \cap c
\]
が成り立つ.
また定理 \ref{sthmcupbasis}と推論法則 \ref{dedeqch}により
\[
\tag{16}
  x \in a - b \vee x \in a \cap c \leftrightarrow x \in (a - b) \cup (a \cap c)
\]
が成り立つ.
以上の(9), (13), (14), (15), (16)から, 推論法則 \ref{dedeqtrans}によって
\[
\tag{17}
  x \in a - (b - c) \leftrightarrow x \in (a - b) \cup (a \cap c)
\]
が成り立つことがわかる.
いま$x$は定数でなく, はじめに述べたように$a - (b - c)$及び$(a - b) \cup (a \cap c)$の中に
自由変数として現れないから, 
(17)から, 定理 \ref{sthmset=}によって$a - (b - c) = (a - b) \cup (a \cap c)$が成り立つ.
\halmos


%kokoko

\mathstrut
\begin{thm}
\label{sthmwnsetmake}%定理
$R$と$S$を関係式, $x$を文字とし, $R$は$x$について集合を作り得るとする.
このとき$R \wedge \neg S$は$x$について集合を作り得る.
さらに$S$も$x$について集合を作り得るならば, 
\[
  \{x|R \wedge \neg S\} = \{x|R\} - \{x|S\}
\]
が成り立つ.
\end{thm}


\noindent{\bf 証明}
~主張の前半部分は定理 \ref{sthmwsetmake}によって明らかに成り立つ.

以下主張の後半部分を示す.
いま$R$と$S$が共に$x$について集合を作り得るとする.
このとき, いま述べたように$R \wedge \neg S$も$x$について集合を作り得る.
いま$y$を$R$及び$S$の中に自由変数として現れない, 定数でない文字とすれば, 
変数法則 \ref{valfund}, \ref{valwedge}, \ref{valiset}, \ref{val-}からわかるように, 
$y$は$\{x|R \wedge \neg S\}$及び$\{x|R\} - \{x|S\}$の中に自由変数として現れない.
そして定理 \ref{sthmisetbasis}より
\[
  y \in \{x|R \wedge \neg S\} \leftrightarrow (y|x)(R \wedge \neg S), \\
\]
が成り立つが, 代入法則 \ref{substfund}, \ref{substwedge}により, この記号列は
\[
\tag{1}
  y \in \{x|R \wedge \neg S\} \leftrightarrow (y|x)(R) \wedge \neg (y|x)(S)
\]
と一致するから, これが定理となる.
また定理 \ref{sthmisetbasis}と推論法則 \ref{dedeqch}により, 
\begin{align*}
  \tag{2}
  (y|x)(R) &\leftrightarrow y \in \{x|R\}, \\
  \mbox{} \\
  \tag{3}
  (y|x)(S) &\leftrightarrow y \in \{x|S\}
\end{align*}
が共に成り立つ.
この(3)から, 推論法則 \ref{dedeqcp}によって
\[
\tag{4}
  \neg (y|x)(S) \leftrightarrow y \notin \{x|S\}
\]
が成り立つ.
そこで(2), (4)から, 推論法則 \ref{dedaddeqw}によって
\[
\tag{5}
  (y|x)(R) \wedge \neg (y|x)(S) \leftrightarrow y \in \{x|R\} \wedge y \notin \{x|S\}
\]
が成り立つ.
また定理 \ref{sthm-basis}と推論法則 \ref{dedeqch}により, 
\[
\tag{6}
  y \in \{x|R\} \wedge y \notin \{x|S\} \leftrightarrow y \in \{x|R\} - \{x|S\}
\]
が成り立つ.
そこで(1), (5), (6)から, 推論法則 \ref{dedeqtrans}によって
\[
\tag{7}
  y \in \{x|R \wedge \neg S\} \leftrightarrow y \in \{x|R\} - \{x|S\}
\]
が成り立つ.
いま$y$は定数でなく, 上述のように$\{x|R \wedge \neg S\}$及び$\{x|R\} - \{x|S\}$の中に
自由変数として現れないから, (7)から, 定理 \ref{sthmset=}によって
$\{x|R \wedge \neg S\} = \{x|R\} - \{x|S\}$が成り立つ.
\halmos









\mathstrut
\begin{thm}
\label{sthmrwnssetsep}%定理
$a$を集合, $R$と$S$を関係式とし, $x$を$a$の中に自由変数として現れない文字とする.
このとき
\[
  \{x \in a|R \wedge \neg S\} = \{x \in a|R\} - \{x \in a|S\}
\]
が成り立つ.
\end{thm}


\noindent{\bf 証明}
~$y$を$a$, $R$, $S$のいずれの記号列の中にも自由変数として現れない, 定数でない文字とする.
このとき変数法則 \ref{valfund}, \ref{valwedge}, \ref{valsset}, \ref{val-}からわかるように, 
$y$は$\{x \in a|R \wedge \neg S\}$及び$\{x \in a|R\} - \{x \in a|S\}$の中に自由変数として現れない.
また$x$が$a$の中に自由変数として現れないことから, 定理 \ref{sthmssetbasis}より
\[
  y \in \{x \in a|R \wedge \neg S\} \leftrightarrow y \in a \wedge (y|x)(R \wedge \neg S)
\]
が成り立つが, 代入法則 \ref{substfund}, \ref{substwedge}により, この記号列は
\[
\tag{1}
  y \in \{x \in a|R \wedge \neg S\} \leftrightarrow y \in a \wedge ((y|x)(R) \wedge \neg (y|x)(S))
\]
と一致するから, これが定理となる.
またThm \ref{1awb1wclaw1bwc1}と推論法則 \ref{dedeqch}により
\begin{align*}
  \tag{2}
  y \in a \wedge ((y|x)(R) \wedge \neg (y|x)(S)) &\leftrightarrow (y \in a \wedge (y|x)(R)) \wedge \neg (y|x)(S), \\
  \mbox{} \\
  \tag{3}
  (y|x)(R) \wedge (y \in a \wedge y \notin a) &\leftrightarrow ((y|x)(R) \wedge y \in a) \wedge y \notin a
\end{align*}
が共に成り立つ.
またThm \ref{awblbwa}より
$(y|x)(R) \wedge y \in a \leftrightarrow y \in a \wedge (y|x)(R)$が成り立つから, 
推論法則 \ref{dedaddeqw}により
\[
\tag{4}
  ((y|x)(R) \wedge y \in a) \wedge y \notin a \leftrightarrow (y \in a \wedge (y|x)(R)) \wedge y \notin a
\]
が成り立つ.
そこで(3), (4)から, 推論法則 \ref{dedeqtrans}によって
\[
\tag{5}
  (y|x)(R) \wedge (y \in a \wedge y \notin a) \leftrightarrow (y \in a \wedge (y|x)(R)) \wedge y \notin a
\]
が成り立つ.
またThm \ref{n1awna1}より
$\neg (y \in a \wedge y \notin a)$が成り立つから, 
推論法則 \ref{dednw}により
\[
\tag{6}
  \neg ((y|x)(R) \wedge (y \in a \wedge y \notin a))
\]
が成り立つ.
そこで(5), (6)から, 推論法則 \ref{dedeqfund}によって
\[
  \neg ((y \in a \wedge (y|x)(R)) \wedge y \notin a)
\]
が成り立つ.
そこで推論法則 \ref{dedavblbtrue2}によって
\[
  ((y \in a \wedge (y|x)(R)) \wedge y \notin a) \vee ((y \in a \wedge (y|x)(R)) \wedge \neg (y|x)(S)) 
  \leftrightarrow (y \in a \wedge (y|x)(R)) \wedge \neg (y|x)(S)
\]
が成り立ち, これから推論法則 \ref{dedeqch}によって
\[
\tag{7}
  (y \in a \wedge (y|x)(R)) \wedge \neg (y|x)(S) \leftrightarrow 
  ((y \in a \wedge (y|x)(R)) \wedge y \notin a) \vee ((y \in a \wedge (y|x)(R)) \wedge \neg (y|x)(S))
\]
が成り立つ.
またThm \ref{aw1bvc1l1awb1v1awc1}と推論法則 \ref{dedeqch}により
\begin{multline*}
\tag{8}
  ((y \in a \wedge (y|x)(R)) \wedge y \notin a) \vee ((y \in a \wedge (y|x)(R)) \wedge \neg (y|x)(S)) \\
  \leftrightarrow (y \in a \wedge (y|x)(R)) \wedge (y \notin a \vee \neg (y|x)(S))
\end{multline*}
が成り立つ.
また$x$が$a$の中に自由変数として現れないことから, 定理 \ref{sthmssetbasis}と推論法則 \ref{dedeqch}により
\begin{align*}
  \tag{9}
  y \in a \wedge (y|x)(R) &\leftrightarrow y \in \{x \in a|R\}, \\
  \mbox{} \\
  \tag{10}
  y \in a \wedge (y|x)(S) &\leftrightarrow y \in \{x \in a|S\}
\end{align*}
が共に成り立つ.
そこでこの(10)から, 推論法則 \ref{dedeqcp}によって
\[
\tag{11}
  \neg (y \in a \wedge (y|x)(S)) \leftrightarrow y \notin \{x \in a|S\}
\]
が成り立つ.
また Thm \ref{n1awb1lnavnb}と推論法則 \ref{dedeqch}により
\[
\tag{12}
  y \notin a \vee \neg (y|x)(S) \leftrightarrow \neg (y \in a \wedge (y|x)(S))
\]
が成り立つ.
そこで(11), (12)から, 推論法則 \ref{dedeqtrans}によって
\[
\tag{13}
  y \notin a \vee \neg (y|x)(S) \leftrightarrow y \notin \{x \in a|S\}
\]
が成り立つ.
そこで(9), (13)から, 推論法則 \ref{dedaddeqw}によって
\[
\tag{14}
  (y \in a \wedge (y|x)(R)) \wedge (y \notin a \vee \neg (y|x)(S)) \leftrightarrow 
  y \in \{x \in a|R\} \wedge y \notin \{x \in a|S\}
\]
が成り立つ.
また定理 \ref{sthm-basis}と推論法則 \ref{dedeqch}により
\[
\tag{15}
  y \in \{x \in a|R\} \wedge y \notin \{x \in a|S\} \leftrightarrow y \in \{x \in a|R\} - \{x \in a|S\}
\]
が成り立つ.
そこで(1), (2), (7), (8), (14), (15)から, 推論法則 \ref{dedeqtrans}によって
\[
\tag{16}
  y \in \{x \in a|R \wedge \neg S\} \leftrightarrow y \in \{x \in a|R\} - \{x \in a|S\}
\]
が成り立つことがわかる.
いま$y$は定数でなく, はじめに述べたように$\{x \in a|R \wedge \neg S\}$及び
$\{x \in a|R\} - \{x \in a|S\}$の中に自由変数として現れないから, 
(16)から, 定理 \ref{sthmset=}によって
$\{x \in a|R \wedge \neg S\} = \{x \in a|R\} - \{x \in a|S\}$が成り立つ.
\halmos




\mathstrut
\begin{thm}
\label{sthma-bsetsep}%定理
$a$と$b$を集合, $R$を関係式とし, $x$を$a$及び$b$の中に自由変数として現れない文字とする.
このとき
\[
  \{x \in a - b|R\} = \{x \in a|R\} - b, ~~
  \{x \in a - b|R\} = \{x \in a|R\} - \{x \in b|R\}
\]
が成り立つ.
\end{thm}


\noindent{\bf 証明}
~$y$を$a$, $b$, $R$のいずれの記号列の中にも自由変数として現れない, 定数でない文字とする.
このとき変数法則 \ref{valsset}, \ref{val-}からわかるように, $y$は
$\{x \in a - b|R\}$, $\{x \in a|R\} - b$, $\{x \in a|R\} - \{x \in b|R\}$の
いずれの記号列の中にも自由変数として現れない.
また$x$は$a$及び$b$の中に自由変数として現れないから, 変数法則 \ref{val-}により, 
$x$は$a - b$の中にも自由変数として現れない.
そこで定理 \ref{sthmssetbasis}より
\[
\tag{1}
  y \in \{x \in a - b|R\} \leftrightarrow y \in a - b \wedge (y|x)(R)
\]
が成り立つ.
また定理 \ref{sthm-basis}より
$y \in a - b \leftrightarrow y \in a \wedge y \notin b$が成り立つから, 
推論法則 \ref{dedaddeqw}により
\[
\tag{2}
  y \in a - b \wedge (y|x)(R) \leftrightarrow (y \in a \wedge y \notin b) \wedge (y|x)(R)
\]
が成り立つ.
またThm \ref{1awb1wclaw1bwc1}より
\[
\tag{3}
  (y \in a \wedge y \notin b) \wedge (y|x)(R) \leftrightarrow y \in a \wedge (y \notin b \wedge (y|x)(R))
\]
が成り立つ.
またThm \ref{awblbwa}より
$y \notin b \wedge (y|x)(R) \leftrightarrow (y|x)(R) \wedge y \notin b$が成り立つから, 
推論法則 \ref{dedaddeqw}により
\[
\tag{4}
  y \in a \wedge (y \notin b \wedge (y|x)(R)) \leftrightarrow y \in a \wedge ((y|x)(R) \wedge y \notin b)
\]
が成り立つ.
またThm \ref{1awb1wclaw1bwc1}と推論法則 \ref{dedeqch}により
\begin{align*}
  \tag{5}
  y \in a \wedge ((y|x)(R) \wedge y \notin b) &\leftrightarrow (y \in a \wedge (y|x)(R)) \wedge y \notin b, \\
  \mbox{} \\
  \tag{6}
  y \in a \wedge ((y|x)(R) \wedge \neg (y|x)(R)) &\leftrightarrow (y \in a \wedge (y|x)(R)) \wedge \neg (y|x)(R)
\end{align*}
が共に成り立つ.
またThm \ref{n1awna1}より
$\neg ((y|x)(R) \wedge \neg (y|x)(R))$が成り立つから, 
推論法則 \ref{dednw}により
\[
\tag{7}
  \neg (y \in a \wedge ((y|x)(R) \wedge \neg (y|x)(R)))
\]
が成り立つ.
そこで(6), (7)から, 推論法則 \ref{dedeqfund}によって
\[
  \neg ((y \in a \wedge (y|x)(R)) \wedge \neg (y|x)(R))
\]
が成り立つ.
そこで推論法則 \ref{dedavblbtrue2}によって
\[
  ((y \in a \wedge (y|x)(R)) \wedge y \notin b) \vee ((y \in a \wedge (y|x)(R)) \wedge \neg (y|x)(R)) 
  \leftrightarrow (y \in a \wedge (y|x)(R)) \wedge y \notin b
\]
が成り立ち, これから推論法則 \ref{dedeqch}によって
\[
\tag{8}
  (y \in a \wedge (y|x)(R)) \wedge y \notin b \leftrightarrow 
  ((y \in a \wedge (y|x)(R)) \wedge y \notin b) \vee ((y \in a \wedge (y|x)(R)) \wedge \neg (y|x)(R))
\]
が成り立つ.
またThm \ref{aw1bvc1l1awb1v1awc1}と推論法則 \ref{dedeqch}により
\begin{multline*}
\tag{9}
  ((y \in a \wedge (y|x)(R)) \wedge y \notin b) \vee ((y \in a \wedge (y|x)(R)) \wedge \neg (y|x)(R)) \\
  \leftrightarrow (y \in a \wedge (y|x)(R)) \wedge (y \notin b \vee \neg (y|x)(R))
\end{multline*}
が成り立つ.
また$x$が$a$及び$b$の中に自由変数として現れないことから, 
定理 \ref{sthmssetbasis}と推論法則 \ref{dedeqch}により
\begin{align*}
  \tag{10}
  y \in a \wedge (y|x)(R) &\leftrightarrow y \in \{x \in a|R\}, \\
  \mbox{} \\
  \tag{11}
  y \in b \wedge (y|x)(R) &\leftrightarrow y \in \{x \in b|R\}
\end{align*}
が共に成り立つ.
そこでこの(10)から, 推論法則 \ref{dedaddeqw}によって
\[
\tag{12}
  (y \in a \wedge (y|x)(R)) \wedge y \notin b \leftrightarrow y \in \{x \in a|R\} \wedge y \notin b
\]
が成り立つ.
また(11)から, 推論法則 \ref{dedeqcp}によって
\[
\tag{13}
  \neg (y \in b \wedge (y|x)(R)) \leftrightarrow y \notin \{x \in b|R\}
\]
が成り立つ.
またThm \ref{n1awb1lnavnb}と推論法則 \ref{dedeqch}により
\[
\tag{14}
  y \notin b \vee \neg (y|x)(R) \leftrightarrow \neg (y \in b \wedge (y|x)(R))
\]
が成り立つ.
そこで(13), (14)から, 推論法則 \ref{dedeqtrans}によって
\[
\tag{15}
  y \notin b \vee \neg (y|x)(R) \leftrightarrow y \notin \{x \in b|R\}
\]
が成り立つ.
そこで(10), (15)から, 推論法則 \ref{dedaddeqw}によって
\[
\tag{16}
  (y \in a \wedge (y|x)(R)) \wedge (y \notin b \vee \neg (y|x)(R)) \leftrightarrow 
  y \in \{x \in a|R\} \wedge y \notin \{x \in b|R\}
\]
が成り立つ.
また定理 \ref{sthm-basis}と推論法則 \ref{dedeqch}により
\begin{align*}
  \tag{17}
  y \in \{x \in a|R\} \wedge y \notin b &\leftrightarrow y \in \{x \in a|R\} - b, \\
  \mbox{} \\
  \tag{18}
  y \in \{x \in a|R\} \wedge y \notin \{x \in b|R\} &\leftrightarrow y \in \{x \in a|R\} - \{x \in b|R\}
\end{align*}
が共に成り立つ.
そこで(1)---(5), (12), (17)から, 推論法則 \ref{dedeqtrans}によって
\[
\tag{19}
  y \in \{x \in a - b|R\} \leftrightarrow y \in \{x \in a|R\} - b
\]
が成り立ち, (1)---(5), (8), (9), (16), (18)から, 同じく推論法則 \ref{dedeqtrans}によって
\[
\tag{20}
  y \in \{x \in a - b|R\} \leftrightarrow y \in \{x \in a|R\} - \{x \in b|R\}
\]
が成り立つことがわかる.
いま$y$は定数でなく, はじめに述べたように$\{x \in a - b|R\}$, 
$\{x \in a|R\} - b$, $\{x \in a|R\} - \{x \in b|R\}$の
いずれの記号列の中にも自由変数として現れないから, 
(19), (20)から, 定理 \ref{sthmset=}によって
$\{x \in a - b|R\} = \{x \in a|R\} - b$及び
$\{x \in a - b|R\} = \{x \in a|R\} - \{x \in b|R\}$が成り立つ.
\halmos





%[3]確認済



\mathstrut
\noindent
[\textbf{4}] \textbf{空集合}



















\mathstrut
\begin{thm}
\label{sthmsetsepempty}%定理
$a$を集合, $R$を関係式とし, $x$を$a$の中に自由変数として現れない文字とする.
このとき
\[
  \forall x(x \in a \to \neg R) \leftrightarrow \{x \in a|R\} = \phi
\]
が成り立つ.
特に, 
\[
  \forall x(\neg R) \to \{x \in a|R\} = \phi
\]
が成り立つ.
\end{thm}


\noindent{\bf 証明}
~まず前者が成り立つことを示す.
Thm \ref{thmspallfund}と推論法則 \ref{dedeqch}により, 
\[
\tag{1}
  \forall x(x \in a \to \neg R) \leftrightarrow \forall_{x \in a}x(\neg R)
\]
が成り立つ.
またThm \ref{thmeaspquandm}と推論法則 \ref{dedeqch}により, 
\[
  \forall_{x \in a}x(\neg R) \leftrightarrow \neg \exists_{x \in a}x(R)
\]
が成り立つが, 定義からこの記号列は
\[
\tag{2}
  \forall_{x \in a}x(\neg R) \leftrightarrow \neg \exists x(x \in a \wedge R)
\]
であるから, これが定理となる.
またThm \ref{thmeaquandm}より
\[
\tag{3}
  \neg \exists x(x \in a \wedge R) \leftrightarrow \forall x(\neg (x \in a \wedge R))
\]
が成り立つ.
またいま$\tau_{x}(\neg (\neg (x \in a \wedge R) \leftrightarrow x \notin \{x \in a|R\}))$を
$T$と書けば, $T$は集合であり, 
$x$が$a$の中に自由変数として現れないことから, 
定理 \ref{sthmssetbasis}と推論法則 \ref{dedeqch}により
\[
  T \in a \wedge (T|x)(R) \leftrightarrow T \in \{x \in a|R\}
\]
が成り立つ.
そこで推論法則 \ref{dedeqcp}により
\[
  \neg (T \in a \wedge (T|x)(R)) \leftrightarrow T \notin \{x \in a|R\}
\]
が成り立つが, $x$は$a$の中に自由変数として現れず, 変数法則 \ref{valsset}より
$x$は$\{x \in a|R\}$の中にも自由変数として現れないから, 
代入法則 \ref{substfree}, \ref{substfund}, \ref{substwedge}, \ref{substequiv}により, 
この記号列は
\[
  (T|x)(\neg (x \in a \wedge R) \leftrightarrow x \notin \{x \in a|R\})
\]
と一致する.
よってこれが定理となる.
そこで$T$の定義から, 推論法則 \ref{dedallfund}によって
\[
  \forall x(\neg (x \in a \wedge R) \leftrightarrow x \notin \{x \in a|R\})
\]
が成り立ち, これから推論法則 \ref{dedalleqquansep}によって
\[
\tag{4}
  \forall x(\neg (x \in a \wedge R)) \leftrightarrow \forall x(x \notin \{x \in a|R\})
\]
が成り立つ.
また上述のように$x$は$\{x \in a|R\}$の中に自由変数として現れないから, 
定理 \ref{sthmnotinempty}と推論法則 \ref{dedeqch}により, 
\[
\tag{5}
  \forall x(x \notin \{x \in a|R\}) \leftrightarrow \{x \in a|R\} = \phi
\]
が成り立つ.
そこで(1)---(5)から, 推論法則 \ref{dedeqtrans}によって
$\forall x(x \in a \to \neg R) \leftrightarrow \{x \in a|R\} = \phi$が
成り立つことがわかる.

次に後者が成り立つことを示す.
$\tau_{x}(\neg (x \in a \to \neg R))$を$U$と書けば, $U$は集合であり, 
schema S1の適用により
\[
  \neg (U|x)(R) \to (U \in a \to \neg (U|x)(R))
\]
が成り立つ.
ここで$x$が$a$の中に自由変数として現れないことから, 
代入法則 \ref{substfree}, \ref{substfund}によりこの記号列は
\[
  (U|x)(\neg R \to (x \in a \to \neg R))
\]
と一致する.
よってこれが定理となる.
そこで$U$の定義から, 推論法則 \ref{dedtquanfund}により
\[
\tag{6}
  \forall x(\neg R) \to \forall x(x \in a \to \neg R)
\]
が成り立つ.
また上に示したように
$\forall x(x \in a \to \neg R) \leftrightarrow \{x \in a|R\} = \phi$が成り立つから, 
推論法則 \ref{dedequiv}により
\[
\tag{7}
  \forall x(x \in a \to \neg R) \to \{x \in a|R\} = \phi
\]
が成り立つ.
そこで(6), (7)から, 推論法則 \ref{dedmmp}によって
$\forall x(\neg R) \to \{x \in a|R\} = \phi$が成り立つ.
\halmos




\mathstrut
\begin{thm}
\label{sthmsetsepa=empty}%定理
$a$を集合, $R$を関係式とし, $x$を$a$の中に自由変数として現れない文字とする.
このとき
\[
  a = \phi \to \{x \in a|R\} = \phi
\]
が成り立つ.
またこのことから, 次の($*$)が成り立つ: 

($*$) ~~$a$が空ならば, $\{x \in a|R\}$は空である.
        特に, $\{x \in \phi|R\}$は空である.
\end{thm}


\noindent{\bf 証明}
~$x$が$a$の中に自由変数として現れないことから, 
定理 \ref{sthmssetsubseta}より
\[
\tag{1}
  \{x \in a|R\} \subset a
\]
が成り立つ.
また定理 \ref{sthm=tsubseteq}より
\[
  a = \phi \to (\{x \in a|R\} \subset a \leftrightarrow \{x \in a|R\} \subset \phi)
\]
が成り立つから, これから推論法則 \ref{dedprewedge}によって
\[
  a = \phi \to (\{x \in a|R\} \subset a \to \{x \in a|R\} \subset \phi)
\]
が成り立ち, これから推論法則 \ref{dedch}によって
\[
\tag{2}
  \{x \in a|R\} \subset a \to (a = \phi \to \{x \in a|R\} \subset \phi)
\]
が成り立つ.
そこで(1), (2)から, 推論法則 \ref{dedmp}によって
\[
\tag{3}
  a = \phi \to \{x \in a|R\} \subset \phi
\]
が成り立つ.
また定理 \ref{sthmemptysubset=eq}と推論法則 \ref{dedequiv}により, 
\[
\tag{4}
  \{x \in a|R\} \subset \phi \to \{x \in a|R\} = \phi
\]
が成り立つ.
そこで(3), (4)から, 推論法則 \ref{dedmmp}によって
\[
\tag{5}
  a = \phi \to \{x \in a|R\} = \phi
\]
が成り立つ.
この(5)から, 推論法則 \ref{dedmp}によって, $a$が空のとき$\{x \in a|R\}$も空となることがわかる.
特にThm \ref{x=x}より$\phi = \phi$が成り立ち, 
また変数法則 \ref{valempty}より$x$は$\phi$の中に自由変数として現れないから, 
$\{x \in \phi|R\}$は空である.
\halmos




\mathstrut
\begin{thm}
\label{sthmobjectsetempty}%定理
$a$と$T$を集合とし, $x$を$a$の中に自由変数として現れない文字とする.
このとき
\[
  a = \phi \leftrightarrow \{T|x \in a\} = \phi
\]
が成り立つ.
またこのことから, 特に次の($*$)が成り立つ: 

($*$) ~~$a$が空ならば, $\{T|x \in a\}$は空である.
        特に$\{T|x \in \phi\}$は空である.
\end{thm}


\noindent{\bf 証明}
~まず前半を示す.
推論法則 \ref{dedequiv}があるから, 
\begin{align*}
  \tag{1}
  &a = \phi \to \{T|x \in a\} = \phi, \\
  \mbox{} \\
  \tag{2}
  &\{T|x \in a\} = \phi \to a = \phi
\end{align*}
が共に成り立つことを示せば良い.

(1)の証明: 
$y$を$x$と異なり, $a$及び$T$の中に自由変数として現れない文字とする.
このとき変数法則 \ref{valoset}により, $y$は$\{T|x \in a\}$の中に自由変数として現れない.
そこで$\tau_{y}(y \in \{T|x \in a\})$を$U$と書けば, 
$U$は集合であり, 定理 \ref{sthmelm&empty}と推論法則 \ref{dedequiv}により
\[
\tag{3}
  \{T|x \in a\} \neq \phi \to U \in \{T|x \in a\}
\]
が成り立つ.
また$y$が$x$と異なることから, 変数法則 \ref{valfund}, \ref{valtau}, \ref{valoset}により
$x$は$U$の中に自由変数として現れず, 
仮定より$x$は$a$の中にも自由変数として現れないから, 
定理 \ref{sthmosetbasis}と推論法則 \ref{dedequiv}により
\[
\tag{4}
  U \in \{T|x \in a\} \to \exists x(x \in a \wedge U = T)
\]
が成り立つ.
またThm \ref{thmexw}より
\[
  \exists x(x \in a \wedge U = T) \to \exists x(x \in a) \wedge \exists x(U = T)
\]
が成り立つから, 推論法則 \ref{dedprewedge}により
\[
\tag{5}
  \exists x(x \in a \wedge U = T) \to \exists x(x \in a)
\]
が成り立つ.
また$x$が$a$の中に自由変数として現れないことから, 
定理 \ref{sthmnotemptyeqexin}と推論法則 \ref{dedequiv}により
\[
\tag{6}
  \exists x(x \in a) \to a \neq \phi
\]
が成り立つ.
そこで(3)---(6)から, 推論法則 \ref{dedmmp}によって
\[
  \{T|x \in a\} \neq \phi \to a \neq \phi
\]
が成り立ち, これから推論法則 \ref{deds3}によって(1)が成り立つ.

(2)の証明: 
$\tau_{x}(x \in a)$を$V$と書けば, $V$は集合であり, 
仮定より$x$は$a$の中に自由変数として現れないから, 
定理 \ref{sthmelm&empty}と推論法則 \ref{dedequiv}により
\[
\tag{7}
  a \neq \phi \to V \in a
\]
が成り立つ.
また$x$が$a$の中に自由変数として現れないことから, 定理 \ref{sthmosetfund}より
\[
\tag{8}
  V \in a \to (V|x)(T) \in \{T|x \in a\}
\]
が成り立つ.
また定理 \ref{sthmnotemptyeqexin}より
\[
\tag{9}
  (V|x)(T) \in \{T|x \in a\} \to \{T|x \in a\} \neq \phi
\]
が成り立つ.
そこで(7), (8), (9)から, 推論法則 \ref{dedmmp}によって
\[
  a \neq \phi \to \{T|x \in a\} \neq \phi
\]
が成り立ち, これから推論法則 \ref{deds3}によって(2)が成り立つ.

さていま示したように(1)が成り立つから, $a$が空であるとすれば, 
推論法則 \ref{dedmp}によって$\{T|x \in a\}$も
空となることがわかる.
特に, Thm \ref{x=x}より$\phi = \phi$が成り立ち, 
変数法則 \ref{valempty}より$x$は$\phi$の中に自由変数として現れないから, 
$\{T|x \in \phi\}$は空である.
故に($*$)が成り立つ.
\halmos




\mathstrut
\begin{thm}
\label{sthmconstobjectset}%定理
$a$と$T$を集合とし, $x$をこれらの中に自由変数として現れない文字とする.
このとき, $a$が空でなければ, 
\[
  \{T|x \in a\} = \{T\}
\]
が成り立つ.
\end{thm}


\noindent{\bf 証明}
~$y$を$x$と異なり, $a$及び$T$の中に自由変数として現れない, 定数でない文字とする.
このとき変数法則 \ref{valoset}により, 
$y$は$\{T|x \in a\}$の中に自由変数として現れない.
また変数法則 \ref{valnset}からわかるように, $y$は$\{T\}$の中にも自由変数として現れない.
そして$x$が$y$と異なり, $a$の中に自由変数として現れないことから, 定理 \ref{sthmosetbasis}より
\[
\tag{1}
  y \in \{T|x \in a\} \leftrightarrow \exists x(x \in a \wedge y = T)
\]
が成り立つ.
また$x$が$y$と異なり, $T$の中に自由変数として現れないことから, 
変数法則 \ref{valfund}より$x$は$y = T$の中に自由変数として現れないから, 
Thm \ref{thmexwrfree}より
\[
\tag{2}
  \exists x(x \in a \wedge y = T) \leftrightarrow \exists x(x \in a) \wedge y = T
\]
が成り立つ.
また仮定より, $x$は$a$の中に自由変数として現れず, $a$が空でないから, 
定理 \ref{sthmnotemptyeqexin}により
\[
  \exists x(x \in a)
\]
が成り立つ.
そこで推論法則 \ref{dedawblatrue2}により, 
\[
\tag{3}
  \exists x(x \in a) \wedge y = T \leftrightarrow y = T
\]
が成り立つ.
また定理 \ref{sthmsingletonbasis}と推論法則 \ref{dedeqch}により
\[
\tag{4}
  y = T \leftrightarrow y \in \{T\}
\]
が成り立つ.
そこで(1)---(4)から, 推論法則 \ref{dedeqtrans}によって
\[
\tag{5}
  y \in \{T|x \in a\} \leftrightarrow y \in \{T\}
\]
が成り立つことがわかる.
いま$y$は定数でなく, はじめに述べたように$\{T|x \in a\}$及び
$\{T\}$の中に自由変数として現れないから, (5)から, 定理 \ref{sthmset=}によって
$\{T|x \in a\} = \{T\}$が成り立つ.
\halmos




%旧sthmcupempty跡地




\mathstrut
\begin{thm}
\label{sthmcapempty}%定理
$a$を集合とするとき, 
\[
  a \cap \phi = \phi, ~~
  \phi \cap a = \phi
\]
が成り立つ.
\end{thm}


\noindent{\bf 証明}
~定理 \ref{sthmemptysubset}より
\[
  \phi \subset a
\]
が成り立つ.
また定理 \ref{sthmcapsubset=}と推論法則 \ref{dedequiv}により, 
\[
  \phi \subset a \to \phi \cap a = \phi
\]
が成り立つ.
そこでこれらから, 推論法則 \ref{dedmp}によって$\phi \cap a = \phi$が成り立つ.
また定理 \ref{sthmcapch}より$a \cap \phi = \phi \cap a$が成り立つから, 
推論法則 \ref{ded=trans}によって$a \cap \phi = \phi$も成り立つ.
\halmos




\mathstrut
\begin{thm}
\label{sthmcapsingletonempty}%定理
$a$と$b$を集合とするとき, 
\[
  a \neq b \leftrightarrow \{a\} \cap \{b\} = \phi
\]
が成り立つ.
\end{thm}


\noindent{\bf 証明}
~推論法則 \ref{dedequiv}があるから, 
$a \neq b \to \{a\} \cap \{b\} = \phi$と
$\{a\} \cap \{b\} = \phi \to a \neq b$が共に成り立つことを示せば良い.

まず前者が成り立つことを示す.
いま$x$を$a$及び$b$の中に自由変数として現れない文字とし, 
$\tau_{x}(x \in \{a\} \cap \{b\})$を$T$と書く.
$T$は集合であり, 変数法則 \ref{valnset}, \ref{valcap}からわかるように
$x$は$\{a\} \cap \{b\}$の中に自由変数として現れないので, 
定理 \ref{sthmelm&empty}と推論法則 \ref{dedequiv}により
\[
\tag{1}
  T \notin \{a\} \cap \{b\} \to \{a\} \cap \{b\} = \phi
\]
が成り立つ.
また定理 \ref{sthmcapelement}と推論法則 \ref{dedequiv}により
\[
\tag{2}
  T \in \{a\} \cap \{b\} \to T \in \{a\} \wedge T \in \{b\}
\]
が成り立つ.
また定理 \ref{sthmsingletonbasis}と推論法則 \ref{dedequiv}により
\begin{align*}
  \tag{3}
  T \in \{a\} &\to T = a, \\
  \mbox{} \\
  \tag{4}
  T \in \{b\} &\to T = b
\end{align*}
が共に成り立つ.
またThm \ref{x=yty=x}より
\[
\tag{5}
  T = a \to a = T
\]
が成り立つ.
そこで(3), (5)から, 推論法則 \ref{dedmmp}によって
\[
\tag{6}
  T \in \{a\} \to a = T
\]
が成り立つ.
そこで(6), (4)から, 推論法則 \ref{dedfromaddw}によって
\[
\tag{7}
  T \in \{a\} \wedge T \in \{b\} \to a = T \wedge T = b
\]
が成り立つ.
またThm \ref{x=ywy=ztx=z}より
\[
\tag{8}
  a = T \wedge T = b \to a = b
\]
が成り立つ.
そこで(2), (7), (8)から, 推論法則 \ref{dedmmp}によって
\[
  T \in \{a\} \cap \{b\} \to a = b
\]
が成り立ち, これから推論法則 \ref{dedcp}によって
\[
\tag{9}
  a \neq b \to T \notin \{a\} \cap \{b\}
\]
が成り立つ.
そこで(9), (1)から, 推論法則 \ref{dedmmp}によって
$a \neq b \to \{a\} \cap \{b\} = \phi$が成り立つ.

次に後者が成り立つことを示す.
定理 \ref{sthmcapsingleton}と推論法則 \ref{dedequiv}により
\[
\tag{10}
  a = b \to \{a\} \cap \{b\} = \{a\}
\]
が成り立つ.
またThm \ref{x=yt1x=zly=z1}より
\[
  \{a\} \cap \{b\} = \{a\} \to (\{a\} \cap \{b\} = \phi \leftrightarrow \{a\} = \phi)
\]
が成り立つから, 推論法則 \ref{dedprewedge}により
\[
\tag{11}
  \{a\} \cap \{b\} = \{a\} \to (\{a\} \cap \{b\} = \phi \to \{a\} = \phi)
\]
が成り立つ.
またThm \ref{1atb1t1nbtna1}より
\[
  (\{a\} \cap \{b\} = \phi \to \{a\} = \phi) \to (\{a\} \neq \phi \to \{a\} \cap \{b\} \neq \phi)
\]
が成り立つから, 推論法則 \ref{dedch}により
\[
  \{a\} \neq \phi \to ((\{a\} \cap \{b\} = \phi \to \{a\} = \phi) \to \{a\} \cap \{b\} \neq \phi)
\]
が成り立つ.
いま定理 \ref{sthmsunotempty}より
$\{a\} \neq \phi$が成り立つから, これらから, 
推論法則 \ref{dedmp}によって
\[
\tag{12}
  (\{a\} \cap \{b\} = \phi \to \{a\} = \phi) \to \{a\} \cap \{b\} \neq \phi
\]
が成り立つ.
そこで(10), (11), (12)から, 推論法則 \ref{dedmmp}によって
\[
  a = b \to \{a\} \cap \{b\} \neq \phi
\]
が成り立ち, これから推論法則 \ref{dedcp}によって
$\{a\} \cap \{b\} = \phi \to a \neq b$が成り立つ.
\halmos




\mathstrut
{\small
\noindent
{\bf 例 1.}~%例
$x$を文字とするとき, 
\[
  \{\phi|x \in \{\{\phi\}\}\} \cap \{\phi|x \in \{\phi\}\} 
  \not\subset \{\phi|x \in \{\{\phi\}\} \cap \{\phi\}\}
\]
が成り立つ.

実際変数法則 \ref{valnset}, \ref{valcap}, \ref{valempty}からわかるように, 
$x$は$\phi$, $\{\phi\}$, $\{\{\phi\}\}$, $\{\{\phi\}\} \cap \{\phi\}$のいずれの記号列の中にも
自由変数として現れない.
また定理 \ref{sthmsunotempty}より, $\{\phi\}$と$\{\{\phi\}\}$は共に空でない.
そこで定理 \ref{sthmconstobjectset}より
\[
  \{\phi|x \in \{\{\phi\}\}\} = \{\phi\}, ~~
  \{\phi|x \in \{\phi\}\} = \{\phi\}
\]
が共に成り立ち, これらから, 定理 \ref{sthmcap=}によって
\[
\tag{1}
  \{\phi|x \in \{\{\phi\}\}\} \cap \{\phi|x \in \{\phi\}\} = \{\phi\} \cap \{\phi\}
\]
が成り立つ.
また定理 \ref{sthmcapidempotent}より
\[
\tag{2}
  \{\phi\} \cap \{\phi\} = \{\phi\}
\]
が成り立つ.
そこで(1), (2)から推論法則 \ref{ded=trans}によって
\[
\tag{3}
  \{\phi|x \in \{\{\phi\}\}\} \cap \{\phi|x \in \{\phi\}\} = \{\phi\}
\]
が成り立つ.
いま定理 \ref{sthmsingletonfund}より$\phi \in \{\phi\}$が成り立つから, 
これと(3)が成り立つことから, 定理 \ref{sthm=tineq}によって
\[
\tag{4}
  \phi \in \{\phi|x \in \{\{\phi\}\}\} \cap \{\phi|x \in \{\phi\}\}
\]
が成り立つことがわかる.
また上述のように$\{\phi\} \neq \phi$が成り立ち, 
定理 \ref{sthmcapsingletonempty}と推論法則 \ref{dedequiv}により
$\{\phi\} \neq \phi \to \{\{\phi\}\} \cap \{\phi\} = \phi$が成り立つから, 
これらから, 推論法則 \ref{dedmp}によって
$\{\{\phi\}\} \cap \{\phi\} = \phi$が成り立つ.
このことと, 上述のように$x$が$\{\{\phi\}\} \cap \{\phi\}$の中に
自由変数として現れないことから, 定理 \ref{sthmobjectsetempty}により
\[
  \{\phi|x \in \{\{\phi\}\} \cap \{\phi\}\} = \phi
\]
が成り立つ.
そこで定理 \ref{sthm=tineq}により
\[
\tag{5}
  \phi \in \{\phi|x \in \{\{\phi\}\} \cap \{\phi\}\} \leftrightarrow \phi \in \phi
\]
が成り立つ.
いま定理 \ref{sthmnotinempty}より$\phi \notin \phi$が成り立つから, 
これと(5)が成り立つことから, 推論法則 \ref{dedeqfund}によって
\[
\tag{6}
  \phi \notin \{\phi|x \in \{\{\phi\}\} \cap \{\phi\}\}
\]
が成り立つことがわかる.
(4)と(6)から, 定理 \ref{sthmnotsubsetbasis}により
\[
  \{\phi|x \in \{\{\phi\}\}\} \cap \{\phi|x \in \{\phi\}\} 
  \not\subset \{\phi|x \in \{\{\phi\}\} \cap \{\phi\}\}
\]
が成り立つ. ------
}




\mathstrut
\begin{thm}
\label{sthm-subset=}%定理
$a$と$b$を集合とするとき, 
\[
  a \subset b \leftrightarrow a - b = \phi
\]
が成り立つ.
\end{thm}


\noindent{\bf 証明}
~推論法則 \ref{dedequiv}があるから, $a \subset b \to a - b = \phi$と
$a - b = \phi \to a \subset b$が共に成り立つことを示せば良い.
以下$x$を$a$及び$b$の中に自由変数として現れない文字とする.

まず前者が成り立つことを示す.
$\tau_{x}(x \in a - b)$を$T$と書けば, 
$T$は集合であり, 定理 \ref{sthmsubsetbasis}より
\[
\tag{1}
  a \subset b \to (T \in a \to T \in b)
\]
が成り立つ.
またThm \ref{n1atb1tawnb}より
$T \in a \wedge T \notin b \to \neg (T \in a \to T \in b)$が
成り立つから, 推論法則 \ref{dedcp}により
\[
\tag{2}
  (T \in a \to T \in b) \to \neg (T \in a \wedge T \notin b)
\]
が成り立つ.
また定理 \ref{sthm-basis}と推論法則 \ref{dedequiv}により
$T \in a - b \to T \in a \wedge T \notin b$が
成り立つから, 推論法則 \ref{dedcp}により
\[
\tag{3}
  \neg (T \in a \wedge T \notin b) \to T \notin a - b
\]
が成り立つ.
またいま$x$が$a$及び$b$の中に自由変数として現れないことから, 
変数法則 \ref{val-}により$x$は$a - b$の中にも自由変数として現れないから, 
このことと$T$の定義から, 定理 \ref{sthmelm&empty}と推論法則 \ref{dedequiv}により
\[
\tag{4}
  T \notin a - b \to a - b = \phi
\]
が成り立つ.
そこで(1)---(4)から, 推論法則 \ref{dedmmp}によって
$a \subset b \to a - b = \phi$が成り立つことがわかる.

次に後者が成り立つことを示す.
$\tau_{x}(\neg (x \in a \to x \in b))$を$U$と書けば, $U$は集合であり, 
定理 \ref{sthmnotinempty}より
\[
\tag{5}
  a - b = \phi \to U \notin a - b
\]
が成り立つ.
また定理 \ref{sthm-basis}と推論法則 \ref{dedequiv}により
$U \in a \wedge U \notin b \to U \in a - b$が成り立つから, 
推論法則 \ref{dedcp}により
\[
\tag{6}
  U \notin a - b \to \neg (U \in a \wedge U \notin b)
\]
が成り立つ.
またThm \ref{n1atb1tawnb}より
$\neg (U \in a \to U \in b) \to U \in a \wedge U \notin b$が成り立つから, 
推論法則 \ref{dedcp}により
\[
  \neg (U \in a \wedge U \notin b) \to (U \in a \to U \in b)
\]
が成り立つ.
ここで$x$が$a$及び$b$の中に自由変数として現れないことから, 
代入法則 \ref{substfree}, \ref{substfund}により, この記号列は
\[
\tag{7}
  \neg (U \in a \wedge U \notin b) \to (U|x)(x \in a \to x \in b)
\]
と一致する.
よってこれが定理となる.
また$U$の定義から, Thm \ref{thmallfund1}と推論法則 \ref{dedequiv}により
\[
  (U|x)(x \in a \to x \in b) \to \forall x(x \in a \to x \in b)
\]
が成り立つが, 
いま$x$が$a$及び$b$の中に自由変数として現れないので, 
定義からこの記号列は
\[
\tag{8}
  (U|x)(x \in a \to x \in b) \to a \subset b
\]
と同じである.
よってこれが定理となる.
そこで(5)---(8)から, 推論法則 \ref{dedmmp}によって
$a - b = \phi \to a \subset b$が成り立つことがわかる.
\halmos




\mathstrut
\begin{thm}
\label{sthm-emptyyet}%定理%koko
$a$を集合とするとき, 
\[
  a - a = \phi, ~~
  a - \phi = a, ~~
  \phi - a = \phi
\]
が成り立つ.
\end{thm}


\noindent{\bf 証明}
~
$a - a = \phi$の証明: 
定理 \ref{sthm-subset=}と推論法則 \ref{dedequiv}により
$a \subset a \to a - a = \phi$が成り立ち, 
定理 \ref{sthmsubsetself}より$a \subset a$が成り立つから, 
推論法則 \ref{dedmp}によって
$a - a = \phi$が成り立つ.

$a - \phi = a$の証明: 
$x$を, $a$の中に自由変数として現れない, 定数でない文字とする.
このとき変数法則 \ref{val-}, \ref{valempty}からわかるように, 
$x$は$a - \phi$の中にも自由変数として現れない.
また定理 \ref{sthm-basis}より
\[
  x \in a - \phi \leftrightarrow x \in a \wedge x \notin \phi
\]
が成り立つ.
また定理 \ref{sthmnotinempty}より$x \notin \phi$が成り立つから, 
推論法則 \ref{dedawblatrue2}により
\[
  x \in a \wedge x \notin \phi \leftrightarrow x \in a
\]
が成り立つ.
そこでこれらから, 推論法則 \ref{dedeqtrans}によって
\[
\tag{$*$}
  x \in a - \phi \leftrightarrow x \in a
\]
が成り立つ.
いま$x$は定数でなく, 上述のように$a$及び$a - \phi$の中に自由変数として現れないから, 
($*$)から, 定理 \ref{sthmset=}によって$a - \phi = a$が成り立つ.

$\phi - a = \phi$の証明: 
定理 \ref{sthma-bsubseta}より$\phi - a \subset \phi$が成り立つから, 
定理 \ref{sthmemptysubset=eq}によって$\phi - a = \phi$が成り立つ.
\halmos




\mathstrut
{\small
\noindent
{\bf 例 2.}~%例
$x$を文字とするとき, 
\[
  \{\phi|x \in \{\{\phi\}\} - \{\phi\}\} \not\subset 
  \{\phi|x \in \{\{\phi\}\}\} - \{\phi|x \in \{\phi\}\}
\]
が成り立つ.

実際例1で述べたように
\[
  \{\phi|x \in \{\{\phi\}\}\} = \{\phi\}, ~~
  \{\phi|x \in \{\phi\}\} = \{\phi\}
\]
が共に成り立つから, 定理 \ref{sthm-=}により
\[
\tag{1}
  \{\phi|x \in \{\{\phi\}\}\} - \{\phi|x \in \{\phi\}\} = \{\phi\} - \{\phi\}
\]
が成り立つ.
また定理 \ref{sthm-emptyyet}より
\[
\tag{2}
  \{\phi\} - \{\phi\} = \phi
\]
が成り立つ.
そこで(1), (2)から, 推論法則 \ref{ded=trans}によって
\[
  \{\phi|x \in \{\{\phi\}\}\} - \{\phi|x \in \{\phi\}\} = \phi
\]
が成り立ち, これから定理 \ref{sthm=tineq}によって
\[
\tag{3}
  \phi \in \{\phi|x \in \{\{\phi\}\}\} - \{\phi|x \in \{\phi\}\} \leftrightarrow \phi \in \phi
\]
が成り立つ.
いま定理 \ref{sthmnotinempty}より$\phi \notin \phi$が成り立つから, 
これと(3)が成り立つことから, 推論法則 \ref{dedeqfund}によって
\[
\tag{4}
  \phi \notin \{\phi|x \in \{\{\phi\}\}\} - \{\phi|x \in \{\phi\}\}
\]
が成り立つことがわかる.
また定理 \ref{sthmsunotempty}より
$\{\phi\} \neq \phi$が成り立ち, 定理 \ref{sthmsingleton-}と推論法則 \ref{dedequiv}により
$\{\phi\} \neq \phi \to \{\{\phi\}\} - \{\phi\} = \{\{\phi\}\}$が成り立つから, 
推論法則 \ref{dedmp}によって
\[
\tag{5}
  \{\{\phi\}\} - \{\phi\} = \{\{\phi\}\}
\]
が成り立つ.
いま変数法則 \ref{valnset}, \ref{val-}, \ref{valempty}によって
$x$が$\{\{\phi\}\} - \{\phi\}$及び$\{\{\phi\}\}$の中に自由変数として現れないことがわかるから, 
このことと(5)が成り立つことから, 定理 \ref{sthmoset=}により
\[
  \{\phi|x \in \{\{\phi\}\} - \{\phi\}\} = \{\phi|x \in \{\{\phi\}\}\}, 
\]
が成り立つ.
また上述のように$\{\phi|x \in \{\{\phi\}\}\} = \{\phi\}$が成り立つから, これらから, 
推論法則 \ref{ded=trans}によって
\[
\tag{6}
  \{\phi|x \in \{\{\phi\}\} - \{\phi\}\} = \{\phi\}
\]
が成り立つ.
いま定理 \ref{sthmsingletonfund}より$\phi \in \{\phi\}$が成り立つから, 
これと(6)が成り立つことから, 定理 \ref{sthm=tineq}によって
\[
\tag{7}
  \phi \in \{\phi|x \in \{\{\phi\}\} - \{\phi\}\}
\]
が成り立つことがわかる.
(4)と(7)から, 定理 \ref{sthmnotsubsetbasis}によって
\[
  \{\phi|x \in \{\{\phi\}\} - \{\phi\}\} \not\subset 
  \{\phi|x \in \{\{\phi\}\}\} - \{\phi|x \in \{\phi\}\}
\]
が成り立つ. ------
}




\mathstrut
\begin{thm}
\label{sthm-singletonempty}%定理
$a$と$b$を集合とするとき, 
\[
  a = b \leftrightarrow \{a\} - \{b\} = \phi
\]
が成り立つ.
\end{thm}


\noindent{\bf 証明}
~定理 \ref{sthmsingleton=subset}より
\[
  a = b \leftrightarrow \{a\} \subset \{b\}
\]
が成り立ち, 定理 \ref{sthm-subset=}より
\[
  \{a\} \subset \{b\} \leftrightarrow \{a\} - \{b\} = \phi
\]
が成り立つから, これらから, 推論法則 \ref{dedeqtrans}によって
$a = b \leftrightarrow \{a\} - \{b\} = \phi$が成り立つ.
\halmos
%[4]確認済



\mathstrut
\noindent
[\textbf{5}] \textbf{補集合}




\newpage
\setcounter{defi}{0}
\section{順序対}%%%%%%%%%%%%%%%%%%%%%%%%%%%%%%%%%%%%%%%%




この節では, \S 3で定義した非順序対を基にして, 
二つの集合の順序付けられた組(順序対)を構成し, その性質を述べる.




\mathstrut
\begin{defi}
\label{defpair}%定義
$a$と$b$を記号列とするとき, 
記号列$\{\{a\}, \{a, b\}\}$を$(a, b)$と書き表し, これを
$a$と$b$の\textbf{順序対}(ordered pair)
あるいは単に\textbf{対}(pair)という.
\end{defi}




\mathstrut
\begin{valu}
\label{valpair}%変数
$a$, $b$を記号列とし, $x$を文字とする.
$x$が$a$及び$b$の中に自由変数として現れなければ, 
$x$は$(a, b)$の中に自由変数として現れない.
\end{valu}


\noindent{\bf 証明}
~このとき変数法則 \ref{valnset}より
$x$は$\{a\}$及び$\{a, b\}$の中に自由変数として現れないから, 
やはり変数法則 \ref{valnset}より, 
$x$は$\{\{a\}, \{a, b\}\}$, 即ち$(a, b)$の中にも自由変数として現れない.
\halmos




\mathstrut
\begin{subs}
\label{substpair}%代入
$a$, $b$, $c$を記号列とし, $x$を文字とするとき, 
\[
  (c|x)((a, b)) \equiv ((c|x)(a), (c|x)(b))
\]
が成り立つ.
\end{subs}


\noindent{\bf 証明}
~定義から$(a, b)$は$\{\{a\}, \{a, b\}\}$だから, 
代入法則 \ref{substnset}により
\[
  (c|x)((a, b)) \equiv \{(c|x)(\{a\}), (c|x)(\{a, b\})\}
\]
が成り立つ.
また同じく代入法則 \ref{substnset}により
\[
  (c|x)(\{a\}) \equiv \{(c|x)(a)\}, ~~
  (c|x)(\{a, b\}) \equiv \{(c|x)(a), (c|x)(b)\}
\]
が成り立つ.
そこでこれらから, $(c|x)((a, b))$が$\{\{(c|x)(a)\}, \{(c|x)(a), (c|x)(b)\}\}$と一致することがわかる.
これは$((c|x)(a), (c|x)(b))$と書き表される記号列である.
\halmos




\mathstrut
\begin{form}
\label{formpair}%構成
$a$と$b$が集合ならば, $(a, b)$は集合である.
\end{form}


\noindent{\bf 証明}
~このとき構成法則 \ref{formnset}により
$\{a\}$と$\{a, b\}$が共に集合となるから, 
やはり構成法則 \ref{formnset}により, 
$\{\{a\}, \{a, b\}\}$, 即ち$(a, b)$も集合となる.
\halmos




\mathstrut
順序対に対して成り立つ定理のうち, 最も重要なものは次の定理 \ref{sthmpair}である.




\mathstrut
\begin{thm}
\label{sthmpair}%定理
$a$, $b$, $c$, $d$を集合とするとき, 
\[
  (a, b) = (c, d) \leftrightarrow a = c \wedge b = d
\]
が成り立つ.
またこのことから, 次の($*$)が成り立つ: 

($*$) ~~$(a, b) = (c, d)$が成り立つならば, $a = c$と$b = d$が共に成り立つ.
        逆に$a = c$と$b = d$が共に成り立つならば, $(a, b) = (c, d)$が成り立つ.
\end{thm}


\noindent{\bf 証明}
~まず前半を示す.
推論法則 \ref{dedequiv}があるから, 
\begin{align*}
  \tag{1}
  &(a, b) = (c, d) \to a = c \wedge b = d, \\
  \mbox{} \\
  \tag{2}
  &a = c \wedge b = d \to (a, b) = (c, d)
\end{align*}
が共に成り立つことを示せば良い.

(1)の証明: 
はじめに
\[
\tag{3}
  (a, b) = (c, d) \to a = c
\]
が成り立つことを示す.
定理 \ref{sthmuopairfund}より$\{a\} \in (a, b)$が成り立つ($(a, b)$の定義に注意)から, 
推論法則 \ref{dedatawbtrue2}により
\[
\tag{4}
  (a, b) = (c, d) \to (a, b) = (c, d) \wedge \{a\} \in (a, b)
\]
が成り立つ.
また定理 \ref{sthm=&in}より
\[
\tag{5}
  (a, b) = (c, d) \wedge \{a\} \in (a, b) \to \{a\} \in (c, d)
\]
が成り立つ.
また定理 \ref{sthmuopairbasis}と推論法則 \ref{dedequiv}により
\[
\tag{6}
  \{a\} \in (c, d) \to \{a\} = \{c\} \vee \{a\} = \{c, d\}
\]
が成り立つ.
また定理 \ref{sthmsingleton=}と推論法則 \ref{dedequiv}により
\[
\tag{7}
  \{a\} = \{c\} \to a = c
\]
が成り立つ.
また定理 \ref{sthmuopairfund}より$c \in \{c, d\}$が成り立つから, 
推論法則 \ref{dedatawbtrue2}により
\[
\tag{8}
  \{a\} = \{c, d\} \to \{a\} = \{c, d\} \wedge c \in \{c, d\}
\]
が成り立つ.
また定理 \ref{sthm=&in}より
\[
\tag{9}
  \{a\} = \{c, d\} \wedge c \in \{c, d\} \to c \in \{a\}
\]
が成り立つ.
また定理 \ref{sthmsingletonbasis}と推論法則 \ref{dedequiv}により
\[
\tag{10}
  c \in \{a\} \to c = a
\]
が成り立つ.
またThm \ref{x=yty=x}より
\[
\tag{11}
  c = a \to a = c
\]
が成り立つ.
そこで(8)---(11)から, 推論法則 \ref{dedmmp}によって
\[
\tag{12}
  \{a\} = \{c, d\} \to a = c
\]
が成り立つことがわかる.
そこで(7), (12)から, 推論法則 \ref{deddil}によって
\[
\tag{13}
  \{a\} = \{c\} \vee \{a\} = \{c, d\} \to a = c
\]
が成り立つ.
そこで(4), (5), (6), (13)から, 推論法則 \ref{dedmmp}によって(3)が成り立つことがわかる.

さて次に
\[
\tag{14}
  (a, b) = (c, d) \to b = c \vee b = d
\]
が成り立つことを示す.
定理 \ref{sthmuopairfund}より$\{a, b\} \in (a, b)$が成り立つから, 
推論法則 \ref{dedatawbtrue2}により
\[
\tag{15}
  (a, b) = (c, d) \to (a, b) = (c, d) \wedge \{a, b\} \in (a, b)
\]
が成り立つ.
また定理 \ref{sthm=&in}より
\[
\tag{16}
  (a, b) = (c, d) \wedge \{a, b\} \in (a, b) \to \{a, b\} \in (c, d)
\]
が成り立つ.
また定理 \ref{sthmuopairbasis}と推論法則 \ref{dedequiv}により
\[
\tag{17}
  \{a, b\} \in (c, d) \to \{a, b\} = \{c\} \vee \{a, b\} = \{c, d\}
\]
が成り立つ.
また定理 \ref{sthmuopairfund}より$b \in \{a, b\}$が成り立つから, 
推論法則 \ref{dedatawbtrue2}により
\begin{align*}
  \tag{18}
  \{a, b\} = \{c\} &\to \{a, b\} = \{c\} \wedge b \in \{a, b\}, \\
  \mbox{} \\
  \tag{19}
  \{a, b\} = \{c, d\} &\to \{a, b\} = \{c, d\} \wedge b \in \{a, b\}
\end{align*}
が共に成り立つ.
また定理 \ref{sthm=&in}より
\begin{align*}
  \tag{20}
  \{a, b\} = \{c\} \wedge b \in \{a, b\} &\to b \in \{c\}, \\
  \mbox{} \\
  \tag{21}
  \{a, b\} = \{c, d\} \wedge b \in \{a, b\} &\to b \in \{c, d\}
\end{align*}
が共に成り立つ.
また定理 \ref{sthmsingletonbasis}と推論法則 \ref{dedequiv}により
\[
\tag{22}
  b \in \{c\} \to b = c
\]
が成り立つ.
またThm \ref{atavb}より
\[
\tag{23}
  b = c \to b = c \vee b = d
\]
が成り立つ.
また定理 \ref{sthmuopairbasis}と推論法則 \ref{dedequiv}により
\[
\tag{24}
  b \in \{c, d\} \to b = c \vee b = d
\]
が成り立つ.
そこで(18), (20), (22), (23)から, 推論法則 \ref{dedmmp}によって
\[
\tag{25}
  \{a, b\} = \{c\} \to b = c \vee b = d
\]
が成り立ち, (19), (21), (24)から, 同じく推論法則 \ref{dedmmp}によって
\[
\tag{26}
  \{a, b\} = \{c, d\} \to b = c \vee b = d
\]
が成り立つことがわかる.
そこで(25), (26)から, 推論法則 \ref{deddil}によって
\[
\tag{27}
  \{a, b\} = \{c\} \vee \{a, b\} = \{c, d\} \to b = c \vee b = d
\]
が成り立つ.
そこで(15), (16), (17), (27)から, 推論法則 \ref{dedmmp}によって(14)が成り立つことがわかる.

最後に(1)が成り立つことを示す.
いま示したように(14)が成り立つから, $a$と$c$, $b$と$d$を入れ替えた
\[
\tag{28}
  (c, d) = (a, b) \to d = a \vee d = b
\]
も成り立つ.
またThm \ref{x=yty=x}より
\[
\tag{29}
  (a, b) = (c, d) \to (c, d) = (a, b)
\]
が成り立つ.
同じくThm \ref{x=yty=x}より
$d = a \to a = d$と$d = b \to b = d$が共に成り立つから, 
推論法則 \ref{dedfromaddv}により
\[
\tag{30}
  d = a \vee d = b \to a = d \vee b = d
\]
が成り立つ.
そこで(29), (28), (30)から, 推論法則 \ref{dedmmp}によって
\[
\tag{31}
  (a, b) = (c, d) \to a = d \vee b = d
\]
が成り立つ.
そこで(14), (31)から, 推論法則 \ref{dedprewedge}によって
\[
\tag{32}
  (a, b) = (c, d) \to (b = c \vee b = d) \wedge (a = d \vee b = d)
\]
が成り立つ.
またThm \ref{1avb1w1avc1tav1bwc1}より
\[
\tag{33}
  (b = c \vee b = d) \wedge (a = d \vee b = d) \to (b = c \wedge a = d) \vee b = d
\]
が成り立つ.
そこで(32), (33)から, 推論法則 \ref{dedmmp}によって
\[
\tag{34}
  (a, b) = (c, d) \to (b = c \wedge a = d) \vee b = d
\]
が成り立つ.
そこで(3), (34)から, 推論法則 \ref{dedprewedge}によって
\[
\tag{35}
  (a, b) = (c, d) \to a = c \wedge ((b = c \wedge a = d) \vee b = d)
\]
が成り立つ.
またThm \ref{aw1bvc1t1awb1v1awc1}より
\[
\tag{36}
  a = c \wedge ((b = c \wedge a = d) \vee b = d) \to (a = c \wedge (b = c \wedge a = d)) \vee (a = c \wedge b = d)
\]
が成り立つ.
またThm \ref{x=yty=x}より$a = c \to c = a$が成り立つから, 
推論法則 \ref{dedaddw}により
\[
\tag{37}
  a = c \wedge (b = c \wedge a = d) \to c = a \wedge (b = c \wedge a = d)
\]
が成り立つ.
またThm \ref{aw1bwc1t1awb1wc}より
\[
\tag{38}
  c = a \wedge (b = c \wedge a = d) \to (c = a \wedge b = c) \wedge a = d
\]
が成り立つ.
またThm \ref{awbtbwa}より
\[
  c = a \wedge b = c \to b = c \wedge c = a
\]
が成り立ち, 
Thm \ref{x=ywy=ztx=z}より
\[
  b = c \wedge c = a \to b = a
\]
が成り立つから, 
推論法則 \ref{dedmmp}によって
\[
  c = a \wedge b = c \to b = a
\]
が成り立つ.
そこで推論法則 \ref{dedaddw}によって
\[
\tag{39}
  (c = a \wedge b = c) \wedge a = d \to b = a \wedge a = d
\]
が成り立つ.
またThm \ref{x=ywy=ztx=z}より
\[
\tag{40}
  b = a \wedge a = d \to b = d
\]
が成り立つ.
そこで(37)---(40)から, 推論法則 \ref{dedmmp}によって
\[
\tag{41}
  a = c \wedge (b = c \wedge a = d) \to b = d
\]
が成り立つ.
またThm \ref{awbta}より
\[
\tag{42}
  a = c \wedge (b = c \wedge a = d) \to a = c
\]
が成り立つ.
そこで(41), (42)から, 推論法則 \ref{dedprewedge}によって
\[
  a = c \wedge (b = c \wedge a = d) \to a = c \wedge b = d
\]
が成り立ち, これから推論法則 \ref{dedavbtbtrue1}によって
\[
\tag{43}
  (a = c \wedge (b = c \wedge a = d)) \vee (a = c \wedge b = d) \to a = c \wedge b = d
\]
が成り立つ.
そこで(35), (36), (43)から, 推論法則 \ref{dedmmp}によって
(1)が成り立つ.

(2)の証明: 
Thm \ref{awbta}より$a = c \wedge b = d \to a = c$が成り立ち, 
定理 \ref{sthmsingleton=}と推論法則 \ref{dedequiv}により
$a = c \to \{a\} = \{c\}$が成り立つから, これらから, 推論法則 \ref{dedmmp}によって
\[
\tag{44}
  a = c \wedge b = d \to \{a\} = \{c\}
\]
が成り立つ.
また定理 \ref{sthmuopair=}より
\[
\tag{45}
  a = c \wedge b = d \to \{a, b\} = \{c, d\}
\]
が成り立つ.
そこで(44), (45)から, 推論法則 \ref{dedprewedge}によって
\[
\tag{46}
  a = c \wedge b = d \to \{a\} = \{c\} \wedge \{a, b\} = \{c, d\}
\]
が成り立つ.
また定理 \ref{sthmuopair=}より
\[
  \{a\} = \{c\} \wedge \{a, b\} = \{c, d\} \to \{\{a\}, \{a, b\}\} = \{\{c\}, \{c, d\}\}
\]
が成り立つが, 定義からこれは
\[
\tag{47}
  \{a\} = \{c\} \wedge \{a, b\} = \{c, d\} \to (a, b) = (c, d)
\]
であるから, これが定理となる.
そこで(46), (47)から, 推論法則 \ref{dedmmp}によって(2)が成り立つ.

最後に($*$)が成り立つことを示す.
いま$(a, b) = (c, d)$が成り立つとする.
このとき上に示したように(1)が成り立つから, これらから, 推論法則 \ref{dedmp}によって
$a = c \wedge b = d$が成り立つ.
そこで推論法則 \ref{dedwedge}によって$a = c$と$b = d$が共に成り立つ.
逆に$a = c$と$b = d$が共に成り立つならば, 推論法則 \ref{dedwedge}によって
$a = c \wedge b = d$が成り立ち, また上に示したように(2)が成り立つから, 
これらから, 推論法則 \ref{dedmp}によって$(a, b) = (c, d)$が成り立つ.
\halmos




\mathstrut
この定理の特別な場合として, 次の定理が成り立つことを示しておく.




\mathstrut
\begin{thm}
\label{sthmpairweak}%定理
$a$, $b$, $c$を集合とするとき, 
\[
  a = b \leftrightarrow (a, c) = (b, c), ~~
  a = b \leftrightarrow (c, a) = (c, b)
\]
が成り立つ.
またこのことから, 特に次の($*$)が成り立つ: 

($*$) ~~$a = b$が成り立つならば, $(a, c) = (b, c)$と$(c, a) = (c, b)$が共に成り立つ.
\end{thm}


\noindent{\bf 証明}
~Thm \ref{x=x}より$c = c$が成り立つから, 推論法則 \ref{dedawblatrue2}により
\[
  a = b \wedge c = c \leftrightarrow a = b
\]
が成り立つ.
そこで推論法則 \ref{dedeqch}によって
\[
\tag{1}
  a = b \leftrightarrow a = b \wedge c = c
\]
が成り立つ.
またThm \ref{awblbwa}より
\[
  a = b \wedge c = c \leftrightarrow c = c \wedge a = b
\]
が成り立つから, これと(1)から, 推論法則 \ref{dedeqtrans}によって
\[
\tag{2}
  a = b \leftrightarrow c = c \wedge a = b
\]
が成り立つ.
また定理 \ref{sthmpair}より
\[
  (a, c) = (b, c) \leftrightarrow a = b \wedge c = c, ~~
  (c, a) = (c, b) \leftrightarrow c = c \wedge a = b
\]
が共に成り立つから, 推論法則 \ref{dedeqch}によって
\begin{align*}
  \tag{3}
  a = b \wedge c = c &\leftrightarrow (a, c) = (b, c), \\
  \mbox{} \\
  \tag{4}
  c = c \wedge a = b &\leftrightarrow (c, a) = (c, b)
\end{align*}
が共に成り立つ.
そこで(1)と(3), (2)と(4)から, それぞれ推論法則 \ref{dedeqtrans}によって
\[
  a = b \leftrightarrow (a, c) = (b, c), ~~
  a = b \leftrightarrow (c, a) = (c, b)
\]
が成り立つことがわかる.
($*$)が成り立つことは, これらと推論法則 \ref{dedeqfund}から明らかである.
\halmos




\mathstrut
\begin{defo}
\label{bigpair}%変形
$a$を記号列とする.
また$x$と$y$, $z$と$w$を, それぞれ互いに異なり, 
いずれも$a$の中に自由変数として現れない文字とする.
このとき
\[
  \exists x(\exists y(a = (x, y))) \equiv \exists z(\exists w(a = (z, w)))
\]
が成り立つ.
\end{defo}


\noindent{\bf 証明}
~$u$と$v$を, 互いに異なり, 共に$x$, $y$, $z$, $w$のいずれとも異なり, $a$の
中に自由変数として現れない文字とする.
このとき変数法則 \ref{valfund}, \ref{valquan}, \ref{valpair}からわかるように, 
$u$は$\exists y(a = (x, y))$の中に自由変数として現れないから, 
代入法則 \ref{substquantrans}により
\[
  \exists x(\exists y(a = (x, y))) \equiv \exists u((u|x)(\exists y(a = (x, y))))
\]
が成り立つ.
また$y$が$x$とも$u$とも異なることから, 代入法則 \ref{substquan}により
\[
  (u|x)(\exists y(a = (x, y))) \equiv \exists y((u|x)(a = (x, y)))
\]
が成り立つ.
また$x$が$y$と異なり, $a$の中に自由変数として現れないことから, 
代入法則 \ref{substfree}, \ref{substfund}, \ref{substpair}により
\[
  (u|x)(a = (x, y)) \equiv a = (u, y)
\]
が成り立つ.
よってこれらから, 
\[
\tag{1}
  \exists x(\exists y(a = (x, y))) \equiv \exists u(\exists y(a = (u, y)))
\]
が成り立つ.
また$v$が$u$とも$y$とも異なり, $a$の中に自由変数として現れないことから, 
変数法則 \ref{valfund}, \ref{valpair}により
$v$は$a = (u, y)$の中に自由変数として現れないから, 
代入法則 \ref{substquantrans}により
\[
  \exists y(a = (u, y)) \equiv \exists v((v|y)(a = (u, y)))
\]
が成り立つ.
また$y$が$u$と異なり, $a$の中に自由変数として現れないことから, 
代入法則 \ref{substfree}, \ref{substfund}, \ref{substpair}により
\[
  (v|y)(a = (u, y)) \equiv a = (u, v)
\]
が成り立つ.
よってこれらから, 
\[
\tag{2}
  \exists u(\exists y(a = (u, y))) \equiv \exists u(\exists v(a = (u, v)))
\]
が成り立つ.
そこで(1)と(2)から, $\exists x(\exists y(a = (x, y)))$と
$\exists u(\exists v(a = (u, v)))$が一致することがわかる.
ここまでの議論と全く同様に, $\exists z(\exists w(a = (z, w)))$も
$\exists u(\exists v(a = (u, v)))$と一致する.
故に$\exists x(\exists y(a = (x, y)))$と$\exists z(\exists w(a = (z, w)))$は
同一の記号列である.
\halmos




\mathstrut
\begin{defi}
\label{defbigpair}%定義
$a$を記号列とする.
また$x$と$y$, $z$と$w$を, それぞれ互いに異なり, 
いずれも$a$の中に自由変数として現れない文字とする.
このとき上記の変形法則 \ref{bigpair}によれば, 
$\exists x(\exists y(a = (x, y)))$と$\exists z(\exists w(a = (z, w)))$は
同一の記号列となる.
$a$に対して定まるこの記号列を, ${\rm Pair}(a)$と書き表す.
\end{defi}




\mathstrut
\begin{valu}
\label{valbigpair}%変数
$a$を記号列とし, $x$を文字とする.
$x$が$a$の中に自由変数として現れなければ, 
$x$は${\rm Pair}(a)$の中に自由変数として現れない.
\end{valu}


\noindent{\bf 証明}
~$y$を$x$と異なり, $a$の中に自由変数として現れない文字とすれば, 
定義から${\rm Pair}(a)$は$\exists x(\exists y(a = (x, y)))$と同じである.
変数法則 \ref{valquan}より, $x$はこの中に自由変数として現れない.
\halmos




\mathstrut
\begin{subs}
\label{substbigpair}%代入
$a$と$b$を記号列とし, $x$を文字とする.
このとき
\[
  (b|x)({\rm Pair}(a)) \equiv {\rm Pair}((b|x)(a))
\]
が成り立つ.
\end{subs}


\noindent{\bf 証明}
~$y$と$z$を, 互いに異なり, 共に$x$と異なり, 
$a$及び$b$の中に自由変数として現れない文字とする.
このとき定義から${\rm Pair}(a)$は$\exists y(\exists z(a = (y, z)))$だから, 
$y$が$x$と異なり, $b$の中に自由変数として現れないことから, 
代入法則 \ref{substquan}により
\[
  (b|x)({\rm Pair}(a)) \equiv \exists y((b|x)(\exists z(a = (y, z))))
\]
が成り立つ.
同様に, $z$が$x$と異なり, $b$の中に自由変数として現れないことから, 
代入法則 \ref{substquan}により
\[
  (b|x)(\exists z(a = (y, z))) \equiv \exists z((b|x)(a = (y, z)))
\]
が成り立つ.
また$x$が$y$とも$z$とも異なることから, 変数法則 \ref{valpair}より
$x$は$(y, z)$の中に自由変数として現れないから, 代入法則 \ref{substfree}, \ref{substfund}により
\[
  (b|x)(a = (y, z)) \equiv (b|x)(a) = (y, z)
\]
が成り立つ.
よってこれらから, 
\[
  (b|x)({\rm Pair}(a)) \equiv \exists y(\exists z((b|x)(a) = (y, z)))
\]
が成り立つ.
ところでいま$y$と$z$は共に$a$及び$b$の中に自由変数として現れないから, 
変数法則 \ref{valsubst}により, $y$と$z$は共に$(b|x)(a)$の中にも
自由変数として現れない.
また$y$と$z$は異なる文字である.
そこで定義から$\exists y(\exists z((b|x)(a) = (y, z)))$は
${\rm Pair}((b|x)(a))$と同じである.
故に$(b|x)({\rm Pair}(a))$と${\rm Pair}((b|x)(a))$は同一の記号列である.
\halmos




\mathstrut
\begin{form}
\label{formbigpair}%構成
$a$を集合とするとき, ${\rm Pair}(a)$は関係式である.
\end{form}


\noindent{\bf 証明}
~$x$と$y$を, 互いに異なり, 共に$a$の中に自由変数として現れない文字とすれば, 
定義から${\rm Pair}(a)$は$\exists x(\exists y(a = (x, y)))$である.
いま構成法則 \ref{formfund}, \ref{formpair}により$(x, y)$は集合となるから, 
これと$a$が集合であることから, 構成法則 \ref{formfund}, \ref{formquan}により
$\exists x(\exists y(a = (x, y)))$が関係式となることがわかる.
故に本法則が成り立つ.
\halmos




\mathstrut
$a$を集合とする.
関係式${\rm Pair}(a)$が定理となるとき, \textbf{${\bm a}$は対である}という.




\mathstrut
\begin{thm}
\label{sthm=bigpaireq}%定理
$a$と$b$を集合とするとき, 
\[
  a = b \to ({\rm Pair}(a) \leftrightarrow {\rm Pair}(b))
\]
が成り立つ.
またこのことから, 次の($*$)が成り立つ: 

($*$) ~~$a = b$が成り立つならば, ${\rm Pair}(a) \leftrightarrow {\rm Pair}(b)$が成り立つ.
\end{thm}


\noindent{\bf 証明}
~$x$を文字とするとき, Thm \ref{thms5eq}より
\[
  a = b \to ((a|x)({\rm Pair}(x)) \leftrightarrow (b|x)({\rm Pair}(x)))
\]
が成り立つが, 代入法則 \ref{substbigpair}によればこの記号列は
\[
  a = b \to ({\rm Pair}(a) \leftrightarrow {\rm Pair}(b))
\]
と一致するから, これが定理となる.
($*$)が成り立つことは, これと推論法則 \ref{dedmp}により明らか.
\halmos




\mathstrut
\begin{thm}
\label{sthmbigpair}%定理
$a$, $b$, $c$を集合とする.
このとき
\[
  a = (b, c) \to {\rm Pair}(a)
\]
が成り立つ.
またこのことから, 次の($*$)が成り立つ: 

($*$) ~~$a = (b, c)$が成り立つならば, $a$は対である.
\end{thm}


\noindent{\bf 証明}
~$x$と$y$を, 互いに異なり, 共に$a$の中に自由変数として現れない文字とする.
また$y$は$b$の中にも自由変数として現れないとする.
このとき代入法則 \ref{substfree}, \ref{substfund}, \ref{substpair}からわかるように, 
$a = (b, c)$は$(c|y)(a = (b, y))$と一致し, 
$a = (b, y)$は$(b|x)(a = (x, y))$と一致するから, 
$a = (b, c)$は$(c|y)((b|x)(a = (x, y)))$と一致する.
いまschema S4の適用により
$(c|y)((b|x)(a = (x, y))) \to \exists y((b|x)(a = (x, y)))$が成り立つから, 
従って
\[
  a = (b, c) \to \exists y((b|x)(a = (x, y)))
\]
が成り立つ.
ここで$y$が$x$と異なり, $b$の中に自由変数として現れないことから, 
代入法則 \ref{substquan}により, 上記の記号列は
\[
\tag{1}
  a = (b, c) \to (b|x)(\exists y(a = (x, y)))
\]
と一致する.
よってこれが定理となる.
また再びschema S4の適用により, 
\[
  (b|x)(\exists y(a = (x, y))) \to \exists x(\exists y(a = (x, y)))
\]
が成り立つが, $x$と$y$が互いに異なり, 共に$a$の中に自由変数として現れないことから, 
この記号列は
\[
\tag{2}
  (b|x)(\exists y(a = (x, y))) \to {\rm Pair}(a)
\]
と同じである.
よってこれが定理となる.
(1), (2)から, 推論法則 \ref{dedmmp}により
$a = (b, c) \to {\rm Pair}(a)$が成り立つ.
($*$)が成り立つことは, これと推論法則 \ref{dedmp}により明らか.
\halmos




\mathstrut
\begin{thm}
\label{sthmbigpairpair}%定理
$a$と$b$を集合とするとき, $(a, b)$は対である.
\end{thm}


\noindent{\bf 証明}
~Thm \ref{x=x}より$(a, b) = (a, b)$が成り立つから, 
定理 \ref{sthmbigpair}により, $(a, b)$は対である.
\halmos




\mathstrut
\begin{thm}
\label{sthmpairsetofa}%定理
$a$, $b$, $c$を集合とし, $x$を$a$の中に自由変数として現れない文字とする.
このとき
\[
  (b, c) \in a \leftrightarrow (b, c) \in \{x \in a|{\rm Pair}(x)\}
\]
が成り立つ.
\end{thm}


\noindent{\bf 証明}
~定理 \ref{sthmbigpairpair}より${\rm Pair}((b, c))$が成り立つから, 
推論法則 \ref{dedawblatrue2}により
\[
  (b, c) \in a \wedge {\rm Pair}((b, c)) \leftrightarrow (b, c) \in a
\]
が成り立ち, これから推論法則 \ref{dedeqch}によって
\[
\tag{1}
  (b, c) \in a \leftrightarrow (b, c) \in a \wedge {\rm Pair}((b, c))
\]
が成り立つ.
また$x$が$a$の中に自由変数として現れないという仮定から, 
定理 \ref{sthmssetbasis}と推論法則 \ref{dedeqch}により
\[
  (b, c) \in a \wedge ((b, c)|x)({\rm Pair}(x)) \leftrightarrow (b, c) \in \{x \in a|{\rm Pair}(x)\}
\]
が成り立つが, 代入法則 \ref{substbigpair}によりこの記号列は
\[
\tag{2}
  (b, c) \in a \wedge {\rm Pair}((b, c)) \leftrightarrow (b, c) \in \{x \in a|{\rm Pair}(x)\}
\]
と一致するから, これが定理となる.
そこで(1), (2)から, 推論法則 \ref{dedeqtrans}によって
\[
  (b, c) \in a \leftrightarrow (b, c) \in \{x \in a|{\rm Pair}(x)\}
\]
が成り立つ.
\halmos




\mathstrut
\begin{defo}
\label{projection}%変形
$a$を記号列とする.
また$x$と$y$, $z$と$w$を, それぞれ互いに異なり, 
いずれも$a$の中に自由変数として現れない文字とする.
このとき
\begin{align*}
  \tau_{x}(\exists y(a = (x, y))) &\equiv \tau_{z}(\exists w(a = (z, w))), \\
  \mbox{} \\
  \tau_{y}(\exists x(a = (x, y))) &\equiv \tau_{w}(\exists z(a = (z, w)))
\end{align*}
が成り立つ.
\end{defo}


\noindent{\bf 証明}
~$u$と$v$を, 互いに異なり, 共に$x$, $y$, $z$, $w$のいずれとも異なり, $a$の
中に自由変数として現れない文字とする.
このとき変数法則 \ref{valfund}, \ref{valquan}, \ref{valpair}からわかるように, 
$u$は$\exists y(a = (x, y))$の中に自由変数として現れず, 
$v$は$\exists x(a = (x, y))$の中に自由変数として現れないから, 
代入法則 \ref{substtautrans}により
\begin{align*}
  \tau_{x}(\exists y(a = (x, y))) &\equiv \tau_{u}((u|x)(\exists y(a = (x, y)))), \\
  \mbox{} \\
  \tau_{y}(\exists x(a = (x, y))) &\equiv \tau_{v}((v|y)(\exists x(a = (x, y))))
\end{align*}
が共に成り立つ.
また$y$が$x$とも$u$とも異なり, $x$が$y$とも$v$とも異なることから, 代入法則 \ref{substquan}により
\begin{align*}
  (u|x)(\exists y(a = (x, y))) &\equiv \exists y((u|x)(a = (x, y))), \\
  \mbox{} \\
  (v|y)(\exists x(a = (x, y))) &\equiv \exists x((v|y)(a = (x, y)))
\end{align*}
が共に成り立つ.
また$x$と$y$が互いに異なり, 共に$a$の中に自由変数として現れないことから, 
代入法則 \ref{substfree}, \ref{substfund}, \ref{substpair}により
\begin{align*}
  (u|x)(a = (x, y)) &\equiv a = (u, y), \\
  \mbox{} \\
  (v|y)(a = (x, y)) &\equiv a = (x, v)
\end{align*}
が共に成り立つ.
よってこれらから, 
\begin{align*}
  \tag{1}
  \tau_{x}(\exists y(a = (x, y))) &\equiv \tau_{u}(\exists y(a = (u, y))), \\
  \mbox{} \\
  \tag{2}
  \tau_{y}(\exists x(a = (x, y))) &\equiv \tau_{v}(\exists x(a = (x, v)))
\end{align*}
が共に成り立つ.
また$u$と$v$が互いに異なり, 共に$x$とも$y$とも異なり, $a$の中に自由変数として現れないことから, 
変数法則 \ref{valfund}, \ref{valpair}により
$v$は$a = (u, y)$の中に自由変数として現れず, 
$u$は$a = (x, v)$の中に自由変数として現れないから, 
代入法則 \ref{substquantrans}により
\begin{align*}
  \exists y(a = (u, y)) &\equiv \exists v((v|y)(a = (u, y))), \\
  \mbox{} \\
  \exists x(a = (x, v)) &\equiv \exists u((u|x)(a = (x, v)))
\end{align*}
が共に成り立つ.
また$x$と$y$が共に$u$とも$v$とも異なり, $a$の中に自由変数として現れないことから, 
代入法則 \ref{substfree}, \ref{substfund}, \ref{substpair}により
\begin{align*}
  (v|y)(a = (u, y)) &\equiv a = (u, v), \\
  \mbox{} \\
  (u|x)(a = (x, v)) &\equiv a = (u, v)
\end{align*}
が共に成り立つ.
よってこれらから, 
\begin{align*}
  \tag{3}
  \tau_{u}(\exists y(a = (u, y))) &\equiv \tau_{u}(\exists v(a = (u, v))), \\
  \mbox{} \\
  \tag{4}
  \tau_{v}(\exists x(a = (x, v))) &\equiv \tau_{v}(\exists u(a = (u, v)))
\end{align*}
が共に成り立つ.
そこで(1)と(3), (2)と(4)から, それぞれ
\begin{align*}
  \tau_{x}(\exists y(a = (x, y))) &\equiv \tau_{u}(\exists v(a = (u, v))), \\
  \mbox{} \\
  \tau_{y}(\exists x(a = (x, y))) &\equiv \tau_{v}(\exists u(a = (u, v)))
\end{align*}
が成り立つことがわかる.
ここまでの議論と全く同様に, 
\begin{align*}
  \tau_{z}(\exists w(a = (z, w))) &\equiv \tau_{u}(\exists v(a = (u, v))), \\
  \mbox{} \\
  \tau_{w}(\exists z(a = (z, w))) &\equiv \tau_{v}(\exists u(a = (u, v)))
\end{align*}
も成り立つ.
故に$\tau_{x}(\exists y(a = (x, y)))$と$\tau_{z}(\exists w(a = (z, w)))$, 
$\tau_{y}(\exists x(a = (x, y)))$と$\tau_{w}(\exists z(a = (z, w)))$は
それぞれ一致する.
\halmos




\mathstrut
\begin{defi}
\label{defpr}%定義
$a$を記号列とする.
また$x$と$y$, $z$と$w$を, それぞれ互いに異なり, 
いずれも$a$の中に自由変数として現れない文字とする.
このとき上記の変形法則 \ref{projection}によれば, 
$\tau_{x}(\exists y(a = (x, y)))$と$\tau_{z}(\exists w(a = (z, w)))$は
同一の記号列であり, 
$\tau_{y}(\exists x(a = (x, y)))$と$\tau_{w}(\exists z(a = (z, w)))$も
同一の記号列である.
$a$に対して定まるこれらの記号列を, それぞれ${\rm pr}_{1}(a)$, ${\rm pr}_{2}(a)$
と書き表す.
${\rm pr}_{1}(a)$を$a$の\textbf{第一座標}(あるいは\textbf{第一射影}), 
${\rm pr}_{2}(a)$を$a$の\textbf{第二座標}(あるいは\textbf{第二射影})という.
\end{defi}




\mathstrut
\begin{valu}
\label{valpr}%変数
$a$を記号列とし, $x$を文字とする.
$x$が$a$の中に自由変数として現れなければ, 
$x$は${\rm pr}_{1}(a)$及び${\rm pr}_{2}(a)$の中に自由変数として現れない.
\end{valu}


\noindent{\bf 証明}
~$y$を$x$と異なり, $a$の中に自由変数として現れない文字とすれば, 定義から
${\rm pr}_{1}(a)$は$\tau_{x}(\exists y(a = (x, y)))$と同じであり, 
${\rm pr}_{2}(a)$は$\tau_{x}(\exists y(a = (y, x)))$と同じである.
変数法則 \ref{valtau}より, $x$はこれらの記号列の中に自由変数として現れない.
\halmos




\mathstrut
\begin{subs}
\label{substpr}%代入
$a$と$b$を記号列とし, $x$を文字とするとき, 
\begin{align*}
  (b|x)({\rm pr}_{1}(a)) &\equiv {\rm pr}_{1}((b|x)(a)), \\
  \mbox{} \\
  (b|x)({\rm pr}_{2}(a)) &\equiv {\rm pr}_{2}((b|x)(a))
\end{align*}
が成り立つ.
\end{subs}


\noindent{\bf 証明}
~$y$と$z$を, 互いに異なり, 共に$x$と異なり, 
$a$及び$b$の中に自由変数として現れない文字とする.
このとき定義から${\rm pr}_{1}(a)$は$\tau_{y}(\exists z(a = (y, z)))$と同じであり, 
${\rm pr}_{2}(a)$は$\tau_{z}(\exists y(a = (y, z)))$と同じだから, 
$y$と$z$が共に$x$と異なり, $b$の中に自由変数として現れないことから, 
代入法則 \ref{substtau}により
\begin{align*}
  (b|x)({\rm pr}_{1}(a)) &\equiv \tau_{y}((b|x)(\exists z(a = (y, z)))), \\
  \mbox{} \\
  (b|x)({\rm pr}_{2}(a)) &\equiv \tau_{z}((b|x)(\exists y(a = (y, z))))
\end{align*}
が共に成り立つ.
同じ根拠から, 代入法則 \ref{substquan}により
\begin{align*}
  (b|x)(\exists z(a = (y, z))) &\equiv \exists z((b|x)(a = (y, z))), \\
  \mbox{} \\
  (b|x)(\exists y(a = (y, z))) &\equiv \exists y((b|x)(a = (y, z)))
\end{align*}
が共に成り立つ.
また$x$が$y$とも$z$とも異なることから, 変数法則 \ref{valpair}により
$x$は$(y, z)$の中に自由変数として現れないから, 代入法則 \ref{substfree}, \ref{substfund}により
\[
  (b|x)(a = (y, z)) \equiv (b|x)(a) = (y, z)
\]
が成り立つ.
よってこれらから, 
\begin{align*}
  (b|x)({\rm pr}_{1}(a)) &\equiv \tau_{y}(\exists z((b|x)(a) = (y, z))), \\
  \mbox{} \\
  (b|x)({\rm pr}_{2}(a)) &\equiv \tau_{z}(\exists y((b|x)(a) = (y, z)))
\end{align*}
が共に成り立つ.
ところでいま$y$と$z$は共に$a$及び$b$の中に自由変数として現れないから, 
変数法則 \ref{valsubst}により, $y$と$z$は共に$(b|x)(a)$の中にも
自由変数として現れない.
また$y$と$z$は異なる文字である.
そこで定義から$\tau_{y}(\exists z((b|x)(a) = (y, z)))$は
${\rm pr}_{1}((b|x)(a))$と同じであり, 
$\tau_{z}(\exists y((b|x)(a) = (y, z)))$は
${\rm pr}_{2}((b|x)(a))$と同じである.
故に$(b|x)({\rm pr}_{1}(a))$と${\rm pr}_{1}((b|x)(a))$, 
$(b|x)({\rm pr}_{2}(a))$と${\rm pr}_{2}((b|x)(a))$はそれぞれ一致する.
\halmos




\mathstrut
\begin{form}
\label{formpr}%構成
$a$が集合ならば, ${\rm pr}_{1}(a)$と${\rm pr}_{2}(a)$は集合である.
\end{form}


\noindent{\bf 証明}
~$x$と$y$を, 互いに異なり, 共に$a$の中に自由変数として現れない文字とすれば, 
定義から${\rm pr}_{1}(a)$は$\tau_{x}(\exists y(a = (x, y)))$であり, 
${\rm pr}_{2}(a)$は$\tau_{y}(\exists x(a = (x, y)))$である.
いま構成法則 \ref{formfund}, \ref{formpair}により$(x, y)$は集合となるから, 
これと$a$が集合であることから, 構成法則 \ref{formfund}, \ref{formquan}により
$\exists y(a = (x, y))$と$\exists x(a = (x, y))$は共に関係式となり, 
従って再び構成法則 \ref{formfund}により, 
$\tau_{x}(\exists y(a = (x, y)))$と$\tau_{y}(\exists x(a = (x, y)))$は
共に集合となる.
故に本法則が成り立つ.
\halmos




\mathstrut
\begin{thm}
\label{sthmpr=}%定理
$a$と$b$を集合とするとき, 
\[
  a = b \to {\rm pr}_{1}(a) = {\rm pr}_{1}(b), ~~
  a = b \to {\rm pr}_{2}(a) = {\rm pr}_{2}(b)
\]
が成り立つ.
またこのことから, 次の($*$)が成り立つ: 

($*$) ~~$a = b$が成り立つならば, ${\rm pr}_{1}(a) = {\rm pr}_{1}(b)$と
        ${\rm pr}_{2}(a) = {\rm pr}_{2}(b)$が共に成り立つ.
\end{thm}


\noindent{\bf 証明}
~$x$を文字とするとき, Thm \ref{T=Ut1TV=UV1}より
\[
  a = b \to (a|x)({\rm pr}_{1}(x)) = (b|x)({\rm pr}_{1}(x)), ~~
  a = b \to (a|x)({\rm pr}_{2}(x)) = (b|x)({\rm pr}_{2}(x))
\]
が共に成り立つが, 代入法則 \ref{substpr}によれば, これらの記号列はそれぞれ
\[
  a = b \to {\rm pr}_{1}(a) = {\rm pr}_{1}(b), ~~
  a = b \to {\rm pr}_{2}(a) = {\rm pr}_{2}(b)
\]
と一致するから, これらが定理となる.
($*$)が成り立つことは, これらと推論法則 \ref{dedmp}から明らか.
\halmos




\mathstrut
\begin{thm}
\label{sthmbigpairpr}%定理
$a$を集合とするとき, 
\[
  {\rm Pair}(a) \leftrightarrow a = ({\rm pr}_{1}(a), {\rm pr}_{2}(a))
\]
が成り立つ.
またこのことから, 特に次の($*$)が成り立つ: 

($*$) ~~$a$が対ならば, $a = ({\rm pr}_{1}(a), {\rm pr}_{2}(a))$が成り立つ.
\end{thm}


\noindent{\bf 証明}
~まず${\rm Pair}(a) \leftrightarrow a = ({\rm pr}_{1}(a), {\rm pr}_{2}(a))$が成り立つことを示す.
推論法則 \ref{dedequiv}があるから, 
${\rm Pair}(a) \to a = ({\rm pr}_{1}(a), {\rm pr}_{2}(a))$と
$a = ({\rm pr}_{1}(a), {\rm pr}_{2}(a)) \to {\rm Pair}(a)$が共に成り立つことを
示せば良いが, この後者は定理 \ref{sthmbigpair}によって成り立つから, 
前者が成り立つことのみ示せば良い.
いま$x$と$y$を, 互いに異なり, 共に$a$の中に自由変数として現れない文字とすれば, 
定義から${\rm Pair}(a)$は$\exists x(\exists y(a = (x, y)))$と同じであり, 
これは$(\tau_{x}(\exists y(a = (x, y)))|x)(\exists y(a = (x, y)))$と同じである.
また$\tau_{x}(\exists y(a = (x, y)))$は${\rm pr}_{1}(a)$と同じである.
よって${\rm Pair}(a)$は
\[
  ({\rm pr}_{1}(a)|x)(\exists y(a = (x, y)))
\]
と同じである.
ここで$y$が$a$の中に自由変数として現れないことから, 変数法則 \ref{valpr}により
$y$は${\rm pr}_{1}(a)$の中に自由変数として現れない.
また$x$と$y$は異なる文字である.
そこで代入法則 \ref{substquan}により, 上記の記号列は
\[
  \exists y(({\rm pr}_{1}(a)|x)(a = (x, y)))
\]
と一致する.
また$x$が$y$と異なり, $a$の中に自由変数として現れないことから, 
代入法則 \ref{substfree}, \ref{substfund}, \ref{substpair}により, 
この記号列は
\[
  \exists y(a = ({\rm pr}_{1}(a), y))
\]
と一致する.
またいま$\tau_{y}(a = ({\rm pr}_{1}(a), y))$を$T$と書けば, $T$は集合であり, 
定義から上記の記号列は
\[
  (T|y)(a = ({\rm pr}_{1}(a), y))
\]
である.
ここで$y$が$a$の中に自由変数として現れず, 上述のように${\rm pr}_{1}(a)$の中にも
自由変数として現れないことから, 代入法則 \ref{substfree}, \ref{substfund}, \ref{substpair}により, 
この記号列は
\[
  a = ({\rm pr}_{1}(a), T)
\]
と一致する.
以上のことから, 
\[
\tag{1}
  {\rm Pair}(a) \equiv a = ({\rm pr}_{1}(a), T)
\]
が成り立つことがわかる.
同様に, $\exists y(\exists x(a = (x, y)))$は
$(\tau_{y}(\exists x(a = (x, y)))|y)(\exists x(a = (x, y)))$と同じであり, 
$x$と$y$に対する仮定から, $\tau_{y}(\exists x(a = (x, y)))$は${\rm pr}_{2}(a)$と同じであるから, 
$\exists y(\exists x(a = (x, y)))$は
\[
  ({\rm pr}_{2}(a)|y)(\exists x(a = (x, y)))
\]
と同じである.
ここで$x$が$a$の中に自由変数として現れないことから, 変数法則 \ref{valpr}により
$x$は${\rm pr}_{2}(a)$の中にも自由変数として現れない.
また$x$と$y$は異なる文字である.
そこで代入法則 \ref{substquan}により, 上記の記号列は
\[
  \exists x(({\rm pr}_{2}(a)|y)(a = (x, y)))
\]
と一致する.
また$y$が$x$と異なり, $a$の中に自由変数として現れないことから, 
代入法則 \ref{substfree}, \ref{substfund}, \ref{substpair}により, 
この記号列は
\[
  \exists x(a = (x, {\rm pr}_{2}(a)))
\]
と一致する.
またいま$\tau_{x}(a = (x, {\rm pr}_{2}(a)))$を$U$と書けば, $U$は集合であり, 
定義から上記の記号列は
\[
  (U|x)(a = (x, {\rm pr}_{2}(a)))
\]
である.
ここで$x$が$a$の中に自由変数として現れず, 上述のように${\rm pr}_{2}(a)$の中にも
自由変数として現れないことから, 代入法則 \ref{substfree}, \ref{substfund}, \ref{substpair}により, 
この記号列は
\[
  a = (U, {\rm pr}_{2}(a))
\]
と一致する.
以上のことから, 
\[
\tag{2}
  \exists y(\exists x(a = (x, y))) \equiv a = (U, {\rm pr}_{2}(a))
\]
が成り立つことがわかる.
さていまThm \ref{thmexch}と推論法則 \ref{dedequiv}により
\[
  \exists x(\exists y(a = (x, y))) \to \exists y(\exists x(a = (x, y)))
\]
が成り立つが, はじめに述べたように$\exists x(\exists y(a = (x, y)))$は${\rm Pair}(a)$と同じであり, 
また(2)が成り立つから, この記号列は
\[
\tag{3}
  {\rm Pair}(a) \to a = (U, {\rm pr}_{2}(a))
\]
と一致し, これが定理となる.
またThm \ref{x=yty=x}より
\[
  a = ({\rm pr}_{1}(a), T) \to ({\rm pr}_{1}(a), T) = a
\]
が成り立つが, (1)よりこの記号列は
\[
\tag{4}
  {\rm Pair}(a) \to ({\rm pr}_{1}(a), T) = a
\]
と一致するから, これが定理となる.
そこで(3), (4)から, 推論法則 \ref{dedprewedge}によって
\[
\tag{5}
  {\rm Pair}(a) \to ({\rm pr}_{1}(a), T) = a \wedge a = (U, {\rm pr}_{2}(a))
\]
が成り立つ.
またThm \ref{x=ywy=ztx=z}より
\[
\tag{6}
  ({\rm pr}_{1}(a), T) = a \wedge a = (U, {\rm pr}_{2}(a)) \to ({\rm pr}_{1}(a), T) = (U, {\rm pr}_{2}(a))
\]
が成り立つ.
また定理 \ref{sthmpair}と推論法則 \ref{dedequiv}により
\[
\tag{7}
  ({\rm pr}_{1}(a), T) = (U, {\rm pr}_{2}(a)) \to {\rm pr}_{1}(a) = U \wedge T = {\rm pr}_{2}(a)
\]
が成り立つ.
またThm \ref{awbta}より
\[
\tag{8}
  {\rm pr}_{1}(a) = U \wedge T = {\rm pr}_{2}(a) \to T = {\rm pr}_{2}(a)
\]
が成り立つ.
また定理 \ref{sthmpairweak}と推論法則 \ref{dedequiv}により
\[
\tag{9}
  T = {\rm pr}_{2}(a) \to ({\rm pr}_{1}(a), T) = ({\rm pr}_{1}(a), {\rm pr}_{2}(a))
\]
が成り立つ.
そこで(5)---(9)から, 推論法則 \ref{dedmmp}によって
\[
\tag{10}
  {\rm Pair}(a) \to ({\rm pr}_{1}(a), T) = ({\rm pr}_{1}(a), {\rm pr}_{2}(a))
\]
が成り立つ.
いま(1)が成り立つことに注意すると, この(10)から, 推論法則 \ref{dedatawbtrue1}によって
\[
\tag{11}
  {\rm Pair}(a) \to a = ({\rm pr}_{1}(a), T) \wedge ({\rm pr}_{1}(a), T) = ({\rm pr}_{1}(a), {\rm pr}_{2}(a))
\]
が成り立つことがわかる.
またThm \ref{x=ywy=ztx=z}より
\[
\tag{12}
  a = ({\rm pr}_{1}(a), T) \wedge ({\rm pr}_{1}(a), T) = ({\rm pr}_{1}(a), {\rm pr}_{2}(a)) \to a = ({\rm pr}_{1}(a), {\rm pr}_{2}(a))
\]
が成り立つ.
そこで(11), (12)から, 推論法則 \ref{dedmmp}によって
\[
\tag{13}
  {\rm Pair}(a) \to a = ({\rm pr}_{1}(a), {\rm pr}_{2}(a))
\]
が成り立つ.

さていま$a$が対であるとすると, これと(13)から, 推論法則 \ref{dedmp}によって
$a = ({\rm pr}_{1}(a), {\rm pr}_{2}(a))$が成り立つ.
故に($*$)が成り立つ.
\halmos




\mathstrut
\begin{thm}
\label{sthmprpair}%定理
$a$と$b$を集合とするとき, 
\[
  {\rm pr}_{1}((a, b)) = a, ~~
  {\rm pr}_{2}((a, b)) = b
\]
が成り立つ.
\end{thm}


\noindent{\bf 証明}
~定理 \ref{sthmbigpairpair}より$(a, b)$は対だから, 
定理 \ref{sthmbigpairpr}により
$(a, b) = ({\rm pr}_{1}((a, b)), {\rm pr}_{2}((a, b)))$が成り立つ.
そこで定理 \ref{sthmpair}により
$a = {\rm pr}_{1}((a, b))$と$b = {\rm pr}_{2}((a, b))$が共に成り立ち, 
これらから, 推論法則 \ref{ded=ch}によって
${\rm pr}_{1}((a, b)) = a$と${\rm pr}_{2}((a, b)) = b$が共に成り立つ.
\halmos




\mathstrut
\begin{thm}
\label{sthmpairpreq}%定理
$a$, $b$, $c$を集合とするとき, 
\[
  a = (b, c) \leftrightarrow {\rm Pair}(a) \wedge (b = {\rm pr}_{1}(a) \wedge c = {\rm pr}_{2}(a))
\]
が成り立つ.
\end{thm}


\noindent{\bf 証明}
~推論法則 \ref{dedequiv}があるから, 
\begin{align*}
  \tag{1}
  &a = (b, c) \to {\rm Pair}(a) \wedge (b = {\rm pr}_{1}(a) \wedge c = {\rm pr}_{2}(a)), \\
  \mbox{} \\
  \tag{2}
  &{\rm Pair}(a) \wedge (b = {\rm pr}_{1}(a) \wedge c = {\rm pr}_{2}(a)) \to a = (b, c)
\end{align*}
が共に成り立つことを示せば良い.

(1)の証明: 
Thm \ref{x=yty=x}より
\[
\tag{3}
  a = (b, c) \to (b, c) = a
\]
が成り立つ.
また定理 \ref{sthmbigpair}より
\[
\tag{4}
  a = (b, c) \to {\rm Pair}(a)
\]
が成り立ち, 定理 \ref{sthmbigpairpr}と推論法則 \ref{dedequiv}により
\[
\tag{5}
  {\rm Pair}(a) \to a = ({\rm pr}_{1}(a), {\rm pr}_{2}(a))
\]
が成り立つから, (4), (5)から, 推論法則 \ref{dedmmp}によって
\[
\tag{6}
  a = (b, c) \to a = ({\rm pr}_{1}(a), {\rm pr}_{2}(a))
\]
が成り立つ.
そこで(3), (6)から, 推論法則 \ref{dedprewedge}によって
\[
\tag{7}
  a = (b, c) \to (b, c) = a \wedge a = ({\rm pr}_{1}(a), {\rm pr}_{2}(a))
\]
が成り立つ.
またThm \ref{x=ywy=ztx=z}より
\[
\tag{8}
  (b, c) = a \wedge a = ({\rm pr}_{1}(a), {\rm pr}_{2}(a)) \to (b, c) = ({\rm pr}_{1}(a), {\rm pr}_{2}(a))
\]
が成り立ち, 定理 \ref{sthmpair}と推論法則 \ref{dedequiv}により
\[
\tag{9}
  (b, c) = ({\rm pr}_{1}(a), {\rm pr}_{2}(a)) \to b = {\rm pr}_{1}(a) \wedge c = {\rm pr}_{2}(a)
\]
が成り立つから, (7), (8), (9)から, 推論法則 \ref{dedmmp}によって
\[
\tag{10}
  a = (b, c) \to b = {\rm pr}_{1}(a) \wedge c = {\rm pr}_{2}(a)
\]
が成り立つ.
そこで(4), (10)から, 推論法則 \ref{dedprewedge}によって
(1)が成り立つ.

(2)の証明: 
定理 \ref{sthmpair}と推論法則 \ref{dedequiv}により
\[
  b = {\rm pr}_{1}(a) \wedge c = {\rm pr}_{2}(a) \to (b, c) = ({\rm pr}_{1}(a), {\rm pr}_{2}(a))
\]
が成り立ち, Thm \ref{x=yty=x}より
\[
  (b, c) = ({\rm pr}_{1}(a), {\rm pr}_{2}(a)) \to ({\rm pr}_{1}(a), {\rm pr}_{2}(a)) = (b, c)
\]
が成り立つから, 推論法則 \ref{dedmmp}によって
\[
  b = {\rm pr}_{1}(a) \wedge c = {\rm pr}_{2}(a) \to ({\rm pr}_{1}(a), {\rm pr}_{2}(a)) = (b, c)
\]
が成り立ち, これと(5)から, 推論法則 \ref{dedfromaddw}によって
\[
\tag{11}
  {\rm Pair}(a) \wedge (b = {\rm pr}_{1}(a) \wedge c = {\rm pr}_{2}(a)) \to 
  a = ({\rm pr}_{1}(a), {\rm pr}_{2}(a)) \wedge ({\rm pr}_{1}(a), {\rm pr}_{2}(a)) = (b, c)
\]
が成り立つ.
またThm \ref{x=ywy=ztx=z}より
\[
\tag{12}
  a = ({\rm pr}_{1}(a), {\rm pr}_{2}(a)) \wedge ({\rm pr}_{1}(a), {\rm pr}_{2}(a)) = (b, c) \to a = (b, c)
\]
が成り立つ.
そこで(11), (12)から, 推論法則 \ref{dedmmp}によって(2)が成り立つ.
\halmos
%section6確認済
%まだ定理追加するかも?



\newpage
\setcounter{defi}{0}
\section{積}%%%%%%%%%%%%%%%%%%%%%%%%%%%%%%%%%%%%%%%%%%%%%%%%%%%%%%%%%%%%%%%%%%%%




この節では, 二つの集合の積を定義し, その性質について述べる.
これによって, グラフ及び函数という概念を導入することができるようになる(\S 8 -- 11).




\mathstrut
\begin{defo}
\label{product}%変形
$a$と$b$を記号列とする.
また$x$, $y$, $z$を, どの二つも互いに異なり, 
どの一つも$a$及び$b$の中に
自由変数として現れない文字とする.
同様に$u$, $v$, $w$を, どの二つも互いに異なり, 
どの一つも$a$及び$b$の中に
自由変数として現れない文字とする.
このとき
\[
  \{z|\exists x(\exists y((x \in a \wedge y \in b) \wedge z = (x, y)))\} \equiv 
  \{w|\exists u(\exists v((u \in a \wedge v \in b) \wedge w = (u, v)))\}
\]
が成り立つ.
\end{defo}


\noindent{\bf 証明}
~$p$, $q$, $r$を, どの二つも互いに異なり, どの一つも
$x$, $y$, $z$, $u$, $v$, $w$のいずれとも異なり, 
$a$及び$b$の中に自由変数として現れない文字とする.
このとき変数法則 \ref{valfund}, \ref{valwedge}, \ref{valquan}, \ref{valpair}からわかるように, 
$r$は$\exists x(\exists y((x \in a \wedge y \in b) \wedge z = (x, y)))$の中に
自由変数として現れないから, 代入法則 \ref{substisettrans}により
\[
\tag{1}
  \{z|\exists x(\exists y((x \in a \wedge y \in b) \wedge z = (x, y)))\} \equiv 
  \{r|(r|z)(\exists x(\exists y((x \in a \wedge y \in b) \wedge z = (x, y))))\}
\]
が成り立つ.
また$x$と$y$が共に$z$とも$r$とも異なることから, 代入法則 \ref{substquan}により
\[
\tag{2}
  (r|z)(\exists x(\exists y((x \in a \wedge y \in b) \wedge z = (x, y)))) \equiv 
  \exists x(\exists y((r|z)((x \in a \wedge y \in b) \wedge z = (x, y))))
\]
が成り立つ.
また$z$が$x$とも$y$とも異なり, $a$及び$b$の中に自由変数として現れないことから, 
代入法則 \ref{substfree}, \ref{substfund}, \ref{substwedge}, \ref{substpair}により
\[
\tag{3}
  (r|z)((x \in a \wedge y \in b) \wedge z = (x, y)) \equiv 
  (x \in a \wedge y \in b) \wedge r = (x, y)
\]
が成り立つ.
よって(1), (2), (3)から, 
\[
\tag{4}
  \{z|\exists x(\exists y((x \in a \wedge y \in b) \wedge z = (x, y)))\} \equiv 
  \{r|\exists x(\exists y((x \in a \wedge y \in b) \wedge r = (x, y)))\}
\]
が成り立つことがわかる.
また$p$が$x$, $y$, $r$のいずれとも異なり, $a$及び$b$の中に自由変数として現れないことから, 
変数法則 \ref{valfund}, \ref{valwedge}, \ref{valquan}, \ref{valpair}によって
$p$が$\exists y((x \in a \wedge y \in b) \wedge r = (x, y))$の中に
自由変数として現れないことがわかるから, 
代入法則 \ref{substquantrans}により
\[
\tag{5}
  \exists x(\exists y((x \in a \wedge y \in b) \wedge r = (x, y))) \equiv 
  \exists p((p|x)(\exists y((x \in a \wedge y \in b) \wedge r = (x, y))))
\]
が成り立つ.
また$y$が$x$とも$p$とも異なることから, 代入法則 \ref{substquan}により
\[
\tag{6}
  (p|x)(\exists y((x \in a \wedge y \in b) \wedge r = (x, y))) \equiv 
  \exists y((p|x)((x \in a \wedge y \in b) \wedge r = (x, y)))
\]
が成り立つ.
また$x$が$y$とも$r$とも異なり, $a$及び$b$の中に自由変数として現れないことから, 
代入法則 \ref{substfree}, \ref{substfund}, \ref{substwedge}, \ref{substpair}により
\[
\tag{7}
  (p|x)((x \in a \wedge y \in b) \wedge r = (x, y)) \equiv 
  (p \in a \wedge y \in b) \wedge r = (p, y)
\]
が成り立つ.
よって(5), (6), (7)から, 
\[
\tag{8}
  \exists x(\exists y((x \in a \wedge y \in b) \wedge r = (x, y))) \equiv 
  \exists p(\exists y((p \in a \wedge y \in b) \wedge r = (p, y)))
\]
が成り立つことがわかる.
また$q$が$y$, $p$, $r$のいずれとも異なり, $a$及び$b$の中に自由変数として現れないことから, 
変数法則 \ref{valfund}, \ref{valwedge}, \ref{valpair}によって
$q$が$(p \in a \wedge y \in b) \wedge r = (p, y)$の中に
自由変数として現れないことがわかるから, 
代入法則 \ref{substquantrans}により
\[
\tag{9}
  \exists y((p \in a \wedge y \in b) \wedge r = (p, y)) \equiv 
  \exists q((q|y)((p \in a \wedge y \in b) \wedge r = (p, y)))
\]
が成り立つ.
また$y$が$p$とも$r$とも異なり, $a$及び$b$の中に自由変数として現れないことから, 
代入法則 \ref{substfree}, \ref{substfund}, \ref{substwedge}, \ref{substpair}により
\[
\tag{10}
  (q|y)((p \in a \wedge y \in b) \wedge r = (p, y)) \equiv 
  (p \in a \wedge q \in b) \wedge r = (p, q)
\]
が成り立つ.
よって(9), (10)から, 
\[
\tag{11}
  \exists y((p \in a \wedge y \in b) \wedge r = (p, y)) \equiv 
  \exists q((p \in a \wedge q \in b) \wedge r = (p, q))
\]
が成り立つことがわかる.
そこで(4), (8), (11)から, 
\[
  \{z|\exists x(\exists y((x \in a \wedge y \in b) \wedge z = (x, y)))\} \equiv 
  \{r|\exists p(\exists q((p \in a \wedge q \in b) \wedge r = (p, q)))\}
\]
が成り立つ.
ここまでの議論と全く同様にして
\[
  \{w|\exists u(\exists v((u \in a \wedge v \in b) \wedge w = (u, v)))\} \equiv 
  \{r|\exists p(\exists q((p \in a \wedge q \in b) \wedge r = (p, q)))\}
\]
も成り立つから, 従って$\{z|\exists x(\exists y((x \in a \wedge y \in b) \wedge z = (x, y)))\}$と
$\{w|\exists u(\exists v((u \in a \wedge v \in b) \wedge w = (u, v)))\}$は一致する.
\halmos




\mathstrut
\begin{defi}
\label{defproduct}%定義
$a$と$b$を記号列とする.
また$x$, $y$, $z$を, どの二つも互いに異なり, 
どの一つも$a$及び$b$の中に
自由変数として現れない文字とする.
同様に$u$, $v$, $w$を, どの二つも互いに異なり, 
どの一つも$a$及び$b$の中に
自由変数として現れない文字とする.
このとき上記の変形法則 \ref{product}によれば, 
$\{z|\exists x(\exists y((x \in a \wedge y \in b) \wedge z = (x, y)))\}$と
$\{w|\exists u(\exists v((u \in a \wedge v \in b) \wedge w = (u, v)))\}$という
二つの記号列は一致する.
$a$と$b$に対して定まるこの記号列を, 
$a$と$b$の\textbf{直積}(direct product)
または単に\textbf{積}(product)といい, $(a) \times (b)$と書き表す.
括弧は適宜省略する.
\end{defi}




\mathstrut
\begin{valu}
\label{valproduct}%変数
$a$と$b$を記号列とし, $x$を文字とする.
$x$が$a$及び$b$の中に自由変数として現れなければ, 
$x$は$a \times b$の中に自由変数として現れない.
\end{valu}


\noindent{\bf 証明}
~このとき$u$と$v$を, 互いに異なり, 共に$x$と異なり, 
$a$及び$b$の中に自由変数として現れない文字とすれば, 
定義から$a \times b$は
$\{x|\exists u(\exists v((u \in a \wedge v \in b) \wedge x = (u, v)))\}$と同じである.
変数法則 \ref{valiset}より, $x$はこの中に自由変数として現れない.
\halmos




\mathstrut
\begin{subs}
\label{substproduct}%代入
$a$, $b$, $c$を記号列とし, $x$を文字とするとき, 
\[
  (c|x)(a \times b) \equiv (c|x)(a) \times (c|x)(b)
\]
が成り立つ.
\end{subs}


\noindent{\bf 証明}
~$u$, $v$, $w$を, どの二つも互いに異なり, 
どの一つも$x$と異なり, $a$, $b$, $c$のいずれの記号列の中にも
自由変数として現れない文字とする.
このとき定義から, $a \times b$は
$\{w|\exists u(\exists v((u \in a \wedge v \in b) \wedge w = (u, v)))\}$と同じである.
このことと, $w$が$x$と異なり, $c$の中に自由変数として現れないことから, 
代入法則 \ref{substiset}により
\[
  (c|x)(a \times b) \equiv 
  \{w|(c|x)(\exists u(\exists v((u \in a \wedge v \in b) \wedge w = (u, v))))\}
\]
が成り立つ.
また$u$と$v$が共に$x$と異なり, $c$の中に自由変数として現れないことから, 
代入法則 \ref{substquan}により
\[
  (c|x)(\exists u(\exists v((u \in a \wedge v \in b) \wedge w = (u, v)))) \equiv 
  \exists u(\exists v((c|x)((u \in a \wedge v \in b) \wedge w = (u, v))))
\]
が成り立つ.
また$x$が$u$, $v$, $w$のいずれとも異なることと
代入法則 \ref{substfund}, \ref{substwedge}, \ref{substpair}により, 
\[
  (c|x)((u \in a \wedge v \in b) \wedge w = (u, v)) \equiv 
  (u \in (c|x)(a) \wedge v \in (c|x)(b)) \wedge w = (u, v)
\]
が成り立つ.
よってこれらから, $(c|x)(a \times b)$が
\[
\tag{$*$}
  \{w|\exists u(\exists v((u \in (c|x)(a) \wedge v \in (c|x)(b)) \wedge w = (u, v)))\}
\]
と一致することがわかる.
いま$u$, $v$, $w$は$a$, $b$, $c$のいずれの記号列の中にも自由変数として現れないから, 
変数法則 \ref{valsubst}より, $u$, $v$, $w$は$(c|x)(a)$及び$(c|x)(b)$の中に自由変数として現れない.
また$u$, $v$, $w$はどの二つも互いに異なる.
よって定義から, 上記の記号列($*$)は$(c|x)(a) \times (c|x)(b)$と同じである.
故に本法則が成り立つ.
\halmos




\mathstrut
\begin{form}
\label{formproduct}%構成
$a$と$b$が集合ならば, $a \times b$は集合である.
\end{form}


\noindent{\bf 証明}
~$x$, $y$, $z$を, どの二つも互いに異なり, 
どの一つも$a$及び$b$の中に
自由変数として現れない文字とする.
このとき定義から, $a \times b$は
$\{z|\exists x(\exists y((x \in a \wedge y \in b) \wedge z = (x, y)))\}$と同じである.
$a$と$b$が集合のとき, 
構成法則 \ref{formfund}, \ref{formwedge}, \ref{formquan}, \ref{formiset}, \ref{formpair}から
直ちにわかるように, これは集合である.
\halmos




\mathstrut
\begin{thm}
\label{sthmproductsetmake}%定理
$a$と$b$を集合とする.
また$x$, $y$, $z$を, どの二つも互いに異なり, どの一つも
$a$及び$b$の中に自由変数として現れない文字とする.
このとき, 関係式
$\exists x(\exists y((x \in a \wedge y \in b) \wedge z = (x, y)))$は
$z$について集合を作り得る.
\end{thm}




これを示すために, まず次の補題 \ref{lemma1}, 補題 \ref{lemma2}を証明する.




\mathstrut
\begin{lemma}
\label{lemma1}%補題
$a$と$b$を集合とする.
また$x$, $y$, $z$を, どの二つも互いに異なり, 
どの一つも$a$及び$b$の中に自由変数として現れない文字とする.
このとき
\[
  \forall z(\exists x(\exists y((x \in a \wedge y \in b) \wedge z = (x, y))) \leftrightarrow 
  \exists y(y \in b \wedge \exists x(x \in a \wedge z = (x, y))))
\]
が成り立つ.
\end{lemma}


\noindent{\bf 証明}
~$u$, $v$, $w$を, どの二つも互いに異なり, 
どの一つも$x$, $y$, $z$のいずれとも異なり, 
$a$及び$b$の中に自由変数として現れない, 定数でない文字とする.
このときThm \ref{awblbwa}より
$u \in a \wedge v \in b \leftrightarrow v \in b \wedge u \in a$が成り立つから, 
推論法則 \ref{dedaddeqw}により
\[
\tag{1}
  (u \in a \wedge v \in b) \wedge w = (u, v) \leftrightarrow (v \in b \wedge u \in a) \wedge w = (u, v)
\]
が成り立つ.
またThm \ref{1awb1wclaw1bwc1}より
\[
\tag{2}
  (v \in b \wedge u \in a) \wedge w = (u, v) \leftrightarrow v \in b \wedge (u \in a \wedge w = (u, v))
\]
が成り立つ.
そこで(1), (2)から, 推論法則 \ref{dedeqtrans}によって
\[
  (u \in a \wedge v \in b) \wedge w = (u, v) \leftrightarrow v \in b \wedge (u \in a \wedge w = (u, v))
\]
が成り立ち, これと$u$が定数でないことから, 推論法則 \ref{dedalleqquansepconst}によって
\[
\tag{3}
  \exists u((u \in a \wedge v \in b) \wedge w = (u, v)) \leftrightarrow 
  \exists u(v \in b \wedge (u \in a \wedge w = (u, v)))
\]
が成り立つ.
ここで$x$が$u$, $v$, $w$のいずれとも異なり, $a$及び$b$の中に自由変数として現れないことから, 
変数法則 \ref{valfund}, \ref{valwedge}, \ref{valpair}によって
$x$が$(u \in a \wedge v \in b) \wedge w = (u, v)$の中にも
$v \in b \wedge (u \in a \wedge w = (u, v))$の中にも自由変数として現れないことがわかるから, 
代入法則 \ref{substquantrans}により, 
\begin{align*}
  \exists u((u \in a \wedge v \in b) \wedge w = (u, v)) &\equiv \exists x((x|u)((u \in a \wedge v \in b) \wedge w = (u, v))), \\
  \mbox{} \\
  \exists u(v \in b \wedge (u \in a \wedge w = (u, v))) &\equiv \exists x((x|u)(v \in b \wedge (u \in a \wedge w = (u, v))))
\end{align*}
が成り立つ.
また$u$が$v$とも$w$とも異なり, $a$及び$b$の中に自由変数として現れないことから, 
代入法則 \ref{substfree}, \ref{substfund}, \ref{substwedge}, \ref{substpair}により, 
\begin{align*}
  (x|u)((u \in a \wedge v \in b) \wedge w = (u, v)) &\equiv (x \in a \wedge v \in b) \wedge w = (x, v), \\
  \mbox{} \\
  (x|u)(v \in b \wedge (u \in a \wedge w = (u, v))) &\equiv v \in b \wedge (x \in a \wedge w = (x, v))
\end{align*}
が成り立つ.
そこでこれらから, (3)が
\[
\tag{4}
  \exists x((x \in a \wedge v \in b) \wedge w = (x, v)) \leftrightarrow 
  \exists x(v \in b \wedge (x \in a \wedge w = (x, v)))
\]
と一致することがわかり, これが定理となる.
また$x$が$v$と異なり, $b$の中に自由変数として現れないことから, 
変数法則 \ref{valfund}により$x$は$v \in b$の中に自由変数として現れないから, 
Thm \ref{thmexwrfree}より
\[
\tag{5}
  \exists x(v \in b \wedge (x \in a \wedge w = (x, v))) \leftrightarrow 
  v \in b \wedge \exists x(x \in a \wedge w = (x, v))
\]
が成り立つ.
そこで(4), (5)から, 推論法則 \ref{dedeqtrans}によって
\[
  \exists x((x \in a \wedge v \in b) \wedge w = (x, v)) \leftrightarrow 
  v \in b \wedge \exists x(x \in a \wedge w = (x, v))
\]
が成り立ち, これと$v$が定数でないことから, 推論法則 \ref{dedalleqquansepconst}によって
\[
\tag{6}
  \exists v(\exists x((x \in a \wedge v \in b) \wedge w = (x, v))) \leftrightarrow 
  \exists v(v \in b \wedge \exists x(x \in a \wedge w = (x, v)))
\]
が成り立つ.
ここで$y$が$x$, $v$, $w$のいずれとも異なり, $a$及び$b$の中に自由変数として現れないことから, 
変数法則 \ref{valfund}, \ref{valwedge}, \ref{valquan}, \ref{valpair}によって
$y$が$\exists x((x \in a \wedge v \in b) \wedge w = (x, v))$の中にも
$v \in b \wedge \exists x(x \in a \wedge w = (x, v))$の中にも自由変数として現れないことがわかるから, 
代入法則 \ref{substquantrans}により, 
\begin{align*}
  \exists v(\exists x((x \in a \wedge v \in b) \wedge w = (x, v))) &\equiv \exists y((y|v)(\exists x((x \in a \wedge v \in b) \wedge w = (x, v)))), \\
  \mbox{} \\
  \exists v(v \in b \wedge \exists x(x \in a \wedge w = (x, v))) &\equiv \exists y((y|v)(v \in b \wedge \exists x(x \in a \wedge w = (x, v))))
\end{align*}
が成り立つ.
また$v$が$b$の中に自由変数として現れないことから, 
代入法則 \ref{substfree}, \ref{substfund}, \ref{substwedge}により, 
\[
  (y|v)(v \in b \wedge \exists x(x \in a \wedge w = (x, v))) \equiv y \in b \wedge (y|v)(\exists x(x \in a \wedge w = (x, v)))
\]
が成り立つ.
また$x$が$y$とも$v$とも異なることから, 代入法則 \ref{substquan}により, 
\begin{align*}
  (y|v)(\exists x((x \in a \wedge v \in b) \wedge w = (x, v))) &\equiv \exists x((y|v)((x \in a \wedge v \in b) \wedge w = (x, v))), \\
  \mbox{} \\
  (y|v)(\exists x(x \in a \wedge w = (x, v))) &\equiv \exists x((y|v)(x \in a \wedge w = (x, v)))
\end{align*}
が成り立つ.
また$v$が$x$とも$w$とも異なり, $a$及び$b$の中に自由変数として現れないことから, 
代入法則 \ref{substfree}, \ref{substfund}, \ref{substwedge}, \ref{substpair}により, 
\begin{align*}
  (y|v)((x \in a \wedge v \in b) \wedge w = (x, v)) &\equiv (x \in a \wedge y \in b) \wedge w = (x, y), \\
  \mbox{} \\
  (y|v)(x \in a \wedge w = (x, v)) &\equiv x \in a \wedge w = (x, y)
\end{align*}
が成り立つ.
そこでこれらから, (6)が
\[
\tag{7}
  \exists y(\exists x((x \in a \wedge y \in b) \wedge w = (x, y))) \leftrightarrow 
  \exists y(y \in b \wedge \exists x(x \in a \wedge w = (x, y)))
\]
と一致することがわかり, これが定理となる.
またThm \ref{thmexch}より
\[
\tag{8}
  \exists x(\exists y((x \in a \wedge y \in b) \wedge w = (x, y))) \leftrightarrow 
  \exists y(\exists x((x \in a \wedge y \in b) \wedge w = (x, y)))
\]
が成り立つ.
そこで(7), (8)から, 推論法則 \ref{dedeqtrans}によって
\[
\tag{9}
  \exists x(\exists y((x \in a \wedge y \in b) \wedge w = (x, y))) \leftrightarrow 
  \exists y(y \in b \wedge \exists x(x \in a \wedge w = (x, y)))
\]
が成り立ち, これと$w$が定数でないことから, 推論法則 \ref{dedltthmquan}によって
\[
\tag{10}
  \forall w(\exists x(\exists y((x \in a \wedge y \in b) \wedge w = (x, y))) \leftrightarrow 
  \exists y(y \in b \wedge \exists x(x \in a \wedge w = (x, y))))
\]
が成り立つ.
ここで$z$が$x$, $y$, $w$のいずれとも異なり, $a$及び$b$の中に自由変数として現れないことから, 
変数法則 \ref{valfund}, \ref{valwedge}, \ref{valequiv}, \ref{valquan}, \ref{valpair}によって
$z$が(9)の中に自由変数として現れないことがわかるから, 代入法則 \ref{substquantrans}により, 
(10)は
\[
  \forall z((z|w)(\exists x(\exists y((x \in a \wedge y \in b) \wedge w = (x, y))) \leftrightarrow 
  \exists y(y \in b \wedge \exists x(x \in a \wedge w = (x, y)))))
\]
と一致する.
また代入法則 \ref{substequiv}により, 
この記号列は
\[
  \forall z((z|w)(\exists x(\exists y((x \in a \wedge y \in b) \wedge w = (x, y)))) \leftrightarrow 
  (z|w)(\exists y(y \in b \wedge \exists x(x \in a \wedge w = (x, y)))))
\]
と一致する.
また$x$と$y$が共に$z$とも$w$とも異なることから, 代入法則 \ref{substquan}により, 
この記号列は
\[
\tag{11}
  \forall z(\exists x(\exists y((z|w)((x \in a \wedge y \in b) \wedge w = (x, y)))) \leftrightarrow 
  \exists y((z|w)(y \in b \wedge \exists x(x \in a \wedge w = (x, y)))))
\]
と一致する.
そこでこれが(10)と一致し, 定理となる.
いま$w$は$x$とも$y$とも異なり, $a$及び$b$の中に自由変数として現れないから, 
代入法則 \ref{substfree}, \ref{substfund}, \ref{substwedge}, \ref{substpair}により, 
\[
  (z|w)((x \in a \wedge y \in b) \wedge w = (x, y)) \equiv 
  (x \in a \wedge y \in b) \wedge z = (x, y)
\]
が成り立つ.
また$w$が$y$と異なり, $b$の中に自由変数として現れないことから, 
代入法則 \ref{substfree}, \ref{substfund}, \ref{substwedge}により, 
\[
  (z|w)(y \in b \wedge \exists x(x \in a \wedge w = (x, y))) \equiv 
  y \in b \wedge (z|w)(\exists x(x \in a \wedge w = (x, y)))
\]
が成り立つ.
また$x$が$z$とも$w$とも異なることから, 代入法則 \ref{substquan}により, 
\[
  (z|w)(\exists x(x \in a \wedge w = (x, y))) \equiv 
  \exists x((z|w)(x \in a \wedge w = (x, y)))
\]
が成り立つ.
また$w$が$x$とも$y$とも異なり, $a$の中に自由変数として現れないことから, 
代入法則 \ref{substfree}, \ref{substfund}, \ref{substwedge}, \ref{substpair}により, 
\[
  (z|w)(x \in a \wedge w = (x, y)) \equiv x \in a \wedge z = (x, y)
\]
が成り立つ.
そこでこれらから, (11)が
\[
  \forall z(\exists x(\exists y((x \in a \wedge y \in b) \wedge z = (x, y))) \leftrightarrow 
  \exists y(y \in b \wedge \exists x(x \in a \wedge z = (x, y))))
\]
と一致することがわかる.
故にこれが定理となる.
\halmos




\mathstrut
\begin{lemma}
\label{lemma2}%補題
$a$と$b$を集合とする.
また$x$, $y$, $z$を, どの二つも互いに異なり, 
どの一つも$a$及び$b$の中に自由変数として現れない文字とする.
このとき, 関係式$\exists y(y \in b \wedge \exists x(x \in a \wedge z = (x, y)))$は
$z$について集合を作り得る.
\end{lemma}


\noindent{\bf 証明}
~$u$と$v$を, 互いに異なり, 
共に$x$, $y$, $z$のいずれとも異なり, $a$の中に自由変数として現れない文字とする.
また$v$は定数でないとする.
また$\exists x(x \in a \wedge z = (x, y))$を$R$と書く.
$R$は関係式であり, 変数法則 \ref{valfund}, \ref{valwedge}, \ref{valquan}, \ref{valpair}からわかるように, 
$u$と$v$は共に$R$の中に自由変数として現れない.
また$x$が$y$とも$v$とも異なることから, 代入法則 \ref{substquan}により
\[
  (v|y)(R) \equiv \exists x((v|y)(x \in a \wedge z = (x, y)))
\]
が成り立つ.
また$y$が$x$及び$z$と異なり, $a$の中に自由変数として現れないことから, 
代入法則 \ref{substfree}, \ref{substfund}, \ref{substwedge}, \ref{substpair}により
\[
  (v|y)(x \in a \wedge z = (x, y)) \equiv x \in a \wedge z = (x, v)
\]
が成り立つ.
よってこれらから, 
\[
\tag{$*$}
  (v|y)(R) \equiv \exists x(x \in a \wedge z = (x, v))
\]
が成り立つ.
さていま$x$と$z$は互いに異なり, 共に$a$の中に自由変数として現れない.
また$z$は$v$とも異なるから, 変数法則 \ref{valpair}により, 
$z$は$(x, v)$の中に自由変数として現れない.
そこでこのことと($*$)から, 定理 \ref{sthmosetsm}より
$(v|y)(R)$が$z$について集合を作り得る, 即ち
\[
\tag{1}
  {\rm Set}_{z}((v|y)(R))
\]
が成り立つことがわかる.
また上述のように$u$は$R$の中に自由変数として現れないから, 
このことと$u$が$v$と異なることから, 変数法則 \ref{valsubst}により
$u$は$(v|y)(R)$の中に自由変数として現れない.
また$u$は$z$と異なる文字である.
そこでこのことから, 定理 \ref{sthmsmimp}と推論法則 \ref{dedequiv}により, 
\[
\tag{2}
  {\rm Set}_{z}((v|y)(R)) \to \exists u(\forall z((v|y)(R) \to z \in u))
\]
が成り立つ.
そこで(1), (2)から, 推論法則 \ref{dedmp}によって
\[
  \exists u(\forall z((v|y)(R) \to z \in u))
\]
が成り立つ.
ここで$y$が$z$とも$u$とも異なることと
代入法則 \ref{substfund}により, この記号列は
\[
  \exists u(\forall z((v|y)(R \to z \in u)))
\]
と一致する.
また$z$と$u$が共に$y$とも$v$とも異なることから, 
代入法則 \ref{substquan}により, この記号列は
\[
  (v|y)(\exists u(\forall z(R \to z \in u)))
\]
と一致する.
よってこれが定理となる.
そこで$v$が定数でないことから, 推論法則 \ref{dedltthmquan}により
\[
  \forall v((v|y)(\exists u(\forall z(R \to z \in u))))
\]
が成り立つ.
ここで$v$が$z$及び$u$と異なり, 上述のように$R$の中に自由変数として現れないことから, 
変数法則 \ref{valfund}, \ref{valquan}によって
$v$が$\exists u(\forall z(R \to z \in u))$の中にも自由変数として現れないことがわかるから, 
代入法則 \ref{substquantrans}により, 上記の記号列は
\[
\tag{3}
  \forall y(\exists u(\forall z(R \to z \in u)))
\]
と一致する.
よってこれが定理となる.
さていま仮定より$y$と$z$は異なる文字である.
また$u$と$v$は共に$y$とも$z$とも異なる文字であり, 
上述のようにこれらは$R$の中に自由変数として現れない.
よってschema S7の適用により, 
\[
\tag{4}
  \forall y(\exists u(\forall z(R \to z \in u))) \to 
  \forall v({\rm Set}_{z}(\exists y(y \in v \wedge R)))
\]
が成り立つ.
そこで(3), (4)から, 推論法則 \ref{dedmp}によって
\[
  \forall v({\rm Set}_{z}(\exists y(y \in v \wedge R)))
\]
が成り立ち, これから推論法則 \ref{dedfromallthm}によって
\[
  (b|v)({\rm Set}_{z}(\exists y(y \in v \wedge R)))
\]
が成り立つ.
ここで$z$が$v$と異なり, $b$の中に自由変数として現れないことから, 
代入法則 \ref{substsm}により, この記号列は
\[
  {\rm Set}_{z}((b|v)(\exists y(y \in v \wedge R)))
\]
と一致する.
また$y$も$v$と異なり, $b$の中に自由変数として現れないから, 
代入法則 \ref{substquan}により, この記号列は
\[
  {\rm Set}_{z}(\exists y((b|v)(y \in v \wedge R)))
\]
と一致する.
更に, $v$が$y$と異なり, 上述のように$R$の中に自由変数として現れないことから, 
代入法則 \ref{substfree}, \ref{substwedge}により, この記号列は
\[
  {\rm Set}_{z}(\exists y(y \in b \wedge R))
\]
と一致する.
よってこれが定理となる.
即ち, 関係式$\exists y(y \in b \wedge \exists x(x \in a \wedge z = (x, y)))$は
$z$について集合を作り得る.
\halmos




\mathstrut
\noindent{\bf 定理 \ref{sthmproductsetmake}の証明}
~$\exists x(\exists y((x \in a \wedge y \in b) \wedge z = (x, y)))$を$R$と書き, 
$\exists y(y \in b \wedge \exists x(x \in a \wedge z = (x, y)))$を$S$と書く.
これらは共に関係式であり, 補題 \ref{lemma1}により$\forall z(R \leftrightarrow S)$が成り立つ.
また補題 \ref{lemma2}により, Sは$z$について集合を作り得る.
そこで定理 \ref{sthmalleqsm}により, $R$は$z$について集合を作り得る.
\halmos




\mathstrut
\begin{thm}
\label{sthmproductelement}%定理
$a$, $b$, $c$を集合とするとき, 
\[
  c \in a \times b \leftrightarrow 
  {\rm Pair}(c) \wedge ({\rm pr}_{1}(c) \in a \wedge {\rm pr}_{2}(c) \in b)
\]
が成り立つ.
\end{thm}


\noindent{\bf 証明}
~$x$, $y$, $z$を, どの二つも互いに異なり, どの一つも
$a$, $b$, $c$のいずれの記号列の中にも自由変数として現れない文字とする.
このとき定義から, $a \times b$は
$\{z|\exists x(\exists y((x \in a \wedge y \in b) \wedge z = (x, y)))\}$と同じである.
また定理 \ref{sthmproductsetmake}より, 
関係式$\exists x(\exists y((x \in a \wedge y \in b) \wedge z = (x, y)))$は
$z$について集合を作り得る.
そこで定理 \ref{sthmisetbasis}より
\[
\tag{1}
  c \in a \times b \leftrightarrow 
  (c|z)(\exists x(\exists y((x \in a \wedge y \in b) \wedge z = (x, y))))
\]
が成り立つ.
ここで$x$と$y$が共に$z$と異なり, $c$の中に自由変数として現れないことから, 
代入法則 \ref{substquan}により
\[
  (c|z)(\exists x(\exists y((x \in a \wedge y \in b) \wedge z = (x, y)))) \equiv 
  \exists x(\exists y((c|z)((x \in a \wedge y \in b) \wedge z = (x, y))))
\]
が成り立つ.
また$z$が$x$とも$y$とも異なり, $a$及び$b$の中に自由変数として現れないことから, 
代入法則 \ref{substfree}, \ref{substfund}, \ref{substwedge}, \ref{substpair}により
\[
  (c|z)((x \in a \wedge y \in b) \wedge z = (x, y)) \equiv 
  (x \in a \wedge y \in b) \wedge c = (x, y)
\]
が成り立つ.
よって(1)は
\[
\tag{2}
  c \in a \times b \leftrightarrow 
  \exists x(\exists y((x \in a \wedge y \in b) \wedge c = (x, y)))
\]
と一致し, これが定理となる.
さていま$\tau_{x}(\exists y((x \in a \wedge y \in b) \wedge c = (x, y)))$を$T$と書けば, 
$T$は集合であり, 
変数法則 \ref{valtau}, \ref{valquan}により$y$は
$T$の中に自由変数として現れない.
そして定義から, 
\[
\tag{3}
  \exists x(\exists y((x \in a \wedge y \in b) \wedge c = (x, y))) \equiv 
  (T|x)(\exists y((x \in a \wedge y \in b) \wedge c = (x, y)))
\]
である.
また$y$が$x$と異なり, $T$の中に自由変数として現れないことから, 
代入法則 \ref{substquan}により
\[
\tag{4}
  (T|x)(\exists y((x \in a \wedge y \in b) \wedge c = (x, y))) \equiv 
  \exists y((T|x)((x \in a \wedge y \in b) \wedge c = (x, y)))
\]
が成り立つ.
また$x$が$y$と異なり, $a$, $b$, $c$の中に自由変数として現れないことから, 
代入法則 \ref{substfree}, \ref{substfund}, \ref{substwedge}, \ref{substpair}により
\[
\tag{5}
  (T|x)((x \in a \wedge y \in b) \wedge c = (x, y)) \equiv 
  (T \in a \wedge y \in b) \wedge c = (T, y)
\]
が成り立つ.
そこで(3), (4), (5)から, 
\[
\tag{6}
  \exists x(\exists y((x \in a \wedge y \in b) \wedge c = (x, y))) \equiv 
  \exists y((T \in a \wedge y \in b) \wedge c = (T, y))
\]
が成り立つことがわかる.
またいま$\tau_{y}((T \in a \wedge y \in b) \wedge c = (T, y))$を$U$と書けば, 
$U$は集合であり, 定義から
\[
\tag{7}
  \exists y((T \in a \wedge y \in b) \wedge c = (T, y)) \equiv 
  (U|y)((T \in a \wedge y \in b) \wedge c = (T, y))
\]
である.
また$y$は$a$, $b$, $c$の中に自由変数として現れず, 
上述のように$T$の中にも自由変数として現れないから, 
代入法則 \ref{substfree}, \ref{substfund}, \ref{substwedge}, \ref{substpair}により
\[
\tag{8}
  (U|y)((T \in a \wedge y \in b) \wedge c = (T, y)) \equiv 
  (T \in a \wedge U \in b) \wedge c = (T, U)
\]
が成り立つ.
そこで(6), (7), (8)から, 
\[
  \exists x(\exists y((x \in a \wedge y \in b) \wedge c = (x, y))) \equiv 
  (T \in a \wedge U \in b) \wedge c = (T, U)
\]
が成り立つことがわかるから, これにより(2)が
\[
  c \in a \times b \leftrightarrow 
  (T \in a \wedge U \in b) \wedge c = (T, U)
\]
と一致することがわかる.
故にこれが定理となる.
そこで特に, 推論法則 \ref{dedequiv}により
\[
\tag{9}
  c \in a \times b \to
  (T \in a \wedge U \in b) \wedge c = (T, U)
\]
が成り立つ.
またThm \ref{awbtbwa}より
\[
\tag{10}
  (T \in a \wedge U \in b) \wedge c = (T, U) \to
  c = (T, U) \wedge (T \in a \wedge U \in b)
\]
が成り立つ.
また定理 \ref{sthmpairpreq}と推論法則 \ref{dedequiv}により
\[
  c = (T, U) \to {\rm Pair}(c) \wedge (T = {\rm pr}_{1}(c) \wedge U = {\rm pr}_{2}(c))
\]
が成り立つから, 推論法則 \ref{dedaddw}により
\[
\tag{11}
  c = (T, U) \wedge (T \in a \wedge U \in b) \to 
  ({\rm Pair}(c) \wedge (T = {\rm pr}_{1}(c) \wedge U = {\rm pr}_{2}(c))) \wedge (T \in a \wedge U \in b)
\]
が成り立つ.
またThm \ref{1awb1wctaw1bwc1}より
\begin{multline*}
\tag{12}
  ({\rm Pair}(c) \wedge (T = {\rm pr}_{1}(c) \wedge U = {\rm pr}_{2}(c))) \wedge (T \in a \wedge U \in b) \\
  \to {\rm Pair}(c) \wedge ((T = {\rm pr}_{1}(c) \wedge U = {\rm pr}_{2}(c)) \wedge (T \in a \wedge U \in b))
\end{multline*}
及び
\[
\tag{13}
  (T = {\rm pr}_{1}(c) \wedge U = {\rm pr}_{2}(c)) \wedge (T \in a \wedge U \in b) \to 
  T = {\rm pr}_{1}(c) \wedge (U = {\rm pr}_{2}(c) \wedge (T \in a \wedge U \in b))
\]
が成り立つ.
またThm \ref{awbtbwa}より
\[
\tag{14}
  U = {\rm pr}_{2}(c) \wedge (T \in a \wedge U \in b) \to 
  (T \in a \wedge U \in b) \wedge U = {\rm pr}_{2}(c)
\]
が成り立ち, 
Thm \ref{1awb1wctaw1bwc1}より
\[
\tag{15}
  (T \in a \wedge U \in b) \wedge U = {\rm pr}_{2}(c) \to 
  T \in a \wedge (U \in b \wedge U = {\rm pr}_{2}(c))
\]
が成り立つ.
またThm \ref{awbtbwa}より
\[
  U \in b \wedge U = {\rm pr}_{2}(c) \to 
  U = {\rm pr}_{2}(c) \wedge U \in b
\]
が成り立つから, 推論法則 \ref{dedaddw}により
\[
\tag{16}
  T \in a \wedge (U \in b \wedge U = {\rm pr}_{2}(c)) \to 
  T \in a \wedge (U = {\rm pr}_{2}(c) \wedge U \in b)
\]
が成り立つ.
そこで(14), (15), (16)から, 推論法則 \ref{dedmmp}によって
\[
  U = {\rm pr}_{2}(c) \wedge (T \in a \wedge U \in b) \to 
  T \in a \wedge (U = {\rm pr}_{2}(c) \wedge U \in b)
\]
が成り立ち, これから推論法則 \ref{dedaddw}によって
\[
\tag{17}
  T = {\rm pr}_{1}(c) \wedge (U = {\rm pr}_{2}(c) \wedge (T \in a \wedge U \in b)) \to 
  T = {\rm pr}_{1}(c) \wedge (T \in a \wedge (U = {\rm pr}_{2}(c) \wedge U \in b))
\]
が成り立つ.
またThm \ref{aw1bwc1t1awb1wc}より
\[
\tag{18}
  T = {\rm pr}_{1}(c) \wedge (T \in a \wedge (U = {\rm pr}_{2}(c) \wedge U \in b)) \to 
  (T = {\rm pr}_{1}(c) \wedge T \in a) \wedge (U = {\rm pr}_{2}(c) \wedge U \in b)
\]
が成り立つ.
また定理 \ref{sthm=&in}より
\[
  T = {\rm pr}_{1}(c) \wedge T \in a \to {\rm pr}_{1}(c) \in a, ~~
  U = {\rm pr}_{2}(c) \wedge U \in b \to {\rm pr}_{2}(c) \in b
\]
が共に成り立つから, 推論法則 \ref{dedfromaddw}により
\[
\tag{19}
  (T = {\rm pr}_{1}(c) \wedge T \in a) \wedge (U = {\rm pr}_{2}(c) \wedge U \in b) \to
  {\rm pr}_{1}(c) \in a \wedge {\rm pr}_{2}(c) \in b
\]
が成り立つ.
そこで(13), (17), (18), (19)から, 推論法則 \ref{dedmmp}によって
\[
  (T = {\rm pr}_{1}(c) \wedge U = {\rm pr}_{2}(c)) \wedge (T \in a \wedge U \in b) \to 
  {\rm pr}_{1}(c) \in a \wedge {\rm pr}_{2}(c) \in b
\]
が成り立ち, これから推論法則 \ref{dedaddw}によって
\[
\tag{20}
  {\rm Pair}(c) \wedge ((T = {\rm pr}_{1}(c) \wedge U = {\rm pr}_{2}(c)) \wedge (T \in a \wedge U \in b)) \to 
  {\rm Pair}(c) \wedge ({\rm pr}_{1}(c) \in a \wedge {\rm pr}_{2}(c) \in b)
\]
が成り立つ.
そこで(9), (10), (11), (12), (20)から, 推論法則 \ref{dedmmp}によって
\[
\tag{21}
  c \in a \times b \to {\rm Pair}(c) \wedge ({\rm pr}_{1}(c) \in a \wedge {\rm pr}_{2}(c) \in b)
\]
が成り立つことがわかる.
またThm \ref{awbtbwa}より
\[
\tag{22}
  {\rm Pair}(c) \wedge ({\rm pr}_{1}(c) \in a \wedge {\rm pr}_{2}(c) \in b) \to 
  ({\rm pr}_{1}(c) \in a \wedge {\rm pr}_{2}(c) \in b) \wedge {\rm Pair}(c)
\]
が成り立つ.
また定理 \ref{sthmbigpairpr}と推論法則 \ref{dedequiv}により
${\rm Pair}(c) \to c = ({\rm pr}_{1}(c), {\rm pr}_{2}(c))$が
成り立つから, 推論法則 \ref{dedaddw}により
\[
\tag{23}
  ({\rm pr}_{1}(c) \in a \wedge {\rm pr}_{2}(c) \in b) \wedge {\rm Pair}(c) \to 
  ({\rm pr}_{1}(c) \in a \wedge {\rm pr}_{2}(c) \in b) \wedge c = ({\rm pr}_{1}(c), {\rm pr}_{2}(c))
\]
が成り立つ.
そこで(22), (23)から, 推論法則 \ref{dedmmp}によって
\[
\tag{24}
  {\rm Pair}(c) \wedge ({\rm pr}_{1}(c) \in a \wedge {\rm pr}_{2}(c) \in b) \to 
  ({\rm pr}_{1}(c) \in a \wedge {\rm pr}_{2}(c) \in b) \wedge c = ({\rm pr}_{1}(c), {\rm pr}_{2}(c))
\]
が成り立つ.
いま$y$は$c$の中に自由変数として現れないから, 変数法則 \ref{valpr}により, 
$y$は${\rm pr}_{1}(c)$の中に自由変数として現れない.
また$y$は$a$及び$b$の中にも自由変数として現れない.
そこで代入法則 \ref{substfree}, \ref{substfund}, \ref{substwedge}, \ref{substpair}により, 
\[
\tag{25}
  ({\rm pr}_{1}(c) \in a \wedge {\rm pr}_{2}(c) \in b) \wedge c = ({\rm pr}_{1}(c), {\rm pr}_{2}(c)) \equiv 
  ({\rm pr}_{2}(c)|y)(({\rm pr}_{1}(c) \in a \wedge y \in b) \wedge c = ({\rm pr}_{1}(c), y))
\]
が成り立つ.
また$x$が$y$と異なり, $a$, $b$, $c$の中に自由変数として現れないことから, 
同じく代入法則 \ref{substfree}, \ref{substfund}, \ref{substwedge}, \ref{substpair}により
\[
\tag{26}
  ({\rm pr}_{1}(c) \in a \wedge y \in b) \wedge c = ({\rm pr}_{1}(c), y) \equiv 
  ({\rm pr}_{1}(c)|x)((x \in a \wedge y \in b) \wedge c = (x, y))
\]
が成り立つ.
そこで(25), (26)から, (24)が
\[
\tag{27}
  {\rm Pair}(c) \wedge ({\rm pr}_{1}(c) \in a \wedge {\rm pr}_{2}(c) \in b) \to 
  ({\rm pr}_{2}(c)|y)(({\rm pr}_{1}(c)|x)((x \in a \wedge y \in b) \wedge c = (x, y)))
\]
と一致することがわかり, これが定理となる.
またschema S4の適用により
\[
  ({\rm pr}_{2}(c)|y)(({\rm pr}_{1}(c)|x)((x \in a \wedge y \in b) \wedge c = (x, y))) \to 
  \exists y(({\rm pr}_{1}(c)|x)((x \in a \wedge y \in b) \wedge c = (x, y)))
\]
が成り立つ.
ここで$y$が$x$と異なり, 上述のように${\rm pr}_{1}(c)$の中に自由変数として現れないことから, 
代入法則 \ref{substquan}により
$\exists y(({\rm pr}_{1}(c)|x)((x \in a \wedge y \in b) \wedge c = (x, y)))$は
$({\rm pr}_{1}(c)|x)(\exists y((x \in a \wedge y \in b) \wedge c = (x, y)))$と一致するから, 
上記の記号列は
\[
\tag{28}
  ({\rm pr}_{2}(c)|y)(({\rm pr}_{1}(c)|x)((x \in a \wedge y \in b) \wedge c = (x, y))) \to 
  ({\rm pr}_{1}(c)|x)(\exists y((x \in a \wedge y \in b) \wedge c = (x, y)))
\]
と一致する.
よってこれが定理となる.
また同じくschema S4の適用により, 
\[
\tag{29}
  ({\rm pr}_{1}(c)|x)(\exists y((x \in a \wedge y \in b) \wedge c = (x, y))) \to 
  \exists x(\exists y((x \in a \wedge y \in b) \wedge c = (x, y)))
\]
が成り立つ.
また(2)から, 推論法則 \ref{dedequiv}により
\[
\tag{30}
  \exists x(\exists y((x \in a \wedge y \in b) \wedge c = (x, y))) \to 
  c \in a \times b
\]
が成り立つ.
そこで(27)---(30)から, 推論法則 \ref{dedmmp}によって
\[
\tag{31}
  {\rm Pair}(c) \wedge ({\rm pr}_{1}(c) \in a \wedge {\rm pr}_{2}(c) \in b) \to c \in a \times b
\]
が成り立つことがわかる.
(21), (31)から, 推論法則 \ref{dedequiv}によって
\[
  c \in a \times b \leftrightarrow {\rm Pair}(c) \wedge ({\rm pr}_{1}(c) \in a \wedge {\rm pr}_{2}(c) \in b)
\]
が成り立つ.
\halmos




\mathstrut
\begin{thm}
\label{sthmpairinproduct}%定理
$a$, $b$, $c$, $d$を集合とするとき, 
\[
  (a, b) \in c \times d \leftrightarrow a \in c \wedge b \in d
\]
が成り立つ.
またこのことから, 次の($*$)が成り立つ: 

($*$) ~~$(a, b) \in c \times d$が成り立つならば, $a \in c$と$b \in d$が共に成り立つ.
        逆に$a \in c$と$b \in d$が共に成り立つならば, $(a, b) \in c \times d$が成り立つ.
\end{thm}


\noindent{\bf 証明}
~まず前半を示す.
定理 \ref{sthmproductelement}より
\[
\tag{1}
  (a, b) \in c \times d \leftrightarrow 
  {\rm Pair}((a, b)) \wedge ({\rm pr}_{1}((a, b)) \in c \wedge {\rm pr}_{2}((a, b)) \in d)
\]
が成り立つ.
また定理 \ref{sthmbigpairpair}より${\rm Pair}((a, b))$が成り立つから, 
推論法則 \ref{dedawblatrue2}により
\[
\tag{2}
  {\rm Pair}((a, b)) \wedge ({\rm pr}_{1}((a, b)) \in c \wedge {\rm pr}_{2}((a, b)) \in d) \leftrightarrow 
  {\rm pr}_{1}((a, b)) \in c \wedge {\rm pr}_{2}((a, b)) \in d
\]
が成り立つ.
また定理 \ref{sthmprpair}より
${\rm pr}_{1}((a, b)) = a$と${\rm pr}_{2}((a, b)) = b$が共に成り立つから, 
定理 \ref{sthm=tineq}により
\[
  {\rm pr}_{1}((a, b)) \in c \leftrightarrow a \in c, ~~
  {\rm pr}_{2}((a, b)) \in d \leftrightarrow b \in d
\]
が共に成り立ち, これらから, 推論法則 \ref{dedaddeqw}によって
\[
\tag{3}
  {\rm pr}_{1}((a, b)) \in c \wedge {\rm pr}_{2}((a, b)) \in d \leftrightarrow 
  a \in c \wedge b \in d
\]
が成り立つ.
そこで(1), (2), (3)から, 推論法則 \ref{dedeqtrans}によって
\[
\tag{4}
  (a, b) \in c \times d \leftrightarrow a \in c \wedge b \in d
\]
が成り立つ.

さていま$(a, b) \in c \times d$が成り立つとすると, これと今示した(4)から, 
推論法則 \ref{dedeqfund}によって$a \in c \wedge b \in d$が成り立つから, 
推論法則 \ref{dedwedge}により$a \in c$と$b \in d$が共に成り立つ.
逆に$a \in c$と$b \in d$が共に成り立つならば, 推論法則 \ref{dedwedge}により
$a \in c \wedge b \in d$が成り立つから, これと(4)から, 推論法則 \ref{dedeqfund}によって
$(a, b) \in c \times d$が成り立つ.
これで($*$)が成り立つことが示された.
\halmos




\mathstrut
\begin{thm}
\label{sthmproductsubset}%定理
\mbox{}

1) $a$, $b$, $c$を集合とするとき, 
   \[
     a \subset b \to a \times c \subset b \times c, ~~
     a \subset b \to c \times a \subset c \times b
   \]
   が成り立つ.
   またこのことから, 次の($*$)が成り立つ: 
   
   ($*$) ~~$a \subset b$が成り立つならば, 
           $a \times c \subset b \times c$及び$c \times a \subset c \times b$が成り立つ.

2) $a$, $b$, $c$, $d$を集合とするとき, 
   \[
     a \subset c \wedge b \subset d \to a \times b \subset c \times d
   \]
   が成り立つ.
   またこのことから, 次の($**$)が成り立つ: 
   
   ($**$) ~~$a \subset c$と$b \subset d$が共に成り立つならば, 
            $a \times b \subset c \times d$が成り立つ.
\end{thm}


\noindent{\bf 証明}
~1)
$x$を$a$, $b$, $c$のいずれの記号列の中にも自由変数として現れない文字とする.
このとき変数法則 \ref{valproduct}により, $x$は
$a \times c$, $b \times c$, $c \times a$, $c \times b$のいずれの記号列の中にも
自由変数として現れない.
またいま$\tau_{x}(\neg (x \in a \times c \to x \in b \times c))$, 
$\tau_{x}(\neg (x \in c \times a \to x \in c \times b))$をそれぞれ$T$, $U$と
書けば, これらは集合であり, 定理 \ref{sthmsubsetbasis}より
\begin{align*}
  \tag{1}
  a \subset b &\to ({\rm pr}_{1}(T) \in a \to {\rm pr}_{1}(T) \in b), \\
  \mbox{} \\
  \tag{2}
  a \subset b &\to ({\rm pr}_{2}(U) \in a \to {\rm pr}_{2}(U) \in b)
\end{align*}
が共に成り立つ.
またThm \ref{1atb1t1awctbwc1}より
\begin{multline*}
\tag{3}
  ({\rm pr}_{1}(T) \in a \to {\rm pr}_{1}(T) \in b) \\
  \to (({\rm Pair}(T) \wedge {\rm pr}_{2}(T) \in c) \wedge {\rm pr}_{1}(T) \in a 
  \to ({\rm Pair}(T) \wedge {\rm pr}_{2}(T) \in c) \wedge {\rm pr}_{1}(T) \in b), 
\end{multline*}
\begin{multline*}
\tag{4}
  ({\rm pr}_{2}(U) \in a \to {\rm pr}_{2}(U) \in b) \\
  \to (({\rm Pair}(U) \wedge {\rm pr}_{1}(U) \in c) \wedge {\rm pr}_{2}(U) \in a 
  \to ({\rm Pair}(U) \wedge {\rm pr}_{1}(U) \in c) \wedge {\rm pr}_{2}(U) \in b)
\end{multline*}
が共に成り立つ.
また定理 \ref{sthmproductelement}と推論法則 \ref{dedequiv}により, 
\begin{align*}
  \tag{5}
  &T \in a \times c \to {\rm Pair}(T) \wedge ({\rm pr}_{1}(T) \in a \wedge {\rm pr}_{2}(T) \in c), \\
  \mbox{} \\
  \tag{6}
  &U \in c \times a \to {\rm Pair}(U) \wedge ({\rm pr}_{1}(U) \in c \wedge {\rm pr}_{2}(U) \in a), \\
  \mbox{} \\
  \tag{7}
  &{\rm Pair}(T) \wedge ({\rm pr}_{1}(T) \in b \wedge {\rm pr}_{2}(T) \in c) \to T \in b \times c, \\
  \mbox{} \\
  \tag{8}
  &{\rm Pair}(U) \wedge ({\rm pr}_{1}(U) \in c \wedge {\rm pr}_{2}(U) \in b) \to U \in c \times b
\end{align*}
がすべて成り立つ.
またThm \ref{awbtbwa}より
\begin{align*}
  {\rm pr}_{1}(T) \in a \wedge {\rm pr}_{2}(T) \in c &\to {\rm pr}_{2}(T) \in c \wedge {\rm pr}_{1}(T) \in a, \\
  \mbox{} \\
  {\rm pr}_{2}(T) \in c \wedge {\rm pr}_{1}(T) \in b &\to {\rm pr}_{1}(T) \in b \wedge {\rm pr}_{2}(T) \in c
\end{align*}
が共に成り立つから, 推論法則 \ref{dedaddw}により
\begin{align*}
  \tag{9}
  {\rm Pair}(T) \wedge ({\rm pr}_{1}(T) \in a \wedge {\rm pr}_{2}(T) \in c) &\to
  {\rm Pair}(T) \wedge ({\rm pr}_{2}(T) \in c \wedge {\rm pr}_{1}(T) \in a), \\
  \mbox{} \\
  \tag{10}
  {\rm Pair}(T) \wedge ({\rm pr}_{2}(T) \in c \wedge {\rm pr}_{1}(T) \in b) &\to
  {\rm Pair}(T) \wedge ({\rm pr}_{1}(T) \in b \wedge {\rm pr}_{2}(T) \in c)
\end{align*}
が共に成り立つ.
またThm \ref{aw1bwc1t1awb1wc}より
\begin{align*}
  \tag{11}
  {\rm Pair}(T) \wedge ({\rm pr}_{2}(T) \in c \wedge {\rm pr}_{1}(T) \in a) &\to
  ({\rm Pair}(T) \wedge {\rm pr}_{2}(T) \in c) \wedge {\rm pr}_{1}(T) \in a, \\
  \mbox{} \\
  \tag{12}
  {\rm Pair}(U) \wedge ({\rm pr}_{1}(U) \in c \wedge {\rm pr}_{2}(U) \in a) &\to
  ({\rm Pair}(U) \wedge {\rm pr}_{1}(U) \in c) \wedge {\rm pr}_{2}(U) \in a
\end{align*}
が共に成り立つ.
またThm \ref{1awb1wctaw1bwc1}より
\begin{align*}
  \tag{13}
  ({\rm Pair}(T) \wedge {\rm pr}_{2}(T) \in c) \wedge {\rm pr}_{1}(T) \in b &\to
  {\rm Pair}(T) \wedge ({\rm pr}_{2}(T) \in c \wedge {\rm pr}_{1}(T) \in b), \\
  \mbox{} \\
  \tag{14}
  ({\rm Pair}(U) \wedge {\rm pr}_{1}(U) \in c) \wedge {\rm pr}_{2}(U) \in b &\to
  {\rm Pair}(U) \wedge ({\rm pr}_{1}(U) \in c \wedge {\rm pr}_{2}(U) \in b)
\end{align*}
が共に成り立つ.
そこで推論法則 \ref{dedmmp}を, (5)と(9)と(11), (6)と(12), (13)と(10)と(7), (14)と(8)にそれぞれ
この順で適用して, 
\begin{align*}
  \tag{15}
  &T \in a \times c \to ({\rm Pair}(T) \wedge {\rm pr}_{2}(T) \in c) \wedge {\rm pr}_{1}(T) \in a, \\
  \mbox{} \\
  \tag{16}
  &U \in c \times a \to ({\rm Pair}(U) \wedge {\rm pr}_{1}(U) \in c) \wedge {\rm pr}_{2}(U) \in a, \\
  \mbox{} \\
  \tag{17}
  &({\rm Pair}(T) \wedge {\rm pr}_{2}(T) \in c) \wedge {\rm pr}_{1}(T) \in b \to T \in b \times c, \\
  \mbox{} \\
  \tag{18}
  &({\rm Pair}(U) \wedge {\rm pr}_{1}(U) \in c) \wedge {\rm pr}_{2}(U) \in b \to U \in c \times b
\end{align*}
がすべて成り立つことがわかる.
そこで(15)と(16)にそれぞれ推論法則 \ref{dedaddf}を適用して, 
\begin{multline*}
\tag{19}
  (({\rm Pair}(T) \wedge {\rm pr}_{2}(T) \in c) \wedge {\rm pr}_{1}(T) \in a 
  \to ({\rm Pair}(T) \wedge {\rm pr}_{2}(T) \in c) \wedge {\rm pr}_{1}(T) \in b) \\
  \to (T \in a \times c \to ({\rm Pair}(T) \wedge {\rm pr}_{2}(T) \in c) \wedge {\rm pr}_{1}(T) \in b), 
\end{multline*}
\begin{multline*}
\tag{20}
  (({\rm Pair}(U) \wedge {\rm pr}_{1}(U) \in c) \wedge {\rm pr}_{2}(U) \in a 
  \to ({\rm Pair}(U) \wedge {\rm pr}_{1}(U) \in c) \wedge {\rm pr}_{2}(U) \in b) \\
  \to (U \in c \times a \to ({\rm Pair}(U) \wedge {\rm pr}_{1}(U) \in c) \wedge {\rm pr}_{2}(U) \in b)
\end{multline*}
が共に成り立つ.
また(17)と(18)にそれぞれ推論法則 \ref{dedaddb}を適用して, 
\begin{align*}
  \tag{21}
  (T \in a \times c \to ({\rm Pair}(T) \wedge {\rm pr}_{2}(T) \in c) \wedge {\rm pr}_{1}(T) \in b) &\to 
  (T \in a \times c \to T \in b \times c), \\
  \mbox{} \\
  \tag{22}
  (U \in c \times a \to ({\rm Pair}(U) \wedge {\rm pr}_{1}(U) \in c) \wedge {\rm pr}_{2}(U) \in b) &\to 
  (U \in c \times a \to U \in c \times b)
\end{align*}
が共に成り立つ.
また$T$と$U$の定義から, Thm \ref{thmallfund1}と推論法則 \ref{dedequiv}により
\begin{align*}
  (T|x)(x \in a \times c \to x \in b \times c) &\to \forall x(x \in a \times c \to x \in b \times c), \\
  \mbox{} \\
  (U|x)(x \in c \times a \to x \in c \times b) &\to \forall x(x \in c \times a \to x \in c \times b)
\end{align*}
が共に成り立つが, はじめに述べたように$x$は$a \times c$, $b \times c$, $c \times a$, $c \times b$の
いずれの記号列の中にも自由変数として現れないから, 
代入法則 \ref{substfree}, \ref{substfund}及び定義によれば, これらの記号列はそれぞれ
\begin{align*}
  \tag{23}
  (T \in a \times c \to T \in b \times c) &\to a \times c \subset b \times c, \\
  \mbox{} \\
  \tag{24}
  (U \in c \times a \to U \in c \times b) &\to c \times a \subset c \times b
\end{align*}
と一致する.
よってこれらが共に定理となる.
そこで(1), (3), (19), (21), (23)から, 推論法則 \ref{dedmmp}によって
\[
  a \subset b \to a \times c \subset b \times c
\]
が成り立ち, (2), (4), (20), (22), (24)から, 同じく推論法則 \ref{dedmmp}によって
\[
  a \subset b \to c \times a \subset c \times b
\]
が成り立つことがわかる.
($*$)が成り立つことは, これらと推論法則 \ref{dedmp}によって明らかである.

\noindent
2)
まず前半を示す.
1)より
\[
  a \subset c \to a \times b \subset c \times b, ~~
  b \subset d \to c \times b \subset c \times d
\]
が共に成り立つから, 推論法則 \ref{dedfromaddw}により
\[
\tag{25}
  a \subset c \wedge b \subset d \to a \times b \subset c \times b \wedge c \times b \subset c \times d
\]
が成り立つ.
また定理 \ref{sthmsubsettrans}より
\[
\tag{26}
  a \times b \subset c \times b \wedge c \times b \subset c \times d \to a \times b \subset c \times d
\]
が成り立つ.
そこで(25), (26)から, 推論法則 \ref{dedmmp}によって
\[
\tag{27}
  a \subset c \wedge b \subset d \to a \times b \subset c \times d
\]
が成り立つ.

いま$a \subset c$と$b \subset d$が共に成り立つとすれば, 推論法則 \ref{dedwedge}によって
$a \subset c \wedge b \subset d$が成り立つから, これと(27)から, 
推論法則 \ref{dedmp}によって$a \times b \subset c \times d$が成り立つ.
これで($**$)が成り立つことが示された.
\halmos




\mathstrut
\begin{thm}
\label{sthmproductsubsetinverse}%定理
$a$, $b$, $c$, $d$を集合とするとき, 
\[
  a \neq \phi \to (a \times b \subset c \times d \to b \subset d), ~~
  b \neq \phi \to (a \times b \subset c \times d \to a \subset c)
\]
が成り立つ.
またこのことから, 
\[
  a \neq \phi \wedge b \neq \phi \to (a \times b \subset c \times d \to a \subset c \wedge b \subset d)
\]
が成り立ち, 更に次の($*$)が成り立つ: 

($*$) ~~$a \times b \subset c \times d$が成り立つとき, 
        $a$が空でなければ$b \subset d$が成り立ち, 
        $b$が空でなければ$a \subset c$が成り立つ.
\end{thm}


\noindent{\bf 証明}
~まず
\begin{align*}
  \tag{1}
  a \neq \phi &\to (a \times b \subset c \times d \to b \subset d), \\
  \mbox{} \\
  \tag{2}
  b \neq \phi &\to (a \times b \subset c \times d \to a \subset c)
\end{align*}
が共に成り立つことを示す.
$x$を$a$, $b$, $c$, $d$のいずれの記号列の中にも自由変数として現れない文字とし, 
$\tau_{x}(x \in a)$を$T$, $\tau_{x}(x \in b)$を$U$と書けば, これらは集合であり, 
定理 \ref{sthmelm&empty}と推論法則 \ref{dedequiv}により
\begin{align*}
  \tag{3}
  a \neq \phi &\to T \in a, \\
  \mbox{} \\
  \tag{4}
  b \neq \phi &\to U \in b
\end{align*}
が共に成り立つ.
また$\tau_{x}(\neg (x \in b \to x \in d))$を$V$, $\tau_{x}(\neg (x \in a \to x \in c))$を$W$と
書けば, これらも集合であり, 定理 \ref{sthmsubsetbasis}より
\begin{align*}
  \tag{5}
  a \times b \subset c \times d &\to ((T, V) \in a \times b \to (T, V) \in c \times d), \\
  \mbox{} \\
  \tag{6}
  a \times b \subset c \times d &\to ((W, U) \in a \times b \to (W, U) \in c \times d)
\end{align*}
が共に成り立つ.
また定理 \ref{sthmpairinproduct}と推論法則 \ref{dedequiv}により, 
\begin{align*}
  \tag{7}
  &T \in a \wedge V \in b \to (T, V) \in a \times b, \\
  \mbox{} \\
  \tag{8}
  &W \in a \wedge U \in b \to (W, U) \in a \times b, \\
  \mbox{} \\
  \tag{9}
  &(T, V) \in c \times d \to T \in c \wedge V \in d, \\
  \mbox{} \\
  \tag{10}
  &(W, U) \in c \times d \to W \in c \wedge U \in d
\end{align*}
がすべて成り立つ.
またThm \ref{awbtbwa}より
\[
  U \in b \wedge W \in a \to W \in a \wedge U \in b
\]
が成り立つから, これと(8)から, 推論法則 \ref{dedmmp}によって
\[
\tag{11}
  U \in b \wedge W \in a \to (W, U) \in a \times b
\]
が成り立つ.
そこで(7)と(11)にそれぞれ推論法則 \ref{dedaddf}を適用して, 
\begin{align*}
  \tag{12}
  ((T, V) \in a \times b \to (T, V) \in c \times d) &\to (T \in a \wedge V \in b \to (T, V) \in c \times d), \\
  \mbox{} \\
  \tag{13}
  ((W, U) \in a \times b \to (W, U) \in c \times d) &\to (U \in b \wedge W \in a \to (W, U) \in c \times d)
\end{align*}
が共に成り立つ.
またThm \ref{awbta}より
\begin{align*}
  \tag{14}
  T \in c \wedge V \in d &\to V \in d, \\
  \mbox{} \\
  \tag{15}
  W \in c \wedge U \in d &\to W \in c
\end{align*}
が共に成り立つ.
そこで(9)と(14), (10)と(15)にそれぞれ推論法則 \ref{dedmmp}を適用して, 
\begin{align*}
  (T, V) \in c \times d &\to V \in d, \\
  \mbox{} \\
  (W, U) \in c \times d &\to W \in c
\end{align*}
が共に成り立つから, これらにそれぞれ推論法則 \ref{dedaddb}を適用して, 
\begin{align*}
  \tag{16}
  (T \in a \wedge V \in b \to (T, V) \in c \times d) &\to (T \in a \wedge V \in b \to V \in d), \\
  \mbox{} \\
  \tag{17}
  (U \in b \wedge W \in a \to (W, U) \in c \times d) &\to (U \in b \wedge W \in a \to W \in c)
\end{align*}
が共に成り立つ.
またThm \ref{1awbtc1t1at1btc11}より
\begin{align*}
  \tag{18}
  (T \in a \wedge V \in b \to V \in d) &\to (T \in a \to (V \in b \to V \in d)), \\
  \mbox{} \\
  \tag{19}
  (U \in b \wedge W \in a \to W \in c) &\to (U \in b \to (W \in a \to W \in c))
\end{align*}
が共に成り立つ.
また$V$と$W$の定義から, Thm \ref{thmallfund1}と推論法則 \ref{dedmmp}によって
\begin{align*}
  (V|x)(x \in b \to x \in d) &\to \forall x(x \in b \to x \in d), \\
  \mbox{} \\
  (W|x)(x \in a \to x \in c) &\to \forall x(x \in a \to x \in c)
\end{align*}
が共に成り立つが, いま$x$は$a$, $b$, $c$, $d$のいずれの記号列の中にも
自由変数として現れないから, 代入法則 \ref{substfree}, \ref{substfund}及び定義によれば, 
これらはそれぞれ
\begin{align*}
  (V \in b \to V \in d) &\to b \subset d, \\
  \mbox{} \\
  (W \in a \to W \in c) &\to a \subset c
\end{align*}
と一致する.
よってこれらが定理となる.
そこでこれらにそれぞれ推論法則 \ref{dedaddb}を適用して, 
\begin{align*}
  \tag{20}
  (T \in a \to (V \in b \to V \in d)) &\to (T \in a \to b \subset d), \\
  \mbox{} \\
  \tag{21}
  (U \in b \to (W \in a \to W \in c)) &\to (U \in b \to a \subset c)
\end{align*}
が共に成り立つ.
そこで(5), (12), (16), (18), (20)から, 推論法則 \ref{dedmmp}によって
\[
\tag{22}
  a \times b \subset c \times d \to (T \in a \to b \subset d)
\]
が成り立ち, (6), (13), (17), (19), (21)から, 同じく推論法則 \ref{dedmmp}によって
\[
\tag{23}
  a \times b \subset c \times d \to (U \in b \to a \subset c)
\]
が成り立つことがわかる.
そこで(22), (23)にそれぞれ推論法則 \ref{dedch}を適用して, 
\begin{align*}
  \tag{24}
  T \in a &\to (a \times b \subset c \times d \to b \subset d), \\
  \mbox{} \\
  \tag{25}
  U \in b &\to (a \times b \subset c \times d \to a \subset c)
\end{align*}
が共に成り立つ.
そこで(3)と(24), (4)と(25)にそれぞれ推論法則 \ref{dedmmp}を適用して, 
(1)と(2)が共に成り立つことがわかる.

次に
\[
\tag{26}
  a \neq \phi \wedge b \neq \phi \to (a \times b \subset c \times d \to a \subset c \wedge b \subset d)
\]
が成り立つことを示す.
まず今示したように, (1)と(2)が共に成り立つから, 
推論法則 \ref{dedfromaddw}により
\[
\tag{27}
  a \neq \phi \wedge b \neq \phi \to 
  (a \times b \subset c \times d \to b \subset d) \wedge (a \times b \subset c \times d \to a \subset c)
\]
が成り立つ.
またThm \ref{awbtbwa}より
\begin{multline*}
\tag{28}
  (a \times b \subset c \times d \to b \subset d) \wedge (a \times b \subset c \times d \to a \subset c) \\
  \to (a \times b \subset c \times d \to a \subset c) \wedge (a \times b \subset c \times d \to b \subset d)
\end{multline*}
が成り立つ.
またThm \ref{1atb1w1atc1t1atbwc1}より
\[
\tag{29}
  (a \times b \subset c \times d \to a \subset c) \wedge (a \times b \subset c \times d \to b \subset d) \to 
  (a \times b \subset c \times d \to a \subset c \wedge b \subset d)
\]
が成り立つ.
そこで(27), (28), (29)から, 推論法則 \ref{dedmmp}によって(26)が成り立つ.

最後に($*$)であるが, これが成り立つことは, (1)と(2)が共に成り立つことと
推論法則 \ref{dedmp}によって明らかである.
\halmos




\mathstrut
\begin{thm}
\label{sthmproductsubseteq}%定理
\mbox{}

1)
$a$, $b$, $c$を集合とするとき, 
\[
  c \neq \phi \to (a \subset b \leftrightarrow a \times c \subset b \times c), ~~
  c \neq \phi \to (a \subset b \leftrightarrow c \times a \subset c \times b)
\]
が成り立つ.
またこのことから, 次の($*$)が成り立つ: 

($*$) ~~$c$が空でなければ, 
        $a \subset b \leftrightarrow a \times c \subset b \times c$と
        $a \subset b \leftrightarrow c \times a \subset c \times b$が共に成り立つ.

2)
$a$, $b$, $c$, $d$を集合とするとき, 
\[
  a \neq \phi \wedge b \neq \phi \to (a \subset c \wedge b \subset d \leftrightarrow a \times b \subset c \times d)
\]
が成り立つ.
またこのことから, 次の($**$)が成り立つ: 

($**$) ~~$a$が空でなく, $b$も空でなければ, 
         $a \subset c \wedge b \subset d \leftrightarrow a \times b \subset c \times d$が成り立つ.
\end{thm}


\noindent{\bf 証明}
~1)
定理 \ref{sthmproductsubset}より
\[
  a \subset b \to a \times c \subset b \times c, ~~
  a \subset b \to c \times a \subset c \times b
\]
が共に成り立つから, 推論法則 \ref{deds1}により, 
\begin{align*}
  \tag{1}
  c \neq \phi &\to (a \subset b \to a \times c \subset b \times c), \\
  \mbox{} \\
  \tag{2}
  c \neq \phi &\to (a \subset b \to c \times a \subset c \times b)
\end{align*}
が共に成り立つ.
また定理 \ref{sthmproductsubsetinverse}より
\begin{align*}
  \tag{3}
  c \neq \phi &\to (a \times c \subset b \times c \to a \subset b), \\
  \mbox{} \\
  \tag{4}
  c \neq \phi &\to (c \times a \subset c \times b \to a \subset b)
\end{align*}
が共に成り立つ.
そこで(1)と(3)から, 推論法則 \ref{dedprewedge}によって
\[
  c \neq \phi \to (a \subset b \leftrightarrow a \times c \subset b \times c)
\]
が成り立ち, (2)と(4)から, 同じく推論法則 \ref{dedprewedge}によって
\[
  c \neq \phi \to (a \subset b \leftrightarrow c \times a \subset c \times b)
\]
が成り立つ.
($*$)が成り立つことは, これらと推論法則 \ref{dedmp}によって明らかである.

\noindent
2)
定理 \ref{sthmproductsubset}より
\[
  a \subset c \wedge b \subset d \to a \times b \subset c \times d
\]
が成り立つから, 推論法則 \ref{deds1}により
\[
\tag{5}
  a \neq \phi \wedge b \neq \phi \to (a \subset c \wedge b \subset d \to a \times b \subset c \times d)
\]
が成り立つ.
また定理 \ref{sthmproductsubsetinverse}より
\[
\tag{6}
  a \neq \phi \wedge b \neq \phi \to (a \times b \subset c \times d \to a \subset c \wedge b \subset d)
\]
が成り立つ.
そこで(5), (6)から, 推論法則 \ref{dedprewedge}によって
\[
\tag{7}
  a \neq \phi \wedge b \neq \phi \to (a \subset c \wedge b \subset d \leftrightarrow a \times b \subset c \times d)
\]
が成り立つ.

いま$a$と$b$が共に空でないとすれば, 推論法則 \ref{dedwedge}により
$a \neq \phi \wedge b \neq \phi$が成り立つから, これと(7)から, 
推論法則 \ref{dedmp}によって
$a \subset c \wedge b \subset d \leftrightarrow a \times b \subset c \times d$が
成り立つ.
これで($**$)が成り立つことも示された.
\halmos




\mathstrut
\begin{thm}
\label{sthmproduct=}%定理
\mbox{}

1) $a$, $b$, $c$を集合とするとき, 
   \[
     a = b \to a \times c = b \times c, ~~
     a = b \to c \times a = c \times b
   \]
   が成り立つ.
   またこのことから, 次の($*$)が成り立つ: 
   
   ($*$) ~~$a = b$が成り立つならば, $a \times c = b \times c$と
           $c \times a = c \times b$が共に成り立つ.

2) $a$, $b$, $c$, $d$を集合とするとき, 
   \[
     a = c \wedge b = d \to a \times b = c \times d
   \]
   が成り立つ.
   またこのことから, 次の($**$)が成り立つ: 
   
   ($**$) ~~$a = c$と$b = d$が共に成り立つならば, 
            $a \times b = c \times d$が成り立つ.
\end{thm}


\noindent{\bf 証明}
~1)
$x$を$c$の中に自由変数として現れない文字とするとき, Thm \ref{T=Ut1TV=UV1}より
\[
  a = b \to (a|x)(x \times c) = (b|x)(x \times c), ~~
  a = b \to (a|x)(c \times x) = (b|x)(c \times x)
\]
が共に成り立つが, 代入法則 \ref{substfree}, \ref{substproduct}によれば, これらの記号列はそれぞれ
\[
  a = b \to a \times c = b \times c, ~~
  a = b \to c \times a = c \times b
\]
と一致するから, これらが定理となる.
($*$)が成り立つことは, これらと推論法則 \ref{dedmp}によって明らか.

\noindent
2)
1)より
$a = c \to a \times b = c \times b$と
$b = d \to c \times b = c \times d$が共に成り立つから, 
推論法則 \ref{dedfromaddw}により
\[
  a = c \wedge b = d \to a \times b = c \times b \wedge c \times b = c \times d
\]
が成り立つ.
またThm \ref{x=ywy=ztx=z}より
\[
  a \times b = c \times b \wedge c \times b = c \times d \to a \times b = c \times d
\]
が成り立つ.
そこでこれらから, 推論法則 \ref{dedmmp}によって
\[
\tag{1}
  a = c \wedge b = d \to a \times b = c \times d
\]
が成り立つ.

いま$a = c$と$b = d$が共に成り立つならば, 推論法則 \ref{dedwedge}により
$a = c \wedge b = d$が成り立つから, これと(1)から, 推論法則 \ref{dedmp}によって
$a \times b = c \times d$が成り立つ.
これで($**$)が成り立つことも示された.
\halmos




\mathstrut
\begin{thm}
\label{sthmproduct=inverse}%定理
\mbox{}

1)
$a$, $b$, $c$を集合とするとき, 
\[
  c \neq \phi \to (a \times c = b \times c \to a = b), ~~
  c \neq \phi \to (c \times a = c \times b \to a = b)
\]
が成り立つ.
またこのことから, 次の($*$)が成り立つ: 

($*$) ~~$c$が空でないとき, $a \times c = b \times c$が成り立つならば
        $a = b$が成り立ち, $c \times a = c \times b$が成り立つならば
        $a = b$が成り立つ.

2)
$a$, $b$, $c$, $d$を集合とするとき, 
\[
  a \neq \phi \wedge c \neq \phi \to (a \times b = c \times d \to b = d), ~~
  b \neq \phi \wedge d \neq \phi \to (a \times b = c \times d \to a = c)
\]
が成り立つ.
またこのことから, 
\[
  (a \neq \phi \wedge b \neq \phi) \wedge (c \neq \phi \wedge d \neq \phi) \to (a \times b = c \times d \to a = c \wedge b = d)
\]
が成り立ち, 更に次の($**$)が成り立つ: 

($**$) ~~$a \times b = c \times d$が成り立つとき, $a$と$c$が共に空でなければ$b = d$が成り立ち, 
         $b$と$d$が共に空でなければ$a = c$が成り立つ.
\end{thm}


\noindent{\bf 証明}
~1)
定理 \ref{sthmproductsubsetinverse}より
\begin{align*}
  &c \neq \phi \to (a \times c \subset b \times c \to a \subset b), ~~
  c \neq \phi \to (b \times c \subset a \times c \to b \subset a), \\
  \mbox{} \\
  &c \neq \phi \to (c \times a \subset c \times b \to a \subset b), ~~
  c \neq \phi \to (c \times b \subset c \times a \to b \subset a)
\end{align*}
がすべて成り立つから, このはじめの二つから, 推論法則 \ref{dedprewedge}によって
\[
\tag{1}
  c \neq \phi \to (a \times c \subset b \times c \to a \subset b) \wedge (b \times c \subset a \times c \to b \subset a)
\]
が成り立ち, 後の二つから, 同じく推論法則 \ref{dedprewedge}によって
\[
\tag{2}
  c \neq \phi \to (c \times a \subset c \times b \to a \subset b) \wedge (c \times b \subset c \times a \to b \subset a)
\]
が成り立つ.
またThm \ref{1atb1w1ctd1t1awctbwd1}より
\begin{multline*}
\tag{3}
  (a \times c \subset b \times c \to a \subset b) \wedge (b \times c \subset a \times c \to b \subset a) \\
  \to (a \times c \subset b \times c \wedge b \times c \subset a \times c \to a \subset b \wedge b \subset a), 
\end{multline*}
\begin{multline*}
\tag{4}
  (c \times a \subset c \times b \to a \subset b) \wedge (c \times b \subset c \times a \to b \subset a) \\
  \to (c \times a \subset c \times b \wedge c \times b \subset c \times a \to a \subset b \wedge b \subset a)
\end{multline*}
が共に成り立つ.
また定理 \ref{sthmaxiom1}と推論法則 \ref{dedequiv}により, 
\begin{align*}
  \tag{5}
  a \times c = b \times c &\to a \times c \subset b \times c \wedge b \times c \subset a \times c, \\
  \mbox{} \\
  \tag{6}
  c \times a = c \times b &\to c \times a \subset c \times b \wedge c \times b \subset c \times a, \\
  \mbox{} \\
  \tag{7}
  a \subset b \wedge b \subset a &\to a = b
\end{align*}
がすべて成り立つ.
そこで(5)と(6)にそれぞれ推論法則 \ref{dedaddf}を適用して, 
\begin{align*}
  \tag{8}
  (a \times c \subset b \times c \wedge b \times c \subset a \times c \to a \subset b \wedge b \subset a) &\to 
  (a \times c = b \times c \to a \subset b \wedge b \subset a), \\
  \mbox{} \\
  \tag{9}
  (c \times a \subset c \times b \wedge c \times b \subset c \times a \to a \subset b \wedge b \subset a) &\to 
  (c \times a = c \times b \to a \subset b \wedge b \subset a)
\end{align*}
が共に成り立つ.
また(7)に推論法則 \ref{dedaddb}を適用して, 
\begin{align*}
  \tag{10}
  (a \times c = b \times c \to a \subset b \wedge b \subset a) &\to (a \times c = b \times c \to a = b), \\
  \mbox{} \\
  \tag{11}
  (c \times a = c \times b \to a \subset b \wedge b \subset a) &\to (c \times a = c \times b \to a = b)
\end{align*}
が共に成り立つ.
そこで(1), (3), (8), (10)から, 推論法則 \ref{dedmmp}によって
\[
  c \neq \phi \to (a \times c = b \times c \to a = b)
\]
が成り立ち, (2), (4), (9), (11)から, 同じく推論法則 \ref{dedmmp}によって
\[
  c \neq \phi \to (c \times a = c \times b \to a = b)
\]
が成り立つことがわかる.
($*$)が成り立つことは, これらと推論法則 \ref{dedmp}によって明らかである.

\noindent
2)
定理 \ref{sthmproductsubsetinverse}より
\begin{align*}
  &a \neq \phi \to (a \times b \subset c \times d \to b \subset d), ~~
  c \neq \phi \to (c \times d \subset a \times b \to d \subset b), \\
  \mbox{} \\
  &b \neq \phi \to (a \times b \subset c \times d \to a \subset c), ~~
  d \neq \phi \to (c \times d \subset a \times b \to c \subset a)
\end{align*}
がすべて成り立つから, このはじめの二つから, 推論法則 \ref{dedfromaddw}によって
\[
\tag{12}
  a \neq \phi \wedge c \neq \phi \to (a \times b \subset c \times d \to b \subset d) \wedge (c \times d \subset a \times b \to d \subset b)
\]
が成り立ち, 後の二つから, 同じく推論法則 \ref{dedfromaddw}によって
\[
\tag{13}
  b \neq \phi \wedge d \neq \phi \to (a \times b \subset c \times d \to a \subset c) \wedge (c \times d \subset a \times b \to c \subset a)
\]
が成り立つ.
またThm \ref{1atb1w1ctd1t1awctbwd1}より
\begin{multline*}
\tag{14}
  (a \times b \subset c \times d \to b \subset d) \wedge (c \times d \subset a \times b \to d \subset b) \\
  \to (a \times b \subset c \times d \wedge c \times d \subset a \times b \to b \subset d \wedge d \subset b), 
\end{multline*}
\begin{multline*}
\tag{15}
  (a \times b \subset c \times d \to a \subset c) \wedge (c \times d \subset a \times b \to c \subset a) \\
  \to (a \times b \subset c \times d \wedge c \times d \subset a \times b \to a \subset c \wedge c \subset a)
\end{multline*}
が共に成り立つ.
また定理 \ref{sthmaxiom1}と推論法則 \ref{dedequiv}により, 
\begin{align*}
  \tag{16}
  a \times b = c \times d &\to a \times b \subset c \times d \wedge c \times d \subset a \times b, \\
  \mbox{} \\
  \tag{17}
  b \subset d \wedge d \subset b &\to b = d, \\
  \mbox{} \\
  \tag{18}
  a \subset c \wedge c \subset a &\to a = c
\end{align*}
がすべて成り立つ.
そこで(16)に推論法則 \ref{dedaddf}を適用して, 
\begin{align*}
  \tag{19}
  (a \times b \subset c \times d \wedge c \times d \subset a \times b \to b \subset d \wedge d \subset b) &\to 
  (a \times b = c \times d \to b \subset d \wedge d \subset b), \\
  \mbox{} \\
  \tag{20}
  (a \times b \subset c \times d \wedge c \times d \subset a \times b \to a \subset c \wedge c \subset a) &\to 
  (a \times b = c \times d \to a \subset c \wedge c \subset a)
\end{align*}
が共に成り立つ.
また(17)と(18)にそれぞれ推論法則 \ref{dedaddb}を適用して, 
\begin{align*}
  \tag{21}
  (a \times b = c \times d \to b \subset d \wedge d \subset b) &\to (a \times b = c \times d \to b = d), \\
  \mbox{} \\
  \tag{22}
  (a \times b = c \times d \to a \subset c \wedge c \subset a) &\to (a \times b = c \times d \to a = c)
\end{align*}
が共に成り立つ.
そこで(12), (14), (19), (21)から, 推論法則 \ref{dedmmp}によって
\[
\tag{23}
  a \neq \phi \wedge c \neq \phi \to (a \times b = c \times d \to b = d)
\]
が成り立ち, (13), (15), (20), (22)から, 同じく推論法則 \ref{dedmmp}によって
\[
\tag{24}
  b \neq \phi \wedge d \neq \phi \to (a \times b = c \times d \to a = c)
\]
が成り立つことがわかる.
そこで(23), (24)から, 推論法則 \ref{dedfromaddw}によって
\[
\tag{25}
  (a \neq \phi \wedge c \neq \phi) \wedge (b \neq \phi \wedge d \neq \phi) \to 
  (a \times b = c \times d \to b = d) \wedge (a \times b = c \times d \to a = c)
\]
が成り立つ.
またThm \ref{1atb1w1atc1t1atbwc1}より
\[
\tag{26}
  (a \times b = c \times d \to b = d) \wedge (a \times b = c \times d \to a = c) \to 
  (a \times b = c \times d \to b = d \wedge a = c)
\]
が成り立つ.
またThm \ref{awbtbwa}より$b = d \wedge a = c \to a = c \wedge b = d$が成り立つから, 
推論法則 \ref{dedaddb}により
\[
\tag{27}
  (a \times b = c \times d \to b = d \wedge a = c) \to (a \times b = c \times d \to a = c \wedge b = d)
\]
が成り立つ.
またThm \ref{1awb1wctaw1bwc1}より
\begin{align*}
  \tag{28}
  (a \neq \phi \wedge b \neq \phi) \wedge (c \neq \phi \wedge d \neq \phi) &\to 
  a \neq \phi \wedge (b \neq \phi \wedge (c \neq \phi \wedge d \neq \phi)), \\
  \mbox{} \\
  \tag{29}
  (c \neq \phi \wedge b \neq \phi) \wedge d \neq \phi &\to 
  c \neq \phi \wedge (b \neq \phi \wedge d \neq \phi)
\end{align*}
が共に成り立つ.
そこで(29)から, 推論法則 \ref{dedaddw}によって
\[
\tag{30}
  a \neq \phi \wedge ((c \neq \phi \wedge b \neq \phi) \wedge d \neq \phi) \to 
  a \neq \phi \wedge (c \neq \phi \wedge (b \neq \phi \wedge d \neq \phi))
\]
が成り立つ.
またThm \ref{aw1bwc1t1awb1wc}より
\begin{align*}
  \tag{31}
  b \neq \phi \wedge (c \neq \phi \wedge d \neq \phi) &\to 
  (b \neq \phi \wedge c \neq \phi) \wedge d \neq \phi, \\
  \mbox{} \\
  \tag{32}
  a \neq \phi \wedge (c \neq \phi \wedge (b \neq \phi \wedge d \neq \phi)) &\to 
  (a \neq \phi \wedge c \neq \phi) \wedge (b \neq \phi \wedge d \neq \phi)
\end{align*}
が共に成り立つ.
そこで(31)から, 推論法則 \ref{dedaddw}によって
\[
\tag{33}
  a \neq \phi \wedge (b \neq \phi \wedge (c \neq \phi \wedge d \neq \phi)) \to 
  a \neq \phi \wedge ((b \neq \phi \wedge c \neq \phi) \wedge d \neq \phi)
\]
が成り立つ.
またThm \ref{awbtbwa}より$b \neq \phi \wedge c \neq \phi \to c \neq \phi \wedge b \neq \phi$が成り立つから, 
推論法則 \ref{dedaddw}を二回用いて
\[
\tag{34}
  a \neq \phi \wedge ((b \neq \phi \wedge c \neq \phi) \wedge d \neq \phi) \to 
  a \neq \phi \wedge ((c \neq \phi \wedge b \neq \phi) \wedge d \neq \phi)
\]
が成り立つ.
そこで(28), (33), (34), (30), (32), (25), (26), (27)にこの順で推論法則 \ref{dedmmp}を適用していき, 
\[
  (a \neq \phi \wedge b \neq \phi) \wedge (c \neq \phi \wedge d \neq \phi) \to 
  (a \times b = c \times d \to a = c \wedge b = d)
\]
が成り立つことがわかる.

さていま$a \times b = c \times d$が成り立つとする.
このとき, $a$と$c$が共に空でなければ, 推論法則 \ref{dedwedge}によって
$a \neq \phi \wedge c \neq \phi$が成り立つから, これと
(23)から, 推論法則 \ref{dedmp}によって
$a \times b = c \times d \to b = d$が成り立ち, これと
$a \times b = c \times d$から, 再び推論法則 \ref{dedmp}によって
$b = d$が成り立つ.
また$b$と$d$が共に空でなければ, 推論法則 \ref{dedwedge}によって
$b \neq \phi \wedge d \neq \phi$が成り立つから, これと
(24)から, 推論法則 \ref{dedmp}によって
$a \times b = c \times d \to a = c$が成り立ち, これと
$a \times b = c \times d$から, 再び推論法則 \ref{dedmp}によって
$a = c$が成り立つ.
これで($**$)が成り立つことも示された.
\halmos




\mathstrut
\begin{thm}
\label{sthmproduct=eq}%定理
\mbox{}

1)
$a$, $b$, $c$を集合とするとき, 
\[
  c \neq \phi \to (a = b \leftrightarrow a \times c = b \times c), ~~
  c \neq \phi \to (a = b \leftrightarrow c \times a = c \times b)
\]
が成り立つ.
またこのことから, 次の($*$)が成り立つ: 

($*$) ~~$c$が空でなければ, $a = b \leftrightarrow a \times c = b \times c$と
        $a = b \leftrightarrow c \times a = c \times b$が共に成り立つ.

2)
$a$, $b$, $c$, $d$を集合とするとき, 
\[
  (a \neq \phi \wedge b \neq \phi) \wedge (c \neq \phi \wedge d \neq \phi) \to 
  (a = c \wedge b = d \leftrightarrow a \times b = c \times d)
\]
が成り立つ.
またこのことから, 次の($**$)が成り立つ: 

($**$) ~~$a$, $b$, $c$, $d$がいずれも空でなければ, 
         $a = c \wedge b = d \leftrightarrow a \times b = c \times d$が成り立つ.
\end{thm}


\noindent{\bf 証明}
~1)
定理 \ref{sthmproduct=}より
\[
  a = b \to a \times c = b \times c, ~~
  a = b \to c \times a = c \times b
\]
が共に成り立つから, 推論法則 \ref{deds1}により, 
\[
  c \neq \phi \to (a = b \to a \times c = b \times c), ~~
  c \neq \phi \to (a = b \to c \times a = c \times b)
\]
が共に成り立つ.
また定理 \ref{sthmproduct=inverse}より
\[
  c \neq \phi \to (a \times c = b \times c \to a = b), ~~
  c \neq \phi \to (c \times a = c \times b \to a = b)
\]
が共に成り立つ.
そこでこれらから, 推論法則 \ref{dedprewedge}によって
\[
  c \neq \phi \to (a = b \leftrightarrow a \times c = b \times c), ~~
  c \neq \phi \to (a = b \leftrightarrow c \times a = c \times b)
\]
が共に成り立つことがわかる.
($*$)が成り立つことは, これらと推論法則 \ref{dedmp}によって明らかである.

\noindent
2)
定理 \ref{sthmproduct=}より$
a = c \wedge b = d \to a \times b = c \times d$が成り立つから, 
推論法則 \ref{deds1}により
\[
  (a \neq \phi \wedge b \neq \phi) \wedge (c \neq \phi \wedge d \neq \phi) \to 
  (a = c \wedge b = d \to a \times b = c \times d)
\]
が成り立つ.
また定理 \ref{sthmproduct=inverse}より
\[
  (a \neq \phi \wedge b \neq \phi) \wedge (c \neq \phi \wedge d \neq \phi) \to 
  (a \times b = c \times d \to a = c \wedge b = d)
\]
が成り立つ.
そこでこれらから, 推論法則 \ref{dedprewedge}によって
\[
\tag{1}
  (a \neq \phi \wedge b \neq \phi) \wedge (c \neq \phi \wedge d \neq \phi) \to 
  (a = c \wedge b = d \leftrightarrow a \times b = c \times d)
\]
が成り立つ.

いま$a$, $b$, $c$, $d$がいずれも空でないとすれば, 推論法則 \ref{dedwedge}により
$(a \neq \phi \wedge b \neq \phi) \wedge (c \neq \phi \wedge d \neq \phi)$が成り立つから, 
これと(1)から, 推論法則 \ref{dedmp}によって
$a = c \wedge b = d \leftrightarrow a \times b = c \times d$が成り立つ.
これで($**$)が成り立つことが示された.
\halmos




\mathstrut
\begin{thm}
\label{sthmuopairproduct}%定理
$a$, $b$, $c$を集合とするとき, 
\[
  \{a, b\} \times \{c\} = \{(a, c), (b, c)\}, ~~
  \{a\} \times \{b, c\} = \{(a, b), (a, c)\}
\]
が成り立つ.
特に, 
\[
  \{a\} \times \{b\} = \{(a, b)\}
\]
が成り立つ.
\end{thm}


\noindent{\bf 証明}
~一般に, $a$, $b$, $c$, $d$, $e$を集合とするとき
\[
\tag{1}
  {\rm Pair}(a) \wedge a \in \{(b, c), (d, e)\} \leftrightarrow a \in \{(b, c), (d, e)\}
\]
が成り立つ.
実際定理 \ref{sthmuopairbasis}と推論法則 \ref{dedequiv}により, 
\[
\tag{2}
  a \in \{(b, c), (d, e)\} \to a = (b, c) \vee a = (d, e)
\]
が成り立つ.
また定理 \ref{sthmbigpair}より
\[
  a = (b, c) \to {\rm Pair}(a), ~~
  a = (d, e) \to {\rm Pair}(a)
\]
が共に成り立つから, 推論法則 \ref{deddil}によって
\[
\tag{3}
  a = (b, c) \vee a = (d, e) \to {\rm Pair}(a)
\]
が成り立つ.
そこで(2), (3)から, 推論法則 \ref{dedmmp}によって
\[
  a \in \{(b, c), (d, e)\} \to {\rm Pair}(a)
\]
が成り立ち, これから推論法則 \ref{dedawblatrue1}によって(1)が成り立つことがわかる.

さていま$x$を$a$, $b$, $c$のいずれの記号列の中にも自由変数として現れない, 定数でない文字とする.
このとき変数法則 \ref{valnset}, \ref{valpair}, \ref{valproduct}からわかるように, 
$x$は$\{a, b\} \times \{c\}$, $\{(a, c), (b, c)\}$, $\{a\} \times \{b, c\}$, $\{(a, b), (a, c)\}$の
いずれの記号列の中にも自由変数として現れない.
そして定理 \ref{sthmproductelement}より
\begin{align*}
  \tag{4}
  x \in \{a, b\} \times \{c\} &\leftrightarrow {\rm Pair}(x) \wedge ({\rm pr}_{1}(x) \in \{a, b\} \wedge {\rm pr}_{2}(x) \in \{c\}), \\
  \mbox{} \\
  \tag{5}
  x \in \{a\} \times \{b, c\} &\leftrightarrow {\rm Pair}(x) \wedge ({\rm pr}_{1}(x) \in \{a\} \wedge {\rm pr}_{2}(x) \in \{b, c\})
\end{align*}
が共に成り立つ.
また定理 \ref{sthmbigpairpr}より
\[
\tag{6}
  {\rm Pair}(x) \leftrightarrow x = ({\rm pr}_{1}(x), {\rm pr}_{2}(x))
\]
が成り立つ.
また定理 \ref{sthmuopairbasis}より
\begin{align*}
  \tag{7}
  {\rm pr}_{1}(x) \in \{a, b\} &\leftrightarrow {\rm pr}_{1}(x) = a \vee {\rm pr}_{1}(x) = b, \\
  \mbox{} \\
  \tag{8}
  {\rm pr}_{2}(x) \in \{b, c\} &\leftrightarrow {\rm pr}_{2}(x) = b \vee {\rm pr}_{2}(x) = c
\end{align*}
が共に成り立ち, 定理 \ref{sthmsingletonbasis}より
\begin{align*}
  \tag{9}
  {\rm pr}_{2}(x) \in \{c\} &\leftrightarrow {\rm pr}_{2}(x) = c, \\
  \mbox{} \\
  \tag{10}
  {\rm pr}_{1}(x) \in \{a\} &\leftrightarrow {\rm pr}_{1}(x) = a
\end{align*}
が共に成り立つ.
そこで(7)と(9), (10)と(8)にそれぞれ推論法則 \ref{dedaddeqw}を適用して, 
\begin{align*}
  \tag{11}
  {\rm pr}_{1}(x) \in \{a, b\} \wedge {\rm pr}_{2}(x) \in \{c\} &\leftrightarrow 
  ({\rm pr}_{1}(x) = a \vee {\rm pr}_{1}(x) = b) \wedge {\rm pr}_{2}(x) = c, \\
  \mbox{} \\
  \tag{12}
  {\rm pr}_{1}(x) \in \{a\} \wedge {\rm pr}_{2}(x) \in \{b, c\} &\leftrightarrow 
  {\rm pr}_{1}(x) = a \wedge ({\rm pr}_{2}(x) = b \vee {\rm pr}_{2}(x) = c)
\end{align*}
が共に成り立つ.
またThm \ref{aw1bvc1l1awb1v1awc1}より
\begin{align*}
  \tag{13}
  ({\rm pr}_{1}(x) = a \vee {\rm pr}_{1}(x) = b) \wedge {\rm pr}_{2}(x) = c &\leftrightarrow 
  ({\rm pr}_{1}(x) = a \wedge {\rm pr}_{2}(x) = c) \vee ({\rm pr}_{1}(x) = b \wedge {\rm pr}_{2}(x) = c), \\
  \mbox{} \\
  \tag{14}
  {\rm pr}_{1}(x) = a \wedge ({\rm pr}_{2}(x) = b \vee {\rm pr}_{2}(x) = c) &\leftrightarrow 
  ({\rm pr}_{1}(x) = a \wedge {\rm pr}_{2}(x) = b) \vee ({\rm pr}_{1}(x) = a \wedge {\rm pr}_{2}(x) = c)
\end{align*}
が共に成り立つ.
また定理 \ref{sthmpair}と推論法則 \ref{dedeqch}により, 
\begin{align*}
  \tag{15}
  {\rm pr}_{1}(x) = a \wedge {\rm pr}_{2}(x) = c &\leftrightarrow ({\rm pr}_{1}(x), {\rm pr}_{2}(x)) = (a, c), \\
  \mbox{} \\
  \tag{16}
  {\rm pr}_{1}(x) = b \wedge {\rm pr}_{2}(x) = c &\leftrightarrow ({\rm pr}_{1}(x), {\rm pr}_{2}(x)) = (b, c), \\
  \mbox{} \\
  \tag{17}
  {\rm pr}_{1}(x) = a \wedge {\rm pr}_{2}(x) = b &\leftrightarrow ({\rm pr}_{1}(x), {\rm pr}_{2}(x)) = (a, b)
\end{align*}
がすべて成り立つ.
そこで(15)と(16)から, 推論法則 \ref{dedaddeqv}によって
\begin{multline*}
\tag{18}
  ({\rm pr}_{1}(x) = a \wedge {\rm pr}_{2}(x) = c) \vee ({\rm pr}_{1}(x) = b \wedge {\rm pr}_{2}(x) = c) \\
  \leftrightarrow ({\rm pr}_{1}(x), {\rm pr}_{2}(x)) = (a, c) \vee ({\rm pr}_{1}(x), {\rm pr}_{2}(x)) = (b, c)
\end{multline*}
が成り立ち, (17)と(15)から, 同じく推論法則 \ref{dedaddeqv}によって
\begin{multline*}
\tag{19}
  ({\rm pr}_{1}(x) = a \wedge {\rm pr}_{2}(x) = b) \vee ({\rm pr}_{1}(x) = a \wedge {\rm pr}_{2}(x) = c) \\
  \leftrightarrow ({\rm pr}_{1}(x), {\rm pr}_{2}(x)) = (a, b) \vee ({\rm pr}_{1}(x), {\rm pr}_{2}(x)) = (a, c)
\end{multline*}
が成り立つ.
また定理 \ref{sthmuopairbasis}と推論法則 \ref{dedeqch}により, 
\begin{align*}
  \tag{20}
  ({\rm pr}_{1}(x), {\rm pr}_{2}(x)) = (a, c) \vee ({\rm pr}_{1}(x), {\rm pr}_{2}(x)) = (b, c) &\leftrightarrow 
  ({\rm pr}_{1}(x), {\rm pr}_{2}(x)) \in \{(a, c), (b, c)\}, \\
  \mbox{} \\
  \tag{21}
  ({\rm pr}_{1}(x), {\rm pr}_{2}(x)) = (a, b) \vee ({\rm pr}_{1}(x), {\rm pr}_{2}(x)) = (a, c) &\leftrightarrow 
  ({\rm pr}_{1}(x), {\rm pr}_{2}(x)) \in \{(a, b), (a, c)\}
\end{align*}
が共に成り立つ.
そこで(11), (13), (18), (20)から, 推論法則 \ref{dedeqtrans}によって
\[
\tag{22}
  {\rm pr}_{1}(x) \in \{a, b\} \wedge {\rm pr}_{2}(x) \in \{c\} \leftrightarrow 
  ({\rm pr}_{1}(x), {\rm pr}_{2}(x)) \in \{(a, c), (b, c)\}
\]
が成り立ち, (12), (14), (19), (21)から, 同じく推論法則 \ref{dedeqtrans}によって
\[
\tag{23}
  {\rm pr}_{1}(x) \in \{a\} \wedge {\rm pr}_{2}(x) \in \{b, c\} \leftrightarrow 
  ({\rm pr}_{1}(x), {\rm pr}_{2}(x)) \in \{(a, b), (a, c)\}
\]
が成り立つことがわかる.
そこで(6)と(22)から, 推論法則 \ref{dedaddeqw}によって
\begin{multline*}
\tag{24}
  {\rm Pair}(x) \wedge ({\rm pr}_{1}(x) \in \{a, b\} \wedge {\rm pr}_{2}(x) \in \{c\}) \\
  \leftrightarrow x = ({\rm pr}_{1}(x), {\rm pr}_{2}(x)) \wedge ({\rm pr}_{1}(x), {\rm pr}_{2}(x)) \in \{(a, c), (b, c)\}
\end{multline*}
が成り立ち, (6)と(23)から, 同じく推論法則 \ref{dedaddeqw}によって
\begin{multline*}
\tag{25}
  {\rm Pair}(x) \wedge ({\rm pr}_{1}(x) \in \{a\} \wedge {\rm pr}_{2}(x) \in \{b, c\}) \\
  \leftrightarrow x = ({\rm pr}_{1}(x), {\rm pr}_{2}(x)) \wedge ({\rm pr}_{1}(x), {\rm pr}_{2}(x)) \in \{(a, b), (a, c)\}
\end{multline*}
が成り立つ.
またThm \ref{thmfroms5eq}より
\begin{multline*}
  x = ({\rm pr}_{1}(x), {\rm pr}_{2}(x)) \wedge (x|x)(x \in \{(a, c), (b, c)\}) \\
  \leftrightarrow x = ({\rm pr}_{1}(x), {\rm pr}_{2}(x)) \wedge (({\rm pr}_{1}(x), {\rm pr}_{2}(x))|x)(x \in \{(a, c), (b, c)\}), 
\end{multline*}
\begin{multline*}
  x = ({\rm pr}_{1}(x), {\rm pr}_{2}(x)) \wedge (x|x)(x \in \{(a, b), (a, c)\}) \\
  \leftrightarrow x = ({\rm pr}_{1}(x), {\rm pr}_{2}(x)) \wedge (({\rm pr}_{1}(x), {\rm pr}_{2}(x))|x)(x \in \{(a, b), (a, c)\})
\end{multline*}
が共に成り立つが, 上述のように$x$は$\{(a, c), (b, c)\}$及び$\{(a, b), (a, c)\}$の中に
自由変数として現れないから, このことと代入法則 \ref{substsame}, \ref{substfree}, \ref{substfund}により, 
上記の記号列はそれぞれ
\begin{align*}
  x = ({\rm pr}_{1}(x), {\rm pr}_{2}(x)) \wedge x \in \{(a, c), (b, c)\} &\leftrightarrow 
  x = ({\rm pr}_{1}(x), {\rm pr}_{2}(x)) \wedge ({\rm pr}_{1}(x), {\rm pr}_{2}(x)) \in \{(a, c), (b, c)\}, \\
  \mbox{} \\
  x = ({\rm pr}_{1}(x), {\rm pr}_{2}(x)) \wedge x \in \{(a, b), (a, c)\} &\leftrightarrow 
  x = ({\rm pr}_{1}(x), {\rm pr}_{2}(x)) \wedge ({\rm pr}_{1}(x), {\rm pr}_{2}(x)) \in \{(a, b), (a, c)\}
\end{align*}
と一致する.
よってこれらが共に定理となる.
そこでこれらにそれぞれ推論法則 \ref{dedeqch}を適用して, 
\begin{multline*}
\tag{26}
  x = ({\rm pr}_{1}(x), {\rm pr}_{2}(x)) \wedge ({\rm pr}_{1}(x), {\rm pr}_{2}(x)) \in \{(a, c), (b, c)\} \\
  \leftrightarrow x = ({\rm pr}_{1}(x), {\rm pr}_{2}(x)) \wedge x \in \{(a, c), (b, c)\}, 
\end{multline*}
\begin{multline*}
\tag{27}
  x = ({\rm pr}_{1}(x), {\rm pr}_{2}(x)) \wedge ({\rm pr}_{1}(x), {\rm pr}_{2}(x)) \in \{(a, b), (a, c)\} \\
  \leftrightarrow x = ({\rm pr}_{1}(x), {\rm pr}_{2}(x)) \wedge x \in \{(a, b), (a, c)\}
\end{multline*}
が共に成り立つ.
また(6)に推論法則 \ref{dedeqch}を適用して
\[
  x = ({\rm pr}_{1}(x), {\rm pr}_{2}(x)) \leftrightarrow {\rm Pair}(x)
\]
が成り立つから, これに推論法則 \ref{dedaddeqw}を適用して, 
\begin{align*}
  \tag{28}
  x = ({\rm pr}_{1}(x), {\rm pr}_{2}(x)) \wedge x \in \{(a, c), (b, c)\} &\leftrightarrow 
  {\rm Pair}(x) \wedge x \in \{(a, c), (b, c)\}, \\
  \mbox{} \\
  \tag{29}
  x = ({\rm pr}_{1}(x), {\rm pr}_{2}(x)) \wedge x \in \{(a, b), (a, c)\} &\leftrightarrow 
  {\rm Pair}(x) \wedge x \in \{(a, b), (a, c)\}
\end{align*}
が共に成り立つ.
また(1)が成り立つことから, 
\begin{align*}
  \tag{30}
  {\rm Pair}(x) \wedge x \in \{(a, c), (b, c)\} &\leftrightarrow x \in \{(a, c), (b, c)\}, \\
  \mbox{} \\
  \tag{31}
  {\rm Pair}(x) \wedge x \in \{(a, b), (a, c)\} &\leftrightarrow x \in \{(a, b), (a, c)\}
\end{align*}
が共に成り立つ.
そこで(4), (24), (26), (28), (30)から, 推論法則 \ref{dedeqtrans}によって
\[
\tag{32}
  x \in \{a, b\} \times \{c\} \leftrightarrow x \in \{(a, c), (b, c)\}
\]
が成り立ち, (5), (25), (27), (29), (31)から, 同じく推論法則 \ref{dedeqtrans}によって
\[
\tag{33}
  x \in \{a\} \times \{b, c\} \leftrightarrow x \in \{(a, b), (a, c)\}
\]
が成り立つことがわかる.
いま$x$は定数でなく, 上述のように$\{a, b\} \times \{c\}$, $\{(a, c), (b, c)\}$, 
$\{a\} \times \{b, c\}$, $\{(a, b), (a, c)\}$のいずれの記号列の中にも自由変数として現れないから, 
(32)と(33)から, 定理 \ref{sthmset=}によって
\[
  \{a, b\} \times \{c\} = \{(a, c), (b, c)\}, ~~
  \{a\} \times \{b, c\} = \{(a, b), (a, c)\}
\]
が共に成り立つ.
そこで特に, この後者で$c$を$b$に置き換えて, 
$\{a\} \times \{b, b\} = \{(a, b), (a, b)\}$, 
即ち$\{a\} \times \{b\} = \{(a, b)\}$も成り立つことがわかる.
\halmos




\mathstrut
\begin{thm}
\label{sthmcupproduct}%定理
$a$, $b$, $c$を集合とするとき, 
\[
  (a \cup b) \times c = (a \times c) \cup (b \times c), ~~
  a \times (b \cup c) = (a \times b) \cup (a \times c)
\]
が成り立つ.
\end{thm}


\noindent{\bf 証明}
~$x$を, $a$, $b$, $c$の中に自由変数として現れない, 定数でない文字とする.
このとき変数法則 \ref{valcup}, \ref{valproduct}からわかるように, 
$x$は$(a \cup b) \times c$, $(a \times c) \cup (b \times c)$, 
$a \times (b \cup c)$, $(a \times b) \cup (a \times c)$のいずれの記号列の中にも
自由変数として現れない.
そして定理 \ref{sthmproductelement}より, 
\begin{align*}
  \tag{1}
  x \in (a \cup b) \times c &\leftrightarrow 
  {\rm Pair}(x) \wedge ({\rm pr}_{1}(x) \in a \cup b \wedge {\rm pr}_{2}(x) \in c), \\
  \mbox{}& \\
  \tag{2}
  x \in a \times (b \cup c) &\leftrightarrow 
  {\rm Pair}(x) \wedge ({\rm pr}_{1}(x) \in a \wedge {\rm pr}_{2}(x) \in b \cup c)
\end{align*}
が共に成り立つ.
また定理 \ref{sthmcupbasis}より
\begin{align*}
  {\rm pr}_{1}(x) \in a \cup b &\leftrightarrow {\rm pr}_{1}(x) \in a \vee {\rm pr}_{1}(x) \in b, \\
  \mbox{}& \\
  {\rm pr}_{2}(x) \in b \cup c &\leftrightarrow {\rm pr}_{2}(x) \in b \vee {\rm pr}_{2}(x) \in c
\end{align*}
が共に成り立つから, 推論法則 \ref{dedaddeqw}によって
\begin{align*}
  \tag{3}
  {\rm pr}_{1}(x) \in a \cup b \wedge {\rm pr}_{2}(x) \in c &\leftrightarrow 
  ({\rm pr}_{1}(x) \in a \vee {\rm pr}_{1}(x) \in b) \wedge {\rm pr}_{2}(x) \in c, \\
  \mbox{}& \\
  \tag{4}
  {\rm pr}_{1}(x) \in a \wedge {\rm pr}_{2}(x) \in b \cup c &\leftrightarrow 
  {\rm pr}_{1}(x) \in a \wedge ({\rm pr}_{2}(x) \in b \vee {\rm pr}_{2}(x) \in c)
\end{align*}
が共に成り立つ.
またThm \ref{aw1bvc1l1awb1v1awc1}より
\begin{align*}
  \tag{5}
  ({\rm pr}_{1}(x) \in a \vee {\rm pr}_{1}(x) \in b) \wedge {\rm pr}_{2}(x) \in c &\leftrightarrow 
  ({\rm pr}_{1}(x) \in a \wedge {\rm pr}_{2}(x) \in c) \vee ({\rm pr}_{1}(x) \in b \wedge {\rm pr}_{2}(x) \in c), \\
  \mbox{}& \\
  \tag{6}
  {\rm pr}_{1}(x) \in a \wedge ({\rm pr}_{2}(x) \in b \vee {\rm pr}_{2}(x) \in c) &\leftrightarrow 
  ({\rm pr}_{1}(x) \in a \wedge {\rm pr}_{2}(x) \in b) \vee ({\rm pr}_{1}(x) \in a \wedge {\rm pr}_{2}(x) \in c)
\end{align*}
が共に成り立つ.
そこで(3)と(5), (4)と(6)にそれぞれ推論法則 \ref{dedeqtrans}を適用して, 
\begin{align*}
  {\rm pr}_{1}(x) \in a \cup b \wedge {\rm pr}_{2}(x) \in c &\leftrightarrow 
  ({\rm pr}_{1}(x) \in a \wedge {\rm pr}_{2}(x) \in c) \vee ({\rm pr}_{1}(x) \in b \wedge {\rm pr}_{2}(x) \in c), \\
  \mbox{}& \\
  {\rm pr}_{1}(x) \in a \wedge {\rm pr}_{2}(x) \in b \cup c &\leftrightarrow 
  ({\rm pr}_{1}(x) \in a \wedge {\rm pr}_{2}(x) \in b) \vee ({\rm pr}_{1}(x) \in a \wedge {\rm pr}_{2}(x) \in c)
\end{align*}
が共に成り立ち, これらから, 推論法則 \ref{dedaddeqw}によって
\begin{multline*}
\tag{7}
  {\rm Pair}(x) \wedge ({\rm pr}_{1}(x) \in a \cup b \wedge {\rm pr}_{2}(x) \in c) \\
  \leftrightarrow {\rm Pair}(x) \wedge (({\rm pr}_{1}(x) \in a \wedge {\rm pr}_{2}(x) \in c) \vee ({\rm pr}_{1}(x) \in b \wedge {\rm pr}_{2}(x) \in c)), 
\end{multline*}
\begin{multline*}
\tag{8}
  {\rm Pair}(x) \wedge ({\rm pr}_{1}(x) \in a \wedge {\rm pr}_{2}(x) \in b \cup c) \\
  \leftrightarrow {\rm Pair}(x) \wedge (({\rm pr}_{1}(x) \in a \wedge {\rm pr}_{2}(x) \in b) \vee ({\rm pr}_{1}(x) \in a \wedge {\rm pr}_{2}(x) \in c))
\end{multline*}
が共に成り立つ.
またThm \ref{aw1bvc1l1awb1v1awc1}より
\begin{multline*}
\tag{9}
  {\rm Pair}(x) \wedge (({\rm pr}_{1}(x) \in a \wedge {\rm pr}_{2}(x) \in c) \vee ({\rm pr}_{1}(x) \in b \wedge {\rm pr}_{2}(x) \in c)) \\
  \leftrightarrow ({\rm Pair}(x) \wedge ({\rm pr}_{1}(x) \in a \wedge {\rm pr}_{2}(x) \in c)) \vee ({\rm Pair}(x) \wedge ({\rm pr}_{1}(x) \in b \wedge {\rm pr}_{2}(x) \in c)), 
\end{multline*}
\begin{multline*}
\tag{10}
  {\rm Pair}(x) \wedge (({\rm pr}_{1}(x) \in a \wedge {\rm pr}_{2}(x) \in b) \vee ({\rm pr}_{1}(x) \in a \wedge {\rm pr}_{2}(x) \in c)) \\
  \leftrightarrow ({\rm Pair}(x) \wedge ({\rm pr}_{1}(x) \in a \wedge {\rm pr}_{2}(x) \in b)) \vee ({\rm Pair}(x) \wedge ({\rm pr}_{1}(x) \in a \wedge {\rm pr}_{2}(x) \in c))
\end{multline*}
が共に成り立つ.
また定理 \ref{sthmproductelement}と推論法則 \ref{dedeqch}によって
\begin{align*}
  {\rm Pair}(x) \wedge ({\rm pr}_{1}(x) \in a \wedge {\rm pr}_{2}(x) \in c) &\leftrightarrow x \in a \times c, \\
  \mbox{}& \\
  {\rm Pair}(x) \wedge ({\rm pr}_{1}(x) \in b \wedge {\rm pr}_{2}(x) \in c) &\leftrightarrow x \in b \times c, \\
  \mbox{}& \\
  {\rm Pair}(x) \wedge ({\rm pr}_{1}(x) \in a \wedge {\rm pr}_{2}(x) \in b) &\leftrightarrow x \in a \times b
\end{align*}
がすべて成り立つから, 推論法則 \ref{dedaddeqv}によって
\begin{multline*}
\tag{11}
  ({\rm Pair}(x) \wedge ({\rm pr}_{1}(x) \in a \wedge {\rm pr}_{2}(x) \in c)) \vee ({\rm Pair}(x) \wedge ({\rm pr}_{1}(x) \in b \wedge {\rm pr}_{2}(x) \in c)) \\
  \leftrightarrow x \in a \times c \vee x \in b \times c,
\end{multline*}
\begin{multline*}
\tag{12}
  ({\rm Pair}(x) \wedge ({\rm pr}_{1}(x) \in a \wedge {\rm pr}_{2}(x) \in b)) \vee ({\rm Pair}(x) \wedge ({\rm pr}_{1}(x) \in a \wedge {\rm pr}_{2}(x) \in c)) \\
  \leftrightarrow x \in a \times b \vee x \in a \times c
\end{multline*}
が共に成り立つ.
また定理 \ref{sthmcupbasis}と推論法則 \ref{dedeqch}により, 
\begin{align*}
  \tag{13}
  x \in a \times c \vee x \in b \times c &\leftrightarrow x \in (a \times c) \cup (b \times c), \\
  \mbox{}& \\
  \tag{14}
  x \in a \times b \vee x \in a \times c &\leftrightarrow x \in (a \times b) \cup (a \times c)
\end{align*}
が共に成り立つ.
以上の(1), (7), (9), (11), (13)から, 推論法則 \ref{dedeqtrans}によって
\[
\tag{15}
  x \in (a \cup b) \times c \leftrightarrow x \in (a \times c) \cup (b \times c)
\]
が成り立ち, 
(2), (8), (10), (12), (14)から, やはり推論法則 \ref{dedeqtrans}によって
\[
\tag{16}
  x \in a \times (b \cup c) \leftrightarrow x \in (a \times b) \cup (a \times c)
\]
が成り立つことがわかる.
いま$x$は定数でなく, はじめに述べたように$(a \cup b) \times c$, $(a \times c) \cup (b \times c)$, 
$a \times (b \cup c)$, $(a \times b) \cup (a \times c)$のいずれの記号列の中にも
自由変数として現れないから, (15), (16)から, 定理 \ref{sthmset=}によって
$(a \cup b) \times c = (a \times c) \cup (b \times c)$と
$a \times (b \cup c) = (a \times b) \cup (a \times c)$が共に成り立つ.
\halmos




\mathstrut
\begin{thm}
\label{sthmcapproduct}%定理
$a$, $b$, $c$を集合とするとき, 
\[
  (a \cap b) \times c = (a \times c) \cap (b \times c), ~~
  a \times (b \cap c) = (a \times b) \cap (a \times c)
\]
が成り立つ.
\end{thm}


\noindent{\bf 証明}
~$x$を, $a$, $b$, $c$の中に自由変数として現れない, 定数でない文字とする.
このとき変数法則 \ref{valcap}, \ref{valproduct}からわかるように, 
$x$は$(a \cap b) \times c$, $(a \times c) \cap (b \times c)$, 
$a \times (b \cap c)$, $(a \times b) \cap (a \times c)$のいずれの記号列の中にも
自由変数として現れない.
そして定理 \ref{sthmproductelement}より, 
\begin{align*}
  \tag{1}
  x \in (a \cap b) \times c &\leftrightarrow 
  {\rm Pair}(x) \wedge ({\rm pr}_{1}(x) \in a \cap b \wedge {\rm pr}_{2}(x) \in c), \\
  \mbox{}& \\
  \tag{2}
  x \in a \times (b \cap c) &\leftrightarrow 
  {\rm Pair}(x) \wedge ({\rm pr}_{1}(x) \in a \wedge {\rm pr}_{2}(x) \in b \cap c)
\end{align*}
が共に成り立つ.
また定理 \ref{sthmcapelement}より
\begin{align*}
  {\rm pr}_{1}(x) \in a \cap b &\leftrightarrow {\rm pr}_{1}(x) \in a \wedge {\rm pr}_{1}(x) \in b, \\
  \mbox{}& \\
  {\rm pr}_{2}(x) \in b \cap c &\leftrightarrow {\rm pr}_{2}(x) \in b \wedge {\rm pr}_{2}(x) \in c
\end{align*}
が共に成り立つから, 推論法則 \ref{dedaddeqw}によって
\begin{align*}
  \tag{3}
  {\rm pr}_{1}(x) \in a \cap b \wedge {\rm pr}_{2}(x) \in c &\leftrightarrow 
  ({\rm pr}_{1}(x) \in a \wedge {\rm pr}_{1}(x) \in b) \wedge {\rm pr}_{2}(x) \in c, \\
  \mbox{}& \\
  \tag{4}
  {\rm pr}_{1}(x) \in a \wedge {\rm pr}_{2}(x) \in b \cap c &\leftrightarrow 
  {\rm pr}_{1}(x) \in a \wedge ({\rm pr}_{2}(x) \in b \wedge {\rm pr}_{2}(x) \in c)
\end{align*}
が共に成り立つ.
またThm \ref{aw1bwc1l1awb1w1awc1}より
\begin{align*}
  \tag{5}
  ({\rm pr}_{1}(x) \in a \wedge {\rm pr}_{1}(x) \in b) \wedge {\rm pr}_{2}(x) \in c &\leftrightarrow 
  ({\rm pr}_{1}(x) \in a \wedge {\rm pr}_{2}(x) \in c) \wedge ({\rm pr}_{1}(x) \in b \wedge {\rm pr}_{2}(x) \in c), \\
  \mbox{}& \\
  \tag{6}
  {\rm pr}_{1}(x) \in a \wedge ({\rm pr}_{2}(x) \in b \wedge {\rm pr}_{2}(x) \in c) &\leftrightarrow 
  ({\rm pr}_{1}(x) \in a \wedge {\rm pr}_{2}(x) \in b) \wedge ({\rm pr}_{1}(x) \in a \wedge {\rm pr}_{2}(x) \in c)
\end{align*}
が共に成り立つ.
そこで(3)と(5), (4)と(6)にそれぞれ推論法則 \ref{dedeqtrans}を適用して, 
\begin{align*}
  {\rm pr}_{1}(x) \in a \cap b \wedge {\rm pr}_{2}(x) \in c &\leftrightarrow 
  ({\rm pr}_{1}(x) \in a \wedge {\rm pr}_{2}(x) \in c) \wedge ({\rm pr}_{1}(x) \in b \wedge {\rm pr}_{2}(x) \in c), \\
  \mbox{}& \\
  {\rm pr}_{1}(x) \in a \wedge {\rm pr}_{2}(x) \in b \cap c &\leftrightarrow 
  ({\rm pr}_{1}(x) \in a \wedge {\rm pr}_{2}(x) \in b) \wedge ({\rm pr}_{1}(x) \in a \wedge {\rm pr}_{2}(x) \in c)
\end{align*}
が共に成り立ち, これらから, 推論法則 \ref{dedaddeqw}によって
\begin{multline*}
\tag{7}
  {\rm Pair}(x) \wedge ({\rm pr}_{1}(x) \in a \cap b \wedge {\rm pr}_{2}(x) \in c) \\
  \leftrightarrow {\rm Pair}(x) \wedge (({\rm pr}_{1}(x) \in a \wedge {\rm pr}_{2}(x) \in c) \wedge ({\rm pr}_{1}(x) \in b \wedge {\rm pr}_{2}(x) \in c)), 
\end{multline*}
\begin{multline*}
\tag{8}
  {\rm Pair}(x) \wedge ({\rm pr}_{1}(x) \in a \wedge {\rm pr}_{2}(x) \in b \cap c) \\
  \leftrightarrow {\rm Pair}(x) \wedge (({\rm pr}_{1}(x) \in a \wedge {\rm pr}_{2}(x) \in b) \wedge ({\rm pr}_{1}(x) \in a \wedge {\rm pr}_{2}(x) \in c))
\end{multline*}
が共に成り立つ.
またThm \ref{aw1bwc1l1awb1w1awc1}より
\begin{multline*}
\tag{9}
  {\rm Pair}(x) \wedge (({\rm pr}_{1}(x) \in a \wedge {\rm pr}_{2}(x) \in c) \wedge ({\rm pr}_{1}(x) \in b \wedge {\rm pr}_{2}(x) \in c)) \\
  \leftrightarrow ({\rm Pair}(x) \wedge ({\rm pr}_{1}(x) \in a \wedge {\rm pr}_{2}(x) \in c)) \wedge ({\rm Pair}(x) \wedge ({\rm pr}_{1}(x) \in b \wedge {\rm pr}_{2}(x) \in c)), 
\end{multline*}
\begin{multline*}
\tag{10}
  {\rm Pair}(x) \wedge (({\rm pr}_{1}(x) \in a \wedge {\rm pr}_{2}(x) \in b) \wedge ({\rm pr}_{1}(x) \in a \wedge {\rm pr}_{2}(x) \in c)) \\
  \leftrightarrow ({\rm Pair}(x) \wedge ({\rm pr}_{1}(x) \in a \wedge {\rm pr}_{2}(x) \in b)) \wedge ({\rm Pair}(x) \wedge ({\rm pr}_{1}(x) \in a \wedge {\rm pr}_{2}(x) \in c))
\end{multline*}
が共に成り立つ.
また定理 \ref{sthmproductelement}と推論法則 \ref{dedeqch}によって
\begin{align*}
  {\rm Pair}(x) \wedge ({\rm pr}_{1}(x) \in a \wedge {\rm pr}_{2}(x) \in c) &\leftrightarrow x \in a \times c, \\
  \mbox{}& \\
  {\rm Pair}(x) \wedge ({\rm pr}_{1}(x) \in b \wedge {\rm pr}_{2}(x) \in c) &\leftrightarrow x \in b \times c, \\
  \mbox{}& \\
  {\rm Pair}(x) \wedge ({\rm pr}_{1}(x) \in a \wedge {\rm pr}_{2}(x) \in b) &\leftrightarrow x \in a \times b
\end{align*}
がすべて成り立つから, 推論法則 \ref{dedaddeqw}によって
\begin{multline*}
\tag{11}
  ({\rm Pair}(x) \wedge ({\rm pr}_{1}(x) \in a \wedge {\rm pr}_{2}(x) \in c)) \wedge ({\rm Pair}(x) \wedge ({\rm pr}_{1}(x) \in b \wedge {\rm pr}_{2}(x) \in c)) \\
  \leftrightarrow x \in a \times c \wedge x \in b \times c,
\end{multline*}
\begin{multline*}
\tag{12}
  ({\rm Pair}(x) \wedge ({\rm pr}_{1}(x) \in a \wedge {\rm pr}_{2}(x) \in b)) \wedge ({\rm Pair}(x) \wedge ({\rm pr}_{1}(x) \in a \wedge {\rm pr}_{2}(x) \in c)) \\
  \leftrightarrow x \in a \times b \wedge x \in a \times c
\end{multline*}
が共に成り立つ.
また定理 \ref{sthmcapelement}と推論法則 \ref{dedeqch}により, 
\begin{align*}
  \tag{13}
  x \in a \times c \wedge x \in b \times c &\leftrightarrow x \in (a \times c) \cap (b \times c), \\
  \mbox{}& \\
  \tag{14}
  x \in a \times b \wedge x \in a \times c &\leftrightarrow x \in (a \times b) \cap (a \times c)
\end{align*}
が成り立つ.
以上の(1), (7), (9), (11), (13)から, 推論法則 \ref{dedeqtrans}によって
\[
\tag{15}
  x \in (a \cap b) \times c \leftrightarrow x \in (a \times c) \cap (b \times c)
\]
が成り立ち, 
(2), (8), (10), (12), (14)から, やはり推論法則 \ref{dedeqtrans}によって
\[
\tag{16}
  x \in a \times (b \cap c) \leftrightarrow x \in (a \times b) \cap (a \times c)
\]
が成り立つことがわかる.
いま$x$は定数でなく, はじめに述べたように$(a \cap b) \times c$, $(a \times c) \cap (b \times c)$, 
$a \times (b \cap c)$, $(a \times b) \cap (a \times c)$のいずれの記号列の中にも
自由変数として現れないから, (15), (16)から, 定理 \ref{sthmset=}によって
$(a \cap b) \times c = (a \times c) \cap (b \times c)$と
$a \times (b \cap c) = (a \times b) \cap (a \times c)$が共に成り立つ.
\halmos




\mathstrut
\begin{thm}
\label{sthmcapproduct2}%定理
$a$, $b$, $c$, $d$を集合とするとき, 
\[
  (a \cap b) \times (c \cap d) = (a \times c) \cap (b \times d), ~~
  (a \cap b) \times (c \cap d) = (a \times d) \cap (b \times c)
\]
が成り立つ.
\end{thm}


\noindent{\bf 証明}
~まず前者が成り立つことを示す.
$x$を$a$, $b$, $c$, $d$のいずれの記号列の中にも自由変数として現れない, 定数でない文字とする.
このとき変数法則 \ref{valcap}, \ref{valproduct}からわかるように, $x$は
$(a \cap b) \times (c \cap d)$及び$(a \times c) \cap (b \times d)$の中に自由変数として現れない.
そして定理 \ref{sthmproductelement}より
\[
\tag{1}
  x \in (a \cap b) \times (c \cap d) \leftrightarrow 
  {\rm Pair}(x) \wedge ({\rm pr}_{1}(x) \in a \cap b \wedge {\rm pr}_{2}(x) \in c \cap d)
\]
が成り立つ.
また定理 \ref{sthmcapelement}より
\[
  {\rm pr}_{1}(x) \in a \cap b \leftrightarrow {\rm pr}_{1}(x) \in a \wedge {\rm pr}_{1}(x) \in b, ~~
  {\rm pr}_{2}(x) \in c \cap d \leftrightarrow {\rm pr}_{2}(x) \in c \wedge {\rm pr}_{2}(x) \in d
\]
が共に成り立つから, 推論法則 \ref{dedaddeqw}によって
\[
\tag{2}
  {\rm pr}_{1}(x) \in a \cap b \wedge {\rm pr}_{2}(x) \in c \cap d \leftrightarrow 
  ({\rm pr}_{1}(x) \in a \wedge {\rm pr}_{1}(x) \in b) \wedge ({\rm pr}_{2}(x) \in c \wedge {\rm pr}_{2}(x) \in d)
\]
が成り立つ.
またThm \ref{1awb1wclaw1bwc1}より
\begin{multline*}
\tag{3}
  ({\rm pr}_{1}(x) \in a \wedge {\rm pr}_{1}(x) \in b) \wedge ({\rm pr}_{2}(x) \in c \wedge {\rm pr}_{2}(x) \in d) \\
  \leftrightarrow {\rm pr}_{1}(x) \in a \wedge ({\rm pr}_{1}(x) \in b \wedge ({\rm pr}_{2}(x) \in c \wedge {\rm pr}_{2}(x) \in d))
\end{multline*}
が成り立つ.
またThm \ref{1awb1wclaw1bwc1}と推論法則 \ref{dedeqch}により
\[
\tag{4}
  {\rm pr}_{1}(x) \in b \wedge ({\rm pr}_{2}(x) \in c \wedge {\rm pr}_{2}(x) \in d) \leftrightarrow 
  ({\rm pr}_{1}(x) \in b \wedge {\rm pr}_{2}(x) \in c) \wedge {\rm pr}_{2}(x) \in d
\]
が成り立つ.
またThm \ref{awblbwa}より
\[
  {\rm pr}_{1}(x) \in b \wedge {\rm pr}_{2}(x) \in c \leftrightarrow {\rm pr}_{2}(x) \in c \wedge {\rm pr}_{1}(x) \in b
\]
が成り立つから, 推論法則 \ref{dedaddeqw}によって
\[
\tag{5}
  ({\rm pr}_{1}(x) \in b \wedge {\rm pr}_{2}(x) \in c) \wedge {\rm pr}_{2}(x) \in d \leftrightarrow 
  ({\rm pr}_{2}(x) \in c \wedge {\rm pr}_{1}(x) \in b) \wedge {\rm pr}_{2}(x) \in d
\]
が成り立つ.
またThm \ref{1awb1wclaw1bwc1}より
\[
\tag{6}
  ({\rm pr}_{2}(x) \in c \wedge {\rm pr}_{1}(x) \in b) \wedge {\rm pr}_{2}(x) \in d \leftrightarrow 
  {\rm pr}_{2}(x) \in c \wedge ({\rm pr}_{1}(x) \in b \wedge {\rm pr}_{2}(x) \in d)
\]
が成り立つ.
そこで(4), (5), (6)から, 推論法則 \ref{dedeqtrans}によって
\[
  {\rm pr}_{1}(x) \in b \wedge ({\rm pr}_{2}(x) \in c \wedge {\rm pr}_{2}(x) \in d) \leftrightarrow 
  {\rm pr}_{2}(x) \in c \wedge ({\rm pr}_{1}(x) \in b \wedge {\rm pr}_{2}(x) \in d)
\]
が成り立ち, これから推論法則 \ref{dedaddeqw}によって
\begin{multline*}
\tag{7}
  {\rm pr}_{1}(x) \in a \wedge ({\rm pr}_{1}(x) \in b \wedge ({\rm pr}_{2}(x) \in c \wedge {\rm pr}_{2}(x) \in d)) \\
  \leftrightarrow {\rm pr}_{1}(x) \in a \wedge ({\rm pr}_{2}(x) \in c \wedge ({\rm pr}_{1}(x) \in b \wedge {\rm pr}_{2}(x) \in d))
\end{multline*}
が成り立つ.
またThm \ref{1awb1wclaw1bwc1}と推論法則 \ref{dedeqch}により
\begin{multline*}
\tag{8}
  {\rm pr}_{1}(x) \in a \wedge ({\rm pr}_{2}(x) \in c \wedge ({\rm pr}_{1}(x) \in b \wedge {\rm pr}_{2}(x) \in d)) \\
  \leftrightarrow ({\rm pr}_{1}(x) \in a \wedge {\rm pr}_{2}(x) \in c) \wedge ({\rm pr}_{1}(x) \in b \wedge {\rm pr}_{2}(x) \in d)
\end{multline*}
が成り立つ.
そこで(2), (3), (7), (8)から, 推論法則 \ref{dedeqtrans}によって
\[
  {\rm pr}_{1}(x) \in a \cap b \wedge {\rm pr}_{2}(x) \in c \cap d \leftrightarrow 
  ({\rm pr}_{1}(x) \in a \wedge {\rm pr}_{2}(x) \in c) \wedge ({\rm pr}_{1}(x) \in b \wedge {\rm pr}_{2}(x) \in d)
\]
が成り立ち, これから推論法則 \ref{dedaddeqw}によって
\begin{multline*}
\tag{9}
  {\rm Pair}(x) \wedge ({\rm pr}_{1}(x) \in a \cap b \wedge {\rm pr}_{2}(x) \in c \cap d) \\
  \leftrightarrow {\rm Pair}(x) \wedge (({\rm pr}_{1}(x) \in a \wedge {\rm pr}_{2}(x) \in c) \wedge ({\rm pr}_{1}(x) \in b \wedge {\rm pr}_{2}(x) \in d))
\end{multline*}
が成り立つ.
またThm \ref{aw1bwc1l1awb1w1awc1}より
\begin{multline*}
\tag{10}
  {\rm Pair}(x) \wedge (({\rm pr}_{1}(x) \in a \wedge {\rm pr}_{2}(x) \in c) \wedge ({\rm pr}_{1}(x) \in b \wedge {\rm pr}_{2}(x) \in d)) \\
  \leftrightarrow ({\rm Pair}(x) \wedge ({\rm pr}_{1}(x) \in a \wedge {\rm pr}_{2}(x) \in c)) \wedge ({\rm Pair}(x) \wedge ({\rm pr}_{1}(x) \in b \wedge {\rm pr}_{2}(x) \in d))
\end{multline*}
が成り立つ.
また定理 \ref{sthmproductelement}と推論法則 \ref{dedeqch}により
\begin{align*}
  {\rm Pair}(x) \wedge ({\rm pr}_{1}(x) \in a \wedge {\rm pr}_{2}(x) \in c) &\leftrightarrow x \in a \times c, \\
  \mbox{} \\
  {\rm Pair}(x) \wedge ({\rm pr}_{1}(x) \in b \wedge {\rm pr}_{2}(x) \in d) &\leftrightarrow x \in b \times d
\end{align*}
が共に成り立つから, 推論法則 \ref{dedaddeqw}によって
\begin{multline*}
  \tag{11}
  ({\rm Pair}(x) \wedge ({\rm pr}_{1}(x) \in a \wedge {\rm pr}_{2}(x) \in c)) \wedge ({\rm Pair}(x) \wedge ({\rm pr}_{1}(x) \in b \wedge {\rm pr}_{2}(x) \in d)) \\
  \leftrightarrow x \in a \times c \wedge x \in b \times d
\end{multline*}
が成り立つ.
また定理 \ref{sthmcapelement}と推論法則 \ref{dedeqch}により
\[
\tag{12}
  x \in a \times c \wedge x \in b \times d \leftrightarrow x \in (a \times c) \cap (b \times d)
\]
が成り立つ.
そこで(1), (9)---(12)から, 推論法則 \ref{dedeqtrans}によって
\[
\tag{13}
  x \in (a \cap b) \times (c \cap d) \leftrightarrow x \in (a \times c) \cap (b \times d)
\]
が成り立つことがわかる.
いま$x$は定数でなく, 上述のように$(a \cap b) \times (c \cap d)$及び$(a \times c) \cap (b \times d)$の中に
自由変数として現れないから, (13)から, 定理 \ref{sthmset=}によって
\[
\tag{14}
  (a \cap b) \times (c \cap d) = (a \times c) \cap (b \times d)
\]
が成り立つ.

次に後者が成り立つことを示す.
定理 \ref{sthmcapch}より$c \cap d = d \cap c$が成り立つから, 
定理 \ref{sthmproduct=}により
\[
  (a \cap b) \times (c \cap d) = (a \cap b) \times (d \cap c)
\]
が成り立つ.
またいま示したように(14)が成り立つから, (14)で$c$と$d$を入れ替えた
\[
  (a \cap b) \times (d \cap c) = (a \times d) \cap (b \times c)
\]
も成り立つ.
そこでこれらから, 推論法則 \ref{ded=trans}によって
$(a \cap b) \times (c \cap d) = (a \times d) \cap (b \times c)$が成り立つ.
\halmos




\mathstrut
\begin{thm}
\label{sthm-product}%定理
$a$, $b$, $c$を集合とするとき, 
\[
  (a - b) \times c = (a \times c) - (b \times c), ~~
  a \times (b - c) = (a \times b) - (a \times c)
\]
が成り立つ.
\end{thm}


\noindent{\bf 証明}
~まず次の($*$)が成り立つことを示しておく:

\mathstrut
($*$) ~~$A$, $B$, $C$を関係式とするとき, 
        \[
          \neg ((A \wedge (B \wedge C)) \wedge \neg A), ~~
          \neg ((A \wedge (B \wedge C)) \wedge \neg B), ~~
          \neg ((A \wedge (B \wedge C)) \wedge \neg C)
        \]
        が成り立つ.

\mathstrut
$\neg ((A \wedge (B \wedge C)) \wedge \neg A)$の証明: 
Thm \ref{awblbwa}より
$A \wedge (B \wedge C) \leftrightarrow (B \wedge C) \wedge A$が
成り立つから, 推論法則 \ref{dedaddeqw}により
\[
  (A \wedge (B \wedge C)) \wedge \neg A \leftrightarrow ((B \wedge C) \wedge A) \wedge \neg A
\]
が成り立つ.
またThm \ref{1awb1wclaw1bwc1}より
\[
  ((B \wedge C) \wedge A) \wedge \neg A \leftrightarrow (B \wedge C) \wedge (A \wedge \neg A)
\]
が成り立つ.
そこでこれらから, 推論法則 \ref{dedeqtrans}によって
\[
\tag{1}
  (A \wedge (B \wedge C)) \wedge \neg A \leftrightarrow (B \wedge C) \wedge (A \wedge \neg A)
\]
が成り立つ.
またThm \ref{n1awna1}より$\neg (A \wedge \neg A)$が成り立つから, 
推論法則 \ref{dednw}により
\[
\tag{2}
  \neg ((B \wedge C) \wedge (A \wedge \neg A))
\]
が成り立つ.
そこで(1), (2)から, 推論法則 \ref{dedeqfund}によって
$\neg ((A \wedge (B \wedge C)) \wedge \neg A)$が成り立つ.

$\neg ((A \wedge (B \wedge C)) \wedge \neg B)$の証明: 
Thm \ref{awblbwa}より
$B \wedge C \leftrightarrow C \wedge B$が成り立つから, 
推論法則 \ref{dedaddeqw}を二回用いて, 
\[
  (A \wedge (B \wedge C)) \wedge \neg B \leftrightarrow (A \wedge (C \wedge B)) \wedge \neg B
\]
が成り立つ.
またThm \ref{1awb1wclaw1bwc1}より
\[
  (A \wedge (C \wedge B)) \wedge \neg B \leftrightarrow A \wedge ((C \wedge B) \wedge \neg B)
\]
が成り立つ.
同じくThm \ref{1awb1wclaw1bwc1}より
$(C \wedge B) \wedge \neg B \leftrightarrow C \wedge (B \wedge \neg B)$が成り立つから, 
推論法則 \ref{dedaddeqw}により
\[
  A \wedge ((C \wedge B) \wedge \neg B) \leftrightarrow A \wedge (C \wedge (B \wedge \neg B))
\]
が成り立つ.
また, Thm \ref{1awb1wclaw1bwc1}と推論法則 \ref{dedeqch}により
\[
  A \wedge (C \wedge (B \wedge \neg B)) \leftrightarrow (A \wedge C) \wedge (B \wedge \neg B)
\]
が成り立つ.
そこでこれらから, 推論法則 \ref{dedeqtrans}によって
\[
\tag{3}
  (A \wedge (B \wedge C)) \wedge \neg B \leftrightarrow (A \wedge C) \wedge (B \wedge \neg B)
\]
が成り立つ.
またThm \ref{n1awna1}より$\neg (B \wedge \neg B)$が成り立つから, 
推論法則 \ref{dednw}により
\[
\tag{4}
  \neg ((A \wedge C) \wedge (B \wedge \neg B))
\]
が成り立つ.
そこで(3), (4)から, 推論法則 \ref{dedeqfund}によって
$\neg ((A \wedge (B \wedge C)) \wedge \neg B)$が成り立つ.

$\neg ((A \wedge (B \wedge C)) \wedge \neg C)$の証明: 
Thm \ref{1awb1wclaw1bwc1}と推論法則 \ref{dedeqch}により
$A \wedge (B \wedge C) \leftrightarrow (A \wedge B) \wedge C$が
成り立つから, 推論法則 \ref{dedaddeqw}により
\[
  (A \wedge (B \wedge C)) \wedge \neg C \leftrightarrow ((A \wedge B) \wedge C) \wedge \neg C
\]
が成り立つ.
またThm \ref{1awb1wclaw1bwc1}より
\[
  ((A \wedge B) \wedge C) \wedge \neg C \leftrightarrow (A \wedge B) \wedge (C \wedge \neg C)
\]
が成り立つ.
そこでこれらから, 推論法則 \ref{dedeqtrans}によって
\[
\tag{5}
  (A \wedge (B \wedge C)) \wedge \neg C \leftrightarrow (A \wedge B) \wedge (C \wedge \neg C)
\]
が成り立つ.
またThm \ref{n1awna1}より$\neg (C \wedge \neg C)$が成り立つから, 
推論法則 \ref{dednw}により
\[
\tag{6}
  \neg ((A \wedge B) \wedge (C \wedge \neg C))
\]
が成り立つ.
そこで(5), (6)から, 推論法則 \ref{dedeqfund}によって
$\neg ((A \wedge (B \wedge C)) \wedge \neg C)$が成り立つ.
以上で($*$)が示された.

さて次に$(a - b) \times c = (a \times c) - (b \times c)$が成り立つことを示す.
いま$x$を, $a$, $b$, $c$の中に自由変数として現れない, 定数でない文字とする.
このとき変数法則 \ref{val-}, \ref{valproduct}からわかるように, 
$x$は$(a - b) \times c$及び$(a \times c) - (b \times c)$の
中にも自由変数として現れない.
そして定理 \ref{sthmproductelement}より, 
\[
\tag{7}
  x \in (a - b) \times c \leftrightarrow 
  {\rm Pair}(x) \wedge ({\rm pr}_{1}(x) \in a - b \wedge {\rm pr}_{2}(x) \in c)
\]
が成り立つ.
また定理 \ref{sthm-basis}より
\[
  {\rm pr}_{1}(x) \in a - b \leftrightarrow {\rm pr}_{1}(x) \in a \wedge {\rm pr}_{1}(x) \notin b
\]
が成り立つから, 推論法則 \ref{dedaddeqw}により
\[
\tag{8}
  {\rm pr}_{1}(x) \in a - b \wedge {\rm pr}_{2}(x) \in c \leftrightarrow 
  ({\rm pr}_{1}(x) \in a \wedge {\rm pr}_{1}(x) \notin b) \wedge {\rm pr}_{2}(x) \in c
\]
が成り立つ.
またThm \ref{1awb1wclaw1bwc1}より
\[
\tag{9}
  ({\rm pr}_{1}(x) \in a \wedge {\rm pr}_{1}(x) \notin b) \wedge {\rm pr}_{2}(x) \in c \leftrightarrow 
  {\rm pr}_{1}(x) \in a \wedge ({\rm pr}_{1}(x) \notin b \wedge {\rm pr}_{2}(x) \in c)
\]
が成り立つ.
またThm \ref{awblbwa}より
\[
  {\rm pr}_{1}(x) \notin b \wedge {\rm pr}_{2}(x) \in c \leftrightarrow 
  {\rm pr}_{2}(x) \in c \wedge {\rm pr}_{1}(x) \notin b
\]
が成り立つから, 推論法則 \ref{dedaddeqw}により
\[
\tag{10}
  {\rm pr}_{1}(x) \in a \wedge ({\rm pr}_{1}(x) \notin b \wedge {\rm pr}_{2}(x) \in c) \leftrightarrow 
  {\rm pr}_{1}(x) \in a \wedge ({\rm pr}_{2}(x) \in c \wedge {\rm pr}_{1}(x) \notin b)
\]
が成り立つ.
またThm \ref{1awb1wclaw1bwc1}と推論法則 \ref{dedeqch}により
\[
\tag{11}
  {\rm pr}_{1}(x) \in a \wedge ({\rm pr}_{2}(x) \in c \wedge {\rm pr}_{1}(x) \notin b) \leftrightarrow 
  ({\rm pr}_{1}(x) \in a \wedge {\rm pr}_{2}(x) \in c) \wedge {\rm pr}_{1}(x) \notin b
\]
が成り立つ.
そこで(8)---(11)から, 推論法則 \ref{dedeqtrans}によって
\[
  {\rm pr}_{1}(x) \in a - b \wedge {\rm pr}_{2}(x) \in c \leftrightarrow 
  ({\rm pr}_{1}(x) \in a \wedge {\rm pr}_{2}(x) \in c) \wedge {\rm pr}_{1}(x) \notin b
\]
が成り立ち, これから推論法則 \ref{dedaddeqw}によって
\[
\tag{12}
  {\rm Pair}(x) \wedge ({\rm pr}_{1}(x) \in a - b \wedge {\rm pr}_{2}(x) \in c) \leftrightarrow 
  {\rm Pair}(x) \wedge (({\rm pr}_{1}(x) \in a \wedge {\rm pr}_{2}(x) \in c) \wedge {\rm pr}_{1}(x) \notin b)
\]
が成り立つ.
またThm \ref{1awb1wclaw1bwc1}と推論法則 \ref{dedeqch}により
\[
\tag{13}
  {\rm Pair}(x) \wedge (({\rm pr}_{1}(x) \in a \wedge {\rm pr}_{2}(x) \in c) \wedge {\rm pr}_{1}(x) \notin b) \leftrightarrow 
  ({\rm Pair}(x) \wedge ({\rm pr}_{1}(x) \in a \wedge {\rm pr}_{2}(x) \in c)) \wedge {\rm pr}_{1}(x) \notin b
\]
が成り立つ.
そこでいま${\rm Pair}(x) \wedge ({\rm pr}_{1}(x) \in a \wedge {\rm pr}_{2}(x) \in c)$を$R$と書くとき, 
(7), (12), (13)から, 推論法則 \ref{dedeqtrans}によって
\[
\tag{14}
  x \in (a - b) \times c \leftrightarrow R \wedge {\rm pr}_{1}(x) \notin b
\]
が成り立つ.
また$R$の定義から, ($*$)により
$\neg (R \wedge \neg {\rm Pair}(x))$が成り立つから, 
推論法則 \ref{dedavblbtrue2}により
\[
  (R \wedge \neg {\rm Pair}(x)) \vee (R \wedge {\rm pr}_{1}(x) \notin b) \leftrightarrow 
  R \wedge {\rm pr}_{1}(x) \notin b
\]
が成り立ち, これから推論法則 \ref{dedeqch}によって
\[
\tag{15}
  R \wedge {\rm pr}_{1}(x) \notin b \leftrightarrow 
  (R \wedge \neg {\rm Pair}(x)) \vee (R \wedge {\rm pr}_{1}(x) \notin b)
\]
が成り立つ.
またThm \ref{aw1bvc1l1awb1v1awc1}と推論法則 \ref{dedeqch}により
\[
\tag{16}
  (R \wedge \neg {\rm Pair}(x)) \vee (R \wedge {\rm pr}_{1}(x) \notin b) \leftrightarrow 
  R \wedge (\neg {\rm Pair}(x) \vee {\rm pr}_{1}(x) \notin b)
\]
が成り立つ.
またThm \ref{n1awb1lnavnb}と推論法則 \ref{dedeqch}により
\[
  \neg {\rm Pair}(x) \vee {\rm pr}_{1}(x) \notin b \leftrightarrow 
  \neg ({\rm Pair}(x) \wedge {\rm pr}_{1}(x) \in b)
\]
が成り立つから, 推論法則 \ref{dedaddeqw}により
\[
\tag{17}
  R \wedge (\neg {\rm Pair}(x) \vee {\rm pr}_{1}(x) \notin b) \leftrightarrow 
  R \wedge \neg ({\rm Pair}(x) \wedge {\rm pr}_{1}(x) \in b)
\]
が成り立つ.
また$R$の定義から, ($*$)により
$\neg (R \wedge {\rm pr}_{2}(x) \notin c)$が成り立つから, 
推論法則 \ref{dedavblbtrue2}により
\[
  (R \wedge \neg ({\rm Pair}(x) \wedge {\rm pr}_{1}(x) \in b)) \vee (R \wedge {\rm pr}_{2}(x) \notin c) \leftrightarrow 
  R \wedge \neg ({\rm Pair}(x) \wedge {\rm pr}_{1}(x) \in b)
\]
が成り立ち, これから推論法則 \ref{dedeqch}によって
\[
\tag{18}
  R \wedge \neg ({\rm Pair}(x) \wedge {\rm pr}_{1}(x) \in b) \leftrightarrow 
  (R \wedge \neg ({\rm Pair}(x) \wedge {\rm pr}_{1}(x) \in b)) \vee (R \wedge {\rm pr}_{2}(x) \notin c)
\]
が成り立つ.
またThm \ref{aw1bvc1l1awb1v1awc1}と推論法則 \ref{dedeqch}により
\[
\tag{19}
  (R \wedge \neg ({\rm Pair}(x) \wedge {\rm pr}_{1}(x) \in b)) \vee (R \wedge {\rm pr}_{2}(x) \notin c) \leftrightarrow 
  R \wedge (\neg ({\rm Pair}(x) \wedge {\rm pr}_{1}(x) \in b) \vee {\rm pr}_{2}(x) \notin c)
\]
が成り立つ.
またThm \ref{n1awb1lnavnb}と推論法則 \ref{dedeqch}により
\[
\tag{20}
  \neg ({\rm Pair}(x) \wedge {\rm pr}_{1}(x) \in b) \vee {\rm pr}_{2}(x) \notin c \leftrightarrow 
  \neg (({\rm Pair}(x) \wedge {\rm pr}_{1}(x) \in b) \wedge {\rm pr}_{2}(x) \in c)
\]
が成り立つ.
またThm \ref{1awb1wclaw1bwc1}より
\[
\tag{21}
  ({\rm Pair}(x) \wedge {\rm pr}_{1}(x) \in b) \wedge {\rm pr}_{2}(x) \in c \leftrightarrow 
  {\rm Pair}(x) \wedge ({\rm pr}_{1}(x) \in b \wedge {\rm pr}_{2}(x) \in c)
\]
が成り立つ.
また定理 \ref{sthmproductelement}と推論法則 \ref{dedeqch}により
\[
\tag{22}
  {\rm Pair}(x) \wedge ({\rm pr}_{1}(x) \in b \wedge {\rm pr}_{2}(x) \in c) \leftrightarrow 
  x \in b \times c
\]
が成り立つ.
そこで(21), (22)から, 推論法則 \ref{dedeqtrans}によって
\[
  ({\rm Pair}(x) \wedge {\rm pr}_{1}(x) \in b) \wedge {\rm pr}_{2}(x) \in c \leftrightarrow 
  x \in b \times c
\]
が成り立ち, これから推論法則 \ref{dedeqcp}によって
\[
\tag{23}
  \neg (({\rm Pair}(x) \wedge {\rm pr}_{1}(x) \in b) \wedge {\rm pr}_{2}(x) \in c) \leftrightarrow 
  x \notin b \times c
\]
が成り立つ.
そこで(20), (23)から, 再び推論法則 \ref{dedeqtrans}によって
\[
  \neg ({\rm Pair}(x) \wedge {\rm pr}_{1}(x) \in b) \vee {\rm pr}_{2}(x) \notin c \leftrightarrow 
  x \notin b \times c
\]
が成り立ち, これから推論法則 \ref{dedaddeqw}によって
\[
\tag{24}
  R \wedge (\neg ({\rm Pair}(x) \wedge {\rm pr}_{1}(x) \in b) \vee {\rm pr}_{2}(x) \notin c) \leftrightarrow 
  R \wedge x \notin b \times c
\]
が成り立つ.
また$R$の定義から, 定理 \ref{sthmproductelement}と推論法則 \ref{dedeqch}によって
$R \leftrightarrow x \in a \times c$が成り立つから, 
推論法則 \ref{dedaddeqw}により
\[
\tag{25}
  R \wedge x \notin b \times c \leftrightarrow x \in a \times c \wedge x \notin b \times c
\]
が成り立つ.
また定理 \ref{sthm-basis}と推論法則 \ref{dedeqch}により
\[
\tag{26}
  x \in a \times c \wedge x \notin b \times c \leftrightarrow x \in (a \times c) - (b \times c)
\]
が成り立つ.
以上の(14)---(19), (24)---(26)から, 推論法則 \ref{dedeqtrans}によって
\[
\tag{27}
  x \in (a - b) \times c \leftrightarrow x \in (a \times c) - (b \times c)
\]
が成り立つことがわかる.
いま$x$は定数でなく, 上述のように$(a - b) \times c$及び
$(a \times c) - (b \times c)$の中に
自由変数として現れないから, (27)から, 定理 \ref{sthmset=}によって
$(a - b) \times c = (a \times c) - (b \times c)$が成り立つ.

最後に$a \times (b - c) = (a \times b) - (a \times c)$が成り立つことを示す.
いま$x$を上と同じ文字とする.
このとき変数法則 \ref{val-}, \ref{valproduct}からわかるように, 
$x$は$a \times (b - c)$及び$(a \times b) - (a \times c)$の
中に自由変数として現れない.
そして定理 \ref{sthmproductelement}より, 
\[
\tag{28}
  x \in a \times (b - c) \leftrightarrow 
  {\rm Pair}(x) \wedge ({\rm pr}_{1}(x) \in a \wedge {\rm pr}_{2}(x) \in b - c)
\]
が成り立つ.
また定理 \ref{sthm-basis}より
\[
  {\rm pr}_{2}(x) \in b - c \leftrightarrow {\rm pr}_{2}(x) \in b \wedge {\rm pr}_{2}(x) \notin c
\]
が成り立つから, 推論法則 \ref{dedaddeqw}により
\[
\tag{29}
  {\rm pr}_{1}(x) \in a \wedge {\rm pr}_{2}(x) \in b - c \leftrightarrow 
  {\rm pr}_{1}(x) \in a \wedge ({\rm pr}_{2}(x) \in b \wedge {\rm pr}_{2}(x) \notin c)
\]
が成り立つ.
またThm \ref{1awb1wclaw1bwc1}と推論法則 \ref{dedeqch}により
\[
\tag{30}
  {\rm pr}_{1}(x) \in a \wedge ({\rm pr}_{2}(x) \in b \wedge {\rm pr}_{2}(x) \notin c) \leftrightarrow 
  ({\rm pr}_{1}(x) \in a \wedge {\rm pr}_{2}(x) \in b) \wedge {\rm pr}_{2}(x) \notin c
\]
が成り立つ.
そこで(29), (30)から, 推論法則 \ref{dedeqtrans}によって
\[
  {\rm pr}_{1}(x) \in a \wedge {\rm pr}_{2}(x) \in b - c \leftrightarrow 
  ({\rm pr}_{1}(x) \in a \wedge {\rm pr}_{2}(x) \in b) \wedge {\rm pr}_{2}(x) \notin c
\]
が成り立ち, これから推論法則 \ref{dedaddeqw}によって
\[
\tag{31}
  {\rm Pair}(x) \wedge ({\rm pr}_{1}(x) \in a \wedge {\rm pr}_{2}(x) \in b - c) \leftrightarrow 
  {\rm Pair}(x) \wedge (({\rm pr}_{1}(x) \in a \wedge {\rm pr}_{2}(x) \in b) \wedge {\rm pr}_{2}(x) \notin c)
\]
が成り立つ.
またThm \ref{1awb1wclaw1bwc1}と推論法則 \ref{dedeqch}により
\[
\tag{32}
  {\rm Pair}(x) \wedge (({\rm pr}_{1}(x) \in a \wedge {\rm pr}_{2}(x) \in b) \wedge {\rm pr}_{2}(x) \notin c) \leftrightarrow 
  ({\rm Pair}(x) \wedge ({\rm pr}_{1}(x) \in a \wedge {\rm pr}_{2}(x) \in b)) \wedge {\rm pr}_{2}(x) \notin c
\]
が成り立つ.
そこでいま${\rm Pair}(x) \wedge ({\rm pr}_{1}(x) \in a \wedge {\rm pr}_{2}(x) \in b)$を$S$と書くとき, 
(28), (31), (32)から, 推論法則 \ref{dedeqtrans}によって
\[
\tag{33}
  x \in a \times (b - c) \leftrightarrow S \wedge {\rm pr}_{2}(x) \notin c
\]
が成り立つ.
また$S$の定義から, ($*$)により
$\neg (S \wedge {\rm pr}_{1}(x) \notin a)$が成り立つから, 
推論法則 \ref{dedavblbtrue2}により
\[
  (S \wedge {\rm pr}_{1}(x) \notin a) \vee (S \wedge {\rm pr}_{2}(x) \notin c) \leftrightarrow 
  S \wedge {\rm pr}_{2}(x) \notin c
\]
が成り立ち, これから推論法則 \ref{dedeqch}によって
\[
\tag{34}
  S \wedge {\rm pr}_{2}(x) \notin c \leftrightarrow 
  (S \wedge {\rm pr}_{1}(x) \notin a) \vee (S \wedge {\rm pr}_{2}(x) \notin c)
\]
が成り立つ.
またThm \ref{aw1bvc1l1awb1v1awc1}と推論法則 \ref{dedeqch}により
\[
\tag{35}
  (S \wedge {\rm pr}_{1}(x) \notin a) \vee (S \wedge {\rm pr}_{2}(x) \notin c) \leftrightarrow 
  S \wedge ({\rm pr}_{1}(x) \notin a \vee {\rm pr}_{2}(x) \notin c)
\]
が成り立つ.
またThm \ref{n1awb1lnavnb}と推論法則 \ref{dedeqch}により
\[
  {\rm pr}_{1}(x) \notin a \vee {\rm pr}_{2}(x) \notin c \leftrightarrow 
  \neg ({\rm pr}_{1}(x) \in a \wedge {\rm pr}_{2}(x) \in c)
\]
が成り立つから, 推論法則 \ref{dedaddeqw}により
\[
\tag{36}
  S \wedge ({\rm pr}_{1}(x) \notin a \vee {\rm pr}_{2}(x) \notin c) \leftrightarrow 
  S \wedge \neg ({\rm pr}_{1}(x) \in a \wedge {\rm pr}_{2}(x) \in c)
\]
が成り立つ.
また$S$の定義から, ($*$)により
$\neg (S \wedge \neg {\rm Pair}(x))$が成り立つから, 
推論法則 \ref{dedavblbtrue2}により
\[
  (S \wedge \neg {\rm Pair}(x)) \vee (S \wedge \neg ({\rm pr}_{1}(x) \in a \wedge {\rm pr}_{2}(x) \in c)) \leftrightarrow 
  S \wedge \neg ({\rm pr}_{1}(x) \in a \wedge {\rm pr}_{2}(x) \in c)
\]
が成り立ち, これから推論法則 \ref{dedeqch}によって
\[
\tag{37}
  S \wedge \neg ({\rm pr}_{1}(x) \in a \wedge {\rm pr}_{2}(x) \in c) \leftrightarrow 
  (S \wedge \neg {\rm Pair}(x)) \vee (S \wedge \neg ({\rm pr}_{1}(x) \in a \wedge {\rm pr}_{2}(x) \in c))
\]
が成り立つ.
またThm \ref{aw1bvc1l1awb1v1awc1}と推論法則 \ref{dedeqch}により
\[
\tag{38}
  (S \wedge \neg {\rm Pair}(x)) \vee (S \wedge \neg ({\rm pr}_{1}(x) \in a \wedge {\rm pr}_{2}(x) \in c)) \leftrightarrow 
  S \wedge (\neg {\rm Pair}(x) \vee \neg ({\rm pr}_{1}(x) \in a \wedge {\rm pr}_{2}(x) \in c))
\]
が成り立つ.
またThm \ref{n1awb1lnavnb}と推論法則 \ref{dedeqch}により
\[
\tag{39}
  \neg {\rm Pair}(x) \vee \neg ({\rm pr}_{1}(x) \in a \wedge {\rm pr}_{2}(x) \in c) \leftrightarrow 
  \neg ({\rm Pair}(x) \wedge ({\rm pr}_{1}(x) \in a \wedge {\rm pr}_{2}(x) \in c))
\]
が成り立つ.
また定理 \ref{sthmproductelement}と推論法則 \ref{dedeqch}により
\[
  {\rm Pair}(x) \wedge ({\rm pr}_{1}(x) \in a \wedge {\rm pr}_{2}(x) \in c) \leftrightarrow 
  x \in a \times c
\]
が成り立つから, これから推論法則 \ref{dedeqcp}によって
\[
\tag{40}
  \neg ({\rm Pair}(x) \wedge ({\rm pr}_{1}(x) \in a \wedge {\rm pr}_{2}(x) \in c)) \leftrightarrow 
  x \notin a \times c
\]
が成り立つ.
そこで(39), (40)から, 推論法則 \ref{dedeqtrans}によって
\[
  \neg {\rm Pair}(x) \vee \neg ({\rm pr}_{1}(x) \in a \wedge {\rm pr}_{2}(x) \in c) \leftrightarrow 
  x \notin a \times c
\]
が成り立ち, これから推論法則 \ref{dedaddeqw}によって
\[
\tag{41}
  S \wedge (\neg {\rm Pair}(x) \vee \neg ({\rm pr}_{1}(x) \in a \wedge {\rm pr}_{2}(x) \in c)) \leftrightarrow 
  S \wedge x \notin a \times c
\]
が成り立つ.
また$S$の定義から, 定理 \ref{sthmproductelement}と推論法則 \ref{dedeqch}によって
$S \leftrightarrow x \in a \times b$が成り立つから, 
推論法則 \ref{dedaddeqw}により
\[
\tag{42}
  S \wedge x \notin a \times c \leftrightarrow x \in a \times b \wedge x \notin a \times c
\]
が成り立つ.
また定理 \ref{sthm-basis}と推論法則 \ref{dedeqch}により
\[
\tag{43}
  x \in a \times b \wedge x \notin a \times c \leftrightarrow x \in (a \times b) - (a \times c)
\]
が成り立つ.
以上の(33)---(38), (41)---(43)から, 推論法則 \ref{dedeqtrans}によって
\[
\tag{44}
  x \in a \times (b - c) \leftrightarrow x \in (a \times b) - (a \times c)
\]
が成り立つことがわかる.
いま$x$は定数でなく, 上述のように$a \times (b - c)$及び
$(a \times b) - (a \times c)$の中に
自由変数として現れないから, (44)から, 定理 \ref{sthmset=}によって
$a \times (b - c) = (a \times b) - (a \times c)$が成り立つ.
\halmos




\mathstrut
\begin{thm}
\label{sthmemptyproducteq}%定理
$a$と$b$を集合とするとき, 
\[
  a \times b = \phi \leftrightarrow a = \phi \vee b = \phi, ~~
  a \times b \neq \phi \leftrightarrow a \neq \phi \wedge b \neq \phi
\]
が成り立つ.
またこのことから, 次の($*$)が成り立つ: 

($*$) ~~$a$が空ならば, $a \times b$は空である.
        また$b$が空ならば, $a \times b$は空である.
        また$a$と$b$が共に空でなければ, $a \times b$は空でない.
        また$a \times b$が空でなければ, $a$と$b$は共に空でない.
\end{thm}


\noindent{\bf 証明}
~まず前半を示す.
$x$を$a$及び$b$の中に自由変数として現れない文字とする.
このとき変数法則 \ref{valproduct}より, $x$は$a \times b$の中にも
自由変数として現れない.
そこでいま$\tau_{x}(x \in a \times b)$を$T$と書けば, 
$T$は集合であり, 定理 \ref{sthmelm&empty}と推論法則 \ref{dedequiv}により
\[
\tag{1}
  a \times b \neq \phi \to T \in a \times b
\]
が成り立つ.
また定理 \ref{sthmproductelement}と推論法則 \ref{dedequiv}により
\[
\tag{2}
  T \in a \times b \to {\rm Pair}(T) \wedge ({\rm pr}_{1}(T) \in a \wedge {\rm pr}_{2}(T) \in b)
\]
が成り立つ.
またThm \ref{awbta}より
\[
\tag{3}
  {\rm Pair}(T) \wedge ({\rm pr}_{1}(T) \in a \wedge {\rm pr}_{2}(T) \in b) \to 
  {\rm pr}_{1}(T) \in a \wedge {\rm pr}_{2}(T) \in b
\]
が成り立つ.
また定理 \ref{sthmnotemptyeqexin}より
\[
  {\rm pr}_{1}(T) \in a \to a \neq \phi, ~~
  {\rm pr}_{2}(T) \in b \to b \neq \phi
\]
が共に成り立つから, 推論法則 \ref{dedfromaddw}により
\[
\tag{4}
  {\rm pr}_{1}(T) \in a \wedge {\rm pr}_{2}(T) \in b \to a \neq \phi \wedge b \neq \phi
\]
が成り立つ.
そこで(1)---(4)から, 推論法則 \ref{dedmmp}によって
\[
\tag{5}
  a \times b \neq \phi \to a \neq \phi \wedge b \neq \phi
\]
が成り立つことがわかる.
またいま$\tau_{x}(x \in a)$, $\tau_{x}(x \in b)$をそれぞれ$U$, $V$と書けば, 
これらは集合であり, $x$が$a$及び$b$の中に自由変数として現れないことから, 
定理 \ref{sthmelm&empty}と推論法則 \ref{dedequiv}により
\[
  a \neq \phi \to U \in a, ~~
  b \neq \phi \to V \in b
\]
が共に成り立つ.
そこでこれらから, 推論法則 \ref{dedfromaddw}によって
\[
\tag{6}
  a \neq \phi \wedge b \neq \phi \to U \in a \wedge V \in b
\]
が成り立つ.
また定理 \ref{sthmpairinproduct}と推論法則 \ref{dedequiv}により
\[
\tag{7}
  U \in a \wedge V \in b \to (U, V) \in a \times b
\]
が成り立つ.
また定理 \ref{sthmnotemptyeqexin}より
\[
\tag{8}
  (U, V) \in a \times b \to a \times b \neq \phi
\]
が成り立つ.
そこで(6), (7), (8)から, 推論法則 \ref{dedmmp}によって
\[
\tag{9}
  a \neq \phi \wedge b \neq \phi \to a \times b \neq \phi
\]
が成り立つことがわかる.
そこで(5), (9)から, 推論法則 \ref{dedequiv}によって
\[
\tag{10}
  a \times b \neq \phi \leftrightarrow a \neq \phi \wedge b \neq \phi
\]
が成り立つ.
またThm \ref{n1awb1lnavnb}と推論法則 \ref{dedeqch}により
\[
\tag{11}
  a \neq \phi \wedge b \neq \phi \leftrightarrow \neg (a = \phi \vee b = \phi)
\]
が成り立つ.
そこで(10), (11)から, 推論法則 \ref{dedeqtrans}によって
\[
  a \times b \neq \phi \leftrightarrow \neg (a = \phi \vee b = \phi)
\]
が成り立ち, これから推論法則 \ref{dedeqcp}によって
\[
\tag{12}
  a \times b = \phi \leftrightarrow a = \phi \vee b = \phi
\]
が成り立つ.

次に($*$)を示す.
いま$a$が空であるとすると, 推論法則 \ref{dedvee}により
$a = \phi \vee b = \phi$が成り立つから, これと(12)から, 
推論法則 \ref{dedeqfund}によって$a \times b = \phi$が成り立つ.
$b$が空であるときも同様に, 推論法則 \ref{dedvee}により
$a = \phi \vee b = \phi$が成り立ち, これと(12)から, 
推論法則 \ref{dedeqfund}によって$a \times b = \phi$が成り立つ.
また$a$と$b$が共に空でなければ, 推論法則 \ref{dedwedge}により
$a \neq \phi \wedge b \neq \phi$が成り立つから, これと(10)から, 
推論法則 \ref{dedeqfund}によって$a \times b \neq \phi$が成り立つ.
逆に$a \times b$が空でなければ, これと(10)から, 推論法則 \ref{dedeqfund}によって
$a \neq \phi \wedge b \neq \phi$が成り立つから, 推論法則 \ref{dedwedge}によって
$a \neq \phi$と$b \neq \phi$が共に成り立つ.
\halmos




\mathstrut
\begin{thm}
\label{sthmemptyproduct}%定理
$a$を集合とするとき, 
\[
  a \times \phi = \phi, ~~
  \phi \times a = \phi
\]
が成り立つ.
\end{thm}


\noindent{\bf 証明}
~Thm \ref{x=x}より$\phi = \phi$が成り立つから, 
定理 \ref{sthmemptyproducteq}により
$a \times \phi = \phi$と$\phi \times a = \phi$が共に成り立つ.
\halmos




\mathstrut
定理 \ref{sthmemptyproducteq}を用いて, 定理 \ref{sthmproduct=eq}の2)を改良しておく.




\mathstrut
\begin{thm}
\label{sthmproduct=mk2}%定理
$a$, $b$, $c$, $d$を集合とするとき, 
\begin{align*}
  a \neq \phi \wedge b \neq \phi &\to (a = c \wedge b = d \leftrightarrow a \times b = c \times d), \\
  \mbox{} \\
  c \neq \phi \wedge d \neq \phi &\to (a = c \wedge b = d \leftrightarrow a \times b = c \times d)
\end{align*}
が成り立つ.
またこのことから, 次の($*$)が成り立つ: 

($*$) ~~$a$と$b$が共に空でなければ, $a = c \wedge b = d \leftrightarrow a \times b = c \times d$が成り立つ.
        また$c$と$d$が共に空でなければ, $a = c \wedge b = d \leftrightarrow a \times b = c \times d$が成り立つ.
\end{thm}


\noindent{\bf 証明}
~まず
\begin{align*}
  \tag{1}
  a \neq \phi \wedge b \neq \phi &\to (a = c \wedge b = d \leftrightarrow a \times b = c \times d), \\
  \mbox{} \\
  \tag{2}
  c \neq \phi \wedge d \neq \phi &\to (a = c \wedge b = d \leftrightarrow a \times b = c \times d)
\end{align*}
が共に成り立つことを示す.
定理 \ref{sthmproduct=inverse}より
\[
\tag{3}
  (a \neq \phi \wedge b \neq \phi) \wedge (c \neq \phi \wedge d \neq \phi) \to (a \times b = c \times d \to a = c \wedge b = d)
\]
が成り立つ.
また定理 \ref{sthmemptyproducteq}より
\[
  a \times b \neq \phi \leftrightarrow a \neq \phi \wedge b \neq \phi, ~~
  c \times d \neq \phi \leftrightarrow c \neq \phi \wedge d \neq \phi
\]
が共に成り立つから, これらから, 推論法則 \ref{dedequiv}によって
\begin{align*}
  \tag{4}
  &a \times b \neq \phi \to a \neq \phi \wedge b \neq \phi, \\
  \mbox{} \\
  \tag{5}
  &a \neq \phi \wedge b \neq \phi \to a \times b \neq \phi, \\
  \mbox{} \\
  \tag{6}
  &c \times d \neq \phi \to c \neq \phi \wedge d \neq \phi, \\
  \mbox{} \\
  \tag{7}
  &c \neq \phi \wedge d \neq \phi \to c \times d \neq \phi
\end{align*}
がすべて成り立つ.
そこで(4), (6)から, 推論法則 \ref{dedfromaddw}によって
\[
\tag{8}
  a \times b \neq \phi \wedge c \times d \neq \phi \to (a \neq \phi \wedge b \neq \phi) \wedge (c \neq \phi \wedge d \neq \phi)
\]
が成り立つ.
そこで(8), (3)から, 推論法則 \ref{dedmmp}によって
\[
  a \times b \neq \phi \wedge c \times d \neq \phi \to (a \times b = c \times d \to a = c \wedge b = d)
\]
が成り立ち, これから推論法則 \ref{dedch}によって
\[
  a \times b = c \times d \to (a \times b \neq \phi \wedge c \times d \neq \phi \to a = c \wedge b = d)
\]
が成り立ち, これから推論法則 \ref{dedtwch}によって
\[
\tag{9}
  a \times b = c \times d \wedge (a \times b \neq \phi \wedge c \times d \neq \phi) \to a = c \wedge b = d
\]
が成り立つ.
またThm \ref{x=yt1x=zly=z1}より
\[
  a \times b = c \times d \to (a \times b = \phi \leftrightarrow c \times d = \phi)
\]
が成り立つから, 推論法則 \ref{dedprewedge}によって
\begin{align*}
  \tag{10}
  a \times b = c \times d \to (c \times d = \phi \to a \times b = \phi), \\
  \mbox{} \\
  \tag{11}
  a \times b = c \times d \to (a \times b = \phi \to c \times d = \phi)
\end{align*}
が共に成り立つ.
またThm \ref{1atb1t1nbtna1}より
\begin{align*}
  \tag{12}
  (c \times d = \phi \to a \times b = \phi) \to (a \times b \neq \phi \to c \times d \neq \phi), \\
  \mbox{} \\
  \tag{13}
  (a \times b = \phi \to c \times d = \phi) \to (c \times d \neq \phi \to a \times b \neq \phi)
\end{align*}
が共に成り立つ.
そこで(10)と(12), (11)と(13)から, それぞれ推論法則 \ref{dedmmp}によって
\[
  a \times b = c \times d \to (a \times b \neq \phi \to c \times d \neq \phi), ~~
  a \times b = c \times d \to (c \times d \neq \phi \to a \times b \neq \phi)
\]
が共に成り立ち, これらから, 推論法則 \ref{dedtwch}によって
\[
  a \times b = c \times d \wedge a \times b \neq \phi \to c \times d \neq \phi, ~~
  a \times b = c \times d \wedge c \times d \neq \phi \to a \times b \neq \phi
\]
が共に成り立つ.
そこで推論法則 \ref{dedatawbtrue1}によって
\begin{align*}
  \tag{14}
  a \times b = c \times d \wedge a \times b \neq \phi &\to (a \times b = c \times d \wedge a \times b \neq \phi) \wedge c \times d \neq \phi, \\
  \mbox{} \\
  \tag{15}
  a \times b = c \times d \wedge c \times d \neq \phi &\to (a \times b = c \times d \wedge c \times d \neq \phi) \wedge a \times b \neq \phi
\end{align*}
が共に成り立つ.
またThm \ref{1awb1wctaw1bwc1}より
\begin{align*}
  \tag{16}
  (a \times b = c \times d \wedge a \times b \neq \phi) \wedge c \times d \neq \phi &\to 
  a \times b = c \times d \wedge (a \times b \neq \phi \wedge c \times d \neq \phi), \\
  \mbox{} \\
  \tag{17}
  (a \times b = c \times d \wedge c \times d \neq \phi) \wedge a \times b \neq \phi &\to 
  a \times b = c \times d \wedge (c \times d \neq \phi \wedge a \times b \neq \phi)
\end{align*}
が共に成り立つ.
またThm \ref{awbtbwa}より
\[
  c \times d \neq \phi \wedge a \times b \neq \phi \to a \times b \neq \phi \wedge c \times d \neq \phi
\]
が成り立つから, 推論法則 \ref{dedaddw}により
\[
\tag{18}
  a \times b = c \times d \wedge (c \times d \neq \phi \wedge a \times b \neq \phi) \to 
  a \times b = c \times d \wedge (a \times b \neq \phi \wedge c \times d \neq \phi)
\]
が成り立つ.
そこで(14), (16), (9)から, 推論法則 \ref{dedmmp}によって
\[
  a \times b = c \times d \wedge a \times b \neq \phi \to a = c \wedge b = d
\]
が成り立ち, (15), (17), (18), (9)から, 同じく推論法則 \ref{dedmmp}によって
\[
  a \times b = c \times d \wedge c \times d \neq \phi \to a = c \wedge b = d
\]
が成り立つことがわかる.
そこでこれらにそれぞれ推論法則 \ref{dedtwch}を適用して
\begin{align*}
  a \times b = c \times d &\to (a \times b \neq \phi \to a = c \wedge b = d), \\
  \mbox{} \\
  a \times b = c \times d &\to (c \times d \neq \phi \to a = c \wedge b = d)
\end{align*}
が共に成り立つから, これらにそれぞれ推論法則 \ref{dedch}を適用して
\begin{align*}
  \tag{19}
  a \times b \neq \phi &\to (a \times b = c \times d \to a = c \wedge b = d), \\
  \mbox{} \\
  \tag{20}
  c \times d \neq \phi &\to (a \times b = c \times d \to a = c \wedge b = d)
\end{align*}
が共に成り立つ.
そこで(5)と(19)から, 推論法則 \ref{dedmmp}によって
\[
\tag{21}
  a \neq \phi \wedge b \neq \phi \to (a \times b = c \times d \to a = c \wedge b = d)
\]
が成り立ち, 
(7)と(20)から, 同じく推論法則 \ref{dedmmp}によって
\[
\tag{22}
  c \neq \phi \wedge d \neq \phi \to (a \times b = c \times d \to a = c \wedge b = d)
\]
が成り立つ.
また定理 \ref{sthmproduct=}より
\[
  a = c \wedge b = d \to a \times b = c \times d
\]
が成り立つから, 推論法則 \ref{deds1}により
\begin{align*}
  \tag{23}
  a \neq \phi \wedge b \neq \phi \to (a = c \wedge b = d \to a \times b = c \times d), \\
  \mbox{} \\
  \tag{24}
  c \neq \phi \wedge d \neq \phi \to (a = c \wedge b = d \to a \times b = c \times d)
\end{align*}
が共に成り立つ.
そこで(23)と(21)から, 推論法則 \ref{dedprewedge}によって(1)が成り立ち, 
(24)と(22)から, 同じく推論法則 \ref{dedprewedge}によって(2)が成り立つ.

さていま$a$と$b$が共に空でないとすると, 推論法則 \ref{dedwedge}により
$a \neq \phi \wedge b \neq \phi$が成り立つから, これと(1)から, 推論法則 \ref{dedmp}によって
$a = c \wedge b = d \leftrightarrow a \times b = c \times d$が成り立つ.
同様に, $c$と$d$が共に空でなければ, 推論法則 \ref{dedwedge}によって
$c \neq \phi \wedge d \neq \phi$が成り立つから, これと(2)から, 推論法則 \ref{dedmp}によって
$a = c \wedge b = d \leftrightarrow a \times b = c \times d$が成り立つ.
これで($*$)が成り立つことが示された.
\halmos
%section7確認済



\newpage
\setcounter{defi}{0}
\section{グラフ}%%%%%%%%%%%%%%%%%%%%%%%%%%%%%%%%%%%%%%%%%%%%%%%%%%%%%%%%%%%%%%%




この節ではグラフを定義し, その性質を調べる.
グラフとは, その元がすべて対であるような集合のことである.
後で見るように, 集合$a$がグラフであるということは, 
$a$がある集合$b$, $c$の積$b \times c$の部分集合であるということと同じことである.
なおグラフのより詳しい性質については\S 9, \S 10でも論じる.




\mathstrut
\begin{defo}
\label{graph}%変形
$a$を記号列とし, $x$と$y$を共に$a$の中に自由変数として現れない文字とする.
このとき
\[
  \forall x(x \in a \to {\rm Pair}(x)) \equiv \forall y(y \in a \to {\rm Pair}(y))
\]
が成り立つ.
\end{defo}


\noindent{\bf 証明}
~$x$と$y$が同じ文字であるときは明らか.
$x$と$y$が異なる文字であるとき, このことと
$y$が$a$の中に自由変数として現れないという仮定から, 
変数法則 \ref{valfund}, \ref{valbigpair}により
$y$は$x \in a \to {\rm Pair}(x)$の中に自由変数として現れないから, 
代入法則 \ref{substquantrans}により
\[
  \forall x(x \in a \to {\rm Pair}(x)) \equiv \forall y((y|x)(x \in a \to {\rm Pair}(x)))
\]
が成り立つ.
また$x$が$a$の中に自由変数として現れないという仮定から, 
代入法則 \ref{substfree}, \ref{substfund}, \ref{substbigpair}により
\[
  (y|x)(x \in a \to {\rm Pair}(x)) \equiv y \in a \to {\rm Pair}(y)
\]
が成り立つ.
よってこれらから, $\forall x(x \in a \to {\rm Pair}(x))$が$\forall y(y \in a \to {\rm Pair}(y))$と
一致することがわかる.
\halmos




\mathstrut
\begin{defi}
\label{defgraph}%定義
$a$を記号列とし, $x$と$y$を共に$a$の中に自由変数として現れない文字とする.
このとき上記の変形法則 \ref{graph}によれば, 
$\forall x(x \in a \to {\rm Pair}(x))$と$\forall y(y \in a \to {\rm Pair}(y))$という
二つの記号列は一致する.
$a$に対して定まるこの記号列を, ${\rm Graph}(a)$と書き表す.
\end{defi}




\mathstrut
\begin{valu}
\label{valgraph}%変数
$a$を記号列とし, $x$を文字とする.
$x$が$a$の中に自由変数として現れなければ, $x$は${\rm Graph}(a)$の中に
自由変数として現れない.
\end{valu}


\noindent{\bf 証明}
~このとき定義から${\rm Graph}(a)$は
$\forall x(x \in a \to {\rm Pair}(x))$と同じだから, 
変数法則 \ref{valquan}により, $x$はこの中に
自由変数として現れない.
\halmos




\mathstrut
\begin{subs}
\label{substgraph}%代入
$a$と$b$を記号列とし, $x$を文字とするとき, 
\[
  (b|x)({\rm Graph}(a)) \equiv {\rm Graph}((b|x)(a))
\]
が成り立つ.
\end{subs}


\noindent{\bf 証明}
~$y$を$x$と異なり, $a$及び$b$の中に自由変数として現れない文字とすれば, 
定義から${\rm Graph}(a)$は$\forall y(y \in a \to {\rm Pair}(y))$と同じだから, 
代入法則 \ref{substquan}により
\[
  (b|x)({\rm Graph}(a)) \equiv \forall y((b|x)(y \in a \to {\rm Pair}(y)))
\]
が成り立つ.
また$x$が$y$と異なることから, 変数法則 \ref{valbigpair}により
$x$は${\rm Pair}(y)$の中に自由変数として現れないから, 
代入法則 \ref{substfree}, \ref{substfund}により
\[
  (b|x)(y \in a \to {\rm Pair}(y)) \equiv y \in (b|x)(a) \to {\rm Pair}(y)
\]
が成り立つ.
よってこれらから, $(b|x)({\rm Graph}(a))$が
\[
  \forall y(y \in (b|x)(a) \to {\rm Pair}(y))
\]
と一致することがわかる.
ここで$y$が$a$及び$b$の中に自由変数として現れないことから, 
変数法則 \ref{valsubst}により$y$は$(b|x)(a)$の中にも自由変数として現れないから, 
定義から上記の記号列は${\rm Graph}((b|x)(a))$と同じである.
故に本法則が成り立つ.
\halmos




\mathstrut
\begin{form}
\label{formgraph}%構成
$a$が集合ならば, ${\rm Graph}(a)$は関係式である.
\end{form}


\noindent{\bf 証明}
~$x$を$a$の中に自由変数として現れない文字とするとき, 定義から
${\rm Graph}(a)$は$\forall x(x \in a \to {\rm Pair}(x))$と同じである.
よって$a$が集合ならば, 構成法則 \ref{formfund}, \ref{formquan}, \ref{formbigpair}から
すぐわかるように, ${\rm Graph}(a)$は関係式である.
\halmos




\mathstrut
$a$を集合とする.
関係式${\rm Graph}(a)$が定理となるとき, 
$a$は\textbf{グラフ}である, あるいは, 
$a$は\textbf{対の集合}であるという.

さて定義から直ちに次の定理が得られる.




\mathstrut
\begin{thm}
\label{sthmgraphbasis}%定理
$a$と$b$を集合とするとき, 
\[
  {\rm Graph}(a) \to (b \in a \to {\rm Pair}(b))
\]
が成り立つ.
\end{thm}


\noindent{\bf 証明}
~$x$を$a$の中に自由変数として現れない文字とすれば, 定義から
${\rm Graph}(a)$は$\forall x(x \in a \to {\rm Pair}(x))$と同じだから, 
Thm \ref{thmallfund2}より
\[
  {\rm Graph}(a) \to (b|x)(x \in a \to {\rm Pair}(x))
\]
が成り立つ.
$x$が$a$の中に自由変数として現れないことから, 
代入法則 \ref{substfree}, \ref{substfund}, \ref{substbigpair}によりこの記号列は
\[
  {\rm Graph}(a) \to (b \in a \to {\rm Pair}(b))
\]
と一致するから, これが定理となる.
\halmos




\mathstrut
\begin{thm}
\label{sthmgraphsubset}%定理
$a$と$b$を集合とするとき, 
\[
  a \subset b \to ({\rm Graph}(b) \to {\rm Graph}(a))
\]
が成り立つ.
またこのことから, 次の($*$)が成り立つ: 

($*$) ~~$a \subset b$が成り立つとき, $b$がグラフならば, $a$はグラフである.
\end{thm}


\noindent{\bf 証明}
~$x$を$a$及び$b$の中に自由変数として現れない文字とする.
また
\[
  T \equiv \tau_{x}(\neg ((x \in b \to {\rm Pair}(x)) \to (x \in a \to {\rm Pair}(x))))
\]
と置く.
$T$は集合であり, 定理 \ref{sthmsubsetbasis}より
\[
\tag{1}
  a \subset b \to (T \in a \to T \in b)
\]
が成り立つ.
またThm \ref{1atb1t11btc1t1atc11}より
\[
  (T \in a \to T \in b) \to ((T \in b \to {\rm Pair}(T)) \to (T \in a \to {\rm Pair}(T)))
\]
が成り立つ.
ここで$x$が$a$及び$b$の中に自由変数として現れないことから, 
代入法則 \ref{substfree}, \ref{substfund}, \ref{substbigpair}により, 
この記号列は
\[
\tag{2}
  (T \in a \to T \in b) \to (T|x)((x \in b \to {\rm Pair}(x)) \to (x \in a \to {\rm Pair}(x)))
\]
と一致する.
よってこれが定理となる.
また$T$の定義から, Thm \ref{thmallfund1}と推論法則 \ref{dedequiv}により
\[
\tag{3}
  (T|x)((x \in b \to {\rm Pair}(x)) \to (x \in a \to {\rm Pair}(x))) \to 
  \forall x((x \in b \to {\rm Pair}(x)) \to (x \in a \to {\rm Pair}(x)))
\]
が成り立つ.
またThm \ref{thmalltallsep}より
\[
  \forall x((x \in b \to {\rm Pair}(x)) \to (x \in a \to {\rm Pair}(x))) \to 
  (\forall x(x \in b \to {\rm Pair}(x)) \to \forall x(x \in a \to {\rm Pair}(x)))
\]
が成り立つが, いま$x$が$a$及び$b$の中に自由変数として現れないので, 定義からこの記号列は
\[
\tag{4}
  \forall x((x \in b \to {\rm Pair}(x)) \to (x \in a \to {\rm Pair}(x))) \to 
  ({\rm Graph}(b) \to {\rm Graph}(a))
\]
と同じである.
よってこれが定理となる.
そこで(1)---(4)から, 推論法則 \ref{dedmmp}によって
\[
  a \subset b \to ({\rm Graph}(b) \to {\rm Graph}(a))
\]
が成り立つことがわかる.
($*$)が成り立つことは, これと推論法則 \ref{dedmp}によって明らかである.
\halmos




\mathstrut
\begin{thm}
\label{sthmgraph=}%定理
$a$と$b$を集合とするとき, 
\[
  a = b \to ({\rm Graph}(a) \leftrightarrow {\rm Graph}(b))
\]
が成り立つ.
またこのことから, 次の($*$)が成り立つ: 

($*$) ~~$a = b$が成り立つとき, $a$がグラフならば$b$はグラフであり, 
        $b$がグラフならば$a$はグラフである.
\end{thm}


\noindent{\bf 証明}
~$x$を文字とするとき, Thm \ref{thms5eq}より
\[
  a = b \to ((a|x)({\rm Graph}(x)) \leftrightarrow (b|x)({\rm Graph}(x)))
\]
が成り立つが, 代入法則 \ref{substgraph}によればこの記号列は
\[
\tag{1}
  a = b \to ({\rm Graph}(a) \leftrightarrow {\rm Graph}(b))
\]
と一致するから, これが定理となる.

いま$a = b$が成り立つとすれば, これと(1)から推論法則 \ref{dedmp}によって
${\rm Graph}(a) \leftrightarrow {\rm Graph}(b)$が成り立つから, 
推論法則 \ref{dedeqfund}により, $a$がグラフならば$b$はグラフであり, 
$b$がグラフならば$a$はグラフである.
\halmos




\mathstrut
\begin{thm}
\label{sthmgraphpairsubset}%定理
$a$と$b$を集合とする.
また$x$と$y$を, 互いに異なり, 共に$a$及び$b$の中に自由変数として現れない文字とする.
このとき
\begin{align*}
  a \subset b &\to \forall x(\forall y((x, y) \in a \to (x, y) \in b)), \\
  \mbox{} \\
  {\rm Graph}(a) &\to (a \subset b \leftrightarrow \forall x(\forall y((x, y) \in a \to (x, y) \in b)))
\end{align*}
が成り立つ.
またこのことから, 次の1), 2), 3)が成り立つ.

1)
$a$がグラフならば, 
\[
  a \subset b \leftrightarrow \forall x(\forall y((x, y) \in a \to (x, y) \in b))
\]
が成り立つ.

2)
$a$がグラフであるとき, 
$\forall x(\forall y((x, y) \in a \to (x, y) \in b))$が成り立つならば, $a \subset b$が成り立つ.

3)
$a$がグラフであるとき, 
$x$と$y$が共に定数でなく, 
$(x, y) \in a \to (x, y) \in b$が成り立つならば, $a \subset b$が成り立つ.
\end{thm}


\noindent{\bf 証明}
~まず
\[
\tag{1}
  a \subset b \to \forall x(\forall y((x, y) \in a \to (x, y) \in b))
\]
が成り立つことを示す.
$\tau_{x}(\neg \forall y((x, y) \in a \to (x, y) \in b))$を$T$と書けば, 
$T$は集合であり, 変数法則 \ref{valfund}, \ref{valtau}, \ref{valquan}によってわかるように, 
$y$は$T$の中に自由変数として現れない.
また$\tau_{y}(\neg ((T, y) \in a \to (T, y) \in b))$を$U$と書けば, 
$U$も集合である.
そして定理 \ref{sthmsubsetbasis}より
\[
  a \subset b \to ((T, U) \in a \to (T, U) \in b)
\]
が成り立つが, 仮定より$y$は$a$及び$b$の中に自由変数として現れず, 
上述のように$T$の中にも自由変数として現れないから, 
代入法則 \ref{substfree}, \ref{substfund}, \ref{substpair}により, この記号列は
\[
\tag{2}
  a \subset b \to (U|y)((T, y) \in a \to (T, y) \in b)
\]
と一致する.
よってこれが定理となる.
また$U$の定義から, Thm \ref{thmallfund1}と推論法則 \ref{dedequiv}により
\[
  (U|y)((T, y) \in a \to (T, y) \in b) \to \forall y((T, y) \in a \to (T, y) \in b)
\]
が成り立つが, 
仮定より$x$は$y$と異なり, $a$及び$b$の中に自由変数として現れないから, 
代入法則 \ref{substfree}, \ref{substfund}, \ref{substpair}により, この記号列は
\[
  (U|y)((T, y) \in a \to (T, y) \in b) \to \forall y((T|x)((x, y) \in a \to (x, y) \in b))
\]
と一致する.
また$y$が$x$と異なり, 上述のように$T$の中に自由変数として現れないことから, 
代入法則 \ref{substquan}により, この記号列は
\[
\tag{3}
  (U|y)((T, y) \in a \to (T, y) \in b) \to (T|x)(\forall y((x, y) \in a \to (x, y) \in b))
\]
と一致する.
よってこれが定理となる.
また$T$の定義から, Thm \ref{thmallfund1}と推論法則 \ref{dedequiv}により
\[
\tag{4}
  (T|x)(\forall y((x, y) \in a \to (x, y) \in b)) \to \forall x(\forall y((x, y) \in a \to (x, y) \in b))
\]
が成り立つ.
そこで(2), (3), (4)から, 推論法則 \ref{dedmmp}によって
(1)が成り立つことがわかる.

次に
\[
\tag{5}
  {\rm Graph}(a) \to (a \subset b \leftrightarrow \forall x(\forall y((x, y) \in a \to (x, y) \in b)))
\]
が成り立つことを示す.
$\tau_{x}(\neg (x \in a \to x \in b))$を$V$と書けば, $V$は集合である.
また仮定より$y$は$x$と異なり, $a$及び$b$の中に自由変数として現れないから, 
変数法則 \ref{valfund}, \ref{valtau}によってわかるように, $y$は$V$の中に
自由変数として現れない.
そこで特に変数法則 \ref{valpr}により, $y$は
${\rm pr}_{1}(V)$の中にも自由変数として現れない.
また定理 \ref{sthmgraphbasis}より
\[
  {\rm Graph}(a) \to (V \in a \to {\rm Pair}(V))
\]
が成り立つから, これに推論法則 \ref{dedch}を適用して
\[
  V \in a \to ({\rm Graph}(a) \to {\rm Pair}(V))
\]
が成り立ち, これに推論法則 \ref{dedtwch}を適用して
\[
\tag{6}
  V \in a \wedge {\rm Graph}(a) \to {\rm Pair}(V)
\]
が成り立つ.
また定理 \ref{sthmbigpairpr}と推論法則 \ref{dedequiv}により
\[
\tag{7}
  {\rm Pair}(V) \to V = ({\rm pr}_{1}(V), {\rm pr}_{2}(V))
\]
が成り立つ.
そこで(6), (7)から, 推論法則 \ref{dedmmp}によって
\[
\tag{8}
  V \in a \wedge {\rm Graph}(a) \to V = ({\rm pr}_{1}(V), {\rm pr}_{2}(V))
\]
が成り立つ.
またThm \ref{thmallfund2}より
\[
\tag{9}
  \forall x(\forall y((x, y) \in a \to (x, y) \in b)) \to 
  ({\rm pr}_{1}(V)|x)(\forall y((x, y) \in a \to (x, y) \in b))
\]
が成り立つ.
ここで$y$が$x$と異なり, 上述のように${\rm pr}_{1}(V)$の中に自由変数として現れないことから, 
代入法則 \ref{substquan}により
\[
\tag{10}
  ({\rm pr}_{1}(V)|x)(\forall y((x, y) \in a \to (x, y) \in b)) \equiv 
  \forall y(({\rm pr}_{1}(V)|x)((x, y) \in a \to (x, y) \in b))
\]
が成り立つ.
また$x$が$y$と異なり, $a$及び$b$の中に自由変数として現れないことから, 
代入法則 \ref{substfree}, \ref{substfund}, \ref{substpair}により
\[
\tag{11}
  ({\rm pr}_{1}(V)|x)((x, y) \in a \to (x, y) \in b) \equiv 
  ({\rm pr}_{1}(V), y) \in a \to ({\rm pr}_{1}(V), y) \in b
\]
が成り立つ.
そこで(10)と(11)から, (9)が
\[
\tag{12}
  \forall x(\forall y((x, y) \in a \to (x, y) \in b)) \to 
  \forall y(({\rm pr}_{1}(V), y) \in a \to ({\rm pr}_{1}(V), y) \in b)
\]
と一致することがわかるから, これが定理となる.
またThm \ref{thmallfund2}より
\[
\tag{13}
  \forall y(({\rm pr}_{1}(V), y) \in a \to ({\rm pr}_{1}(V), y) \in b) \to 
  ({\rm pr}_{2}(V)|y)(({\rm pr}_{1}(V), y) \in a \to ({\rm pr}_{1}(V), y) \in b)
\]
が成り立つ.
ここで$y$が$a$及び$b$の中に自由変数として現れず, 上述のように${\rm pr}_{1}(V)$の中にも
自由変数として現れないことから, 
代入法則 \ref{substfree}, \ref{substfund}, \ref{substpair}により
\[
  ({\rm pr}_{2}(V)|y)(({\rm pr}_{1}(V), y) \in a \to ({\rm pr}_{1}(V), y) \in b) \equiv 
  ({\rm pr}_{1}(V), {\rm pr}_{2}(V)) \in a \to ({\rm pr}_{1}(V), {\rm pr}_{2}(V)) \in b
\]
が成り立つから, (13)は
\[
\tag{14}
  \forall y(({\rm pr}_{1}(V), y) \in a \to ({\rm pr}_{1}(V), y) \in b) \to 
  (({\rm pr}_{1}(V), {\rm pr}_{2}(V)) \in a \to ({\rm pr}_{1}(V), {\rm pr}_{2}(V)) \in b)
\]
と一致する.
よってこれが定理となる.
そこで(12), (14)から, 推論法則 \ref{dedmmp}によって
\[
  \forall x(\forall y((x, y) \in a \to (x, y) \in b)) \to 
  (({\rm pr}_{1}(V), {\rm pr}_{2}(V)) \in a \to ({\rm pr}_{1}(V), {\rm pr}_{2}(V)) \in b)
\]
が成り立つが, $x$が$a$及び$b$の中に自由変数として現れないことから, 
代入法則 \ref{substfree}, \ref{substfund}により, この記号列は
\[
\tag{15}
  \forall x(\forall y((x, y) \in a \to (x, y) \in b)) \to 
  (({\rm pr}_{1}(V), {\rm pr}_{2}(V))|x)(x \in a \to x \in b)
\]
と一致する.
よってこれが定理となる.
そこで(8), (15)から, 推論法則 \ref{dedfromaddw}によって
\begin{multline*}
\tag{16}
  (V \in a \wedge {\rm Graph}(a)) \wedge \forall x(\forall y((x, y) \in a \to (x, y) \in b)) \\
  \to V = ({\rm pr}_{1}(V), {\rm pr}_{2}(V)) \wedge (({\rm pr}_{1}(V), {\rm pr}_{2}(V))|x)(x \in a \to x \in b)
\end{multline*}
が成り立つ.
またThm \ref{aw1bwc1t1awb1wc}より
\begin{multline*}
\tag{17}
  V \in a \wedge ({\rm Graph}(a) \wedge \forall x(\forall y((x, y) \in a \to (x, y) \in b))) \\
  \to (V \in a \wedge {\rm Graph}(a)) \wedge \forall x(\forall y((x, y) \in a \to (x, y) \in b))
\end{multline*}
が成り立つ.
またThm \ref{thmfroms5t}より
\[
  V = ({\rm pr}_{1}(V), {\rm pr}_{2}(V)) \wedge (({\rm pr}_{1}(V), {\rm pr}_{2}(V))|x)(x \in a \to x \in b) \to 
  (V|x)(x \in a \to x \in b)
\]
が成り立つが, $x$が$a$及び$b$の中に自由変数として現れないことから, 
代入法則 \ref{substfree}, \ref{substfund}により, この記号列は
\[
\tag{18}
  V = ({\rm pr}_{1}(V), {\rm pr}_{2}(V)) \wedge (({\rm pr}_{1}(V), {\rm pr}_{2}(V))|x)(x \in a \to x \in b) \to 
  (V \in a \to V \in b)
\]
と一致する.
よってこれが定理となる.
そこで(17), (16), (18)から, 推論法則 \ref{dedmmp}によって
\[
  V \in a \wedge ({\rm Graph}(a) \wedge \forall x(\forall y((x, y) \in a \to (x, y) \in b))) \to (V \in a \to V \in b)
\]
が成り立つから, これに推論法則 \ref{dedtwch}を適用して
\[
  V \in a \to ({\rm Graph}(a) \wedge \forall x(\forall y((x, y) \in a \to (x, y) \in b)) \to (V \in a \to V \in b))
\]
が成り立ち, これに推論法則 \ref{dedch}を適用して
\[
\tag{19}
  {\rm Graph}(a) \wedge \forall x(\forall y((x, y) \in a \to (x, y) \in b)) \to (V \in a \to (V \in a \to V \in b))
\]
が成り立つ.
またThm \ref{1at1atb11t1atb1}より
\[
\tag{20}
  (V \in a \to (V \in a \to V \in b)) \to (V \in a \to V \in b)
\]
が成り立つ.
また$V$の定義から, Thm \ref{thmallfund1}と推論法則 \ref{dedequiv}により
\[
  (V|x)(x \in a \to x \in b) \to \forall x(x \in a \to x \in b)
\]
が成り立つが, $x$が$a$及び$b$の中に自由変数として現れないことから, 
代入法則 \ref{substfree}, \ref{substfund}及び定義により, この記号列は
\[
\tag{21}
  (V \in a \to V \in b) \to a \subset b
\]
と一致する.
よってこれが定理となる.
そこで(19), (20), (21)から, 推論法則 \ref{dedmmp}によって
\[
  {\rm Graph}(a) \wedge \forall x(\forall y((x, y) \in a \to (x, y) \in b)) \to a \subset b
\]
が成り立ち, これから推論法則 \ref{dedtwch}によって
\[
\tag{22}
  {\rm Graph}(a) \to (\forall x(\forall y((x, y) \in a \to (x, y) \in b)) \to a \subset b)
\]
が成り立つ.
また(1)から, 推論法則 \ref{deds1}により
\[
\tag{23}
  {\rm Graph}(a) \to (a \subset b \to \forall x(\forall y((x, y) \in a \to (x, y) \in b)))
\]
が成り立つ.
そこで(23), (22)から, 推論法則 \ref{dedprewedge}によって
(5)が成り立つ.

\noindent
1)
このとき上で示したように(5)が成り立つから, 
1)が成り立つことはこれと推論法則 \ref{dedmp}によって明らか.

\noindent
2)
このとき1)により
$a \subset b \leftrightarrow \forall x(\forall y((x, y) \in a \to (x, y) \in b))$が成り立つから, 
2)が成り立つことはこれと推論法則 \ref{dedeqfund}によって明らか.

\noindent
3)
このとき推論法則 \ref{dedltthmquan}により
$\forall x(\forall y((x, y) \in a \to (x, y) \in b))$が成り立つから, 
2)によって$a \subset b$が成り立つ.
\halmos




\mathstrut
\begin{thm}
\label{sthmgraphpair=}%定理
$a$と$b$を集合とする.
また$x$と$y$を, 互いに異なり, 共に$a$及び$b$の中に自由変数として現れない文字とする.
このとき
\begin{align*}
  a = b &\to \forall x(\forall y((x, y) \in a \leftrightarrow (x, y) \in b)), \\
  \mbox{} \\
  {\rm Graph}(a) \wedge {\rm Graph}(b) &\to (a = b \leftrightarrow \forall x(\forall y((x, y) \in a \leftrightarrow (x, y) \in b)))
\end{align*}
が成り立つ.
またこのことから, 次の1), 2), 3)が成り立つ.

1)
$a$と$b$が共にグラフならば, 
\[
  a = b \leftrightarrow \forall x(\forall y((x, y) \in a \leftrightarrow (x, y) \in b))
\]
が成り立つ.

2)
$a$と$b$が共にグラフであるとき, 
$\forall x(\forall y((x, y) \in a \leftrightarrow (x, y) \in b))$が成り立つならば, $a = b$が成り立つ.

3)
$a$と$b$が共にグラフであるとき, 
$x$と$y$が共に定数でなく, 
$(x, y) \in a \leftrightarrow (x, y) \in b$が成り立つならば, $a = b$が成り立つ.
\end{thm}


\noindent{\bf 証明}
~まず
\[
\tag{1}
  a = b \to \forall x(\forall y((x, y) \in a \leftrightarrow (x, y) \in b))
\]
が成り立つことを示す.
$\tau_{x}(\neg \forall y((x, y) \in a \leftrightarrow (x, y) \in b))$を$T$と書けば, 
$T$は集合であり, 変数法則 \ref{valfund}, \ref{valtau}, \ref{valquan}によってわかるように, 
$y$は$T$の中に自由変数として現れない.
また$\tau_{y}(\neg ((T, y) \in a \leftrightarrow (T, y) \in b))$を$U$と書けば, 
$U$も集合である.
そして定理 \ref{sthm=tineq}より
\[
  a = b \to ((T, U) \in a \leftrightarrow (T, U) \in b)
\]
が成り立つが, 仮定より$y$は$a$及び$b$の中に自由変数として現れず, 
上述のように$T$の中にも自由変数として現れないから, 
代入法則 \ref{substfree}, \ref{substfund}, \ref{substequiv}, \ref{substpair}により, この記号列は
\[
\tag{2}
  a = b \to (U|y)((T, y) \in a \leftrightarrow (T, y) \in b)
\]
と一致する.
よってこれが定理となる.
また$U$の定義から, Thm \ref{thmallfund1}と推論法則 \ref{dedequiv}により
\[
  (U|y)((T, y) \in a \leftrightarrow (T, y) \in b) \to \forall y((T, y) \in a \leftrightarrow (T, y) \in b)
\]
が成り立つが, 
仮定より$x$は$y$と異なり, $a$及び$b$の中に自由変数として現れないから, 
代入法則 \ref{substfree}, \ref{substfund}, \ref{substequiv}, \ref{substpair}により, この記号列は
\[
  (U|y)((T, y) \in a \leftrightarrow (T, y) \in b) \to \forall y((T|x)((x, y) \in a \leftrightarrow (x, y) \in b))
\]
と一致する.
また$y$が$x$と異なり, 上述のように$T$の中に自由変数として現れないことから, 
代入法則 \ref{substquan}により, この記号列は
\[
\tag{3}
  (U|y)((T, y) \in a \leftrightarrow (T, y) \in b) \to (T|x)(\forall y((x, y) \in a \leftrightarrow (x, y) \in b))
\]
と一致する.
よってこれが定理となる.
また$T$の定義から, Thm \ref{thmallfund1}と推論法則 \ref{dedequiv}により
\[
\tag{4}
  (T|x)(\forall y((x, y) \in a \leftrightarrow (x, y) \in b)) \to \forall x(\forall y((x, y) \in a \leftrightarrow (x, y) \in b))
\]
が成り立つ.
そこで(2), (3), (4)から, 推論法則 \ref{dedmmp}によって
(1)が成り立つことがわかる.

次に
\[
\tag{5}
  {\rm Graph}(a) \wedge {\rm Graph}(b) \to (a = b \leftrightarrow \forall x(\forall y((x, y) \in a \leftrightarrow (x, y) \in b)))
\]
が成り立つことを示す.
定理 \ref{sthmgraphpairsubset}より
\begin{align*}
  {\rm Graph}(a) &\to (a \subset b \leftrightarrow \forall x(\forall y((x, y) \in a \to (x, y) \in b))), \\
  \mbox{} \\
  {\rm Graph}(b) &\to (b \subset a \leftrightarrow \forall x(\forall y((x, y) \in b \to (x, y) \in a)))
\end{align*}
が共に成り立つから, 推論法則 \ref{dedprewedge}により
\begin{align*}
  {\rm Graph}(a) &\to (\forall x(\forall y((x, y) \in a \to (x, y) \in b)) \to a \subset b), \\
  \mbox{} \\
  {\rm Graph}(b) &\to (\forall x(\forall y((x, y) \in b \to (x, y) \in a)) \to b \subset a)
\end{align*}
が共に成り立つ.
そこでこれらから, 推論法則 \ref{dedfromaddw}によって
\begin{multline*}
\tag{6}
  {\rm Graph}(a) \wedge {\rm Graph}(b) \\
  \to (\forall x(\forall y((x, y) \in a \to (x, y) \in b)) \to a \subset b) \wedge (\forall x(\forall y((x, y) \in b \to (x, y) \in a)) \to b \subset a)
\end{multline*}
が成り立つ.
またThm \ref{1atb1w1ctd1t1awctbwd1}より
\begin{multline*}
\tag{7}
  (\forall x(\forall y((x, y) \in a \to (x, y) \in b)) \to a \subset b) \wedge (\forall x(\forall y((x, y) \in b \to (x, y) \in a)) \to b \subset a) \\
  \to (\forall x(\forall y((x, y) \in a \to (x, y) \in b)) \wedge \forall x(\forall y((x, y) \in b \to (x, y) \in a)) \to a \subset b \wedge b \subset a)
\end{multline*}
が成り立つ.
いま$u$を$x$とも$y$とも異なり, $a$及び$b$の中に自由変数として現れない, 定数でない文字とする.
このときThm \ref{thmallw}と推論法則 \ref{dedequiv}により
\[
  \forall y((u, y) \in a \leftrightarrow (u, y) \in b) \to 
  \forall y((u, y) \in a \to (u, y) \in b) \wedge \forall y((u, y) \in b \to (u, y) \in a)
\]
が成り立つから, $u$が定数でないことから, 推論法則 \ref{dedalltquansepconst}により
\[
\tag{8}
  \forall u(\forall y((u, y) \in a \leftrightarrow (u, y) \in b)) \to 
  \forall u(\forall y((u, y) \in a \to (u, y) \in b) \wedge \forall y((u, y) \in b \to (u, y) \in a))
\]
が成り立つ.
またThm \ref{thmallw}と推論法則 \ref{dedequiv}により
\begin{multline*}
\tag{9}
  \forall u(\forall y((u, y) \in a \to (u, y) \in b) \wedge \forall y((u, y) \in b \to (u, y) \in a)) \\
  \to \forall u(\forall y((u, y) \in a \to (u, y) \in b)) \wedge \forall u(\forall y((u, y) \in b \to (u, y) \in a))
\end{multline*}
が成り立つ.
そこで(8), (9)から, 推論法則 \ref{dedmmp}によって
\[
\tag{10}
  \forall u(\forall y((u, y) \in a \leftrightarrow (u, y) \in b)) \to 
  \forall u(\forall y((u, y) \in a \to (u, y) \in b)) \wedge \forall u(\forall y((u, y) \in b \to (u, y) \in a))
\]
が成り立つ.
ここで$x$が$y$とも$u$とも異なり, $a$及び$b$の中に自由変数として現れないことから, 
変数法則 \ref{valfund}, \ref{valequiv}, \ref{valquan}, \ref{valpair}によって
$x$が
\[
  \forall y((u, y) \in a \leftrightarrow (u, y) \in b), ~~
  \forall y((u, y) \in a \to (u, y) \in b), ~~
  \forall y((u, y) \in b \to (u, y) \in a)
\]
のいずれの記号列の中にも自由変数として現れないことが
わかるから, 代入法則 \ref{substquantrans}により
\begin{align*}
  \tag{11}
  \forall u(\forall y((u, y) \in a \leftrightarrow (u, y) \in b)) &\equiv 
  \forall x((x|u)(\forall y((u, y) \in a \leftrightarrow (u, y) \in b))), \\
  \mbox{} \\
  \tag{12}
  \forall u(\forall y((u, y) \in a \to (u, y) \in b)) &\equiv 
  \forall x((x|u)(\forall y((u, y) \in a \to (u, y) \in b))), \\
  \mbox{} \\
  \tag{13}
  \forall u(\forall y((u, y) \in b \to (u, y) \in a)) &\equiv 
  \forall x((x|u)(\forall y((u, y) \in b \to (u, y) \in a)))
\end{align*}
が成り立つ.
また$y$が$x$とも$u$とも異なることから, 代入法則 \ref{substquan}により
\begin{align*}
  \tag{14}
  (x|u)(\forall y((u, y) \in a \leftrightarrow (u, y) \in b)) &\equiv 
  \forall y((x|u)((u, y) \in a \leftrightarrow (u, y) \in b)), \\
  \mbox{} \\
  \tag{15}
  (x|u)(\forall y((u, y) \in a \to (u, y) \in b)) &\equiv 
  \forall y((x|u)((u, y) \in a \to (u, y) \in b)), \\
  \mbox{} \\
  \tag{16}
  (x|u)(\forall y((u, y) \in b \to (u, y) \in a)) &\equiv 
  \forall y((x|u)((u, y) \in b \to (u, y) \in a))
\end{align*}
が成り立つ.
また$u$が$y$と異なり, $a$及び$b$の中に自由変数として現れないことから, 
代入法則 \ref{substfree}, \ref{substfund}, \ref{substequiv}, \ref{substpair}により
\begin{align*}
  \tag{17}
  (x|u)((u, y) \in a \leftrightarrow (u, y) \in b) &\equiv (x, y) \in a \leftrightarrow (x, y) \in b, \\
  \mbox{} \\
  \tag{18}
  (x|u)((u, y) \in a \to (u, y) \in b) &\equiv (x, y) \in a \to (x, y) \in b, \\
  \mbox{} \\
  \tag{19}
  (x|u)((u, y) \in b \to (u, y) \in a) &\equiv (x, y) \in b \to (x, y) \in a
\end{align*}
が成り立つ.
そこで(11)---(19)から, (10)が
\[
  \forall x(\forall y((x, y) \in a \leftrightarrow (x, y) \in b)) \to 
  \forall x(\forall y((x, y) \in a \to (x, y) \in b)) \wedge \forall x(\forall y((x, y) \in b \to (x, y) \in a))
\]
と一致することがわかる.
故にこれが定理となる.
そこでこれに推論法則 \ref{dedaddf}を適用して, 
\begin{multline*}
\tag{20}
  (\forall x(\forall y((x, y) \in a \to (x, y) \in b)) \wedge \forall x(\forall y((x, y) \in b \to (x, y) \in a)) \to a \subset b \wedge b \subset a) \\
  \to (\forall x(\forall y((x, y) \in a \leftrightarrow (x, y) \in b)) \to a \subset b \wedge b \subset a)
\end{multline*}
が成り立つ.
また定理 \ref{sthmaxiom1}と推論法則 \ref{dedequiv}により
\[
  a \subset b \wedge b \subset a \to a = b
\]
が成り立つから, これに推論法則 \ref{dedaddb}を適用して, 
\[
\tag{21}
  (\forall x(\forall y((x, y) \in a \leftrightarrow (x, y) \in b)) \to a \subset b \wedge b \subset a) \to 
  (\forall x(\forall y((x, y) \in a \leftrightarrow (x, y) \in b)) \to a = b)
\]
が成り立つ.
そこで(6), (7), (20), (21)から, 推論法則 \ref{dedmmp}によって
\[
\tag{22}
  {\rm Graph}(a) \wedge {\rm Graph}(b) \to (\forall x(\forall y((x, y) \in a \leftrightarrow (x, y) \in b)) \to a = b)
\]
が成り立つことがわかる.
また(1)から, 推論法則 \ref{deds1}により
\[
\tag{23}
  {\rm Graph}(a) \wedge {\rm Graph}(b) \to (a = b \to \forall x(\forall y((x, y) \in a \leftrightarrow (x, y) \in b)))
\]
が成り立つ.
そこで(23), (22)から, 推論法則 \ref{dedprewedge}によって(5)が成り立つ.

\noindent
1)
このとき推論法則 \ref{dedwedge}により${\rm Graph}(a) \wedge {\rm Graph}(b)$が成り立ち, 
また上で示したように(5)が成り立つから, 
これらから, 推論法則 \ref{dedmp}によって
$a = b \leftrightarrow \forall x(\forall y((x, y) \in a \leftrightarrow (x, y) \in b))$が成り立つ.

\noindent
2)
このとき1)により
$a = b \leftrightarrow \forall x(\forall y((x, y) \in a \leftrightarrow (x, y) \in b))$が成り立つから, 
2)が成り立つことはこれと推論法則 \ref{dedeqfund}によって明らか.

\noindent
3)
このとき推論法則 \ref{dedltthmquan}により
$\forall x(\forall y((x, y) \in a \leftrightarrow (x, y) \in b))$が成り立つから, 
2)によって$a = b$が成り立つ.
\halmos




\mathstrut
\begin{thm}
\label{sthmuopairgraph}%定理
$a$と$b$を集合とするとき, 
\[
  {\rm Graph}(\{a, b\}) \leftrightarrow {\rm Pair}(a) \wedge {\rm Pair}(b)
\]
が成り立つ.
またこのことから, 次の($*$)が成り立つ: 

($*$) ~~$\{a, b\}$がグラフならば, $a$と$b$は共に対である.
        逆に$a$と$b$が共に対ならば, $\{a, b\}$はグラフである.
\end{thm}


\noindent{\bf 証明}
~まず前半を示す.
推論法則 \ref{dedequiv}があるから, 
\begin{align*}
  \tag{1}
  &{\rm Graph}(\{a, b\}) \to {\rm Pair}(a) \wedge {\rm Pair}(b), \\
  \mbox{} \\
  \tag{2}
  &{\rm Pair}(a) \wedge {\rm Pair}(b) \to {\rm Graph}(\{a, b\})
\end{align*}
が共に成り立つことを示せば良い.

(1)の証明: 
定理 \ref{sthmgraphbasis}より
\[
  {\rm Graph}(\{a, b\}) \to (a \in \{a, b\} \to {\rm Pair}(a)), ~~
  {\rm Graph}(\{a, b\}) \to (b \in \{a, b\} \to {\rm Pair}(b))
\]
が共に成り立つから, 推論法則 \ref{dedch}により
\begin{align*}
  \tag{3}
  a \in \{a, b\} &\to ({\rm Graph}(\{a, b\}) \to {\rm Pair}(a)), \\
  \mbox{} \\
  \tag{4}
  b \in \{a, b\} &\to ({\rm Graph}(\{a, b\}) \to {\rm Pair}(b))
\end{align*}
が共に成り立つ.
また定理 \ref{sthmuopairfund}より
\begin{align*}
  \tag{5}
  a &\in \{a, b\}, \\
  \mbox{} \\
  \tag{6}
  b &\in \{a, b\}
\end{align*}
が共に成り立つ.
そこで(5)と(3), (6)と(4)から, それぞれ推論法則 \ref{dedmp}によって
\[
  {\rm Graph}(\{a, b\}) \to {\rm Pair}(a), ~~
  {\rm Graph}(\{a, b\}) \to {\rm Pair}(b)
\]
が共に成り立つ.
そこでこれらから, 推論法則 \ref{dedprewedge}によって(1)が成り立つ.

(2)の証明: 
$x$を$a$及び$b$の中に自由変数として現れない文字とする.
このとき変数法則 \ref{valnset}により, $x$は$\{a, b\}$の中に
自由変数として現れない.
また$\tau_{x}(\neg (x \in \{a, b\} \to {\rm Pair}(x)))$を$T$と書く.
$T$は集合であり, 定理 \ref{sthm=bigpaireq}より
\[
  T = a \to ({\rm Pair}(T) \leftrightarrow {\rm Pair}(a)), ~~
  T = b \to ({\rm Pair}(T) \leftrightarrow {\rm Pair}(b))
\]
が共に成り立つ.
そこでこれらにそれぞれ推論法則 \ref{dedprewedge}を適用して
\[
  T = a \to ({\rm Pair}(a) \to {\rm Pair}(T)), ~~
  T = b \to ({\rm Pair}(b) \to {\rm Pair}(T))
\]
が共に成り立つから, これらにそれぞれ推論法則 \ref{dedch}を適用して, 
\[
  {\rm Pair}(a) \to (T = a \to {\rm Pair}(T)), ~~
  {\rm Pair}(b) \to (T = b \to {\rm Pair}(T))
\]
が共に成り立つ.
そこでこれらから, 推論法則 \ref{dedfromaddw}によって
\[
\tag{7}
  {\rm Pair}(a) \wedge {\rm Pair}(b) \to (T = a \to {\rm Pair}(T)) \wedge (T = b \to {\rm Pair}(T))
\]
が成り立つ.
またThm \ref{1atc1w1btc1t1avbtc1}より
\[
\tag{8}
  (T = a \to {\rm Pair}(T)) \wedge (T = b \to {\rm Pair}(T)) \to 
  (T = a \vee T = b \to {\rm Pair}(T))
\]
が成り立つ.
また定理 \ref{sthmuopairbasis}と推論法則 \ref{dedequiv}により
\[
  T \in \{a, b\} \to T = a \vee T = b
\]
が成り立つから, 推論法則 \ref{dedaddf}により
\[
\tag{9}
  (T = a \vee T = b \to {\rm Pair}(T)) \to (T \in \{a, b\} \to {\rm Pair}(T))
\]
が成り立つ.
また$T$の定義から, Thm \ref{thmallfund1}と推論法則 \ref{dedequiv}により
\[
  (T|x)(x \in \{a, b\} \to {\rm Pair}(x)) \to \forall x(x \in \{a, b\} \to {\rm Pair}(x))
\]
が成り立つが, 上述のように$x$は$\{a, b\}$の中に自由変数として現れないから, 
代入法則 \ref{substfree}, \ref{substfund}, \ref{substbigpair}及び定義より, 
この記号列は
\[
\tag{10}
  (T \in \{a, b\} \to {\rm Pair}(T)) \to {\rm Graph}(\{a, b\})
\]
と一致する.
よってこれが定理となる.
そこで(7)---(10)から, 推論法則 \ref{dedmmp}によって(2)が成り立つことがわかる.

さていま$\{a, b\}$がグラフであるとする.
このときこれと(1)から, 推論法則 \ref{dedmp}によって
${\rm Pair}(a) \wedge {\rm Pair}(b)$が成り立つから, 推論法則 \ref{dedwedge}により
${\rm Pair}(a)$と${\rm Pair}(b)$が共に成り立つ.
逆に$a$と$b$が共に対ならば, 推論法則 \ref{dedwedge}により
${\rm Pair}(a) \wedge {\rm Pair}(b)$が成り立つから, これと
(2)から, 推論法則 \ref{dedmp}によって${\rm Graph}(\{a, b\})$が成り立つ.
これで($*$)が成り立つことが示された.
\halmos




\mathstrut
\begin{thm}
\label{sthmsingletongraph}%定理
$a$を集合とするとき, 
\[
  {\rm Graph}(\{a\}) \leftrightarrow {\rm Pair}(a)
\]
が成り立つ.
またこのことから, 次の($*$)が成り立つ: 

($*$) ~~$\{a\}$がグラフならば, $a$は対である.
        逆に$a$が対ならば, $\{a\}$はグラフである.
\end{thm}


\noindent{\bf 証明}
~定理 \ref{sthmuopairgraph}より
\[
  {\rm Graph}(\{a\}) \leftrightarrow {\rm Pair}(a) \wedge {\rm Pair}(a)
\]
が成り立ち, Thm \ref{awala}より
\[
  {\rm Pair}(a) \wedge {\rm Pair}(a) \leftrightarrow {\rm Pair}(a)
\]
が成り立つから, これらから, 推論法則 \ref{dedeqtrans}によって
${\rm Graph}(\{a\}) \leftrightarrow {\rm Pair}(a)$が成り立つ.
($*$)が成り立つことは, これと推論法則 \ref{dedeqfund}によって明らか.
\halmos




\mathstrut
\begin{thm}
\label{sthmsetsepgraph}%定理
$a$を集合, $R$を関係式とし, $x$を$a$の中に自由変数として現れない文字とする.
このとき
\[
  {\rm Graph}(a) \to {\rm Graph}(\{x \in a|R\})
\]
が成り立つ.
またこのことから, 次の($*$)が成り立つ: 

($*$) ~~$a$がグラフならば, $\{x \in a|R\}$はグラフである.
\end{thm}


\noindent{\bf 証明}
~このとき定理 \ref{sthmssetsubseta}より
\[
  \{x \in a|R\} \subset a
\]
が成り立ち, 定理 \ref{sthmgraphsubset}より
\[
  \{x \in a|R\} \subset a \to ({\rm Graph}(a) \to {\rm Graph}(\{x \in a|R\}))
\]
が成り立つから, これらから, 推論法則 \ref{dedmp}によって
${\rm Graph}(a) \to {\rm Graph}(\{x \in a|R\})$が成り立つ.
($*$)が成り立つことはこれと推論法則 \ref{dedmp}によって明らか.
\halmos




\mathstrut
\begin{thm}
\label{sthmpairsetofagraph}%定理
$a$を集合とし, $x$を$a$の中に自由変数として現れない文字とする.
このとき, $\{x \in a|{\rm Pair}(x)\}$はグラフである.
また, 
\[
  {\rm Graph}(a) \leftrightarrow a = \{x \in a|{\rm Pair}(x)\}
\]
が成り立つ.
更にこのことから, 次の($*$)が成り立つ: 

($*$) ~~$a$がグラフならば, $a = \{x \in a|{\rm Pair}(x)\}$が成り立つ.
\end{thm}


\noindent{\bf 証明}
~まず$\{x \in a|{\rm Pair}(x)\}$がグラフであることを示す.
$u$を$x$と異なり, $a$の中に自由変数として現れない, 定数でない文字とする.
このとき変数法則 \ref{valsset}, \ref{valbigpair}からわかるように, 
$u$は$\{x \in a|{\rm Pair}(x)\}$の中に自由変数として現れない.
また$x$が$a$の中に自由変数として現れないという仮定から, 
定理 \ref{sthmssetbasis}と推論法則 \ref{dedequiv}により
\[
  u \in \{x \in a|{\rm Pair}(x)\} \to u \in a \wedge (u|x)({\rm Pair}(x))
\]
が成り立つ.
またThm \ref{awbta}より
\[
  u \in a \wedge (u|x)({\rm Pair}(x)) \to (u|x)({\rm Pair}(x))
\]
が成り立つ.
そこでこれらから, 推論法則 \ref{dedmmp}によって
\[
  u \in \{x \in a|{\rm Pair}(x)\} \to (u|x)({\rm Pair}(x))
\]
が成り立つが, 代入法則 \ref{substbigpair}によりこの記号列は
\[
  u \in \{x \in a|{\rm Pair}(x)\} \to {\rm Pair}(u)
\]
と一致するから, これが定理となる.
そこで$u$が定数でないことから, 推論法則 \ref{dedltthmquan}により
\[
  \forall u(u \in \{x \in a|{\rm Pair}(x)\} \to {\rm Pair}(u))
\]
が成り立つ.
上述のように$u$は$\{x \in a|{\rm Pair}(x)\}$の中に自由変数として現れないから, 
定義からこの記号列は${\rm Graph}(\{x \in a|{\rm Pair}(x)\})$と同じである.
故にこれが定理となる.

次に${\rm Graph}(a) \leftrightarrow a = \{x \in a|{\rm Pair}(x)\}$が成り立つことを示す.
推論法則 \ref{dedequiv}があるから, 
\begin{align*}
  \tag{1}
  &{\rm Graph}(a) \to a = \{x \in a|{\rm Pair}(x)\}, \\
  \mbox{} \\
  \tag{2}
  &a = \{x \in a|{\rm Pair}(x)\} \to {\rm Graph}(a)
\end{align*}
が共に成り立つことを示せば良い.

(1)の証明: 
$\tau_{x}(\neg (x \in a \to x \in \{x \in a|{\rm Pair}(x)\}))$を$T$と書けば, 
$T$は集合であり, 定理 \ref{sthmgraphbasis}より
\[
\tag{3}
  {\rm Graph}(a) \to (T \in a \to {\rm Pair}(T))
\]
が成り立つ.
またThm \ref{ata}より
\[
  T \in a \to T \in a
\]
が成り立ち, Thm \ref{1cta1t11ctb1t1ctawb11}より
\[
  (T \in a \to T \in a) \to ((T \in a \to {\rm Pair}(T)) \to (T \in a \to T \in a \wedge {\rm Pair}(T)))
\]
が成り立つから, これらから, 推論法則 \ref{dedmp}によって
\[
\tag{4}
  (T \in a \to {\rm Pair}(T)) \to (T \in a \to T \in a \wedge {\rm Pair}(T))
\]
が成り立つ.
また$x$が$a$の中に自由変数として現れないことから, 
定理 \ref{sthmssetbasis}と推論法則 \ref{dedequiv}により
\[
  T \in a \wedge (T|x)({\rm Pair}(x)) \to T \in \{x \in a|{\rm Pair}(x)\}
\]
が成り立つが, 代入法則 \ref{substbigpair}によりこの記号列は
\[
  T \in a \wedge {\rm Pair}(T) \to T \in \{x \in a|{\rm Pair}(x)\}
\]
と一致するから, これが定理となる.
そこでこれに推論法則 \ref{dedaddb}を適用して, 
\[
\tag{5}
  (T \in a \to T \in a \wedge {\rm Pair}(T)) \to (T \in a \to T \in \{x \in a|{\rm Pair}(x)\})
\]
が成り立つ.
また$T$の定義から, Thm \ref{thmallfund1}と推論法則 \ref{dedequiv}により
\[
  (T|x)(x \in a \to x \in \{x \in a|{\rm Pair}(x)\}) \to \forall x(x \in a \to x \in \{x \in a|{\rm Pair}(x)\})
\]
が成り立つが, 仮定より$x$は$a$の中に自由変数として現れず, 
また変数法則 \ref{valsset}により$x$は$\{x \in a|{\rm Pair}(x)\}$の中にも自由変数として現れないから, 
代入法則 \ref{substfree}, \ref{substfund}及び定義より, この記号列は
\[
\tag{6}
  (T \in a \to T \in \{x \in a|{\rm Pair}(x)\}) \to a \subset \{x \in a|{\rm Pair}(x)\}
\]
と一致する.
よってこれが定理となる.
また$x$が$a$の中に自由変数として現れないことから, 定理 \ref{sthmssetsubseta}より
\[
  \{x \in a|{\rm Pair}(x)\} \subset a
\]
が成り立つから, 推論法則 \ref{dedatawbtrue2}により
\[
\tag{7}
  a \subset \{x \in a|{\rm Pair}(x)\} \to a \subset \{x \in a|{\rm Pair}(x)\} \wedge \{x \in a|{\rm Pair}(x)\} \subset a
\]
が成り立つ.
また定理 \ref{sthmaxiom1}と推論法則 \ref{dedequiv}により
\[
\tag{8}
  a \subset \{x \in a|{\rm Pair}(x)\} \wedge \{x \in a|{\rm Pair}(x)\} \subset a \to a = \{x \in a|{\rm Pair}(x)\}
\]
が成り立つ.
そこで(3)---(8)から, 推論法則 \ref{dedmmp}によって
(1)が成り立つことがわかる.

(2)の証明: 
定理 \ref{sthm=tsubset}より
\[
\tag{9}
  a = \{x \in a|{\rm Pair}(x)\} \to a \subset \{x \in a|{\rm Pair}(x)\}
\]
が成り立つ.
また定理 \ref{sthmgraphsubset}より
\[
  a \subset \{x \in a|{\rm Pair}(x)\} \to ({\rm Graph}(\{x \in a|{\rm Pair}(x)\}) \to {\rm Graph}(a))
\]
が成り立つから, 推論法則 \ref{dedch}により
\[
\tag{10}
  {\rm Graph}(\{x \in a|{\rm Pair}(x)\}) \to (a \subset \{x \in a|{\rm Pair}(x)\} \to {\rm Graph}(a))
\]
が成り立つ.
また上に示したように${\rm Graph}(\{x \in a|{\rm Pair}(x)\})$が成り立つから, 
これと(10)から, 推論法則 \ref{dedmp}によって
\[
\tag{11}
  a \subset \{x \in a|{\rm Pair}(x)\} \to {\rm Graph}(a)
\]
が成り立つ.
(9), (11)から, 推論法則 \ref{dedmmp}によって(2)が成り立つ.

($*$)が成り立つことは, (1)と推論法則 \ref{dedmp}によって明らかである.
\halmos




\mathstrut
\begin{thm}
\label{sthmobjectsetgraph}%定理
$a$と$T$を集合とし, $x$を$a$の中に自由変数として現れない文字とする.
このとき
\[
  \forall x(x \in a \to {\rm Pair}(T)) \leftrightarrow {\rm Graph}(\{T|x \in a\})
\]
が成り立つ.
特に
\[
  \forall x({\rm Pair}(T)) \to {\rm Graph}(\{T|x \in a\})
\]
が成り立つ.
またこれらから, 次の1) -- 4)が成り立つ.

1)
$\forall x(x \in a \to {\rm Pair}(T))$が成り立つならば, $\{T|x \in a\}$はグラフである.
逆に$\{T|x \in a\}$がグラフならば, $\forall x(x \in a \to {\rm Pair}(T))$が成り立つ.

2)
$x$が定数でなく, $x \in a \to {\rm Pair}(T)$が成り立つならば, $\{T|x \in a\}$はグラフである.

3)
$\forall x({\rm Pair}(T))$が成り立つならば, $\{T|x \in a\}$はグラフである.

4)
$x$が定数でなく, $T$が対ならば, $\{T|x \in a\}$はグラフである.
\end{thm}


\noindent{\bf 証明}
~まず$\forall x(x \in a \to {\rm Pair}(T)) \leftrightarrow {\rm Graph}(\{T|x \in a\})$が
成り立つことを示す.
推論法則 \ref{dedequiv}があるから, 
\begin{align*}
  \tag{1}
  &\forall x(x \in a \to {\rm Pair}(T)) \to {\rm Graph}(\{T|x \in a\}), \\
  \mbox{} \\
  \tag{2}
  &{\rm Graph}(\{T|x \in a\}) \to \forall x(x \in a \to {\rm Pair}(T))
\end{align*}
が共に成り立つことを示せば良い.

(1)の証明: 
$y$を$x$と異なり, $a$及び$T$の中に自由変数として現れない文字とする.
このとき変数法則 \ref{valoset}により, $y$は$\{T|x \in a\}$の中に自由変数として現れない.
また$\tau_{y}(\neg (y \in \{T|x \in a\} \to {\rm Pair}(y)))$を$U$と書けば, 
$U$は集合であり, 変数法則 \ref{valfund}, \ref{valtau}, \ref{valoset}, \ref{valbigpair}によって
わかるように, $x$は$U$の中に自由変数として現れない.
また$\tau_{x}(x \in a \wedge U = T)$を$V$と書けば, $V$も集合である.
そしてThm \ref{thmallfund2}より
\[
  \forall x(x \in a \to {\rm Pair}(T)) \to (V|x)(x \in a \to {\rm Pair}(T))
\]
が成り立つが, 仮定より$x$は$a$の中に自由変数として現れないから, 
代入法則 \ref{substfree}, \ref{substfund}, \ref{substbigpair}により, 
この記号列は
\[
\tag{3}
  \forall x(x \in a \to {\rm Pair}(T)) \to (V \in a \to {\rm Pair}((V|x)(T)))
\]
と一致する.
よってこれが定理となる.
またThm \ref{1atb1t1awctbwc1}より
\[
\tag{4}
  (V \in a \to {\rm Pair}((V|x)(T))) \to 
  (V \in a \wedge U = (V|x)(T) \to {\rm Pair}((V|x)(T)) \wedge U = (V|x)(T))
\]
が成り立つ.
また上述のように$x$が$a$及び$U$の中に自由変数として現れないことから, 
定理 \ref{sthmosetbasis}と推論法則 \ref{dedequiv}により
\[
  U \in \{T|x \in a\} \to \exists x(x \in a \wedge U = T)
\]
が成り立つ.
ここで$V$の定義から, この記号列は
\[
  U \in \{T|x \in a\} \to (V|x)(x \in a \wedge U = T)
\]
と一致する.
またいま述べたように$x$は$a$及び$U$の中に
自由変数として現れないから, 代入法則 \ref{substfree}, \ref{substfund}, \ref{substwedge}により, 
この記号列は
\[
  U \in \{T|x \in a\} \to V \in a \wedge U = (V|x)(T)
\]
と一致する.
よってこれが定理となる.
そこでこれに推論法則 \ref{dedaddf}を適用して, 
\begin{multline*}
\tag{5}
  (V \in a \wedge U = (V|x)(T) \to {\rm Pair}((V|x)(T)) \wedge U = (V|x)(T)) \\
  \to (U \in \{T|x \in a\} \to {\rm Pair}((V|x)(T)) \wedge U = (V|x)(T))
\end{multline*}
が成り立つ.
また定理 \ref{sthm=bigpaireq}より
\[
  U = (V|x)(T) \to ({\rm Pair}(U) \leftrightarrow {\rm Pair}((V|x)(T)))
\]
が成り立つから, 推論法則 \ref{dedprewedge}により
\[
  U = (V|x)(T) \to ({\rm Pair}((V|x)(T)) \to {\rm Pair}(U))
\]
が成り立つ.
そこでこれに推論法則 \ref{dedch}を適用して
\[
  {\rm Pair}((V|x)(T)) \to (U = (V|x)(T) \to {\rm Pair}(U))
\]
が成り立ち, これに推論法則 \ref{dedtwch}を適用して
\[
  {\rm Pair}((V|x)(T)) \wedge U = (V|x)(T) \to {\rm Pair}(U)
\]
が成り立つ.
そこでこれに推論法則 \ref{dedaddb}を適用して, 
\[
\tag{6}
  (U \in \{T|x \in a\} \to {\rm Pair}((V|x)(T)) \wedge U = (V|x)(T)) \to (U \in \{T|x \in a\} \to {\rm Pair}(U))
\]
が成り立つ.
また$U$の定義から, Thm \ref{thmallfund1}と推論法則 \ref{dedequiv}により
\[
  (U|y)(y \in \{T|x \in a\} \to {\rm Pair}(y)) \to \forall y(y \in \{T|x \in a\} \to {\rm Pair}(y))
\]
が成り立つが, 上述のように$y$は$\{T|x \in a\}$の中に自由変数として現れないから, 
代入法則 \ref{substfree}, \ref{substfund}, \ref{substbigpair}及び定義によれば, 
この記号列は
\[
\tag{7}
  (U \in \{T|x \in a\} \to {\rm Pair}(U)) \to {\rm Graph}(\{T|x \in a\})
\]
と一致する.
よってこれが定理となる.
そこで(3)---(7)から, 推論法則 \ref{dedmmp}によって
(1)が成り立つことがわかる.

(2)の証明: 
$\tau_{x}(\neg (x \in a \to {\rm Pair}(T)))$を$W$と書けば, $W$は集合であり, 
定理 \ref{sthmgraphbasis}より
\[
\tag{8}
  {\rm Graph}(\{T|x \in a\}) \to ((W|x)(T) \in \{T|x \in a\} \to {\rm Pair}((W|x)(T)))
\]
が成り立つ.
また$x$が$a$の中に自由変数として現れないことから, 定理 \ref{sthmosetfund}より
\[
  W \in a \to (W|x)(T) \in \{T|x \in a\}
\]
が成り立つ.
そこでこれに推論法則 \ref{dedaddf}を適用して, 
\[
\tag{9}
  ((W|x)(T) \in \{T|x \in a\} \to {\rm Pair}((W|x)(T))) \to (W \in a \to {\rm Pair}((W|x)(T)))
\]
が成り立つ.
また$W$の定義から, Thm \ref{thmallfund1}と推論法則 \ref{dedequiv}により
\[
  (W|x)(x \in a \to {\rm Pair}(T)) \to \forall x(x \in a \to {\rm Pair}(T))
\]
が成り立つが, $x$が$a$の中に自由変数として現れないことから, 
代入法則 \ref{substfree}, \ref{substfund}, \ref{substbigpair}により, この記号列は
\[
\tag{10}
  (W \in a \to {\rm Pair}((W|x)(T))) \to \forall x(x \in a \to {\rm Pair}(T))
\]
と一致する.
よってこれが定理となる.
そこで(8), (9), (10)から, 推論法則 \ref{dedmmp}によって
(2)が成り立つことがわかる.

次に
\[
\tag{11}
  \forall x({\rm Pair}(T)) \to {\rm Graph}(\{T|x \in a\})
\]
が成り立つことを示す.
$W$は上と同じとするとき, schema S1の適用により
\[
  (W|x)({\rm Pair}(T)) \to (W \in a \to (W|x)({\rm Pair}(T)))
\]
が成り立つが, $x$が$a$の中に自由変数として現れないことから, 
代入法則 \ref{substfree}, \ref{substfund}により, この記号列は
\[
  (W|x)({\rm Pair}(T) \to (x \in a \to {\rm Pair}(T)))
\]
と一致する.
よってこれが定理となる.
そこで$W$の定義から, 推論法則 \ref{dedtquanfund}によって
\[
\tag{12}
  \forall x({\rm Pair}(T)) \to \forall x(x \in a \to {\rm Pair}(T))
\]
が成り立つ.
また上に示したように(1)が成り立つ.
そこで(12), (1)から, 推論法則 \ref{dedmmp}によって
(11)が成り立つ.

\noindent
1)
上で示したように
$\forall x(x \in a \to {\rm Pair}(T)) \leftrightarrow {\rm Graph}(\{T|x \in a\})$が成り立つから, 
これと推論法則 \ref{dedeqfund}によって1)が成り立つことがわかる.

\noindent
2)
このとき推論法則 \ref{dedltthmquan}により
$\forall x(x \in a \to {\rm Pair}(T))$が成り立つから, 
1)により2)が成り立つ.

\noindent
3)
上で示したように(11)が成り立つから, これと推論法則 \ref{dedmp}によって
3)が成り立つことがわかる.

\noindent
4)
このとき推論法則 \ref{dedltthmquan}により
$\forall x({\rm Pair}(T))$が成り立つから, 3)により4)が成り立つ.
\halmos




\mathstrut
\begin{thm}
\label{sthmobjectsetgraph2}%定理
$a$, $T$, $U$を集合とし, $x$を$a$の中に自由変数として現れない文字とする.
このとき, $\{(T, U)|x \in a\}$はグラフである.
\end{thm}


\noindent{\bf 証明}
~$\tau_{x}(\neg {\rm Pair}((T, U)))$を$V$と書けば, $V$は集合であり, 
定理 \ref{sthmbigpairpair}より
\[
  {\rm Pair}(((V|x)(T), (V|x)(U)))
\]
が成り立つが, 代入法則 \ref{substpair}, \ref{substbigpair}によりこの記号列は
\[
  (V|x)({\rm Pair}((T, U)))
\]
と一致するから, これが定理となる.
そこで$V$の定義から, 推論法則 \ref{dedallfund}によって
\[
  \forall x({\rm Pair}((T, U)))
\]
が成り立つ.
このことと, $x$が$a$の中に自由変数として現れないことから, 
定理 \ref{sthmobjectsetgraph}によって$\{(T, U)|x \in a\}$がグラフとなることがわかる.
\halmos




\mathstrut
\begin{thm}
\label{sthmcupgraph}%定理
$a$と$b$を集合とするとき, 
\[
  {\rm Graph}(a) \wedge {\rm Graph}(b) \leftrightarrow {\rm Graph}(a \cup b)
\]
が成り立つ.
またこのことから, 次の($*$)が成り立つ: 

($*$) ~~$a$と$b$が共にグラフならば, $a \cup b$はグラフである.
        逆に$a \cup b$がグラフならば, $a$と$b$は共にグラフである.
\end{thm}


\noindent{\bf 証明}
~まず前半を示す.
推論法則 \ref{dedequiv}があるから, 
\begin{align*}
  \tag{1}
  &{\rm Graph}(a) \wedge {\rm Graph}(b) \to {\rm Graph}(a \cup b), \\
  \mbox{} \\
  \tag{2}
  &{\rm Graph}(a \cup b) \to {\rm Graph}(a) \wedge {\rm Graph}(b)
\end{align*}
が共に成り立つことを示せば良い.

(1)の証明: 
$x$を$a$及び$b$の中に自由変数として現れない文字とする.
このとき変数法則 \ref{valcup}により, $x$は$a \cup b$の中にも自由変数として現れない.
また$\tau_{x}(\neg (x \in a \cup b \to {\rm Pair}(x)))$を$T$と書けば, 
$T$は集合であり, 定理 \ref{sthmgraphbasis}より
\[
  {\rm Graph}(a) \to (T \in a \to {\rm Pair}(T)), ~~
  {\rm Graph}(b) \to (T \in b \to {\rm Pair}(T))
\]
が共に成り立つ.
そこで推論法則 \ref{dedfromaddw}により, 
\[
\tag{3}
  {\rm Graph}(a) \wedge {\rm Graph}(b) \to (T \in a \to {\rm Pair}(T)) \wedge (T \in b \to {\rm Pair}(T))
\]
が成り立つ.
またThm \ref{1atc1w1btc1t1avbtc1}より
\[
\tag{4}
  (T \in a \to {\rm Pair}(T)) \wedge (T \in b \to {\rm Pair}(T)) \to 
  (T \in a \vee T \in b \to {\rm Pair}(T))
\]
が成り立つ.
また定理 \ref{sthmcupbasis}と推論法則 \ref{dedequiv}により
\[
  T \in a \cup b \to T \in a \vee T \in b
\]
が成り立つから, 推論法則 \ref{dedaddf}により
\[
\tag{5}
  (T \in a \vee T \in b \to {\rm Pair}(T)) \to (T \in a \cup b \to {\rm Pair}(T))
\]
が成り立つ.
また$T$の定義から, Thm \ref{thmallfund1}と推論法則 \ref{dedequiv}により
\[
  (T|x)(x \in a \cup b \to {\rm Pair}(x)) \to \forall x(x \in a \cup b \to {\rm Pair}(x))
\]
が成り立つが, 上述のように$x$は$a \cup b$の中に自由変数として現れないから, 
代入法則 \ref{substfree}, \ref{substfund}, \ref{substbigpair}及び定義より, 
この記号列は
\[
\tag{6}
  (T \in a \cup b \to {\rm Pair}(T)) \to {\rm Graph}(a \cup b)
\]
と一致する.
よってこれが定理となる.
そこで(3)---(6)から, 推論法則 \ref{dedmmp}によって
(1)が成り立つことがわかる.

(2)の証明: 
定理 \ref{sthmsubsetcup}より
\[
  a \subset a \cup b, ~~
  b \subset a \cup b
\]
が共に成り立ち, 定理 \ref{sthmgraphsubset}より
\[
  a \subset a \cup b \to ({\rm Graph}(a \cup b) \to {\rm Graph}(a)), ~~
  b \subset a \cup b \to ({\rm Graph}(a \cup b) \to {\rm Graph}(b))
\]
が共に成り立つから, これらから, 推論法則 \ref{dedmp}によって
\[
  {\rm Graph}(a \cup b) \to {\rm Graph}(a), ~~
  {\rm Graph}(a \cup b) \to {\rm Graph}(b)
\]
が共に成り立つ.
そこでこれらから, 推論法則 \ref{dedprewedge}によって(2)が成り立つ.

さていま$a$と$b$が共にグラフであるとする.
このとき推論法則 \ref{dedwedge}により${\rm Graph}(a) \wedge {\rm Graph}(b)$が成り立つ.
また上に示したように(1)が成り立つ.
そこでこれらから, 推論法則 \ref{dedmp}によって${\rm Graph}(a \cup b)$が成り立つ.
逆に$a \cup b$がグラフであるとき, これと(2)が成り立つことから, 
推論法則 \ref{dedmp}によって${\rm Graph}(a) \wedge {\rm Graph}(b)$が成り立つ.
そこで推論法則 \ref{dedwedge}により, ${\rm Graph}(a)$と${\rm Graph}(b)$が共に成り立つ.
これで($*$)が成り立つことが示された.
\halmos




\mathstrut
\begin{thm}
\label{sthmcapgraph}%定理
$a$と$b$を集合とするとき, 
\[
  {\rm Graph}(a) \vee {\rm Graph}(b) \to {\rm Graph}(a \cap b)
\]
が成り立つ.
またこのことから, 次の($*$)が成り立つ: 

($*$) ~~$a$がグラフならば, $a \cap b$はグラフである.
        また$b$がグラフならば, $a \cap b$はグラフである.
\end{thm}


\noindent{\bf 証明}
~定理 \ref{sthmcap}より
\[
  a \cap b \subset a, ~~
  a \cap b \subset b
\]
が共に成り立ち, 定理 \ref{sthmgraphsubset}より
\[
  a \cap b \subset a \to ({\rm Graph}(a) \to {\rm Graph}(a \cap b)), ~~
  a \cap b \subset b \to ({\rm Graph}(b) \to {\rm Graph}(a \cap b))
\]
が共に成り立つから, これらから, 推論法則 \ref{dedmp}によって
\[
  {\rm Graph}(a) \to {\rm Graph}(a \cap b), ~~
  {\rm Graph}(b) \to {\rm Graph}(a \cap b)
\]
が共に成り立つ.
そこでこれらから, 推論法則 \ref{deddil}によって
\[
\tag{1}
  {\rm Graph}(a) \vee {\rm Graph}(b) \to {\rm Graph}(a \cap b)
\]
が成り立つ.

さていま$a$がグラフであるとすれば, 推論法則 \ref{dedvee}により
${\rm Graph}(a) \vee {\rm Graph}(b)$が成り立つから, これと(1)から, 
推論法則 \ref{dedmp}によって${\rm Graph}(a \cap b)$が成り立つ.
また$b$がグラフであるとすれば, 同様に推論法則 \ref{dedvee}により
${\rm Graph}(a) \vee {\rm Graph}(b)$が成り立つから, これと(1)から, 
推論法則 \ref{dedmp}によって${\rm Graph}(a \cap b)$が成り立つ.
これで($*$)が成り立つことが示された.
\halmos




\mathstrut
\begin{thm}
\label{sthm-graph}%定理
$a$と$b$を集合とするとき, 
\[
  {\rm Graph}(a) \to {\rm Graph}(a - b)
\]
が成り立つ.
またこのことから, 次の($*$)が成り立つ: 

($*$) ~~$a$がグラフならば, $a - b$はグラフである.
\end{thm}


\noindent{\bf 証明}
~定理 \ref{sthma-bsubseta}より
\[
  a - b \subset a
\]
が成り立ち, 定理 \ref{sthmgraphsubset}より
\[
  a - b \subset a \to ({\rm Graph}(a) \to {\rm Graph}(a - b))
\]
が成り立つから, これらから, 推論法則 \ref{dedmp}によって
${\rm Graph}(a) \to {\rm Graph}(a - b)$が成り立つ.
($*$)が成り立つことはこれと推論法則 \ref{dedmp}によって明らか.
\halmos




\mathstrut
\begin{thm}
\label{sthmemptygraph}%定理
$\phi$はグラフである.
\end{thm}


\noindent{\bf 証明}
~$x$を文字とするとき, 定理 \ref{sthmemptytspin}より
\[
  \forall x(x \in \phi \to {\rm Pair}(x))
\]
が成り立つが, 変数法則 \ref{valempty}により$x$は
$\phi$の中に自由変数として現れないから, 定義よりこの記号列は
${\rm Graph}(\phi)$と同じである.
よってこれが定理となる.
\halmos




\mathstrut
\begin{thm}
\label{sthmproductsubsetgraph}%定理
$a$, $b$, $c$を集合とするとき, 
\[
  c \subset a \times b \to {\rm Graph}(c)
\]
が成り立つ.
またこのことから, 次の($*$)が成り立つ: 

($*$) ~~$c \subset a \times b$が成り立つならば, $c$はグラフである.
\end{thm}


\noindent{\bf 証明}
~$x$を$c$の中に自由変数として現れない文字とする.
また$\tau_{x}(\neg (x \in c \to {\rm Pair}(x)))$を$T$と書く.
$T$は集合であり, 定理 \ref{sthmsubsetbasis}より
\[
\tag{1}
  c \subset a \times b \to (T \in c \to T \in a \times b)
\]
が成り立つ.
また定理 \ref{sthmproductelement}と推論法則 \ref{dedequiv}により, 
\[
\tag{2}
  T \in a \times b \to {\rm Pair}(T) \wedge ({\rm pr}_{1}(T) \in a \wedge {\rm pr}_{2}(T) \in b)
\]
が成り立つ.
またThm \ref{awbta}より
\[
\tag{3}
  {\rm Pair}(T) \wedge ({\rm pr}_{1}(T) \in a \wedge {\rm pr}_{2}(T) \in b) \to {\rm Pair}(T)
\]
が成り立つ.
そこで(2), (3)から, 推論法則 \ref{dedmmp}によって
\[
  T \in a \times b \to {\rm Pair}(T)
\]
が成り立ち, これから推論法則 \ref{dedaddb}によって
\[
  (T \in c \to T \in a \times b) \to (T \in c \to {\rm Pair}(T))
\]
が成り立つ.
ここで$x$が$c$の中に自由変数として現れないことから, 
代入法則 \ref{substfree}, \ref{substfund}, \ref{substbigpair}により, 
上記の記号列は
\[
\tag{4}
  (T \in c \to T \in a \times b) \to (T|x)(x \in c \to {\rm Pair}(x))
\]
と一致する.
よってこれが定理となる.
また$T$の定義から, Thm \ref{thmallfund1}と推論法則 \ref{dedequiv}により
\[
  (T|x)(x \in c \to {\rm Pair}(x)) \to \forall x(x \in c \to {\rm Pair}(x))
\]
が成り立つが, いま$x$が$c$の中に自由変数として現れないので, 定義からこの記号列は
\[
\tag{5}
  (T|x)(x \in c \to {\rm Pair}(x)) \to {\rm Graph}(c)
\]
と同じである.
よってこれが定理となる.
そこで(1), (4), (5)から, 推論法則 \ref{dedmmp}によって
$c \subset a \times b \to {\rm Graph}(c)$が成り立つことがわかる.
($*$)が成り立つことは, これと推論法則 \ref{dedmp}によって明らかである.
\halmos




\mathstrut
$a$, $b$, $c$を集合とする.
$c \subset a \times b$が成り立つとき, 上記の定理 \ref{sthmproductsubsetgraph}によれば, 
$c$はグラフである.
このとき$c$を\textbf{${\bm a}$から${\bm b}$への対応}ともいう.

上記の定理 \ref{sthmproductsubsetgraph}から直ちに次の定理が得られる.




\mathstrut
\begin{thm}
\label{sthmproductgraph}%定理
$a$と$b$を集合とするとき, $a \times b$はグラフである.
\end{thm}


\noindent{\bf 証明}
~定理 \ref{sthmsubsetself}より$a \times b \subset a \times b$が成り立つから, 
定理 \ref{sthmproductsubsetgraph}より$a \times b$はグラフである.
\halmos




\mathstrut
\begin{defo}
\label{projectionset}%変形
$a$を記号列とする.
また$x$と$y$を, 互いに異なり, 共に$a$の中に自由変数として現れない文字とする.
同様に, $z$と$w$を, 互いに異なり, 共に$a$の中に自由変数として現れない文字とする.
このとき
\[
  \{x|\exists y((x, y) \in a)\} \equiv \{z|\exists w((z, w) \in a)\}, ~~
  \{y|\exists x((x, y) \in a)\} \equiv \{w|\exists z((z, w) \in a)\}
\]
が成り立つ.
\end{defo}


\noindent{\bf 証明}
~$u$と$v$を, 互いに異なり, 共に$x$, $y$, $z$, $w$のいずれとも異なり, 
$a$の中に自由変数として現れない文字とする.
このとき変数法則 \ref{valfund}, \ref{valquan}, \ref{valpair}からわかるように, 
$u$は$\exists y((x, y) \in a)$の中に自由変数として現れず, 
$v$は$\exists x((x, y) \in a)$の中に自由変数として現れない.
そこで代入法則 \ref{substisettrans}により, 
\begin{align*}
  \tag{1}
  \{x|\exists y((x, y) \in a)\} &\equiv \{u|(u|x)(\exists y((x, y) \in a))\}, \\
  \mbox{}& \\
  \tag{2}
  \{y|\exists x((x, y) \in a)\} &\equiv \{v|(v|y)(\exists x((x, y) \in a))\}
\end{align*}
が成り立つ.
また$y$が$x$とも$u$とも異なること, $x$が$y$とも$v$とも異なることから, 
それぞれ代入法則 \ref{substquan}により, 
\begin{align*}
  \tag{3}
  (u|x)(\exists y((x, y) \in a)) &\equiv \exists y((u|x)((x, y) \in a)), \\
  \mbox{}& \\
  \tag{4}
  (v|y)(\exists x((x, y) \in a)) &\equiv \exists x((v|y)((x, y) \in a))
\end{align*}
が成り立つ.
また$x$と$y$が互いに異なり, 共に$a$の中に自由変数として現れないことから, 
代入法則 \ref{substfree}, \ref{substfund}, \ref{substpair}により, 
\begin{align*}
  \tag{5}
  (u|x)((x, y) \in a) &\equiv (u, y) \in a, \\
  \mbox{}& \\
  \tag{6}
  (v|y)((x, y) \in a) &\equiv (x, v) \in a
\end{align*}
が成り立つ.
そこで(1), (3), (5)から
\[
\tag{7}
  \{x|\exists y((x, y) \in a)\} \equiv \{u|\exists y((u, y) \in a)\}
\]
が成り立ち, (2), (4), (6)から
\[
\tag{8}
  \{y|\exists x((x, y) \in a)\} \equiv \{v|\exists x((x, v) \in a)\}
\]
が成り立つことがわかる.
また, $u$と$v$が互いに異なり, 共に$x$とも$y$とも異なり, 
$a$の中に自由変数として現れないことから, 
変数法則 \ref{valfund}, \ref{valpair}により
$v$は$(u, y) \in a$の中に自由変数として現れず, 
$u$は$(x, v) \in a$の中に自由変数として現れないことがわかるから, 
代入法則 \ref{substquantrans}により, 
\begin{align*}
  \tag{9}
  \exists y((u, y) \in a) &\equiv \exists v((v|y)((u, y) \in a)), \\
  \mbox{}& \\
  \tag{10}
  \exists x((x, v) \in a) &\equiv \exists u((u|x)((x, v) \in a))
\end{align*}
が成り立つ.
また$x$と$y$が共に$u$とも$v$とも異なり, 
$a$の中に自由変数として現れないことから, 
代入法則 \ref{substfree}, \ref{substfund}, \ref{substpair}により, 
\begin{align*}
  \tag{11}
  (v|y)((u, y) \in a) &\equiv (u, v) \in a, \\
  \mbox{}& \\
  \tag{12}
  (u|x)((x, v) \in a) &\equiv (u, v) \in a
\end{align*}
が成り立つ.
そこで(7), (9), (11)から
\[
  \{x|\exists y((x, y) \in a)\} \equiv \{u|\exists v((u, v) \in a)\}
\]
が成り立ち, (8), (10), (12)から
\[
  \{y|\exists x((x, y) \in a)\} \equiv \{v|\exists u((u, v) \in a)\}
\]
が成り立つことがわかる.
ここまでの議論と全く同様にして, 
\begin{align*}
  \{z|\exists w((z, w) \in a)\} &\equiv \{u|\exists v((u, v) \in a)\}, \\
  \mbox{}& \\
  \{w|\exists z((z, w) \in a)\} &\equiv \{v|\exists u((u, v) \in a)\}
\end{align*}
も成り立つから, 
従って$\{x|\exists y((x, y) \in a)\}$と$\{z|\exists w((z, w) \in a)\}$, 
$\{y|\exists x((x, y) \in a)\}$と$\{w|\exists z((z, w) \in a)\}$は
それぞれ一致する.
\halmos




\mathstrut
\begin{defi}
\label{defprset}%定義
$a$を記号列とする.
また$x$と$y$を, 互いに異なり, 共に$a$の中に自由変数として現れない文字とする.
同様に, $z$と$w$を, 互いに異なり, 共に$a$の中に自由変数として現れない文字とする.
このとき上記の変形法則 \ref{projectionset}によれば, 
$\{x|\exists y((x, y) \in a)\}$と$\{z|\exists w((z, w) \in a)\}$という二つの記号列は一致し, 
$\{y|\exists x((x, y) \in a)\}$と$\{w|\exists z((z, w) \in a)\}$という二つの記号列も一致する.
$a$に対して定まるこれらの記号列のうち, 前者を${\rm pr}_{1}\langle a \rangle$, 後者を
${\rm pr}_{2}\langle a \rangle$と書き表す.
\end{defi}




\mathstrut
\begin{valu}
\label{valprset}%変数
$a$を記号列とし, $x$を文字とする.
$x$が$a$の中に自由変数として現れなければ, 
$x$は${\rm pr}_{1}\langle a \rangle$及び
${\rm pr}_{2}\langle a \rangle$の中に自由変数として現れない.
\end{valu}


\noindent{\bf 証明}
~$y$を$x$と異なり, $a$の中に自由変数として現れない文字とすれば, 
定義から${\rm pr}_{1}\langle a \rangle$は$\{x|\exists y((x, y) \in a)\}$, 
${\rm pr}_{2}\langle a \rangle$は$\{x|\exists y((y, x) \in a)\}$と同じである.
変数法則 \ref{valiset}により, $x$はこれらの記号列の中に自由変数として現れない.
\halmos




\mathstrut
\begin{subs}
\label{substprset}%代入
$a$と$b$を記号列とし, $x$を文字とするとき, 
\[
  (b|x)({\rm pr}_{1}\langle a \rangle) \equiv {\rm pr}_{1}\langle (b|x)(a) \rangle, ~~
  (b|x)({\rm pr}_{2}\langle a \rangle) \equiv {\rm pr}_{2}\langle (b|x)(a) \rangle
\]
が成り立つ.
\end{subs}


\noindent{\bf 証明}
~$y$と$z$を, 互いに異なり, 共に$x$と異なり, 
$a$及び$b$の中に自由変数として現れない文字とする.
このとき定義から, ${\rm pr}_{1}\langle a \rangle$は
$\{y|\exists z((y, z) \in a)\}$と同じであり, 
${\rm pr}_{2}\langle a \rangle$は
$\{z|\exists y((y, z) \in a)\}$と同じであるから, 
代入法則 \ref{substiset}により, 
\begin{align*}
  \tag{1}
  (b|x)({\rm pr}_{1}\langle a \rangle) &\equiv \{y|(b|x)(\exists z((y, z) \in a))\}, \\
  \mbox{}& \\
  \tag{2}
  (b|x)({\rm pr}_{2}\langle a \rangle) &\equiv \{z|(b|x)(\exists y((y, z) \in a))\}
\end{align*}
が成り立つ.
また$y$と$z$が共に$x$と異なり, $b$の中に自由変数として現れないことから, 
代入法則 \ref{substquan}により, 
\begin{align*}
  \tag{3}
  (b|x)(\exists z((y, z) \in a)) &\equiv \exists z((b|x)((y, z) \in a)), \\
  \mbox{}& \\
  \tag{4}
  (b|x)(\exists y((y, z) \in a)) &\equiv \exists y((b|x)((y, z) \in a))
\end{align*}
が成り立つ.
また$x$が$y$とも$z$とも異なることから, 変数法則 \ref{valpair}により
$x$は$(y, z)$の中に自由変数として現れないから, 
代入法則 \ref{substfree}, \ref{substfund}により, 
\[
\tag{5}
  (b|x)((y, z) \in a) \equiv (y, z) \in (b|x)(a)
\]
が成り立つ.
そこで(1), (3), (5)から
\[
\tag{6}
  (b|x)({\rm pr}_{1}\langle a \rangle) \equiv \{y|\exists z((y, z) \in (b|x)(a))\}
\]
が成り立ち, (2), (4), (5)から
\[
\tag{7}
  (b|x)({\rm pr}_{2}\langle a \rangle) \equiv \{z|\exists y((y, z) \in (b|x)(a))\}
\]
が成り立つことがわかる.
いま$y$と$z$は共に$a$及び$b$の中に自由変数として現れないから, 
変数法則 \ref{valsubst}により, これらは共に$(b|x)(a)$の中にも自由変数として現れない.
また$y$と$z$は異なる文字である.
そこで定義から, $\{y|\exists z((y, z) \in (b|x)(a))\}$は
${\rm pr}_{1}\langle (b|x)(a) \rangle$と同じであり, 
$\{z|\exists y((y, z) \in (b|x)(a))\}$は
${\rm pr}_{2}\langle (b|x)(a) \rangle$と同じである.
このことと(6), (7)から本法則が成り立つことがわかる.
\halmos




\mathstrut
\begin{form}
\label{formprset}%構成
$a$が集合ならば, ${\rm pr}_{1}\langle a \rangle$と${\rm pr}_{2}\langle a \rangle$は
集合である.
\end{form}


\noindent{\bf 証明}
~$x$と$y$を, 互いに異なり, 共に$a$の中に自由変数として現れない文字とすれば, 
定義から${\rm pr}_{1}\langle a \rangle$は$\{x|\exists y((x, y) \in a)\}$と同じであり, 
${\rm pr}_{2}\langle a \rangle$は$\{y|\exists x((x, y) \in a)\}$と同じである.
$a$が集合であるとき, これらが集合となることは
構成法則 \ref{formfund}, \ref{formquan}, \ref{formiset}, \ref{formpair}によって
直ちにわかる.
\halmos




\mathstrut
\begin{thm}
\label{sthmprsetmake}%定理
$a$を集合とする.
また$x$と$y$を, 互いに異なり, 共に$a$の中に自由変数として現れない文字とする.
このとき, 関係式$\exists y((x, y) \in a)$, $\exists x((x, y) \in a)$は, 
それぞれ$x$, $y$について集合を作り得る.
\end{thm}


\noindent{\bf 証明}
~$u$と$v$を, 共に$x$とも$y$とも異なり, $a$の中に自由変数として現れない, 定数でない文字とする.
また$\tau_{x}((x, v) \in a)$を$T$と書き, $\tau_{y}((u, y) \in a)$を$U$と書く.
このとき$T$と$U$は共に集合であり, 定義から
\[
  \exists y((u, y) \in a) \equiv (U|y)((u, y) \in a), ~~
  \exists x((x, v) \in a) \equiv (T|x)((x, v) \in a)
\]
である.
また$x$, $y$がそれぞれ$v$, $u$と異なり, 共に$a$の中に自由変数として現れないことから, 
代入法則 \ref{substfree}, \ref{substfund}, \ref{substpair}により, 
\[
  (U|y)((u, y) \in a) \equiv (u, U) \in a, ~~
  (T|x)((x, v) \in a) \equiv (T, v) \in a
\]
が成り立つ.
これらのことから, $\exists y((u, y) \in a)$が$(u, U) \in a$と一致し, 
$\exists x((x, v) \in a)$が$(T, v) \in a$と一致することがわかる.
いま定理 \ref{sthmprpair}と推論法則 \ref{ded=ch}により
\[
  u = {\rm pr}_{1}((u, U)), ~~
  v = {\rm pr}_{2}((T, v))
\]
が共に成り立つから, 従って推論法則 \ref{dedatawbtrue2}により, 
\begin{align*}
  \tag{1}
  \exists y((u, y) \in a) &\to (u, U) \in a \wedge u = {\rm pr}_{1}((u, U)), \\
  \mbox{}& \\
  \tag{2}
  \exists x((x, v) \in a) &\to (T, v) \in a \wedge v = {\rm pr}_{2}((T, v))
\end{align*}
が共に成り立つ.
ここで$z$を$x$, $y$, $u$, $v$のいずれとも異なり, 
$a$の中に自由変数として現れない文字とすれば, 
代入法則 \ref{substfree}, \ref{substfund}, \ref{substwedge}, \ref{substpr}により, 
\begin{align*}
  (u, U) \in a \wedge u = {\rm pr}_{1}((u, U)) &\equiv ((u, U)|z)(z \in a \wedge u = {\rm pr}_{1}(z)), \\
  \mbox{}& \\
  (T, v) \in a \wedge v = {\rm pr}_{2}((T, v)) &\equiv ((T, v)|z)(z \in a \wedge v = {\rm pr}_{2}(z))
\end{align*}
が成り立つから, (1), (2)はそれぞれ
\begin{align*}
  \tag{3}
  \exists y((u, y) \in a) &\to ((u, U)|z)(z \in a \wedge u = {\rm pr}_{1}(z)), \\
  \mbox{}& \\
  \tag{4}
  \exists x((x, v) \in a) &\to ((T, v)|z)(z \in a \wedge v = {\rm pr}_{2}(z))
\end{align*}
と一致する.
従ってこれらが共に定理となる.
またschema S4の適用により, 
\begin{align*}
  \tag{5}
  ((u, U)|z)(z \in a \wedge u = {\rm pr}_{1}(z)) &\to \exists z(z \in a \wedge u = {\rm pr}_{1}(z)), \\
  \mbox{}& \\
  \tag{6}
  ((T, v)|z)(z \in a \wedge v = {\rm pr}_{2}(z)) &\to \exists z(z \in a \wedge v = {\rm pr}_{2}(z))
\end{align*}
が共に成り立つ.
そこで(3)と(5), (4)と(6)から, それぞれ推論法則 \ref{dedmmp}によって
\begin{align*}
  \exists y((u, y) \in a) &\to \exists z(z \in a \wedge u = {\rm pr}_{1}(z)), \\
  \mbox{}& \\
  \exists x((x, v) \in a) &\to \exists z(z \in a \wedge v = {\rm pr}_{2}(z))
\end{align*}
が共に成り立つ.
また$u$と$v$は共に定数でないから, これらから, 推論法則 \ref{dedltthmquan}によって
\begin{align*}
  \tag{7}
  \forall u(\exists y((u, y) \in a) &\to \exists z(z \in a \wedge u = {\rm pr}_{1}(z))), \\
  \mbox{}& \\
  \tag{8}
  \forall v(\exists x((x, v) \in a) &\to \exists z(z \in a \wedge v = {\rm pr}_{2}(z)))
\end{align*}
が共に成り立つ.
いま$x$と$y$が互いに異なり, 共に$u$, $v$, $z$のいずれとも異なり, 
$a$の中に自由変数として現れないことから, 
変数法則 \ref{valfund}, \ref{valwedge}, \ref{valquan}, \ref{valpair}, \ref{valpr}によって
$x$, $y$がそれぞれ$\exists y((u, y) \in a) \to \exists z(z \in a \wedge u = {\rm pr}_{1}(z))$, 
$\exists x((x, v) \in a) \to \exists z(z \in a \wedge v = {\rm pr}_{2}(z))$の中に
自由変数として現れないことがわかる.
そこで代入法則 \ref{substquantrans}により, 
\begin{multline*}
\tag{9}
  \forall u(\exists y((u, y) \in a) \to \exists z(z \in a \wedge u = {\rm pr}_{1}(z))) \\
  \equiv \forall x((x|u)(\exists y((u, y) \in a) \to \exists z(z \in a \wedge u = {\rm pr}_{1}(z)))), 
\end{multline*}
\begin{multline*}
\tag{10}
  \forall v(\exists x((x, v) \in a) \to \exists z(z \in a \wedge v = {\rm pr}_{2}(z))) \\
  \equiv \forall y((y|v)(\exists x((x, v) \in a) \to \exists z(z \in a \wedge v = {\rm pr}_{2}(z))))
\end{multline*}
が成り立つ.
また, $y$と$z$が共に$x$とも$u$とも異なること, $x$と$z$が共に$y$とも$v$とも異なることから, 
代入法則 \ref{substfund}, \ref{substquan}により, 
\begin{multline*}
\tag{11}
  (x|u)(\exists y((u, y) \in a) \to \exists z(z \in a \wedge u = {\rm pr}_{1}(z))) \\
  \equiv \exists y((x|u)((u, y) \in a)) \to \exists z((x|u)(z \in a \wedge u = {\rm pr}_{1}(z))), 
\end{multline*}
\begin{multline*}
\tag{12}
  (y|v)(\exists x((x, v) \in a) \to \exists z(z \in a \wedge v = {\rm pr}_{2}(z))) \\
  \equiv \exists x((y|v)((x, v) \in a)) \to \exists z((y|v)(z \in a \wedge v = {\rm pr}_{2}(z)))
\end{multline*}
が成り立つ.
また$u$と$v$が共に$x$, $y$, $z$のいずれとも異なり, $a$の中に自由変数として現れないことから, 
代入法則 \ref{substfree}, \ref{substfund}, \ref{substwedge}, \ref{substpair}, \ref{substpr}により, 
\begin{align*}
  \tag{13}
  (x|u)((u, y) \in a) &\equiv (x, y) \in a, \\
  \mbox{}& \\
  \tag{14}
  (x|u)(z \in a \wedge u = {\rm pr}_{1}(z)) &\equiv z \in a \wedge x = {\rm pr}_{1}(z), \\
  \mbox{}& \\
  \tag{15}
  (y|v)((x, v) \in a) &\equiv (x, y) \in a, \\
  \mbox{}& \\
  \tag{16}
  (y|v)(z \in a \wedge v = {\rm pr}_{2}(z)) &\equiv z \in a \wedge y = {\rm pr}_{2}(z)
\end{align*}
が成り立つ.
そこで(9), (11), (13), (14)から, (7)が
\[
\tag{17}
  \forall x(\exists y((x, y) \in a) \to \exists z(z \in a \wedge x = {\rm pr}_{1}(z)))
\]
と一致することがわかり, これが定理となる.
また(10), (12), (15), (16)から, (8)が
\[
\tag{18}
  \forall y(\exists x((x, y) \in a) \to \exists z(z \in a \wedge y = {\rm pr}_{2}(z)))
\]
と一致することがわかり, これも定理となる.
さていま$z$は$x$とも$y$とも異なり, $a$の中に自由変数として現れない.
またこのことから, 変数法則 \ref{valpr}により, 
$x$は${\rm pr}_{1}(z)$の中に自由変数として現れず, 
$y$は${\rm pr}_{2}(z)$の中に自由変数として現れない.
また$x$と$y$は共に$a$の中に自由変数として現れない.
そこで定理 \ref{sthmosetsm}より, 
$\exists z(z \in a \wedge x = {\rm pr}_{1}(z))$と
$\exists z(z \in a \wedge y = {\rm pr}_{2}(z))$は, 
それぞれ$x$, $y$について集合を作り得る.
このことと(17), (18)が成り立つことから, 
定理 \ref{sthmalltsm}により, 
$\exists y((x, y) \in a)$と$\exists x((x, y) \in a)$が
それぞれ$x$, $y$について集合を作り得ることがわかる.
\halmos




\mathstrut
\begin{thm}
\label{sthmprsetelement}%定理
$a$と$b$を集合とする.
また$x$と$y$を, 共に$a$及び$b$の中に自由変数として現れない文字とする.
このとき
\[
  b \in {\rm pr}_{1}\langle a \rangle \leftrightarrow \exists y((b, y) \in a), ~~
  b \in {\rm pr}_{2}\langle a \rangle \leftrightarrow \exists x((x, b) \in a)
\]
が成り立つ.
\end{thm}


\noindent{\bf 証明}
~$u$, $v$を, それぞれ$y$, $x$と異なり, 共に$a$及び$b$の中に自由変数として現れない文字とする.
このとき定義から, ${\rm pr}_{1}\langle a \rangle$は$\{u|\exists y((u, y) \in a)\}$と同じであり, 
${\rm pr}_{2}\langle a \rangle$は$\{v|\exists x((x, v) \in a)\}$と同じである.
またこれらの文字に対する仮定から, 
定理 \ref{sthmprsetmake}より$\exists y((u, y) \in a)$, $\exists x((x, v) \in a)$は
それぞれ$u$, $v$について集合を作り得るから, 従って定理 \ref{sthmisetbasis}より, 
\begin{align*}
  \tag{1}
  b \in {\rm pr}_{1}\langle a \rangle &\leftrightarrow (b|u)(\exists y((u, y) \in a)), \\
  \mbox{}& \\
  \tag{2}
  b \in {\rm pr}_{2}\langle a \rangle &\leftrightarrow (b|v)(\exists x((x, v) \in a))
\end{align*}
が共に成り立つ.
いま$y$, $x$がそれぞれ$u$, $v$と異なり, 共に$b$の中に自由変数として現れないことから, 
代入法則 \ref{substquan}により, 
\begin{align*}
  \tag{3}
  (b|u)(\exists y((u, y) \in a)) &\equiv \exists y((b|u)((u, y) \in a)), \\
  \mbox{}& \\
  \tag{4}
  (b|v)(\exists x((x, v) \in a)) &\equiv \exists x((b|v)((x, v) \in a))
\end{align*}
が成り立つ.
また$u$, $v$がそれぞれ$y$, $x$と異なり, 共に$a$の中に自由変数として現れないことから, 
代入法則 \ref{substfree}, \ref{substfund}, \ref{substpair}により, 
\begin{align*}
  \tag{5}
  (b|u)((u, y) \in a) &\equiv (b, y) \in a, \\
  \mbox{}& \\
  \tag{6}
  (b|v)((x, v) \in a) &\equiv (x, b) \in a
\end{align*}
が成り立つ.
そこで(3)と(5)から, (1)が
\[
  b \in {\rm pr}_{1}\langle a \rangle \leftrightarrow \exists y((b, y) \in a)
\]
と一致することがわかり, これが定理となる.
また(4)と(6)から, (2)が
\[
  b \in {\rm pr}_{2}\langle a \rangle \leftrightarrow \exists x((x, b) \in a)
\]
と一致することがわかり, これも定理となる.
\halmos




\mathstrut
\begin{thm}
\label{sthmprsetsubset}%定理
$a$と$b$を集合とするとき, 
\[
  a \subset b \to {\rm pr}_{1}\langle a \rangle \subset {\rm pr}_{1}\langle b \rangle, ~~
  a \subset b \to {\rm pr}_{2}\langle a \rangle \subset {\rm pr}_{2}\langle b \rangle
\]
が成り立つ.
またこのことから, 次の($*$)が成り立つ: 

($*$) ~~$a \subset b$が成り立つならば, 
        ${\rm pr}_{1}\langle a \rangle \subset {\rm pr}_{1}\langle b \rangle$と
        ${\rm pr}_{2}\langle a \rangle \subset {\rm pr}_{2}\langle b \rangle$が
        共に成り立つ.
\end{thm}


\noindent{\bf 証明}
~$x$と$y$を, 互いに異なり, 共に$a$及び$b$の中に自由変数として現れない文字とする.
このとき変数法則 \ref{valprset}により, $x$と$y$は共に
${\rm pr}_{1}\langle a \rangle$, ${\rm pr}_{1}\langle b \rangle$, ${\rm pr}_{2}\langle a \rangle$, 
${\rm pr}_{2}\langle b \rangle$のいずれの記号列の中にも自由変数として現れない.
また
\[
  T \equiv \tau_{x}(\neg (x \in {\rm pr}_{1}\langle a \rangle \to x \in {\rm pr}_{1}\langle b \rangle)), ~~
  U \equiv \tau_{y}(\neg (y \in {\rm pr}_{2}\langle a \rangle \to y \in {\rm pr}_{2}\langle b \rangle))
\]
と置けば, これらは共に集合であり, 変数法則 \ref{valtau}により
$x$, $y$はそれぞれ$T$, $U$の中に自由変数として現れない.
また$x$と$y$が互いに異なり, 上述のようにこれらが共に
${\rm pr}_{1}\langle a \rangle$, ${\rm pr}_{1}\langle b \rangle$, ${\rm pr}_{2}\langle a \rangle$, 
${\rm pr}_{2}\langle b \rangle$のいずれの記号列の中にも自由変数として現れないことから, 
変数法則 \ref{valfund}, \ref{valtau}によって$x$, $y$が
それぞれ$U$, $T$の中にも自由変数として現れないことがわかる.
そこで定理 \ref{sthmprsetelement}と推論法則 \ref{dedequiv}により, 
\begin{align*}
  \tag{1}
  T \in {\rm pr}_{1}\langle a \rangle &\to \exists y((T, y) \in a), \\
  \mbox{}& \\
  \tag{2}
  U \in {\rm pr}_{2}\langle a \rangle &\to \exists x((x, U) \in a)
\end{align*}
が共に成り立つ.
ここで$\tau_{y}((T, y) \in a)$を$V$と書き, $\tau_{x}((x, U) \in a)$を$W$と書けば, 
これらは集合であり, 定義から(1), (2)はそれぞれ, 
\begin{align*}
  \tag{3}
  T \in {\rm pr}_{1}\langle a \rangle &\to (V|y)((T, y) \in a), \\
  \mbox{}& \\
  \tag{4}
  U \in {\rm pr}_{2}\langle a \rangle &\to (W|x)((x, U) \in a)
\end{align*}
と一致する.
また$x$と$y$は共に$a$の中に自由変数として現れず, 上述のように
$T$及び$U$の中にも自由変数として現れないから, 
代入法則 \ref{substfree}, \ref{substfund}, \ref{substpair}により, 
(3), (4)はそれぞれ
\begin{align*}
  \tag{5}
  T \in {\rm pr}_{1}\langle a \rangle &\to (T, V) \in a, \\
  \mbox{}& \\
  \tag{6}
  U \in {\rm pr}_{2}\langle a \rangle &\to (W, U) \in a
\end{align*}
と一致する.
そこで(1)と(5), (2)と(6)がそれぞれ一致し, (5)と(6)が共に定理となる.
そこで推論法則 \ref{dedaddw}により, 
\begin{align*}
  \tag{7}
  a \subset b \wedge T \in {\rm pr}_{1}\langle a \rangle &\to a \subset b \wedge (T, V) \in a, \\
  \mbox{}& \\
  \tag{8}
  a \subset b \wedge U \in {\rm pr}_{2}\langle a \rangle &\to a \subset b \wedge (W, U) \in a
\end{align*}
が共に成り立つ.
また定理 \ref{sthmsubsetbasis}より, 
\begin{align*}
  a \subset b &\to ((T, V) \in a \to (T, V) \in b), \\
  \mbox{}& \\
  a \subset b &\to ((W, U) \in a \to (W, U) \in b)
\end{align*}
が共に成り立つから, これらから, 推論法則 \ref{dedtwch}によって
\begin{align*}
  \tag{9}
  a \subset b \wedge (T, V) \in a &\to (T, V) \in b, \\
  \mbox{}& \\
  \tag{10}
  a \subset b \wedge (W, U) \in a &\to (W, U) \in b
\end{align*}
が共に成り立つ.
またschema S4の適用により
\begin{align*}
  (V|y)((T, y) \in b) &\to \exists y((T, y) \in b), \\
  \mbox{} \\
  (W|x)((x, U) \in b) &\to \exists x((x, U) \in b)
\end{align*}
が共に成り立つが, 
仮定より$x$と$y$は共に$b$の中に自由変数として現れず, 
また上述のようにこれらは$T$及び$U$の中にも自由変数として現れないから, 
代入法則 \ref{substfree}, \ref{substfund}, \ref{substpair}により, 
上記の記号列はそれぞれ
\begin{align*}
  \tag{11}
  (T, V) \in b &\to \exists y((T, y) \in b), \\
  \mbox{}& \\
  \tag{12}
  (W, U) \in b &\to \exists x((x, U) \in b)
\end{align*}
と一致する.
よってこれらが共に定理となる.
またいま述べたように, $x$と$y$が共に$b$, $T$, $U$の中に自由変数として現れないことから, 
定理 \ref{sthmprsetelement}と推論法則 \ref{dedequiv}により
\begin{align*}
  \tag{13}
  \exists y((T, y) \in b) &\to T \in {\rm pr}_{1}\langle b \rangle, \\
  \mbox{}& \\
  \tag{14}
  \exists x((x, U) \in b) &\to U \in {\rm pr}_{2}\langle b \rangle
\end{align*}
が共に成り立つ.
そこで(7), (9), (11), (13)から, 推論法則 \ref{dedmmp}によって
\[
  a \subset b \wedge T \in {\rm pr}_{1}\langle a \rangle \to T \in {\rm pr}_{1}\langle b \rangle
\]
が成り立ち, (8), (10), (12), (14)から, 同じく推論法則 \ref{dedmmp}によって
\[
  a \subset b \wedge U \in {\rm pr}_{2}\langle a \rangle \to U \in {\rm pr}_{2}\langle b \rangle
\]
が成り立つことがわかる.
そこでこれらから, 推論法則 \ref{dedtwch}によって
\begin{align*}
  \tag{15}
  a \subset b &\to (T \in {\rm pr}_{1}\langle a \rangle \to T \in {\rm pr}_{1}\langle b \rangle), \\
  \mbox{}& \\
  \tag{16}
  a \subset b &\to (U \in {\rm pr}_{2}\langle a \rangle \to U \in {\rm pr}_{2}\langle b \rangle)
\end{align*}
が共に成り立つ.
また$T$と$U$の定義から, Thm \ref{thmallfund1}と推論法則 \ref{dedequiv}により
\begin{align*}
  (T|x)(x \in {\rm pr}_{1}\langle a \rangle \to x \in {\rm pr}_{1}\langle b \rangle) &\to 
  \forall x(x \in {\rm pr}_{1}\langle a \rangle \to x \in {\rm pr}_{1}\langle b \rangle), \\
  \mbox{}& \\
  (U|y)(y \in {\rm pr}_{2}\langle a \rangle \to y \in {\rm pr}_{2}\langle b \rangle) &\to 
  \forall y(y \in {\rm pr}_{2}\langle a \rangle \to y \in {\rm pr}_{2}\langle b \rangle)
\end{align*}
が共に成り立つが, 上述のように
$x$と$y$は共に${\rm pr}_{1}\langle a \rangle$, ${\rm pr}_{1}\langle b \rangle$, 
${\rm pr}_{2}\langle a \rangle$, ${\rm pr}_{2}\langle b \rangle$のいずれの記号列の中にも
自由変数として現れないから, 代入法則 \ref{substfree}, \ref{substfund}及び定義によれば, 上記の記号列はそれぞれ
\begin{align*}
  \tag{17}
  (T \in {\rm pr}_{1}\langle a \rangle \to T \in {\rm pr}_{1}\langle b \rangle) &\to 
  {\rm pr}_{1}\langle a \rangle \subset {\rm pr}_{1}\langle b \rangle, \\
  \mbox{}& \\
  \tag{18}
  (U \in {\rm pr}_{2}\langle a \rangle \to U \in {\rm pr}_{2}\langle b \rangle) &\to 
  {\rm pr}_{2}\langle a \rangle \subset {\rm pr}_{2}\langle b \rangle
\end{align*}
と一致する.
よってこれらが共に定理となる.
そこで(15), (17)から, 推論法則 \ref{dedmmp}によって
\[
  a \subset b \to {\rm pr}_{1}\langle a \rangle \subset {\rm pr}_{1}\langle b \rangle
\]
が成り立ち, 
(16), (18)から, 同じく推論法則 \ref{dedmmp}によって
\[
  a \subset b \to {\rm pr}_{2}\langle a \rangle \subset {\rm pr}_{2}\langle b \rangle
\]
が成り立つ.
($*$)が成り立つことは, これらと推論法則 \ref{dedmp}によって明らかである.
\halmos




\mathstrut
\begin{thm}
\label{sthmprset=}%定理
$a$と$b$を集合とするとき, 
\[
  a = b \to {\rm pr}_{1}\langle a \rangle = {\rm pr}_{1}\langle b \rangle, ~~
  a = b \to {\rm pr}_{2}\langle a \rangle = {\rm pr}_{2}\langle b \rangle
\]
が成り立つ.
またこのことから, 次の($*$)が成り立つ: 

($*$) ~~$a = b$が成り立つならば, 
        ${\rm pr}_{1}\langle a \rangle = {\rm pr}_{1}\langle b \rangle$と
        ${\rm pr}_{2}\langle a \rangle = {\rm pr}_{2}\langle b \rangle$が共に成り立つ.
\end{thm}


\noindent{\bf 証明}
~$x$を文字とするとき, Thm \ref{T=Ut1TV=UV1}より
\[
  a = b \to (a|x)({\rm pr}_{1}\langle x \rangle) = (b|x)({\rm pr}_{1}\langle x \rangle), ~~
  a = b \to (a|x)({\rm pr}_{2}\langle x \rangle) = (b|x)({\rm pr}_{2}\langle x \rangle)
\]
が共に成り立つが, 代入法則 \ref{substprset}によりこれらはそれぞれ
\[
  a = b \to {\rm pr}_{1}\langle a \rangle = {\rm pr}_{1}\langle b \rangle, ~~
  a = b \to {\rm pr}_{2}\langle a \rangle = {\rm pr}_{2}\langle b \rangle
\]
と一致するから, これらが共に定理となる.
($*$)が成り立つことは, これらと推論法則 \ref{dedmp}によって明らかである.
\halmos




\mathstrut
\begin{thm}
\label{sthmuopairprset}%定理
$a$, $b$, $c$, $d$を集合とするとき, 
\[
  {\rm pr}_{1}\langle \{(a, b), (c, d)\} \rangle = \{a, c\}, ~~
  {\rm pr}_{2}\langle \{(a, b), (c, d)\} \rangle = \{b, d\}
\]
が成り立つ.
\end{thm}


\noindent{\bf 証明}
~$x$と$y$を, 互いに異なり, 共に$a$, $b$, $c$, $d$のいずれの記号列の中にも自由変数として現れない, 
定数でない文字とする.
このとき変数法則 \ref{valnset}により, $x$と$y$は共に
$\{a, c\}$及び$\{b, d\}$の中に自由変数として現れない.
また変数法則 \ref{valnset}, \ref{valpair}からわかるように, 
$x$と$y$は共に$\{(a, b), (c, d)\}$の中にも自由変数として現れない.
そこで定理 \ref{sthmprsetelement}より
\begin{align*}
  \tag{1}
  x \in {\rm pr}_{1}\langle \{(a, b), (c, d)\} \rangle &\leftrightarrow 
  \exists y((x, y) \in \{(a, b), (c, d)\}), \\
  \mbox{} \\
  \tag{2}
  y \in {\rm pr}_{2}\langle \{(a, b), (c, d)\} \rangle &\leftrightarrow 
  \exists x((x, y) \in \{(a, b), (c, d)\})
\end{align*}
が共に成り立つ.
また定理 \ref{sthmuopairbasis}より
\[
\tag{3}
  (x, y) \in \{(a, b), (c, d)\} \leftrightarrow (x, y) = (a, b) \vee (x, y) = (c, d)
\]
が成り立つ.
また定理 \ref{sthmpair}より
\[
  (x, y) = (a, b) \leftrightarrow x = a \wedge y = b, ~~
  (x, y) = (c, d) \leftrightarrow x = c \wedge y = d
\]
が共に成り立つから, 
推論法則 \ref{dedaddeqv}により
\[
\tag{4}
  (x, y) = (a, b) \vee (x, y) = (c, d) \leftrightarrow (x = a \wedge y = b) \vee (x = c \wedge y = d)
\]
が成り立つ.
そこで(3)と(4)から, 推論法則 \ref{dedeqtrans}によって
\[
  (x, y) \in \{(a, b), (c, d)\} \leftrightarrow (x = a \wedge y = b) \vee (x = c \wedge y = d)
\]
が成り立つ.
いま$x$と$y$は共に定数でないから, これから推論法則 \ref{dedalleqquansepconst}によって
\begin{align*}
  \tag{5}
  \exists y((x, y) \in \{(a, b), (c, d)\}) &\leftrightarrow \exists y((x = a \wedge y = b) \vee (x = c \wedge y = d)), \\
  \mbox{} \\
  \tag{6}
  \exists x((x, y) \in \{(a, b), (c, d)\}) &\leftrightarrow \exists x((x = a \wedge y = b) \vee (x = c \wedge y = d))
\end{align*}
が共に成り立つ.
またThm \ref{thmexv}より
\begin{align*}
  \tag{7}
  \exists y((x = a \wedge y = b) \vee (x = c \wedge y = d)) &\leftrightarrow \exists y(x = a \wedge y = b) \vee \exists y(x = c \wedge y = d), \\
  \mbox{} \\
  \tag{8}
  \exists x((x = a \wedge y = b) \vee (x = c \wedge y = d)) &\leftrightarrow \exists x(x = a \wedge y = b) \vee \exists x(x = c \wedge y = d)
\end{align*}
が共に成り立つ.
また$x$と$y$が互いに異なり, 共に$a$, $b$, $c$, $d$のいずれの記号列の中にも
自由変数として現れないことから, 変数法則 \ref{valfund}によって, $y$が
$x = a$及び$x = c$の中に自由変数として現れず, $x$が$y = b$及び$y = d$の中に
自由変数として現れないことがわかるから, 
Thm \ref{thmexwrfree}より
\begin{align*}
  \exists y(x = a \wedge y = b) \leftrightarrow x = a \wedge \exists y(y = b)&, ~~
  \exists y(x = c \wedge y = d) \leftrightarrow x = c \wedge \exists y(y = d), \\
  \mbox{} \\
  \exists x(x = a \wedge y = b) \leftrightarrow \exists x(x = a) \wedge y = b&, ~~
  \exists x(x = c \wedge y = d) \leftrightarrow \exists x(x = c) \wedge y = d
\end{align*}
がすべて成り立つ.
そこでこのはじめの二つ, 後の二つから, 推論法則 \ref{dedaddeqv}によって
\begin{align*}
  \tag{9}
  \exists y(x = a \wedge y = b) \vee \exists y(x = c \wedge y = d) &\leftrightarrow 
  (x = a \wedge \exists y(y = b)) \vee (x = c \wedge \exists y(y = d)), \\
  \mbox{} \\
  \tag{10}
  \exists x(x = a \wedge y = b) \vee \exists x(x = c \wedge y = d) &\leftrightarrow 
  (\exists x(x = a) \wedge y = b) \vee (\exists x(x = c) \wedge y = d)
\end{align*}
がそれぞれ成り立つ.
またThm \ref{x=x}より
\[
  b = b, ~~
  d = d, ~~
  a = a, ~~
  c = c
\]
がすべて成り立つが, 
いま$x$と$y$は共に$a$, $b$, $c$, $d$のいずれの記号列の中にも
自由変数として現れないので, 
代入法則 \ref{substfree}, \ref{substfund}により, これらの記号列はそれぞれ
\[
  (b|y)(y = b), ~~
  (d|y)(y = d), ~~
  (a|x)(x = a), ~~
  (c|x)(x = c)
\]
と一致する.
よってこれらがすべて定理となる.
そこで推論法則 \ref{deds4}により, 
\[
  \exists y(y = b), ~~
  \exists y(y = d), ~~
  \exists x(x = a), ~~
  \exists x(x = c)
\]
がすべて成り立つ.
そこで推論法則 \ref{dedawblatrue2}により, 
\begin{align*}
  x = a \wedge \exists y(y = b) \leftrightarrow x = a&, ~~
  x = c \wedge \exists y(y = d) \leftrightarrow x = c, \\
  \mbox{} \\
  \exists x(x = a) \wedge y = b \leftrightarrow y = b&, ~~
  \exists x(x = c) \wedge y = d \leftrightarrow y = d
\end{align*}
がすべて成り立つ.
そこでこのはじめの二つ, 後の二つから, 推論法則 \ref{dedaddeqv}により
\begin{align*}
  \tag{11}
  (x = a \wedge \exists y(y = b)) \vee (x = c \wedge \exists y(y = d)) &\leftrightarrow 
  x = a \vee x = c, \\
  \mbox{} \\
  \tag{12}
  (\exists x(x = a) \wedge y = b) \vee (\exists x(x = c) \wedge y = d) &\leftrightarrow 
  y = b \vee y = d
\end{align*}
がそれぞれ成り立つ.
また定理 \ref{sthmuopairbasis}と推論法則 \ref{dedeqch}により
\begin{align*}
  \tag{13}
  x = a \vee x = c &\leftrightarrow x \in \{a, c\}, \\
  \mbox{} \\
  \tag{14}
  y = b \vee y = d &\leftrightarrow y \in \{b, d\}
\end{align*}
が共に成り立つ.
そこで(1), (5), (7), (9), (11), (13)から, 推論法則 \ref{dedeqtrans}によって
\[
\tag{15}
  x \in {\rm pr}_{1}\langle \{(a, b), (c, d)\} \rangle \leftrightarrow 
  x \in \{a, c\}
\]
が成り立ち, 
(2), (6), (8), (10), (12), (14)から, 同じく推論法則 \ref{dedeqtrans}によって
\[
\tag{16}
  y \in {\rm pr}_{2}\langle \{(a, b), (c, d)\} \rangle \leftrightarrow 
  y \in \{b, d\}
\]
が成り立つことがわかる.
いま$x$と$y$は共に定数でなく, はじめに述べたように, これらは共に$\{a, c\}$, $\{b, d\}$, 
$\{(a, b), (c, d)\}$のいずれの記号列の中にも自由変数として現れない.
そこで変数法則 \ref{valprset}により, これらは共に
${\rm pr}_{1}\langle \{(a, b), (c, d)\} \rangle$及び${\rm pr}_{2}\langle \{(a, b), (c, d)\} \rangle$の中にも
自由変数として現れない.
これらのことと, (15), (16)が成り立つことから, 
定理 \ref{sthmset=}により
\[
  {\rm pr}_{1}\langle \{(a, b), (c, d)\} \rangle = \{a, c\}, ~~
  {\rm pr}_{2}\langle \{(a, b), (c, d)\} \rangle = \{b, d\}
\]
が共に成り立つ.
\halmos




\mathstrut
\begin{thm}
\label{sthmsingletonprset}%定理
$a$と$b$を集合とするとき, 
\[
  {\rm pr}_{1}\langle \{(a, b)\} \rangle = \{a\}, ~~
  {\rm pr}_{2}\langle \{(a, b)\} \rangle = \{b\}
\]
が成り立つ.
\end{thm}


\noindent{\bf 証明}
~定理 \ref{sthmuopairprset}より
\[
  {\rm pr}_{1}\langle \{(a, b), (a, b)\} \rangle = \{a, a\}, ~~
  {\rm pr}_{2}\langle \{(a, b), (a, b)\} \rangle = \{b, b\}
\]
が共に成り立つが, 定義からこれらはそれぞれ
\[
  {\rm pr}_{1}\langle \{(a, b)\} \rangle = \{a\}, ~~
  {\rm pr}_{2}\langle \{(a, b)\} \rangle = \{b\}
\]
であるから, これらが共に定理となる.
\halmos




\mathstrut
\begin{thm}
\label{sthmpairsetofaprset}%定理
$a$を集合とし, $x$を$a$の中に自由変数として現れない文字とする.
このとき
\[
  {\rm pr}_{1}\langle a \rangle = {\rm pr}_{1}\langle \{x \in a|{\rm Pair}(x)\} \rangle, ~~
  {\rm pr}_{2}\langle a \rangle = {\rm pr}_{2}\langle \{x \in a|{\rm Pair}(x)\} \rangle
\]
が成り立つ.
\end{thm}


\noindent{\bf 証明}
~$u$と$v$を, 互いに異なり, 共に$x$と異なり, $a$の中に自由変数として現れない, 
定数でない文字とする.
このとき定理 \ref{sthmprsetelement}より
\begin{align*}
  \tag{1}
  u \in {\rm pr}_{1}\langle a \rangle &\leftrightarrow \exists v((u, v) \in a), \\
  \mbox{} \\
  \tag{2}
  v \in {\rm pr}_{2}\langle a \rangle &\leftrightarrow \exists u((u, v) \in a)
\end{align*}
が共に成り立つ.
また変数法則 \ref{valsset}, \ref{valbigpair}により, $u$と$v$は共に
$\{x \in a|{\rm Pair}(x)\}$の中に自由変数として現れないから, 
定理 \ref{sthmprsetelement}と推論法則 \ref{dedeqch}により
\begin{align*}
  \tag{3}
  \exists v((u, v) \in \{x \in a|{\rm Pair}(x)\}) &\leftrightarrow 
  u \in {\rm pr}_{1}\langle \{x \in a|{\rm Pair}(x)\} \rangle, \\
  \mbox{} \\
  \tag{4}
  \exists u((u, v) \in \{x \in a|{\rm Pair}(x)\}) &\leftrightarrow 
  v \in {\rm pr}_{2}\langle \{x \in a|{\rm Pair}(x)\} \rangle
\end{align*}
が共に成り立つ.
また$x$が$a$の中に自由変数として現れないことから, 定理 \ref{sthmpairsetofa}より
\[
  (u, v) \in a \leftrightarrow (u, v) \in \{x \in a|{\rm Pair}(x)\}
\]
が成り立つ.
いま$u$と$v$は共に定数でないので, これから推論法則 \ref{dedalleqquansepconst}によって
\begin{align*}
  \tag{5}
  \exists v((u, v) \in a) &\leftrightarrow \exists v((u, v) \in \{x \in a|{\rm Pair}(x)\}), \\
  \mbox{} \\
  \tag{6}
  \exists u((u, v) \in a) &\leftrightarrow \exists u((u, v) \in \{x \in a|{\rm Pair}(x)\})
\end{align*}
が共に成り立つ.
そこで(1), (5), (3)から, 推論法則 \ref{dedeqtrans}によって
\[
\tag{7}
  u \in {\rm pr}_{1}\langle a \rangle \leftrightarrow 
  u \in {\rm pr}_{1}\langle \{x \in a|{\rm Pair}(x)\} \rangle
\]
が成り立ち, (2), (6), (4)から, 同じく推論法則 \ref{dedeqtrans}によって
\[
\tag{8}
  v \in {\rm pr}_{2}\langle a \rangle \leftrightarrow 
  v \in {\rm pr}_{2}\langle \{x \in a|{\rm Pair}(x)\} \rangle
\]
が成り立つことがわかる.
いま$u$と$v$は共に$a$の中に自由変数として現れず, 上述のように
$\{x \in a|{\rm Pair}(x)\}$の中にも自由変数として現れないから, 
変数法則 \ref{valprset}により, これらは共に
${\rm pr}_{1}\langle a \rangle$, ${\rm pr}_{1}\langle \{x \in a|{\rm Pair}(x)\} \rangle$, 
${\rm pr}_{2}\langle a \rangle$, ${\rm pr}_{2}\langle \{x \in a|{\rm Pair}(x)\} \rangle$の
いずれの記号列の中にも自由変数として現れない.
また$u$と$v$は共に定数でない.
そこでこれらのことと, (7)と(8)が共に成り立つことから, 
定理 \ref{sthmset=}により
${\rm pr}_{1}\langle a \rangle = {\rm pr}_{1}\langle \{x \in a|{\rm Pair}(x)\} \rangle$と
${\rm pr}_{2}\langle a \rangle = {\rm pr}_{2}\langle \{x \in a|{\rm Pair}(x)\} \rangle$が
共に成り立つ.
\halmos




\mathstrut
\begin{thm}
\label{sthmprsetsubsetobjectset}%定理
$a$を集合とし, $z$を$a$の中に自由変数として現れない文字とする.
このとき
\[
  {\rm pr}_{1}\langle a \rangle \subset \{{\rm pr}_{1}(z)|z \in a\}, ~~
  {\rm pr}_{2}\langle a \rangle \subset \{{\rm pr}_{2}(z)|z \in a\}
\]
が成り立つ.
\end{thm}


\noindent{\bf 証明}
~$x$と$y$を, 互いに異なり, 共に$z$と異なり, $a$の中に自由変数として現れない, 
定数でない文字とする.
このとき変数法則 \ref{valoset}, \ref{valpr}, \ref{valprset}からわかるように, 
$x$と$y$は共に${\rm pr}_{1}\langle a \rangle$, ${\rm pr}_{2}\langle a \rangle$, 
$\{{\rm pr}_{1}(z)|z \in a\}$, $\{{\rm pr}_{2}(z)|z \in a\}$のいずれの記号列の中にも
自由変数として現れない.
そして定理 \ref{sthmprsetelement}と推論法則 \ref{dedequiv}により
\[
  x \in {\rm pr}_{1}\langle a \rangle \to \exists y((x, y) \in a), ~~
  y \in {\rm pr}_{2}\langle a \rangle \to \exists x((x, y) \in a)
\]
が共に成り立つ.
ここで$\tau_{y}((x, y) \in a)$を$T$, $\tau_{x}((x, y) \in a)$を$U$と書けば, 
これらは共に集合であり, 定義から上記の記号列はそれぞれ
\[
  x \in {\rm pr}_{1}\langle a \rangle \to (T|y)((x, y) \in a), ~~
  y \in {\rm pr}_{2}\langle a \rangle \to (U|x)((x, y) \in a)
\]
と同じである.
また上述のように$x$と$y$は互いに異なり, 共に$a$の中に自由変数として現れないから, 
代入法則 \ref{substfree}, \ref{substfund}, \ref{substpair}により, これらの記号列はそれぞれ
\begin{align*}
  \tag{1}
  x \in {\rm pr}_{1}\langle a \rangle &\to (x, T) \in a, \\
  \mbox{} \\
  \tag{2}
  y \in {\rm pr}_{2}\langle a \rangle &\to (U, y) \in a
\end{align*}
と一致する.
よってこれらが共に定理となる.
また定理 \ref{sthmprpair}より
\[
  {\rm pr}_{1}((x, T)) = x, ~~
  {\rm pr}_{2}((U, y)) = y
\]
が共に成り立つから, 推論法則 \ref{ded=ch}により
\[
  x = {\rm pr}_{1}((x, T)), ~~
  y = {\rm pr}_{2}((U, y))
\]
が共に成り立つ.
そこでこれらから, 推論法則 \ref{dedatawbtrue2}により
\begin{align*}
  \tag{3}
  (x, T) \in a &\to (x, T) \in a \wedge x = {\rm pr}_{1}((x, T)), \\
  \mbox{} \\
  \tag{4}
  (U, y) \in a &\to (U, y) \in a \wedge y = {\rm pr}_{2}((U, y))
\end{align*}
が共に成り立つ.
またschema S4の適用により
\begin{align*}
  ((x, T)|z)(z \in a \wedge x = {\rm pr}_{1}(z)) &\to \exists z(z \in a \wedge x = {\rm pr}_{1}(z)), \\
  \mbox{} \\
  ((U, y)|z)(z \in a \wedge y = {\rm pr}_{2}(z)) &\to \exists z(z \in a \wedge y = {\rm pr}_{2}(z))
\end{align*}
が共に成り立つが, $z$は$x$とも$y$とも異なり, 
仮定より$a$の中に自由変数として現れないから, 
代入法則 \ref{substfree}, \ref{substfund}, \ref{substwedge}, \ref{substpr}により, 
これらの記号列はそれぞれ
\begin{align*}
  \tag{5}
  (x, T) \in a \wedge x = {\rm pr}_{1}((x, T)) &\to \exists z(z \in a \wedge x = {\rm pr}_{1}(z)), \\
  \mbox{} \\
  \tag{6}
  (U, y) \in a \wedge y = {\rm pr}_{2}((U, y)) &\to \exists z(z \in a \wedge y = {\rm pr}_{2}(z))
\end{align*}
と一致する.
よってこれらが共に定理となる.
またいま述べたように, $z$は$x$とも$y$とも異なり, $a$の中に自由変数として現れないから, 
定理 \ref{sthmosetbasis}と推論法則 \ref{dedequiv}により
\begin{align*}
  \tag{7}
  \exists z(z \in a \wedge x = {\rm pr}_{1}(z)) &\to x \in \{{\rm pr}_{1}(z)|z \in a\}, \\
  \mbox{} \\
  \tag{8}
  \exists z(z \in a \wedge y = {\rm pr}_{2}(z)) &\to y \in \{{\rm pr}_{2}(z)|z \in a\}
\end{align*}
が共に成り立つ.
そこで(1), (3), (5), (7)から, 推論法則 \ref{dedmmp}によって
\[
\tag{9}
  x \in {\rm pr}_{1}\langle a \rangle \to x \in \{{\rm pr}_{1}(z)|z \in a\}
\]
が成り立ち, (2), (4), (6), (8)から, 同じく推論法則 \ref{dedmmp}によって
\[
\tag{10}
  y \in {\rm pr}_{2}\langle a \rangle \to y \in \{{\rm pr}_{2}(z)|z \in a\}
\]
が成り立つことがわかる.
いま$x$と$y$は共に定数でなく, 上述のようにこれらは共に
${\rm pr}_{1}\langle a \rangle$, ${\rm pr}_{2}\langle a \rangle$, 
$\{{\rm pr}_{1}(z)|z \in a\}$, $\{{\rm pr}_{2}(z)|z \in a\}$のいずれの記号列の中にも
自由変数として現れないから, 
このことと(9), (10)が共に成り立つことから, 
定理 \ref{sthmsubsetconst}によって
\[
  {\rm pr}_{1}\langle a \rangle \subset \{{\rm pr}_{1}(z)|z \in a\}, ~~
  {\rm pr}_{2}\langle a \rangle \subset \{{\rm pr}_{2}(z)|z \in a\}
\]
が共に成り立つことがわかる.
\halmos




\mathstrut
\begin{thm}
\label{sthmprset=objectset}%定理
$a$を集合とし, $z$を$a$の中に自由変数として現れない文字とする.
このとき
\[
  {\rm Graph}(a) \to {\rm pr}_{1}\langle a \rangle = \{{\rm pr}_{1}(z)|z \in a\}, ~~
  {\rm Graph}(a) \to {\rm pr}_{2}\langle a \rangle = \{{\rm pr}_{2}(z)|z \in a\}
\]
が成り立つ.
またこのことから, 次の($*$)が成り立つ: 

($*$) ~~$a$がグラフならば, ${\rm pr}_{1}\langle a \rangle = \{{\rm pr}_{1}(z)|z \in a\}$と
        ${\rm pr}_{2}\langle a \rangle = \{{\rm pr}_{2}(z)|z \in a\}$が共に成り立つ.
\end{thm}


\noindent{\bf 証明}
~$x$と$y$を, 互いに異なり, 共に$z$と異なり, $a$の中に自由変数として現れない, 
定数でない文字とする.
このとき変数法則 \ref{valoset}, \ref{valpr}, \ref{valprset}からわかるように, 
$x$と$y$は共に${\rm pr}_{1}\langle a \rangle$, ${\rm pr}_{2}\langle a \rangle$, 
$\{{\rm pr}_{1}(z)|z \in a\}$, $\{{\rm pr}_{2}(z)|z \in a\}$のいずれの記号列の中にも
自由変数として現れない.
そして$z$は$x$とも$y$とも異なり, 仮定より$a$の中に自由変数として現れないから, 
定理 \ref{sthmosetbasis}と推論法則 \ref{dedequiv}により
\begin{align*}
  x \in \{{\rm pr}_{1}(z)|z \in a\} &\to \exists z(z \in a \wedge x = {\rm pr}_{1}(z)), \\
  \mbox{} \\
  y \in \{{\rm pr}_{2}(z)|z \in a\} &\to \exists z(z \in a \wedge y = {\rm pr}_{2}(z))
\end{align*}
が共に成り立つ.
ここで$\tau_{z}(z \in a \wedge x = {\rm pr}_{1}(z))$を$T$, 
$\tau_{z}(z \in a \wedge y = {\rm pr}_{2}(z))$を$U$と書けば, 
これらは共に集合であり, 定義から上記の記号列はそれぞれ
\begin{align*}
  x \in \{{\rm pr}_{1}(z)|z \in a\} &\to (T|z)(z \in a \wedge x = {\rm pr}_{1}(z)), \\
  \mbox{} \\
  y \in \{{\rm pr}_{2}(z)|z \in a\} &\to (U|z)(z \in a \wedge y = {\rm pr}_{2}(z))
\end{align*}
と同じである.
またいま述べたように, $z$は$x$とも$y$とも異なり, $a$の中に自由変数として現れないから, 
代入法則 \ref{substfree}, \ref{substfund}, \ref{substwedge}, \ref{substpr}により, 
これらの記号列はそれぞれ
\begin{align*}
  \tag{1}
  x \in \{{\rm pr}_{1}(z)|z \in a\} &\to T \in a \wedge x = {\rm pr}_{1}(T), \\
  \mbox{} \\
  \tag{2}
  y \in \{{\rm pr}_{2}(z)|z \in a\} &\to U \in a \wedge y = {\rm pr}_{2}(U)
\end{align*}
と一致する.
よってこれらが共に定理となる.
そこで特に, 推論法則 \ref{dedprewedge}により
\begin{align*}
  \tag{3}
  x \in \{{\rm pr}_{1}(z)|z \in a\} &\to T \in a, \\
  \mbox{} \\
  \tag{4}
  y \in \{{\rm pr}_{2}(z)|z \in a\} &\to U \in a
\end{align*}
が共に成り立つ.
また(1), (2)から, 推論法則 \ref{dedaddw}により
\begin{align*}
  \tag{5}
  {\rm Graph}(a) \wedge x \in \{{\rm pr}_{1}(z)|z \in a\} &\to {\rm Graph}(a) \wedge (T \in a \wedge x = {\rm pr}_{1}(T)), \\
  \mbox{} \\
  \tag{6}
  {\rm Graph}(a) \wedge y \in \{{\rm pr}_{2}(z)|z \in a\} &\to {\rm Graph}(a) \wedge (U \in a \wedge y = {\rm pr}_{2}(U))
\end{align*}
が共に成り立つ.
またThm \ref{aw1bwc1t1awb1wc}より
\begin{align*}
  \tag{7}
  {\rm Graph}(a) \wedge (T \in a \wedge x = {\rm pr}_{1}(T)) &\to 
  ({\rm Graph}(a) \wedge T \in a) \wedge x = {\rm pr}_{1}(T), \\
  \mbox{} \\
  \tag{8}
  {\rm Graph}(a) \wedge (U \in a \wedge y = {\rm pr}_{2}(U)) &\to 
  ({\rm Graph}(a) \wedge U \in a) \wedge y = {\rm pr}_{2}(U)
\end{align*}
が共に成り立つ.
また定理 \ref{sthmgraphbasis}より
\begin{align*}
  {\rm Graph}(a) &\to (T \in a \to {\rm Pair}(T)), \\
  \mbox{} \\
  {\rm Graph}(a) &\to (U \in a \to {\rm Pair}(U))
\end{align*}
が共に成り立つから, 推論法則 \ref{dedtwch}により
\begin{align*}
  \tag{9}
  {\rm Graph}(a) \wedge T \in a &\to {\rm Pair}(T), \\
  \mbox{} \\
  \tag{10}
  {\rm Graph}(a) \wedge U \in a &\to {\rm Pair}(U)
\end{align*}
が共に成り立つ.
また定理 \ref{sthmbigpairpr}と推論法則 \ref{dedequiv}により
\begin{align*}
  \tag{11}
  {\rm Pair}(T) &\to T = ({\rm pr}_{1}(T), {\rm pr}_{2}(T)), \\
  \mbox{} \\
  \tag{12}
  {\rm Pair}(U) &\to U = ({\rm pr}_{1}(U), {\rm pr}_{2}(U))
\end{align*}
が共に成り立つ.
そこで(9)と(11), (10)と(12)から, 推論法則 \ref{dedmmp}によって
\begin{align*}
  {\rm Graph}(a) \wedge T \in a &\to T = ({\rm pr}_{1}(T), {\rm pr}_{2}(T)), \\
  \mbox{} \\
  {\rm Graph}(a) \wedge U \in a &\to U = ({\rm pr}_{1}(U), {\rm pr}_{2}(U))
\end{align*}
が共に成り立つ.
そこでこれらに推論法則 \ref{dedaddw}を適用して, 
\begin{align*}
  \tag{13}
  ({\rm Graph}(a) \wedge T \in a) \wedge x = {\rm pr}_{1}(T) &\to 
  T = ({\rm pr}_{1}(T), {\rm pr}_{2}(T)) \wedge x = {\rm pr}_{1}(T), \\
  \mbox{} \\
  \tag{14}
  ({\rm Graph}(a) \wedge U \in a) \wedge y = {\rm pr}_{2}(U) &\to 
  U = ({\rm pr}_{1}(U), {\rm pr}_{2}(U)) \wedge y = {\rm pr}_{2}(U)
\end{align*}
が共に成り立つ.
またThm \ref{x=yty=x}より
\begin{align*}
  \tag{15}
  x = {\rm pr}_{1}(T) &\to {\rm pr}_{1}(T) = x, \\
  \mbox{} \\
  \tag{16}
  y = {\rm pr}_{2}(U) &\to {\rm pr}_{2}(U) = y
\end{align*}
が共に成り立つ.
また定理 \ref{sthmpairweak}と推論法則 \ref{dedequiv}により
\begin{align*}
  \tag{17}
  {\rm pr}_{1}(T) = x &\to ({\rm pr}_{1}(T), {\rm pr}_{2}(T)) = (x, {\rm pr}_{2}(T)), \\
  \mbox{} \\
  \tag{18}
  {\rm pr}_{2}(U) = y &\to ({\rm pr}_{1}(U), {\rm pr}_{2}(U)) = ({\rm pr}_{1}(U), y)
\end{align*}
が共に成り立つ.
そこで(15)と(17), (16)と(18)から, 推論法則 \ref{dedmmp}によって
\begin{align*}
  x = {\rm pr}_{1}(T) \to ({\rm pr}_{1}(T), {\rm pr}_{2}(T)) = (x, {\rm pr}_{2}(T)), \\
  \mbox{} \\
  y = {\rm pr}_{2}(U) \to ({\rm pr}_{1}(U), {\rm pr}_{2}(U)) = ({\rm pr}_{1}(U), y)
\end{align*}
が共に成り立つ.
そこでこれらに推論法則 \ref{dedaddw}を適用して, 
\begin{align*}
  \tag{19}
  T = ({\rm pr}_{1}(T), {\rm pr}_{2}(T)) \wedge x = {\rm pr}_{1}(T) 
  &\to T = ({\rm pr}_{1}(T), {\rm pr}_{2}(T)) \wedge ({\rm pr}_{1}(T), {\rm pr}_{2}(T)) = (x, {\rm pr}_{2}(T)), \\
  \mbox{} \\
  \tag{20}
  U = ({\rm pr}_{1}(U), {\rm pr}_{2}(U)) \wedge y = {\rm pr}_{2}(U) 
  &\to U = ({\rm pr}_{1}(U), {\rm pr}_{2}(U)) \wedge ({\rm pr}_{1}(U), {\rm pr}_{2}(U)) = ({\rm pr}_{1}(U), y)
\end{align*}
が共に成り立つ.
またThm \ref{x=ywy=ztx=z}より
\begin{align*}
  \tag{21}
  T = ({\rm pr}_{1}(T), {\rm pr}_{2}(T)) \wedge ({\rm pr}_{1}(T), {\rm pr}_{2}(T)) = (x, {\rm pr}_{2}(T)) &\to 
  T = (x, {\rm pr}_{2}(T)), \\
  \mbox{} \\
  \tag{22}
  U = ({\rm pr}_{1}(U), {\rm pr}_{2}(U)) \wedge ({\rm pr}_{1}(U), {\rm pr}_{2}(U)) = ({\rm pr}_{1}(U), y) &\to 
  U = ({\rm pr}_{1}(U), y)
\end{align*}
が共に成り立つ.
そこで(5), (7), (13), (19), (21)から, 推論法則 \ref{dedmmp}によって
\[
\tag{23}
  {\rm Graph}(a) \wedge x \in \{{\rm pr}_{1}(z)|z \in a\} \to T = (x, {\rm pr}_{2}(T))
\]
が成り立ち, (6), (8), (14), (20), (22)から, 同じく推論法則 \ref{dedmmp}によって
\[
\tag{24}
  {\rm Graph}(a) \wedge y \in \{{\rm pr}_{2}(z)|z \in a\} \to U = ({\rm pr}_{1}(U), y)
\]
が成り立つことがわかる.
またThm \ref{awbta}より
\begin{align*}
  {\rm Graph}(a) \wedge x \in \{{\rm pr}_{1}(z)|z \in a\} &\to x \in \{{\rm pr}_{1}(z)|z \in a\}, \\
  \mbox{} \\
  {\rm Graph}(a) \wedge y \in \{{\rm pr}_{2}(z)|z \in a\} &\to y \in \{{\rm pr}_{2}(z)|z \in a\}
\end{align*}
が共に成り立つから, この前者と(3), 後者と(4)から, 推論法則 \ref{dedmmp}により
\begin{align*}
  \tag{25}
  {\rm Graph}(a) \wedge x \in \{{\rm pr}_{1}(z)|z \in a\} &\to T \in a, \\
  \mbox{} \\
  \tag{26}
  {\rm Graph}(a) \wedge y \in \{{\rm pr}_{2}(z)|z \in a\} &\to U \in a
\end{align*}
が共に成り立つ.
そこで(23)と(25), (24)と(26)から, 推論法則 \ref{dedprewedge}によって
\begin{align*}
  \tag{27}
  {\rm Graph}(a) \wedge x \in \{{\rm pr}_{1}(z)|z \in a\} &\to T = (x, {\rm pr}_{2}(T)) \wedge T \in a, \\
  \mbox{} \\
  \tag{28}
  {\rm Graph}(a) \wedge y \in \{{\rm pr}_{2}(z)|z \in a\} &\to U = ({\rm pr}_{1}(U), y) \wedge U \in a
\end{align*}
が共に成り立つ.
また定理 \ref{sthm=&in}より
\begin{align*}
  \tag{29}
  T = (x, {\rm pr}_{2}(T)) \wedge T \in a &\to (x, {\rm pr}_{2}(T)) \in a, \\
  \mbox{} \\
  \tag{30}
  U = ({\rm pr}_{1}(U), y) \wedge U \in a &\to ({\rm pr}_{1}(U), y) \in a
\end{align*}
が共に成り立つ.
またschema S4の適用により
\begin{align*}
  ({\rm pr}_{2}(T)|y)((x, y) \in a) &\to \exists y((x, y) \in a), \\
  \mbox{} \\
  ({\rm pr}_{1}(U)|x)((x, y) \in a) &\to \exists x((x, y) \in a)
\end{align*}
が共に成り立つ.
ここで$x$と$y$が互いに異なり, 共に$a$の中に自由変数として現れないことから, 
代入法則 \ref{substfree}, \ref{substfund}, \ref{substpair}により, これらの記号列はそれぞれ
\begin{align*}
  \tag{31}
  (x, {\rm pr}_{2}(T)) \in a &\to \exists y((x, y) \in a), \\
  \mbox{} \\
  \tag{32}
  ({\rm pr}_{1}(U), y) \in a &\to \exists x((x, y) \in a)
\end{align*}
と一致する.
よってこれらが共に定理となる.
またいま述べたように, $x$と$y$が互いに異なり, 共に$a$の中に自由変数として現れないことから, 
定理 \ref{sthmprsetelement}と推論法則 \ref{dedequiv}により
\begin{align*}
  \tag{33}
  \exists y((x, y) \in a) &\to x \in {\rm pr}_{1}\langle a \rangle, \\
  \mbox{} \\
  \tag{34}
  \exists x((x, y) \in a) &\to y \in {\rm pr}_{2}\langle a \rangle
\end{align*}
が共に成り立つ.
そこで(27), (29), (31), (33)から, 推論法則 \ref{dedmmp}によって
\[
  {\rm Graph}(a) \wedge x \in \{{\rm pr}_{1}(z)|z \in a\} \to x \in {\rm pr}_{1}\langle a \rangle
\]
が成り立ち, (28), (30), (32), (34)から, 同じく推論法則 \ref{dedmmp}によって
\[
  {\rm Graph}(a) \wedge y \in \{{\rm pr}_{2}(z)|z \in a\} \to y \in {\rm pr}_{2}\langle a \rangle
\]
が成り立つことがわかる.
そこでこれらに推論法則 \ref{dedtwch}を適用して, 
\begin{align*}
  \tag{35}
  {\rm Graph}(a) &\to (x \in \{{\rm pr}_{1}(z)|z \in a\} \to x \in {\rm pr}_{1}\langle a \rangle), \\
  \mbox{} \\
  \tag{36}
  {\rm Graph}(a) &\to (y \in \{{\rm pr}_{2}(z)|z \in a\} \to y \in {\rm pr}_{2}\langle a \rangle)
\end{align*}
が共に成り立つ.
いま$x$と$y$は共に$a$の中に自由変数として現れないから, 
変数法則 \ref{valgraph}により, これらは共に${\rm Graph}(a)$の中に自由変数として現れない.
また$x$と$y$は共に定数でない.
そこでこれらのことと, (35), (36)が共に成り立つことから, 推論法則 \ref{dedalltquansepfreeconst}によって
\begin{align*}
  {\rm Graph}(a) &\to \forall x(x \in \{{\rm pr}_{1}(z)|z \in a\} \to x \in {\rm pr}_{1}\langle a \rangle), \\
  \mbox{} \\
  {\rm Graph}(a) &\to \forall y(y \in \{{\rm pr}_{2}(z)|z \in a\} \to y \in {\rm pr}_{2}\langle a \rangle)
\end{align*}
が共に成り立つ.
はじめに述べたように, $x$と$y$は共に${\rm pr}_{1}\langle a \rangle$, ${\rm pr}_{2}\langle a \rangle$, 
$\{{\rm pr}_{1}(z)|z \in a\}$, $\{{\rm pr}_{2}(z)|z \in a\}$のいずれの記号列の中にも
自由変数として現れないから, 定義により上記の記号列はそれぞれ
\begin{align*}
  \tag{37}
  {\rm Graph}(a) &\to \{{\rm pr}_{1}(z)|z \in a\} \subset {\rm pr}_{1}\langle a \rangle, \\
  \mbox{} \\
  \tag{38}
  {\rm Graph}(a) &\to \{{\rm pr}_{2}(z)|z \in a\} \subset {\rm pr}_{2}\langle a \rangle
\end{align*}
と同じである.
よってこれらが共に定理となる.
また$z$が$a$の中に自由変数として現れないことから, 定理 \ref{sthmprsetsubsetobjectset}より
\[
  {\rm pr}_{1}\langle a \rangle \subset \{{\rm pr}_{1}(z)|z \in a\}, ~~
  {\rm pr}_{2}\langle a \rangle \subset \{{\rm pr}_{2}(z)|z \in a\}
\]
が共に成り立つ.
そこでこれらから, 推論法則 \ref{dedatawbtrue2}により
\begin{align*}
  \tag{39}
  \{{\rm pr}_{1}(z)|z \in a\} \subset {\rm pr}_{1}\langle a \rangle &\to 
  {\rm pr}_{1}\langle a \rangle \subset \{{\rm pr}_{1}(z)|z \in a\} \wedge \{{\rm pr}_{1}(z)|z \in a\} \subset {\rm pr}_{1}\langle a \rangle, \\
  \mbox{} \\
  \tag{40}
  \{{\rm pr}_{2}(z)|z \in a\} \subset {\rm pr}_{2}\langle a \rangle &\to 
  {\rm pr}_{2}\langle a \rangle \subset \{{\rm pr}_{2}(z)|z \in a\} \wedge \{{\rm pr}_{2}(z)|z \in a\} \subset {\rm pr}_{2}\langle a \rangle
\end{align*}
が共に成り立つ.
また定理 \ref{sthmaxiom1}と推論法則 \ref{dedequiv}により
\begin{align*}
  \tag{41}
  {\rm pr}_{1}\langle a \rangle \subset \{{\rm pr}_{1}(z)|z \in a\} \wedge \{{\rm pr}_{1}(z)|z \in a\} \subset {\rm pr}_{1}\langle a \rangle &\to 
  {\rm pr}_{1}\langle a \rangle = \{{\rm pr}_{1}(z)|z \in a\}, \\
  \mbox{} \\
  \tag{42}
  {\rm pr}_{2}\langle a \rangle \subset \{{\rm pr}_{2}(z)|z \in a\} \wedge \{{\rm pr}_{2}(z)|z \in a\} \subset {\rm pr}_{2}\langle a \rangle &\to 
  {\rm pr}_{2}\langle a \rangle = \{{\rm pr}_{2}(z)|z \in a\}
\end{align*}
が共に成り立つ.
そこで(37), (39), (41)から, 推論法則 \ref{dedmmp}によって
\[
  {\rm Graph}(a) \to {\rm pr}_{1}\langle a \rangle = \{{\rm pr}_{1}(z)|z \in a\}
\]
が成り立ち, (38), (40), (42)から, 同じく推論法則 \ref{dedmmp}によって
\[
  {\rm Graph}(a) \to {\rm pr}_{2}\langle a \rangle = \{{\rm pr}_{2}(z)|z \in a\}
\]
が成り立つことがわかる.
($*$)が成り立つことは, これらと推論法則 \ref{dedmp}から明らかである.
\halmos




\mathstrut
\begin{thm}
\label{sthmcupprset}%定理
$a$と$b$を集合とするとき, 
\[
  {\rm pr}_{1}\langle a \cup b \rangle = {\rm pr}_{1}\langle a \rangle \cup {\rm pr}_{1}\langle b \rangle, ~~
  {\rm pr}_{2}\langle a \cup b \rangle = {\rm pr}_{2}\langle a \rangle \cup {\rm pr}_{2}\langle b \rangle
\]
が成り立つ.
\end{thm}


\noindent{\bf 証明}
~$x$と$y$を互いに異なり, 共に$a$及び$b$の中に自由変数として現れない, 
定数でない文字とする.
このとき定理 \ref{sthmprsetelement}と推論法則 \ref{dedeqch}により
\begin{align*}
  &\exists y((x, y) \in a) \leftrightarrow x \in {\rm pr}_{1}\langle a \rangle, ~~
  \exists y((x, y) \in b) \leftrightarrow x \in {\rm pr}_{1}\langle b \rangle, \\
  \mbox{} \\
  &\exists x((x, y) \in a) \leftrightarrow y \in {\rm pr}_{2}\langle a \rangle, ~~
  \exists x((x, y) \in b) \leftrightarrow y \in {\rm pr}_{2}\langle b \rangle
\end{align*}
がすべて成り立つ.
そこでこのはじめの二つから, 推論法則 \ref{dedaddeqv}によって
\[
\tag{1}
  \exists y((x, y) \in a) \vee \exists y((x, y) \in b) \leftrightarrow 
  x \in {\rm pr}_{1}\langle a \rangle \vee x \in {\rm pr}_{1}\langle b \rangle
\]
が成り立ち, 後の二つから, 同じく推論法則 \ref{dedaddeqv}によって
\[
\tag{2}
  \exists x((x, y) \in a) \vee \exists x((x, y) \in b) \leftrightarrow 
  y \in {\rm pr}_{2}\langle a \rangle \vee y \in {\rm pr}_{2}\langle b \rangle
\]
が成り立つ.
また変数法則 \ref{valcup}により, $x$と$y$は共に$a \cup b$の中にも自由変数として現れないから, 
再び定理 \ref{sthmprsetelement}より
\begin{align*}
  \tag{3}
  x \in {\rm pr}_{1}\langle a \cup b\rangle &\leftrightarrow \exists y((x, y) \in a \cup b), \\
  \mbox{} \\
  \tag{4}
  y \in {\rm pr}_{2}\langle a \cup b\rangle &\leftrightarrow \exists x((x, y) \in a \cup b)
\end{align*}
が共に成り立つ.
また定理 \ref{sthmcupbasis}より
\[
  (x, y) \in a \cup b \leftrightarrow (x, y) \in a \vee (x, y) \in b
\]
が成り立つから, $x$と$y$が共に定数でないことから, 推論法則 \ref{dedalleqquansepconst}により
\begin{align*}
  \tag{5}
  \exists y((x, y) \in a \cup b) &\leftrightarrow \exists y((x, y) \in a \vee (x, y) \in b), \\
  \mbox{} \\
  \tag{6}
  \exists x((x, y) \in a \cup b) &\leftrightarrow \exists x((x, y) \in a \vee (x, y) \in b)
\end{align*}
が共に成り立つ.
またThm \ref{thmexv}より
\begin{align*}
  \tag{7}
  \exists y((x, y) \in a \vee (x, y) \in b) &\leftrightarrow \exists y((x, y) \in a) \vee \exists y((x, y) \in b), \\
  \mbox{} \\
  \tag{8}
  \exists x((x, y) \in a \vee (x, y) \in b) &\leftrightarrow \exists x((x, y) \in a) \vee \exists x((x, y) \in b)
\end{align*}
が共に成り立つ.
また定理 \ref{sthmcupbasis}と推論法則 \ref{dedeqch}により
\begin{align*}
  \tag{9}
  x \in {\rm pr}_{1}\langle a \rangle \vee x \in {\rm pr}_{1}\langle b \rangle &\leftrightarrow 
  x \in {\rm pr}_{1}\langle a \rangle \cup {\rm pr}_{1}\langle b \rangle, \\
  \mbox{} \\
  \tag{10}
  y \in {\rm pr}_{2}\langle a \rangle \vee y \in {\rm pr}_{2}\langle b \rangle &\leftrightarrow 
  y \in {\rm pr}_{2}\langle a \rangle \cup {\rm pr}_{2}\langle b \rangle
\end{align*}
が共に成り立つ.
そこで(3), (5), (7), (1), (9)にこの順で推論法則 \ref{dedeqtrans}を適用し, 
\[
\tag{11}
  x \in {\rm pr}_{1}\langle a \cup b\rangle \leftrightarrow 
  x \in {\rm pr}_{1}\langle a \rangle \cup {\rm pr}_{1}\langle b \rangle
\]
が成り立つことがわかる.
同様に, (4), (6), (8), (2), (10)にこの順で推論法則 \ref{dedeqtrans}を適用し, 
\[
\tag{12}
  y \in {\rm pr}_{2}\langle a \cup b\rangle \leftrightarrow 
  y \in {\rm pr}_{2}\langle a \rangle \cup {\rm pr}_{2}\langle b \rangle
\]
が成り立つこともわかる.
いま$x$と$y$は共に$a$及び$b$の中に自由変数として現れないから, 
変数法則 \ref{valcup}, \ref{valprset}からわかるように, これらは共に
${\rm pr}_{1}\langle a \cup b\rangle$, ${\rm pr}_{1}\langle a \rangle \cup {\rm pr}_{1}\langle b \rangle$, 
${\rm pr}_{2}\langle a \cup b\rangle$, ${\rm pr}_{2}\langle a \rangle \cup {\rm pr}_{2}\langle b \rangle$の
いずれの記号列の中にも自由変数として現れない.
また$x$と$y$は共に定数でない.
そこでこれらのことと, (11), (12)が共に成り立つことから, 
定理 \ref{sthmset=}により
${\rm pr}_{1}\langle a \cup b \rangle = {\rm pr}_{1}\langle a \rangle \cup {\rm pr}_{1}\langle b \rangle$と
${\rm pr}_{2}\langle a \cup b \rangle = {\rm pr}_{2}\langle a \rangle \cup {\rm pr}_{2}\langle b \rangle$が
共に成り立つ.
\halmos




\mathstrut
\begin{thm}
\label{sthmcapprset}%定理
$a$と$b$を集合とするとき, 
\[
  {\rm pr}_{1}\langle a \cap b \rangle \subset {\rm pr}_{1}\langle a \rangle \cap {\rm pr}_{1}\langle b \rangle, ~~
  {\rm pr}_{2}\langle a \cap b \rangle \subset {\rm pr}_{2}\langle a \rangle \cap {\rm pr}_{2}\langle b \rangle
\]
が成り立つ.
\end{thm}


\noindent{\bf 証明}
~定理 \ref{sthmcap}より
\[
  a \cap b \subset a, ~~
  a \cap b \subset b
\]
が共に成り立つから, 定理 \ref{sthmprsetsubset}により
\begin{align*}
  \tag{1}
  {\rm pr}_{1}\langle a \cap b \rangle &\subset {\rm pr}_{1}\langle a \rangle, \\
  \mbox{} \\
  \tag{2}
  {\rm pr}_{2}\langle a \cap b \rangle &\subset {\rm pr}_{2}\langle a \rangle, \\
  \mbox{} \\
  \tag{3}
  {\rm pr}_{1}\langle a \cap b \rangle &\subset {\rm pr}_{1}\langle b \rangle, \\
  \mbox{} \\
  \tag{4}
  {\rm pr}_{2}\langle a \cap b \rangle &\subset {\rm pr}_{2}\langle b \rangle
\end{align*}
がすべて成り立つ.
そこで(1)と(3), (2)と(4)から, それぞれ定理 \ref{sthmcapdil}によって
\[
  {\rm pr}_{1}\langle a \cap b \rangle \subset {\rm pr}_{1}\langle a \rangle \cap {\rm pr}_{1}\langle b \rangle, ~~
  {\rm pr}_{2}\langle a \cap b \rangle \subset {\rm pr}_{2}\langle a \rangle \cap {\rm pr}_{2}\langle b \rangle
\]
が成り立つことがわかる.
\halmos




\mathstrut
\begin{thm}
\label{sthm-prset}%定理
$a$と$b$を集合とするとき, 
\[
  {\rm pr}_{1}\langle a \rangle - {\rm pr}_{1}\langle b \rangle \subset {\rm pr}_{1}\langle a - b \rangle, ~~
  {\rm pr}_{2}\langle a \rangle - {\rm pr}_{2}\langle b \rangle \subset {\rm pr}_{2}\langle a - b \rangle
\]
が成り立つ.
\end{thm}


\noindent{\bf 証明}
~$x$と$y$を, 互いに異なり, 共に$a$及び$b$の中に自由変数として現れない, 
定数でない文字とする.
このとき定理 \ref{sthmprsetelement}と推論法則 \ref{dedequiv}により, 
\begin{align*}
  \tag{1}
  &x \in {\rm pr}_{1}\langle a \rangle \to \exists y((x, y) \in a), \\
  \mbox{} \\
  \tag{2}
  &y \in {\rm pr}_{2}\langle a \rangle \to \exists x((x, y) \in a), \\
  \mbox{} \\
  \tag{3}
  &\exists y((x, y) \in b) \to x \in {\rm pr}_{1}\langle b \rangle, \\
  \mbox{} \\
  \tag{4}
  &\exists x((x, y) \in b) \to y \in {\rm pr}_{2}\langle b \rangle
\end{align*}
がすべて成り立つ.
そこで(3)と(4)から, 推論法則 \ref{dedcp}によって
\begin{align*}
  \tag{5}
  x \notin {\rm pr}_{1}\langle b \rangle &\to \neg \exists y((x, y) \in b), \\
  \mbox{} \\
  \tag{6}
  y \notin {\rm pr}_{2}\langle b \rangle &\to \neg \exists x((x, y) \in b)
\end{align*}
が共に成り立つ.
また変数法則 \ref{val-}により, $x$と$y$は共に$a - b$の中にも自由変数として現れないから, 
同じく定理 \ref{sthmprsetelement}と推論法則 \ref{dedequiv}により, 
\begin{align*}
  \tag{7}
  \exists y((x, y) \in a - b) &\to x \in {\rm pr}_{1}\langle a - b \rangle, \\
  \mbox{} \\
  \tag{8}
  \exists x((x, y) \in a - b) &\to y \in {\rm pr}_{2}\langle a - b \rangle
\end{align*}
が共に成り立つ.
また定理 \ref{sthm-basis}と推論法則 \ref{dedequiv}により, 
\begin{align*}
  \tag{9}
  x \in {\rm pr}_{1}\langle a \rangle - {\rm pr}_{1}\langle b \rangle &\to 
  x \in {\rm pr}_{1}\langle a \rangle \wedge x \notin {\rm pr}_{1}\langle b \rangle, \\
  \mbox{} \\
  \tag{10}
  y \in {\rm pr}_{2}\langle a \rangle - {\rm pr}_{2}\langle b \rangle &\to 
  y \in {\rm pr}_{2}\langle a \rangle \wedge y \notin {\rm pr}_{2}\langle b \rangle
\end{align*}
が共に成り立つ.
またThm \ref{thmeaquandm}と推論法則 \ref{dedequiv}により, 
\begin{align*}
  \tag{11}
  \neg \exists y((x, y) \in b) &\to \forall y((x, y) \notin b), \\
  \mbox{} \\
  \tag{12}
  \neg \exists x((x, y) \in b) &\to \forall x((x, y) \notin b)
\end{align*}
が共に成り立つ.
そこで(5)と(11), (6)と(12)から, それぞれ推論法則 \ref{dedmmp}によって
\begin{align*}
  \tag{13}
  x \notin {\rm pr}_{1}\langle b \rangle &\to \forall y((x, y) \notin b), \\
  \mbox{} \\
  \tag{14}
  y \notin {\rm pr}_{2}\langle b \rangle &\to \forall x((x, y) \notin b)
\end{align*}
が共に成り立つ.
そこで(1)と(13), (2)と(14)から, それぞれ推論法則 \ref{dedfromaddw}によって
\begin{align*}
  \tag{15}
  x \in {\rm pr}_{1}\langle a \rangle \wedge x \notin {\rm pr}_{1}\langle b \rangle &\to 
  \exists y((x, y) \in a) \wedge \forall y((x, y) \notin b), \\
  \mbox{} \\
  \tag{16}
  y \in {\rm pr}_{2}\langle a \rangle \wedge y \notin {\rm pr}_{2}\langle b \rangle &\to 
  \exists x((x, y) \in a) \wedge \forall x((x, y) \notin b)
\end{align*}
が共に成り立つ.
またThm \ref{thmexw2}より
\begin{align*}
  \tag{17}
  \exists y((x, y) \in a) \wedge \forall y((x, y) \notin b) &\to 
  \exists y((x, y) \in a \wedge (x, y) \notin b), \\
  \mbox{} \\
  \tag{18}
  \exists x((x, y) \in a) \wedge \forall x((x, y) \notin b) &\to 
  \exists x((x, y) \in a \wedge (x, y) \notin b)
\end{align*}
が共に成り立つ.
また定理 \ref{sthm-basis}と推論法則 \ref{dedequiv}により
\[
  (x, y) \in a \wedge (x, y) \notin b \to (x, y) \in a - b
\]
が成り立つから, $x$と$y$が共に定数でないことから, 推論法則 \ref{dedalltquansepconst}によって
\begin{align*}
  \tag{19}
  \exists y((x, y) \in a \wedge (x, y) \notin b) &\to \exists y((x, y) \in a - b), \\
  \mbox{} \\
  \tag{20}
  \exists x((x, y) \in a \wedge (x, y) \notin b) &\to \exists x((x, y) \in a - b)
\end{align*}
が共に成り立つ.
そこで(9), (15), (17), (19), (7)にこの順で推論法則 \ref{dedmmp}を適用して, 
\[
\tag{21}
  x \in {\rm pr}_{1}\langle a \rangle - {\rm pr}_{1}\langle b \rangle \to x \in {\rm pr}_{1}\langle a - b \rangle
\]
が成り立つ.
同様に, (10), (16), (18), (20), (8)にこの順で推論法則 \ref{dedmmp}を適用して, 
\[
\tag{22}
  y \in {\rm pr}_{2}\langle a \rangle - {\rm pr}_{2}\langle b \rangle \to y \in {\rm pr}_{2}\langle a - b \rangle
\]
が成り立つ.
いま$x$と$y$は共に$a$及び$b$の中に自由変数として現れないから, 
変数法則 \ref{val-}, \ref{valprset}からわかるように, これらは共に
${\rm pr}_{1}\langle a \rangle - {\rm pr}_{1}\langle b \rangle$, 
${\rm pr}_{1}\langle a - b \rangle$, 
${\rm pr}_{2}\langle a \rangle - {\rm pr}_{2}\langle b \rangle$, 
${\rm pr}_{2}\langle a - b \rangle$の
いずれの記号列の中にも自由変数として現れない.
また$x$と$y$は共に定数でない.
そこでこのことと(21), (22)が共に成り立つことから, 
定理 \ref{sthmsubsetconst}により
\[
  {\rm pr}_{1}\langle a \rangle - {\rm pr}_{1}\langle b \rangle \subset {\rm pr}_{1}\langle a - b \rangle, ~~
  {\rm pr}_{2}\langle a \rangle - {\rm pr}_{2}\langle b \rangle \subset {\rm pr}_{2}\langle a - b \rangle
\]
が共に成り立つ.
\halmos




\mathstrut
\begin{thm}
\label{sthmemptyprset}%定理
$a$を集合とするとき, 
\[
  a = \phi \leftrightarrow {\rm Graph}(a) \wedge {\rm pr}_{1}\langle a \rangle = \phi, ~~
  a = \phi \leftrightarrow {\rm Graph}(a) \wedge {\rm pr}_{2}\langle a \rangle = \phi
\]
が成り立つ.
またこのことから, 次の1), 2), 3)が成り立つ.

1)
$a$が空ならば, ${\rm pr}_{1}\langle a \rangle$と${\rm pr}_{2}\langle a \rangle$は共に空である.
特に, ${\rm pr}_{1}\langle \phi \rangle$と${\rm pr}_{2}\langle \phi \rangle$は共に空である.

2)
$a$がグラフならば, $a = \phi \leftrightarrow {\rm pr}_{1}\langle a \rangle = \phi$と
$a = \phi \leftrightarrow {\rm pr}_{2}\langle a \rangle = \phi$が共に成り立つ.

3)
$a$がグラフであるとき, ${\rm pr}_{1}\langle a \rangle$が空ならば$a$は空であり, 
${\rm pr}_{2}\langle a \rangle$が空ならば$a$は空である.
\end{thm}


\noindent{\bf 証明}
~まず
\[
  a = \phi \leftrightarrow {\rm Graph}(a) \wedge {\rm pr}_{1}\langle a \rangle = \phi, ~~
  a = \phi \leftrightarrow {\rm Graph}(a) \wedge {\rm pr}_{2}\langle a \rangle = \phi
\]
が共に成り立つことを示す.
推論法則 \ref{dedequiv}があるから, 
\begin{align*}
  \tag{1}
  &a = \phi \to {\rm Graph}(a) \wedge {\rm pr}_{1}\langle a \rangle = \phi, \\
  \mbox{} \\
  \tag{2}
  &a = \phi \to {\rm Graph}(a) \wedge {\rm pr}_{2}\langle a \rangle = \phi, \\
  \mbox{} \\
  \tag{3}
  &{\rm Graph}(a) \wedge {\rm pr}_{1}\langle a \rangle = \phi \to a = \phi, \\
  \mbox{} \\
  \tag{4}
  &{\rm Graph}(a) \wedge {\rm pr}_{2}\langle a \rangle = \phi \to a = \phi
\end{align*}
がすべて成り立つことを示せば良い.

(1)と(2)の証明: 
定理 \ref{sthmgraph=}より
\[
  a = \phi \to ({\rm Graph}(a) \leftrightarrow {\rm Graph}(\phi))
\]
が成り立つから, 推論法則 \ref{dedprewedge}によって
\[
  a = \phi \to ({\rm Graph}(\phi) \to {\rm Graph}(a))
\]
が成り立ち, これから推論法則 \ref{dedch}によって
\[
\tag{5}
  {\rm Graph}(\phi) \to (a = \phi \to {\rm Graph}(a))
\]
が成り立つ.
また定理 \ref{sthmemptygraph}より${\rm Graph}(\phi)$が成り立つ.
そこでこれと(5)から, 推論法則 \ref{dedmp}によって
\[
\tag{6}
  a = \phi \to {\rm Graph}(a)
\]
が成り立つ.
さていま$x$と$y$を, 互いに異なり, 共に$a$の中に自由変数として現れない文字とする.
また$\tau_{x}(x \in {\rm pr}_{1}\langle a \rangle)$を$T$と書き, 
$\tau_{y}(y \in {\rm pr}_{2}\langle a \rangle)$を$U$と書く.
このとき$T$と$U$は共に集合であり, 
変数法則 \ref{valfund}, \ref{valtau}, \ref{valprset}からわかるように, 
$y$は$T$の中に自由変数として現れず, $x$は$U$の中に自由変数として現れない.
また$\tau_{y}(\neg ((T, y) \notin a))$を$V$と書き, 
$\tau_{x}(\neg ((x, U) \notin a))$を$W$と書けば, これらも集合である.
そして定理 \ref{sthmnotinempty}より
\[
  a = \phi \to (T, V) \notin a, ~~
  a = \phi \to (W, U) \notin a
\]
が共に成り立つが, いま$x$と$y$は共に$a$の中に自由変数として現れず, 
上述のように$y$, $x$はそれぞれ$T$, $U$の中に自由変数として現れないから, 
代入法則 \ref{substfree}, \ref{substfund}, \ref{substpair}により, 
これらの記号列はそれぞれ
\begin{align*}
  \tag{7}
  a = \phi &\to (V|y)((T, y) \notin a), \\
  \mbox{} \\
  \tag{8}
  a = \phi &\to (W|x)((x, U) \notin a)
\end{align*}
と一致する.
よってこれらが共に定理となる.
また$V$と$W$の定義から, Thm \ref{thmallfund1}と推論法則 \ref{dedequiv}により
\begin{align*}
  \tag{9}
  (V|y)((T, y) \notin a) &\to \forall y((T, y) \notin a), \\
  \mbox{} \\
  \tag{10}
  (W|x)((x, U) \notin a) &\to \forall x((x, U) \notin a)
\end{align*}
が共に成り立つ.
またThm \ref{thmeaquandm}と推論法則 \ref{dedequiv}により
\begin{align*}
  \tag{11}
  \forall y((T, y) \notin a) &\to \neg \exists y((T, y) \in a), \\
  \mbox{} \\
  \tag{12}
  \forall x((x, U) \notin a) &\to \neg \exists x((x, U) \in a)
\end{align*}
が共に成り立つ.
また上述のように, $y$は$a$及び$T$の中に自由変数として現れず, 
$x$は$a$及び$U$の中に自由変数として現れないから, 
定理 \ref{sthmprsetelement}と推論法則 \ref{dedequiv}により
\[
  T \in {\rm pr}_{1}\langle a \rangle \to \exists y((T, y) \in a), ~~
  U \in {\rm pr}_{2}\langle a \rangle \to \exists x((x, U) \in a)
\]
が共に成り立つ.
そこでこれらから, 推論法則 \ref{dedcp}によって
\begin{align*}
  \tag{13}
  \neg \exists y((T, y) \in a) &\to T \notin {\rm pr}_{1}\langle a \rangle, \\
  \mbox{} \\
  \tag{14}
  \neg \exists x((x, U) \in a) &\to U \notin {\rm pr}_{2}\langle a \rangle
\end{align*}
が共に成り立つ.
また$x$と$y$が共に$a$の中に自由変数として現れないことから, 
変数法則 \ref{valprset}によりこれらは共に${\rm pr}_{1}\langle a \rangle$及び
${\rm pr}_{2}\langle a \rangle$の中にも自由変数として現れないから, 
このことと$T$, $U$の定義から, 定理 \ref{sthmelm&empty}と推論法則 \ref{dedequiv}により
\begin{align*}
  \tag{15}
  T \notin {\rm pr}_{1}\langle a \rangle &\to {\rm pr}_{1}\langle a \rangle = \phi, \\
  \mbox{} \\
  \tag{16}
  U \notin {\rm pr}_{2}\langle a \rangle &\to {\rm pr}_{2}\langle a \rangle = \phi
\end{align*}
が共に成り立つ.
そこで(7), (9), (11), (13), (15)から,推論法則 \ref{dedmmp}によって
\[
\tag{17}
  a = \phi \to {\rm pr}_{1}\langle a \rangle = \phi
\]
が成り立ち, (8), (10), (12), (14), (16)から, 同じく推論法則 \ref{dedmmp}によって
\[
\tag{18}
  a = \phi \to {\rm pr}_{2}\langle a \rangle = \phi
\]
が成り立つことがわかる.
そこで(6)と(17)から, 推論法則 \ref{dedprewedge}によって(1)が成り立ち, 
(6)と(18)から, 同じく推論法則 \ref{dedprewedge}によって(2)が成り立つ.

(3)と(4)の証明: 
$x$と$y$は上と同じとする.
また$z$を$x$とも$y$とも異なり, $a$の中に自由変数として現れない文字とし, 
$\tau_{z}(z \in a)$を$X$と書く.
このとき$X$は集合であり, 変数法則 \ref{valfund}, \ref{valtau}, \ref{valpr}からわかるように, 
$x$と$y$は共に${\rm pr}_{1}(X)$及び${\rm pr}_{2}(X)$の中に自由変数として現れない.
また定理 \ref{sthmgraphbasis}より
\[
\tag{19}
  {\rm Graph}(a) \to (X \in a \to {\rm Pair}(X))
\]
が成り立つ.
また定理 \ref{sthmbigpairpr}と推論法則 \ref{dedequiv}により
\[
  {\rm Pair}(X) \to X = ({\rm pr}_{1}(X), {\rm pr}_{2}(X))
\]
が成り立つから, 推論法則 \ref{dedaddb}により
\[
\tag{20}
  (X \in a \to {\rm Pair}(X)) \to (X \in a \to X = ({\rm pr}_{1}(X), {\rm pr}_{2}(X)))
\]
が成り立つ.
またThm \ref{1cta1t11ctb1t1ctawb11}より
\begin{multline*}
  (X \in a \to X = ({\rm pr}_{1}(X), {\rm pr}_{2}(X))) \\
  \to ((X \in a \to X \in a) \to (X \in a \to X = ({\rm pr}_{1}(X), {\rm pr}_{2}(X)) \wedge X \in a))
\end{multline*}
が成り立つから, 推論法則 \ref{dedch}により
\begin{multline*}
\tag{21}
  (X \in a \to X \in a) \\
  \to ((X \in a \to X = ({\rm pr}_{1}(X), {\rm pr}_{2}(X))) \to 
  (X \in a \to X = ({\rm pr}_{1}(X), {\rm pr}_{2}(X)) \wedge X \in a))
\end{multline*}
が成り立つ.
またThm \ref{ata}より$X \in a \to X \in a$が成り立つ.
そこでこれと(21)から, 推論法則 \ref{dedmp}によって
\[
\tag{22}
  (X \in a \to X = ({\rm pr}_{1}(X), {\rm pr}_{2}(X))) \to 
  (X \in a \to X = ({\rm pr}_{1}(X), {\rm pr}_{2}(X)) \wedge X \in a)
\]
が成り立つ.
そこで(19), (20), (22)から, 推論法則 \ref{dedmmp}によって
\[
  {\rm Graph}(a) \to (X \in a \to X = ({\rm pr}_{1}(X), {\rm pr}_{2}(X)) \wedge X \in a)
\]
が成り立ち, これから推論法則 \ref{dedtwch}によって
\[
\tag{23}
  {\rm Graph}(a) \wedge X \in a \to X = ({\rm pr}_{1}(X), {\rm pr}_{2}(X)) \wedge X \in a
\]
が成り立つ.
また定理 \ref{sthm=&in}より
\[
\tag{24}
  X = ({\rm pr}_{1}(X), {\rm pr}_{2}(X)) \wedge X \in a \to 
  ({\rm pr}_{1}(X), {\rm pr}_{2}(X)) \in a
\]
が成り立つ.
またschema S4の適用により
\begin{align*}
  ({\rm pr}_{2}(X)|y)(({\rm pr}_{1}(X), y) \in a) &\to \exists y(({\rm pr}_{1}(X), y) \in a), \\
  \mbox{} \\
  ({\rm pr}_{1}(X)|x)((x, {\rm pr}_{2}(X)) \in a) &\to \exists x((x, {\rm pr}_{2}(X)) \in a)
\end{align*}
が共に成り立つが, $x$と$y$は共に$a$の中に自由変数として現れず, 
上述のようにこれらは共に${\rm pr}_{1}(X)$及び${\rm pr}_{2}(X)$の中にも自由変数として現れないから, 
代入法則 \ref{substfree}, \ref{substfund}, \ref{substpair}により, これらの記号列はそれぞれ
\begin{align*}
  \tag{25}
  ({\rm pr}_{1}(X), {\rm pr}_{2}(X)) \in a &\to \exists y(({\rm pr}_{1}(X), y) \in a), \\
  \mbox{} \\
  \tag{26}
  ({\rm pr}_{1}(X), {\rm pr}_{2}(X)) \in a &\to \exists x((x, {\rm pr}_{2}(X)) \in a)
\end{align*}
と一致する.
よってこれらが共に定理となる.
また今述べたように, 
$x$と$y$が共に$a$, ${\rm pr}_{1}(X)$, ${\rm pr}_{2}(X)$のいずれの記号列の中にも自由変数として現れないことから, 
定理 \ref{sthmprsetelement}と推論法則 \ref{dedequiv}により
\begin{align*}
  \tag{27}
  \exists y(({\rm pr}_{1}(X), y) \in a) &\to {\rm pr}_{1}(X) \in {\rm pr}_{1}\langle a \rangle, \\
  \mbox{} \\
  \tag{28}
  \exists x((x, {\rm pr}_{2}(X)) \in a) &\to {\rm pr}_{2}(X) \in {\rm pr}_{2}\langle a \rangle
\end{align*}
が共に成り立つ.
また定理 \ref{sthmnotemptyeqexin}より
\begin{align*}
  \tag{29}
  {\rm pr}_{1}(X) \in {\rm pr}_{1}\langle a \rangle &\to {\rm pr}_{1}\langle a \rangle \neq \phi, \\
  \mbox{} \\
  \tag{30}
  {\rm pr}_{2}(X) \in {\rm pr}_{2}\langle a \rangle &\to {\rm pr}_{2}\langle a \rangle \neq \phi
\end{align*}
が共に成り立つ.
そこで(23), (24), (25), (27), (29)から, 推論法則 \ref{dedmmp}によって
\[
  {\rm Graph}(a) \wedge X \in a \to {\rm pr}_{1}\langle a \rangle \neq \phi
\]
が成り立ち, (23), (24), (26), (28), (30)から, 同じく推論法則 \ref{dedmmp}によって
\[
  {\rm Graph}(a) \wedge X \in a \to {\rm pr}_{2}\langle a \rangle \neq \phi
\]
が成り立つことがわかる.
そこでこれらにそれぞれ推論法則 \ref{dedtwch}を適用して, 
\begin{align*}
  \tag{31}
  {\rm Graph}(a) &\to (X \in a \to {\rm pr}_{1}\langle a \rangle \neq \phi), \\
  \mbox{} \\
  \tag{32}
  {\rm Graph}(a) &\to (X \in a \to {\rm pr}_{2}\langle a \rangle \neq \phi)
\end{align*}
が共に成り立つ.
またThm \ref{1atnb1t1btna1}より
\begin{align*}
  \tag{33}
  (X \in a \to {\rm pr}_{1}\langle a \rangle \neq \phi) &\to 
  ({\rm pr}_{1}\langle a \rangle = \phi \to X \notin a), \\
  \mbox{} \\
  \tag{34}
  (X \in a \to {\rm pr}_{2}\langle a \rangle \neq \phi) &\to 
  ({\rm pr}_{2}\langle a \rangle = \phi \to X \notin a)
\end{align*}
が共に成り立つ.
そこで(31)と(33), (32)と(34)から, それぞれ推論法則 \ref{dedmmp}によって
\begin{align*}
  {\rm Graph}(a) &\to ({\rm pr}_{1}\langle a \rangle = \phi \to X \notin a), \\
  \mbox{} \\
  {\rm Graph}(a) &\to ({\rm pr}_{2}\langle a \rangle = \phi \to X \notin a)
\end{align*}
が共に成り立ち, これらから, 推論法則 \ref{dedtwch}によって
\begin{align*}
  \tag{35}
  {\rm Graph}(a) \wedge {\rm pr}_{1}\langle a \rangle = \phi &\to X \notin a, \\
  \mbox{} \\
  \tag{36}
  {\rm Graph}(a) \wedge {\rm pr}_{2}\langle a \rangle = \phi &\to X \notin a
\end{align*}
が共に成り立つ.
また$z$が$a$の中に自由変数として現れないことと$X$の定義から, 
定理 \ref{sthmelm&empty}と推論法則 \ref{dedequiv}により
\[
\tag{37}
  X \notin a \to a = \phi
\]
が成り立つ.
そこで(35)と(37)から, 推論法則 \ref{dedmmp}によって(3)が成り立ち, 
(36)と(37)から, 同じく推論法則 \ref{dedmmp}によって(4)が成り立つ.

\noindent
1)
上で示したように(17)と(18)が共に成り立つから, 
$a$が空ならば, 推論法則 \ref{dedmp}によって
${\rm pr}_{1}\langle a \rangle$と${\rm pr}_{2}\langle a \rangle$は
共に空となる.
特にThm \ref{x=x}より$\phi = \phi$が成り立つから, 
${\rm pr}_{1}\langle \phi \rangle$と${\rm pr}_{2}\langle \phi \rangle$は
共に空である.

\noindent
2)
上で示したように, 
\begin{align*}
  \tag{38}
  a = \phi &\leftrightarrow {\rm Graph}(a) \wedge {\rm pr}_{1}\langle a \rangle = \phi, \\
  \mbox{} \\
  \tag{39}
  a = \phi &\leftrightarrow {\rm Graph}(a) \wedge {\rm pr}_{2}\langle a \rangle = \phi
\end{align*}
が共に成り立つ.
また$a$がグラフであるという仮定から, 推論法則 \ref{dedawblatrue2}により
\begin{align*}
  \tag{40}
  {\rm Graph}(a) \wedge {\rm pr}_{1}\langle a \rangle = \phi &\leftrightarrow {\rm pr}_{1}\langle a \rangle = \phi, \\
  \mbox{} \\
  \tag{41}
  {\rm Graph}(a) \wedge {\rm pr}_{2}\langle a \rangle = \phi &\leftrightarrow {\rm pr}_{2}\langle a \rangle = \phi
\end{align*}
が共に成り立つ.
そこで(38)と(40)から, 推論法則 \ref{dedeqtrans}によって
$a = \phi \leftrightarrow {\rm pr}_{1}\langle a \rangle = \phi$が成り立ち, 
(39)と(41)から, 同じく推論法則 \ref{dedeqtrans}によって
$a = \phi \leftrightarrow {\rm pr}_{2}\langle a \rangle = \phi$が成り立つ.

\noindent
3)
このとき2)により
$a = \phi \leftrightarrow {\rm pr}_{1}\langle a \rangle = \phi$と
$a = \phi \leftrightarrow {\rm pr}_{2}\langle a \rangle = \phi$が共に成り立つから, 
3)が成り立つことはこれらと推論法則 \ref{dedeqfund}によって明らかである.
\halmos




\mathstrut
\begin{thm}
\label{sthmproductprset}%定理
$a$と$b$を集合とするとき, 
\begin{align*}
  {\rm pr}_{1}\langle a \times b \rangle \subset a&, ~~
  {\rm pr}_{2}\langle a \times b \rangle \subset b, \\
  \mbox{} \\
  b \neq \phi \to {\rm pr}_{1}\langle a \times b \rangle = a&, ~~
  a \neq \phi \to {\rm pr}_{2}\langle a \times b \rangle = b
\end{align*}
が成り立つ.
またこのことから, 次の($*$)が成り立つ: 

($*$) ~~$b$が空でなければ, ${\rm pr}_{1}\langle a \times b \rangle = a$が成り立つ.
        また$a$が空でなければ, ${\rm pr}_{2}\langle a \times b \rangle = b$が成り立つ.
\end{thm}


\noindent{\bf 証明}
~$x$と$y$を, 互いに異なり, 共に$a$及び$b$の中に自由変数として現れない, 定数でない文字とする.
このとき変数法則 \ref{valproduct}により$x$と$y$は共に$a \times b$の中に自由変数として現れないから, 
定理 \ref{sthmprsetelement}より
\begin{align*}
  \tag{1}
  x \in {\rm pr}_{1}\langle a \times b \rangle &\leftrightarrow \exists y((x, y) \in a \times b), \\
  \mbox{} \\
  \tag{2}
  y \in {\rm pr}_{2}\langle a \times b \rangle &\leftrightarrow \exists x((x, y) \in a \times b)
\end{align*}
が共に成り立つ.
また定理 \ref{sthmpairinproduct}より
\[
  (x, y) \in a \times b \leftrightarrow x \in a \wedge y \in b
\]
が成り立つ.
$x$と$y$は共に定数でないから, これから推論法則 \ref{dedalleqquansepconst}によって
\begin{align*}
  \tag{3}
  \exists y((x, y) \in a \times b) &\leftrightarrow \exists y(x \in a \wedge y \in b), \\
  \mbox{} \\
  \tag{4}
  \exists x((x, y) \in a \times b) &\leftrightarrow \exists x(x \in a \wedge y \in b)
\end{align*}
が共に成り立つ.
またThm \ref{thmquanwch}より
\[
\tag{5}
  \exists x(x \in a \wedge y \in b) \leftrightarrow \exists x(y \in b \wedge x \in a)
\]
が成り立つ.
また$x$と$y$が互いに異なり, 共に$a$及び$b$の中に自由変数として現れないことから, 
変数法則 \ref{valfund}により$y$は$x \in a$の中に自由変数として現れず, 
$x$は$y \in b$の中に自由変数として現れないから, Thm \ref{thmexwrfree}より
\begin{align*}
  \tag{6}
  \exists y(x \in a \wedge y \in b) &\leftrightarrow x \in a \wedge \exists y(y \in b), \\
  \mbox{} \\
  \tag{7}
  \exists x(y \in b \wedge x \in a) &\leftrightarrow y \in b \wedge \exists x(x \in a)
\end{align*}
が共に成り立つ.
また$x$と$y$が共に$a$及び$b$の中に自由変数として現れないことから, 
定理 \ref{sthmnotemptyeqexin}と推論法則 \ref{dedeqch}により
\[
  \exists y(y \in b) \leftrightarrow b \neq \phi, ~~
  \exists x(x \in a) \leftrightarrow a \neq \phi
\]
が共に成り立つから, 推論法則 \ref{dedaddeqw}により
\begin{align*}
  \tag{8}
  x \in a \wedge \exists y(y \in b) &\leftrightarrow x \in a \wedge b \neq \phi, \\
  \mbox{} \\
  \tag{9}
  y \in b \wedge \exists x(x \in a) &\leftrightarrow y \in b \wedge a \neq \phi
\end{align*}
が共に成り立つ.
そこで(1), (3), (6), (8)から, 推論法則 \ref{dedeqtrans}によって
\[
  x \in {\rm pr}_{1}\langle a \times b \rangle \leftrightarrow x \in a \wedge b \neq \phi
\]
が成り立ち, (2), (4), (5), (7), (9)から, 同じく推論法則 \ref{dedeqtrans}によって
\[
  y \in {\rm pr}_{2}\langle a \times b \rangle \leftrightarrow y \in b \wedge a \neq \phi
\]
が成り立つことがわかる.
そこでこれらから, 推論法則 \ref{dedequiv}によって
\begin{align*}
  \tag{10}
  &x \in {\rm pr}_{1}\langle a \times b \rangle \to x \in a \wedge b \neq \phi, \\
  \mbox{} \\
  \tag{11}
  &x \in a \wedge b \neq \phi \to x \in {\rm pr}_{1}\langle a \times b \rangle, \\
  \mbox{} \\
  \tag{12}
  &y \in {\rm pr}_{2}\langle a \times b \rangle \to y \in b \wedge a \neq \phi, \\
  \mbox{} \\
  \tag{13}
  &y \in b \wedge a \neq \phi \to y \in {\rm pr}_{2}\langle a \times b \rangle
\end{align*}
がすべて成り立つ.
またThm \ref{awbta}より
\begin{align*}
  \tag{14}
  x \in a \wedge b \neq \phi &\to x \in a, \\
  \mbox{} \\
  \tag{15}
  y \in b \wedge a \neq \phi &\to y \in b
\end{align*}
が共に成り立つ.
そこで(10)と(14), (12)と(15)から, 推論法則 \ref{dedmmp}によって
\begin{align*}
  \tag{16}
  x \in {\rm pr}_{1}\langle a \times b \rangle &\to x \in a, \\
  \mbox{} \\
  \tag{17}
  y \in {\rm pr}_{2}\langle a \times b \rangle &\to y \in b
\end{align*}
が共に成り立つ.
いま$x$と$y$は共に$a$及び$b$の中に自由変数として現れず, 
従って変数法則 \ref{valproduct}, \ref{valprset}により, これらは共に
${\rm pr}_{1}\langle a \times b \rangle$及び${\rm pr}_{2}\langle a \times b \rangle$の中にも
自由変数として現れない.
また$x$と$y$は共に定数でない.
そこでこのことと, (16)と(17)が共に成り立つことから, 
定理 \ref{sthmsubsetconst}により
\[
  {\rm pr}_{1}\langle a \times b \rangle \subset a, ~~
  {\rm pr}_{2}\langle a \times b \rangle \subset b
\]
が共に成り立つ.
そこでこれらから, 推論法則 \ref{deds1}によって
\begin{align*}
  \tag{18}
  b \neq \phi &\to {\rm pr}_{1}\langle a \times b \rangle \subset a, \\
  \mbox{} \\
  \tag{19}
  a \neq \phi &\to {\rm pr}_{2}\langle a \times b \rangle \subset b
\end{align*}
が共に成り立つ.
また(11)と(13)から, 推論法則 \ref{dedtwch}によって
\[
  x \in a \to (b \neq \phi \to x \in {\rm pr}_{1}\langle a \times b \rangle), ~~
  y \in b \to (a \neq \phi \to y \in {\rm pr}_{2}\langle a \times b \rangle)
\]
が共に成り立ち, これらから, 推論法則 \ref{dedch}によって
\begin{align*}
  \tag{20}
  b \neq \phi &\to (x \in a \to x \in {\rm pr}_{1}\langle a \times b \rangle), \\
  \mbox{} \\
  \tag{21}
  a \neq \phi &\to (y \in b \to y \in {\rm pr}_{2}\langle a \times b \rangle)
\end{align*}
が共に成り立つ.
いま$x$と$y$は共に$a$及び$b$の中に自由変数として現れないから, 
変数法則 \ref{valfund}, \ref{valempty}により, これらは共に
$a \neq \phi$及び$b \neq \phi$の中に自由変数として現れない.
また$x$と$y$は共に定数でない.
そこでこれらのことと, (20), (21)が共に成り立つことから, 
推論法則 \ref{dedalltquansepfreeconst}によって
\[
  b \neq \phi \to \forall x(x \in a \to x \in {\rm pr}_{1}\langle a \times b \rangle), ~~
  a \neq \phi \to \forall y(y \in b \to y \in {\rm pr}_{2}\langle a \times b \rangle)
\]
が共に成り立つが, 
上述のように$x$は$a$及び${\rm pr}_{1}\langle a \times b \rangle$の中に
自由変数として現れず, $y$は$b$及び${\rm pr}_{2}\langle a \times b \rangle$の中に自由変数として現れないから, 
定義によりこれらの記号列はそれぞれ
\begin{align*}
  \tag{22}
  b \neq \phi &\to a \subset {\rm pr}_{1}\langle a \times b \rangle, \\
  \mbox{} \\
  \tag{23}
  a \neq \phi &\to b \subset {\rm pr}_{2}\langle a \times b \rangle
\end{align*}
と同じである.
よってこれらが共に定理となる.
そこで(18)と(22), (19)と(23)から, それぞれ推論法則 \ref{dedprewedge}によって
\begin{align*}
  \tag{24}
  b \neq \phi &\to {\rm pr}_{1}\langle a \times b \rangle \subset a \wedge a \subset {\rm pr}_{1}\langle a \times b \rangle, \\
  \mbox{} \\
  \tag{25}
  a \neq \phi &\to {\rm pr}_{2}\langle a \times b \rangle \subset b \wedge b \subset {\rm pr}_{2}\langle a \times b \rangle
\end{align*}
が共に成り立つ.
また定理 \ref{sthmaxiom1}と推論法則 \ref{dedequiv}により
\begin{align*}
  \tag{26}
  {\rm pr}_{1}\langle a \times b \rangle \subset a \wedge a \subset {\rm pr}_{1}\langle a \times b \rangle &\to 
  {\rm pr}_{1}\langle a \times b \rangle = a, \\
  \mbox{} \\
  \tag{27}
  {\rm pr}_{2}\langle a \times b \rangle \subset b \wedge b \subset {\rm pr}_{2}\langle a \times b \rangle &\to 
  {\rm pr}_{2}\langle a \times b \rangle = b
\end{align*}
が共に成り立つ.
そこで(24)と(26), (25)と(27)から, それぞれ推論法則 \ref{dedmmp}によって
\[
  b \neq \phi \to {\rm pr}_{1}\langle a \times b \rangle = a, ~~
  a \neq \phi \to {\rm pr}_{2}\langle a \times b \rangle = b
\]
が共に成り立つ.
($*$)が成り立つことは, これらと推論法則 \ref{dedmp}によって明らかである.
\halmos




\mathstrut
\begin{thm}
\label{sthmprsetminimality}%定理
$a$, $b$, $c$を集合とするとき, 
\[
  a \subset b \times c \to {\rm pr}_{1}\langle a \rangle \subset b \wedge {\rm pr}_{2}\langle a \rangle \subset c
\]
が成り立つ.
またこのことから, 次の($*$)が成り立つ: 

($*$) ~~$a \subset b \times c$が成り立つならば, 
        ${\rm pr}_{1}\langle a \rangle \subset b$と${\rm pr}_{2}\langle a \rangle \subset c$が
        共に成り立つ.
\end{thm}


\noindent{\bf 証明}
~定理 \ref{sthmprsetsubset}より
\begin{align*}
  \tag{1}
  a \subset b \times c &\to {\rm pr}_{1}\langle a \rangle \subset {\rm pr}_{1}\langle b \times c \rangle, \\
  \mbox{}& \\
  \tag{2}
  a \subset b \times c &\to {\rm pr}_{2}\langle a \rangle \subset {\rm pr}_{2}\langle b \times c \rangle
\end{align*}
が共に成り立つ.
また定理 \ref{sthmproductprset}より
\[
  {\rm pr}_{1}\langle b \times c \rangle \subset b, ~~
  {\rm pr}_{2}\langle b \times c \rangle \subset c
\]
が共に成り立つから, 推論法則 \ref{dedatawbtrue2}によって
\begin{align*}
  \tag{3}
  {\rm pr}_{1}\langle a \rangle \subset {\rm pr}_{1}\langle b \times c \rangle &\to 
  {\rm pr}_{1}\langle a \rangle \subset {\rm pr}_{1}\langle b \times c \rangle \wedge {\rm pr}_{1}\langle b \times c \rangle \subset b, \\
  \mbox{}& \\
  \tag{4}
  {\rm pr}_{2}\langle a \rangle \subset {\rm pr}_{2}\langle b \times c \rangle &\to 
  {\rm pr}_{2}\langle a \rangle \subset {\rm pr}_{2}\langle b \times c \rangle \wedge {\rm pr}_{2}\langle b \times c \rangle \subset c
\end{align*}
が共に成り立つ.
また定理 \ref{sthmsubsettrans}より
\begin{align*}
  \tag{5}
  {\rm pr}_{1}\langle a \rangle \subset {\rm pr}_{1}\langle b \times c \rangle \wedge {\rm pr}_{1}\langle b \times c \rangle \subset b &\to 
  {\rm pr}_{1}\langle a \rangle \subset b, \\
  \mbox{}& \\
  \tag{6}
  {\rm pr}_{2}\langle a \rangle \subset {\rm pr}_{2}\langle b \times c \rangle \wedge {\rm pr}_{2}\langle b \times c \rangle \subset c &\to
  {\rm pr}_{2}\langle a \rangle \subset c
\end{align*}
が共に成り立つ.
そこで(1), (3), (5)から, 推論法則 \ref{dedmmp}によって
\[
\tag{7}
  a \subset b \times c \to {\rm pr}_{1}\langle a \rangle \subset b
\]
が成り立ち, (2), (4), (6)から, 同じく推論法則 \ref{dedmmp}によって
\[
\tag{8}
  a \subset b \times c \to {\rm pr}_{2}\langle a \rangle \subset c
\]
が成り立つことがわかる.
(7), (8)から, 推論法則 \ref{dedprewedge}によって
$a \subset b \times c \to {\rm pr}_{1}\langle a \rangle \subset b \wedge {\rm pr}_{2}\langle a \rangle \subset c$が
成り立つ.
($*$)が成り立つことは, (7), (8)と推論法則 \ref{dedmp}によって明らかである.
\halmos




\mathstrut
\begin{thm}
\label{sthmpairelementinprset}%定理
$a$, $b$, $c$を集合とするとき, 
\[
  (a, b) \in c \to a \in {\rm pr}_{1}\langle c \rangle \wedge b \in {\rm pr}_{2}\langle c \rangle
\]
が成り立つ.
\end{thm}


\noindent{\bf 証明}
~$x$と$y$を, 共に$a$, $b$, $c$のいずれの記号列の中にも自由変数として現れない文字とすれば, 
schema S4の適用により
\[
  (b|y)((a, y) \in c) \to \exists y((a, y) \in c), ~~
  (a|x)((x, b) \in c) \to \exists x((x, b) \in c)
\]
が共に成り立つが, 
代入法則 \ref{substfree}, \ref{substfund}, \ref{substpair}により
$(b|y)((a, y) \in c)$と$(a|x)((x, b) \in c)$は共に$(a, b) \in c$と一致するから, 
上記の記号列はそれぞれ
\begin{align*}
  \tag{1}
  (a, b) \in c &\to \exists y((a, y) \in c), \\
  \mbox{} \\
  \tag{2}
  (a, b) \in c &\to \exists x((x, b) \in c)
\end{align*}
と一致する.
よってこれらが共に定理となる.
また$x$と$y$に対する仮定から, 定理 \ref{sthmprsetelement}と推論法則 \ref{dedequiv}により
\begin{align*}
  \tag{3}
  \exists y((a, y) \in c) &\to a \in {\rm pr}_{1}\langle c \rangle, \\
  \mbox{} \\
  \tag{4}
  \exists x((x, b) \in c) &\to b \in {\rm pr}_{2}\langle c \rangle
\end{align*}
が共に成り立つ.
そこで(1)と(3), (2)と(4)から, それぞれ推論法則 \ref{dedmmp}によって
\[
  (a, b) \in c \to a \in {\rm pr}_{1}\langle c \rangle, ~~
  (a, b) \in c \to b \in {\rm pr}_{2}\langle c \rangle
\]
が共に成り立ち, これらから, 推論法則 \ref{dedprewedge}によって
\[
  (a, b) \in c \to a \in {\rm pr}_{1}\langle c \rangle \wedge b \in {\rm pr}_{2}\langle c \rangle
\]
が成り立つ.
\halmos




\mathstrut
\begin{thm}
\label{sthmgraphprset}%定理
$a$を集合とするとき, 
\[
  {\rm Graph}(a) \leftrightarrow a \subset {\rm pr}_{1}\langle a \rangle \times {\rm pr}_{2}\langle a \rangle
\]
が成り立つ.
またこのことから, 次の($*$)が成り立つ: 

($*$) ~~$a$がグラフならば, $a \subset {\rm pr}_{1}\langle a \rangle \times {\rm pr}_{2}\langle a \rangle$が成り立つ.
\end{thm}


\noindent{\bf 証明}
~推論法則 \ref{dedequiv}があるから, 
${\rm Graph}(a) \to a \subset {\rm pr}_{1}\langle a \rangle \times {\rm pr}_{2}\langle a \rangle$と
$a \subset {\rm pr}_{1}\langle a \rangle \times {\rm pr}_{2}\langle a \rangle \to {\rm Graph}(a)$が
共に成り立つことを示せば良いが, 
この後者は定理 \ref{sthmproductsubsetgraph}によって成り立つから, 
前者が成り立つことのみを示せば良い.

いま$x$を$a$の中に自由変数として現れない文字とし, 
$\tau_{x}(\neg (x \in a \to x \in {\rm pr}_{1}\langle a \rangle \times {\rm pr}_{2}\langle a \rangle))$を
$T$と書く.
このとき$T$は集合であり, 定理 \ref{sthmgraphbasis}より
\[
  {\rm Graph}(a) \to (T \in a \to {\rm Pair}(T))
\]
が成り立つ.
そこで推論法則 \ref{dedtwch}により
\[
\tag{1}
  {\rm Graph}(a) \wedge T \in a \to {\rm Pair}(T)
\]
が成り立つ.
また定理 \ref{sthmbigpairpr}と推論法則 \ref{dedequiv}により
\[
\tag{2}
  {\rm Pair}(T) \to T = ({\rm pr}_{1}(T), {\rm pr}_{2}(T))
\]
が成り立つ.
そこで(1), (2)から, 推論法則 \ref{dedmmp}によって
\[
\tag{3}
  {\rm Graph}(a) \wedge T \in a \to T = ({\rm pr}_{1}(T), {\rm pr}_{2}(T))
\]
が成り立つ.
またThm \ref{awbta}より
\[
\tag{4}
  {\rm Graph}(a) \wedge T \in a \to T \in a
\]
が成り立つ.
そこで(3), (4)から, 推論法則 \ref{dedprewedge}によって
\[
\tag{5}
  {\rm Graph}(a) \wedge T \in a \to T = ({\rm pr}_{1}(T), {\rm pr}_{2}(T)) \wedge T \in a
\]
が成り立つ.
また定理 \ref{sthm=&in}より
\[
\tag{6}
  T = ({\rm pr}_{1}(T), {\rm pr}_{2}(T)) \wedge T \in a \to ({\rm pr}_{1}(T), {\rm pr}_{2}(T)) \in a
\]
が成り立つ.
また定理 \ref{sthmpairelementinprset}より
\[
\tag{7}
  ({\rm pr}_{1}(T), {\rm pr}_{2}(T)) \in a \to 
  {\rm pr}_{1}(T) \in {\rm pr}_{1}\langle a \rangle \wedge {\rm pr}_{2}(T) \in {\rm pr}_{2}\langle a \rangle
\]
が成り立つ.
また定理 \ref{sthmpairinproduct}と推論法則 \ref{dedequiv}により
\[
\tag{8}
  {\rm pr}_{1}(T) \in {\rm pr}_{1}\langle a \rangle \wedge {\rm pr}_{2}(T) \in {\rm pr}_{2}\langle a \rangle \to 
  ({\rm pr}_{1}(T), {\rm pr}_{2}(T)) \in {\rm pr}_{1}\langle a \rangle \times {\rm pr}_{2}\langle a \rangle
\]
が成り立つ.
そこで(5)---(8)から, 推論法則 \ref{dedmmp}によって
\[
  {\rm Graph}(a) \wedge T \in a \to ({\rm pr}_{1}(T), {\rm pr}_{2}(T)) \in {\rm pr}_{1}\langle a \rangle \times {\rm pr}_{2}\langle a \rangle
\]
が成り立ち, これと(3)から, 推論法則 \ref{dedprewedge}によって
\[
\tag{9}
  {\rm Graph}(a) \wedge T \in a \to 
  T = ({\rm pr}_{1}(T), {\rm pr}_{2}(T)) \wedge ({\rm pr}_{1}(T), {\rm pr}_{2}(T)) \in {\rm pr}_{1}\langle a \rangle \times {\rm pr}_{2}\langle a \rangle
\]
が成り立つ.
また定理 \ref{sthm=&in}より
\[
\tag{10}
  T = ({\rm pr}_{1}(T), {\rm pr}_{2}(T)) \wedge ({\rm pr}_{1}(T), {\rm pr}_{2}(T)) \in {\rm pr}_{1}\langle a \rangle \times {\rm pr}_{2}\langle a \rangle \to 
  T \in {\rm pr}_{1}\langle a \rangle \times {\rm pr}_{2}\langle a \rangle
\]
が成り立つ.
そこで(9), (10)から, 推論法則 \ref{dedmmp}によって
\[
  {\rm Graph}(a) \wedge T \in a \to T \in {\rm pr}_{1}\langle a \rangle \times {\rm pr}_{2}\langle a \rangle
\]
が成り立ち, これから推論法則 \ref{dedtwch}によって
\[
\tag{11}
  {\rm Graph}(a) \to (T \in a \to T \in {\rm pr}_{1}\langle a \rangle \times {\rm pr}_{2}\langle a \rangle)
\]
が成り立つ.
また$T$の定義から, Thm \ref{thmallfund1}と推論法則 \ref{dedequiv}により
\[
  (T|x)(x \in a \to x \in {\rm pr}_{1}\langle a \rangle \times {\rm pr}_{2}\langle a \rangle) \to 
  \forall x(x \in a \to x \in {\rm pr}_{1}\langle a \rangle \times {\rm pr}_{2}\langle a \rangle)
\]
が成り立つが, 
いま$x$は$a$の中に自由変数として現れず, 
従って変数法則 \ref{valproduct}, \ref{valprset}により$x$は
${\rm pr}_{1}\langle a \rangle \times {\rm pr}_{2}\langle a \rangle$の中にも自由変数として現れないから, 
代入法則 \ref{substfree}, \ref{substfund}及び定義によれば, この記号列は
\[
\tag{12}
  (T \in a \to T \in {\rm pr}_{1}\langle a \rangle \times {\rm pr}_{2}\langle a \rangle) \to 
  a \subset {\rm pr}_{1}\langle a \rangle \times {\rm pr}_{2}\langle a \rangle
\]
と一致する.
よってこれが定理となる.
そこで(11), (12)から, 推論法則 \ref{dedmmp}によって
\[
  {\rm Graph}(a) \to a \subset {\rm pr}_{1}\langle a \rangle \times {\rm pr}_{2}\langle a \rangle
\]
が成り立つ.
($*$)が成り立つことは, これと推論法則 \ref{dedmp}によって明らかである.
\halmos




\mathstrut
\begin{thm}
\label{sthmgrapheqproductsubset}%定理
$a$を集合とする.
また$x$と$y$を, 互いに異なり, 共に$a$の中に自由変数として現れない文字とする.
このとき
\[
  {\rm Graph}(a) \leftrightarrow \exists x(\exists y(a \subset x \times y))
\]
が成り立つ.
\end{thm}


\noindent{\bf 証明}
~推論法則 \ref{dedequiv}があるから, 
\begin{align*}
  \tag{1}
  &{\rm Graph}(a) \to \exists x(\exists y(a \subset x \times y)), \\
  \mbox{} \\
  \tag{2}
  &\exists x(\exists y(a \subset x \times y)) \to {\rm Graph}(a)
\end{align*}
が共に成り立つことを示せば良い.

(1)の証明: 
定理 \ref{sthmgraphprset}と推論法則 \ref{dedequiv}により
\[
  {\rm Graph}(a) \to a \subset {\rm pr}_{1}\langle a \rangle \times {\rm pr}_{2}\langle a \rangle
\]
が成り立つが, いま$y$は$a$の中に自由変数として現れず, 
従って変数法則 \ref{valprset}により, $y$は${\rm pr}_{1}\langle a \rangle$の中にも
自由変数として現れないから, 代入法則 \ref{substfree}, \ref{substsubset}, \ref{substproduct}により, 
この記号列は
\[
\tag{3}
  {\rm Graph}(a) \to ({\rm pr}_{2}\langle a \rangle|y)(a \subset {\rm pr}_{1}\langle a \rangle \times y)
\]
と一致する.
よってこれが定理となる.
またschema S4の適用により
\[
  ({\rm pr}_{2}\langle a \rangle|y)(a \subset {\rm pr}_{1}\langle a \rangle \times y) \to 
  \exists y(a \subset {\rm pr}_{1}\langle a \rangle \times y)
\]
が成り立つ.
ここで$x$が$y$と異なり, $a$の中に自由変数として現れないことから, 
代入法則 \ref{substfree}, \ref{substsubset}, \ref{substproduct}により, 
上記の記号列は
\[
  ({\rm pr}_{2}\langle a \rangle|y)(a \subset {\rm pr}_{1}\langle a \rangle \times y) \to 
  \exists y(({\rm pr}_{1}\langle a \rangle|x)(a \subset x \times y))
\]
と一致する.
また$y$が$x$と異なり, 上述のように${\rm pr}_{1}\langle a \rangle$の中に
自由変数として現れないことから, 代入法則 \ref{substquan}により, この記号列は
\[
\tag{4}
  ({\rm pr}_{2}\langle a \rangle|y)(a \subset {\rm pr}_{1}\langle a \rangle \times y) \to 
  ({\rm pr}_{1}\langle a \rangle|x)(\exists y(a \subset x \times y))
\]
と一致する.
よってこれが定理となる.
また再びschema S4の適用により
\[
\tag{5}
  ({\rm pr}_{1}\langle a \rangle|x)(\exists y(a \subset x \times y)) \to 
  \exists x(\exists y(a \subset x \times y))
\]
が成り立つ.
そこで(3), (4), (5)から, 推論法則 \ref{dedmmp}によって
(1)が成り立つことがわかる.

(2)の証明: 
$\tau_{x}(\exists y(a \subset x \times y))$を$T$と書けば, 
$T$は集合であり, 変数法則 \ref{valtau}, \ref{valquan}からわかるように, 
$y$はこの中に自由変数として現れない.
そして定義から, 
\[
\tag{6}
  \exists x(\exists y(a \subset x \times y)) \equiv (T|x)(\exists y(a \subset x \times y))
\]
である.
また$y$が$x$と異なり, いま述べたように$T$の中に自由変数として現れないことから, 
代入法則 \ref{substquan}により, 
\[
\tag{7}
  (T|x)(\exists y(a \subset x \times y)) \equiv \exists y((T|x)(a \subset x \times y))
\]
が成り立つ.
また$x$が$y$と異なり, $a$の中に自由変数として現れないことから, 
代入法則 \ref{substfree}, \ref{substsubset}, \ref{substproduct}により, 
\[
\tag{8}
  \exists y((T|x)(a \subset x \times y)) \equiv \exists y(a \subset T \times y)
\]
が成り立つ.
またいま$\tau_{y}(a \subset T \times y)$を$U$と書けば, $U$は集合であり, 
定義から, 
\[
\tag{9}
  \exists y(a \subset T \times y) \equiv (U|y)(a \subset T \times y)
\]
である.
また$y$が$a$の中に自由変数として現れず, 上述のように$T$の中にも
自由変数として現れないことから, 
代入法則 \ref{substfree}, \ref{substsubset}, \ref{substproduct}により, 
\[
\tag{10}
  (U|y)(a \subset T \times y) \equiv a \subset T \times U
\]
が成り立つ.
以上の(6)---(10)から, 
\[
\tag{11}
  \exists x(\exists y(a \subset x \times y)) \equiv a \subset T \times U
\]
が成り立つことがわかる.
いま定理 \ref{sthmproductsubsetgraph}より
\[
  a \subset T \times U \to {\rm Graph}(a)
\]
が成り立つが, (11)により, これは(2)と一致する.
故に(2)が成り立つ.
\halmos




\mathstrut
この定理からわかるように, $a$がグラフであるとは, 
$a$がある積の部分集合であるということと同じことである.




\mathstrut
\begin{defo}
\label{valueset}%変形
$a$と$b$を記号列とする.
また$x$と$y$を, 互いに異なり, 共に$a$及び$b$の中に自由変数として現れない文字とする.
同様に, $z$と$w$を, 互いに異なり, 共に$a$及び$b$の中に自由変数として現れない文字とする.
このとき
\[
  \{y|\exists x(x \in b \wedge (x, y) \in a)\} \equiv \{w|\exists z(z \in b \wedge (z, w) \in a)\}
\]
が成り立つ.
\end{defo}


\noindent{\bf 証明}
~$u$と$v$を, 互いに異なり, 共に$x$, $y$, $z$, $w$のいずれとも異なり, 
$a$及び$b$の中に自由変数として現れない文字とする.
このとき変数法則 \ref{valfund}, \ref{valwedge}, \ref{valpair}からわかるように, 
$u$は$x \in b \wedge (x, y) \in a$の中に自由変数として現れない.
そこで代入法則 \ref{substquantrans}により
\[
\tag{1}
  \exists x(x \in b \wedge (x, y) \in a) \equiv \exists u((u|x)(x \in b \wedge (x, y) \in a))
\]
が成り立つ.
また$x$が$y$と異なり, $a$及び$b$の中に自由変数として現れないことから, 
代入法則 \ref{substfree}, \ref{substfund}, \ref{substwedge}, \ref{substpair}により
\[
\tag{2}
  (u|x)(x \in b \wedge (x, y) \in a) \equiv u \in b \wedge (u, y) \in a
\]
が成り立つ.
そこで(1)と(2)から, 
\[
\tag{3}
  \{y|\exists x(x \in b \wedge (x, y) \in a)\} \equiv \{y|\exists u(u \in b \wedge (u, y) \in a)\}
\]
が成り立つことがわかる.
また$v$が$y$とも$u$とも異なり, $a$及び$b$の中に自由変数として現れないことから, 
変数法則 \ref{valfund}, \ref{valwedge}, \ref{valquan}, \ref{valpair}によって
$v$が$\exists u(u \in b \wedge (u, y) \in a)$の中に自由変数として現れないことがわかるから, 
代入法則 \ref{substisettrans}により
\[
\tag{4}
  \{y|\exists u(u \in b \wedge (u, y) \in a)\} \equiv \{v|(v|y)(\exists u(u \in b \wedge (u, y) \in a))\}
\]
が成り立つ.
また$u$が$y$とも$v$とも異なることから, 代入法則 \ref{substquan}により
\[
\tag{5}
  (v|y)(\exists u(u \in b \wedge (u, y) \in a)) \equiv \exists u((v|y)(u \in b \wedge (u, y) \in a))
\]
が成り立つ.
また$y$が$u$と異なり, $a$及び$b$の中に自由変数として現れないことから, 
代入法則 \ref{substfree}, \ref{substfund}, \ref{substwedge}, \ref{substpair}により
\[
\tag{6}
  (v|y)(u \in b \wedge (u, y) \in a) \equiv u \in b \wedge (u, v) \in a
\]
が成り立つ.
そこで(4), (5), (6)から, 
\[
\tag{7}
  \{y|\exists u(u \in b \wedge (u, y) \in a)\} \equiv \{v|\exists u(u \in b \wedge (u, v) \in a)\}
\]
が成り立つことがわかる.
そこで(3)と(7)からわかるように, $\{y|\exists x(x \in b \wedge (x, y) \in a)\}$は
$\{v|\exists u(u \in b \wedge (u, v) \in a)\}$と一致する.
ここまでの議論と全く同様にして, $\{w|\exists z(z \in b \wedge (z, w) \in a)\}$も
$\{v|\exists u(u \in b \wedge (u, v) \in a)\}$と一致する.
故に本法則が成り立つ.
\halmos




\mathstrut
\begin{defi}
\label{defvalueset}%定義
$a$と$b$を記号列とする.
また$x$と$y$を, 互いに異なり, 共に$a$及び$b$の中に自由変数として現れない文字とする.
同様に, $z$と$w$を, 互いに異なり, 共に$a$及び$b$の中に自由変数として現れない文字とする.
このとき上記の変形法則 \ref{valueset}によれば, 
$\{y|\exists x(x \in b \wedge (x, y) \in a)\}$と
$\{w|\exists z(z \in b \wedge (z, w) \in a)\}$という
二つの記号列は一致する.
$a$と$b$に対して定まるこの記号列を, 
$(a)[b]$と書き表す.
括弧を省略して単に$a[b]$と書き表すことも多い.
\end{defi}




\mathstrut
\begin{valu}
\label{valvalueset}%変数
$a$と$b$を記号列とし, $x$を文字とする.
$x$が$a$及び$b$の中に自由変数として現れなければ, 
$x$は$a[b]$の中に自由変数として現れない.
\end{valu}


\noindent{\bf 証明}
~$y$を$x$と異なり, $a$及び$b$の中に自由変数として現れない文字とすれば, 
定義から$a[b]$は$\{x|\exists y(y \in b \wedge (y, x) \in a)\}$と同じだから, 
変数法則 \ref{valiset}により, $x$はこの中に自由変数として現れない.
\halmos




\mathstrut
\begin{subs}
\label{substvalueset}%代入
$a$, $b$, $c$を記号列とし, $x$を文字とするとき, 
\[
  (c|x)(a[b]) \equiv (c|x)(a)[(c|x)(b)]
\]
が成り立つ.
\end{subs}


\noindent{\bf 証明}
~$u$と$v$を, 互いに異なり, 共に$x$と異なり, 
$a$, $b$, $c$のいずれの記号列の中にも自由変数として現れない文字とする.
このとき定義から$a[b]$は$\{v|\exists u(u \in b \wedge (u, v) \in a)\}$と
同じである.
そこで$v$が$x$と異なり, $c$の中に自由変数として現れないということから, 
代入法則 \ref{substiset}により
\[
  (c|x)(a[b]) \equiv \{v|(c|x)(\exists u(u \in b \wedge (u, v) \in a))\}
\]
が成り立つ.
また$u$も$x$と異なり, $c$の中に自由変数として現れないから, 
代入法則 \ref{substquan}により
\[
  (c|x)(\exists u(u \in b \wedge (u, v) \in a)) \equiv \exists u((c|x)(u \in b \wedge (u, v) \in a))
\]
が成り立つ.
また$x$が$u$とも$v$とも異なることと
代入法則 \ref{substfund}, \ref{substwedge}, \ref{substpair}により, 
\[
  (c|x)(u \in b \wedge (u, v) \in a) \equiv u \in (c|x)(b) \wedge (u, v) \in (c|x)(a)
\]
が成り立つ.
以上から, $(c|x)(a[b])$が
\[
\tag{$*$}
  \{v|\exists u(u \in (c|x)(b) \wedge (u, v) \in (c|x)(a))\}
\]
と一致することがわかる.
いま$u$と$v$が共に$a$, $b$, $c$のいずれの記号列の中にも自由変数として現れないことから, 
変数法則 \ref{valsubst}により 
$u$と$v$は共に$(c|x)(a)$及び$(c|x)(b)$の中にも自由変数として現れない.
また$u$と$v$は異なる文字である.
そこで定義から, ($*$)は$(c|x)(a)[(c|x)(b)]$と同じである.
故に$(c|x)(a[b])$は$(c|x)(a)[(c|x)(b)]$と一致する.
\halmos




\mathstrut
\begin{form}
\label{formvalueset}%構成
$a$と$b$が集合ならば, $a[b]$は集合である.
\end{form}


\noindent{\bf 証明}
~$x$と$y$を, 互いに異なり, 共に$a$及び$b$の中に自由変数として現れない文字とすれば, 
定義から$a[b]$は$\{y|\exists x(x \in b \wedge (x, y) \in a)\}$と同じである.
そこで$a$と$b$が集合ならば, 
構成法則 \ref{formfund}, \ref{formwedge}, \ref{formquan}, \ref{formiset}, \ref{formpair}によって
直ちにわかるように, $a[b]$は集合である.
\halmos




\mathstrut
$a$と$b$を集合とするとき, 集合$a[b]$を, $a$による$b$の\textbf{像}という.




\mathstrut
\begin{thm}
\label{sthmvaluesetmake}%定理
$a$と$b$を集合とする.
また$x$と$y$を, 互いに異なり, 共に$a$及び$b$の中に自由変数として現れない文字とする.
このとき, 関係式$\exists x(x \in b \wedge (x, y) \in a)$は$y$について集合を作り得る.
\end{thm}


\noindent{\bf 証明}
~$u$と$v$を, 共に$x$とも$y$とも異なり, $a$の中に自由変数として現れない文字とする.
また$\tau_{x}(\neg (\exists u(\forall y((x, y) \in a \to y \in u))))$を$T$と書く.
このとき$T$は集合であり, 変数法則 \ref{valfund}, \ref{valtau}, \ref{valquan}によってわかるように, 
$y$と$u$は共に$T$の中に自由変数として現れない.
また$\tau_{y}(\neg ((T, y) \in a \to y \in {\rm pr}_{2}\langle a \rangle))$を$U$と書けば, 
$U$も集合である.
そして定理 \ref{sthmpairelementinprset}より
\[
  (T, U) \in a \to T \in {\rm pr}_{1}\langle a \rangle \wedge U \in {\rm pr}_{2}\langle a \rangle
\]
が成り立つ.
またThm \ref{awbta}より
\[
  T \in {\rm pr}_{1}\langle a \rangle \wedge U \in {\rm pr}_{2}\langle a \rangle
  \to U \in {\rm pr}_{2}\langle a \rangle
\]
が成り立つ.
そこでこれらから, 推論法則 \ref{dedmmp}によって
\[
  (T, U) \in a \to U \in {\rm pr}_{2}\langle a \rangle
\]
が成り立つ.
いま$y$は$a$の中に自由変数として現れず, 従って
変数法則 \ref{valprset}により, $y$は${\rm pr}_{2}\langle a \rangle$の中にも
自由変数として現れない.
また上述のように, $y$は$T$の中にも自由変数として現れない.
そこで代入法則 \ref{substfree}, \ref{substfund}, \ref{substpair}により, 
上記の記号列は
\[
  (U|y)((T, y) \in a \to y \in {\rm pr}_{2}\langle a \rangle)
\]
と一致する.
よってこれが定理となる.
そこで$U$の定義から, 推論法則 \ref{dedallfund}によって
\[
  \forall y((T, y) \in a \to y \in {\rm pr}_{2}\langle a \rangle)
\]
が成り立つが, いま$u$が$y$と異なり, $a$の中に自由変数として現れず, 
上述のように$T$の中にも自由変数として現れないことから, 
代入法則 \ref{substfree}, \ref{substfund}, \ref{substpair}により, 
この記号列は
\[
  \forall y(({\rm pr}_{2}\langle a \rangle|u)((T, y) \in a \to y \in u))
\]
と一致する.
また$y$は$u$と異なり, 上述のように${\rm pr}_{2}\langle a \rangle$の中に
自由変数として現れないから, 代入法則 \ref{substquan}により, 
この記号列は
\[
  ({\rm pr}_{2}\langle a \rangle|u)(\forall y((T, y) \in a \to y \in u))
\]
と一致する.
よってこれが定理となる.
そこで推論法則 \ref{deds4}により, 
\[
  \exists u(\forall y((T, y) \in a \to y \in u))
\]
が成り立つ.
ここで$x$が$y$とも$u$とも異なり, $a$の中に自由変数として現れないことから, 
代入法則 \ref{substfree}, \ref{substfund}, \ref{substpair}により, 
上記の記号列は
\[
  \exists u(\forall y((T|x)((x, y) \in a \to y \in u)))
\]
と一致する.
また$y$と$u$は共に$x$と異なり, 上述のように$T$の中に自由変数として現れないから, 
代入法則 \ref{substquan}により, この記号列は
\[
  (T|x)(\exists u(\forall y((x, y) \in a \to y \in u)))
\]
と一致する.
よってこれが定理となる.
そこで$T$の定義から, 推論法則 \ref{dedallfund}によって
\[
\tag{1}
  \forall x(\exists u(\forall y((x, y) \in a \to y \in u)))
\]
が成り立つ.
さていま仮定より$x$と$y$は互いに異なる文字である.
また$u$と$v$は共に$x$とも$y$とも異なり, $a$の中に自由変数として現れないから, 
変数法則 \ref{valfund}, \ref{valpair}によってわかるように, 
これらは関係式$(x, y) \in a$の中にも自由変数として現れない.
そこでschema S7の適用により, 
\[
\tag{2}
  \forall x(\exists u(\forall y((x, y) \in a \to y \in u))) \to 
  \forall v({\rm Set}_{y}(\exists x(x \in v \wedge (x, y) \in a)))
\]
が成り立つ.
そこで(1), (2)から, 推論法則 \ref{dedmp}によって
\[
  \forall v({\rm Set}_{y}(\exists x(x \in v \wedge (x, y) \in a)))
\]
が成り立ち, これから推論法則 \ref{dedfromallthm}によって
\[
  (b|v)({\rm Set}_{y}(\exists x(x \in v \wedge (x, y) \in a)))
\]
が成り立つ.
ここで$y$が$v$と異なり, $b$の中に自由変数として現れないことから, 
代入法則 \ref{substsm}により, 上記の記号列は
\[
  {\rm Set}_{y}((b|v)(\exists x(x \in v \wedge (x, y) \in a)))
\]
と一致する.
また$x$も$v$と異なり, $b$の中に自由変数として現れないから, 
代入法則 \ref{substquan}により, この記号列は
\[
  {\rm Set}_{y}(\exists x((b|v)(x \in v \wedge (x, y) \in a)))
\]
と一致する.
更に, $v$が$x$と異なり, 上述のように$(x, y) \in a$の中に自由変数として現れないことから, 
代入法則 \ref{substfree}, \ref{substwedge}により, この記号列は
\[
  {\rm Set}_{y}(\exists x(x \in b \wedge (x, y) \in a))
\]
と一致する.
よってこれが定理となる.
言い換えれば, $\exists x(x \in b \wedge (x, y) \in a)$という関係式は, 
$y$について集合を作り得る.
\halmos




\mathstrut
\begin{thm}
\label{sthmvaluesetelement}%定理
$a$, $b$, $c$を集合とし, $x$をこれらの中に自由変数として現れない文字とする.
このとき
\[
  c \in a[b] \leftrightarrow \exists x(x \in b \wedge (x, c) \in a)
\]
が成り立つ.
\end{thm}


\noindent{\bf 証明}
~$y$を, $x$と異なり, $a$及び$b$の中に自由変数として現れない文字とする.
このとき定義から, $a[b]$は$\{y|\exists x(x \in b \wedge (x, y) \in a)\}$と同じである.
また定理 \ref{sthmvaluesetmake}より, $\exists x(x \in b \wedge (x, y) \in a)$は$y$について集合を作り得る.
そこで定理 \ref{sthmisetbasis}より
\[
\tag{1}
  c \in a[b] \leftrightarrow (c|y)(\exists x(x \in b \wedge (x, y) \in a))
\]
が成り立つ.
いま$x$が$y$と異なり, $c$の中に自由変数として現れないことから, 
代入法則 \ref{substquan}により
\[
\tag{2}
  (c|y)(\exists x(x \in b \wedge (x, y) \in a)) \equiv \exists x((c|y)(x \in b \wedge (x, y) \in a))
\]
が成り立つ.
また$y$が$x$と異なり, $a$及び$b$の中に自由変数として現れないことから, 
代入法則 \ref{substfree}, \ref{substfund}, \ref{substwedge}, \ref{substpair}により
\[
\tag{3}
  (c|y)(x \in b \wedge (x, y) \in a) \equiv x \in b \wedge (x, c) \in a
\]
が成り立つ.
よって(2)と(3)から, (1)が
\[
  c \in a[b] \leftrightarrow \exists x(x \in b \wedge (x, c) \in a)
\]
と一致することがわかり, これが定理となる.
\halmos




\mathstrut
次は上記の定理 \ref{sthmvaluesetelement}から直ちに得られる定理であるが, 
以降の定理の証明の中でしばしば引用する.




\mathstrut
\begin{thm}
\label{sthmvaluesetbasis}%定理
$a$, $b$, $T$, $U$を集合とするとき, 
\[
  T \in b \wedge (T, U) \in a \to U \in a[b]
\]
が成り立つ.
\end{thm}


\noindent{\bf 証明}
~$x$を$a$, $b$, $U$のいずれの記号列の中にも自由変数として現れない文字とする.
このときschema S4の適用により
\[
  (T|x)(x \in b \wedge (x, U) \in a) \to \exists x(x \in b \wedge (x, U) \in a)
\]
が成り立つが, 代入法則 \ref{substfree}, \ref{substfund}, \ref{substwedge}, \ref{substpair}によれば, 
この記号列は
\[
\tag{1}
  T \in b \wedge (T, U) \in a \to \exists x(x \in b \wedge (x, U) \in a)
\]
と一致するから, これが定理となる.
また$x$に対する仮定から, 定理 \ref{sthmvaluesetelement}と推論法則 \ref{dedequiv}により
\[
\tag{2}
  \exists x(x \in b \wedge (x, U) \in a) \to U \in a[b]
\]
が成り立つ.
そこで(1), (2)から, 推論法則 \ref{dedmmp}によって
$T \in b \wedge (T, U) \in a \to U \in a[b]$が成り立つ.
\halmos




\mathstrut
\begin{thm}
\label{sthmvaluesetsubset}%定理
\mbox{}

1)
$a$, $b$, $c$を集合とするとき, 
\[
  a \subset b \to a[c] \subset b[c], ~~
  a \subset b \to c[a] \subset c[b]
\]
が成り立つ.
またこのことから, 次の($*$)が成り立つ: 

($*$) ~~$a \subset b$が成り立つならば, $a[c] \subset b[c]$と$c[a] \subset c[b]$が共に成り立つ.

2)
$a$, $b$, $c$, $d$を集合とするとき, 
\[
  a \subset c \wedge b \subset d \to a[b] \subset c[d]
\]
が成り立つ.
またこのことから, 次の($**$)が成り立つ: 

($**$) ~~$a \subset c$と$b \subset d$が共に成り立つならば, $a[b] \subset c[d]$が成り立つ.
\end{thm}


\noindent{\bf 証明}
~1)
$x$を$a$, $b$, $c$のいずれの記号列の中にも自由変数として現れない文字とする.
このとき変数法則 \ref{valvalueset}により, $x$は
$a[c]$, $b[c]$, $c[a]$, $c[b]$のいずれの記号列の中にも自由変数として現れない.
また$\tau_{x}(\neg (x \in a[c] \to x \in b[c]))$を$T$と書き, 
$\tau_{x}(\neg (x \in c[a] \to x \in c[b]))$を$U$と書けば, これらは共に集合であり, 
変数法則 \ref{valtau}により, $x$はこれらの中にも自由変数として現れない.
これらのことから, 特に$x$は$a$, $c$, $T$, $U$の中に自由変数として現れないから, 
定理 \ref{sthmvaluesetelement}と推論法則 \ref{dedequiv}により
\begin{align*}
  T \in a[c] &\to \exists x(x \in c \wedge (x, T) \in a), \\
  \mbox{} \\
  U \in c[a] &\to \exists x(x \in a \wedge (x, U) \in c)
\end{align*}
が共に成り立つ.
ここで$\tau_{x}(x \in c \wedge (x, T) \in a)$を$V$と書き, 
$\tau_{x}(x \in a \wedge (x, U) \in c)$を$W$と書けば, 
これらは共に集合であり, 定義から上記の記号列はそれぞれ
\begin{align*}
  T \in a[c] &\to (V|x)(x \in c \wedge (x, T) \in a), \\
  \mbox{} \\
  U \in c[a] &\to (W|x)(x \in a \wedge (x, U) \in c)
\end{align*}
である.
また上述のように$x$は$a$, $c$, $T$, $U$の中に自由変数として現れないから, 
代入法則 \ref{substfree}, \ref{substfund}, \ref{substwedge}, \ref{substpair}により, 
これらの記号列はそれぞれ
\begin{align*}
  T \in a[c] &\to V \in c \wedge (V, T) \in a, \\
  \mbox{} \\
  U \in c[a] &\to W \in a \wedge (W, U) \in c
\end{align*}
と一致する.
よってこれらが共に定理となる.
そこで推論法則 \ref{dedaddw}により, 
\begin{align*}
  \tag{1}
  a \subset b \wedge T \in a[c] &\to a \subset b \wedge (V \in c \wedge (V, T) \in a), \\
  \mbox{} \\
  \tag{2}
  a \subset b \wedge U \in c[a] &\to a \subset b \wedge (W \in a \wedge (W, U) \in c)
\end{align*}
が共に成り立つ.
また定理 \ref{sthmsubsetbasis}より
\begin{align*}
  a \subset b &\to ((V, T) \in a \to (V, T) \in b), \\
  \mbox{} \\
  a \subset b &\to (W \in a \to W \in b)
\end{align*}
が共に成り立ち, 
Thm \ref{1atb1t1awctbwc1}より
\begin{align*}
  ((V, T) \in a \to (V, T) \in b) &\to (V \in c \wedge (V, T) \in a \to V \in c \wedge (V, T) \in b), \\
  \mbox{} \\
  (W \in a \to W \in b) &\to (W \in a \wedge (W, U) \in c \to W \in b \wedge (W, U) \in c)
\end{align*}
が共に成り立つから, これらから, 推論法則 \ref{dedmmp}によって
\begin{align*}
  a \subset b &\to (V \in c \wedge (V, T) \in a \to V \in c \wedge (V, T) \in b), \\
  \mbox{} \\
  a \subset b &\to (W \in a \wedge (W, U) \in c \to W \in b \wedge (W, U) \in c)
\end{align*}
が共に成り立つ.
そこで推論法則 \ref{dedtwch}により, 
\begin{align*}
  \tag{3}
  a \subset b \wedge (V \in c \wedge (V, T) \in a) &\to V \in c \wedge (V, T) \in b, \\
  \mbox{} \\
  \tag{4}
  a \subset b \wedge (W \in a \wedge (W, U) \in c) &\to W \in b \wedge (W, U) \in c
\end{align*}
が共に成り立つ.
また定理 \ref{sthmvaluesetbasis}より
\begin{align*}
  \tag{5}
  V \in c \wedge (V, T) \in b &\to T \in b[c], \\
  \mbox{} \\
  \tag{6}
  W \in b \wedge (W, U) \in c &\to U \in c[b]
\end{align*}
が共に成り立つ.
そこで(1), (3), (5)から, 推論法則 \ref{dedmmp}によって
\[
  a \subset b \wedge T \in a[c] \to T \in b[c]
\]
が成り立ち, (2), (4), (6)から, 同じく推論法則 \ref{dedmmp}によって
\[
  a \subset b \wedge U \in c[a] \to U \in c[b]
\]
が成り立つことがわかる.
そこでこれらから, 推論法則 \ref{dedtwch}によって
\begin{align*}
  \tag{7}
  a \subset b &\to (T \in a[c] \to T \in b[c]), \\
  \mbox{} \\
  \tag{8}
  a \subset b &\to (U \in c[a] \to U \in c[b])
\end{align*}
が共に成り立つ.
また$T$, $U$の定義から, Thm \ref{thmallfund1}と推論法則 \ref{dedequiv}により
\begin{align*}
  (T|x)(x \in a[c] \to x \in b[c]) \to \forall x(x \in a[c] \to x \in b[c]), \\
  \mbox{} \\
  (U|x)(x \in c[a] \to x \in c[b]) \to \forall x(x \in c[a] \to x \in c[b])
\end{align*}
が共に成り立つが, 上述のように$x$は
$a[c]$, $b[c]$, $c[a]$, $c[b]$のいずれの記号列の中にも自由変数として現れないから, 
代入法則 \ref{substfree}, \ref{substfund}及び定義によれば, これらの記号列はそれぞれ
\begin{align*}
  \tag{9}
  (T \in a[c] \to T \in b[c]) &\to a[c] \subset b[c], \\
  \mbox{} \\
  \tag{10}
  (U \in c[a] \to U \in c[b]) &\to c[a] \subset c[b]
\end{align*}
と一致する.
よってこれらが共に定理となる.
そこで(7)と(9), (8)と(10)にそれぞれ推論法則 \ref{dedmmp}を適用して, 
\[
  a \subset b \to a[c] \subset b[c], ~~
  a \subset b \to c[a] \subset c[b]
\]
が共に成り立つ.
($*$)が成り立つことは, これらと推論法則 \ref{dedmp}によって明らかである.

\noindent
2)
1)より
\[
  a \subset c \to a[b] \subset c[b], ~~
  b \subset d \to c[b] \subset c[d]
\]
が共に成り立つから, 推論法則 \ref{dedfromaddw}によって
\[
\tag{11}
  a \subset c \wedge b \subset d \to a[b] \subset c[b] \wedge c[b] \subset c[d]
\]
が成り立つ.
また定理 \ref{sthmsubsettrans}より
\[
\tag{12}
  a[b] \subset c[b] \wedge c[b] \subset c[d] \to a[b] \subset c[d]
\]
が成り立つ.
そこで(11), (12)から, 推論法則 \ref{dedmmp}によって
\[
\tag{13}
  a \subset c \wedge b \subset d \to a[b] \subset c[d]
\]
が成り立つ.

いま$a \subset c$と$b \subset d$が共に成り立つとすれば, 
推論法則 \ref{dedwedge}によって$a \subset c \wedge b \subset d$が成り立つから, 
これと(13)から, 推論法則 \ref{dedmp}によって
$a[b] \subset c[d]$が成り立つ.
故に($**$)が成り立つ.
\halmos




\mathstrut
\begin{thm}
\label{sthmvalueset=}%定理
\mbox{}

1)
$a$, $b$, $c$を集合とするとき, 
\[
  a = b \to a[c] = b[c], ~~
  a = b \to c[a] = c[b]
\]
が成り立つ.
またこのことから, 次の($*$)が成り立つ: 

($*$) ~~$a = b$が成り立つならば, $a[c] = b[c]$と$c[a] = c[b]$が共に成り立つ.

2)
$a$, $b$, $c$, $d$を集合とするとき, 
\[
  a = c \wedge b = d \to a[b] = c[d]
\]
が成り立つ.
またこのことから, 次の($**$)が成り立つ: 

($**$) ~~$a = c$と$b = d$が共に成り立つならば, $a[b] = c[d]$が成り立つ.
\end{thm}


\noindent{\bf 証明}
~1)
$x$を$c$の中に自由変数として現れない文字とすれば, 
Thm \ref{T=Ut1TV=UV1}より
\[
  a = b \to (a|x)(x[c]) = (b|x)(x[c]), ~~
  a = b \to (a|x)(c[x]) = (b|x)(c[x])
\]
が共に成り立つが, 代入法則 \ref{substfree}, \ref{substvalueset}によれば
これらの記号列はそれぞれ
\[
  a = b \to a[c] = b[c], ~~
  a = b \to c[a] = c[b]
\]
と一致するから, これらが共に定理となる.
($*$)が成り立つことは, これらと推論法則 \ref{dedmp}によって明らかである.

\noindent
2)
1)より
\[
  a = c \to a[b] = c[b], ~~
  b = d \to c[b] = c[d]
\]
が共に成り立つから, これらから, 推論法則 \ref{dedfromaddw}によって
\[
\tag{1}
  a = c \wedge b = d \to a[b] = c[b] \wedge c[b] = c[d]
\]
が成り立つ.
またThm \ref{x=ywy=ztx=z}より
\[
\tag{2}
  a[b] = c[b] \wedge c[b] = c[d] \to a[b] = c[d]
\]
が成り立つ.
そこで(1), (2)から, 推論法則 \ref{dedmmp}によって
\[
\tag{3}
  a = c \wedge b = d \to a[b] = c[d]
\]
が成り立つ.

いま$a = c$と$b = d$が共に成り立つとすれば, 推論法則 \ref{dedwedge}によって
$a = c \wedge b = d$が成り立つから, これと(3)から, 推論法則 \ref{dedmp}によって
$a[b] = c[d]$が成り立つ.
故に($**$)が成り立つ.
\halmos




\mathstrut
\begin{thm}
\label{sthmpairsetofavalueset}%定理
$a$と$b$を集合とし, $x$を$a$の中に自由変数として現れない文字とする.
このとき
\[
  \{x \in a|{\rm Pair}(x)\}[b] = a[b]
\]
が成り立つ.
\end{thm}


\noindent{\bf 証明}
~$y$と$u$を, 互いに異なり, 共に$x$と異なり, $a$及び$b$の中に自由変数として現れない, 
定数でない文字とする.
このとき変数法則 \ref{valsset}, \ref{valbigpair}からわかるように, $u$は$\{x \in a|{\rm Pair}(x)\}$の中にも
自由変数として現れないから, 
定理 \ref{sthmvaluesetelement}より
\[
\tag{1}
  y \in \{x \in a|{\rm Pair}(x)\}[b] \leftrightarrow \exists u(u \in b \wedge (u, y) \in \{x \in a|{\rm Pair}(x)\})
\]
が成り立つ.
また仮定より$x$は$a$の中に自由変数として現れないから, 
定理 \ref{sthmssetbasis}より
\[
  (u, y) \in \{x \in a|{\rm Pair}(x)\} \leftrightarrow (u, y) \in a \wedge ((u, y)|x)({\rm Pair}(x))
\]
が成り立つ.
代入法則 \ref{substbigpair}により, この記号列は
\[
\tag{2}
  (u, y) \in \{x \in a|{\rm Pair}(x)\} \leftrightarrow (u, y) \in a \wedge {\rm Pair}((u, y))
\]
と一致するから, これが定理となる.
また定理 \ref{sthmbigpairpair}より${\rm Pair}((u, y))$が成り立つから, 
推論法則 \ref{dedawblatrue2}により
\[
\tag{3}
  (u, y) \in a \wedge {\rm Pair}((u, y)) \leftrightarrow (u, y) \in a
\]
が成り立つ.
そこで(2), (3)から, 推論法則 \ref{dedeqtrans}によって
\[
  (u, y) \in \{x \in a|{\rm Pair}(x)\} \leftrightarrow (u, y) \in a
\]
が成り立ち, これから推論法則 \ref{dedaddeqw}によって
\[
  u \in b \wedge (u, y) \in \{x \in a|{\rm Pair}(x)\} \leftrightarrow u \in b \wedge (u, y) \in a
\]
が成り立つ.
$u$は定数でないので, これから推論法則 \ref{dedalleqquansepconst}によって
\[
\tag{4}
  \exists u(u \in b \wedge (u, y) \in \{x \in a|{\rm Pair}(x)\}) \leftrightarrow \exists u(u \in b \wedge (u, y) \in a)
\]
が成り立つ.
また$u$が$y$と異なり, $a$及び$b$の中に自由変数として現れないことから, 
定理 \ref{sthmvaluesetelement}と推論法則 \ref{dedeqch}により
\[
\tag{5}
  \exists u(u \in b \wedge (u, y) \in a) \leftrightarrow y \in a[b]
\]
が成り立つ.
そこで(1), (4), (5)から, 推論法則 \ref{dedeqtrans}によって
\[
\tag{6}
  y \in \{x \in a|{\rm Pair}(x)\}[b] \leftrightarrow y \in a[b]
\]
が成り立つことがわかる.
いま$y$は$x$と異なり, $a$及び$b$の中に自由変数として現れないから, 
変数法則 \ref{valsset}, \ref{valbigpair}, \ref{valvalueset}によってわかるように, 
$y$は$\{x \in a|{\rm Pair}(x)\}[b]$及び$a[b]$の中に自由変数として現れない.
また$y$は定数でない.
これらのことと, (6)が成り立つことから, 
定理 \ref{sthmset=}によって$\{x \in a|{\rm Pair}(x)\}[b] = a[b]$が成り立つ.
\halmos




\mathstrut
\begin{thm}
\label{sthmobjectsetvalueset}%定理
$a$, $b$, $T$を集合とし, $x$を$a$及び$b$の中に自由変数として現れない文字とする.
このとき
\[
  \{(x, T)|x \in a\}[b] = \{T|x \in a \cap b\}
\]
が成り立つ.
\end{thm}


\noindent{\bf 証明}
~はじめに$x$が定数でない場合に定理が成り立つことを示す.
$y$と$u$を, 互いに異なり, 共に$x$と異なり, $a$, $b$, $T$のいずれの記号列の中にも自由変数として現れない, 
定数でない文字とする.
このとき変数法則 \ref{valoset}, \ref{valpair}によってわかるように, 
$u$は$\{(x, T)|x \in a\}$の中に自由変数として現れない.
このことと, $u$が$y$と異なり, $b$の中にも自由変数として現れないことから, 
定理 \ref{sthmvaluesetelement}より
\[
\tag{1}
  y \in \{(x, T)|x \in a\}[b] \leftrightarrow \exists u(u \in b \wedge (u, y) \in \{(x, T)|x \in a\})
\]
が成り立つ.
また$x$が$y$とも$u$とも異なることから, 変数法則 \ref{valpair}により, 
$x$は$(u, y)$の中に自由変数として現れない.
このことと, 仮定より$x$が$a$の中に自由変数として現れないことから, 
定理 \ref{sthmosetbasis}より
\[
\tag{2}
  (u, y) \in \{(x, T)|x \in a\} \leftrightarrow \exists x(x \in a \wedge (u, y) = (x, T))
\]
が成り立つ.
また定理 \ref{sthmpair}より
\[
  (u, y) = (x, T) \leftrightarrow u = x \wedge y = T
\]
が成り立つから, 推論法則 \ref{dedaddeqw}により
\[
  x \in a \wedge (u, y) = (x, T) \leftrightarrow x \in a \wedge (u = x \wedge y = T)
\]
が成り立つ.
いま$x$は定数でないとしているから, これから推論法則 \ref{dedalleqquansepconst}によって
\[
\tag{3}
  \exists x(x \in a \wedge (u, y) = (x, T)) \leftrightarrow \exists x(x \in a \wedge (u = x \wedge y = T))
\]
が成り立つ.
そこで(2), (3)から, 推論法則 \ref{dedeqtrans}によって
\[
  (u, y) \in \{(x, T)|x \in a\} \leftrightarrow \exists x(x \in a \wedge (u = x \wedge y = T))
\]
が成り立ち, これから推論法則 \ref{dedaddeqw}によって
\[
\tag{4}
  u \in b \wedge (u, y) \in \{(x, T)|x \in a\} \leftrightarrow u \in b \wedge \exists x(x \in a \wedge (u = x \wedge y = T))
\]
が成り立つ.
いま$x$は$u$と異なり, $b$の中に自由変数として現れないから, 
変数法則 \ref{valfund}により, $x$は$u \in b$の中に自由変数として現れない.
そこでThm \ref{thmexwrfree}と推論法則 \ref{dedeqch}により
\[
\tag{5}
  u \in b \wedge \exists x(x \in a \wedge (u = x \wedge y = T)) \leftrightarrow \exists x(u \in b \wedge (x \in a \wedge (u = x \wedge y = T)))
\]
が成り立つ.
またThm \ref{1awb1wclaw1bwc1}と推論法則 \ref{dedeqch}により
\[
\tag{6}
  u \in b \wedge (x \in a \wedge (u = x \wedge y = T)) \leftrightarrow (u \in b \wedge x \in a) \wedge (u = x \wedge y = T)
\]
が成り立つ.
またThm \ref{awblbwa}より
\[
  u \in b \wedge x \in a \leftrightarrow x \in a \wedge u \in b
\]
が成り立つから, 推論法則 \ref{dedaddeqw}により
\[
\tag{7}
  (u \in b \wedge x \in a) \wedge (u = x \wedge y = T) \leftrightarrow (x \in a \wedge u \in b) \wedge (u = x \wedge y = T)
\]
が成り立つ.
またThm \ref{1awb1wclaw1bwc1}と推論法則 \ref{dedeqch}により
\[
\tag{8}
  (x \in a \wedge u \in b) \wedge (u = x \wedge y = T) \leftrightarrow ((x \in a \wedge u \in b) \wedge u = x) \wedge y = T
\]
が成り立つ.
またThm \ref{1awb1wclaw1bwc1}より
\[
  (x \in a \wedge u \in b) \wedge u = x \leftrightarrow x \in a \wedge (u \in b \wedge u = x)
\]
が成り立つから, 推論法則 \ref{dedaddeqw}により
\[
\tag{9}
  ((x \in a \wedge u \in b) \wedge u = x) \wedge y = T \leftrightarrow (x \in a \wedge (u \in b \wedge u = x)) \wedge y = T
\]
が成り立つ.
またThm \ref{awblbwa}より
\[
  u \in b \wedge u = x \leftrightarrow u = x \wedge u \in b
\]
が成り立つから, 推論法則 \ref{dedaddeqw}を二回用いて
\[
\tag{10}
  (x \in a \wedge (u \in b \wedge u = x)) \wedge y = T \leftrightarrow (x \in a \wedge (u = x \wedge u \in b)) \wedge y = T
\]
が成り立つ.
以上の(6)---(10)から, 推論法則 \ref{dedeqtrans}によって
\[
  u \in b \wedge (x \in a \wedge (u = x \wedge y = T)) \leftrightarrow (x \in a \wedge (u = x \wedge u \in b)) \wedge y = T
\]
が成り立つことがわかる.
いま$x$は定数でないので, これから推論法則 \ref{dedalleqquansepconst}によって
\[
\tag{11}
  \exists x(u \in b \wedge (x \in a \wedge (u = x \wedge y = T))) \leftrightarrow \exists x((x \in a \wedge (u = x \wedge u \in b)) \wedge y = T)
\]
が成り立つ.
そこで(4), (5), (11)から, 推論法則 \ref{dedeqtrans}によって
\[
  u \in b \wedge (u, y) \in \{(x, T)|x \in a\} \leftrightarrow \exists x((x \in a \wedge (u = x \wedge u \in b)) \wedge y = T)
\]
が成り立つ.
いま$u$も定数でないので, これから推論法則 \ref{dedalleqquansepconst}によって
\[
\tag{12}
  \exists u(u \in b \wedge (u, y) \in \{(x, T)|x \in a\}) \leftrightarrow \exists u(\exists x((x \in a \wedge (u = x \wedge u \in b)) \wedge y = T))
\]
が成り立つ.
そこで(1), (12)から, 推論法則 \ref{dedeqtrans}によって
\[
  y \in \{(x, T)|x \in a\}[b] \leftrightarrow \exists u(\exists x((x \in a \wedge (u = x \wedge u \in b)) \wedge y = T))
\]
が成り立ち, これから推論法則 \ref{dedequiv}によって
\begin{align*}
  \tag{13}
  &y \in \{(x, T)|x \in a\}[b] \to \exists u(\exists x((x \in a \wedge (u = x \wedge u \in b)) \wedge y = T)), \\
  \mbox{} \\
  \tag{14}
  &\exists u(\exists x((x \in a \wedge (u = x \wedge u \in b)) \wedge y = T)) \to y \in \{(x, T)|x \in a\}[b]
\end{align*}
が共に成り立つ.
また定理 \ref{sthm=&in}より$u = x \wedge u \in b \to x \in b$が成り立つから, 
推論法則 \ref{dedaddw}により
\[
\tag{15}
  x \in a \wedge (u = x \wedge u \in b) \to x \in a \wedge x \in b
\]
が成り立つ.
また定理 \ref{sthmcapelement}と推論法則 \ref{dedequiv}により
\begin{align*}
  \tag{16}
  &x \in a \cap b \to x \in a \wedge x \in b, \\
  \mbox{} \\
  \tag{17}
  &x \in a \wedge x \in b \to x \in a \cap b
\end{align*}
が共に成り立つ.
そこで(15), (17)から, 推論法則 \ref{dedmmp}によって
\[
  x \in a \wedge (u = x \wedge u \in b) \to x \in a \cap b
\]
が成り立ち, これから推論法則 \ref{dedaddw}によって
\[
  (x \in a \wedge (u = x \wedge u \in b)) \wedge y = T \to x \in a \cap b \wedge y = T
\]
が成り立つ.
いま$x$と$u$は共に定数でないから, これに推論法則 \ref{dedalltquansepconst}を二回用いて
\[
  \exists u(\exists x((x \in a \wedge (u = x \wedge u \in b)) \wedge y = T)) \to \exists u(\exists x(x \in a \cap b \wedge y = T))
\]
が成り立つことがわかる.
ここで$u$が$x$とも$y$とも異なり, $a$, $b$, $T$のいずれの記号列の中にも自由変数として現れないことから, 
変数法則 \ref{valfund}, \ref{valwedge}, \ref{valquan}, \ref{valcap}によってわかるように, 
$u$は$\exists x(x \in a \cap b \wedge y = T)$の中に自由変数として現れない.
そこで変形法則 \ref{quanfree}により, 上記の記号列は
\[
\tag{18}
  \exists u(\exists x((x \in a \wedge (u = x \wedge u \in b)) \wedge y = T)) \to \exists x(x \in a \cap b \wedge y = T)
\]
と一致する.
よってこれが定理となる.
また仮定より$x$は$a$及び$b$の中に自由変数として現れないから, 
変数法則 \ref{valcap}により, $x$は$a \cap b$の中に自由変数として現れない.
このことと, $x$が$y$と異なることから, 
定理 \ref{sthmosetbasis}と推論法則 \ref{dedequiv}により
\begin{align*}
  \tag{19}
  &y \in \{T|x \in a \cap b\} \to \exists x(x \in a \cap b \wedge y = T), \\
  \mbox{} \\
  \tag{20}
  &\exists x(x \in a \cap b \wedge y = T) \to y \in \{T|x \in a \cap b\}
\end{align*}
が共に成り立つ.
そこで(13), (18), (20)から, 推論法則 \ref{dedmmp}によって
\[
\tag{21}
  y \in \{(x, T)|x \in a\}[b] \to y \in \{T|x \in a \cap b\}
\]
が成り立つことがわかる.
またThm \ref{x=x}より$x = x$が成り立つから, 
推論法則 \ref{dedatawbtrue2}により
\[
  x \in b \to x = x \wedge x \in b
\]
が成り立ち, これから推論法則 \ref{dedaddw}によって
\[
\tag{22}
  x \in a \wedge x \in b \to x \in a \wedge (x = x \wedge x \in b)
\]
が成り立つ.
そこで(16), (22)から, 推論法則 \ref{dedmmp}によって
\[
  x \in a \cap b \to x \in a \wedge (x = x \wedge x \in b)
\]
が成り立ち, これから推論法則 \ref{dedaddw}によって
\[
  x \in a \cap b \wedge y = T \to (x \in a \wedge (x = x \wedge x \in b)) \wedge y = T
\]
が成り立つ.
ここで$u$が$x$とも$y$とも異なり, $a$, $b$, $T$のいずれの記号列の中にも自由変数として現れないことから, 
代入法則 \ref{substfree}, \ref{substfund}, \ref{substwedge}により, 上記の記号列は
\[
\tag{23}
  x \in a \cap b \wedge y = T \to (x|u)((x \in a \wedge (u = x \wedge u \in b)) \wedge y = T)
\]
と一致する.
よってこれが定理となる.
またschema S4の適用により
\[
\tag{24}
  (x|u)((x \in a \wedge (u = x \wedge u \in b)) \wedge y = T) \to \exists u((x \in a \wedge (u = x \wedge u \in b)) \wedge y = T)
\]
が成り立つ.
そこで(23), (24)から, 推論法則 \ref{dedmmp}によって
\[
  x \in a \cap b \wedge y = T \to \exists u((x \in a \wedge (u = x \wedge u \in b)) \wedge y = T)
\]
が成り立つ.
いま$x$は定数でないので, これから推論法則 \ref{dedalltquansepconst}によって
\[
\tag{25}
  \exists x(x \in a \cap b \wedge y = T) \to \exists x(\exists u((x \in a \wedge (u = x \wedge u \in b)) \wedge y = T))
\]
が成り立つ.
またThm \ref{thmexch}と推論法則 \ref{dedequiv}により
\[
\tag{26}
  \exists x(\exists u((x \in a \wedge (u = x \wedge u \in b)) \wedge y = T)) \to \exists u(\exists x((x \in a \wedge (u = x \wedge u \in b)) \wedge y = T))
\]
が成り立つ.
そこで(19), (25), (26), (14)から, 推論法則 \ref{dedmmp}によって
\[
\tag{27}
  y \in \{T|x \in a \cap b\} \to y \in \{(x, T)|x \in a\}[b]
\]
が成り立つことがわかる.
そして(21), (27)から, 推論法則 \ref{dedequiv}によって
\[
\tag{28}
  y \in \{(x, T)|x \in a\}[b] \leftrightarrow y \in \{T|x \in a \cap b\}
\]
が成り立つ.
いま$y$は$x$と異なり, $a$, $b$, $T$のいずれの記号列の中にも自由変数として現れないから, 
変数法則 \ref{valoset}, \ref{valcap}, \ref{valpair}, \ref{valvalueset}によってわかるように, 
$y$は$\{(x, T)|x \in a\}[b]$及び$\{T|x \in a \cap b\}$の中に自由変数として現れない.
また$y$は定数でない.
このことと, (28)が成り立つことから, 定理 \ref{sthmset=}により
\[
  \{(x, T)|x \in a\}[b] = \{T|x \in a \cap b\}
\]
が成り立つ.

次に$x$が定数でないとは限らない場合を考える.
$v$を$x$と異なり, $a$, $b$, $T$のいずれの記号列の中にも自由変数として現れない, 
定数でない文字とする.
このとき, いま示したことから
\[
  \{(v, (v|x)(T))|v \in a\}[b] = \{(v|x)(T)|v \in a \cap b\}
\]
が成り立つが, 代入法則 \ref{substpair}によりこの記号列は
\[
\tag{29}
  \{(v|x)((x, T))|v \in a\}[b] = \{(v|x)(T)|v \in a \cap b\}
\]
と一致するから, これが定理となる.
いま$v$は$x$と異なり, $T$の中に自由変数として現れないから, 
変数法則 \ref{valpair}により, $v$は$(x, T)$の中に自由変数として現れない.
このことと, $x$と$v$が共に$a$の中に自由変数として現れないことから, 
代入法則 \ref{substosettrans}により
\[
\tag{30}
  \{(v|x)((x, T))|v \in a\} \equiv \{(x, T)|x \in a\}
\]
が成り立つ.
また$x$と$v$は共に$a$及び$b$の中に自由変数として現れないから, 
変数法則 \ref{valcap}により, これらは共に$a \cap b$の中に自由変数として現れない.
このことと, $v$が$T$の中に自由変数として現れないことから, 
同じく代入法則 \ref{substosettrans}により
\[
\tag{31}
  \{(v|x)(T)|v \in a \cap b\} \equiv \{T|x \in a \cap b\}
\]
が成り立つ.
そこで(30)と(31)から, (29)が
\[
  \{(x, T)|x \in a\}[b] = \{T|x \in a \cap b\}
\]
と一致することがわかる.
故にこれが定理となる.
\halmos




\mathstrut
\begin{thm}
\label{sthmcupvalueset}%定理
$a$, $b$, $c$を集合とするとき, 
\[
  (a \cup b)[c] = a[c] \cup b[c], ~~
  c[a \cup b] = c[a] \cup c[b]
\]
が成り立つ.
\end{thm}


\noindent{\bf 証明}
~$x$と$y$を, 互いに異なり, 共に$a$, $b$, $c$のいずれの記号列の中にも自由変数として現れない, 
定数でない文字とする.
このとき変数法則 \ref{valcup}により, $x$は$a \cup b$の中にも自由変数として現れないから, 
定理 \ref{sthmvaluesetelement}より
\begin{align*}
  \tag{1}
  y \in (a \cup b)[c] &\leftrightarrow \exists x(x \in c \wedge (x, y) \in a \cup b), \\
  \mbox{} \\
  \tag{2}
  y \in c[a \cup b] &\leftrightarrow \exists x(x \in a \cup b \wedge (x, y) \in c)
\end{align*}
が共に成り立つ.
また定理 \ref{sthmcupbasis}より
\begin{align*}
  (x, y) \in a \cup b &\leftrightarrow (x, y) \in a \vee (x, y) \in b, \\
  \mbox{} \\
  x \in a \cup b &\leftrightarrow x \in a \vee x \in b
\end{align*}
が共に成り立つから, 推論法則 \ref{dedaddeqw}により
\begin{align*}
  \tag{3}
  x \in c \wedge (x, y) \in a \cup b &\leftrightarrow x \in c \wedge ((x, y) \in a \vee (x, y) \in b), \\
  \mbox{} \\
  \tag{4}
  x \in a \cup b \wedge (x, y) \in c &\leftrightarrow (x \in a \vee x \in b) \wedge (x, y) \in c
\end{align*}
が共に成り立つ.
またThm \ref{aw1bvc1l1awb1v1awc1}より
\begin{align*}
  \tag{5}
  x \in c \wedge ((x, y) \in a \vee (x, y) \in b) &\leftrightarrow 
  (x \in c \wedge (x, y) \in a) \vee (x \in c \wedge (x, y) \in b), \\
  \mbox{} \\
  \tag{6}
  (x \in a \vee x \in b) \wedge (x, y) \in c &\leftrightarrow 
  (x \in a \wedge (x, y) \in c) \vee (x \in b \wedge (x, y) \in c)
\end{align*}
が共に成り立つ.
そこで(3)と(5)から, 推論法則 \ref{dedeqtrans}によって
\[
  x \in c \wedge (x, y) \in a \cup b \leftrightarrow 
  (x \in c \wedge (x, y) \in a) \vee (x \in c \wedge (x, y) \in b)
\]
が成り立ち, (4)と(6)から, 同じく推論法則 \ref{dedeqtrans}によって
\[
  x \in a \cup b \wedge (x, y) \in c \leftrightarrow 
  (x \in a \wedge (x, y) \in c) \vee (x \in b \wedge (x, y) \in c)
\]
が成り立つ.
そこで$x$が定数でないことから, 推論法則 \ref{dedalleqquansepconst}によって
\begin{align*}
  \tag{7}
  \exists x(x \in c \wedge (x, y) \in a \cup b) &\leftrightarrow 
  \exists x((x \in c \wedge (x, y) \in a) \vee (x \in c \wedge (x, y) \in b)), \\
  \mbox{} \\
  \tag{8}
  \exists x(x \in a \cup b \wedge (x, y) \in c) &\leftrightarrow 
  \exists x((x \in a \wedge (x, y) \in c) \vee (x \in b \wedge (x, y) \in c))
\end{align*}
が共に成り立つ.
またThm \ref{thmexv}より
\begin{align*}
  \tag{9}
  \exists x((x \in c \wedge (x, y) \in a) \vee (x \in c \wedge (x, y) \in b)) &\leftrightarrow 
  \exists x(x \in c \wedge (x, y) \in a) \vee \exists x(x \in c \wedge (x, y) \in b), \\
  \mbox{} \\
  \tag{10}
  \exists x((x \in a \wedge (x, y) \in c) \vee (x \in b \wedge (x, y) \in c)) &\leftrightarrow 
  \exists x(x \in a \wedge (x, y) \in c) \vee \exists x(x \in b \wedge (x, y) \in c)
\end{align*}
が共に成り立つ.
また$x$が$y$と異なり, $a$, $b$, $c$のいずれの記号列の中にも自由変数として現れないことから, 
定理 \ref{sthmvaluesetelement}と推論法則 \ref{dedeqch}により
\begin{align*}
  \exists x(x \in c \wedge (x, y) \in a) \leftrightarrow y \in a[c]&, ~~
  \exists x(x \in c \wedge (x, y) \in b) \leftrightarrow y \in b[c], \\
  \mbox{} \\
  \exists x(x \in a \wedge (x, y) \in c) \leftrightarrow y \in c[a]&, ~~
  \exists x(x \in b \wedge (x, y) \in c) \leftrightarrow y \in c[b]
\end{align*}
がすべて成り立つ.
そこでこのはじめの二つから, 推論法則 \ref{dedaddeqv}によって
\[
\tag{11}
  \exists x(x \in c \wedge (x, y) \in a) \vee \exists x(x \in c \wedge (x, y) \in b) \leftrightarrow 
  y \in a[c] \vee y \in b[c]
\]
が成り立ち, 後の二つから, 同じく推論法則 \ref{dedaddeqv}によって
\[
\tag{12}
  \exists x(x \in a \wedge (x, y) \in c) \vee \exists x(x \in b \wedge (x, y) \in c) \leftrightarrow 
  y \in c[a] \vee y \in c[b]
\]
が成り立つ.
また定理 \ref{sthmcupbasis}と推論法則 \ref{dedeqch}により
\begin{align*}
  \tag{13}
  y \in a[c] \vee y \in b[c] &\leftrightarrow y \in a[c] \cup b[c], \\
  \mbox{} \\
  \tag{14}
  y \in c[a] \vee y \in c[b] &\leftrightarrow y \in c[a] \cup c[b]
\end{align*}
が共に成り立つ.
そこで(1), (7), (9), (11), (13)から, 推論法則 \ref{dedeqtrans}によって
\[
\tag{15}
  y \in (a \cup b)[c] \leftrightarrow y \in a[c] \cup b[c]
\]
が成り立ち, (2), (8), (10), (12), (14)から, 同じく推論法則 \ref{dedeqtrans}によって
\[
\tag{16}
  y \in c[a \cup b] \leftrightarrow y \in c[a] \cup c[b]
\]
が成り立つことがわかる.
いま$y$は$a$, $b$, $c$の中に自由変数として現れないから, 
変数法則 \ref{valcup}, \ref{valvalueset}からわかるように, $y$は
$(a \cup b)[c]$, $a[c] \cup b[c]$, $c[a \cup b]$, $c[a] \cup c[b]$の
いずれの記号列の中にも自由変数として現れない.
また$y$は定数でない.
そこでこのことと, (15)と(16)が共に成り立つことから, 定理 \ref{sthmset=}により
$(a \cup b)[c] = a[c] \cup b[c]$と
$c[a \cup b] = c[a] \cup c[b]$が共に成り立つ.
\halmos




\mathstrut
\begin{thm}
\label{sthmcapvalueset}%定理
$a$, $b$, $c$を集合とするとき, 
\[
  (a \cap b)[c] \subset a[c] \cap b[c], ~~
  c[a \cap b] \subset c[a] \cap c[b]
\]
が成り立つ.
\end{thm}


\noindent{\bf 証明}
~定理 \ref{sthmcap}より
\[
  a \cap b \subset a, ~~
  a \cap b \subset b
\]
が共に成り立つから, この前者から, 定理 \ref{sthmvaluesetsubset}によって
\begin{align*}
  \tag{1}
  (a \cap b)[c] &\subset a[c], \\
  \mbox{} \\
  \tag{2}
  c[a \cap b] &\subset c[a]
\end{align*}
が共に成り立ち, 後者から, 同じく定理 \ref{sthmvaluesetsubset}によって
\begin{align*}
  \tag{3}
  (a \cap b)[c] &\subset b[c], \\
  \mbox{} \\
  \tag{4}
  c[a \cap b] &\subset c[b]
\end{align*}
が共に成り立つ.
そこで(1)と(3)から, 定理 \ref{sthmcapdil}によって
$(a \cap b)[c] \subset a[c] \cap b[c]$が成り立ち, 
(2)と(4)から, 同じく定理 \ref{sthmcapdil}によって
$c[a \cap b] \subset c[a] \cap c[b]$が成り立つ.
\halmos




\mathstrut
\begin{thm}
\label{sthmvaluesetcappr1set}%定理
$a$と$b$を集合とするとき, 
\[
  a[b] = a[b \cap {\rm pr}_{1}\langle a \rangle]
\]
が成り立つ.
\end{thm}


\noindent{\bf 証明}
~$x$と$y$を, 互いに異なり, 共に$a$及び$b$の中に自由変数として現れない, 
定数でない文字とする.
このとき定理 \ref{sthmvaluesetelement}と推論法則 \ref{dedequiv}により
\[
\tag{1}
  y \in a[b] \to \exists x(x \in b \wedge (x, y) \in a)
\]
が成り立つ.
また定理 \ref{sthmpairelementinprset}より
\[
  (x, y) \in a \to x \in {\rm pr}_{1}\langle a \rangle \wedge y \in {\rm pr}_{2}\langle a \rangle
\]
が成り立つから, 推論法則 \ref{dedprewedge}により
\[
  (x, y) \in a \to x \in {\rm pr}_{1}\langle a \rangle
\]
が成り立つ.
そこで推論法則 \ref{dedatawbtrue1}により
\[
  (x, y) \in a \to x \in {\rm pr}_{1}\langle a \rangle \wedge (x, y) \in a
\]
が成り立ち, これから推論法則 \ref{dedaddw}によって
\[
\tag{2}
  x \in b \wedge (x, y) \in a \to x \in b \wedge (x \in {\rm pr}_{1}\langle a \rangle \wedge (x, y) \in a)
\]
が成り立つ.
またThm \ref{aw1bwc1t1awb1wc}より
\[
\tag{3}
  x \in b \wedge (x \in {\rm pr}_{1}\langle a \rangle \wedge (x, y) \in a) \to 
  (x \in b \wedge x \in {\rm pr}_{1}\langle a \rangle) \wedge (x, y) \in a
\]
が成り立つ.
また定理 \ref{sthmcapelement}と推論法則 \ref{dedequiv}により
\[
  x \in b \wedge x \in {\rm pr}_{1}\langle a \rangle \to x \in b \cap {\rm pr}_{1}\langle a \rangle
\]
が成り立つから, 推論法則 \ref{dedaddw}により
\[
\tag{4}
  (x \in b \wedge x \in {\rm pr}_{1}\langle a \rangle) \wedge (x, y) \in a \to 
  x \in b \cap {\rm pr}_{1}\langle a \rangle \wedge (x, y) \in a
\]
が成り立つ.
そこで(2), (3), (4)から, 推論法則 \ref{dedmmp}によって
\[
  x \in b \wedge (x, y) \in a \to x \in b \cap {\rm pr}_{1}\langle a \rangle \wedge (x, y) \in a
\]
が成り立つことがわかる.
いま$x$は定数でないから, これから推論法則 \ref{dedalltquansepconst}によって
\[
\tag{5}
  \exists x(x \in b \wedge (x, y) \in a) \to \exists x(x \in b \cap {\rm pr}_{1}\langle a \rangle \wedge (x, y) \in a)
\]
が成り立つ.
また$x$は$a$及び$b$の中に自由変数として現れないから, 変数法則 \ref{valcap}, \ref{valprset}により, 
$x$は$b \cap {\rm pr}_{1}\langle a \rangle$の中にも自由変数として現れない.
このことと, $x$が$y$と異なることから, 
定理 \ref{sthmvaluesetelement}と推論法則 \ref{dedequiv}により
\[
\tag{6}
  \exists x(x \in b \cap {\rm pr}_{1}\langle a \rangle \wedge (x, y) \in a) \to 
  y \in a[b \cap {\rm pr}_{1}\langle a \rangle]
\]
が成り立つ.
そこで(1), (5), (6)から, 推論法則 \ref{dedmmp}によって
\[
\tag{7}
  y \in a[b] \to y \in a[b \cap {\rm pr}_{1}\langle a \rangle]
\]
が成り立つことがわかる.
いま$y$は$a$及び$b$の中に自由変数として現れないから, 
変数法則 \ref{valcap}, \ref{valprset}, \ref{valvalueset}により, 
$y$は$a[b]$及び$a[b \cap {\rm pr}_{1}\langle a \rangle]$の中に自由変数として現れない.
また$y$は定数でない.
そこでこれらのことと, (7)が成り立つことから, 
定理 \ref{sthmsubsetconst}により
\[
\tag{8}
  a[b] \subset a[b \cap {\rm pr}_{1}\langle a \rangle]
\]
が成り立つ.
また定理 \ref{sthmcap}より
$b \cap {\rm pr}_{1}\langle a \rangle \subset b$が成り立つから, 
定理 \ref{sthmvaluesetsubset}により
\[
\tag{9}
  a[b \cap {\rm pr}_{1}\langle a \rangle] \subset a[b]
\]
が成り立つ.
そこで(8), (9)から, 定理 \ref{sthmaxiom1}により
$a[b] = a[b \cap {\rm pr}_{1}\langle a \rangle]$が成り立つ.
\halmos




\mathstrut
\begin{thm}
\label{sthm-valueset}%定理
$a$, $b$, $c$を集合とするとき, 
\[
  a[c] - b[c] \subset (a - b)[c], ~~
  c[a] - c[b] \subset c[a - b]
\]
が成り立つ.
\end{thm}


\noindent{\bf 証明}
~$x$と$y$を, 互いに異なり, 共に$a$, $b$, $c$のいずれの記号列の中にも自由変数として現れない, 
定数でない文字とする.
このとき定理 \ref{sthm-basis}と推論法則 \ref{dedequiv}により
\begin{align*}
  \tag{1}
  y \in a[c] - b[c] &\to y \in a[c] \wedge y \notin b[c], \\
  \mbox{} \\
  \tag{2}
  y \in c[a] - c[b] &\to y \in c[a] \wedge y \notin c[b]
\end{align*}
が共に成り立つ.
また$x$が$y$と異なり, $a$, $b$, $c$のいずれの記号列の中にも自由変数として現れないことから, 
定理 \ref{sthmvaluesetelement}と推論法則 \ref{dedequiv}により
\begin{align*}
  \tag{3}
  &y \in a[c] \to \exists x(x \in c \wedge (x, y) \in a), \\
  \mbox{} \\
  \tag{4}
  &\exists x(x \in c \wedge (x, y) \in b) \to y \in b[c], \\
  \mbox{} \\
  \tag{5}
  &y \in c[a] \to \exists x(x \in a \wedge (x, y) \in c), \\
  \mbox{} \\
  \tag{6}
  &\exists x(x \in b \wedge (x, y) \in c) \to y \in c[b]
\end{align*}
がすべて成り立つ.
そこでこの(4)と(6)から, 推論法則 \ref{dedcp}によって
\begin{align*}
  \tag{7}
  y \notin b[c] &\to \neg \exists x(x \in c \wedge (x, y) \in b), \\
  \mbox{} \\
  \tag{8}
  y \notin c[b] &\to \neg \exists x(x \in b \wedge (x, y) \in c)
\end{align*}
が共に成り立つ.
またThm \ref{thmeaquandm}と推論法則 \ref{dedequiv}により
\begin{align*}
  \tag{9}
  \neg \exists x(x \in c \wedge (x, y) \in b) &\to \forall x(\neg (x \in c \wedge (x, y) \in b)), \\
  \mbox{} \\
  \tag{10}
  \neg \exists x(x \in b \wedge (x, y) \in c) &\to \forall x(\neg (x \in b \wedge (x, y) \in c))
\end{align*}
が共に成り立つ.
またThm \ref{n1awb1tnavnb}より
\[
  \neg (x \in c \wedge (x, y) \in b) \to x \notin c \vee (x, y) \notin b, ~~
  \neg (x \in b \wedge (x, y) \in c) \to x \notin b \vee (x, y) \notin c
\]
が共に成り立つから, $x$が定数でないことから, 推論法則 \ref{dedalltquansepconst}により
\begin{align*}
  \tag{11}
  \forall x(\neg (x \in c \wedge (x, y) \in b)) &\to \forall x(x \notin c \vee (x, y) \notin b), \\
  \mbox{} \\
  \tag{12}
  \forall x(\neg (x \in b \wedge (x, y) \in c)) &\to \forall x(x \notin b \vee (x, y) \notin c)
\end{align*}
が共に成り立つ.
そこで(7), (9), (11)から, 推論法則 \ref{dedmmp}によって
\[
\tag{13}
  y \notin b[c] \to \forall x(x \notin c \vee (x, y) \notin b)
\]
が成り立ち, (8), (10), (12)から, 同じく推論法則 \ref{dedmmp}によって
\[
\tag{14}
  y \notin c[b] \to \forall x(x \notin b \vee (x, y) \notin c)
\]
が成り立つことがわかる.
そこで(3)と(13), (5)と(14)にそれぞれ推論法則 \ref{dedfromaddw}を適用して, 
\begin{align*}
  \tag{15}
  y \in a[c] \wedge y \notin b[c] &\to 
  \exists x(x \in c \wedge (x, y) \in a) \wedge \forall x(x \notin c \vee (x, y) \notin b), \\
  \mbox{} \\
  \tag{16}
  y \in c[a] \wedge y \notin c[b] &\to 
  \exists x(x \in a \wedge (x, y) \in c) \wedge \forall x(x \notin b \vee (x, y) \notin c)
\end{align*}
が共に成り立つ.
またThm \ref{thmexw2}より
\begin{align*}
  \tag{17}
  \exists x(x \in c \wedge (x, y) \in a) \wedge \forall x(x \notin c \vee (x, y) \notin b) &\to 
  \exists x((x \in c \wedge (x, y) \in a) \wedge (x \notin c \vee (x, y) \notin b)), \\
  \mbox{} \\
  \tag{18}
  \exists x(x \in a \wedge (x, y) \in c) \wedge \forall x(x \notin b \vee (x, y) \notin c) &\to 
  \exists x((x \in a \wedge (x, y) \in c) \wedge (x \notin b \vee (x, y) \notin c))
\end{align*}
が共に成り立つ.
またThm \ref{aw1bvc1t1awb1v1awc1}より
\begin{multline*}
\tag{19}
  (x \in c \wedge (x, y) \in a) \wedge (x \notin c \vee (x, y) \notin b) \\
  \to ((x \in c \wedge (x, y) \in a) \wedge x \notin c) \vee ((x \in c \wedge (x, y) \in a) \wedge (x, y) \notin b), 
\end{multline*}
\begin{multline*}
\tag{20}
  (x \in a \wedge (x, y) \in c) \wedge (x \notin b \vee (x, y) \notin c) \\
  \to ((x \in a \wedge (x, y) \in c) \wedge x \notin b) \vee ((x \in a \wedge (x, y) \in c) \wedge (x, y) \notin c)
\end{multline*}
が共に成り立つ.
またThm \ref{n1awna1}より
\[
  \neg (x \in c \wedge x \notin c), ~~
  \neg ((x, y) \in c \wedge (x, y) \notin c)
\]
が共に成り立つから, 推論法則 \ref{dednw}により
\begin{align*}
  \tag{21}
  &\neg ((x \in c \wedge x \notin c) \wedge (x, y) \in a), \\
  \mbox{} \\
  \tag{22}
  &\neg (x \in a \wedge ((x, y) \in c \wedge (x, y) \notin c))
\end{align*}
が共に成り立つ.
またThm \ref{1awb1wclaw1bwc1}より
\begin{align*}
  \tag{23}
  (x \in c \wedge x \notin c) \wedge (x, y) \in a &\leftrightarrow x \in c \wedge (x \notin c \wedge (x, y) \in a), \\
  \mbox{} \\
  \tag{24}
  (x \in a \wedge (x, y) \in c) \wedge (x, y) \notin c &\leftrightarrow x \in a \wedge ((x, y) \in c \wedge (x, y) \notin c)
\end{align*}
が共に成り立つ.
またThm \ref{awblbwa}より
\[
  x \notin c \wedge (x, y) \in a \leftrightarrow (x, y) \in a \wedge x \notin c
\]
が成り立つから, 推論法則 \ref{dedaddeqw}により
\[
\tag{25}
  x \in c \wedge (x \notin c \wedge (x, y) \in a) \leftrightarrow x \in c \wedge ((x, y) \in a \wedge x \notin c)
\]
が成り立つ.
またThm \ref{1awb1wclaw1bwc1}と推論法則 \ref{dedeqch}により
\[
\tag{26}
  x \in c \wedge ((x, y) \in a \wedge x \notin c) \leftrightarrow (x \in c \wedge (x, y) \in a) \wedge x \notin c
\]
が成り立つ.
そこで(23), (25), (26)から, 推論法則 \ref{dedeqtrans}によって
\[
\tag{27}
  (x \in c \wedge x \notin c) \wedge (x, y) \in a \leftrightarrow (x \in c \wedge (x, y) \in a) \wedge x \notin c
\]
が成り立つ.
そこで(21)と(27), (22)と(24)から, 推論法則 \ref{dedeqfund}によって
\begin{align*}
  &\neg ((x \in c \wedge (x, y) \in a) \wedge x \notin c), \\
  \mbox{} \\
  &\neg ((x \in a \wedge (x, y) \in c) \wedge (x, y) \notin c)
\end{align*}
が共に成り立つ.
そこでこれらから, 推論法則 \ref{dedavbtbtrue2}によって
\begin{multline*}
\tag{28}
  ((x \in c \wedge (x, y) \in a) \wedge x \notin c) \vee ((x \in c \wedge (x, y) \in a) \wedge (x, y) \notin b) \\
  \to (x \in c \wedge (x, y) \in a) \wedge (x, y) \notin b, 
\end{multline*}
\[
\tag{29}
  ((x \in a \wedge (x, y) \in c) \wedge x \notin b) \vee ((x \in a \wedge (x, y) \in c) \wedge (x, y) \notin c) 
  \to (x \in a \wedge (x, y) \in c) \wedge x \notin b
\]
が共に成り立つ.
またThm \ref{1awb1wctaw1bwc1}より
\begin{align*}
  \tag{30}
  (x \in c \wedge (x, y) \in a) \wedge (x, y) \notin b &\to x \in c \wedge ((x, y) \in a \wedge (x, y) \notin b), \\
  \mbox{} \\
  \tag{31}
  (x \in a \wedge (x, y) \in c) \wedge x \notin b &\to x \in a \wedge ((x, y) \in c \wedge x \notin b)
\end{align*}
が共に成り立つ.
またThm \ref{awbtbwa}より
\[
  (x, y) \in c \wedge x \notin b \to x \notin b \wedge (x, y) \in c
\]
が成り立つから, 推論法則 \ref{dedaddw}により
\[
\tag{32}
  x \in a \wedge ((x, y) \in c \wedge x \notin b) \to x \in a \wedge (x \notin b \wedge (x, y) \in c)
\]
が成り立つ.
またThm \ref{aw1bwc1t1awb1wc}より
\[
\tag{33}
  x \in a \wedge (x \notin b \wedge (x, y) \in c) \to (x \in a \wedge x \notin b) \wedge (x, y) \in c
\]
が成り立つ.
また定理 \ref{sthm-basis}と推論法則 \ref{dedequiv}により
\[
  (x, y) \in a \wedge (x, y) \notin b \to (x, y) \in a - b, ~~
  x \in a \wedge x \notin b \to x \in a - b
\]
が共に成り立つから, 推論法則 \ref{dedaddw}により
\begin{align*}
  \tag{34}
  x \in c \wedge ((x, y) \in a \wedge (x, y) \notin b) &\to x \in c \wedge (x, y) \in a - b, \\
  \mbox{} \\
  \tag{35}
  (x \in a \wedge x \notin b) \wedge (x, y) \in c &\to x \in a - b \wedge (x, y) \in c
\end{align*}
が共に成り立つ.
そこで(19), (28), (30), (34)から, 推論法則 \ref{dedmmp}によって
\[
  (x \in c \wedge (x, y) \in a) \wedge (x \notin c \vee (x, y) \notin b) \to x \in c \wedge (x, y) \in a - b
\]
が成り立ち, (20), (29), (31), (32), (33), (35)から, 同じく推論法則 \ref{dedmmp}によって
\[
  (x \in a \wedge (x, y) \in c) \wedge (x \notin b \vee (x, y) \notin c) \to x \in a - b \wedge (x, y) \in c
\]
が成り立つことがわかる.
そこでこれらと, $x$が定数でないことから, 推論法則 \ref{dedalltquansepconst}によって
\begin{align*}
  \tag{36}
  \exists x((x \in c \wedge (x, y) \in a) \wedge (x \notin c \vee (x, y) \notin b)) &\to \exists x(x \in c \wedge (x, y) \in a - b), \\
  \mbox{} \\
  \tag{37}
  \exists x((x \in a \wedge (x, y) \in c) \wedge (x \notin b \vee (x, y) \notin c)) &\to \exists x(x \in a - b \wedge (x, y) \in c)
\end{align*}
が共に成り立つ.
いま$x$は$a$及び$b$の中に自由変数として現れないから, 
変数法則 \ref{val-}により, $x$は$a - b$の中に自由変数として現れない.
また$x$は$y$と異なり, $c$の中にも自由変数として現れない.
そこで定理 \ref{sthmvaluesetelement}と推論法則 \ref{dedequiv}により
\begin{align*}
  \tag{38}
  \exists x(x \in c \wedge (x, y) \in a - b) &\to y \in (a - b)[c], \\
  \mbox{} \\
  \tag{39}
  \exists x(x \in a - b \wedge (x, y) \in c) &\to y \in c[a - b]
\end{align*}
が共に成り立つ.
そこで(1), (15), (17), (36), (38)から, 推論法則 \ref{dedmmp}によって
\[
\tag{40}
  y \in a[c] - b[c] \to y \in (a - b)[c]
\]
が成り立ち, (2), (16), (18), (37), (39)から, 同じく推論法則 \ref{dedmmp}によって
\[
\tag{41}
  y \in c[a] - c[b] \to y \in c[a - b]
\]
が成り立つことがわかる.
さて仮定より$y$は$a$, $b$, $c$のいずれの記号列の中にも自由変数として現れないから, 
変数法則 \ref{val-}, \ref{valvalueset}によってわかるように, $y$は
$a[c] - b[c]$, $(a - b)[c]$, $c[a] - c[b]$, $c[a - b]$のいずれの記号列の中にも自由変数として現れない.
また$y$は定数でない.
そこでこのことと, (40)と(41)が共に成り立つことから, 定理 \ref{sthmsubsetconst}により
$a[c] - b[c] \subset (a - b)[c]$と$c[a] - c[b] \subset c[a - b]$が共に成り立つ.
\halmos




\mathstrut
\begin{thm}
\label{sthmvaluesetemptycon}%定理
$a$と$b$を集合とするとき, 
\[
  a[b] = \phi \leftrightarrow b \cap {\rm pr}_{1}\langle a \rangle = \phi
\]
が成り立つ.
またこのことから, 次の($*$)が成り立つ: 

($*$) ~~$a[b]$が空ならば, $b \cap {\rm pr}_{1}\langle a \rangle$は空である.
        逆に$b \cap {\rm pr}_{1}\langle a \rangle$が空ならば, $a[b]$は空である.
\end{thm}


\noindent{\bf 証明}
~$x$を$a$及び$b$の中に自由変数として現れない文字とする.
このとき変数法則 \ref{valvalueset}により, $x$は$a[b]$の中に自由変数として現れない.
そこで$\tau_{x}(x \in a[b])$を$T$と書くとき, $T$は集合であり, 
定理 \ref{sthmelm&empty}と推論法則 \ref{dedequiv}により
\[
\tag{1}
  a[b] \neq \phi \to T \in a[b]
\]
が成り立つ.
また$T$の定義から, 変数法則 \ref{valtau}により
$x$は$T$の中に自由変数として現れない.
このことと, $x$が$a$及び$b$の中にも自由変数として現れないことから, 
定理 \ref{sthmvaluesetelement}と推論法則 \ref{dedequiv}により
\[
  T \in a[b] \to \exists x(x \in b \wedge (x, T) \in a)
\]
が成り立つ.
ここで$\tau_{x}(x \in b \wedge (x, T) \in a)$を$U$と書けば, $U$は集合であり, 
定義から上記の記号列は
\[
  T \in a[b] \to (U|x)(x \in b \wedge (x, T) \in a)
\]
である.
また上述のように$x$は$a$, $b$, $T$のいずれの記号列の中にも自由変数として現れないから, 
代入法則 \ref{substfree}, \ref{substfund}, \ref{substwedge}, \ref{substpair}により, 
この記号列は
\[
\tag{2}
  T \in a[b] \to U \in b \wedge (U, T) \in a
\]
と一致する.
よってこれが定理となる.
また定理 \ref{sthmpairelementinprset}より
\[
  (U, T) \in a \to U \in {\rm pr}_{1}\langle a \rangle \wedge T \in {\rm pr}_{2}\langle a \rangle
\]
が成り立つから, 推論法則 \ref{dedprewedge}によって
\[
  (U, T) \in a \to U \in {\rm pr}_{1}\langle a \rangle
\]
が成り立ち, これから推論法則 \ref{dedaddw}によって
\[
\tag{3}
  U \in b \wedge (U, T) \in a \to U \in b \wedge U \in {\rm pr}_{1}\langle a \rangle
\]
が成り立つ.
また定理 \ref{sthmcapelement}と推論法則 \ref{dedequiv}により
\[
\tag{4}
  U \in b \wedge U \in {\rm pr}_{1}\langle a \rangle \to U \in b \cap {\rm pr}_{1}\langle a \rangle
\]
が成り立つ.
また定理 \ref{sthmnotemptyeqexin}より
\[
\tag{5}
  U \in b \cap {\rm pr}_{1}\langle a \rangle \to b \cap {\rm pr}_{1}\langle a \rangle \neq \phi
\]
が成り立つ.
そこで(1)---(5)から, 推論法則 \ref{dedmmp}によって
\[
\tag{6}
  a[b] \neq \phi \to b \cap {\rm pr}_{1}\langle a \rangle \neq \phi
\]
が成り立つことがわかる.
またいま$x$が$a$及び$b$の中に自由変数として現れないことから, 
変数法則 \ref{valcap}, \ref{valprset}によってわかるように, $x$は$b \cap {\rm pr}_{1}\langle a \rangle$の中に
自由変数として現れない.
そこで$\tau_{x}(x \in b \cap {\rm pr}_{1}\langle a \rangle)$を$V$と書けば, $V$は集合であり, 
定理 \ref{sthmelm&empty}と推論法則 \ref{dedequiv}により
\[
\tag{7}
  b \cap {\rm pr}_{1}\langle a \rangle \neq \phi \to V \in b \cap {\rm pr}_{1}\langle a \rangle
\]
が成り立つ.
また定理 \ref{sthmcapelement}と推論法則 \ref{dedequiv}により
\[
\tag{8}
  V \in b \cap {\rm pr}_{1}\langle a \rangle \to V \in b \wedge V \in {\rm pr}_{1}\langle a \rangle
\]
が成り立つ.
また$V$の定義から, 変数法則 \ref{valtau}により$x$は$V$の中に自由変数として現れない.
このことと, $x$が$a$の中にも自由変数として現れないことから, 
定理 \ref{sthmprsetelement}と推論法則 \ref{dedequiv}により
\[
  V \in {\rm pr}_{1}\langle a \rangle \to \exists x((V, x) \in a)
\]
が成り立つ.
ここで$\tau_{x}((V, x) \in a)$を$W$と書けば, $W$は集合であり, 
定義から上記の記号列は
\[
  V \in {\rm pr}_{1}\langle a \rangle \to (W|x)((V, x) \in a)
\]
である.
また上述のように$x$が$a$及び$V$の中に自由変数として現れないことから, 
代入法則 \ref{substfree}, \ref{substfund}, \ref{substpair}により, 
この記号列は
\[
  V \in {\rm pr}_{1}\langle a \rangle \to (V, W) \in a
\]
と一致する.
よってこれが定理となる.
そこで推論法則 \ref{dedaddw}により
\[
\tag{9}
  V \in b \wedge V \in {\rm pr}_{1}\langle a \rangle \to V \in b \wedge (V, W) \in a
\]
が成り立つ.
また定理 \ref{sthmvaluesetbasis}より
\[
\tag{10}
  V \in b \wedge (V, W) \in a \to W \in a[b]
\]
が成り立つ.
また定理 \ref{sthmnotemptyeqexin}より
\[
\tag{11}
  W \in a[b] \to a[b] \neq \phi
\]
が成り立つ.
そこで(7)---(11)から, 推論法則 \ref{dedmmp}によって
\[
\tag{12}
  b \cap {\rm pr}_{1}\langle a \rangle \neq \phi \to a[b] \neq \phi
\]
が成り立つことがわかる.
そこで(6), (12)から, 推論法則 \ref{dedequiv}によって
\[
  a[b] \neq \phi \leftrightarrow b \cap {\rm pr}_{1}\langle a \rangle \neq \phi
\]
が成り立ち, これから推論法則 \ref{dedeqcp}によって
\[
  a[b] = \phi \leftrightarrow b \cap {\rm pr}_{1}\langle a \rangle = \phi
\]
が成り立つ.
($*$)が成り立つことは, これと推論法則 \ref{dedeqfund}によって明らかである.
\halmos




\mathstrut
\begin{thm}
\label{sthmvaluesetempty}%定理
$a$を集合とするとき, $\phi[a]$と$a[\phi]$は共に空である.
\end{thm}


\noindent{\bf 証明}
~定理 \ref{sthmemptyprset}より
${\rm pr}_{1}\langle \phi \rangle = \phi$が成り立つから, 
定理 \ref{sthmcap=}により
\[
\tag{1}
  a \cap {\rm pr}_{1}\langle \phi \rangle = a \cap \phi
\]
が成り立つ.
また定理 \ref{sthmcapempty}より
\[
\tag{2}
  a \cap \phi = \phi
\]
が成り立つ.
そこで(1), (2)から, 推論法則 \ref{ded=trans}によって
\[
  a \cap {\rm pr}_{1}\langle \phi \rangle = \phi
\]
が成り立ち, 
これから定理 \ref{sthmvaluesetemptycon}によって
$\phi[a] = \phi$が成り立つ.
また定理 \ref{sthmcapempty}より
$\phi \cap {\rm pr}_{1}\langle a \rangle = \phi$が成り立つから, 
やはり定理 \ref{sthmvaluesetemptycon}によって
$a[\phi] = \phi$が成り立つ.
\halmos




\mathstrut
\begin{thm}
\label{sthmvaluesetproduct}%定理
$a$, $b$, $c$を集合とするとき, 
\[
  a \cap c = \phi \to (a \times b)[c] = \phi, ~~
  a \cap c \neq \phi \to (a \times b)[c] = b
\]
が成り立つ.
またこのことから, 次の($*$)が成り立つ: 

($*$) ~~$a \cap c$が空ならば, $(a \times b)[c]$は空である.
        $a \cap c$が空でなければ, $(a \times b)[c] = b$が成り立つ.
\end{thm}


\noindent{\bf 証明}
~$x$と$y$を, 互いに異なり, 共に$a$, $b$, $c$のいずれの記号列の中にも自由変数として現れない, 
定数でない文字とする.
このとき変数法則 \ref{valproduct}により, $x$は$a \times b$の中に自由変数として現れないから, 
このことと, $x$が$y$と異なり, $c$の中に自由変数として現れないことから, 
定理 \ref{sthmvaluesetelement}より
\[
\tag{1}
  y \in (a \times b)[c] \leftrightarrow \exists x(x \in c \wedge (x, y) \in a \times b)
\]
が成り立つ.
また定理 \ref{sthmpairinproduct}より
\[
  (x, y) \in a \times b \leftrightarrow x \in a \wedge y \in b
\]
が成り立つから, 推論法則 \ref{dedaddeqw}により
\[
\tag{2}
  x \in c \wedge (x, y) \in a \times b \leftrightarrow x \in c \wedge (x \in a \wedge y \in b)
\]
が成り立つ.
またThm \ref{1awb1wclaw1bwc1}と推論法則 \ref{dedeqch}により
\[
\tag{3}
  x \in c \wedge (x \in a \wedge y \in b) \leftrightarrow (x \in c \wedge x \in a) \wedge y \in b
\]
が成り立つ.
またThm \ref{awblbwa}より
\[
\tag{4}
  x \in c \wedge x \in a \leftrightarrow x \in a \wedge x \in c
\]
が成り立つ.
また定理 \ref{sthmcapelement}と推論法則 \ref{dedeqch}により
\[
\tag{5}
  x \in a \wedge x \in c \leftrightarrow x \in a \cap c
\]
が成り立つ.
そこで(4), (5)から, 推論法則 \ref{dedeqtrans}によって
\[
  x \in c \wedge x \in a \leftrightarrow x \in a \cap c
\]
が成り立ち, これから推論法則 \ref{dedaddeqw}によって
\[
\tag{6}
  (x \in c \wedge x \in a) \wedge y \in b \leftrightarrow x \in a \cap c \wedge y \in b
\]
が成り立つ.
またThm \ref{awblbwa}より
\[
\tag{7}
  x \in a \cap c \wedge y \in b \leftrightarrow y \in b \wedge x \in a \cap c
\]
が成り立つ.
そこで(2), (3), (6), (7)から, 推論法則 \ref{dedeqtrans}によって
\[
  x \in c \wedge (x, y) \in a \times b \leftrightarrow y \in b \wedge x \in a \cap c
\]
が成り立つことがわかる.
いま$x$は定数でないから, これから推論法則 \ref{dedalleqquansepconst}によって
\[
\tag{8}
  \exists x(x \in c \wedge (x, y) \in a \times b) \leftrightarrow \exists x(y \in b \wedge x \in a \cap c)
\]
が成り立つ.
また$x$が$y$と異なり, $b$の中に自由変数として現れないことから, 
変数法則 \ref{valfund}により, $x$は$y \in b$の中に自由変数として現れないから, 
Thm \ref{thmexwrfree}より
\[
\tag{9}
  \exists x(y \in b \wedge x \in a \cap c) \leftrightarrow y \in b \wedge \exists x(x \in a \cap c)
\]
が成り立つ.
また$x$が$a$及び$c$の中に自由変数として現れないことから, 
変数法則 \ref{valcap}により, $x$は$a \cap c$の中に自由変数として現れないから, 
定理 \ref{sthmnotemptyeqexin}と推論法則 \ref{dedeqch}により
\[
  \exists x(x \in a \cap c) \leftrightarrow a \cap c \neq \phi
\]
が成り立つ.
そこで推論法則 \ref{dedaddeqw}により
\[
\tag{10}
  y \in b \wedge \exists x(x \in a \cap c) \leftrightarrow y \in b \wedge a \cap c \neq \phi
\]
が成り立つ.
そこで(1), (8), (9), (10)から, 推論法則 \ref{dedeqtrans}によって
\[
  y \in (a \times b)[c] \leftrightarrow y \in b \wedge a \cap c \neq \phi
\]
が成り立つことがわかる.
故に推論法則 \ref{dedequiv}により
\begin{align*}
  \tag{11}
  &y \in (a \times b)[c] \to y \in b \wedge a \cap c \neq \phi, \\
  \mbox{} \\
  \tag{12}
  &y \in b \wedge a \cap c \neq \phi \to y \in (a \times b)[c]
\end{align*}
が共に成り立つ.
この(11)から, 推論法則 \ref{dedcp}により
\[
\tag{13}
  \neg (y \in b \wedge a \cap c \neq \phi) \to y \notin (a \times b)[c]
\]
が成り立つ.
またThm \ref{awbta}より
\[
  y \in b \wedge a \cap c \neq \phi \to a \cap c \neq \phi
\]
が成り立つから, 同じく推論法則 \ref{dedcp}により
\[
\tag{14}
  a \cap c = \phi \to \neg (y \in b \wedge a \cap c \neq \phi)
\]
が成り立つ.
そこで(14), (13)から, 推論法則 \ref{dedmmp}によって
\[
\tag{15}
  a \cap c = \phi \to y \notin (a \times b)[c]
\]
が成り立つ.
さていま$y$は$a$及び$c$の中に自由変数として現れないから, 
変数法則 \ref{valfund}, \ref{valcap}, \ref{valempty}によってわかるように, 
$y$は$a \cap c = \phi$の中に自由変数として現れない.
また$y$は定数でない.
そこでこれらのことと, (15)が成り立つことから, 
推論法則 \ref{dedalltquansepfreeconst}によって
\[
\tag{16}
  a \cap c = \phi \to \forall y(y \notin (a \times b)[c])
\]
が成り立つ.
また$y$が$a$, $b$, $c$のいずれの記号列の中にも自由変数として現れないことから, 
変数法則 \ref{valproduct}, \ref{valvalueset}により, $y$が
$(a \times b)[c]$の中に自由変数として現れないことがわかるから, 
定理 \ref{sthmnotinempty}と推論法則 \ref{dedequiv}により
\[
\tag{17}
  \forall y(y \notin (a \times b)[c]) \to (a \times b)[c] = \phi
\]
が成り立つ.
そこで(16), (17)から, 推論法則 \ref{dedmmp}によって
\[
\tag{18}
  a \cap c = \phi \to (a \times b)[c] = \phi
\]
が成り立つ.
また(11)から, 推論法則 \ref{dedprewedge}により
\[
  y \in (a \times b)[c] \to y \in b
\]
が成り立つから, 推論法則 \ref{deds1}により
\[
\tag{19}
  a \cap c \neq \phi \to (y \in (a \times b)[c] \to y \in b)
\]
が成り立つ.
また(12)から, 推論法則 \ref{dedtwch}により
\[
  y \in b \to (a \cap c \neq \phi \to y \in (a \times b)[c])
\]
が成り立ち, これから推論法則 \ref{dedch}により
\[
\tag{20}
  a \cap c \neq \phi \to (y \in b \to y \in (a \times b)[c])
\]
が成り立つ.
そこで(19), (20)から, 推論法則 \ref{dedprewedge}によって
\[
\tag{21}
  a \cap c \neq \phi \to (y \in (a \times b)[c] \leftrightarrow y \in b)
\]
が成り立つ.
さていま$y$は$a$及び$c$の中に自由変数として現れないから, 
変数法則 \ref{valfund}, \ref{valcap}, \ref{valempty}によってわかるように, 
$y$は$a \cap c \neq \phi$の中に自由変数として現れない.
また$y$は定数でない.
そこでこれらのことと, (21)が成り立つことから, 推論法則 \ref{dedalltquansepfreeconst}によって
\[
\tag{22}
  a \cap c \neq \phi \to \forall y(y \in (a \times b)[c] \leftrightarrow y \in b)
\]
が成り立つ.
また上述のように$y$は$(a \times b)[c]$及び$b$の中に自由変数として現れないから, 
定理 \ref{sthmset=}と推論法則 \ref{dedequiv}により
\[
\tag{23}
  \forall y(y \in (a \times b)[c] \leftrightarrow y \in b) \to (a \times b)[c] = b
\]
が成り立つ.
そこで(22), (23)から, 推論法則 \ref{dedmmp}によって
\[
\tag{24}
  a \cap c \neq \phi \to (a \times b)[c] = b
\]
が成り立つ.
($*$)が成り立つことは, (18), (24)が成り立つことと推論法則 \ref{dedmp}によって
明らかである.
\halmos




\mathstrut
\begin{thm}
\label{sthmvaluesetsubsetpr2set}%定理
$a$と$b$を集合とするとき, 
\[
  a[b] \subset {\rm pr}_{2}\langle a \rangle
\]
が成り立つ.
\end{thm}


\noindent{\bf 証明}
~$x$と$y$を, 互いに異なり, 共に$a$及び$b$の中に自由変数として現れない文字とする.
また$y$は定数でないとする.
このとき定理 \ref{sthmvaluesetelement}と推論法則 \ref{dedequiv}により
\[
\tag{1}
  y \in a[b] \to \exists x(x \in b \wedge (x, y) \in a)
\]
が成り立つ.
またThm \ref{thmquanwedge}より
\[
\tag{2}
  \exists x(x \in b \wedge (x, y) \in a) \to \exists x((x, y) \in a)
\]
が成り立つ.
また$x$が$y$と異なり, $a$の中に自由変数として現れないことから, 
定理 \ref{sthmprsetelement}と推論法則 \ref{dedequiv}により
\[
\tag{3}
  \exists x((x, y) \in a) \to y \in {\rm pr}_{2}\langle a \rangle
\]
が成り立つ.
そこで(1), (2), (3)から, 推論法則 \ref{dedmmp}によって
\[
\tag{4}
  y \in a[b] \to y \in {\rm pr}_{2}\langle a \rangle
\]
が成り立つ.
いま$y$は$a$及び$b$の中に自由変数として現れないから, 
変数法則 \ref{valprset}, \ref{valvalueset}により, $y$は
$a[b]$及び${\rm pr}_{2}\langle a \rangle$の中に自由変数として現れない.
また$y$は定数でない.
そこでこのことと, (4)が成り立つことから, 定理 \ref{sthmsubsetconst}により
$a[b] \subset {\rm pr}_{2}\langle a \rangle$が成り立つ.
\halmos




\mathstrut
\begin{thm}
\label{sthmvaluesetpr12set}%定理
$a$を集合とするとき, 
\[
  a[{\rm pr}_{1}\langle a \rangle] = {\rm pr}_{2}\langle a \rangle
\]
が成り立つ.
\end{thm}


\noindent{\bf 証明}
~$x$と$y$を, 互いに異なり, 共に$a$の中に自由変数として現れない文字とする.
また$y$は定数でないとする.
このとき定理 \ref{sthmprsetelement}と推論法則 \ref{dedequiv}により
\[
  y \in {\rm pr}_{2}\langle a \rangle \to \exists x((x, y) \in a)
\]
が成り立つ.
ここで$\tau_{x}((x, y) \in a)$を$T$と書けば, 
定義から上記の記号列は
\[
  y \in {\rm pr}_{2}\langle a \rangle \to (T|x)((x, y) \in a)
\]
である.
また$x$が$y$と異なり, $a$の中に自由変数として現れないことから, 
代入法則 \ref{substfree}, \ref{substfund}, \ref{substpair}により, 
この記号列は
\[
\tag{1}
  y \in {\rm pr}_{2}\langle a \rangle \to (T, y) \in a
\]
と一致する.
よってこれが定理となる.
また定理 \ref{sthmpairelementinprset}より
\[
  (T, y) \in a \to T \in {\rm pr}_{1}\langle a \rangle \wedge y \in {\rm pr}_{2}\langle a \rangle
\]
が成り立つから, 推論法則 \ref{dedprewedge}によって
\[
  (T, y) \in a \to T \in {\rm pr}_{1}\langle a \rangle
\]
が成り立ち, これから推論法則 \ref{dedatawbtrue1}によって
\[
\tag{2}
  (T, y) \in a \to T \in {\rm pr}_{1}\langle a \rangle \wedge (T, y) \in a
\]
が成り立つ.
また定理 \ref{sthmvaluesetbasis}より
\[
\tag{3}
  T \in {\rm pr}_{1}\langle a \rangle \wedge (T, y) \in a \to y \in a[{\rm pr}_{1}\langle a \rangle]
\]
が成り立つ.
そこで(1), (2), (3)から, 推論法則 \ref{dedmmp}によって
\[
\tag{4}
  y \in {\rm pr}_{2}\langle a \rangle \to y \in a[{\rm pr}_{1}\langle a \rangle]
\]
が成り立つことがわかる.
いま$y$は$a$の中に自由変数として現れないから, 
変数法則 \ref{valprset}, \ref{valvalueset}により, 
$y$は${\rm pr}_{2}\langle a \rangle$及び$a[{\rm pr}_{1}\langle a \rangle]$の中に
自由変数として現れない.
また$y$は定数でない.
そこでこのことと, (4)が成り立つことから, 定理 \ref{sthmsubsetconst}により
\[
\tag{5}
  {\rm pr}_{2}\langle a \rangle \subset a[{\rm pr}_{1}\langle a \rangle]
\]
が成り立つ.
また定理 \ref{sthmvaluesetsubsetpr2set}より
\[
\tag{6}
  a[{\rm pr}_{1}\langle a \rangle] \subset {\rm pr}_{2}\langle a \rangle
\]
が成り立つ.
そこで(5)と(6)から, 定理 \ref{sthmaxiom1}によって
$a[{\rm pr}_{1}\langle a \rangle] = {\rm pr}_{2}\langle a \rangle$が
成り立つ.
\halmos




\mathstrut
\begin{thm}
\label{sthmvalueset=pr2set}%定理
$a$と$b$を集合とするとき, 
\[
  {\rm pr}_{1}\langle a \rangle \subset b \to a[b] = {\rm pr}_{2}\langle a \rangle
\]
が成り立つ.
またこのことから, 次の($*$)が成り立つ: 

($*$) ~~${\rm pr}_{1}\langle a \rangle \subset b$が成り立つならば, 
        $a[b] = {\rm pr}_{2}\langle a \rangle$が成り立つ.
\end{thm}


\noindent{\bf 証明}
~定理 \ref{sthmcapsubset=}と推論法則 \ref{dedequiv}により
\[
\tag{1}
  {\rm pr}_{1}\langle a \rangle \subset b \to {\rm pr}_{1}\langle a \rangle \cap b = {\rm pr}_{1}\langle a \rangle
\]
が成り立つ.
また定理 \ref{sthmvalueset=}より
\[
\tag{2}
  {\rm pr}_{1}\langle a \rangle \cap b = {\rm pr}_{1}\langle a \rangle \to 
  a[{\rm pr}_{1}\langle a \rangle \cap b] = a[{\rm pr}_{1}\langle a \rangle]
\]
が成り立つ.
また定理 \ref{sthmvaluesetcappr1set}より
\[
\tag{3}
  a[b] = a[b \cap {\rm pr}_{1}\langle a \rangle]
\]
が成り立つ.
また定理 \ref{sthmcapch}より
\[
  b \cap {\rm pr}_{1}\langle a \rangle = {\rm pr}_{1}\langle a \rangle \cap b
\]
が成り立つから, 定理 \ref{sthmvalueset=}により
\[
\tag{4}
  a[b \cap {\rm pr}_{1}\langle a \rangle] = a[{\rm pr}_{1}\langle a \rangle \cap b]
\]
が成り立つ.
そこで(3), (4)から, 推論法則 \ref{ded=trans}によって
\[
  a[b] = a[{\rm pr}_{1}\langle a \rangle \cap b]
\]
が成り立ち, これから推論法則 \ref{ded=ch}によって
\[
\tag{5}
  a[{\rm pr}_{1}\langle a \rangle \cap b] = a[b]
\]
が成り立つ.
また定理 \ref{sthmvaluesetpr12set}より
\[
\tag{6}
  a[{\rm pr}_{1}\langle a \rangle] = {\rm pr}_{2}\langle a \rangle
\]
が成り立つ.
そこで(5), (6)から, 推論法則 \ref{dedaddeq=}によって
\[
  a[{\rm pr}_{1}\langle a \rangle \cap b] = a[{\rm pr}_{1}\langle a \rangle] \leftrightarrow 
  a[b] = {\rm pr}_{2}\langle a \rangle
\]
が成り立ち, これから推論法則 \ref{dedequiv}によって
\[
\tag{7}
  a[{\rm pr}_{1}\langle a \rangle \cap b] = a[{\rm pr}_{1}\langle a \rangle] \to 
  a[b] = {\rm pr}_{2}\langle a \rangle
\]
が成り立つ.
そこで(1), (2), (7)から, 推論法則 \ref{dedmmp}によって
\[
  {\rm pr}_{1}\langle a \rangle \subset b \to a[b] = {\rm pr}_{2}\langle a \rangle
\]
が成り立つことがわかる.
($*$)が成り立つことは, これと推論法則 \ref{dedmp}によって明らかである.
\halmos




\mathstrut
$a$と$b$を集合とする.
集合$a[\{b\}]$を, $b$による$a$の\textbf{切断面}という.




\mathstrut
\begin{thm}
\label{sthmcutelement}%定理
$a$, $b$, $c$を集合とするとき, 
\[
  c \in a[\{b\}] \leftrightarrow (b, c) \in a
\]
が成り立つ.
\end{thm}


\noindent{\bf 証明}
~$x$を$a$, $b$, $c$のいずれの記号列の中にも自由変数として現れない, 
定数でない文字とする.
このとき変数法則 \ref{valnset}により, $x$は$\{b\}$の中にも自由変数として現れないから, 
定理 \ref{sthmvaluesetelement}より
\[
\tag{1}
  c \in a[\{b\}] \leftrightarrow \exists x(x \in \{b\} \wedge (x, c) \in a)
\]
が成り立つ.
また定理 \ref{sthmsingletonbasis}より
$x \in \{b\} \leftrightarrow x = b$が成り立つから, 
推論法則 \ref{dedaddeqw}により
\[
  x \in \{b\} \wedge (x, c) \in a \leftrightarrow x = b \wedge (x, c) \in a
\]
が成り立つ.
そこで$x$が定数でないことから, 推論法則 \ref{dedalleqquansepconst}により
\[
\tag{2}
  \exists x(x \in \{b\} \wedge (x, c) \in a) \leftrightarrow \exists x(x = b \wedge (x, c) \in a)
\]
が成り立つ.
また定理 \ref{sthmpairweak}と推論法則 \ref{dedequiv}により
$x = b \to (x, c) = (b, c)$が成り立つから, 
推論法則 \ref{dedaddw}により
\[
\tag{3}
  x = b \wedge (x, c) \in a \to (x, c) = (b, c) \wedge (x, c) \in a
\]
が成り立つ.
また定理 \ref{sthm=&in}より
\[
\tag{4}
  (x, c) = (b, c) \wedge (x, c) \in a \to (b, c) \in a
\]
が成り立つ.
そこで(3), (4)から, 推論法則 \ref{dedmmp}によって
\[
  x = b \wedge (x, c) \in a \to (b, c) \in a
\]
が成り立ち, これと$x$が定数でないことから, 推論法則 \ref{dedalltquansepconst}によって
\[
  \exists x(x = b \wedge (x, c) \in a) \to \exists x((b, c) \in a)
\]
が成り立つ.
ここで$x$が$a$, $b$, $c$のいずれの記号列の中にも自由変数として現れないことから, 
変数法則 \ref{valfund}, \ref{valpair}により, $x$は
$(b, c) \in a$の中に自由変数として現れないから, 
変形法則 \ref{quanfree}により, 上記の記号列は
\[
\tag{5}
  \exists x(x = b \wedge (x, c) \in a) \to (b, c) \in a
\]
と一致する.
よってこれが定理となる.
またThm \ref{x=x}より$b = b$が成り立つから, 推論法則 \ref{dedatawbtrue2}により
\[
  (b, c) \in a \to b = b \wedge (b, c) \in a
\]
が成り立つが, いま$x$は$a$, $b$, $c$のいずれの記号列の中にも自由変数として現れないから, 
代入法則 \ref{substfree}, \ref{substfund}, \ref{substwedge}, \ref{substpair}により, 
この記号列は
\[
\tag{6}
  (b, c) \in a \to (b|x)(x = b \wedge (x, c) \in a)
\]
と一致する.
よってこれが定理となる.
またschema S4の適用により
\[
\tag{7}
  (b|x)(x = b \wedge (x, c) \in a) \to \exists x(x = b \wedge (x, c) \in a)
\]
が成り立つ.
そこで(6), (7)から, 推論法則 \ref{dedmmp}によって
\[
\tag{8}
  (b, c) \in a \to \exists x(x = b \wedge (x, c) \in a)
\]
が成り立つ.
そこで(5), (8)から, 推論法則 \ref{dedequiv}によって
\[
\tag{9}
  \exists x(x = b \wedge (x, c) \in a) \leftrightarrow (b, c) \in a
\]
が成り立つ.
そして(1), (2), (9)から, 推論法則 \ref{dedeqtrans}によって
$c \in a[\{b\}] \leftrightarrow (b, c) \in a$が成り立つことがわかる.
\halmos




\mathstrut
\begin{thm}
\label{sthmcutsubset}%定理
$a$と$b$を集合とし, $x$をこれらの中に自由変数として現れない文字とする.
このとき
\[
  {\rm Graph}(a) \to (a \subset b \leftrightarrow \forall x(a[\{x\}] \subset b[\{x\}]))
\]
が成り立つ.
またこのことから, 次の1), 2), 3)が成り立つ.

1)
$a$がグラフならば, $a \subset b \leftrightarrow \forall x(a[\{x\}] \subset b[\{x\}])$が成り立つ.

2)
$a$がグラフで, $\forall x(a[\{x\}] \subset b[\{x\}])$が成り立つならば, $a \subset b$が成り立つ.

3)
$a$がグラフであるとき, $x$が定数でなく, $a[\{x\}] \subset b[\{x\}]$が成り立つならば, $a \subset b$が成り立つ.
\end{thm}


\noindent{\bf 証明}
~まず前半を示す.
$\tau_{x}(\neg (a[\{x\}] \subset b[\{x\}]))$を$T$と書けば, $T$は集合であり, 
定理 \ref{sthmvaluesetsubset}より
\[
\tag{1}
  a \subset b \to a[\{T\}] \subset b[\{T\}]
\]
が成り立つ.
また$T$の定義から, Thm \ref{thmallfund1}と推論法則 \ref{dedequiv}により
\[
  (T|x)(a[\{x\}] \subset b[\{x\}]) \to \forall x(a[\{x\}] \subset b[\{x\}])
\]
が成り立つが, いま$x$は$a$及び$b$の中に自由変数として現れないから, 
代入法則 \ref{substfree}, \ref{substsubset}, \ref{substnset}, \ref{substvalueset}により, 
この記号列は
\[
\tag{2}
  a[\{T\}] \subset b[\{T\}] \to \forall x(a[\{x\}] \subset b[\{x\}])
\]
と一致する.
よってこれが定理となる.
そこで(1), (2)から, 推論法則 \ref{dedmmp}によって
\[
  a \subset b \to \forall x(a[\{x\}] \subset b[\{x\}])
\]
が成り立ち, これから推論法則 \ref{deds1}によって
\[
\tag{3}
  {\rm Graph}(a) \to (a \subset b \to \forall x(a[\{x\}] \subset b[\{x\}]))
\]
が成り立つ.
いま$y$を, $x$と異なり, $a$及び$b$の中に自由変数として現れない文字とする.
このとき$\tau_{x}(\neg \forall y((x, y) \in a \to (x, y) \in b))$を$U$と書けば, 
$U$は集合であり, 変数法則 \ref{valfund}, \ref{valtau}, \ref{valquan}によってわかるように, $y$はこの中に自由変数として現れない.
そしてThm \ref{thmallfund2}より
\[
  \forall x(a[\{x\}] \subset b[\{x\}]) \to (U|x)(a[\{x\}] \subset b[\{x\}])
\]
が成り立つ.
ここで$x$が$a$及び$b$の中に自由変数として現れないことから, 
代入法則 \ref{substfree}, \ref{substsubset}, \ref{substnset}, \ref{substvalueset}により, 上記の記号列は
\[
\tag{4}
  \forall x(a[\{x\}] \subset b[\{x\}]) \to a[\{U\}] \subset b[\{U\}]
\]
と一致する.
よってこれが定理となる.
またいま$\tau_{y}(\neg ((U, y) \in a \to (U, y) \in b))$を$V$と書けば, $V$は集合であり, 
定理 \ref{sthmsubsetbasis}より
\[
\tag{5}
  a[\{U\}] \subset b[\{U\}] \to (V \in a[\{U\}] \to V \in b[\{U\}])
\]
が成り立つ.
また定理 \ref{sthmcutelement}と推論法則 \ref{dedequiv}により
\begin{align*}
  &(U, V) \in a \to V \in a[\{U\}], \\
  \mbox{} \\
  &V \in b[\{U\}] \to (U, V) \in b
\end{align*}
が共に成り立つから, この前者から, 推論法則 \ref{dedaddf}によって
\[
\tag{6}
  (V \in a[\{U\}] \to V \in b[\{U\}]) \to ((U, V) \in a \to V \in b[\{U\}])
\]
が成り立ち, 後者から, 推論法則 \ref{dedaddb}によって
\[
\tag{7}
  ((U, V) \in a \to V \in b[\{U\}]) \to ((U, V) \in a \to (U, V) \in b)
\]
が成り立つ.
また$V$の定義から, Thm \ref{thmallfund1}と推論法則 \ref{dedequiv}により
\[
  (V|y)((U, y) \in a \to (U, y) \in b) \to \forall y((U, y) \in a \to (U, y) \in b)
\]
が成り立つ.
ここで$y$が$a$及び$b$の中に自由変数として現れず, 
上述のように$U$の中にも自由変数として現れないことから, 
代入法則 \ref{substfree}, \ref{substfund}, \ref{substpair}により, 上記の記号列は
\[
  ((U, V) \in a \to (U, V) \in b) \to \forall y((U, y) \in a \to (U, y) \in b)
\]
と一致する.
また$x$が$y$と異なり, $a$及び$b$の中に自由変数として現れないことから, 
同じく代入法則 \ref{substfree}, \ref{substfund}, \ref{substpair}により, この記号列は
\[
  ((U, V) \in a \to (U, V) \in b) \to \forall y((U|x)((x, y) \in a \to (x, y) \in b))
\]
と一致する.
また$y$が$x$と異なり, 上述のように$U$の中に自由変数として現れないことから, 
代入法則 \ref{substquan}により, この記号列は
\[
\tag{8}
  ((U, V) \in a \to (U, V) \in b) \to (U|x)(\forall y((x, y) \in a \to (x, y) \in b))
\]
と一致する.
故にこれが定理となる.
また$U$の定義から, Thm \ref{thmallfund1}と推論法則 \ref{dedequiv}により
\[
\tag{9}
  (U|x)(\forall y((x, y) \in a \to (x, y) \in b)) \to \forall x(\forall y((x, y) \in a \to (x, y) \in b))
\]
が成り立つ.
そこで(4)---(9)から, 推論法則 \ref{dedmmp}によって
\[
  \forall x(a[\{x\}] \subset b[\{x\}]) \to \forall x(\forall y((x, y) \in a \to (x, y) \in b))
\]
が成り立つことがわかり, これから推論法則 \ref{dedaddw}によって
\[
\tag{10}
  {\rm Graph}(a) \wedge \forall x(a[\{x\}] \subset b[\{x\}]) \to {\rm Graph}(a) \wedge \forall x(\forall y((x, y) \in a \to (x, y) \in b))
\]
が成り立つ.
また$x$と$y$が互いに異なり, 共に$a$及び$b$の中に自由変数として現れないことから, 
定理 \ref{sthmgraphpairsubset}より
\[
  {\rm Graph}(a) \to (a \subset b \leftrightarrow \forall x(\forall y((x, y) \in a \to (x, y) \in b)))
\]
が成り立つ.
そこで推論法則 \ref{dedprewedge}により
\[
  {\rm Graph}(a) \to (\forall x(\forall y((x, y) \in a \to (x, y) \in b)) \to a \subset b)
\]
が成り立ち, これから推論法則 \ref{dedtwch}によって
\[
\tag{11}
  {\rm Graph}(a) \wedge \forall x(\forall y((x, y) \in a \to (x, y) \in b)) \to a \subset b
\]
が成り立つ.
そこで(10), (11)から, 推論法則 \ref{dedmmp}によって
\[
  {\rm Graph}(a) \wedge \forall x(a[\{x\}] \subset b[\{x\}]) \to a \subset b
\]
が成り立ち, これから推論法則 \ref{dedtwch}によって
\[
\tag{12}
  {\rm Graph}(a) \to (\forall x(a[\{x\}] \subset b[\{x\}]) \to a \subset b)
\]
が成り立つ.
そして(3), (12)から, 推論法則 \ref{dedprewedge}によって
\[
\tag{13}
  {\rm Graph}(a) \to (a \subset b \leftrightarrow \forall x(a[\{x\}] \subset b[\{x\}]))
\]
が成り立つ.

\noindent
1)
上に示したように(13)が成り立つから, 1)が成り立つことはこれと推論法則 \ref{dedmp}によって明らかである.

\noindent
2)
このとき1)により$a \subset b \leftrightarrow \forall x(a[\{x\}] \subset b[\{x\}])$が成り立つから, 
2)が成り立つことはこれと推論法則 \ref{dedeqfund}によって明らかである.

\noindent
3)
このとき推論法則 \ref{dedltthmquan}によって
$\forall x(a[\{x\}] \subset b[\{x\}])$が成り立つから, 2)によって3)が成り立つ.
\halmos




\mathstrut
\begin{thm}
\label{sthmcut=}%定理
$a$と$b$を集合とし, $x$をこれらの中に自由変数として現れない文字とする.
このとき
\[
  {\rm Graph}(a) \wedge {\rm Graph}(b) \to (a = b \leftrightarrow \forall x(a[\{x\}] = b[\{x\}]))
\]
が成り立つ.
またこのことから, 次の1), 2), 3)が成り立つ.

1)
$a$と$b$が共にグラフならば, $a = b \leftrightarrow \forall x(a[\{x\}] = b[\{x\}])$が成り立つ.

2)
$a$と$b$が共にグラフで, $\forall x(a[\{x\}] = b[\{x\}])$が成り立つならば, $a = b$が成り立つ.

3)
$a$と$b$が共にグラフであるとき, $x$が定数でなく, $a[\{x\}] = b[\{x\}]$が成り立つならば, $a = b$が成り立つ.
\end{thm}


\noindent{\bf 証明}
~$\tau_{x}(\neg (a[\{x\}] = b[\{x\}]))$を$T$と書けば, $T$は集合であり, 
定理 \ref{sthmvalueset=}より
\[
\tag{1}
  a = b \to a[\{T\}] = b[\{T\}]
\]
が成り立つ.
また$T$の定義から, Thm \ref{thmallfund1}と推論法則 \ref{dedequiv}により
\[
  (T|x)(a[\{x\}] = b[\{x\}]) \to \forall x(a[\{x\}] = b[\{x\}])
\]
が成り立つ.
ここで$x$が$a$及び$b$の中に自由変数として現れないことから, 
代入法則 \ref{substfree}, \ref{substfund}, \ref{substnset}, \ref{substvalueset}により, 
この記号列は
\[
\tag{2}
  a[\{T\}] = b[\{T\}] \to \forall x(a[\{x\}] = b[\{x\}])
\]
と一致する.
よってこれが定理となる.
そこで(1), (2)から, 推論法則 \ref{dedmmp}によって
\[
  a = b \to \forall x(a[\{x\}] = b[\{x\}])
\]
が成り立ち, これから推論法則 \ref{deds1}によって
\[
\tag{3}
  {\rm Graph}(a) \wedge {\rm Graph}(b) \to (a = b \to \forall x(a[\{x\}] = b[\{x\}]))
\]
が成り立つ.
また$x$が$a$及び$b$の中に自由変数として現れないことから, 定理 \ref{sthmcutsubset}より
\begin{align*}
  {\rm Graph}(a) &\to (a \subset b \leftrightarrow \forall x(a[\{x\}] \subset b[\{x\}])), \\
  \mbox{} \\
  {\rm Graph}(b) &\to (b \subset a \leftrightarrow \forall x(b[\{x\}] \subset a[\{x\}]))
\end{align*}
が共に成り立つ.
そこでこれらから, 推論法則 \ref{dedprewedge}によって
\begin{align*}
  {\rm Graph}(a) &\to (\forall x(a[\{x\}] \subset b[\{x\}]) \to a \subset b), \\
  \mbox{} \\
  {\rm Graph}(b) &\to (\forall x(b[\{x\}] \subset a[\{x\}]) \to b \subset a)
\end{align*}
が共に成り立ち, これらから推論法則 \ref{dedfromaddw}によって
\[
\tag{4}
  {\rm Graph}(a) \wedge {\rm Graph}(b) \to 
  (\forall x(a[\{x\}] \subset b[\{x\}]) \to a \subset b) \wedge (\forall x(b[\{x\}] \subset a[\{x\}]) \to b \subset a)
\]
が成り立つ.
またThm \ref{1atb1w1ctd1t1awctbwd1}より
\begin{multline*}
\tag{5}
  (\forall x(a[\{x\}] \subset b[\{x\}]) \to a \subset b) \wedge (\forall x(b[\{x\}] \subset a[\{x\}]) \to b \subset a) \\
  \to (\forall x(a[\{x\}] \subset b[\{x\}]) \wedge \forall x(b[\{x\}] \subset a[\{x\}]) \to a \subset b \wedge b \subset a)
\end{multline*}
が成り立つ.
ここで$u$を, $x$と異なり, $a$及び$b$の中に自由変数として現れない, 定数でない文字とする.
このとき定理 \ref{sthmaxiom1}と推論法則 \ref{dedequiv}により
\[
  a[\{u\}] = b[\{u\}] \to a[\{u\}] \subset b[\{u\}] \wedge b[\{u\}] \subset a[\{u\}]
\]
が成り立ち, これと$u$が定数でないことから, 推論法則 \ref{dedalltquansepconst}により
\[
  \forall u(a[\{u\}] = b[\{u\}]) \to \forall u(a[\{u\}] \subset b[\{u\}] \wedge b[\{u\}] \subset a[\{u\}])
\]
が成り立つ.
いま$x$は$u$と異なり, $a$及び$b$の中に自由変数として現れないから, 
変数法則 \ref{valfund}, \ref{valwedge}, \ref{valsubset}, 
\ref{valnset}, \ref{valvalueset}によってわかるように, 
$x$は$a[\{u\}] = b[\{u\}]$及び$a[\{u\}] \subset b[\{u\}] \wedge b[\{u\}] \subset a[\{u\}]$の中に
自由変数として現れない.
そこで代入法則 \ref{substquantrans}により, 上記の記号列は
\[
  \forall x((x|u)(a[\{u\}] = b[\{u\}])) \to \forall x((x|u)(a[\{u\}] \subset b[\{u\}] \wedge b[\{u\}] \subset a[\{u\}]))
\]
と一致する.
また$u$が$a$及び$b$の中に自由変数として現れないことから, 
代入法則 \ref{substfree}, \ref{substfund}, \ref{substwedge}, 
\ref{substsubset}, \ref{substnset}, \ref{substvalueset}により, 
この記号列は
\[
\tag{6}
  \forall x(a[\{x\}] = b[\{x\}]) \to \forall x(a[\{x\}] \subset b[\{x\}] \wedge b[\{x\}] \subset a[\{x\}])
\]
と一致する.
よってこれが定理となる.
またThm \ref{thmallw}と推論法則 \ref{dedequiv}により
\[
\tag{7}
  \forall x(a[\{x\}] \subset b[\{x\}] \wedge b[\{x\}] \subset a[\{x\}]) \to 
  \forall x(a[\{x\}] \subset b[\{x\}]) \wedge \forall x(b[\{x\}] \subset a[\{x\}])
\]
が成り立つ.
そこで(6), (7)から, 推論法則 \ref{dedmmp}によって
\[
  \forall x(a[\{x\}] = b[\{x\}]) \to 
  \forall x(a[\{x\}] \subset b[\{x\}]) \wedge \forall x(b[\{x\}] \subset a[\{x\}])
\]
が成り立ち, これから推論法則 \ref{dedaddf}によって
\begin{multline*}
\tag{8}
  (\forall x(a[\{x\}] \subset b[\{x\}]) \wedge \forall x(b[\{x\}] \subset a[\{x\}]) \to a \subset b \wedge b \subset a) \\
  \to (\forall x(a[\{x\}] = b[\{x\}]) \to a \subset b \wedge b \subset a)
\end{multline*}
が成り立つ.
また定理 \ref{sthmaxiom1}と推論法則 \ref{dedequiv}により
\[
  a \subset b \wedge b \subset a \to a = b
\]
が成り立つから, 推論法則 \ref{dedaddb}により
\[
\tag{9}
  (\forall x(a[\{x\}] = b[\{x\}]) \to a \subset b \wedge b \subset a) \to (\forall x(a[\{x\}] = b[\{x\}]) \to a = b)
\]
が成り立つ.
そこで(4), (5), (8), (9)から, 推論法則 \ref{dedmmp}によって
\[
\tag{10}
  {\rm Graph}(a) \wedge {\rm Graph}(b) \to (\forall x(a[\{x\}] = b[\{x\}]) \to a = b)
\]
が成り立つことがわかる.
そして(3), (10)から, 推論法則 \ref{dedprewedge}によって
\[
\tag{11}
  {\rm Graph}(a) \wedge {\rm Graph}(b) \to (a = b \leftrightarrow \forall x(a[\{x\}] = b[\{x\}]))
\]
が成り立つ.

\noindent
1)
このとき推論法則 \ref{dedwedge}により${\rm Graph}(a) \wedge {\rm Graph}(b)$が成り立つから, 
これと(11)から, 推論法則 \ref{dedmp}によって
$a = b \leftrightarrow \forall x(a[\{x\}] = b[\{x\}])$が成り立つ.

\noindent
2)
このとき1)より$a = b \leftrightarrow \forall x(a[\{x\}] = b[\{x\}])$が成り立つから, 
2)が成り立つことはこれと推論法則 \ref{dedeqfund}によって明らかである.

\noindent
3)
このとき推論法則 \ref{dedltthmquan}によって
$\forall x(a[\{x\}] = b[\{x\}])$が成り立つから, 2)によって3)が成り立つ.
\halmos




\mathstrut
\begin{thm}
\label{sthmcutempty}%定理%証明洗練できそう
$a$と$b$を集合とするとき, 
\[
  b \notin {\rm pr}_{1}\langle a \rangle \leftrightarrow a[\{b\}] = \phi
\]
が成り立つ.
またこのことから, 次の($*$)が成り立つ: 

($*$) ~~$b \notin {\rm pr}_{1}\langle a \rangle$が成り立つならば, $a[\{b\}]$は空である.
        逆に$a[\{b\}]$が空ならば, $b \notin {\rm pr}_{1}\langle a \rangle$が成り立つ.
\end{thm}


\noindent{\bf 証明}
~定理 \ref{sthmsingletonfund}より
$b \in \{b\}$が成り立つから, 推論法則 \ref{dedatawbtrue2}により
\[
\tag{1}
  b \in {\rm pr}_{1}\langle a \rangle \to b \in \{b\} \wedge b \in {\rm pr}_{1}\langle a \rangle
\]
が成り立つ.
また定理 \ref{sthmcapelement}と推論法則 \ref{dedequiv}により
\[
  b \in \{b\} \wedge b \in {\rm pr}_{1}\langle a \rangle \to b \in \{b\} \cap {\rm pr}_{1}\langle a \rangle
\]
が成り立つ.
また定理 \ref{sthmnotemptyeqexin}より
\[
\tag{3}
  b \in \{b\} \cap {\rm pr}_{1}\langle a \rangle \to \{b\} \cap {\rm pr}_{1}\langle a \rangle \neq \phi
\]
が成り立つ.
そこで(1), (2), (3)から, 推論法則 \ref{dedmmp}によって
\[
\tag{4}
  b \in {\rm pr}_{1}\langle a \rangle \to \{b\} \cap {\rm pr}_{1}\langle a \rangle \neq \phi
\]
が成り立つことがわかる.
またいま$x$を$a$及び$b$の中に自由変数として現れない文字とし, 
$\tau_{x}(x \in \{b\} \cap {\rm pr}_{1}\langle a \rangle)$を$T$と書く.
このとき$T$は集合であり, 変数法則 \ref{valnset}, \ref{valcap}, \ref{valprset}によってわかるように, 
$x$は$\{b\} \cap {\rm pr}_{1}\langle a \rangle$の中に自由変数として現れないから, 
定理 \ref{sthmelm&empty}と推論法則 \ref{dedequiv}により
\[
\tag{5}
  \{b\} \cap {\rm pr}_{1}\langle a \rangle \neq \phi \to T \in \{b\} \cap {\rm pr}_{1}\langle a \rangle
\]
が成り立つ.
また定理 \ref{sthmcapelement}と推論法則 \ref{dedequiv}により
\[
\tag{6}
  T \in \{b\} \cap {\rm pr}_{1}\langle a \rangle \to T \in \{b\} \wedge T \in {\rm pr}_{1}\langle a \rangle
\]
が成り立つ.
また定理 \ref{sthmsingletonbasis}と推論法則 \ref{dedequiv}により
$T \in \{b\} \to T = b$が成り立つから, 推論法則 \ref{dedaddw}により
\[
\tag{7}
  T \in \{b\} \wedge T \in {\rm pr}_{1}\langle a \rangle \to T = b \wedge T \in {\rm pr}_{1}\langle a \rangle
\]
が成り立つ.
また定理 \ref{sthm=&in}より
\[
\tag{8}
  T = b \wedge T \in {\rm pr}_{1}\langle a \rangle \to b \in {\rm pr}_{1}\langle a \rangle
\]
が成り立つ.
そこで(5)---(8)から, 推論法則 \ref{dedmmp}によって
\[
\tag{9}
  \{b\} \cap {\rm pr}_{1}\langle a \rangle \neq \phi \to b \in {\rm pr}_{1}\langle a \rangle
\]
が成り立つことがわかる.
そこで(4), (9)から, 推論法則 \ref{dedequiv}によって
\[
  b \in {\rm pr}_{1}\langle a \rangle \leftrightarrow \{b\} \cap {\rm pr}_{1}\langle a \rangle \neq \phi
\]
が成り立ち, これから推論法則 \ref{dedeqcp}によって
\[
\tag{10}
  b \notin {\rm pr}_{1}\langle a \rangle \leftrightarrow \{b\} \cap {\rm pr}_{1}\langle a \rangle = \phi
\]
が成り立つ.
また定理 \ref{sthmvaluesetemptycon}と推論法則 \ref{dedeqch}により
\[
\tag{11}
  \{b\} \cap {\rm pr}_{1}\langle a \rangle = \phi \leftrightarrow a[\{b\}] = \phi
\]
が成り立つ.
そこで(10), (11)から, 推論法則 \ref{dedeqtrans}によって
\[
  b \notin {\rm pr}_{1}\langle a \rangle \leftrightarrow a[\{b\}] = \phi
\]
が成り立つ.
($*$)が成り立つことは, これと推論法則 \ref{dedeqfund}によって明らかである.
\halmos
%section8確認済み



\newpage
\setcounter{defi}{0}
\section{逆グラフ, グラフの合成}%%%%%%%%%%%%%%%%%%%%%%%%%%%%%%%%%%%%%%%%%%%%%%%%%%%%%%%%%%%%%%%




\mathstrut
\begin{defo}
\label{inverse}%変形
$a$を記号列とする.
また$x$, $y$, $z$を, どの二つも互いに異なり, いずれも$a$の中に自由変数として現れない文字とする.
同様に, $u$, $v$, $w$を, どの二つも互いに異なり, いずれも$a$の中に自由変数として現れない文字とする.
このとき
\[
  \{z|\exists x(\exists y((x, y) \in a \wedge z = (y, x)))\} \equiv 
  \{w|\exists u(\exists v((u, v) \in a \wedge w = (v, u)))\}
\]
が成り立つ.
\end{defo}


\noindent{\bf 証明}
~$p$, $q$, $r$を, どの二つも互いに異なり, 
どの一つも$x$, $y$, $z$, $u$, $v$, $w$のいずれとも異なり, 
$a$の中に自由変数として現れない文字とする.
このとき変数法則 \ref{valfund}, \ref{valwedge}, \ref{valquan}, \ref{valpair}によって
わかるように, $r$は$\exists x(\exists y((x, y) \in a \wedge z = (y, x)))$の中に
自由変数として現れないから, 代入法則 \ref{substisettrans}により
\[
\tag{1}
  \{z|\exists x(\exists y((x, y) \in a \wedge z = (y, x)))\} \equiv 
  \{r|(r|z)(\exists x(\exists y((x, y) \in a \wedge z = (y, x))))\}
\]
が成り立つ.
また$x$と$y$が共に$z$とも$r$とも異なることから, 
代入法則 \ref{substquan}により
\[
\tag{2}
  (r|z)(\exists x(\exists y((x, y) \in a \wedge z = (y, x)))) \equiv 
  \exists x(\exists y((r|z)((x, y) \in a \wedge z = (y, x))))
\]
が成り立つ.
また$z$が$x$とも$y$とも異なることから, 変数法則 \ref{valpair}により, 
$z$は$(x, y)$及び$(y, x)$の中に自由変数として現れない.
このことと, $z$が$a$の中にも自由変数として現れないことから, 
代入法則 \ref{substfree}, \ref{substfund}, \ref{substwedge}により
\[
\tag{3}
  (r|z)((x, y) \in a \wedge z = (y, x)) \equiv 
  (x, y) \in a \wedge r = (y, x)
\]
が成り立つ.
そこで(1), (2), (3)から, 
\[
\tag{4}
  \{z|\exists x(\exists y((x, y) \in a \wedge z = (y, x)))\} \equiv 
  \{r|\exists x(\exists y((x, y) \in a \wedge r = (y, x)))\}
\]
が成り立つことがわかる.
また$p$が$x$, $y$, $r$のいずれとも異なり, $a$の中に自由変数として現れないことから, 
変数法則 \ref{valfund}, \ref{valwedge}, \ref{valquan}, \ref{valpair}によって
$p$が$\exists y((x, y) \in a \wedge r = (y, x))$の中に自由変数として現れないことがわかるから, 
代入法則 \ref{substquantrans}により
\[
\tag{5}
  \exists x(\exists y((x, y) \in a \wedge r = (y, x))) \equiv 
  \exists p((p|x)(\exists y((x, y) \in a \wedge r = (y, x))))
\]
が成り立つ.
また$y$が$x$とも$p$とも異なることから, 
代入法則 \ref{substquan}により
\[
\tag{6}
  (p|x)(\exists y((x, y) \in a \wedge r = (y, x))) \equiv 
  \exists y((p|x)((x, y) \in a \wedge r = (y, x)))
\]
が成り立つ.
また$x$が$y$とも$r$とも異なり, $a$の中に自由変数として現れないことから, 
代入法則 \ref{substfree}, \ref{substfund}, \ref{substwedge}, \ref{substpair}により
\[
\tag{7}
  (p|x)((x, y) \in a \wedge r = (y, x)) \equiv 
  (p, y) \in a \wedge r = (y, p)
\]
が成り立つ.
そこで(5), (6), (7)から, 
\[
\tag{8}
  \{r|\exists x(\exists y((x, y) \in a \wedge r = (y, x)))\} \equiv 
  \{r|\exists p(\exists y((p, y) \in a \wedge r = (y, p)))\}
\]
が成り立つことがわかる.
また$q$が$y$, $p$, $r$のいずれとも異なり, $a$の中に自由変数として現れないことから, 
変数法則 \ref{valfund}, \ref{valwedge}, \ref{valpair}によって$q$が
$(p, y) \in a \wedge r = (y, p)$の中に自由変数として現れないことがわかるから, 
代入法則 \ref{substquantrans}により
\[
\tag{9}
  \exists y((p, y) \in a \wedge r = (y, p)) \equiv 
  \exists q((q|y)((p, y) \in a \wedge r = (y, p)))
\]
が成り立つ.
また$y$が$p$とも$r$とも異なり, $a$の中に自由変数として現れないことから, 
代入法則 \ref{substfree}, \ref{substfund}, \ref{substwedge}, \ref{substpair}により
\[
\tag{10}
  (q|y)((p, y) \in a \wedge r = (y, p)) \equiv 
  (p, q) \in a \wedge r = (q, p)
\]
が成り立つ.
そこで(9), (10)から, 
\[
\tag{11}
  \{r|\exists p(\exists y((p, y) \in a \wedge r = (y, p)))\} \equiv 
  \{r|\exists p(\exists q((p, q) \in a \wedge r = (q, p)))\}
\]
が成り立つことがわかる.
以上の(4), (8), (11)から, 
$\{z|\exists x(\exists y((x, y) \in a \wedge z = (y, x)))\}$が
$\{r|\exists p(\exists q((p, q) \in a \wedge r = (q, p)))\}$と一致することがわかる.
ここまでの議論と全く同様にして, 
$\{w|\exists u(\exists v((u, v) \in a \wedge w = (v, u)))\}$も
$\{r|\exists p(\exists q((p, q) \in a \wedge r = (q, p)))\}$と一致することがわかるから, 
従って本法則が成り立つ.
\halmos




\mathstrut
\begin{defi}
\label{definv}%定義
$a$を記号列とする.
また$x$, $y$, $z$を, どの二つも互いに異なり, いずれも$a$の中に自由変数として現れない文字とする.
同様に, $u$, $v$, $w$を, どの二つも互いに異なり, いずれも$a$の中に自由変数として現れない文字とする.
このとき上記の変形法則 \ref{inverse}によれば, 
$\{z|\exists x(\exists y((x, y) \in a \wedge z = (y, x)))\}$と
$\{w|\exists u(\exists v((u, v) \in a \wedge w = (v, u)))\}$という二つの記号列は一致する.
$a$に対して定まるこの記号列を, $(a)^{-1}$と記す(括弧は適宜省略する).
\end{defi}




\mathstrut
\begin{valu}
\label{valinv}%変数
$a$を記号列とし, $x$を文字とする.
$x$が$a$の中に自由変数として現れなければ, $x$は
$a^{-1}$の中に自由変数として現れない.
\end{valu}


\noindent{\bf 証明}
~$u$と$v$を, 互いに異なり, 共に$x$と異なり, $a$の中に自由変数として現れない文字とすれば, 
定義から$a^{-1}$は$\{x|\exists u(\exists v((u, v) \in a \wedge x = (v, u)))\}$と同じである.
変数法則 \ref{valiset}により, $x$はこの中に自由変数として現れない.
\halmos




\mathstrut
\begin{subs}
\label{substinv}%代入
$a$と$b$を記号列とし, $x$を文字とするとき, 
\[
  (b|x)(a^{-1}) \equiv (b|x)(a)^{-1}
\]
が成り立つ.
\end{subs}


\noindent{\bf 証明}
~$u$, $v$, $w$を, どの二つも互いに異なり, いずれも$x$と異なり, $a$及び$b$の中に自由変数として現れない文字とする.
このとき定義から$a^{-1}$は$\{w|\exists u(\exists v((u, v) \in a \wedge w = (v, u)))\}$と同じだから, 
$w$が$x$と異なり, $b$の中に自由変数として現れないということから, 
代入法則 \ref{substiset}により
\[
\tag{1}
  (b|x)(a^{-1}) \equiv \{w|(b|x)(\exists u(\exists v((u, v) \in a \wedge w = (v, u))))\}
\]
が成り立つ.
また$u$と$v$が共に$x$と異なり, $b$の中に自由変数として現れないことから, 
代入法則 \ref{substquan}により
\[
\tag{2}
  (b|x)(\exists u(\exists v((u, v) \in a \wedge w = (v, u)))) \equiv 
  \exists u(\exists v((b|x)((u, v) \in a \wedge w = (v, u))))
\]
が成り立つ.
また$x$が$u$, $v$, $w$のいずれとも異なることから, 変数法則 \ref{valfund}, \ref{valpair}により
$x$は$(u, v)$及び$w = (v, u)$の中に自由変数として現れないから, 
このことと代入法則 \ref{substfree}, \ref{substfund}, \ref{substwedge}により
\[
\tag{3}
  (b|x)((u, v) \in a \wedge w = (v, u)) \equiv 
  (u, v) \in (b|x)(a) \wedge w = (v, u)
\]
が成り立つ.
そこで(1), (2), (3)から, 
\[
\tag{4}
  (b|x)(a^{-1}) \equiv \{w|\exists u(\exists v((u, v) \in (b|x)(a) \wedge w = (v, u)))\}
\]
が成り立つことがわかる.
いま$u$, $v$, $w$はいずれも$a$及び$b$の中に自由変数として現れないから, 
変数法則 \ref{valsubst}により, これらはいずれも$(b|x)(a)$の中に自由変数として現れない.
また$u$, $v$, $w$はどの二つも互いに異なる.
そこで定義から, 
$\{w|\exists u(\exists v((u, v) \in (b|x)(a) \wedge w = (v, u)))\}$は
$(b|x)(a)^{-1}$と書き表される記号列である.
このことと(4)から, 
$(b|x)(a^{-1})$が$(b|x)(a)^{-1}$と一致することがわかる.
\halmos




\mathstrut
\begin{form}
\label{forminv}%構成
$a$が集合ならば, $a^{-1}$は集合である.
\end{form}


\noindent{\bf 証明}
~$x$, $y$, $z$を, どの二つも互いに異なり, いずれも$a$の中に自由変数として現れない文字とすれば, 
定義から$a^{-1}$は$\{z|\exists x(\exists y((x, y) \in a \wedge z = (y, x)))\}$と同じである.
$a$が集合であるとき, 
構成法則 \ref{formfund}, \ref{formwedge}, \ref{formquan}, \ref{formiset}, \ref{formpair}によって
直ちにわかるように, これは集合である.
\halmos




\mathstrut
$a$が集合であるとき, 集合$a^{-1}$を, $a$の\textbf{逆}という.




\mathstrut
\begin{thm}
\label{sthminvsetmake}%定理
$a$を集合とする.
また$x$, $y$, $z$を, どの二つも互いに異なり, いずれも$a$の中に
自由変数として現れない文字とする.
このとき, 関係式$\exists x(\exists y((x, y) \in a \wedge z = (y, x)))$は
$z$について集合を作り得る.
\end{thm}


\noindent{\bf 証明}
~$\exists x(\exists y((x, y) \in a \wedge z = (y, x)))$を$R$と書く.
また$u$を$x$, $y$, $z$のいずれとも異なり, $a$の中に自由変数として現れない, 定数でない文字とする.
このとき変数法則 \ref{valproduct}, \ref{valprset}により, $u$は
${\rm pr}_{2}\langle a \rangle \times {\rm pr}_{1}\langle a \rangle$の中に自由変数として現れない.
また変数法則 \ref{valfund}, \ref{valwedge}, \ref{valquan}, \ref{valpair}からわかるように, 
$u$は$R$の中にも自由変数として現れない.
そして$x$と$y$が共に$z$とも$u$とも異なることから, 代入法則 \ref{substquan}により
\[
  (u|z)(R) \equiv \exists x(\exists y((u|z)((x, y) \in a \wedge z = (y, x))))
\]
が成り立つ.
また$z$が$x$とも$y$とも異なることから, 変数法則 \ref{valpair}により$z$は
$(x, y)$及び$(y, x)$の中に自由変数として現れないから, 
このことと$z$が$a$の中にも自由変数として現れないことから, 
代入法則 \ref{substfree}, \ref{substfund}, \ref{substwedge}により
\[
  (u|z)((x, y) \in a \wedge z = (y, x)) \equiv (x, y) \in a \wedge u = (y, x)
\]
が成り立つ.
そこでこれらから, 
\[
\tag{1}
  (u|z)(R) \equiv \exists x(\exists y((x, y) \in a \wedge u = (y, x)))
\]
が成り立つことがわかる.
またいま$\tau_{x}(\exists y((x, y) \in a \wedge u = (y, x)))$を$T$と書けば, $T$は集合であり, 
変数法則 \ref{valtau}, \ref{valquan}によってわかるように, $y$は$T$の中に自由変数として現れない.
そして定義から
\[
\tag{2}
  \exists x(\exists y((x, y) \in a \wedge u = (y, x))) \equiv (T|x)(\exists y((x, y) \in a \wedge u = (y, x)))
\]
である.
また$y$が$x$と異なり, いま述べたように$T$の中に自由変数として現れないことから, 
代入法則 \ref{substquan}により
\[
\tag{3}
  (T|x)(\exists y((x, y) \in a \wedge u = (y, x))) \equiv \exists y((T|x)((x, y) \in a \wedge u = (y, x)))
\]
が成り立つ.
また$x$が$y$とも$u$とも異なり, $a$の中に自由変数として現れないことから, 
代入法則 \ref{substfree}, \ref{substfund}, \ref{substwedge}, \ref{substpair}により
\[
\tag{4}
  (T|x)((x, y) \in a \wedge u = (y, x)) \equiv (T, y) \in a \wedge u = (y, T)
\]
が成り立つ.
そこで(1)---(4)から, 
\[
\tag{5}
  (u|z)(R) \equiv \exists y((T, y) \in a \wedge u = (y, T))
\]
が成り立つことがわかる.
またいま$\tau_{y}((T, y) \in a \wedge u = (y, T))$を$U$と書けば, $U$は集合であり, 
定義から
\[
\tag{6}
  \exists y((T, y) \in a \wedge u = (y, T)) \equiv (U|y)((T, y) \in a \wedge u = (y, T))
\]
である.
また$y$が$u$と異なり, $a$の中に自由変数として現れず, 上述のように$T$の中にも
自由変数として現れないことから, 
代入法則 \ref{substfree}, \ref{substfund}, \ref{substwedge}, \ref{substpair}により
\[
\tag{7}
  (U|y)((T, y) \in a \wedge u = (y, T)) \equiv (T, U) \in a \wedge u = (U, T)
\]
が成り立つ.
そこで(5), (6), (7)から, 
\[
\tag{8}
  (u|z)(R) \equiv (T, U) \in a \wedge u = (U, T)
\]
が成り立つことがわかる.
さていまThm \ref{awbtbwa}より
\[
  (T, U) \in a \wedge u = (U, T) \to u = (U, T) \wedge (T, U) \in a
\]
が成り立つが, (8)からこの記号列は
\[
\tag{9}
  (u|z)(R) \to u = (U, T) \wedge (T, U) \in a
\]
と一致することがわかるから, これが定理となる.
また定理 \ref{sthmpairelementinprset}より
\[
\tag{10}
  (T, U) \in a \to T \in {\rm pr}_{1}\langle a \rangle \wedge U \in {\rm pr}_{2}\langle a \rangle
\]
が成り立つ.
またThm \ref{awbtbwa}より
\[
\tag{11}
  T \in {\rm pr}_{1}\langle a \rangle \wedge U \in {\rm pr}_{2}\langle a \rangle \to 
  U \in {\rm pr}_{2}\langle a \rangle \wedge T \in {\rm pr}_{1}\langle a \rangle
\]
が成り立つ.
また定理 \ref{sthmpairinproduct}と推論法則 \ref{dedequiv}により
\[
\tag{12}
  U \in {\rm pr}_{2}\langle a \rangle \wedge T \in {\rm pr}_{1}\langle a \rangle \to 
  (U, T) \in {\rm pr}_{2}\langle a \rangle \times {\rm pr}_{1}\langle a \rangle
\]
が成り立つ.
そこで(10), (11), (12)から, 推論法則 \ref{dedmmp}によって
\[
  (T, U) \in a \to (U, T) \in {\rm pr}_{2}\langle a \rangle \times {\rm pr}_{1}\langle a \rangle
\]
が成り立つことがわかる.
そこでこれに推論法則 \ref{dedaddw}を適用して, 
\[
\tag{13}
  u = (U, T) \wedge (T, U) \in a \to u = (U, T) \wedge (U, T) \in {\rm pr}_{2}\langle a \rangle \times {\rm pr}_{1}\langle a \rangle
\]
が成り立つ.
また定理 \ref{sthm=&in}より
\[
\tag{14}
  u = (U, T) \wedge (U, T) \in {\rm pr}_{2}\langle a \rangle \times {\rm pr}_{1}\langle a \rangle \to 
  u \in {\rm pr}_{2}\langle a \rangle \times {\rm pr}_{1}\langle a \rangle
\]
が成り立つ.
そこで(9), (13), (14)から, 推論法則 \ref{dedmmp}によって
\[
  (u|z)(R) \to u \in {\rm pr}_{2}\langle a \rangle \times {\rm pr}_{1}\langle a \rangle
\]
が成り立つことがわかる.
ここで$z$が$a$の中に自由変数として現れないことから, 
変数法則 \ref{valproduct}, \ref{valprset}により$z$は
${\rm pr}_{2}\langle a \rangle \times {\rm pr}_{1}\langle a \rangle$の中に
自由変数として現れないから, 
代入法則 \ref{substfree}, \ref{substfund}により, 上記の記号列は
\[
  (u|z)(R \to z \in {\rm pr}_{2}\langle a \rangle \times {\rm pr}_{1}\langle a \rangle)
\]
と一致する.
よってこれが定理となる.
そこで$u$が定数でないことから, 推論法則 \ref{dedltthmquan}により
\[
  \forall u((u|z)(R \to z \in {\rm pr}_{2}\langle a \rangle \times {\rm pr}_{1}\langle a \rangle))
\]
が成り立つ.
ここで$u$が$z$と異なり, はじめに述べたように$R$及び
${\rm pr}_{2}\langle a \rangle \times {\rm pr}_{1}\langle a \rangle$の中に
自由変数として現れないことから, 変数法則 \ref{valfund}により, $u$は
$R \to z \in {\rm pr}_{2}\langle a \rangle \times {\rm pr}_{1}\langle a \rangle$の中に
自由変数として現れない.
そこで代入法則 \ref{substquantrans}により, 上記の記号列は
\[
\tag{15}
  \forall z(R \to z \in {\rm pr}_{2}\langle a \rangle \times {\rm pr}_{1}\langle a \rangle)
\]
と一致する.
よってこれが定理となる.
上述のように$z$は${\rm pr}_{2}\langle a \rangle \times {\rm pr}_{1}\langle a \rangle$の中に
自由変数として現れないから, この(15)から, 
定理 \ref{sthmalltiset=sset}によって$R$, 即ち
$\exists x(\exists y((x, y) \in a \wedge z = (y, x)))$が$z$について集合を作り得ることがわかる.
\halmos




\mathstrut
\begin{thm}
\label{sthminvelement}%定理
$a$と$b$を集合とするとき, 
\[
  b \in a^{-1} \leftrightarrow {\rm Pair}(b) \wedge ({\rm pr}_{2}(b), {\rm pr}_{1}(b)) \in a
\]
が成り立つ.
\end{thm}


\noindent{\bf 証明}
~$x$, $y$, $z$を, どの二つも互いに異なり, いずれも$a$及び$b$の中に自由変数として現れない文字とする.
このとき定義から, $a^{-1}$は$\{z|\exists x(\exists y((x, y) \in a \wedge z = (y, x)))\}$と同じである.
また定理 \ref{sthminvsetmake}より, $\exists x(\exists y((x, y) \in a \wedge z = (y, x)))$は
$z$について集合を作り得る.
そこで定理 \ref{sthmisetbasis}より
\[
  b \in a^{-1} \leftrightarrow (b|z)(\exists x(\exists y((x, y) \in a \wedge z = (y, x))))
\]
が成り立つ.
ここで$x$と$y$が共に$z$と異なり, $b$の中に自由変数として現れないことから, 
代入法則 \ref{substquan}により, 上記の記号列は
\[
  b \in a^{-1} \leftrightarrow \exists x(\exists y((b|z)((x, y) \in a \wedge z = (y, x))))
\]
と一致する.
また$z$が$x$とも$y$とも異なることから, 変数法則 \ref{valpair}により, 
$z$は$(x, y)$及び$(y, x)$の中に自由変数として現れない.
このことと, $z$が$a$の中にも自由変数として現れないことから, 
代入法則 \ref{substfree}, \ref{substfund}, \ref{substwedge}により, 
上記の記号列は
\[
\tag{1}
  b \in a^{-1} \leftrightarrow \exists x(\exists y((x, y) \in a \wedge b = (y, x)))
\]
と一致する.
よってこれが定理となる.
さていま$\tau_{x}(\exists y((x, y) \in a \wedge b = (y, x)))$を$T$と書けば, 
$T$は集合であり, 変数法則 \ref{valtau}, \ref{valquan}によってわかるように, 
$y$はこの中に自由変数として現れない.
そして定義から
\[
\tag{2}
  \exists x(\exists y((x, y) \in a \wedge b = (y, x))) \equiv (T|x)(\exists y((x, y) \in a \wedge b = (y, x)))
\]
である.
また$y$が$x$と異なり, いま述べたように$T$の中に自由変数として現れないことから, 
代入法則 \ref{substquan}により
\[
\tag{3}
  (T|x)(\exists y((x, y) \in a \wedge b = (y, x))) \equiv \exists y((T|x)((x, y) \in a \wedge b = (y, x)))
\]
が成り立つ.
また$x$が$y$と異なり, $a$及び$b$の中に自由変数として現れないことから, 
代入法則 \ref{substfree}, \ref{substfund}, \ref{substwedge}, \ref{substpair}により
\[
\tag{4}
  (T|x)((x, y) \in a \wedge b = (y, x)) \equiv (T, y) \in a \wedge b = (y, T)
\]
が成り立つ.
そこで(2), (3), (4)から
\[
\tag{5}
  \exists x(\exists y((x, y) \in a \wedge b = (y, x))) \equiv \exists y((T, y) \in a \wedge b = (y, T))
\]
が成り立つことがわかる.
またいま$\tau_{y}((T, y) \in a \wedge b = (y, T))$を$U$と書けば, 
$U$は集合であり, 定義から
\[
\tag{6}
  \exists y((T, y) \in a \wedge b = (y, T)) \equiv (U|y)((T, y) \in a \wedge b = (y, T))
\]
である.
また$y$が$a$及び$b$の中に自由変数として現れず, 
上述のように$T$の中にも自由変数として現れないことから, 
代入法則 \ref{substfree}, \ref{substfund}, \ref{substwedge}, \ref{substpair}により
\[
\tag{7}
  (U|y)((T, y) \in a \wedge b = (y, T)) \equiv (T, U) \in a \wedge b = (U, T)
\]
が成り立つ.
そこで(5), (6), (7)から, 
\[
  \exists x(\exists y((x, y) \in a \wedge b = (y, x))) \equiv (T, U) \in a \wedge b = (U, T)
\]
が成り立つことがわかる.
いまThm \ref{awbtbwa}より
\[
  (T, U) \in a \wedge b = (U, T) \to b = (U, T) \wedge (T, U) \in a
\]
が成り立つから, 従って上記のことから
\[
\tag{8}
  \exists x(\exists y((x, y) \in a \wedge b = (y, x))) \to b = (U, T) \wedge (T, U) \in a
\]
が成り立つ.
また定理 \ref{sthmpairpreq}と推論法則 \ref{dedequiv}により
\[
\tag{9}
  b = (U, T) \to {\rm Pair}(b) \wedge (U = {\rm pr}_{1}(b) \wedge T = {\rm pr}_{2}(b))
\]
が成り立つ.
またThm \ref{awbtbwa}より
\[
\tag{10}
  U = {\rm pr}_{1}(b) \wedge T = {\rm pr}_{2}(b) \to T = {\rm pr}_{2}(b) \wedge U = {\rm pr}_{1}(b)
\]
が成り立つ.
また定理 \ref{sthmpair}と推論法則 \ref{dedequiv}により
\[
\tag{11}
  T = {\rm pr}_{2}(b) \wedge U = {\rm pr}_{1}(b) \to (T, U) = ({\rm pr}_{2}(b), {\rm pr}_{1}(b))
\]
が成り立つ.
そこで(10), (11)から, 推論法則 \ref{dedmmp}によって
\[
  U = {\rm pr}_{1}(b) \wedge T = {\rm pr}_{2}(b) \to (T, U) = ({\rm pr}_{2}(b), {\rm pr}_{1}(b))
\]
が成り立ち, これから推論法則 \ref{dedaddw}によって
\[
\tag{12}
  {\rm Pair}(b) \wedge (U = {\rm pr}_{1}(b) \wedge T = {\rm pr}_{2}(b)) \to {\rm Pair}(b) \wedge (T, U) = ({\rm pr}_{2}(b), {\rm pr}_{1}(b))
\]
が成り立つ.
そこでまた(9), (12)から, 推論法則 \ref{dedmmp}によって
\[
  b = (U, T) \to {\rm Pair}(b) \wedge (T, U) = ({\rm pr}_{2}(b), {\rm pr}_{1}(b))
\]
が成り立ち, これから推論法則 \ref{dedaddw}によって
\[
\tag{13}
  b = (U, T) \wedge (T, U) \in a \to ({\rm Pair}(b) \wedge (T, U) = ({\rm pr}_{2}(b), {\rm pr}_{1}(b))) \wedge (T, U) \in a
\]
が成り立つ.
またThm \ref{1awb1wctaw1bwc1}より
\begin{multline*}
\tag{14}
  ({\rm Pair}(b) \wedge (T, U) = ({\rm pr}_{2}(b), {\rm pr}_{1}(b))) \wedge (T, U) \in a \\
  \to {\rm Pair}(b) \wedge ((T, U) = ({\rm pr}_{2}(b), {\rm pr}_{1}(b)) \wedge (T, U) \in a)
\end{multline*}
が成り立つ.
また定理 \ref{sthm=&in}より
\[
  (T, U) = ({\rm pr}_{2}(b), {\rm pr}_{1}(b)) \wedge (T, U) \in a \to ({\rm pr}_{2}(b), {\rm pr}_{1}(b)) \in a
\]
が成り立つから, 推論法則 \ref{dedaddw}により
\[
\tag{15}
  {\rm Pair}(b) \wedge ((T, U) = ({\rm pr}_{2}(b), {\rm pr}_{1}(b)) \wedge (T, U) \in a) \to {\rm Pair}(b) \wedge ({\rm pr}_{2}(b), {\rm pr}_{1}(b)) \in a
\]
が成り立つ.
そこで(8), (13), (14), (15)から, 推論法則 \ref{dedmmp}によって
\[
\tag{16}
  \exists x(\exists y((x, y) \in a \wedge b = (y, x))) \to {\rm Pair}(b) \wedge ({\rm pr}_{2}(b), {\rm pr}_{1}(b)) \in a
\]
が成り立つことがわかる.
また定理 \ref{sthmbigpairpr}と推論法則 \ref{dedequiv}により
\[
  {\rm Pair}(b) \to b = ({\rm pr}_{1}(b), {\rm pr}_{2}(b))
\]
が成り立つから, 推論法則 \ref{dedaddw}により
\[
\tag{17}
  {\rm Pair}(b) \wedge ({\rm pr}_{2}(b), {\rm pr}_{1}(b)) \in a \to 
  b = ({\rm pr}_{1}(b), {\rm pr}_{2}(b)) \wedge ({\rm pr}_{2}(b), {\rm pr}_{1}(b)) \in a
\]
が成り立つ.
またThm \ref{awbtbwa}より
\[
  b = ({\rm pr}_{1}(b), {\rm pr}_{2}(b)) \wedge ({\rm pr}_{2}(b), {\rm pr}_{1}(b)) \in a \to 
  ({\rm pr}_{2}(b), {\rm pr}_{1}(b)) \in a \wedge b = ({\rm pr}_{1}(b), {\rm pr}_{2}(b))
\]
が成り立つ.
ここで$y$が$b$の中に自由変数として現れないことから, 
変数法則 \ref{valpr}により, $y$は${\rm pr}_{2}(b)$の中にも自由変数として現れない.
このことと, $y$が$a$の中にも自由変数として現れないことから, 
代入法則 \ref{substfree}, \ref{substfund}, \ref{substwedge}, \ref{substpair}により, 
上記の記号列は
\[
\tag{18}
  b = ({\rm pr}_{1}(b), {\rm pr}_{2}(b)) \wedge ({\rm pr}_{2}(b), {\rm pr}_{1}(b)) \in a \to 
  ({\rm pr}_{1}(b)|y)(({\rm pr}_{2}(b), y) \in a \wedge b = (y, {\rm pr}_{2}(b)))
\]
と一致する.
よってこれが定理となる.
またschema S4の適用により
\[
  ({\rm pr}_{1}(b)|y)(({\rm pr}_{2}(b), y) \in a \wedge b = (y, {\rm pr}_{2}(b))) \to 
  \exists y(({\rm pr}_{2}(b), y) \in a \wedge b = (y, {\rm pr}_{2}(b)))
\]
が成り立つが, $x$が$y$と異なり, $a$及び$b$の中に自由変数として現れないことから, 
代入法則 \ref{substfree}, \ref{substfund}, \ref{substwedge}, \ref{substpair}により, 
この記号列は
\[
  ({\rm pr}_{1}(b)|y)(({\rm pr}_{2}(b), y) \in a \wedge b = (y, {\rm pr}_{2}(b))) \to 
  \exists y(({\rm pr}_{2}(b)|x)((x, y) \in a \wedge b = (y, x)))
\]
と一致する.
また$y$が$x$と異なり, 上述のように${\rm pr}_{2}(b)$の中に自由変数として現れないことから, 
代入法則 \ref{substquan}により, この記号列は
\[
\tag{19}
  ({\rm pr}_{1}(b)|y)(({\rm pr}_{2}(b), y) \in a \wedge b = (y, {\rm pr}_{2}(b))) \to 
  ({\rm pr}_{2}(b)|x)(\exists y((x, y) \in a \wedge b = (y, x)))
\]
と一致する.
よってこれが定理となる.
また再びschema S4の適用により
\[
\tag{20}
  ({\rm pr}_{2}(b)|x)(\exists y((x, y) \in a \wedge b = (y, x))) \to 
  \exists x(\exists y((x, y) \in a \wedge b = (y, x)))
\]
が成り立つ.
そこで(17)---(20)から, 推論法則 \ref{dedmmp}によって
\[
\tag{21}
  {\rm Pair}(b) \wedge ({\rm pr}_{2}(b), {\rm pr}_{1}(b)) \in a \to 
  \exists x(\exists y((x, y) \in a \wedge b = (y, x)))
\]
が成り立つことがわかる.
そこで(16), (21)から, 推論法則 \ref{dedequiv}によって
\[
\tag{22}
  \exists x(\exists y((x, y) \in a \wedge b = (y, x))) \leftrightarrow {\rm Pair}(b) \wedge ({\rm pr}_{2}(b), {\rm pr}_{1}(b)) \in a
\]
が成り立つ.
そして(1), (22)から, 推論法則 \ref{dedeqtrans}によって
\[
  b \in a^{-1} \leftrightarrow {\rm Pair}(b) \wedge ({\rm pr}_{2}(b), {\rm pr}_{1}(b)) \in a
\]
が成り立つ.
\halmos




\mathstrut
\begin{thm}
\label{sthmpairininv}%定理
$a$, $b$, $c$を集合とするとき, 
\[
  (b, c) \in a^{-1} \leftrightarrow (c, b) \in a
\]
が成り立つ.
\end{thm}


\noindent{\bf 証明}
~定理 \ref{sthminvelement}より
\[
\tag{1}
  (b, c) \in a^{-1} \leftrightarrow {\rm Pair}((b, c)) \wedge ({\rm pr}_{2}((b, c)), {\rm pr}_{1}((b, c))) \in a
\]
が成り立つ.
また定理 \ref{sthmbigpairpair}より${\rm Pair}((b, c))$が成り立つから, 
推論法則 \ref{dedawblatrue2}により
\[
\tag{2}
  {\rm Pair}((b, c)) \wedge ({\rm pr}_{2}((b, c)), {\rm pr}_{1}((b, c))) \in a \leftrightarrow 
  ({\rm pr}_{2}((b, c)), {\rm pr}_{1}((b, c))) \in a
\]
が成り立つ.
また定理 \ref{sthmprpair}より
\[
  {\rm pr}_{2}((b, c)) = c, ~~
  {\rm pr}_{1}((b, c)) = b
\]
が共に成り立つから, 定理 \ref{sthmpair}により
\[
  ({\rm pr}_{2}((b, c)), {\rm pr}_{1}((b, c))) = (c, b)
\]
が成り立ち, これから定理 \ref{sthm=tineq}により
\[
\tag{3}
  ({\rm pr}_{2}((b, c)), {\rm pr}_{1}((b, c))) \in a \leftrightarrow (c, b) \in a
\]
が成り立つ.
そこで(1), (2), (3)から, 推論法則 \ref{dedeqtrans}によって
$(b, c) \in a^{-1} \leftrightarrow (c, b) \in a$が成り立つことがわかる.
\halmos




\mathstrut
\begin{thm}
\label{sthminvgraph}%定理
$a$を集合とするとき, $a^{-1}$はグラフである.
また
\[
  a^{-1} \subset {\rm pr}_{2}\langle a \rangle \times {\rm pr}_{1}\langle a \rangle
\]
が成り立つ.
\end{thm}


\noindent{\bf 証明}
~$x$, $y$, $z$を, どの二つも互いに異なり, いずれも$a$の中に自由変数として現れない文字とする.
また関係式$\exists x(\exists y((x, y) \in a \wedge z = (y, x)))$を$R$と書く.
このとき定義から$a^{-1}$は$\{z|R\}$と同じである.
また定理 \ref{sthminvsetmake}の証明の中で示したように, 
$z$は${\rm pr}_{2}\langle a \rangle \times {\rm pr}_{1}\langle a \rangle$の中に自由変数として現れず, 
$\forall z(R \to z \in {\rm pr}_{2}\langle a \rangle \times {\rm pr}_{1}\langle a \rangle)$が
成り立つ(定理 \ref{sthminvsetmake}の証明中の(15)).
そこで定理 \ref{sthmalltiset=sset}により, 
$\{z|R\} \subset {\rm pr}_{2}\langle a \rangle \times {\rm pr}_{1}\langle a \rangle$, 
即ち
$a^{-1} \subset {\rm pr}_{2}\langle a \rangle \times {\rm pr}_{1}\langle a \rangle$が
成り立つ.
またこのことから, 定理 \ref{sthmproductsubsetgraph}によってわかるように, $a^{-1}$はグラフである.
\halmos




\mathstrut
$a$を集合とするとき, 上記の定理 \ref{sthminvgraph}によれば, $a$の逆$a^{-1}$はグラフである.
そこで$a^{-1}$を, $a$の\textbf{逆グラフ}ともいう
($a$がグラフであるときにこの言い方をすることが多い).




\mathstrut
\begin{thm}
\label{sthminvinv}%定理
$a$を集合とするとき, 
\begin{align*}
  &(a^{-1})^{-1} \subset a, \\
  \mbox{} \\
  {\rm Graph}&(a) \leftrightarrow (a^{-1})^{-1} = a
\end{align*}
が成り立つ.
またこの後者から, 次のことが成り立つ: 

($*$) ~~$a$がグラフならば, $(a^{-1})^{-1} = a$が成り立つ.
        逆に$(a^{-1})^{-1} = a$が成り立つならば, $a$はグラフである.
\end{thm}


\noindent{\bf 証明}
~まず$(a^{-1})^{-1} \subset a$が成り立つことを示す.
$x$と$y$を, 互いに異なり, 共に$a$の中に自由変数として現れない, 
定数でない文字とする.
このとき変数法則 \ref{valinv}により, 
$x$と$y$は共に$(a^{-1})^{-1}$の中にも自由変数として現れない.
そして定理 \ref{sthmpairininv}より
\begin{align*}
  (x, y) \in (a^{-1})^{-1} &\leftrightarrow (y, x) \in a^{-1}, \\
  \mbox{} \\
  (y, x) \in a^{-1} &\leftrightarrow (x, y) \in a
\end{align*}
が共に成り立つ.
そこでこれらから, 推論法則 \ref{dedeqtrans}によって
\[
\tag{1}
  (x, y) \in (a^{-1})^{-1} \leftrightarrow (x, y) \in a
\]
が成り立ち, 特にこれから推論法則 \ref{dedequiv}によって
\[
\tag{2}
  (x, y) \in (a^{-1})^{-1} \to (x, y) \in a
\]
が成り立つ.
さていま定理 \ref{sthminvgraph}より$(a^{-1})^{-1}$はグラフである.
また上述のように$x$と$y$は共に$(a^{-1})^{-1}$及び$a$の中に自由変数として現れない.
また$x$と$y$は互いに異なり, 共に定数でない.
そこでこれらのことと, (2)が成り立つことから, 定理 \ref{sthmgraphpairsubset}により
$(a^{-1})^{-1} \subset a$が成り立つ.

次に${\rm Graph}(a) \leftrightarrow (a^{-1})^{-1} = a$が成り立つことを示す.
$x$と$y$は上と同じとするとき, 既に示したように(1)が成り立つから, これと
$x$, $y$が共に定数でないことから, 推論法則 \ref{dedltthmquan}により
\[
\tag{3}
  \forall x(\forall y((x, y) \in (a^{-1})^{-1} \leftrightarrow (x, y) \in a))
\]
が成り立つ.
また上で述べたように
\[
\tag{4}
  {\rm Graph}((a^{-1})^{-1})
\]
が成り立つから, 推論法則 \ref{dedatawbtrue2}により
\[
\tag{5}
  {\rm Graph}(a) \to {\rm Graph}((a^{-1})^{-1}) \wedge {\rm Graph}(a)
\]
が成り立つ.
また$x$と$y$が互いに異なり, 上述のように共に$(a^{-1})^{-1}$及び$a$の中に自由変数として現れないことから, 
定理 \ref{sthmgraphpair=}より
\[
  {\rm Graph}((a^{-1})^{-1}) \wedge {\rm Graph}(a) \to 
  ((a^{-1})^{-1} = a \leftrightarrow \forall x(\forall y((x, y) \in (a^{-1})^{-1} \leftrightarrow (x, y) \in a)))
\]
が成り立つ.
そこで推論法則 \ref{dedprewedge}により
\[
\tag{6}
  {\rm Graph}((a^{-1})^{-1}) \wedge {\rm Graph}(a) \to 
  (\forall x(\forall y((x, y) \in (a^{-1})^{-1} \leftrightarrow (x, y) \in a)) \to (a^{-1})^{-1} = a)
\]
が成り立つ.
そこで(5), (6)から, 推論法則 \ref{dedmmp}によって
\[
  {\rm Graph}(a) \to 
  (\forall x(\forall y((x, y) \in (a^{-1})^{-1} \leftrightarrow (x, y) \in a)) \to (a^{-1})^{-1} = a)
\]
が成り立ち, これから推論法則 \ref{dedch}によって
\[
\tag{7}
  \forall x(\forall y((x, y) \in (a^{-1})^{-1} \leftrightarrow (x, y) \in a)) \to 
  ({\rm Graph}(a) \to (a^{-1})^{-1} = a)
\]
が成り立つ.
そこで(3), (7)から, 推論法則 \ref{dedmp}によって
\[
\tag{8}
  {\rm Graph}(a) \to (a^{-1})^{-1} = a
\]
が成り立つ.
また定理 \ref{sthmgraph=}より
\[
  (a^{-1})^{-1} = a \to ({\rm Graph}((a^{-1})^{-1}) \leftrightarrow {\rm Graph}(a))
\]
が成り立つから, 推論法則 \ref{dedprewedge}によって
\[
  (a^{-1})^{-1} = a \to ({\rm Graph}((a^{-1})^{-1}) \to {\rm Graph}(a))
\]
が成り立ち, これから推論法則 \ref{dedch}によって
\[
\tag{9}
  {\rm Graph}((a^{-1})^{-1}) \to ((a^{-1})^{-1} = a \to {\rm Graph}(a))
\]
が成り立つ.
そこで(4), (9)から, 推論法則 \ref{dedmp}によって
\[
\tag{10}
  (a^{-1})^{-1} = a \to {\rm Graph}(a)
\]
が成り立つ.
そして(8), (10)から, 推論法則 \ref{dedequiv}によって
${\rm Graph}(a) \leftrightarrow (a^{-1})^{-1} = a$が成り立つ.
($*$)が成り立つことは, これと推論法則 \ref{dedeqfund}によって明らかである.
\halmos




\mathstrut
\begin{thm}
\label{sthminvsubset}%定理
$a$と$b$を集合とするとき, 
\begin{align*}
  &a \subset b \to a^{-1} \subset b^{-1}, \\
  \mbox{} \\
  {\rm Graph}&(a) \to (a \subset b \leftrightarrow a^{-1} \subset b^{-1})
\end{align*}
が成り立つ.
またこのことから, 次の1), 2), 3)が成り立つ.

1)
$a \subset b$が成り立つならば, $a^{-1} \subset b^{-1}$が成り立つ.

2)
$a$がグラフならば, $a \subset b \leftrightarrow a^{-1} \subset b^{-1}$が成り立つ.

3)
$a$がグラフであるとき, $a^{-1} \subset b^{-1}$が成り立つならば, $a \subset b$が成り立つ.
\end{thm}


\noindent{\bf 証明}
~まず$a \subset b \to a^{-1} \subset b^{-1}$が成り立つことを示す.
$x$と$y$を, 互いに異なり, 共に$a$及び$b$の中に自由変数として現れない, 
定数でない文字とする.
このとき変数法則 \ref{valinv}により, $x$と$y$は共に$a^{-1}$及び$b^{-1}$の中に自由変数として現れない.
そして定理 \ref{sthmsubsetbasis}より
\[
\tag{1}
  a \subset b \to ((x, y) \in a \to (x, y) \in b)
\]
が成り立つ.
また定理 \ref{sthmpairininv}と推論法則 \ref{dedequiv}により
\begin{align*}
  &(y, x) \in a^{-1} \to (x, y) \in a, \\
  \mbox{} \\
  &(x, y) \in b \to (y, x) \in b^{-1}
\end{align*}
が共に成り立つから, この前者から, 推論法則 \ref{dedaddf}によって
\[
\tag{2}
  ((x, y) \in a \to (x, y) \in b) \to ((y, x) \in a^{-1} \to (x, y) \in b)
\]
が成り立ち, 後者から, 推論法則 \ref{dedaddb}によって
\[
\tag{3}
  ((y, x) \in a^{-1} \to (x, y) \in b) \to ((y, x) \in a^{-1} \to (y, x) \in b^{-1})
\]
が成り立つ.
そこで(1), (2), (3)から, 推論法則 \ref{dedmmp}によって
\[
\tag{4}
  a \subset b \to ((y, x) \in a^{-1} \to (y, x) \in b^{-1})
\]
が成り立つことがわかる.
さていま$x$と$y$は共に$a$及び$b$の中に自由変数として現れないから, 
変数法則 \ref{valsubset}により, これらは共に$a \subset b$の中に自由変数として現れない.
また$x$と$y$は共に定数でない.
そこでこれらのことと, (4)が成り立つことから, 推論法則 \ref{dedalltquansepfreeconst}によって
\[
\tag{5}
  a \subset b \to \forall y(\forall x((y, x) \in a^{-1} \to (y, x) \in b^{-1}))
\]
が成り立つことがわかる.
また定理 \ref{sthminvgraph}より$a^{-1}$はグラフであるから, 
このことと, $x$と$y$が互いに異なり, 上述のように共に$a^{-1}$及び$b^{-1}$の中に自由変数として現れないことから, 
定理 \ref{sthmgraphpairsubset}により
\[
  a^{-1} \subset b^{-1} \leftrightarrow \forall y(\forall x((y, x) \in a^{-1} \to (y, x) \in b^{-1}))
\]
が成り立つ.
そこで特に推論法則 \ref{dedequiv}により
\[
\tag{6}
  \forall y(\forall x((y, x) \in a^{-1} \to (y, x) \in b^{-1})) \to a^{-1} \subset b^{-1}
\]
が成り立つ.
(5), (6)から, 推論法則 \ref{dedmmp}によって
\[
\tag{7}
  a \subset b \to a^{-1} \subset b^{-1}
\]
が成り立つ.

次に${\rm Graph}(a) \to (a \subset b \leftrightarrow a^{-1} \subset b^{-1})$が成り立つことを示す.
いま示したように(7)が成り立つから, 推論法則 \ref{deds1}により
\[
\tag{8}
  {\rm Graph}(a) \to (a \subset b \to a^{-1} \subset b^{-1})
\]
が成り立つ.
また(7)において, $a$を$a^{-1}$, $b$を$b^{-1}$にそれぞれ置き換えた
\[
\tag{9}
  a^{-1} \subset b^{-1} \to (a^{-1})^{-1} \subset (b^{-1})^{-1}
\]
も成り立つ.
また定理 \ref{sthminvinv}より
$(b^{-1})^{-1} \subset b$が成り立つから, 
推論法則 \ref{dedatawbtrue2}により
\[
\tag{10}
  (a^{-1})^{-1} \subset (b^{-1})^{-1} \to (a^{-1})^{-1} \subset (b^{-1})^{-1} \wedge (b^{-1})^{-1} \subset b
\]
が成り立つ.
また定理 \ref{sthmsubsettrans}より
\[
\tag{11}
  (a^{-1})^{-1} \subset (b^{-1})^{-1} \wedge (b^{-1})^{-1} \subset b \to (a^{-1})^{-1} \subset b
\]
が成り立つ.
そこで(9), (10), (11)から, 推論法則 \ref{dedmmp}によって
\[
  a^{-1} \subset b^{-1} \to (a^{-1})^{-1} \subset b
\]
が成り立つことがわかり, これから推論法則 \ref{dedaddw}によって
\[
\tag{12}
  {\rm Graph}(a) \wedge a^{-1} \subset b^{-1} \to {\rm Graph}(a) \wedge (a^{-1})^{-1} \subset b
\]
が成り立つ.
また定理 \ref{sthminvinv}と推論法則 \ref{dedequiv}により
\[
\tag{13}
  {\rm Graph}(a) \to (a^{-1})^{-1} = a
\]
が成り立つ.
また定理 \ref{sthm=tsubseteq}より
\[
  (a^{-1})^{-1} = a \to ((a^{-1})^{-1} \subset b \leftrightarrow a \subset b)
\]
が成り立つから, 推論法則 \ref{dedprewedge}により
\[
\tag{14}
  (a^{-1})^{-1} = a \to ((a^{-1})^{-1} \subset b \to a \subset b)
\]
が成り立つ.
そこで(13), (14)から, 推論法則 \ref{dedmmp}によって
\[
  {\rm Graph}(a) \to ((a^{-1})^{-1} \subset b \to a \subset b)
\]
が成り立ち, これから推論法則 \ref{dedtwch}によって
\[
\tag{15}
  {\rm Graph}(a) \wedge (a^{-1})^{-1} \subset b \to a \subset b
\]
が成り立つ.
そこでまた(12), (15)から, 推論法則 \ref{dedmmp}によって
\[
  {\rm Graph}(a) \wedge a^{-1} \subset b^{-1} \to a \subset b
\]
が成り立ち, これから推論法則 \ref{dedtwch}によって
\[
\tag{16}
  {\rm Graph}(a) \to (a^{-1} \subset b^{-1} \to a \subset b)
\]
が成り立つ.
(8), (16)から, 推論法則 \ref{dedprewedge}によって
${\rm Graph}(a) \to (a \subset b \leftrightarrow a^{-1} \subset b^{-1})$が成り立つ.

\noindent
1)
上で示したように$a \subset b \to a^{-1} \subset b^{-1}$が成り立つから, 
1)が成り立つことはこれと推論法則 \ref{dedmp}によって明らかである.

\noindent
2)
上で示したように${\rm Graph}(a) \to (a \subset b \leftrightarrow a^{-1} \subset b^{-1})$が
成り立つから, 2)が成り立つことはこれと推論法則 \ref{dedmp}によって明らかである.

\noindent
3)
このとき2)により$a \subset b \leftrightarrow a^{-1} \subset b^{-1}$が
成り立つから, これと推論法則 \ref{dedeqfund}によって3)が成り立つことがわかる.
\halmos




\mathstrut
\begin{thm}
\label{sthminv=}%定理
$a$と$b$を集合とするとき, 
\begin{align*}
  a = b &\to a^{-1} = b^{-1}, \\
  \mbox{} \\
  {\rm Graph}(a) \wedge {\rm Graph}(b) &\to (a = b \leftrightarrow a^{-1} = b^{-1})
\end{align*}
が成り立つ.
またこのことから, 次の1), 2), 3)が成り立つ.

1)
$a = b$が成り立つならば, $a^{-1} = b^{-1}$が成り立つ.

2)
$a$と$b$が共にグラフならば, $a = b \leftrightarrow a^{-1} = b^{-1}$が成り立つ.

3)
$a$と$b$が共にグラフであるとき, $a^{-1} = b^{-1}$が成り立つならば, $a = b$が成り立つ.
\end{thm}


\noindent{\bf 証明}
~まず$a = b \to a^{-1} = b^{-1}$が成り立つことを示す.
$x$を文字とするとき, Thm \ref{T=Ut1TV=UV1}より
\[
  a = b \to (a|x)(x^{-1}) = (b|x)(x^{-1})
\]
が成り立つが, 代入法則 \ref{substinv}によれば, この記号列は
\[
\tag{1}
  a = b \to a^{-1} = b^{-1}
\]
と一致するから, これが定理となる.

次に${\rm Graph}(a) \wedge {\rm Graph}(b) \to (a = b \leftrightarrow a^{-1} = b^{-1})$が成り立つことを示す.
いま示したように(1)が成り立つから, 推論法則 \ref{deds1}により
\[
\tag{2}
  {\rm Graph}(a) \wedge {\rm Graph}(b) \to (a = b \to a^{-1} = b^{-1})
\]
が成り立つ.
また定理 \ref{sthminvsubset}より
\begin{align*}
  {\rm Graph}(a) &\to (a \subset b \leftrightarrow a^{-1} \subset b^{-1}), \\
  \mbox{} \\
  {\rm Graph}(b) &\to (b \subset a \leftrightarrow b^{-1} \subset a^{-1})
\end{align*}
が共に成り立つから, 推論法則 \ref{dedprewedge}により
\begin{align*}
  {\rm Graph}(a) &\to (a^{-1} \subset b^{-1} \to a \subset b), \\
  \mbox{} \\
  {\rm Graph}(b) &\to (b^{-1} \subset a^{-1} \to b \subset a)
\end{align*}
が共に成り立ち, これらから, 推論法則 \ref{dedfromaddw}により
\[
\tag{3}
  {\rm Graph}(a) \wedge {\rm Graph}(b) \to (a^{-1} \subset b^{-1} \to a \subset b) \wedge (b^{-1} \subset a^{-1} \to b \subset a)
\]
が成り立つ.
またThm \ref{1atb1w1ctd1t1awctbwd1}より
\[
\tag{4}
  (a^{-1} \subset b^{-1} \to a \subset b) \wedge (b^{-1} \subset a^{-1} \to b \subset a) \to 
  (a^{-1} \subset b^{-1} \wedge b^{-1} \subset a^{-1} \to a \subset b \wedge b \subset a)
\]
が成り立つ.
また定理 \ref{sthmaxiom1}と推論法則 \ref{dedequiv}により
\begin{align*}
  &a^{-1} = b^{-1} \to a^{-1} \subset b^{-1} \wedge b^{-1} \subset a^{-1}, \\
  \mbox{} \\
  &a \subset b \wedge b \subset a \to a = b
\end{align*}
が共に成り立つから, この前者から, 推論法則 \ref{dedaddf}によって
\[
\tag{5}
  (a^{-1} \subset b^{-1} \wedge b^{-1} \subset a^{-1} \to a \subset b \wedge b \subset a) \to 
  (a^{-1} = b^{-1} \to a \subset b \wedge b \subset a)
\]
が成り立ち, 後者から, 推論法則 \ref{dedaddb}によって
\[
\tag{6}
  (a^{-1} = b^{-1} \to a \subset b \wedge b \subset a) \to (a^{-1} = b^{-1} \to a = b)
\]
が成り立つ.
そこで(3)---(6)から, 推論法則 \ref{dedmmp}によって
\[
\tag{7}
  {\rm Graph}(a) \wedge {\rm Graph}(b) \to (a^{-1} = b^{-1} \to a = b)
\]
が成り立つことがわかる.
(2), (7)から, 推論法則 \ref{dedprewedge}によって
${\rm Graph}(a) \wedge {\rm Graph}(b) \to (a = b \leftrightarrow a^{-1} = b^{-1})$が成り立つ.

\noindent
1)
上で示したように$a = b \to a^{-1} = b^{-1}$が成り立つから, 
1)が成り立つことはこれと推論法則 \ref{dedmp}によって明らかである.

\noindent
2)
このとき推論法則 \ref{dedwedge}により${\rm Graph}(a) \wedge {\rm Graph}(b)$が成り立つ.
また上で示したように${\rm Graph}(a) \wedge {\rm Graph}(b) \to (a = b \leftrightarrow a^{-1} = b^{-1})$が成り立つ.
そこでこれらから, 推論法則 \ref{dedmp}によって
$a = b \leftrightarrow a^{-1} = b^{-1}$が成り立つ.

\noindent
3)
このとき2)により$a = b \leftrightarrow a^{-1} = b^{-1}$が成り立つから, 
これと推論法則 \ref{dedeqfund}によって3)が成り立つことがわかる.
\halmos




\mathstrut
\begin{thm}
\label{sthmuopairinv}%定理
$a$, $b$, $c$, $d$を集合とするとき, 
\[
  \{(a, b), (c, d)\}^{-1} = \{(b, a), (d, c)\}
\]
が成り立つ.
\end{thm}


\noindent{\bf 証明}
~$x$と$y$を, 互いに異なり, 共に$a$, $b$, $c$, $d$のいずれの記号列の中にも自由変数として現れない, 
定数でない文字とする.
このとき変数法則 \ref{valnset}, \ref{valpair}, \ref{valinv}によってわかるように, 
$x$と$y$は共に$\{(a, b), (c, d)\}^{-1}$及び$\{(b, a), (d, c)\}$の中に自由変数として現れない.
また定理 \ref{sthmpairininv}より
\[
\tag{1}
  (x, y) \in \{(a, b), (c, d)\}^{-1} \leftrightarrow (y, x) \in \{(a, b), (c, d)\}
\]
が成り立つ.
また定理 \ref{sthmuopairbasis}より
\[
\tag{2}
  (y, x) \in \{(a, b), (c, d)\} \leftrightarrow (y, x) = (a, b) \vee (y, x) = (c, d)
\]
が成り立つ.
また定理 \ref{sthmpair}より
\begin{align*}
  \tag{3}
  (y, x) = (a, b) &\leftrightarrow y = a \wedge x = b, \\
  \mbox{} \\
  \tag{4}
  (y, x) = (c, d) &\leftrightarrow y = c \wedge x = d
\end{align*}
が共に成り立つ.
またThm \ref{awblbwa}より
\begin{align*}
  \tag{5}
  y = a \wedge x = b &\leftrightarrow x = b \wedge y = a, \\
  \mbox{} \\
  \tag{6}
  y = c \wedge x = d &\leftrightarrow x = d \wedge y = c
\end{align*}
が共に成り立つ.
また定理 \ref{sthmpair}と推論法則 \ref{dedeqch}により
\begin{align*}
  \tag{7}
  x = b \wedge y = a &\leftrightarrow (x, y) = (b, a), \\
  \mbox{} \\
  \tag{8}
  x = d \wedge y = c &\leftrightarrow (x, y) = (d, c)
\end{align*}
が共に成り立つ.
そこで(3)と(5)と(7), (4)と(6)と(8)から, それぞれ推論法則 \ref{dedeqtrans}によって
\begin{align*}
  (y, x) = (a, b) &\leftrightarrow (x, y) = (b, a), \\
  \mbox{} \\
  (y, x) = (c, d) &\leftrightarrow (x, y) = (d, c)
\end{align*}
が成り立つことがわかる.
そこでこれらから, 推論法則 \ref{dedaddeqv}によって
\[
\tag{9}
  (y, x) = (a, b) \vee (y, x) = (c, d) \leftrightarrow (x, y) = (b, a) \vee (x, y) = (d, c)
\]
が成り立つ.
また定理 \ref{sthmuopairbasis}と推論法則 \ref{dedeqch}により
\[
\tag{10}
  (x, y) = (b, a) \vee (x, y) = (d, c) \leftrightarrow (x, y) \in \{(b, a), (d, c)\}
\]
が成り立つ.
そこで(1), (2), (9), (10)から, 推論法則 \ref{dedeqtrans}によって
\[
\tag{11}
  (x, y) \in \{(a, b), (c, d)\}^{-1} \leftrightarrow (x, y) \in \{(b, a), (d, c)\}
\]
が成り立つことがわかる.
さていま定理 \ref{sthmbigpairpair}, \ref{sthmuopairgraph}, \ref{sthminvgraph}によってわかるように, 
$\{(a, b), (c, d)\}^{-1}$と$\{(b, a), (d, c)\}$は共にグラフである.
また上述のように, $x$と$y$は共にこれらの中に自由変数として現れない.
また$x$と$y$は互いに異なり, 共に定数でない.
そこでこれらのことと, (11)が成り立つことから, 定理 \ref{sthmgraphpair=}によって
$\{(a, b), (c, d)\}^{-1} = \{(b, a), (d, c)\}$が成り立つ.
\halmos




\mathstrut
\begin{thm}
\label{sthmsingletoninv}%定理
$a$と$b$を集合とするとき, 
\[
  \{(a, b)\}^{-1} = \{(b, a)\}
\]
が成り立つ.
\end{thm}


\noindent{\bf 証明}
~定理 \ref{sthmuopairinv}より$\{(a, b), (a, b)\}^{-1} = \{(b, a), (b, a)\}$が成り立つが, 
定義からこの記号列は$\{(a, b)\}^{-1} = \{(b, a)\}$と同じだから, これが定理となる.
\halmos




\mathstrut
\begin{thm}
\label{sthmpairsetofainv}%定理
$a$を集合とし, $x$を$a$の中に自由変数として現れない文字とする.
このとき
\[
  \{x \in a|{\rm Pair}(x)\}^{-1} = a^{-1}
\]
が成り立つ.
\end{thm}


\noindent{\bf 証明}
~$u$と$v$を, 互いに異なり, 共に$x$と異なり, $a$の中に自由変数として現れない, 
定数でない文字とする.
このとき変数法則 \ref{valsset}, \ref{valbigpair}, \ref{valinv}からわかるように, 
$u$と$v$は共に$\{x \in a|{\rm Pair}(x)\}^{-1}$及び$a^{-1}$の中に自由変数として現れない.
そして定理 \ref{sthmpairininv}より
\[
\tag{1}
  (u, v) \in \{x \in a|{\rm Pair}(x)\}^{-1} \leftrightarrow (v, u) \in \{x \in a|{\rm Pair}(x)\}
\]
が成り立つ.
また$x$が$a$の中に自由変数として現れないことから, 定理 \ref{sthmssetbasis}より
\[
  (v, u) \in \{x \in a|{\rm Pair}(x)\} \leftrightarrow (v, u) \in a \wedge ((v, u)|x)({\rm Pair}(x))
\]
が成り立つが, 代入法則 \ref{substbigpair}によりこの記号列は
\[
\tag{2}
  (v, u) \in \{x \in a|{\rm Pair}(x)\} \leftrightarrow (v, u) \in a \wedge {\rm Pair}((v, u))
\]
と一致するから, これが定理となる.
また定理 \ref{sthmbigpairpair}より${\rm Pair}((v, u))$が成り立つから, 
推論法則 \ref{dedawblatrue2}により
\[
\tag{3}
  (v, u) \in a \wedge {\rm Pair}((v, u)) \leftrightarrow (v, u) \in a
\]
が成り立つ.
また定理 \ref{sthmpairininv}と推論法則 \ref{dedeqch}により
\[
\tag{4}
  (v, u) \in a \leftrightarrow (u, v) \in a^{-1}
\]
が成り立つ.
そこで(1)---(4)から, 推論法則 \ref{dedeqtrans}によって
\[
\tag{5}
  (u, v) \in \{x \in a|{\rm Pair}(x)\}^{-1} \leftrightarrow (u, v) \in a^{-1}
\]
が成り立つことがわかる.
さて定理 \ref{sthminvgraph}より$\{x \in a|{\rm Pair}(x)\}^{-1}$と$a^{-1}$は共にグラフである.
また上述のように, $u$と$v$は共にこれらの中に自由変数として現れない.
また$u$と$v$は互いに異なり, 共に定数でない.
そこでこれらのことと, (5)が成り立つことから, 定理 \ref{sthmgraphpair=}によって
$\{x \in a|{\rm Pair}(x)\}^{-1} = a^{-1}$が成り立つ.
\halmos




\mathstrut
\begin{thm}
\label{sthmobjectsetinv}%定理
$a$, $T$, $U$を集合とし, $x$を$a$の中に自由変数として現れない文字とする.
このとき
\[
  \{(T, U)|x \in a\}^{-1} = \{(U, T)|x \in a\}
\]
が成り立つ.
\end{thm}


\noindent{\bf 証明}
~$u$と$v$を, 互いに異なり, 共に$x$と異なり, $a$, $T$, $U$のいずれの記号列の中にも自由変数として現れない, 
定数でない文字とする.
このとき変数法則 \ref{valoset}, \ref{valpair}, \ref{valinv}によってわかるように, 
$u$と$v$は共に$\{(T, U)|x \in a\}^{-1}$及び$\{(U, T)|x \in a\}$の中に自由変数として現れない.
そして定理 \ref{sthmpairininv}より
\[
\tag{1}
  (u, v) \in \{(T, U)|x \in a\}^{-1} \leftrightarrow (v, u) \in \{(T, U)|x \in a\}
\]
が成り立つ.
また$x$が$u$とも$v$とも異なることから, 変数法則 \ref{valpair}により, $x$は
$(v, u)$の中に自由変数として現れない.
このことと, $x$が$a$の中にも自由変数として現れないことから, 
定理 \ref{sthmosetbasis}より
\[
\tag{2}
  (v, u) \in \{(T, U)|x \in a\} \leftrightarrow \exists x(x \in a \wedge (v, u) = (T, U))
\]
が成り立つ.
さてここで$y$を$x$, $u$, $v$のいずれとも異なり, $a$, $T$, $U$のいずれの記号列の中にも自由変数として現れない, 
定数でない文字とする.
このとき変数法則 \ref{valfund}, \ref{valwedge}, \ref{valpair}によってわかるように, 
$y$は$x \in a \wedge (v, u) = (T, U)$及び$x \in a \wedge (u, v) = (U, T)$の中に自由変数として現れない.
また定理 \ref{sthmpair}より
\[
\tag{3}
  (v, u) = ((y|x)(T), (y|x)(U)) \leftrightarrow v = (y|x)(T) \wedge u = (y|x)(U)
\]
が成り立つ.
またThm \ref{awblbwa}より
\[
\tag{4}
  v = (y|x)(T) \wedge u = (y|x)(U) \leftrightarrow u = (y|x)(U) \wedge v = (y|x)(T)
\]
が成り立つ.
また定理 \ref{sthmpair}と推論法則 \ref{dedeqch}により
\[
\tag{5}
  u = (y|x)(U) \wedge v = (y|x)(T) \leftrightarrow (u, v) = ((y|x)(U), (y|x)(T))
\]
が成り立つ.
そこで(3), (4), (5)から, 推論法則 \ref{dedeqtrans}によって
\[
  (v, u) = ((y|x)(T), (y|x)(U)) \leftrightarrow (u, v) = ((y|x)(U), (y|x)(T))
\]
が成り立つことがわかり, これから推論法則 \ref{dedaddeqw}によって
\[
  y \in a \wedge (v, u) = ((y|x)(T), (y|x)(U)) \leftrightarrow y \in a \wedge (u, v) = ((y|x)(U), (y|x)(T))
\]
が成り立つ.
ここで$x$が$u$とも$v$とも異なり, $a$の中に自由変数として現れないことから, 
代入法則 \ref{substfree}, \ref{substfund}, \ref{substwedge}, \ref{substpair}により, 
上記の記号列は
\[
  (y|x)(x \in a \wedge (v, u) = (T, U)) \leftrightarrow (y|x)(x \in a \wedge (u, v) = (U, T))
\]
と一致する.
よってこれが定理となる.
そこで$y$が定数でないことから, 推論法則 \ref{dedalleqquansepconst}により
\[
  \exists y((y|x)(x \in a \wedge (v, u) = (T, U))) \leftrightarrow \exists y((y|x)(x \in a \wedge (u, v) = (U, T)))
\]
が成り立つが, 上述のように$y$は$x \in a \wedge (v, u) = (T, U)$及び
$x \in a \wedge (u, v) = (U, T)$の中に自由変数として現れないから, 
代入法則 \ref{substquantrans}により, この記号列は
\[
\tag{6}
  \exists x(x \in a \wedge (v, u) = (T, U)) \leftrightarrow \exists x(x \in a \wedge (u, v) = (U, T))
\]
と一致する.
よってこれが定理となる.
またいま$x$は$u$とも$v$とも異なるから, 変数法則 \ref{valpair}により, 
$x$は$(u, v)$の中に自由変数として現れない.
このことと, $x$が$a$の中にも自由変数として現れないことから, 
定理 \ref{sthmosetbasis}と推論法則 \ref{dedeqch}により
\[
\tag{7}
  \exists x(x \in a \wedge (u, v) = (U, T)) \leftrightarrow (u, v) \in \{(U, T)|x \in a\}
\]
が成り立つ.
そこで(1), (2), (6), (7)から, 推論法則 \ref{dedeqtrans}によって
\[
\tag{8}
  (u, v) \in \{(T, U)|x \in a\}^{-1} \leftrightarrow (u, v) \in \{(U, T)|x \in a\}
\]
が成り立つことがわかる.
さていま定理 \ref{sthminvgraph}より
$\{(T, U)|x \in a\}^{-1}$はグラフである.
また$x$が$a$の中に自由変数として現れないことから, 
定理 \ref{sthmobjectsetgraph2}より$\{(U, T)|x \in a\}$もグラフである.
また上述のように, $u$と$v$は共にこれらの中に自由変数として現れない.
また$u$と$v$は互いに異なり, 共に定数でない.
そこでこれらのことと, (8)が成り立つことから, 
定理 \ref{sthmgraphpair=}によって
$\{(T, U)|x \in a\}^{-1} = \{(U, T)|x \in a\}$が成り立つ.
\halmos




\mathstrut
\begin{thm}
\label{sthmcupinv}%定理
$a$と$b$を集合とするとき, 
\[
  (a \cup b)^{-1} = a^{-1} \cup b^{-1}
\]
が成り立つ.
\end{thm}


\noindent{\bf 証明}
~$x$と$y$を, 互いに異なり, 共に$a$及び$b$の中に自由変数として現れない, 
定数でない文字とする.
このとき変数法則 \ref{valcup}, \ref{valinv}により, 
$x$と$y$は共に$(a \cup b)^{-1}$及び$a^{-1} \cup b^{-1}$の中に自由変数として現れない.
そして定理 \ref{sthmpairininv}より
\[
\tag{1}
  (x, y) \in (a \cup b)^{-1} \leftrightarrow (y, x) \in a \cup b
\]
が成り立つ.
また定理 \ref{sthmcupbasis}より
\[
\tag{2}
  (y, x) \in a \cup b \leftrightarrow (y, x) \in a \vee (y, x) \in b
\]
が成り立つ.
また定理 \ref{sthmpairininv}と推論法則 \ref{dedeqch}により
\[
  (y, x) \in a \leftrightarrow (x, y) \in a^{-1}, ~~
  (y, x) \in b \leftrightarrow (x, y) \in b^{-1}
\]
が共に成り立つから, 推論法則 \ref{dedaddeqv}により
\[
\tag{3}
  (y, x) \in a \vee (y, x) \in b \leftrightarrow (x, y) \in a^{-1} \vee (x, y) \in b^{-1}
\]
が成り立つ.
また定理 \ref{sthmcupbasis}と推論法則 \ref{dedeqch}により
\[
\tag{4}
  (x, y) \in a^{-1} \vee (x, y) \in b^{-1} \leftrightarrow (x, y) \in a^{-1} \cup b^{-1}
\]
が成り立つ.
そこで(1)---(4)から, 推論法則 \ref{dedeqtrans}によって
\[
\tag{5}
  (x, y) \in (a \cup b)^{-1} \leftrightarrow (x, y) \in a^{-1} \cup b^{-1}
\]
が成り立つことがわかる.
いま定理 \ref{sthmcupgraph}, \ref{sthminvgraph}からわかるように, 
$(a \cup b)^{-1}$と$a^{-1} \cup b^{-1}$は共にグラフである.
また$x$と$y$は互いに異なり, 共に定数でなく, 上述のように
共に$(a \cup b)^{-1}$及び$a^{-1} \cup b^{-1}$の中に自由変数として現れない.
そこでこれらのことと, (5)が成り立つことから, 
定理 \ref{sthmgraphpair=}により
$(a \cup b)^{-1} = a^{-1} \cup b^{-1}$が成り立つ.
\halmos




\mathstrut
\begin{thm}
\label{sthmcapinv}%定理
$a$と$b$を集合とするとき, 
\[
  (a \cap b)^{-1} = a^{-1} \cap b^{-1}
\]
が成り立つ.
\end{thm}


\noindent{\bf 証明}
~$x$と$y$を, 互いに異なり, 共に$a$及び$b$の中に自由変数として現れない, 
定数でない文字とする.
このとき変数法則 \ref{valcap}, \ref{valinv}により, 
$x$と$y$は共に$(a \cap b)^{-1}$及び$a^{-1} \cap b^{-1}$の中に自由変数として現れない.
そして定理 \ref{sthmpairininv}より
\[
\tag{1}
  (x, y) \in (a \cap b)^{-1} \leftrightarrow (y, x) \in a \cap b
\]
が成り立つ.
また定理 \ref{sthmcapelement}より
\[
\tag{2}
  (y, x) \in a \cap b \leftrightarrow (y, x) \in a \wedge (y, x) \in b
\]
が成り立つ.
また定理 \ref{sthmpairininv}と推論法則 \ref{dedeqch}により
\[
  (y, x) \in a \leftrightarrow (x, y) \in a^{-1}, ~~
  (y, x) \in b \leftrightarrow (x, y) \in b^{-1}
\]
が共に成り立つから, 推論法則 \ref{dedaddeqw}により
\[
\tag{3}
  (y, x) \in a \wedge (y, x) \in b \leftrightarrow (x, y) \in a^{-1} \wedge (x, y) \in b^{-1}
\]
が成り立つ.
また定理 \ref{sthmcapelement}と推論法則 \ref{dedeqch}により
\[
\tag{4}
  (x, y) \in a^{-1} \wedge (x, y) \in b^{-1} \leftrightarrow (x, y) \in a^{-1} \cap b^{-1}
\]
が成り立つ.
そこで(1)---(4)から, 推論法則 \ref{dedeqtrans}によって
\[
\tag{5}
  (x, y) \in (a \cap b)^{-1} \leftrightarrow (x, y) \in a^{-1} \cap b^{-1}
\]
が成り立つことがわかる.
いま定理 \ref{sthmcapgraph}, \ref{sthminvgraph}からわかるように, 
$(a \cap b)^{-1}$と$a^{-1} \cap b^{-1}$は共にグラフである.
また$x$と$y$は互いに異なり, 共に定数でなく, 上述のように
共に$(a \cap b)^{-1}$及び$a^{-1} \cap b^{-1}$の中に自由変数として現れない.
そこでこれらのことと, (5)が成り立つことから, 
定理 \ref{sthmgraphpair=}により
$(a \cap b)^{-1} = a^{-1} \cap b^{-1}$が成り立つ.
\halmos




\mathstrut
\begin{thm}
\label{sthm-inv}%定理
$a$と$b$を集合とするとき, 
\[
  (a - b)^{-1} = a^{-1} - b^{-1}
\]
が成り立つ.
\end{thm}


\noindent{\bf 証明}
~$x$と$y$を, 互いに異なり, 共に$a$及び$b$の中に自由変数として現れない, 
定数でない文字とする.
このとき変数法則 \ref{val-}, \ref{valinv}により, 
$x$と$y$は共に$(a - b)^{-1}$及び$a^{-1} - b^{-1}$の中に自由変数として現れない.
そして定理 \ref{sthmpairininv}より
\[
\tag{1}
  (x, y) \in (a - b)^{-1} \leftrightarrow (y, x) \in a - b
\]
が成り立つ.
また定理 \ref{sthm-basis}より
\[
\tag{2}
  (y, x) \in a - b \leftrightarrow (y, x) \in a \wedge (y, x) \notin b
\]
が成り立つ.
また定理 \ref{sthmpairininv}と推論法則 \ref{dedeqch}により
\begin{align*}
  \tag{3}
  (y, x) \in a &\leftrightarrow (x, y) \in a^{-1}, \\
  \mbox{} \\
  \tag{4}
  (y, x) \in b &\leftrightarrow (x, y) \in b^{-1}
\end{align*}
が共に成り立つ.
そこで(4)から, 推論法則 \ref{dedeqcp}によって
\[
  (y, x) \notin b \leftrightarrow (x, y) \notin b^{-1}
\]
が成り立ち, これと(3)から, 推論法則 \ref{dedaddeqw}によって
\[
\tag{5}
  (y, x) \in a \wedge (y, x) \notin b \leftrightarrow (x, y) \in a^{-1} \wedge (x, y) \notin b^{-1}
\]
が成り立つ.
また定理 \ref{sthm-basis}と推論法則 \ref{dedeqch}により
\[
\tag{6}
  (x, y) \in a^{-1} \wedge (x, y) \notin b^{-1} \leftrightarrow (x, y) \in a^{-1} - b^{-1}
\]
が成り立つ.
そこで(1), (2), (5), (6)から, 推論法則 \ref{dedeqtrans}によって
\[
\tag{7}
  (x, y) \in (a - b)^{-1} \leftrightarrow (x, y) \in a^{-1} - b^{-1}
\]
が成り立つことがわかる.
いま定理 \ref{sthm-graph}, \ref{sthminvgraph}からわかるように, 
$(a - b)^{-1}$と$a^{-1} - b^{-1}$は共にグラフである.
また$x$と$y$は互いに異なり, 共に定数でなく, 上述のように
共に$(a - b)^{-1}$及び$a^{-1} - b^{-1}$の中に自由変数として現れない.
そこでこれらのことと, (7)が成り立つことから, 
定理 \ref{sthmgraphpair=}により
$(a - b)^{-1} = a^{-1} - b^{-1}$が成り立つ.
\halmos




\mathstrut
\begin{thm}
\label{sthmemptyinv}%定理
$a$を集合とするとき, 
\begin{align*}
  &a = \phi \to a^{-1} = \phi, \\
  \mbox{} \\
  {\rm Graph}&(a) \to (a = \phi \leftrightarrow a^{-1} = \phi)
\end{align*}
が成り立つ.
またこれらのことから, 次の1) -- 4)が成り立つ.

1)
$a$が空ならば, $a^{-1}$は空である.

2)
$\phi^{-1}$は空である.

3)
$a$がグラフならば, $a = \phi \leftrightarrow a^{-1} = \phi$が成り立つ.

4)
$a$がグラフで, $a^{-1}$が空ならば, $a$は空である.
\end{thm}


\noindent{\bf 証明}
~まず$a = \phi \to a^{-1} = \phi$が成り立つことを示す.
$x$を$a$の中に自由変数として現れない文字とし, $\tau_{x}(x \in a^{-1})$を
$T$と書く.
このとき$T$は集合であり, 変数法則 \ref{valinv}により$x$は$a^{-1}$の中に
自由変数として現れないから, 定理 \ref{sthmelm&empty}と推論法則 \ref{dedequiv}により
\[
\tag{1}
  a^{-1} \neq \phi \to T \in a^{-1}
\]
が成り立つ.
また定理 \ref{sthminvelement}と推論法則 \ref{dedequiv}により
\[
  T \in a^{-1} \to {\rm Pair}(T) \wedge ({\rm pr}_{2}(T), {\rm pr}_{1}(T)) \in a
\]
が成り立つから, 推論法則 \ref{dedprewedge}により
\[
\tag{2}
  T \in a^{-1} \to ({\rm pr}_{2}(T), {\rm pr}_{1}(T)) \in a
\]
が成り立つ.
また定理 \ref{sthmnotemptyeqexin}より
\[
\tag{3}
  ({\rm pr}_{2}(T), {\rm pr}_{1}(T)) \in a \to a \neq \phi
\]
が成り立つ.
そこで(1), (2), (3)から, 推論法則 \ref{dedmmp}によって
\[
  a^{-1} \neq \phi \to a \neq \phi
\]
が成り立つことがわかり, これから推論法則 \ref{deds3}によって
$a = \phi \to a^{-1} = \phi$が成り立つ.

次に${\rm Graph}(a) \to (a = \phi \leftrightarrow a^{-1} = \phi)$が成り立つことを示す.
定理 \ref{sthmemptygraph}より${\rm Graph}(\phi)$が成り立つから, 
推論法則 \ref{dedatawbtrue2}により
\[
\tag{4}
  {\rm Graph}(a) \to {\rm Graph}(a) \wedge {\rm Graph}(\phi)
\]
が成り立つ.
また定理 \ref{sthminv=}より
\[
\tag{5}
  {\rm Graph}(a) \wedge {\rm Graph}(\phi) \to (a = \phi \leftrightarrow a^{-1} = \phi^{-1})
\]
が成り立つ.
またThm \ref{x=x}より$\phi = \phi$が成り立つ.
また上で示したように
$a = \phi \to a^{-1} = \phi$が成り立つから, $a$を$\phi$に置き換えた
$\phi = \phi \to \phi^{-1} = \phi$も成り立つ.
そこでこれらから, 推論法則 \ref{dedmp}によって
$\phi^{-1} = \phi$が成り立ち, これから推論法則 \ref{dedaddeq=}によって
\[
  a^{-1} = \phi^{-1} \leftrightarrow a^{-1} = \phi
\]
が成り立つ.
そこで推論法則 \ref{dedatawbtrue2}により
\[
\tag{6}
  (a = \phi \leftrightarrow a^{-1} = \phi^{-1}) \to 
  (a = \phi \leftrightarrow a^{-1} = \phi^{-1}) \wedge (a^{-1} = \phi^{-1} \leftrightarrow a^{-1} = \phi)
\]
が成り立つ.
またThm \ref{1alb1w1blc1t1alc1}より
\[
\tag{7}
  (a = \phi \leftrightarrow a^{-1} = \phi^{-1}) \wedge (a^{-1} = \phi^{-1} \leftrightarrow a^{-1} = \phi) \to 
  (a = \phi \leftrightarrow a^{-1} = \phi)
\]
が成り立つ.
そこで(4)---(7)から, 推論法則 \ref{dedmmp}によって
${\rm Graph}(a) \to (a = \phi \leftrightarrow a^{-1} = \phi)$が成り立つことがわかる.

\noindent
1)
上で示したように$a = \phi \to a^{-1} = \phi$が成り立つから, 
1)が成り立つことはこれと推論法則 \ref{dedmp}によって明らかである.

\noindent
2)
既に示した.

\noindent
3)
上で示したように
${\rm Graph}(a) \to (a = \phi \leftrightarrow a^{-1} = \phi)$が成り立つから, 
3)が成り立つことはこれと推論法則 \ref{dedmp}によって明らかである.

\noindent
4)
このとき3)により
$a = \phi \leftrightarrow a^{-1} = \phi$が成り立つから, 
4)が成り立つことはこれと推論法則 \ref{dedeqfund}によって明らかである.
\halmos




\mathstrut
\begin{thm}
\label{sthmemptyinv2}%定理
$a$を集合とし, $x$を$a$の中に自由変数として現れない文字とする.
このとき
\[
  a^{-1} = \phi \leftrightarrow \{x \in a|{\rm Pair}(x)\} = \phi
\]
が成り立つ.
またこのことから, 次の($*$)が成り立つ: 

($*$) ~~$a^{-1}$が空ならば, $\{x \in a|{\rm Pair}(x)\}$は空である.
        逆に$\{x \in a|{\rm Pair}(x)\}$が空ならば, $a^{-1}$は空である.
\end{thm}


\noindent{\bf 証明}
~$x$が$a$の中に自由変数として現れないことから, 
定理 \ref{sthmpairsetofagraph}より$\{x \in a|{\rm Pair}(x)\}$はグラフだから, 
定理 \ref{sthmemptyinv}により
\[
\tag{1}
  \{x \in a|{\rm Pair}(x)\} = \phi \leftrightarrow \{x \in a|{\rm Pair}(x)\}^{-1} = \phi
\]
が成り立つ.
また, やはり$x$が$a$の中に自由変数として現れないことから, 
定理 \ref{sthmpairsetofainv}より$\{x \in a|{\rm Pair}(x)\}^{-1} = a^{-1}$が成り立つ.
そこで推論法則 \ref{dedaddeq=}により
\[
\tag{2}
  \{x \in a|{\rm Pair}(x)\}^{-1} = \phi \leftrightarrow a^{-1} = \phi
\]
が成り立つ.
そこで(1), (2)から, 推論法則 \ref{dedeqtrans}によって
$\{x \in a|{\rm Pair}(x)\} = \phi \leftrightarrow a^{-1} = \phi$が成り立ち, 
これから推論法則 \ref{dedeqch}によって
$a^{-1} = \phi \leftrightarrow \{x \in a|{\rm Pair}(x)\} = \phi$が成り立つ.
($*$)が成り立つことは, これと推論法則 \ref{dedeqfund}によって明らかである.
\halmos




\mathstrut
\begin{thm}
\label{sthmproductinv}%定理
$a$と$b$を集合とするとき, 
\[
  (a \times b)^{-1} = b \times a
\]
が成り立つ.
\end{thm}


\noindent{\bf 証明}
~$x$と$y$を, 互いに異なり, 共に$a$及び$b$の中に自由変数として現れない, 
定数でない文字とする.
このとき変数法則 \ref{valproduct}, \ref{valinv}からわかるように, $x$と$y$は共に
$(a \times b)^{-1}$及び$b \times a$の中に自由変数として現れない.
そして定理 \ref{sthmpairininv}より
\[
\tag{1}
  (x, y) \in (a \times b)^{-1} \leftrightarrow (y, x) \in a \times b
\]
が成り立つ.
また定理 \ref{sthmpairinproduct}より
\[
\tag{2}
  (y, x) \in a \times b \leftrightarrow y \in a \wedge x \in b
\]
が成り立つ.
またThm \ref{awblbwa}より
\[
\tag{3}
  y \in a \wedge x \in b \leftrightarrow x \in b \wedge y \in a
\]
が成り立つ.
また定理 \ref{sthmpairinproduct}と推論法則 \ref{dedeqch}により
\[
\tag{4}
  x \in b \wedge y \in a \leftrightarrow (x, y) \in b \times a
\]
が成り立つ.
そこで(1)---(4)から, 推論法則 \ref{dedeqtrans}によって
\[
\tag{5}
  (x, y) \in (a \times b)^{-1} \leftrightarrow (x, y) \in b \times a
\]
が成り立つことがわかる.
いま定理 \ref{sthmproductgraph}, \ref{sthminvgraph}より, 
$(a \times b)^{-1}$と$b \times a$は共にグラフである.
また上述のように, $x$と$y$は共にこれらの中に自由変数として現れない.
また$x$と$y$は互いに異なり, 共に定数でない.
そこでこれらのことと, (5)が成り立つことから, 定理 \ref{sthmgraphpair=}により
$(a \times b)^{-1} = b \times a$が成り立つ.
\halmos




\mathstrut
\begin{thm}
\label{sthmproductsubsetinv}%定理
$a$, $b$, $c$を集合とするとき, 
\begin{align*}
  &a \subset b \times c \to a^{-1} \subset c \times b, \\
  \mbox{} \\
  {\rm Graph}&(a) \to (a \subset b \times c \leftrightarrow a^{-1} \subset c \times b)
\end{align*}
が成り立つ.
またこれらのことから, 次のが成り立つ.

1)
$a \subset b \times c$が成り立つならば, $a^{-1} \subset c \times b$が成り立つ.

2)
$a$がグラフならば, $a \subset b \times c \leftrightarrow a^{-1} \subset c \times b$が成り立つ.

3)
$a$がグラフで, $a^{-1} \subset c \times b$が成り立つならば, $a \subset b \times c$が成り立つ.
\end{thm}


\noindent{\bf 証明}
~まず$a \subset b \times c \to a^{-1} \subset c \times b$が成り立つことを示す.
定理 \ref{sthminvsubset}より
\[
\tag{1}
  a \subset b \times c \to a^{-1} \subset (b \times c)^{-1}
\]
が成り立つ.
また定理 \ref{sthmproductinv}より
$(b \times c)^{-1} = c \times b$が成り立つから, 
定理 \ref{sthm=tsubseteq}により
\[
\tag{2}
  a^{-1} \subset (b \times c)^{-1} \leftrightarrow a^{-1} \subset c \times b
\]
が成り立ち, これから特に推論法則 \ref{dedequiv}によって
\[
\tag{3}
  a^{-1} \subset (b \times c)^{-1} \to a^{-1} \subset c \times b
\]
が成り立つ.
そこで(1), (3)から, 推論法則 \ref{dedmmp}によって
$a \subset b \times c \to a^{-1} \subset c \times b$が成り立つ.

次に${\rm Graph}(a) \to (a \subset b \times c \leftrightarrow a^{-1} \subset c \times b)$が
成り立つことを示す.
定理 \ref{sthminvsubset}より
\[
\tag{4}
  {\rm Graph}(a) \to (a \subset b \times c \leftrightarrow a^{-1} \subset (b \times c)^{-1})
\]
が成り立つ.
また上で示したように(2)が成り立つから, 推論法則 \ref{dedatawbtrue2}により
\begin{multline*}
\tag{5}
  (a \subset b \times c \leftrightarrow a^{-1} \subset (b \times c)^{-1}) \\
  \to (a \subset b \times c \leftrightarrow a^{-1} \subset (b \times c)^{-1}) \wedge 
  (a^{-1} \subset (b \times c)^{-1} \leftrightarrow a^{-1} \subset c \times b)
\end{multline*}
が成り立つ.
またThm \ref{1alb1w1blc1t1alc1}より
\[
\tag{6}
  (a \subset b \times c \leftrightarrow a^{-1} \subset (b \times c)^{-1}) \wedge 
  (a^{-1} \subset (b \times c)^{-1} \leftrightarrow a^{-1} \subset c \times b) \to 
  (a \subset b \times c \leftrightarrow a^{-1} \subset c \times b)
\]
が成り立つ.
そこで(4), (5), (6)から, 推論法則 \ref{dedmmp}によって
${\rm Graph}(a) \to (a \subset b \times c \leftrightarrow a^{-1} \subset c \times b)$が
成り立つことがわかる.

\noindent
1)
上で示したように$a \subset b \times c \to a^{-1} \subset c \times b$が成り立つから, 
1)が成り立つことはこれと推論法則 \ref{dedmp}によって明らかである.

\noindent
2)
上で示したように
${\rm Graph}(a) \to (a \subset b \times c \leftrightarrow a^{-1} \subset c \times b)$が
成り立つから, 2)が成り立つことはこれと推論法則 \ref{dedmp}によって明らかである.

\noindent
3)
このとき2)により
$a \subset b \times c \leftrightarrow a^{-1} \subset c \times b$が成り立つから, 
これと推論法則 \ref{dedeqfund}によって3)が成り立つことがわかる.
\halmos




\mathstrut
\begin{thm}
\label{sthmprsetinv}%定理
$a$を集合とするとき, 
\[
  {\rm pr}_{1}\langle a^{-1} \rangle = {\rm pr}_{2}\langle a \rangle, ~~
  {\rm pr}_{2}\langle a^{-1} \rangle = {\rm pr}_{1}\langle a \rangle
\]
が成り立つ.
\end{thm}


\noindent{\bf 証明}
~$x$と$y$を, 互いに異なり, 共に$a$の中に自由変数として現れない, 定数でない文字とする.
このとき変数法則 \ref{valinv}により, $x$と$y$は共に$a^{-1}$の中に自由変数として現れない.
そこで定理 \ref{sthmprsetelement}より
\begin{align*}
  \tag{1}
  x \in {\rm pr}_{1}\langle a^{-1} \rangle &\leftrightarrow \exists y((x, y) \in a^{-1}), \\
  \mbox{} \\
  \tag{2}
  y \in {\rm pr}_{2}\langle a^{-1} \rangle &\leftrightarrow \exists x((x, y) \in a^{-1})
\end{align*}
が共に成り立つ.
また定理 \ref{sthmpairininv}より
\[
  (x, y) \in a^{-1} \leftrightarrow (y, x) \in a
\]
が成り立つから, これと$x$, $y$が共に定数でないことから, 
推論法則 \ref{dedalleqquansepconst}により
\begin{align*}
  \tag{3}
  \exists y((x, y) \in a^{-1}) &\leftrightarrow \exists y((y, x) \in a), \\
  \mbox{} \\
  \tag{4}
  \exists x((x, y) \in a^{-1}) &\leftrightarrow \exists x((y, x) \in a)
\end{align*}
が共に成り立つ.
また$x$と$y$が互いに異なり, 共に$a$の中に自由変数として現れないことから, 
定理 \ref{sthmprsetelement}と推論法則 \ref{dedeqch}により
\begin{align*}
  \tag{5}
  \exists y((y, x) \in a) \leftrightarrow x \in {\rm pr}_{2}\langle a \rangle, \\
  \mbox{} \\
  \tag{6}
  \exists x((y, x) \in a) \leftrightarrow y \in {\rm pr}_{1}\langle a \rangle
\end{align*}
が共に成り立つ.
そこで(1), (3), (5)から, 推論法則 \ref{dedeqtrans}によって
\[
\tag{7}
  x \in {\rm pr}_{1}\langle a^{-1} \rangle \leftrightarrow x \in {\rm pr}_{2}\langle a \rangle
\]
が成り立ち, (2), (4), (6)から, 同じく推論法則 \ref{dedeqtrans}によって
\[
\tag{8}
  y \in {\rm pr}_{2}\langle a^{-1} \rangle \leftrightarrow y \in {\rm pr}_{1}\langle a \rangle
\]
が成り立つ.
さて上述のように$x$と$y$は共に$a$及び$a^{-1}$の中に自由変数として現れないから, 
変数法則 \ref{valprset}により, 
$x$と$y$は共に${\rm pr}_{1}\langle a \rangle$, ${\rm pr}_{2}\langle a \rangle$, 
${\rm pr}_{1}\langle a^{-1} \rangle$, ${\rm pr}_{2}\langle a^{-1} \rangle$の
いずれの記号列の中にも自由変数として現れない.
また$x$と$y$は共に定数でない.
そこでこれらのことと, (7), (8)が成り立つことから, 
定理 \ref{sthmset=}により
${\rm pr}_{1}\langle a^{-1} \rangle = {\rm pr}_{2}\langle a \rangle$と
${\rm pr}_{2}\langle a^{-1} \rangle = {\rm pr}_{1}\langle a \rangle$が共に成り立つ.
\halmos




\mathstrut
$a$と$b$を集合とするとき, 集合$a^{-1}[b]$を$a$による$b$の\textbf{逆像}という.
これについて次の定理が成り立つ.




\mathstrut
\begin{thm}
\label{sthmvaluesetinv}%定理
$a$と$b$を集合とするとき, 
\[
  b \subset {\rm pr}_{1}\langle a \rangle \leftrightarrow b \subset a^{-1}[a[b]], ~~
  b \subset {\rm pr}_{2}\langle a \rangle \leftrightarrow b \subset a[a^{-1}[b]]
\]
が成り立つ.
またこれらから, 次の1), 2)が成り立つ.

1)
$b \subset {\rm pr}_{1}\langle a \rangle$が成り立つならば, $b \subset a^{-1}[a[b]]$が成り立つ.
逆に$b \subset a^{-1}[a[b]]$が成り立つならば, $b \subset {\rm pr}_{1}\langle a \rangle$が成り立つ.

2)
$b \subset {\rm pr}_{2}\langle a \rangle$が成り立つならば, $b \subset a[a^{-1}[b]]$が成り立つ.
逆に$b \subset a[a^{-1}[b]]$が成り立つならば, $b \subset {\rm pr}_{2}\langle a \rangle$が成り立つ.
\end{thm}


\noindent{\bf 証明}
~まず$b \subset {\rm pr}_{1}\langle a \rangle \leftrightarrow b \subset a^{-1}[a[b]]$が
成り立つことを示す.
推論法則 \ref{dedequiv}があるから, 
\begin{align*}
  \tag{1}
  &b \subset {\rm pr}_{1}\langle a \rangle \to b \subset a^{-1}[a[b]], \\
  \mbox{} \\
  \tag{2}
  &b \subset a^{-1}[a[b]] \to b \subset {\rm pr}_{1}\langle a \rangle
\end{align*}
が共に成り立つことを示せば良い.

(1)の証明: 
$x$を$a$及び$b$の中に自由変数として現れない, 定数でない文字とする.
このとき変数法則 \ref{valvalueset}, \ref{valinv}により, $x$は$a^{-1}[a[b]]$の中にも自由変数として現れない.
またThm \ref{awbta}より
\[
\tag{3}
  b \subset {\rm pr}_{1}\langle a \rangle \wedge x \in b \to x \in b
\]
が成り立つ.
また定理 \ref{sthmsubsetbasis}より
\[
  b \subset {\rm pr}_{1}\langle a \rangle \to (x \in b \to x \in {\rm pr}_{1}\langle a \rangle)
\]
が成り立つから, 推論法則 \ref{dedtwch}により
\[
\tag{4}
  b \subset {\rm pr}_{1}\langle a \rangle \wedge x \in b \to x \in {\rm pr}_{1}\langle a \rangle
\]
が成り立つ.
そこで(3), (4)から, 推論法則 \ref{dedprewedge}によって
\[
\tag{5}
  b \subset {\rm pr}_{1}\langle a \rangle \wedge x \in b \to x \in b \wedge x \in {\rm pr}_{1}\langle a \rangle
\]
が成り立つ.
またいま$y$を$x$と異なり, $a$の中に自由変数として現れない文字とすれば, 
定理 \ref{sthmprsetelement}と推論法則 \ref{dedequiv}により
\[
  x \in {\rm pr}_{1}\langle a \rangle \to \exists y((x, y) \in a)
\]
が成り立つ.
ここで$\tau_{y}((x, y) \in a)$を$T$と書けば, $T$は集合であり, 
定義から上記の記号列は
\[
  x \in {\rm pr}_{1}\langle a \rangle \to (T|y)((x, y) \in a)
\]
と同じである.
また$y$が$x$と異なり, $a$の中に自由変数として現れないことから, 
代入法則 \ref{substfree}, \ref{substfund}, \ref{substpair}により, この記号列は
\[
  x \in {\rm pr}_{1}\langle a \rangle \to (x, T) \in a
\]
と一致する.
よってこれが定理となる.
そこで推論法則 \ref{dedaddw}により, 
\[
\tag{6}
  x \in b \wedge x \in {\rm pr}_{1}\langle a \rangle \to x \in b \wedge (x, T) \in a
\]
が成り立つ.
また定理 \ref{sthmvaluesetbasis}より
\[
\tag{7}
  x \in b \wedge (x, T) \in a \to T \in a[b]
\]
が成り立つ.
またいまThm \ref{awbta}より
\[
  x \in b \wedge (x, T) \in a \to (x, T) \in a
\]
が成り立ち, 定理 \ref{sthmpairininv}と推論法則 \ref{dedequiv}により
\[
  (x, T) \in a \to (T, x) \in a^{-1}
\]
が成り立つから, これらから推論法則 \ref{dedmmp}によって
\[
  x \in b \wedge (x, T) \in a \to (T, x) \in a^{-1}
\]
が成り立つ.
そこでこれと(7)から, 推論法則 \ref{dedprewedge}によって
\[
\tag{8}
  x \in b \wedge (x, T) \in a \to T \in a[b] \wedge (T, x) \in a^{-1}
\]
が成り立つ.
また定理 \ref{sthmvaluesetbasis}より
\[
\tag{9}
  T \in a[b] \wedge (T, x) \in a^{-1} \to x \in a^{-1}[a[b]]
\]
が成り立つ.
以上の(5), (6), (8), (9)から, 推論法則 \ref{dedmmp}によって
\[
  b \subset {\rm pr}_{1}\langle a \rangle \wedge x \in b \to x \in a^{-1}[a[b]]
\]
が成り立つことがわかり, これから推論法則 \ref{dedtwch}により
\[
\tag{10}
  b \subset {\rm pr}_{1}\langle a \rangle \to (x \in b \to x \in a^{-1}[a[b]])
\]
が成り立つ.
さて$x$は$a$及び$b$の中に自由変数として現れないから, 
変数法則 \ref{valsubset}, \ref{valprset}により, $x$は
$b \subset {\rm pr}_{1}\langle a \rangle$の中に自由変数として現れない.
また$x$は定数でない.
そこでこれらのことと, (10)が成り立つことから, 推論法則 \ref{dedalltquansepfreeconst}により
\[
  b \subset {\rm pr}_{1}\langle a \rangle \to \forall x(x \in b \to x \in a^{-1}[a[b]])
\]
が成り立つ.
はじめに述べたように, $x$は$b$及び$a^{-1}[a[b]]$の中に自由変数として現れないから, 
定義よりこの記号列は(1)と同じである.
故に(1)が成り立つ.

(2)の証明: 
定理 \ref{sthmvaluesetsubsetpr2set}より
\[
\tag{11}
  a^{-1}[a[b]] \subset {\rm pr}_{2}\langle a^{-1} \rangle
\]
が成り立つ.
また定理 \ref{sthmprsetinv}より
${\rm pr}_{2}\langle a^{-1} \rangle = {\rm pr}_{1}\langle a \rangle$が成り立つから, 
定理 \ref{sthm=tsubseteq}により
\[
\tag{12}
  a^{-1}[a[b]] \subset {\rm pr}_{2}\langle a^{-1} \rangle 
  \leftrightarrow a^{-1}[a[b]] \subset {\rm pr}_{1}\langle a \rangle
\]
が成り立つ.
そこで(11), (12)から, 推論法則 \ref{dedeqfund}によって
\[
  a^{-1}[a[b]] \subset {\rm pr}_{1}\langle a \rangle
\]
が成り立ち, これから推論法則 \ref{dedatawbtrue2}によって
\[
\tag{13}
  b \subset a^{-1}[a[b]] \to 
  b \subset a^{-1}[a[b]] \wedge a^{-1}[a[b]] \subset {\rm pr}_{1}\langle a \rangle
\]
が成り立つ.
また定理 \ref{sthmsubsettrans}より
\[
\tag{14}
  b \subset a^{-1}[a[b]] \wedge a^{-1}[a[b]] \subset {\rm pr}_{1}\langle a \rangle 
  \to b \subset {\rm pr}_{1}\langle a \rangle
\]
が成り立つ.
そこで(13), (14)から, 推論法則 \ref{dedmmp}によって(2)が成り立つ.

次に$b \subset {\rm pr}_{2}\langle a \rangle \leftrightarrow b \subset a[a^{-1}[b]]$が
成り立つことを示す.
定理 \ref{sthmprsetinv}と推論法則 \ref{ded=ch}により
${\rm pr}_{2}\langle a \rangle = {\rm pr}_{1}\langle a^{-1} \rangle$が成り立つから, 
定理 \ref{sthm=tsubseteq}により
\[
\tag{15}
  b \subset {\rm pr}_{2}\langle a \rangle \leftrightarrow b \subset {\rm pr}_{1}\langle a^{-1} \rangle
\]
が成り立つ.
また上に示したように
$b \subset {\rm pr}_{1}\langle a \rangle \leftrightarrow b \subset a^{-1}[a[b]]$が成り立つから, 
$a$を$a^{-1}$に置き換えた
\[
\tag{16}
  b \subset {\rm pr}_{1}\langle a^{-1} \rangle \leftrightarrow b \subset (a^{-1})^{-1}[a^{-1}[b]]
\]
も成り立つ.
また$x$は上と同じとするとき, これが$a$の中に自由変数として現れないことから, 
定理 \ref{sthmpairsetofainv}と推論法則 \ref{ded=ch}により
$a^{-1} = \{x \in a|{\rm Pair}(x)\}^{-1}$が成り立つ.
そこでこれから定理 \ref{sthminv=}により
$(a^{-1})^{-1} = (\{x \in a|{\rm Pair}(x)\}^{-1})^{-1}$が成り立ち, 
これから定理 \ref{sthmvalueset=}により
\[
\tag{17}
  (a^{-1})^{-1}[a^{-1}[b]] = (\{x \in a|{\rm Pair}(x)\}^{-1})^{-1}[a^{-1}[b]]
\]
が成り立つ.
また$x$が$a$の中に自由変数として現れないことから, 
定理 \ref{sthmpairsetofagraph}より$\{x \in a|{\rm Pair}(x)\}$はグラフであるから, 
定理 \ref{sthminvinv}により
$(\{x \in a|{\rm Pair}(x)\}^{-1})^{-1} = \{x \in a|{\rm Pair}(x)\}$が成り立ち, 
これから定理 \ref{sthmvalueset=}により
\[
\tag{18}
  (\{x \in a|{\rm Pair}(x)\}^{-1})^{-1}[a^{-1}[b]] = \{x \in a|{\rm Pair}(x)\}[a^{-1}[b]]
\]
が成り立つ.
また$x$が$a$の中に自由変数として現れないことから, 
定理 \ref{sthmpairsetofavalueset}により
\[
\tag{19}
  \{x \in a|{\rm Pair}(x)\}[a^{-1}[b]] = a[a^{-1}[b]]
\]
が成り立つ.
そこで(17), (18), (19)から, 推論法則 \ref{ded=trans}によって
\[
  (a^{-1})^{-1}[a^{-1}[b]] = a[a^{-1}[b]]
\]
が成り立つことがわかり, 
これから定理 \ref{sthm=tsubseteq}により
\[
\tag{20}
  b \subset (a^{-1})^{-1}[a^{-1}[b]] \leftrightarrow b \subset a[a^{-1}[b]]
\]
が成り立つ.
以上の(15), (16), (20)から, 推論法則 \ref{dedeqtrans}によって
\[
  b \subset {\rm pr}_{2}\langle a \rangle \leftrightarrow b \subset a[a^{-1}[b]]
\]
が成り立つことがわかる.

1)と2)が成り立つことは, 上で示した二つの定理と推論法則 \ref{dedeqfund}によって明らかである.
\halmos




\mathstrut
\begin{thm}
\label{sthmcutinvelement}%定理
$a$, $b$, $c$を集合とするとき, 
\[
  c \in a^{-1}[\{b\}] \leftrightarrow (c, b) \in a
\]
が成り立つ.
\end{thm}


\noindent{\bf 証明}
~定理 \ref{sthmcutelement}より
\[
  c \in a^{-1}[\{b\}] \leftrightarrow (b, c) \in a^{-1}
\]
が成り立ち, 定理 \ref{sthmpairininv}より
\[
  (b, c) \in a^{-1} \leftrightarrow (c, b) \in a
\]
が成り立つから, これらから, 推論法則 \ref{dedeqtrans}によって
$c \in a^{-1}[\{b\}] \leftrightarrow (c, b) \in a$が成り立つ.
\halmos
%[逆]確認済



\mathstrut
次に二つの集合の合成について述べる.




\mathstrut
\begin{defo}
\label{composition}%変形
$a$と$b$を記号列とする.
また$x$, $y$, $z$, $w$を, どの二つも互いに異なり, 
いずれも$a$及び$b$の中に自由変数として現れない文字とする.
同様に, $x^{*}$, $y^{*}$, $z^{*}$, $w^{*}$を, どの二つも互いに異なり, 
いずれも$a$及び$b$の中に自由変数として現れない文字とする.
このとき
\begin{multline*}
  \{w|\exists x(\exists y(\exists z(((x, y) \in a \wedge (y, z) \in b) \wedge w = (x, z))))\} \\
  \equiv \{w^{*}|\exists x^{*}(\exists y^{*}(\exists z^{*}(((x^{*}, y^{*}) \in a \wedge (y^{*}, z^{*}) \in b) \wedge w^{*} = (x^{*}, z^{*}))))\}
\end{multline*}
が成り立つ.
\end{defo}


\noindent{\bf 証明}
~$p$, $q$, $r$, $s$を, どの二つも互いに異なり, 
どの一つも$x$, $y$, $z$, $w$, $x^{*}$, $y^{*}$, $z^{*}$, $w^{*}$のいずれとも異なり, 
$a$及び$b$の中に自由変数として現れない文字とする.
このとき変数法則 \ref{valfund}, \ref{valwedge}, \ref{valquan}, \ref{valpair}によってわかるように, 
$s$は
$\exists x(\exists y(\exists z(((x, y) \in a \wedge (y, z) \in b) \wedge w = (x, z))))$の中に
自由変数として現れないから, 代入法則 \ref{substisettrans}により
\begin{multline*}
\tag{1}
  \{w|\exists x(\exists y(\exists z(((x, y) \in a \wedge (y, z) \in b) \wedge w = (x, z))))\} \\
  \equiv \{s|(s|w)(\exists x(\exists y(\exists z(((x, y) \in a \wedge (y, z) \in b) \wedge w = (x, z)))))\}
\end{multline*}
が成り立つ.
また$x$, $y$, $z$はいずれも$w$とも$s$とも異なるから, 
代入法則 \ref{substquan}により
\begin{multline*}
\tag{2}
  (s|w)(\exists x(\exists y(\exists z(((x, y) \in a \wedge (y, z) \in b) \wedge w = (x, z))))) \\
  \equiv \exists x(\exists y(\exists z((s|w)(((x, y) \in a \wedge (y, z) \in b) \wedge w = (x, z)))))
\end{multline*}
が成り立つ.
また$w$が$x$, $y$, $z$のいずれとも異なることから, 
変数法則 \ref{valpair}により, $w$は$(x, y)$, $(y, z)$, $(x, z)$のいずれの記号列の中にも自由変数として現れない.
このことと, $w$が$a$及び$b$の中にも自由変数として現れないことから, 
代入法則 \ref{substfree}, \ref{substfund}, \ref{substwedge}により
\[
\tag{3}
  (s|w)(((x, y) \in a \wedge (y, z) \in b) \wedge w = (x, z)) \equiv 
  ((x, y) \in a \wedge (y, z) \in b) \wedge s = (x, z)
\]
が成り立つ.
そこで(1), (2), (3)から, 
\begin{multline*}
\tag{4}
  \{w|\exists x(\exists y(\exists z(((x, y) \in a \wedge (y, z) \in b) \wedge w = (x, z))))\} \\
  \equiv \{s|\exists x(\exists y(\exists z(((x, y) \in a \wedge (y, z) \in b) \wedge s = (x, z))))\}
\end{multline*}
が成り立つことがわかる.
また$p$が$x$, $y$, $z$, $s$のいずれとも異なり, $a$及び$b$の中に自由変数として現れないことから, 
変数法則 \ref{valfund}, \ref{valwedge}, \ref{valquan}, \ref{valpair}によってわかるように, 
$p$は
$\exists y(\exists z(((x, y) \in a \wedge (y, z) \in b) \wedge s = (x, z)))$の中に自由変数として現れない.
そこで代入法則 \ref{substquantrans}により
\begin{multline*}
\tag{5}
  \exists x(\exists y(\exists z(((x, y) \in a \wedge (y, z) \in b) \wedge s = (x, z)))) \\
  \equiv \exists p((p|x)(\exists y(\exists z(((x, y) \in a \wedge (y, z) \in b) \wedge s = (x, z)))))
\end{multline*}
が成り立つ.
また$y$と$z$が共に$x$とも$p$とも異なることから, 
代入法則 \ref{substquan}により
\begin{multline*}
\tag{6}
  (p|x)(\exists y(\exists z(((x, y) \in a \wedge (y, z) \in b) \wedge s = (x, z)))) \\
  \equiv \exists y(\exists z((p|x)(((x, y) \in a \wedge (y, z) \in b) \wedge s = (x, z))))
\end{multline*}
が成り立つ.
また$x$が$y$, $z$, $s$のいずれとも異なり, $a$及び$b$の中に自由変数として現れないことから, 
代入法則 \ref{substfree}, \ref{substfund}, \ref{substwedge}, \ref{substpair}により
\[
\tag{7}
  (p|x)(((x, y) \in a \wedge (y, z) \in b) \wedge s = (x, z)) \equiv 
  ((p, y) \in a \wedge (y, z) \in b) \wedge s = (p, z)
\]
が成り立つ.
そこで(5), (6), (7)から, 
\begin{multline*}
\tag{8}
  \{s|\exists x(\exists y(\exists z(((x, y) \in a \wedge (y, z) \in b) \wedge s = (x, z))))\} \\
  \equiv \{s|\exists p(\exists y(\exists z(((p, y) \in a \wedge (y, z) \in b) \wedge s = (p, z))))\}
\end{multline*}
が成り立つことがわかる.
また$q$が$y$, $z$, $p$, $s$のいずれとも異なり, $a$及び$b$の中に自由変数として現れないことから, 
変数法則 \ref{valfund}, \ref{valwedge}, \ref{valquan}, \ref{valpair}によってわかるように, 
$q$は
$\exists z(((p, y) \in a \wedge (y, z) \in b) \wedge s = (p, z))$の中に自由変数として現れない.
そこで代入法則 \ref{substquantrans}により
\[
\tag{9}
  \exists y(\exists z(((p, y) \in a \wedge (y, z) \in b) \wedge s = (p, z))) \equiv 
  \exists q((q|y)(\exists z(((p, y) \in a \wedge (y, z) \in b) \wedge s = (p, z))))
\]
が成り立つ.
また$z$が$y$とも$q$とも異なることから, 代入法則 \ref{substquan}により
\[
\tag{10}
  (q|y)(\exists z(((p, y) \in a \wedge (y, z) \in b) \wedge s = (p, z))) \equiv 
  \exists z((q|y)(((p, y) \in a \wedge (y, z) \in b) \wedge s = (p, z)))
\]
が成り立つ.
また$y$が$z$, $p$, $s$のいずれとも異なり, $a$及び$b$の中に自由変数として現れないことから, 
代入法則 \ref{substfree}, \ref{substfund}, \ref{substwedge}, \ref{substpair}により
\[
\tag{11}
  (q|y)(((p, y) \in a \wedge (y, z) \in b) \wedge s = (p, z)) \equiv 
  ((p, q) \in a \wedge (q, z) \in b) \wedge s = (p, z)
\]
が成り立つ.
そこで(9), (10), (11)から, 
\begin{multline*}
\tag{12}
  \{s|\exists p(\exists y(\exists z(((p, y) \in a \wedge (y, z) \in b) \wedge s = (p, z))))\} \\
  \equiv \{s|\exists p(\exists q(\exists z(((p, q) \in a \wedge (q, z) \in b) \wedge s = (p, z))))\}
\end{multline*}
が成り立つことがわかる.
また$r$が$z$, $p$, $q$, $s$のいずれとも異なり, $a$及び$b$の中に自由変数として現れないことから, 
変数法則 \ref{valfund}, \ref{valwedge}, \ref{valpair}によってわかるように, 
$r$は
$((p, q) \in a \wedge (q, z) \in b) \wedge s = (p, z)$の中に
自由変数として現れない.
そこで代入法則 \ref{substquantrans}により
\[
\tag{13}
  \exists z(((p, q) \in a \wedge (q, z) \in b) \wedge s = (p, z)) \equiv 
  \exists r((r|z)(((p, q) \in a \wedge (q, z) \in b) \wedge s = (p, z)))
\]
が成り立つ.
また$z$が$p$, $q$, $s$のいずれとも異なり, $a$及び$b$の中に自由変数として現れないことから, 
代入法則 \ref{substfree}, \ref{substfund}, \ref{substwedge}, \ref{substpair}により
\[
\tag{14}
  (r|z)(((p, q) \in a \wedge (q, z) \in b) \wedge s = (p, z)) \equiv 
  ((p, q) \in a \wedge (q, r) \in b) \wedge s = (p, r)
\]
が成り立つ.
そこで(13), (14)から, 
\begin{multline*}
\tag{15}
  \{s|\exists p(\exists q(\exists z(((p, q) \in a \wedge (q, z) \in b) \wedge s = (p, z))))\} \\
  \equiv \{s|\exists p(\exists q(\exists r(((p, q) \in a \wedge (q, r) \in b) \wedge s = (p, r))))\}
\end{multline*}
が成り立つことがわかる.
以上の(4), (8), (12), (15)からわかるように, 
\begin{multline*}
  \{w|\exists x(\exists y(\exists z(((x, y) \in a \wedge (y, z) \in b) \wedge w = (x, z))))\} \\
  \equiv \{s|\exists p(\exists q(\exists r(((p, q) \in a \wedge (q, r) \in b) \wedge s = (p, r))))\}
\end{multline*}
が成り立つ.
またここまでの議論と全く同様にして, 
\begin{multline*}
  \{w^{*}|\exists x^{*}(\exists y^{*}(\exists z^{*}(((x^{*}, y^{*}) \in a \wedge (y^{*}, z^{*}) \in b) \wedge w^{*} = (x^{*}, z^{*}))))\} \\
  \equiv \{s|\exists p(\exists q(\exists r(((p, q) \in a \wedge (q, r) \in b) \wedge s = (p, r))))\}
\end{multline*}
も成り立つ.
故に本法則が成り立つ.
\halmos




\mathstrut
\begin{defi}
\label{defcomp}%定義
$a$と$b$を記号列とする.
また$x$, $y$, $z$, $w$を, どの二つも互いに異なり, 
いずれも$a$及び$b$の中に自由変数として現れない文字とする.
同様に, $x^{*}$, $y^{*}$, $z^{*}$, $w^{*}$を, どの二つも互いに異なり, 
いずれも$a$及び$b$の中に自由変数として現れない文字とする.
このとき上記の変形法則 \ref{composition}によれば, 
$\{w|\exists x(\exists y(\exists z(((x, y) \in a \wedge (y, z) \in b) \wedge w = (x, z))))\}$と
$\{w^{*}|\exists x^{*}(\exists y^{*}(\exists z^{*}(((x^{*}, y^{*}) \in a \wedge (y^{*}, z^{*}) \in b) \wedge w^{*} = (x^{*}, z^{*}))))\}$という
二つの記号列は一致する.
$a$と$b$によって定まるこの記号列を, 
$(b) \circ (a)$と書き表す(括弧は適宜省略する).
\end{defi}




\mathstrut
\begin{valu}
\label{valcomp}%変数
$a$と$b$を記号列とし, $x$を文字とする.
$x$が$a$及び$b$の中に自由変数として現れなければ, 
$x$は$b \circ a$の中に自由変数として現れない.
\end{valu}


\noindent{\bf 証明}
~このとき$u$, $v$, $w$を, どの二つも互いに異なり, いずれも$x$と異なり, 
$a$及び$b$の中に自由変数として現れない文字とすれば, 定義から$b \circ a$は
$\{x|\exists u(\exists v(\exists w(((u, v) \in a \wedge (v, w) \in b) \wedge x = (u, w))))\}$と
同じである.
変数法則 \ref{valiset}により, $x$はこの中に自由変数として現れない.
\halmos




\mathstrut
\begin{subs}
\label{substcomp}%代入
$a$, $b$, $c$を記号列とし, $x$を文字とするとき, 
\[
  (c|x)(b \circ a) \equiv (c|x)(b) \circ (c|x)(a)
\]
が成り立つ.
\end{subs}


\noindent{\bf 証明}
~$p$, $q$, $r$, $s$を, どの二つも互いに異なり, どの一つも
$x$と異なり, $a$, $b$, $c$のいずれの記号列の中にも自由変数として現れない文字とする.
このとき定義から$b \circ a$は
$\{s|\exists p(\exists q(\exists r(((p, q) \in a \wedge (q, r) \in b) \wedge s = (p, r))))\}$と同じである.
そこで$s$が$x$と異なり, $c$の中に自由変数として現れないことから, 
代入法則 \ref{substiset}により
\[
\tag{1}
  (c|x)(b \circ a) \equiv 
  \{s|(c|x)(\exists p(\exists q(\exists r(((p, q) \in a \wedge (q, r) \in b) \wedge s = (p, r)))))\}
\]
が成り立つ.
また$p$, $q$, $r$がいずれも$x$と異なり, $c$の中に自由変数として現れないことから, 
代入法則 \ref{substquan}により
\begin{multline*}
\tag{2}
  (c|x)(\exists p(\exists q(\exists r(((p, q) \in a \wedge (q, r) \in b) \wedge s = (p, r))))) \\
  \equiv \exists p(\exists q(\exists r((c|x)(((p, q) \in a \wedge (q, r) \in b) \wedge s = (p, r)))))
\end{multline*}
が成り立つ.
また$x$が$p$, $q$, $r$, $s$のいずれとも異なることから, 
変数法則 \ref{valfund}, \ref{valpair}により, 
$x$は$(p, q)$, $(q, r)$, $s = (p, r)$のいずれの記号列の中にも自由変数として現れないから, 
このことと代入法則 \ref{substfree}, \ref{substfund}, \ref{substwedge}により
\[
\tag{3}
  (c|x)(((p, q) \in a \wedge (q, r) \in b) \wedge s = (p, r)) \equiv 
  ((p, q) \in (c|x)(a) \wedge (q, r) \in (c|x)(b)) \wedge s = (p, r)
\]
が成り立つ.
以上の(1), (2), (3)から, $(c|x)(b \circ a)$が
\[
\tag{4}
  \{s|\exists p(\exists q(\exists r(((p, q) \in (c|x)(a) \wedge (q, r) \in (c|x)(b)) \wedge s = (p, r))))\}
\]
と一致することがわかる.
いま$p$, $q$, $r$, $s$はどの一つも$a$, $b$, $c$のいずれの記号列の中にも自由変数として現れないから, 
変数法則 \ref{valsubst}により, 
これらはどの一つも$(c|x)(a)$及び$(c|x)(b)$の中に自由変数として現れない.
また$p$, $q$, $r$, $s$はどの二つも互いに異なる.
よって定義から, 上記の(4)は$(c|x)(b) \circ (c|x)(a)$と書き表される記号列である.
故に本法則が成り立つ.
\halmos




\mathstrut
\begin{form}
\label{formcomp}%構成
$a$と$b$が集合ならば, $b \circ a$は集合である.
\end{form}


\noindent{\bf 証明}
~$x$, $y$, $z$, $w$を, どの二つも互いに異なり, いずれも$a$及び$b$の中に自由変数として現れない文字とすれば, 
定義から$b \circ a$は
$\{w|\exists x(\exists y(\exists z(((x, y) \in a \wedge (y, z) \in b) \wedge w = (x, z))))\}$と同じである.
$a$と$b$が共に集合であるとき, 
構成法則 \ref{formfund}, \ref{formwedge}, \ref{formquan}, \ref{formiset}, \ref{formpair}によって
わかるように, これは集合である.
\halmos




\mathstrut
$a$と$b$が集合であるとき, 集合$b \circ a$を$a$と$b$の\textbf{合成}という
(この呼称を用いるときは, $a$と$b$の順序に注意する必要がある).




\mathstrut
\begin{thm}
\label{sthmcompsetmake}%定理
$a$と$b$を集合とし, $x$, $y$, $z$, $w$を, どの二つも互いに異なり, いずれも$a$及び$b$の中に
自由変数として現れない文字とする.
このとき関係式
$\exists x(\exists y(\exists z(((x, y) \in a \wedge (y, z) \in b) \wedge w = (x, z))))$は
$w$について集合を作り得る.
\end{thm}


\noindent{\bf 証明}
~$\exists x(\exists y(\exists z(((x, y) \in a \wedge (y, z) \in b) \wedge w = (x, z))))$を
$R$と書く.
また$u$を$x$, $y$, $z$, $w$のいずれとも異なり, $a$及び$b$の中に自由変数として現れない, 
定数でない文字とする.
このとき変数法則 \ref{valproduct}, \ref{valprset}により, 
$u$は${\rm pr}_{1}\langle a \rangle \times {\rm pr}_{2}\langle b \rangle$の中に自由変数として現れない.
また変数法則 \ref{valfund}, \ref{valwedge}, \ref{valquan}, \ref{valpair}からわかるように, 
$u$は$R$の中にも自由変数として現れない.
そして$x$, $y$, $z$がいずれも$w$とも$u$とも異なることから, 
代入法則 \ref{substquan}により
\[
  (u|w)(R) \equiv \exists x(\exists y(\exists z((u|w)(((x, y) \in a \wedge (y, z) \in b) \wedge w = (x, z)))))
\]
が成り立つ.
また$w$が$x$, $y$, $z$のいずれとも異なり, $a$及び$b$の中に自由変数として現れないことから, 
変数法則 \ref{valfund}, \ref{valwedge}, \ref{valpair}により, $w$は
$(x, y) \in a \wedge (y, z) \in b$及び$(x, z)$の中に自由変数として現れない.
そこで代入法則 \ref{substfree}, \ref{substfund}, \ref{substwedge}により
\[
  (u|w)(((x, y) \in a \wedge (y, z) \in b) \wedge w = (x, z)) \equiv ((x, y) \in a \wedge (y, z) \in b) \wedge u = (x, z)
\]
が成り立つ.
そこでこれらから, 
\[
\tag{1}
  (u|w)(R) \equiv \exists x(\exists y(\exists z(((x, y) \in a \wedge (y, z) \in b) \wedge u = (x, z))))
\]
が成り立つことがわかる.
またいま$\tau_{x}(\exists y(\exists z(((x, y) \in a \wedge (y, z) \in b) \wedge u = (x, z))))$を
$T$と書けば, $T$は集合であり, 変数法則 \ref{valtau}, \ref{valquan}からわかるように, 
$y$と$z$は共にこの中に自由変数として現れない.
そして定義から
\begin{multline*}
\tag{2}
  \exists x(\exists y(\exists z(((x, y) \in a \wedge (y, z) \in b) \wedge u = (x, z)))) \\
  \equiv (T|x)(\exists y(\exists z(((x, y) \in a \wedge (y, z) \in b) \wedge u = (x, z))))
\end{multline*}
である.
また$y$と$z$が共に$x$と異なり, いま述べたように$T$の中に自由変数として現れないことから, 
代入法則 \ref{substquan}により
\begin{multline*}
\tag{3}
  (T|x)(\exists y(\exists z(((x, y) \in a \wedge (y, z) \in b) \wedge u = (x, z)))) \\
  \equiv \exists y(\exists z((T|x)(((x, y) \in a \wedge (y, z) \in b) \wedge u = (x, z))))
\end{multline*}
が成り立つ.
また$x$が$y$, $z$, $u$のいずれとも異なり, $a$及び$b$の中に自由変数として現れないことから, 
代入法則 \ref{substfree}, \ref{substfund}, \ref{substwedge}, \ref{substpair}により
\[
\tag{4}
  (T|x)(((x, y) \in a \wedge (y, z) \in b) \wedge u = (x, z)) \equiv 
  ((T, y) \in a \wedge (y, z) \in b) \wedge u = (T, z)
\]
が成り立つ.
そこで(2), (3), (4)から, 
\[
\tag{5}
  \exists x(\exists y(\exists z(((x, y) \in a \wedge (y, z) \in b) \wedge u = (x, z)))) 
  \equiv \exists y(\exists z(((T, y) \in a \wedge (y, z) \in b) \wedge u = (T, z)))
\]
が成り立つことがわかる.
またいま$\tau_{y}(\exists z(((T, y) \in a \wedge (y, z) \in b) \wedge u = (T, z)))$を$U$と書けば, 
$U$は集合であり, 変数法則 \ref{valtau}, \ref{valquan}によってわかるように, 
$z$はこの中に自由変数として現れない.
そして定義から
\[
\tag{6}
  \exists y(\exists z(((T, y) \in a \wedge (y, z) \in b) \wedge u = (T, z))) 
  \equiv (U|y)(\exists z(((T, y) \in a \wedge (y, z) \in b) \wedge u = (T, z)))
\]
である.
また$z$が$y$と異なり, いま述べたように$U$の中に自由変数として現れないことから, 
代入法則 \ref{substquan}により
\[
\tag{7}
  (U|y)(\exists z(((T, y) \in a \wedge (y, z) \in b) \wedge u = (T, z))) 
  \equiv \exists z((U|y)(((T, y) \in a \wedge (y, z) \in b) \wedge u = (T, z)))
\]
が成り立つ.
また$y$が$z$とも$u$とも異なり, $a$及び$b$の中に自由変数として現れず, 
上述のように$T$の中にも自由変数として現れないことから, 
代入法則 \ref{substfree}, \ref{substfund}, \ref{substwedge}, \ref{substpair}により
\[
\tag{8}
  (U|y)(((T, y) \in a \wedge (y, z) \in b) \wedge u = (T, z)) \equiv 
  ((T, U) \in a \wedge (U, z) \in b) \wedge u = (T, z)
\]
が成り立つ.
そこで(6), (7), (8)から, 
\[
\tag{9}
  \exists y(\exists z(((T, y) \in a \wedge (y, z) \in b) \wedge u = (T, z))) 
  \equiv \exists z(((T, U) \in a \wedge (U, z) \in b) \wedge u = (T, z))
\]
が成り立つことがわかる.
またいま$\tau_{z}(((T, U) \in a \wedge (U, z) \in b) \wedge u = (T, z))$を$V$と書けば, 
$V$は集合であり, 定義から
\[
\tag{10}
  \exists z(((T, U) \in a \wedge (U, z) \in b) \wedge u = (T, z)) \equiv 
  (V|z)(((T, U) \in a \wedge (U, z) \in b) \wedge u = (T, z))
\]
である.
また$z$が$u$と異なり, $a$及び$b$の中に自由変数として現れず, 
上述のように$T$及び$U$の中にも自由変数として現れないことから, 
代入法則 \ref{substfree}, \ref{substfund}, \ref{substwedge}, \ref{substpair}により
\[
\tag{11}
  (V|z)(((T, U) \in a \wedge (U, z) \in b) \wedge u = (T, z)) \equiv 
  ((T, U) \in a \wedge (U, V) \in b) \wedge u = (T, V)
\]
が成り立つ.
そこで(10), (11)から, 
\[
\tag{12}
  \exists z(((T, U) \in a \wedge (U, z) \in b) \wedge u = (T, z)) \equiv 
  ((T, U) \in a \wedge (U, V) \in b) \wedge u = (T, V)
\]
が成り立つことがわかる.
さて以上の(1), (5), (9), (12)からわかるように, 
\[
\tag{13}
  (u|w)(R) \equiv ((T, U) \in a \wedge (U, V) \in b) \wedge u = (T, V)
\]
が成り立つ.
いまThm \ref{awbtbwa}より
\[
  ((T, U) \in a \wedge (U, V) \in b) \wedge u = (T, V) \to 
  u = (T, V) \wedge ((T, U) \in a \wedge (U, V) \in b)
\]
が成り立つが, (13)よりこの記号列は
\[
\tag{14}
  (u|w)(R) \to u = (T, V) \wedge ((T, U) \in a \wedge (U, V) \in b)
\]
と一致するから, これが定理となる.
また定理 \ref{sthmpairelementinprset}より
\begin{align*}
  (T, U) \in a &\to T \in {\rm pr}_{1}\langle a \rangle \wedge U \in {\rm pr}_{2}\langle a \rangle, \\
  \mbox{} \\
  (U, V) \in b &\to U \in {\rm pr}_{1}\langle b \rangle \wedge V \in {\rm pr}_{2}\langle b \rangle
\end{align*}
が共に成り立つから, 推論法則 \ref{dedprewedge}により
\begin{align*}
  (T, U) \in a &\to T \in {\rm pr}_{1}\langle a \rangle, \\
  \mbox{} \\
  (U, V) \in b &\to V \in {\rm pr}_{2}\langle b \rangle
\end{align*}
が共に成り立ち, これらから推論法則 \ref{dedfromaddw}によって
\[
\tag{15}
  (T, U) \in a \wedge (U, V) \in b \to 
  T \in {\rm pr}_{1}\langle a \rangle \wedge V \in {\rm pr}_{2}\langle b \rangle
\]
が成り立つ.
また定理 \ref{sthmpairinproduct}と推論法則 \ref{dedequiv}により
\[
\tag{16}
  T \in {\rm pr}_{1}\langle a \rangle \wedge V \in {\rm pr}_{2}\langle b \rangle \to 
  (T, V) \in {\rm pr}_{1}\langle a \rangle \times {\rm pr}_{2}\langle b \rangle
\]
が成り立つ.
そこで(15), (16)から, 推論法則 \ref{dedmmp}によって
\[
  (T, U) \in a \wedge (U, V) \in b \to 
  (T, V) \in {\rm pr}_{1}\langle a \rangle \times {\rm pr}_{2}\langle b \rangle
\]
が成り立ち, これから推論法則 \ref{dedaddw}によって
\[
\tag{17}
  u = (T, V) \wedge ((T, U) \in a \wedge (U, V) \in b) \to 
  u = (T, V) \wedge (T, V) \in {\rm pr}_{1}\langle a \rangle \times {\rm pr}_{2}\langle b \rangle
\]
が成り立つ.
また定理 \ref{sthm=&in}より
\[
\tag{18}
  u = (T, V) \wedge (T, V) \in {\rm pr}_{1}\langle a \rangle \times {\rm pr}_{2}\langle b \rangle \to 
  u \in {\rm pr}_{1}\langle a \rangle \times {\rm pr}_{2}\langle b \rangle
\]
が成り立つ.
そこで(14), (17), (18)から, 推論法則 \ref{dedmmp}によって
\[
  (u|w)(R) \to u \in {\rm pr}_{1}\langle a \rangle \times {\rm pr}_{2}\langle b \rangle
\]
が成り立つことがわかる.
ここで$w$が$a$及び$b$の中に自由変数として現れないことから, 
変数法則 \ref{valproduct}, \ref{valprset}により, $w$は
${\rm pr}_{1}\langle a \rangle \times {\rm pr}_{2}\langle b \rangle$の中に自由変数として現れないから, 
代入法則 \ref{substfree}, \ref{substfund}により, 上記の記号列は
\[
  (u|w)(R \to w \in {\rm pr}_{1}\langle a \rangle \times {\rm pr}_{2}\langle b \rangle)
\]
と一致する.
よってこれが定理となる.
そこで$u$が定数でないことから, 推論法則 \ref{dedltthmquan}により
\[
  \forall u((u|w)(R \to w \in {\rm pr}_{1}\langle a \rangle \times {\rm pr}_{2}\langle b \rangle))
\]
が成り立つ.
ここで$u$が$w$と異なり, はじめに述べたように$R$及び
${\rm pr}_{1}\langle a \rangle \times {\rm pr}_{2}\langle b \rangle$の中に
自由変数として現れないことから, 変数法則 \ref{valfund}により, 
$u$が$R \to w \in {\rm pr}_{1}\langle a \rangle \times {\rm pr}_{2}\langle b \rangle$の中に
自由変数として現れないことがわかるから, 
代入法則 \ref{substquantrans}により, 上記の記号列は
\[
\tag{19}
  \forall w(R \to w \in {\rm pr}_{1}\langle a \rangle \times {\rm pr}_{2}\langle b \rangle)
\]
と一致する.
よってこれが定理となる.
いま述べたように$w$は${\rm pr}_{1}\langle a \rangle \times {\rm pr}_{2}\langle b \rangle$の中に
自由変数として現れないので, この(19)から, 
定理 \ref{sthmalltiset=sset}によって$R$, 即ち
$\exists x(\exists y(\exists z(((x, y) \in a \wedge (y, z) \in b) \wedge w = (x, z))))$が
$w$について集合を作り得ることがわかる.
\halmos




\mathstrut
\begin{thm}
\label{sthmcompelement}%定理
$a$, $b$, $c$を集合とし, $y$をこれらの中に自由変数として現れない文字とする.
このとき
\[
  c \in b \circ a \leftrightarrow {\rm Pair}(c) \wedge \exists y(({\rm pr}_{1}(c), y) \in a \wedge (y, {\rm pr}_{2}(c)) \in b)
\]
が成り立つ.
\end{thm}


\noindent{\bf 証明}
~$x$, $z$, $w$を, どの二つも互いに異なり, いずれも$y$と異なり, 
$a$, $b$, $c$の中に自由変数として現れない文字とする.
このときこれらのことと$y$に対する仮定から, 
定義により$b \circ a$は
$\{w|\exists x(\exists y(\exists z(((x, y) \in a \wedge (y, z) \in b) \wedge w = (x, z))))\}$と同じである.
また定理 \ref{sthmcompsetmake}より, 
$\exists x(\exists y(\exists z(((x, y) \in a \wedge (y, z) \in b) \wedge w = (x, z))))$は
$w$について集合を作り得る.
そこで定理 \ref{sthmisetbasis}より
\[
  c \in b \circ a \leftrightarrow (c|w)(\exists x(\exists y(\exists z(((x, y) \in a \wedge (y, z) \in b) \wedge w = (x, z)))))
\]
が成り立つ.
ここで$x$, $y$, $z$がいずれも$w$と異なり, $c$の中に自由変数として現れないことから, 
代入法則 \ref{substquan}により上記の記号列は
\[
  c \in b \circ a \leftrightarrow \exists x(\exists y(\exists z((c|w)(((x, y) \in a \wedge (y, z) \in b) \wedge w = (x, z)))))
\]
と一致する.
また$w$が$x$, $y$, $z$のいずれとも異なり, $a$及び$b$の中に自由変数として現れないことから, 
変数法則 \ref{valfund}, \ref{valwedge}, \ref{valpair}により$w$は
$(x, y) \in a \wedge (y, z) \in b$及び$(x, z)$の中に自由変数として現れないから, 
代入法則 \ref{substfree}, \ref{substfund}, \ref{substwedge}により
この記号列は
\[
  c \in b \circ a \leftrightarrow \exists x(\exists y(\exists z(((x, y) \in a \wedge (y, z) \in b) \wedge c = (x, z))))
\]
と一致する.
そこでいま関係式$\exists x(\exists y(\exists z(((x, y) \in a \wedge (y, z) \in b) \wedge c = (x, z))))$を
$R$と書けば, 以上のことから
\[
\tag{1}
  c \in b \circ a \leftrightarrow R
\]
が定理となることがわかる.
またいま$\tau_{x}(\exists y(\exists z(((x, y) \in a \wedge (y, z) \in b) \wedge c = (x, z))))$を$T$と書けば, 
$T$は集合であり, 変数法則 \ref{valtau}, \ref{valquan}により, $y$と$z$は共に
この中に自由変数として現れない.
そして定義から, $R$, 即ち$\exists x(\exists y(\exists z(((x, y) \in a \wedge (y, z) \in b) \wedge c = (x, z))))$は, 
\[
  (T|x)(\exists y(\exists z(((x, y) \in a \wedge (y, z) \in b) \wedge c = (x, z))))
\]
と同じである.
また$y$と$z$が共に$x$と異なり, いま述べたように$T$の中に自由変数として現れないことから, 
代入法則 \ref{substquan}により, 上記の記号列は
\[
  \exists y(\exists z((T|x)(((x, y) \in a \wedge (y, z) \in b) \wedge c = (x, z))))
\]
と一致する.
また$x$が$y$とも$z$とも異なり, $a$, $b$, $c$のいずれの記号列の中にも自由変数として現れないことから, 
代入法則 \ref{substfree}, \ref{substfund}, \ref{substwedge}, \ref{substpair}により, この記号列は
\[
  \exists y(\exists z(((T, y) \in a \wedge (y, z) \in b) \wedge c = (T, z)))
\]
と一致する.
またいま$\tau_{y}(\exists z(((T, y) \in a \wedge (y, z) \in b) \wedge c = (T, z)))$を$U$と書けば, 
$U$は集合であり, 変数法則 \ref{valtau}, \ref{valquan}により, 
$z$はこの中に自由変数として現れない.
そして定義から, 上記の記号列は
\[
  (U|y)(\exists z(((T, y) \in a \wedge (y, z) \in b) \wedge c = (T, z)))
\]
と同じである.
また$z$が$y$と異なり, いま述べたように$U$の中に自由変数として現れないことから, 
代入法則 \ref{substquan}により, この記号列は
\[
  \exists z((U|y)(((T, y) \in a \wedge (y, z) \in b) \wedge c = (T, z)))
\]
と一致する.
また$y$が$z$と異なり, $a$, $b$, $c$のいずれの記号列の中にも自由変数として現れず, 
上述のように$T$の中にも自由変数として現れないことから, 
代入法則 \ref{substfree}, \ref{substfund}, \ref{substwedge}, \ref{substpair}により, この記号列は
\[
  \exists z(((T, U) \in a \wedge (U, z) \in b) \wedge c = (T, z))
\]
と一致する.
またいま$\tau_{z}(((T, U) \in a \wedge (U, z) \in b) \wedge c = (T, z))$を$V$と書けば, 
$V$は集合であり, 定義から上記の記号列は
\[
  (V|z)(((T, U) \in a \wedge (U, z) \in b) \wedge c = (T, z))
\]
と同じである.
また$z$は$a$, $b$, $c$のいずれの記号列の中にも自由変数として現れず, 
上述のように$T$及び$U$の中にも自由変数として現れないから, 
代入法則 \ref{substfree}, \ref{substfund}, \ref{substwedge}, \ref{substpair}により, この記号列は
\[
  ((T, U) \in a \wedge (U, V) \in b) \wedge c = (T, V)
\]
と一致する.
以上のことから, $R$が$((T, U) \in a \wedge (U, V) \in b) \wedge c = (T, V)$と
一致することがわかる.
いまThm \ref{awbta}より
\begin{align*}
  ((T, U) \in a \wedge (U, V) \in b) \wedge c = (T, V) &\to c = (T, V), \\
  \mbox{} \\
  ((T, U) \in a \wedge (U, V) \in b) \wedge c = (T, V) &\to (T, U) \in a \wedge (U, V) \in b
\end{align*}
が共に成り立つから, 従って
\begin{align*}
  \tag{2}
  R &\to c = (T, V), \\
  \mbox{} \\
  \tag{3}
  R &\to (T, U) \in a \wedge (U, V) \in b
\end{align*}
が共に定理となる.
また同じくThm \ref{awbta}より
\begin{align*}
  \tag{4}
  (T, U) \in a \wedge (U, V) \in b &\to (T, U) \in a, \\
  \mbox{} \\
  \tag{5}
  (T, U) \in a \wedge (U, V) \in b &\to (U, V) \in b
\end{align*}
が共に成り立つ.
そこで(3)と(4), (3)と(5)から, それぞれ推論法則 \ref{dedmmp}によって
\begin{align*}
  \tag{6}
  R &\to (T, U) \in a, \\
  \mbox{} \\
  \tag{7}
  R &\to (U, V) \in b
\end{align*}
が成り立つ.
また定理 \ref{sthmpairpreq}と推論法則 \ref{dedequiv}により
\[
  c = (T, V) \to {\rm Pair}(c) \wedge (T = {\rm pr}_{1}(c) \wedge V = {\rm pr}_{2}(c))
\]
が成り立つから, 推論法則 \ref{dedprewedge}によって
\begin{align*}
  \tag{8}
  c = (T, V) &\to {\rm Pair}(c), \\
  \mbox{} \\
  \tag{9}
  c = (T, V) &\to T = {\rm pr}_{1}(c), \\
  \mbox{} \\
  \tag{10}
  c = (T, V) &\to V = {\rm pr}_{2}(c)
\end{align*}
がすべて成り立つことがわかる.
そこで(2)と(8)から, 推論法則 \ref{dedmmp}によって
\[
\tag{11}
  R \to {\rm Pair}(c)
\]
が成り立つ.
また定理 \ref{sthmpairweak}と推論法則 \ref{dedequiv}により
\begin{align*}
  \tag{12}
  T = {\rm pr}_{1}(c) &\to (T, U) = ({\rm pr}_{1}(c), U), \\
  \mbox{} \\
  \tag{13}
  V = {\rm pr}_{2}(c) &\to (U, V) = (U, {\rm pr}_{2}(c))
\end{align*}
が共に成り立つ.
そこで(2), (9), (12)から, 推論法則 \ref{dedmmp}によって
\[
\tag{14}
  R \to (T, U) = ({\rm pr}_{1}(c), U)
\]
が成り立ち, (2), (10), (13)から, 同じく推論法則 \ref{dedmmp}によって
\[
\tag{15}
  R \to (U, V) = (U, {\rm pr}_{2}(c))
\]
が成り立つことがわかる.
そこで(14)と(6), (15)と(7)から, それぞれ推論法則 \ref{dedprewedge}によって
\begin{align*}
  \tag{16}
  R &\to (T, U) = ({\rm pr}_{1}(c), U) \wedge (T, U) \in a, \\
  \mbox{} \\
  \tag{17}
  R &\to (U, V) = (U, {\rm pr}_{2}(c)) \wedge (U, V) \in b
\end{align*}
が成り立つ.
また定理 \ref{sthm=&in}より
\begin{align*}
  \tag{18}
  (T, U) = ({\rm pr}_{1}(c), U) \wedge (T, U) \in a &\to ({\rm pr}_{1}(c), U) \in a, \\
  \mbox{} \\
  \tag{19}
  (U, V) = (U, {\rm pr}_{2}(c)) \wedge (U, V) \in b &\to (U, {\rm pr}_{2}(c)) \in b
\end{align*}
が共に成り立つ.
そこで(16)と(18), (17)と(19)から, それぞれ推論法則 \ref{dedmmp}によって
\[
  R \to ({\rm pr}_{1}(c), U) \in a, ~~
  R \to (U, {\rm pr}_{2}(c)) \in b
\]
が成り立ち, これらから, 推論法則 \ref{dedprewedge}によって
\[
  R \to ({\rm pr}_{1}(c), U) \in a \wedge (U, {\rm pr}_{2}(c)) \in b
\]
が成り立つ.
いま$y$は$c$の中に自由変数として現れないから, 
変数法則 \ref{valpr}により, $y$は${\rm pr}_{1}(c)$及び${\rm pr}_{2}(c)$の中に自由変数として現れない.
このことと, $y$が$a$及び$b$の中にも自由変数として現れないことから, 
代入法則 \ref{substfree}, \ref{substfund}, \ref{substwedge}, \ref{substpair}により, 
上記の記号列は
\[
\tag{20}
  R \to (U|y)(({\rm pr}_{1}(c), y) \in a \wedge (y, {\rm pr}_{2}(c)) \in b)
\]
と一致する.
よってこれが定理となる.
またschema S4の適用により
\[
\tag{21}
  (U|y)(({\rm pr}_{1}(c), y) \in a \wedge (y, {\rm pr}_{2}(c)) \in b) \to 
  \exists y(({\rm pr}_{1}(c), y) \in a \wedge (y, {\rm pr}_{2}(c)) \in b)
\]
が成り立つ.
そこで(20), (21)から, 推論法則 \ref{dedmmp}によって
\[
\tag{22}
  R \to \exists y(({\rm pr}_{1}(c), y) \in a \wedge (y, {\rm pr}_{2}(c)) \in b)
\]
が成り立つ.
そして(11), (22)から, 推論法則 \ref{dedprewedge}によって
\[
\tag{23}
  R \to {\rm Pair}(c) \wedge \exists y(({\rm pr}_{1}(c), y) \in a \wedge (y, {\rm pr}_{2}(c)) \in b)
\]
が成り立つ.
また定理 \ref{sthmbigpairpr}と推論法則 \ref{dedequiv}により
${\rm Pair}(c) \to c = ({\rm pr}_{1}(c), {\rm pr}_{2}(c))$が成り立つから, 
推論法則 \ref{dedaddw}により
\begin{multline*}
\tag{24}
  {\rm Pair}(c) \wedge \exists y(({\rm pr}_{1}(c), y) \in a \wedge (y, {\rm pr}_{2}(c)) \in b) \\
  \to c = ({\rm pr}_{1}(c), {\rm pr}_{2}(c)) \wedge \exists y(({\rm pr}_{1}(c), y) \in a \wedge (y, {\rm pr}_{2}(c)) \in b)
\end{multline*}
が成り立つ.
また上述のように$y$は$c$, ${\rm pr}_{1}(c)$, ${\rm pr}_{2}(c)$の
いずれの記号列の中にも自由変数として現れないから, 
変数法則 \ref{valfund}, \ref{valpair}により, 
$y$は$c = ({\rm pr}_{1}(c), {\rm pr}_{2}(c))$の中に自由変数として現れない.
そこでThm \ref{thmexwrfree}と推論法則 \ref{dedequiv}により
\begin{multline*}
\tag{25}
  c = ({\rm pr}_{1}(c), {\rm pr}_{2}(c)) \wedge \exists y(({\rm pr}_{1}(c), y) \in a \wedge (y, {\rm pr}_{2}(c)) \in b) \\
  \to \exists y(c = ({\rm pr}_{1}(c), {\rm pr}_{2}(c)) \wedge (({\rm pr}_{1}(c), y) \in a \wedge (y, {\rm pr}_{2}(c)) \in b))
\end{multline*}
が成り立つ.
またThm \ref{thmquanwch}と推論法則 \ref{dedequiv}により
\begin{multline*}
  \exists y(c = ({\rm pr}_{1}(c), {\rm pr}_{2}(c)) \wedge (({\rm pr}_{1}(c), y) \in a \wedge (y, {\rm pr}_{2}(c)) \in b)) \\
  \to \exists y((({\rm pr}_{1}(c), y) \in a \wedge (y, {\rm pr}_{2}(c)) \in b) \wedge c = ({\rm pr}_{1}(c), {\rm pr}_{2}(c)))
\end{multline*}
が成り立つ.
いま$z$は$c$の中に自由変数として現れず, 従って変数法則 \ref{valpr}により, 
$z$は${\rm pr}_{1}(c)$の中にも自由変数として現れない.
このことと, $z$が$y$と異なり, $a$及び$b$の中にも自由変数として現れないことから, 
代入法則 \ref{substfree}, \ref{substfund}, \ref{substwedge}, \ref{substpair}により, 
上記の記号列は
\begin{multline*}
  \exists y(c = ({\rm pr}_{1}(c), {\rm pr}_{2}(c)) \wedge (({\rm pr}_{1}(c), y) \in a \wedge (y, {\rm pr}_{2}(c)) \in b)) \\
  \to \exists y(({\rm pr}_{2}(c)|z)((({\rm pr}_{1}(c), y) \in a \wedge (y, z) \in b) \wedge c = ({\rm pr}_{1}(c), z)))
\end{multline*}
と一致する.
また$y$が$z$と異なり, 上述のように${\rm pr}_{2}(c)$の中に自由変数として現れないことから, 
代入法則 \ref{substquan}により, この記号列は
\begin{multline*}
\tag{26}
  \exists y(c = ({\rm pr}_{1}(c), {\rm pr}_{2}(c)) \wedge (({\rm pr}_{1}(c), y) \in a \wedge (y, {\rm pr}_{2}(c)) \in b)) \\
  \to ({\rm pr}_{2}(c)|z)(\exists y((({\rm pr}_{1}(c), y) \in a \wedge (y, z) \in b) \wedge c = ({\rm pr}_{1}(c), z)))
\end{multline*}
と一致する.
よってこれが定理となる.
またschema S4の適用により
\begin{multline*}
\tag{27}
  ({\rm pr}_{2}(c)|z)(\exists y((({\rm pr}_{1}(c), y) \in a \wedge (y, z) \in b) \wedge c = ({\rm pr}_{1}(c), z))) \\
  \to \exists z(\exists y((({\rm pr}_{1}(c), y) \in a \wedge (y, z) \in b) \wedge c = ({\rm pr}_{1}(c), z)))
\end{multline*}
が成り立つ.
またThm \ref{thmexch}と推論法則 \ref{dedequiv}により
\begin{multline*}
  \exists z(\exists y((({\rm pr}_{1}(c), y) \in a \wedge (y, z) \in b) \wedge c = ({\rm pr}_{1}(c), z))) \\
  \to \exists y(\exists z((({\rm pr}_{1}(c), y) \in a \wedge (y, z) \in b) \wedge c = ({\rm pr}_{1}(c), z)))
\end{multline*}
が成り立つ.
ここで$x$が$y$とも$z$とも異なり, $a$, $b$, $c$のいずれの記号列の中にも
自由変数として現れないことから, 
代入法則 \ref{substfree}, \ref{substfund}, \ref{substwedge}, \ref{substpair}により, 
上記の記号列は
\begin{multline*}
  \exists z(\exists y((({\rm pr}_{1}(c), y) \in a \wedge (y, z) \in b) \wedge c = ({\rm pr}_{1}(c), z))) \\
  \to \exists y(\exists z(({\rm pr}_{1}(c)|x)(((x, y) \in a \wedge (y, z) \in b) \wedge c = (x, z))))
\end{multline*}
と一致する.
また$y$と$z$が共に$x$と異なり, 上述のように${\rm pr}_{1}(c)$の中に自由変数として現れないことから, 
代入法則 \ref{substquan}により, この記号列は
\begin{multline*}
\tag{28}
  \exists z(\exists y((({\rm pr}_{1}(c), y) \in a \wedge (y, z) \in b) \wedge c = ({\rm pr}_{1}(c), z))) \\
  \to ({\rm pr}_{1}(c)|x)(\exists y(\exists z(((x, y) \in a \wedge (y, z) \in b) \wedge c = (x, z))))
\end{multline*}
と一致する.
よってこれが定理となる.
またschema S4の適用により
\begin{multline*}
  ({\rm pr}_{1}(c)|x)(\exists y(\exists z(((x, y) \in a \wedge (y, z) \in b) \wedge c = (x, z)))) \\
  \to \exists x(\exists y(\exists z(((x, y) \in a \wedge (y, z) \in b) \wedge c = (x, z))))
\end{multline*}
が成り立つが, $R$の定義からこの記号列は
\[
\tag{29}
  ({\rm pr}_{1}(c)|x)(\exists y(\exists z(((x, y) \in a \wedge (y, z) \in b) \wedge c = (x, z)))) \to R
\]
であるから, これが定理となる.
そこで(24)---(29)から, 推論法則 \ref{dedmmp}によって
\[
\tag{30}
  {\rm Pair}(c) \wedge \exists y(({\rm pr}_{1}(c), y) \in a \wedge (y, {\rm pr}_{2}(c)) \in b) \to R
\]
が成り立つことがわかる.
そこで(23), (30)から, 推論法則 \ref{dedequiv}によって
\[
\tag{31}
  R \leftrightarrow {\rm Pair}(c) \wedge \exists y(({\rm pr}_{1}(c), y) \in a \wedge (y, {\rm pr}_{2}(c)) \in b)
\]
が成り立つ.
そして(1), (31)から, 推論法則 \ref{dedeqtrans}によって
\[
  c \in b \circ a \leftrightarrow {\rm Pair}(c) \wedge \exists y(({\rm pr}_{1}(c), y) \in a \wedge (y, {\rm pr}_{2}(c)) \in b)
\]
が成り立つ.
\halmos




\mathstrut
\begin{thm}
\label{sthmpairincompeq}%定理
$a$, $b$, $T$, $U$を集合とし, $y$をこれらの中に自由変数として現れない文字とする.
このとき
\[
  (T, U) \in b \circ a \leftrightarrow \exists y((T, y) \in a \wedge (y, U) \in b)
\]
が成り立つ.
\end{thm}


\noindent{\bf 証明}
~仮定より$y$は$T$及び$U$の中に自由変数として現れないから, 
変数法則 \ref{valpair}により, $y$は$(T, U)$の中にも自由変数として現れない.
また仮定より$y$は$a$及び$b$の中にも自由変数として現れない.
そこで定理 \ref{sthmcompelement}より
\[
\tag{1}
  (T, U) \in b \circ a \leftrightarrow 
  {\rm Pair}((T, U)) \wedge \exists y(({\rm pr}_{1}((T, U)), y) \in a \wedge (y, {\rm pr}_{2}((T, U))) \in b)
\]
が成り立つ.
また定理 \ref{sthmbigpairpair}より${\rm Pair}((T, U))$が成り立つから, 
推論法則 \ref{dedawblatrue2}により
\begin{multline*}
\tag{2}
  {\rm Pair}((T, U)) \wedge \exists y(({\rm pr}_{1}((T, U)), y) \in a \wedge (y, {\rm pr}_{2}((T, U))) \in b) \\
  \leftrightarrow \exists y(({\rm pr}_{1}((T, U)), y) \in a \wedge (y, {\rm pr}_{2}((T, U))) \in b)
\end{multline*}
が成り立つ.
また定理 \ref{sthmprpair}より
\[
  {\rm pr}_{1}((T, U)) = T, ~~
  {\rm pr}_{2}((T, U)) = U
\]
が共に成り立つから, いま$v$を$y$と異なり, 
$a$, $b$, $T$, $U$のいずれの記号列の中にも自由変数として現れない, 
定数でない文字とすれば, 定理 \ref{sthmpairweak}により
\[
  ({\rm pr}_{1}((T, U)), v) = (T, v), ~~
  (v, {\rm pr}_{2}((T, U))) = (v, U)
\]
が共に成り立つ.
そこで定理 \ref{sthm=tineq}により
\[
  ({\rm pr}_{1}((T, U)), v) \in a \leftrightarrow (T, v) \in a, ~~
  (v, {\rm pr}_{2}((T, U))) \in b \leftrightarrow (v, U) \in b
\]
が共に成り立ち, これらから, 推論法則 \ref{dedaddeqw}により
\[
  ({\rm pr}_{1}((T, U)), v) \in a \wedge (v, {\rm pr}_{2}((T, U))) \in b 
  \leftrightarrow (T, v) \in a \wedge (v, U) \in b
\]
が成り立つ.
ここで$y$が$a$, $b$, $T$, $U$のいずれの記号列の中にも自由変数として現れないことから, 
代入法則 \ref{substfree}, \ref{substfund}, \ref{substwedge}, \ref{substpair}, \ref{substpr}により, 
上記の記号列は
\[
  (v|y)(({\rm pr}_{1}((T, U)), y) \in a \wedge (y, {\rm pr}_{2}((T, U))) \in b) 
  \leftrightarrow (v|y)((T, y) \in a \wedge (y, U) \in b)
\]
と一致する.
よってこれが定理となる.
そこで$v$が定数でないことから, 推論法則 \ref{dedalleqquansepconst}により
\[
  \exists v((v|y)(({\rm pr}_{1}((T, U)), y) \in a \wedge (y, {\rm pr}_{2}((T, U))) \in b)) 
  \leftrightarrow \exists v((v|y)((T, y) \in a \wedge (y, U) \in b))
\]
が成り立つ.
ここで$v$が$y$と異なり, $a$, $b$, $T$, $U$のいずれの記号列の中にも
自由変数として現れないことから, 
変数法則 \ref{valfund}, \ref{valwedge}, \ref{valpair}, \ref{valpr}によってわかるように, 
$v$は$({\rm pr}_{1}((T, U)), y) \in a \wedge (y, {\rm pr}_{2}((T, U))) \in b$及び
$(T, y) \in a \wedge (y, U) \in b$の中に自由変数として現れない.
そこで代入法則 \ref{substquantrans}により, 上記の記号列は
\[
\tag{3}
  \exists y(({\rm pr}_{1}((T, U)), y) \in a \wedge (y, {\rm pr}_{2}((T, U))) \in b) 
  \leftrightarrow \exists y((T, y) \in a \wedge (y, U) \in b)
\]
と一致する.
よってこれが定理となる.
そこで(1), (2), (3)から, 推論法則 \ref{dedeqtrans}によって
\[
  (T, U) \in b \circ a \leftrightarrow \exists y((T, y) \in a \wedge (y, U) \in b)
\]
が成り立つことがわかる.
\halmos




\mathstrut
\begin{thm}
\label{sthmpairincompt}%定理
$a$, $b$, $T$, $U$, $V$を集合とするとき, 
\[
  (T, U) \in a \wedge (U, V) \in b \to (T, V) \in b \circ a
\]
が成り立つ.
またこのことから, 次の($*$)が成り立つ: 

($*$) ~~$(T, U) \in a$と$(U, V) \in b$が共に成り立つならば, 
        $(T, V) \in b \circ a$が成り立つ.
\end{thm}


\noindent{\bf 証明}
~$y$を$a$, $b$, $T$, $V$のいずれの記号列の中にも自由変数として現れない文字とする.
このときschema S4の適用により
\[
  (U|y)((T, y) \in a \wedge (y, V) \in b) \to \exists y((T, y) \in a \wedge (y, V) \in b)
\]
が成り立つが, 代入法則 \ref{substfree}, \ref{substfund}, \ref{substwedge}, \ref{substpair}によれば
この記号列は
\[
\tag{1}
  (T, U) \in a \wedge (U, V) \in b \to \exists y((T, y) \in a \wedge (y, V) \in b)
\]
と一致するから, これが定理となる.
また$y$に対する仮定から, 
定理 \ref{sthmpairincompeq}と推論法則 \ref{dedequiv}により
\[
\tag{2}
  \exists y((T, y) \in a \wedge (y, V) \in b) \to (T, V) \in b \circ a
\]
が成り立つ.
そこで(1), (2)から, 推論法則 \ref{dedmmp}によって
\[
\tag{3}
  (T, U) \in a \wedge (U, V) \in b \to (T, V) \in b \circ a
\]
が成り立つ.

いま$(T, U) \in a$と$(U, V) \in b$が共に成り立つとすれば, 
推論法則 \ref{dedwedge}により$(T, U) \in a \wedge (U, V) \in b$が成り立つから, 
これと(3)から推論法則 \ref{dedmp}によって$(T, V) \in b \circ a$が成り立つ.
故に($*$)が成り立つ.
\halmos




\mathstrut
\begin{thm}
\label{sthmcompgraph}%定理
$a$と$b$を集合とするとき, $b \circ a$はグラフである.
また
\[
  b \circ a \subset {\rm pr}_{1}\langle a \rangle \times {\rm pr}_{2}\langle b \rangle
\]
が成り立つ.
\end{thm}


\noindent{\bf 証明}
~$x$, $y$, $z$, $w$を, どの二つも互いに異なり, いずれも$a$及び$b$の中に自由変数として現れない文字とする.
また関係式$\exists x(\exists y(\exists z(((x, y) \in a \wedge (y, z) \in b) \wedge w = (x, z))))$を$R$と書く.
このとき定義から$b \circ a$は$\{w|R\}$と同じである.
また定理 \ref{sthmcompsetmake}の証明の中で示したように, 
$w$は${\rm pr}_{1}\langle a \rangle \times {\rm pr}_{2}\langle b \rangle$の中に自由変数として現れず, 
$\forall w(R \to w \in {\rm pr}_{1}\langle a \rangle \times {\rm pr}_{2}\langle b \rangle)$が
成り立つ(定理 \ref{sthmcompsetmake}の証明中の(19)).
そこで定理 \ref{sthmalltiset=sset}により, 
$\{w|R\} \subset {\rm pr}_{1}\langle a \rangle \times {\rm pr}_{2}\langle b \rangle$, 
即ち
$b \circ a \subset {\rm pr}_{1}\langle a \rangle \times {\rm pr}_{2}\langle b \rangle$が
成り立つ.
またこのことから, 定理 \ref{sthmproductsubsetgraph}によってわかるように, $b \circ a$はグラフである.
\halmos




\mathstrut
\begin{thm}
\label{sthmcompcomb}%定理
$a$, $b$, $c$を集合とするとき, 
\[
  c \circ (b \circ a) = (c \circ b) \circ a
\]
が成り立つ.
\end{thm}


\noindent{\bf 証明}
~$x$, $y$, $z$, $w$を, どの二つも互いに異なり, どの一つも
$a$, $b$, $c$のいずれの記号列の中にも自由変数として現れない, 定数でない文字とする.
このとき変数法則 \ref{valcomp}により, $z$は$b \circ a$の中にも自由変数として現れないから, 
定理 \ref{sthmpairincompeq}より
\[
\tag{1}
  (x, w) \in c \circ (b \circ a) \leftrightarrow \exists z((x, z) \in b \circ a \wedge (z, w) \in c)
\]
が成り立つ.
また$y$が$x$とも$z$とも異なり, $a$及び$b$の中に自由変数として現れないことから, 
同じく定理 \ref{sthmpairincompeq}より
\[
  (x, z) \in b \circ a \leftrightarrow \exists y((x, y) \in a \wedge (y, z) \in b)
\]
が成り立つ.
そこで推論法則 \ref{dedaddeqw}により
\[
\tag{2}
  (x, z) \in b \circ a \wedge (z, w) \in c \leftrightarrow \exists y((x, y) \in a \wedge (y, z) \in b) \wedge (z, w) \in c
\]
が成り立つ.
また$y$が$z$とも$w$とも異なり, $c$の中に自由変数として現れないことから, 
変数法則 \ref{valfund}, \ref{valpair}により$y$は$(z, w) \in c$の中にも自由変数として現れないから, 
Thm \ref{thmexwrfree}と推論法則 \ref{dedeqch}により
\[
\tag{3}
  \exists y((x, y) \in a \wedge (y, z) \in b) \wedge (z, w) \in c 
  \leftrightarrow \exists y(((x, y) \in a \wedge (y, z) \in b) \wedge (z, w) \in c)
\]
が成り立つ.
またThm \ref{1awb1wclaw1bwc1}より
\[
  ((x, y) \in a \wedge (y, z) \in b) \wedge (z, w) \in c 
  \leftrightarrow (x, y) \in a \wedge ((y, z) \in b \wedge (z, w) \in c)
\]
が成り立つから, $y$が定数でないことから, 推論法則 \ref{dedalleqquansepconst}により
\[
\tag{4}
  \exists y(((x, y) \in a \wedge (y, z) \in b) \wedge (z, w) \in c) 
  \leftrightarrow \exists y((x, y) \in a \wedge ((y, z) \in b \wedge (z, w) \in c))
\]
が成り立つ.
そこで(2), (3), (4)から, 推論法則 \ref{dedeqtrans}によって
\[
  (x, z) \in b \circ a \wedge (z, w) \in c
  \leftrightarrow \exists y((x, y) \in a \wedge ((y, z) \in b \wedge (z, w) \in c))
\]
が成り立つことがわかる.
いま$z$も定数でないから, これから推論法則 \ref{dedalleqquansepconst}により
\[
\tag{5}
  \exists z((x, z) \in b \circ a \wedge (z, w) \in c)
  \leftrightarrow \exists z(\exists y((x, y) \in a \wedge ((y, z) \in b \wedge (z, w) \in c)))
\]
が成り立つ.
またThm \ref{thmexch}より
\[
\tag{6}
  \exists z(\exists y((x, y) \in a \wedge((y, z) \in b \wedge (z, w) \in c)))
  \leftrightarrow \exists y(\exists z((x, y) \in a \wedge ((y, z) \in b \wedge (z, w) \in c)))
\]
が成り立つ.
また$z$が$x$とも$y$とも異なり, $a$の中に自由変数として現れないことから, 
変数法則 \ref{valfund}, \ref{valpair}により$z$は$(x, y) \in a$の中にも
自由変数として現れないから, Thm \ref{thmexwrfree}より
\[
\tag{7}
  \exists z((x, y) \in a \wedge ((y, z) \in b \wedge (z, w) \in c)) 
  \leftrightarrow (x, y) \in a \wedge \exists z((y, z) \in b \wedge (z, w) \in c)
\]
が成り立つ.
また$z$が$y$とも$w$とも異なり, $b$及び$c$の中に自由変数として現れないことから, 
定理 \ref{sthmpairincompeq}と推論法則 \ref{dedeqch}により
\[
  \exists z((y, z) \in b \wedge (z, w) \in c) 
  \leftrightarrow (y, w) \in c \circ b
\]
が成り立つ.
そこで推論法則 \ref{dedaddeqw}により
\[
\tag{8}
  (x, y) \in a \wedge \exists z((y, z) \in b \wedge (z, w) \in c) 
  \leftrightarrow (x, y) \in a \wedge (y, w) \in c \circ b
\]
が成り立つ.
そこで(7), (8)から, 推論法則 \ref{dedeqtrans}によって
\[
  \exists z((x, y) \in a \wedge ((y, z) \in b \wedge (z, w) \in c)) 
  \leftrightarrow (x, y) \in a \wedge (y, w) \in c \circ b
\]
が成り立ち, $y$が定数でないことから, 推論法則 \ref{dedalleqquansepconst}によって
\[
\tag{9}
  \exists y(\exists z((x, y) \in a \wedge ((y, z) \in b \wedge (z, w) \in c))) 
  \leftrightarrow \exists y((x, y) \in a \wedge (y, w) \in c \circ b)
\]
が成り立つ.
また$y$は$b$及び$c$の中に自由変数として現れないから, 
変数法則 \ref{valcomp}により, $y$は$c \circ b$の中に自由変数として現れない.
このことと, $y$が$x$とも$w$とも異なり, $a$の中にも自由変数として現れないことから, 
定理 \ref{sthmpairincompeq}と推論法則 \ref{dedeqch}により
\[
\tag{10}
  \exists y((x, y) \in a \wedge (y, w) \in c \circ b) 
  \leftrightarrow (x, w) \in (c \circ b) \circ a
\]
が成り立つ.
そこで(1), (5), (6), (9), (10)から, 推論法則 \ref{dedeqtrans}によって
\[
\tag{11}
  (x, w) \in c \circ (b \circ a) \leftrightarrow (x, w) \in (c \circ b) \circ a
\]
が成り立つことがわかる.
いま定理 \ref{sthmcompgraph}より, $c \circ (b \circ a)$と$(c \circ b) \circ a$は共にグラフである.
また$x$と$w$は共に$a$, $b$, $c$のいずれの記号列の中にも自由変数として現れないから, 
変数法則 \ref{valcomp}により, これらは共に$c \circ (b \circ a)$及び$(c \circ b) \circ a$の中に
自由変数として現れない.
また$x$と$w$は互いに異なり, 共に定数でない.
そこでこれらのことと, (11)が成り立つことから, 
定理 \ref{sthmgraphpair=}により
$c \circ (b \circ a) = (c \circ b) \circ a$が成り立つ.
\halmos




\mathstrut
\begin{thm}
\label{sthmcompsubset}%定理
\mbox{}

1)
$a$, $b$, $c$を集合とするとき, 
\[
  a \subset b \to c \circ a \subset c \circ b, ~~
  a \subset b \to a \circ c \subset b \circ c
\]
が成り立つ.
またこのことから, 次の($*$)が成り立つ: 

($*$) ~~$a \subset b$が成り立つならば, 
        $c \circ a \subset c \circ b$と$a \circ c \subset b \circ c$が共に成り立つ.

2)
$a$, $b$, $c$, $d$を集合とするとき, 
\[
  a \subset c \wedge b \subset d \to b \circ a \subset d \circ c
\]
が成り立つ.
またこのことから, 次の($**$)が成り立つ: 

($**$) ~~$a \subset c$と$b \subset d$が共に成り立つならば, 
         $b \circ a \subset d \circ c$が成り立つ.
\end{thm}


\noindent{\bf 証明}
~1)
$x$, $y$, $z$を, どの二つも互いに異なり, どの一つも$a$, $b$, $c$のいずれの記号列の中にも
自由変数として現れない, 定数でない文字とする.
このとき定理 \ref{sthmsubsetbasis}より
\begin{align*}
  a \subset b &\to ((x, y) \in a \to (x, y) \in b), \\
  \mbox{} \\
  a \subset b &\to ((y, z) \in a \to (y, z) \in b)
\end{align*}
が共に成り立つ.
いま$y$は定数でなく, 変数法則 \ref{valsubset}により$a \subset b$の中に自由変数として現れないから, 
このことと上記の二つの定理から, 推論法則 \ref{dedalltquansepfreeconst}により
\begin{align*}
  \tag{1}
  a \subset b &\to \forall y((x, y) \in a \to (x, y) \in b), \\
  \mbox{} \\
  \tag{2}
  a \subset b &\to \forall y((y, z) \in a \to (y, z) \in b)
\end{align*}
が共に成り立つ.
またThm \ref{1atb1t1awctbwc1}より
\begin{align*}
  ((x, y) \in a \to (x, y) \in b) &\to 
  ((x, y) \in a \wedge (y, z) \in c \to (x, y) \in b \wedge (y, z) \in c), \\
  \mbox{} \\
  ((y, z) \in a \to (y, z) \in b) &\to 
  ((x, y) \in c \wedge (y, z) \in a \to (x, y) \in c \wedge (y, z) \in b)
\end{align*}
が共に成り立つ.
$y$は定数でないから, これらから推論法則 \ref{dedalltquansepconst}によって
\begin{align*}
  \tag{3}
  \forall y((x, y) \in a \to (x, y) \in b) &\to 
  \forall y((x, y) \in a \wedge (y, z) \in c \to (x, y) \in b \wedge (y, z) \in c), \\
  \mbox{} \\
  \tag{4}
  \forall y((y, z) \in a \to (y, z) \in b) &\to 
  \forall y((x, y) \in c \wedge (y, z) \in a \to (x, y) \in c \wedge (y, z) \in b)
\end{align*}
が共に成り立つ.
またThm \ref{thmalltexsep}より
\begin{multline*}
\tag{5}
  \forall y((x, y) \in a \wedge (y, z) \in c \to (x, y) \in b \wedge (y, z) \in c) \\
  \to (\exists y((x, y) \in a \wedge (y, z) \in c) \to \exists y((x, y) \in b \wedge (y, z) \in c)), 
\end{multline*}
\begin{multline*}
\tag{6}
  \forall y((x, y) \in c \wedge (y, z) \in a \to (x, y) \in c \wedge (y, z) \in b) \\
  \to (\exists y((x, y) \in c \wedge (y, z) \in a) \to \exists y((x, y) \in c \wedge (y, z) \in b))
\end{multline*}
が共に成り立つ.
また$y$は$x$とも$z$とも異なり, $a$, $b$, $c$のいずれの記号列の中にも自由変数として現れないから, 
定理 \ref{sthmpairincompeq}と推論法則 \ref{dedequiv}により
\begin{align*}
  (x, z) \in c \circ a \to \exists y((x, y) \in a \wedge (y, z) \in c)&, ~~
  (x, z) \in a \circ c \to \exists y((x, y) \in c \wedge (y, z) \in a), \\
  \mbox{} \\
  \exists y((x, y) \in b \wedge (y, z) \in c) \to (x, z) \in c \circ b&, ~~
  \exists y((x, y) \in c \wedge (y, z) \in b) \to (x, z) \in b \circ c
\end{align*}
がすべて成り立つ.
そこでこのはじめの二つから, 推論法則 \ref{dedaddf}によって
\begin{multline*}
\tag{7}
  (\exists y((x, y) \in a \wedge (y, z) \in c) \to \exists y((x, y) \in b \wedge (y, z) \in c)) \\
  \to ((x, z) \in c \circ a \to \exists y((x, y) \in b \wedge (y, z) \in c)), 
\end{multline*}
\begin{multline*}
\tag{8}
  (\exists y((x, y) \in c \wedge (y, z) \in a) \to \exists y((x, y) \in c \wedge (y, z) \in b)) \\
  \to ((x, z) \in a \circ c \to \exists y((x, y) \in c \wedge (y, z) \in b))
\end{multline*}
がそれぞれ成り立ち, 後の二つから, 推論法則 \ref{dedaddb}によって
\begin{align*}
  \tag{9}
  ((x, z) \in c \circ a \to \exists y((x, y) \in b \wedge (y, z) \in c)) 
  &\to ((x, z) \in c \circ a \to (x, z) \in c \circ b), \\
  \mbox{} \\
  \tag{10}
  ((x, z) \in a \circ c \to \exists y((x, y) \in c \wedge (y, z) \in b))
  &\to ((x, z) \in a \circ c \to (x, z) \in b \circ c)
\end{align*}
がそれぞれ成り立つ.
そこで(1), (3), (5), (7), (9)から, 推論法則 \ref{dedmmp}によって
\[
\tag{11}
  a \subset b \to ((x, z) \in c \circ a \to (x, z) \in c \circ b)
\]
が成り立ち, (2), (4), (6), (8), (10)から, 同じく推論法則 \ref{dedmmp}によって
\[
\tag{12}
  a \subset b \to ((x, z) \in a \circ c \to (x, z) \in b \circ c)
\]
が成り立つことがわかる.
さていま$x$と$z$は共に$a$及び$b$の中に自由変数として現れないから, 
変数法則 \ref{valsubset}により, これらは共に$a \subset b$の中に自由変数として現れない.
また$x$と$z$は共に定数でない.
そこでこれらのことと, (11), (12)が共に成り立つことから, 推論法則 \ref{dedalltquansepfreeconst}によって
\begin{align*}
  \tag{13}
  a \subset b &\to \forall x(\forall z((x, z) \in c \circ a \to (x, z) \in c \circ b)), \\
  \mbox{} \\
  \tag{14}
  a \subset b &\to \forall x(\forall z((x, z) \in a \circ c \to (x, z) \in b \circ c))
\end{align*}
が共に成り立つことがわかる.
またいま定理 \ref{sthmcompgraph}より, $c \circ a$と$a \circ c$は共にグラフである.
また$x$と$z$は共に$a$, $b$, $c$のいずれの記号列の中にも自由変数として現れないから, 
変数法則 \ref{valcomp}により, これらは共に
$c \circ a$, $c \circ b$, $a \circ c$, $b \circ c$のいずれの記号列の中にも自由変数として現れない.
また$x$と$z$は互いに異なる.
そこでこれらのことから, 定理 \ref{sthmgraphpairsubset}により
\begin{align*}
  c \circ a \subset c \circ b &\leftrightarrow \forall x(\forall z((x, z) \in c \circ a \to (x, z) \in c \circ b)), \\
  \mbox{} \\
  a \circ c \subset b \circ c &\leftrightarrow \forall x(\forall z((x, z) \in a \circ c \to (x, z) \in b \circ c))
\end{align*}
が共に成り立つことがわかり, これらから特に推論法則 \ref{dedequiv}によって
\begin{align*}
  \tag{15}
  \forall x(\forall z((x, z) \in c \circ a \to (x, z) \in c \circ b)) &\to c \circ a \subset c \circ b, \\
  \mbox{} \\
  \tag{16}
  \forall x(\forall z((x, z) \in a \circ c \to (x, z) \in b \circ c)) &\to a \circ c \subset b \circ c
\end{align*}
が共に成り立つ.
そこで(13)と(15), (14)と(16)から, それぞれ推論法則 \ref{dedmmp}によって
\[
  a \subset b \to c \circ a \subset c \circ b, ~~
  a \subset b \to a \circ c \subset b \circ c
\]
が成り立つ.
($*$)が成り立つことは, これらと推論法則 \ref{dedmp}によって明らかである.

\noindent
2)
1)で示したことから, 
\[
  a \subset c \to b \circ a \subset b \circ c, ~~
  b \subset d \to b \circ c \subset d \circ c
\]
が共に成り立つから, 推論法則 \ref{dedfromaddw}により
\[
\tag{17}
  a \subset c \wedge b \subset d \to b \circ a \subset b \circ c \wedge b \circ c \subset d \circ c
\]
が成り立つ.
また定理 \ref{sthmsubsettrans}より
\[
\tag{18}
  b \circ a \subset b \circ c \wedge b \circ c \subset d \circ c \to b \circ a \subset d \circ c
\]
が成り立つ.
そこで(17), (18)から, 推論法則 \ref{dedmmp}によって
\[
\tag{19}
  a \subset c \wedge b \subset d \to b \circ a \subset d \circ c
\]
が成り立つ.

いま$a \subset c$と$b \subset d$が共に成り立つとすれば, 推論法則 \ref{dedwedge}により
$a \subset c \wedge b \subset d$が成り立つから, これと(19)から, 
推論法則 \ref{dedmp}によって$b \circ a \subset d \circ c$が成り立つ.
これで($**$)が成り立つことも示された.
\halmos




\mathstrut
\begin{thm}
\label{sthmcomp=}%定理
\mbox{}

1)
$a$, $b$, $c$を集合とするとき, 
\[
  a = b \to c \circ a = c \circ b, ~~
  a = b \to a \circ c = b \circ c
\]
が成り立つ.
またこのことから, 次の($*$)が成り立つ: 

($*$) ~~$a = b$が成り立つならば, 
        $c \circ a = c \circ b$と$a \circ c = b \circ c$が共に成り立つ.

2)
$a$, $b$, $c$, $d$を集合とするとき, 
\[
  a = c \wedge b = d \to b \circ a = d \circ c
\]
が成り立つ.
またこのことから, 次の($**$)が成り立つ: 

($**$) ~~$a = c$と$b = d$が共に成り立つならば, 
         $b \circ a = d \circ c$が成り立つ.
\end{thm}


\noindent{\bf 証明}
~1)
$x$を$c$の中に自由変数として現れない文字とするとき, 
Thm \ref{T=Ut1TV=UV1}より
\[
  a = b \to (a|x)(c \circ x) = (b|x)(c \circ x), ~~
  a = b \to (a|x)(x \circ c) = (b|x)(x \circ c)
\]
が共に成り立つが, 代入法則 \ref{substfree}, \ref{substcomp}によれば, 
これらの記号列はそれぞれ
\[
  a = b \to c \circ a = c \circ b, ~~
  a = b \to a \circ c = b \circ c
\]
と一致するから, これらが共に定理となる.
($*$)が成り立つことは, これらと推論法則 \ref{dedmp}によって明らかである.

\noindent
2)
1)で示したことより
\[
  a = c \to b \circ a = b \circ c, ~~
  b = d \to b \circ c = d \circ c
\]
が共に成り立つから, 推論法則 \ref{dedfromaddw}により
\[
\tag{1}
  a = c \wedge b = d \to b \circ a = b \circ c \wedge b \circ c = d \circ c
\]
が成り立つ.
またThm \ref{x=ywy=ztx=z}より
\[
\tag{2}
  b \circ a = b \circ c \wedge b \circ c = d \circ c \to b \circ a = d \circ c
\]
が成り立つ.
そこで(1), (2)から, 推論法則 \ref{dedmmp}によって
\[
\tag{3}
  a = c \wedge b = d \to b \circ a = d \circ c
\]
が成り立つ.

いま$a = c$と$b = d$が共に成り立つとすれば, 推論法則 \ref{dedwedge}により
$a = c \wedge b = d$が成り立つから, これと(3)から, 
推論法則 \ref{dedmp}によって$b \circ a = d \circ c$が成り立つ.
これで($**$)が成り立つことも示された.
\halmos




\mathstrut
\begin{thm}
\label{sthmsingletoncomp}%定理
$a$, $b$, $c$を集合とするとき, 
\[
  c \circ \{(a, b)\} = \{a\} \times c[\{b\}], ~~
  \{(a, b)\} \circ c = c^{-1}[\{a\}] \times \{b\}
\]
が成り立つ.
\end{thm}


\noindent{\bf 証明}
~$x$, $y$, $z$を, どの二つも互いに異なり, どの一つも$a$, $b$, $c$のいずれの記号列の中にも
自由変数として現れない, 定数でない文字とする.
このとき変数法則 \ref{valnset}, \ref{valpair}により, 
$y$は$\{(a, b)\}$の中にも自由変数として現れないから, 
定理 \ref{sthmpairincompeq}より
\begin{align*}
  \tag{1}
  (x, z) \in c \circ \{(a, b)\} &\leftrightarrow \exists y((x, y) \in \{(a, b)\} \wedge (y, z) \in c), \\
  \mbox{} \\
  \tag{2}
  (x, z) \in \{(a, b)\} \circ c &\leftrightarrow \exists y((x, y) \in c \wedge (y, z) \in \{(a, b)\})
\end{align*}
が共に成り立つ.
また定理 \ref{sthmsingletonbasis}より
\[
  (x, y) \in \{(a, b)\} \leftrightarrow (x, y) = (a, b), ~~
  (y, z) \in \{(a, b)\} \leftrightarrow (y, z) = (a, b)
\]
が共に成り立ち, 
定理 \ref{sthmpair}より
\[
  (x, y) = (a, b) \leftrightarrow x = a \wedge y = b, ~~
  (y, z) = (a, b) \leftrightarrow y = a \wedge z = b
\]
が共に成り立つから, これらから, 推論法則 \ref{dedeqtrans}によって
\[
  (x, y) \in \{(a, b)\} \leftrightarrow x = a \wedge y = b, ~~
  (y, z) \in \{(a, b)\} \leftrightarrow y = a \wedge z = b
\]
が共に成り立つ.
そこでこれらから, 推論法則 \ref{dedaddeqw}により
\begin{align*}
  \tag{3}
  (x, y) \in \{(a, b)\} \wedge (y, z) \in c &\leftrightarrow (x = a \wedge y = b) \wedge (y, z) \in c, \\
  \mbox{} \\
  \tag{4}
  (x, y) \in c \wedge (y, z) \in \{(a, b)\} &\leftrightarrow (x, y) \in c \wedge (y = a \wedge z = b)
\end{align*}
が共に成り立つ.
またThm \ref{1awb1wclaw1bwc1}より
\[
\tag{5}
  (x = a \wedge y = b) \wedge (y, z) \in c \leftrightarrow x = a \wedge (y = b \wedge (y, z) \in c)
\]
が成り立ち, Thm \ref{1awb1wclaw1bwc1}と推論法則 \ref{dedeqch}により
\[
\tag{6}
  (x, y) \in c \wedge (y = a \wedge z = b) \leftrightarrow ((x, y) \in c \wedge y = a) \wedge z = b
\]
が成り立つ.
そこで(3)と(5), (4)と(6)から, それぞれ推論法則 \ref{dedeqtrans}によって
\begin{align*}
  (x, y) \in \{(a, b)\} \wedge (y, z) \in c &\leftrightarrow x = a \wedge (y = b \wedge (y, z) \in c), \\
  \mbox{} \\
  (x, y) \in c \wedge (y, z) \in \{(a, b)\} &\leftrightarrow ((x, y) \in c \wedge y = a) \wedge z = b
\end{align*}
が成り立つ.
いま$y$は定数でないので, これらから推論法則 \ref{dedalleqquansepconst}によって
\begin{align*}
  \tag{7}
  \exists y((x, y) \in \{(a, b)\} \wedge (y, z) \in c) &\leftrightarrow \exists y(x = a \wedge (y = b \wedge (y, z) \in c)), \\
  \mbox{} \\
  \tag{8}
  \exists y((x, y) \in c \wedge (y, z) \in \{(a, b)\}) &\leftrightarrow \exists y(((x, y) \in c \wedge y = a) \wedge z = b)
\end{align*}
が共に成り立つ.
また$y$が$x$とも$z$とも異なり, $a$及び$b$の中に自由変数として現れないことから, 
変数法則 \ref{valfund}により$y$が$x = a$及び$z = b$の中に自由変数として現れないことが
わかるから, Thm \ref{thmexwrfree}より
\begin{align*}
  \tag{9}
  \exists y(x = a \wedge (y = b \wedge (y, z) \in c)) &\leftrightarrow x = a \wedge \exists y(y = b \wedge (y, z) \in c), \\
  \mbox{} \\
  \tag{10}
  \exists y(((x, y) \in c \wedge y = a) \wedge z = b) &\leftrightarrow \exists y((x, y) \in c \wedge y = a) \wedge z = b
\end{align*}
が共に成り立つ.
またThm \ref{awbtbwa}より
\[
\tag{11}
  (x, y) \in c \wedge y = a \to y = a \wedge (x, y) \in c
\]
が成り立つ.
また定理 \ref{sthmpairweak}と推論法則 \ref{dedequiv}により
\[
  y = b \to (y, z) = (b, z), ~~
  y = a \to (x, y) = (x, a)
\]
が共に成り立つから, 推論法則 \ref{dedaddw}により
\begin{align*}
  \tag{12}
  y = b \wedge (y, z) \in c &\to (y, z) = (b, z) \wedge (y, z) \in c, \\
  \mbox{} \\
  \tag{13}
  y = a \wedge (x, y) \in c &\to (x, y) = (x, a) \wedge (x, y) \in c
\end{align*}
が共に成り立つ.
また定理 \ref{sthm=&in}より
\begin{align*}
  \tag{14}
  (y, z) = (b, z) \wedge (y, z) \in c &\to (b, z) \in c, \\
  \mbox{} \\
  \tag{15}
  (x, y) = (x, a) \wedge (x, y) \in c &\to (x, a) \in c
\end{align*}
が共に成り立つ.
そこで(12), (14)から, 推論法則 \ref{dedmmp}によって
\[
\tag{16}
  y = b \wedge (y, z) \in c \to (b, z) \in c
\]
が成り立ち, (11), (13), (15)から, 同じく推論法則 \ref{dedmmp}によって
\[
\tag{17}
  (x, y) \in c \wedge y = a \to (x, a) \in c
\]
が成り立つことがわかる.
いま$y$は$x$とも$z$とも異なり, $a$, $b$, $c$のいずれの記号列の中にも
自由変数として現れないから, 変数法則 \ref{valfund}, \ref{valpair}により, 
$y$は$(b, z) \in c$及び$(x, a) \in c$の中に自由変数として現れない.
また$y$は定数でない.
そこでこれらのことと, (16), (17)が成り立つことから, 推論法則 \ref{dedalltquansepfreeconst}により
\begin{align*}
  \tag{18}
  \exists y(y = b \wedge (y, z) \in c) &\to (b, z) \in c, \\
  \mbox{} \\
  \tag{19}
  \exists y((x, y) \in c \wedge y = a) &\to (x, a) \in c
\end{align*}
が共に成り立つ.
またいまThm \ref{x=x}より
\[
  b = b, ~~
  a = a
\]
が共に成り立つから, 推論法則 \ref{dedatawbtrue2}により
\[
  (b, z) \in c \to b = b \wedge (b, z) \in c, ~~
  (x, a) \in c \to (x, a) \in c \wedge a = a
\]
が共に成り立つ.
ここで$y$が$x$とも$z$とも異なり, $a$, $b$, $c$のいずれの記号列の中にも
自由変数として現れないことから, 
代入法則 \ref{substfree}, \ref{substfund}, \ref{substwedge}, \ref{substpair}により, 
上記の記号列はそれぞれ
\begin{align*}
  \tag{20}
  (b, z) \in c &\to (b|y)(y = b \wedge (y, z) \in c), \\
  \mbox{} \\
  \tag{21}
  (x, a) \in c &\to (a|y)((x, y) \in c \wedge y = a)
\end{align*}
と一致する.
よってこれらが共に定理となる.
またschema S4の適用により
\begin{align*}
  \tag{22}
  (b|y)(y = b \wedge (y, z) \in c) &\to \exists y(y = b \wedge (y, z) \in c), \\
  \mbox{} \\
  \tag{23}
  (a|y)((x, y) \in c \wedge y = a) &\to \exists y((x, y) \in c \wedge y = a)
\end{align*}
が共に成り立つ.
そこで(20)と(22), (21)と(23)から, それぞれ推論法則 \ref{dedmmp}によって
\begin{align*}
  \tag{24}
  (b, z) \in c &\to \exists y(y = b \wedge (y, z) \in c), \\
  \mbox{} \\
  \tag{25}
  (x, a) \in c &\to \exists y((x, y) \in c \wedge y = a)
\end{align*}
が成り立つ.
そこで(18)と(24), (19)と(25)から, それぞれ推論法則 \ref{dedequiv}によって
\begin{align*}
  \tag{26}
  \exists y(y = b \wedge (y, z) \in c) &\leftrightarrow (b, z) \in c, \\
  \mbox{} \\
  \tag{27}
  \exists y((x, y) \in c \wedge y = a) &\leftrightarrow (x, a) \in c
\end{align*}
が成り立つ.
また定理 \ref{sthmcutelement}と推論法則 \ref{dedeqch}により
\[
\tag{28}
  (b, z) \in c \leftrightarrow z \in c[\{b\}]
\]
が成り立ち, 定理 \ref{sthmcutinvelement}と推論法則 \ref{dedeqch}により
\[
\tag{29}
  (x, a) \in c \leftrightarrow x \in c^{-1}[\{a\}]
\]
が成り立つ.
そこで(26)と(28), (27)と(29)から, それぞれ推論法則 \ref{dedeqtrans}によって
\begin{align*}
\tag{30}
  \exists y(y = b \wedge (y, z) \in c) &\leftrightarrow z \in c[\{b\}], \\
  \mbox{} \\
  \tag{31}
  \exists y((x, y) \in c \wedge y = a) &\leftrightarrow x \in c^{-1}[\{a\}]
\end{align*}
が成り立つ.
また定理 \ref{sthmsingletonbasis}と推論法則 \ref{dedeqch}により
\begin{align*}
  \tag{32}
  x = a &\leftrightarrow x \in \{a\}, \\
  \mbox{} \\
  \tag{33}
  z = b &\leftrightarrow z \in \{b\}
\end{align*}
が共に成り立つ.
そこで(30)と(32), (31)と(33)から, それぞれ推論法則 \ref{dedaddeqw}によって
\begin{align*}
  \tag{34}
  x = a \wedge \exists y(y = b \wedge (y, z) \in c) &\leftrightarrow x \in \{a\} \wedge z \in c[\{b\}], \\
  \mbox{} \\
  \tag{35}
  \exists y((x, y) \in c \wedge y = a) \wedge z = b &\leftrightarrow x \in c^{-1}[\{a\}] \wedge z \in \{b\}
\end{align*}
が成り立つ.
また定理 \ref{sthmpairinproduct}と推論法則 \ref{dedeqch}により
\begin{align*}
  \tag{36}
  x \in \{a\} \wedge z \in c[\{b\}] &\leftrightarrow (x, z) \in \{a\} \times c[\{b\}], \\
  \mbox{} \\
  \tag{37}
  x \in c^{-1}[\{a\}] \wedge z \in \{b\} &\leftrightarrow (x, z) \in c^{-1}[\{a\}] \times \{b\}
\end{align*}
が共に成り立つ.
以上の(1), (7), (9), (34), (36)から, 推論法則 \ref{dedeqtrans}によって
\[
\tag{38}
  (x, z) \in c \circ \{(a, b)\} \leftrightarrow (x, z) \in \{a\} \times c[\{b\}]
\]
が成り立ち, (2), (8), (10), (35), (37)から, 同じく推論法則 \ref{dedeqtrans}によって
\[
\tag{39}
  (x, z) \in \{(a, b)\} \circ c \leftrightarrow (x, z) \in c^{-1}[\{a\}] \times \{b\}
\]
が成り立つことがわかる.
さていま定理 \ref{sthmproductgraph}, \ref{sthmcompgraph}からわかるように, 
$c \circ \{(a, b)\}$, $\{a\} \times c[\{b\}]$, $\{(a, b)\} \circ c$, $c^{-1}[\{a\}] \times \{b\}$は
いずれもグラフである.
また$x$と$z$は共に$a$, $b$, $c$のいずれの記号列の中にも自由変数として現れないから, 
変数法則 \ref{valnset}, \ref{valpair}, \ref{valproduct}, \ref{valvalueset}, \ref{valinv}, \ref{valcomp}によって
わかるように, これらは共に
$c \circ \{(a, b)\}$, $\{a\} \times c[\{b\}]$, $\{(a, b)\} \circ c$, $c^{-1}[\{a\}] \times \{b\}$の
いずれの記号列の中にも自由変数として現れない.
また$x$と$z$は互いに異なり, 共に定数でない.
以上のことと, (38), (39)が成り立つことから, 
定理 \ref{sthmgraphpair=}により
\[
  c \circ \{(a, b)\} = \{a\} \times c[\{b\}], ~~
  \{(a, b)\} \circ c = c^{-1}[\{a\}] \times \{b\}
\]
が成り立つ.
\halmos




\mathstrut
\begin{thm}
\label{sthmpairsetofacomp}%定理
$a$と$b$を集合とする.
また$x$, $y$を, それぞれ$a$, $b$の中に自由変数として現れない文字とする.
このとき
\[
  b \circ a = b \circ \{x \in a|{\rm Pair}(x)\}, ~~
  b \circ a = \{y \in b|{\rm Pair}(y)\} \circ a
\]
が成り立つ.
\end{thm}


\noindent{\bf 証明}
~$u$, $v$, $w$を, どの二つも互いに異なり, 
いずれも$x$とも$y$とも異なり, $a$及び$b$の中に自由変数として現れない, 
定数でない文字とする.
このとき定理 \ref{sthmpairincompeq}より
\[
\tag{1}
  (u, w) \in b \circ a \leftrightarrow \exists v((u, v) \in a \wedge (v, w) \in b)
\]
が成り立つ.
また$x$, $y$がそれぞれ$a$, $b$の中に自由変数として現れないことから, 
定理 \ref{sthmpairsetofa}より
\[
  (u, v) \in a \leftrightarrow (u, v) \in \{x \in a|{\rm Pair}(x)\}, ~~
  (v, w) \in b \leftrightarrow (v, w) \in \{y \in b|{\rm Pair}(y)\}
\]
が共に成り立つ.
そこでこれらから, 推論法則 \ref{dedaddeqw}によって
\begin{align*}
  (u, v) \in a \wedge (v, w) \in b &\leftrightarrow (u, v) \in \{x \in a|{\rm Pair}(x)\} \wedge (v, w) \in b, \\
  \mbox{} \\
  (u, v) \in a \wedge (v, w) \in b &\leftrightarrow (u, v) \in a \wedge (v, w) \in \{y \in b|{\rm Pair}(y)\}
\end{align*}
が共に成り立つ.
いま$v$は定数でないので, これらから推論法則 \ref{dedalleqquansepconst}によって
\begin{align*}
  \tag{2}
  \exists v((u, v) \in a \wedge (v, w) \in b) &\leftrightarrow \exists v((u, v) \in \{x \in a|{\rm Pair}(x)\} \wedge (v, w) \in b), \\
  \mbox{} \\
  \tag{3}
  \exists v((u, v) \in a \wedge (v, w) \in b) &\leftrightarrow \exists v((u, v) \in a \wedge (v, w) \in \{y \in b|{\rm Pair}(y)\})
\end{align*}
が共に成り立つ.
いま$v$は$x$とも$y$とも異なり, $a$及び$b$の中に自由変数として現れないから, 
変数法則 \ref{valsset}, \ref{valbigpair}により, 
$v$は$\{x \in a|{\rm Pair}(x)\}$及び$\{y \in b|{\rm Pair}(y)\}$の中にも自由変数として現れない.
このことと, $v$が$u$とも$w$とも異なることから, 定理 \ref{sthmpairincompeq}と推論法則 \ref{dedeqch}により
\begin{align*}
  \tag{4}
  \exists v((u, v) \in \{x \in a|{\rm Pair}(x)\} \wedge (v, w) \in b) &\leftrightarrow (u, w) \in b \circ \{x \in a|{\rm Pair}(x)\}, \\
  \mbox{} \\
  \tag{5}
  \exists v((u, v) \in a \wedge (v, w) \in \{y \in b|{\rm Pair}(y)\}) &\leftrightarrow (u, w) \in \{y \in b|{\rm Pair}(y)\} \circ a
\end{align*}
が共に成り立つ.
そこで(1), (2), (4)から, 推論法則 \ref{dedeqtrans}によって
\[
\tag{6}
  (u, w) \in b \circ a \leftrightarrow (u, w) \in b \circ \{x \in a|{\rm Pair}(x)\}
\]
が成り立ち, (1), (3), (5)から, 同じく推論法則 \ref{dedeqtrans}によって
\[
\tag{7}
  (u, w) \in b \circ a \leftrightarrow (u, w) \in \{y \in b|{\rm Pair}(y)\} \circ a
\]
が成り立つことがわかる.
さていま定理 \ref{sthmcompgraph}より, $b \circ a$, $b \circ \{x \in a|{\rm Pair}(x)\}$, 
$\{y \in b|{\rm Pair}(y)\} \circ a$はいずれもグラフである.
また$u$と$w$は共に$x$とも$y$とも異なり, $a$及び$b$の中に自由変数として現れないから, 
変数法則 \ref{valsset}, \ref{valbigpair}, \ref{valcomp}によってわかるように, 
これらは共に$b \circ a$, $b \circ \{x \in a|{\rm Pair}(x)\}$, 
$\{y \in b|{\rm Pair}(y)\} \circ a$のいずれの記号列の中にも自由変数として現れない.
また$u$と$w$は互いに異なり, 共に定数でない.
以上のことと, (6)と(7)が成り立つことから, 定理 \ref{sthmgraphpair=}により
\[
  b \circ a = b \circ \{x \in a|{\rm Pair}(x)\}, ~~
  b \circ a = \{y \in b|{\rm Pair}(y)\} \circ a
\]
が共に成り立つ.
\halmos




\mathstrut
\begin{thm}
\label{sthmcupcomp}%定理
$a$, $b$, $c$を集合とするとき, 
\[
  c \circ (a \cup b) = (c \circ a) \cup (c \circ b), ~~
  (a \cup b) \circ c = (a \circ c) \cup (b \circ c)
\]
が成り立つ.
\end{thm}


\noindent{\bf 証明}
~$x$, $y$, $z$を, どの二つも互いに異なり, どの一つも$a$, $b$, $c$のいずれの記号列の中にも
自由変数として現れない, 定数でない文字とする.
このとき変数法則 \ref{valcup}により, $y$は$a \cup b$の中にも自由変数として現れないから, 
定理 \ref{sthmpairincompeq}より
\begin{align*}
  \tag{1}
  (x, z) \in c \circ (a \cup b) &\leftrightarrow \exists y((x, y) \in a \cup b \wedge (y, z) \in c), \\
  \mbox{} \\
  \tag{2}
  (x, z) \in (a \cup b) \circ c &\leftrightarrow \exists y((x, y) \in c \wedge (y, z) \in a \cup b)
\end{align*}
が共に成り立つ.
また定理 \ref{sthmcupbasis}より
\[
  (x, y) \in a \cup b \leftrightarrow (x, y) \in a \vee (x, y) \in b, ~~
  (y, z) \in a \cup b \leftrightarrow (y, z) \in a \vee (y, z) \in b
\]
が共に成り立つから, 推論法則 \ref{dedaddeqw}により
\begin{align*}
  \tag{3}
  (x, y) \in a \cup b \wedge (y, z) \in c &\leftrightarrow ((x, y) \in a \vee (x, y) \in b) \wedge (y, z) \in c, \\
  \mbox{} \\
  \tag{4}
  (x, y) \in c \wedge (y, z) \in a \cup b &\leftrightarrow (x, y) \in c \wedge ((y, z) \in a \vee (y, z) \in b)
\end{align*}
が共に成り立つ.
またThm \ref{aw1bvc1l1awb1v1awc1}より
\begin{align*}
  \tag{5}
  ((x, y) \in a \vee (x, y) \in b) \wedge (y, z) \in c 
  &\leftrightarrow ((x, y) \in a \wedge (y, z) \in c) \vee ((x, y) \in b \wedge (y, z) \in c), \\
  \mbox{} \\
  \tag{6}
  (x, y) \in c \wedge ((y, z) \in a \vee (y, z) \in b) 
  &\leftrightarrow ((x, y) \in c \wedge (y, z) \in a) \vee ((x, y) \in c \wedge (y, z) \in b)
\end{align*}
が共に成り立つ.
そこで(3)と(5), (4)と(6)から, それぞれ推論法則 \ref{dedeqtrans}によって
\begin{align*}
  (x, y) \in a \cup b \wedge (y, z) \in c 
  &\leftrightarrow ((x, y) \in a \wedge (y, z) \in c) \vee ((x, y) \in b \wedge (y, z) \in c), \\
  \mbox{} \\
  (x, y) \in c \wedge (y, z) \in a \cup b 
  &\leftrightarrow ((x, y) \in c \wedge (y, z) \in a) \vee ((x, y) \in c \wedge (y, z) \in b)
\end{align*}
が成り立つ.
いま$y$は定数でないから, これらから, 推論法則 \ref{dedalleqquansepconst}によって
\begin{align*}
  \tag{7}
  \exists y((x, y) \in a \cup b \wedge (y, z) \in c) 
  &\leftrightarrow \exists y(((x, y) \in a \wedge (y, z) \in c) \vee ((x, y) \in b \wedge (y, z) \in c)), \\
  \mbox{} \\
  \tag{8}
  \exists y((x, y) \in c \wedge (y, z) \in a \cup b) 
  &\leftrightarrow \exists y(((x, y) \in c \wedge (y, z) \in a) \vee ((x, y) \in c \wedge (y, z) \in b))
\end{align*}
が共に成り立つ.
またThm \ref{thmexv}より
\begin{multline*}
\tag{9}
  \exists y(((x, y) \in a \wedge (y, z) \in c) \vee ((x, y) \in b \wedge (y, z) \in c)) \\
  \leftrightarrow \exists y((x, y) \in a \wedge (y, z) \in c) \vee \exists y((x, y) \in b \wedge (y, z) \in c), 
\end{multline*}
\begin{multline*}
\tag{10}
  \exists y(((x, y) \in c \wedge (y, z) \in a) \vee ((x, y) \in c \wedge (y, z) \in b)) \\
  \leftrightarrow \exists y((x, y) \in c \wedge (y, z) \in a) \vee \exists y((x, y) \in c \wedge (y, z) \in b)
\end{multline*}
が共に成り立つ.
また$y$が$x$とも$z$とも異なり, $a$, $b$, $c$のいずれの記号列の中にも自由変数として現れないことから, 
定理 \ref{sthmpairincompeq}と推論法則 \ref{dedeqch}により
\begin{align*}
  \exists y((x, y) \in a \wedge (y, z) \in c) \leftrightarrow (x, z) \in c \circ a&, ~~
  \exists y((x, y) \in b \wedge (y, z) \in c) \leftrightarrow (x, z) \in c \circ b, \\
  \mbox{} \\
  \exists y((x, y) \in c \wedge (y, z) \in a) \leftrightarrow (x, z) \in a \circ c&, ~~
  \exists y((x, y) \in c \wedge (y, z) \in b) \leftrightarrow (x, z) \in b \circ c
\end{align*}
がすべて成り立つ.
そこでこのはじめの二つ, 後の二つから, それぞれ推論法則 \ref{dedaddeqv}によって
\begin{align*}
  \tag{11}
  \exists y((x, y) \in a \wedge (y, z) \in c) \vee \exists y((x, y) \in b \wedge (y, z) \in c) 
  &\leftrightarrow (x, z) \in c \circ a \vee (x, z) \in c \circ b, \\
  \mbox{} \\
  \tag{12}
  \exists y((x, y) \in c \wedge (y, z) \in a) \vee \exists y((x, y) \in c \wedge (y, z) \in b) 
  &\leftrightarrow (x, z) \in a \circ c \vee (x, z) \in b \circ c
\end{align*}
が成り立つ.
また定理 \ref{sthmcupbasis}と推論法則 \ref{dedeqch}により
\begin{align*}
  \tag{13}
  (x, z) \in c \circ a \vee (x, z) \in c \circ b &\leftrightarrow (x, z) \in (c \circ a) \cup (c \circ b), \\
  \mbox{} \\
  \tag{14}
  (x, z) \in a \circ c \vee (x, z) \in b \circ c &\leftrightarrow (x, z) \in (a \circ c) \cup (b \circ c)
\end{align*}
が共に成り立つ.
以上の(1), (7), (9), (11), (13)から, 推論法則 \ref{dedeqtrans}によって
\[
\tag{15}
  (x, z) \in c \circ (a \cup b) \leftrightarrow (x, z) \in (c \circ a) \cup (c \circ b)
\]
が成り立ち, (2), (8), (10), (12), (14)から, 同じく推論法則 \ref{dedeqtrans}によって
\[
\tag{16}
  (x, z) \in (a \cup b) \circ c \leftrightarrow (x, z) \in (a \circ c) \cup (b \circ c)
\]
が成り立つことがわかる.
さていま定理 \ref{sthmcupgraph}, \ref{sthmcompgraph}によってわかるように, 
$c \circ (a \cup b)$, $(c \circ a) \cup (c \circ b)$, $(a \cup b) \circ c$, $(a \circ c) \cup (b \circ c)$は
いずれもグラフである.
また$x$と$z$は共に$a$, $b$, $c$のいずれの記号列の中にも自由変数として現れないから, 
変数法則 \ref{valcup}, \ref{valcomp}により, これらは共に
$c \circ (a \cup b)$, $(c \circ a) \cup (c \circ b)$, $(a \cup b) \circ c$, $(a \circ c) \cup (b \circ c)$の
いずれの記号列の中にも自由変数として現れない.
また$x$と$z$は互いに異なり, 共に定数でない.
そこでこれらのことと, (15), (16)が成り立つことから, 
定理 \ref{sthmgraphpair=}により
\[
  c \circ (a \cup b) = (c \circ a) \cup (c \circ b), ~~
  (a \cup b) \circ c = (a \circ c) \cup (b \circ c)
\]
が共に成り立つ.
\halmos




\mathstrut
\begin{thm}
\label{sthmcapcomp}%定理
$a$, $b$, $c$を集合とするとき, 
\[
  c \circ (a \cap b) \subset (c \circ a) \cap (c \circ b), ~~
  (a \cap b) \circ c \subset (a \circ c) \cap (b \circ c)
\]
が成り立つ.
\end{thm}


\noindent{\bf 証明}
~定理 \ref{sthmcap}より
\[
  a \cap b \subset a, ~~
  a \cap b \subset b
\]
が共に成り立つから, 定理 \ref{sthmcompsubset}により
\begin{align*}
  c \circ (a \cap b) \subset c \circ a&, ~~
  c \circ (a \cap b) \subset c \circ b, \\
  \mbox{} \\
  (a \cap b) \circ c \subset a \circ c&, ~~
  (a \cap b) \circ c \subset b \circ c
\end{align*}
がすべて成り立つ.
そこでこのはじめの二つ, 後の二つから, それぞれ定理 \ref{sthmcapdil}により
\[
  c \circ (a \cap b) \subset (c \circ a) \cap (c \circ b), ~~
  (a \cap b) \circ c \subset (a \circ c) \cap (b \circ c)
\]
が成り立つ.
\halmos




\mathstrut
\begin{thm}
\label{sthm-comp}%定理
$a$, $b$, $c$を集合とするとき, 
\[
  (c \circ a) - (c \circ b) \subset c \circ (a - b), ~~
  (a \circ c) - (b \circ c) \subset (a - b) \circ c
\]
が成り立つ.
\end{thm}


\noindent{\bf 証明}
~$x$, $y$, $z$を, どの二つも互いに異なり, どの一つも$a$, $b$, $c$のいずれの
記号列の中にも自由変数として現れない, 定数でない文字とする.
このとき定理 \ref{sthm-basis}と推論法則 \ref{dedequiv}により
\begin{align*}
  \tag{1}
  (x, z) \in (c \circ a) - (c \circ b) &\to (x, z) \in c \circ a \wedge (x, z) \notin c \circ b, \\
  \mbox{} \\
  \tag{2}
  (x, z) \in (a \circ c) - (b \circ c) &\to (x, z) \in a \circ c \wedge (x, z) \notin b \circ c
\end{align*}
が共に成り立つ.
また$y$が$x$とも$z$とも異なり, $a$, $b$, $c$のいずれの記号列の中にも
自由変数として現れないことから, 
定理 \ref{sthmpairincompeq}と推論法則 \ref{dedequiv}により
\begin{align*}
  \tag{3}
  &(x, z) \in c \circ a \to \exists y((x, y) \in a \wedge (y, z) \in c), \\
  \mbox{} \\
  \tag{4}
  &(x, z) \in a \circ c \to \exists y((x, y) \in c \wedge (y, z) \in a), \\
  \mbox{} \\
  \tag{5}
  &\exists y((x, y) \in b \wedge (y, z) \in c) \to (x, z) \in c \circ b, \\
  \mbox{} \\
  \tag{6}
  &\exists y((x, y) \in c \wedge (y, z) \in b) \to (x, z) \in b \circ c
\end{align*}
がすべて成り立つ.
そこでこの(5), (6)から, 推論法則 \ref{dedcp}により
\begin{align*}
  \tag{7}
  (x, z) \notin c \circ b &\to \neg \exists y((x, y) \in b \wedge (y, z) \in c), \\
  \mbox{} \\
  \tag{8}
  (x, z) \notin b \circ c &\to \neg \exists y((x, y) \in c \wedge (y, z) \in b)
\end{align*}
が共に成り立つ.
またThm \ref{thmeaquandm}と推論法則 \ref{dedequiv}により
\begin{align*}
  \tag{9}
  \neg \exists y((x, y) \in b \wedge (y, z) \in c) &\to \forall y(\neg ((x, y) \in b \wedge (y, z) \in c)), \\
  \mbox{} \\
  \tag{10}
  \neg \exists y((x, y) \in c \wedge (y, z) \in b) &\to \forall y(\neg ((x, y) \in c \wedge (y, z) \in b))
\end{align*}
が共に成り立つ.
またThm \ref{n1awb1tnavnb}より
\begin{align*}
  \neg ((x, y) \in b \wedge (y, z) \in c) &\to (x, y) \notin b \vee (y, z) \notin c, \\
  \mbox{} \\
  \neg ((x, y) \in c \wedge (y, z) \in b) &\to (x, y) \notin c \vee (y, z) \notin b
\end{align*}
が共に成り立つ.
いま$y$は定数でないから, これらから推論法則 \ref{dedalltquansepconst}により
\begin{align*}
  \tag{11}
  \forall y(\neg ((x, y) \in b \wedge (y, z) \in c)) &\to \forall y((x, y) \notin b \vee (y, z) \notin c), \\
  \mbox{} \\
  \tag{12}
  \forall y(\neg ((x, y) \in c \wedge (y, z) \in b)) &\to \forall y((x, y) \notin c \vee (y, z) \notin b)
\end{align*}
が共に成り立つ.
そこで(7), (9), (11)から, 推論法則 \ref{dedmmp}によって
\[
\tag{13}
  (x, z) \notin c \circ b \to \forall y((x, y) \notin b \vee (y, z) \notin c)
\]
が成り立ち, 
(8), (10), (12)から, 同じく推論法則 \ref{dedmmp}によって
\[
\tag{14}
  (x, z) \notin b \circ c \to \forall y((x, y) \notin c \vee (y, z) \notin b)
\]
が成り立つことがわかる.
そこで(3)と(13), (4)と(14)から, それぞれ推論法則 \ref{dedfromaddw}によって
\begin{align*}
  \tag{15}
  (x, z) \in c \circ a \wedge (x, z) \notin c \circ b 
  &\to \exists y((x, y) \in a \wedge (y, z) \in c) \wedge \forall y((x, y) \notin b \vee (y, z) \notin c), \\
  \mbox{} \\
  \tag{16}
  (x, z) \in a \circ c \wedge (x, z) \notin b \circ c 
  &\to \exists y((x, y) \in c \wedge (y, z) \in a) \wedge \forall y((x, y) \notin c \vee (y, z) \notin b)
\end{align*}
が成り立つ.
またThm \ref{thmexw2}より
\begin{multline*}
\tag{17}
  \exists y((x, y) \in a \wedge (y, z) \in c) \wedge \forall y((x, y) \notin b \vee (y, z) \notin c) \\
  \to \exists y(((x, y) \in a \wedge (y, z) \in c) \wedge ((x, y) \notin b \vee (y, z) \notin c)), 
\end{multline*}
\begin{multline*}
\tag{18}
  \exists y((x, y) \in c \wedge (y, z) \in a) \wedge \forall y((x, y) \notin c \vee (y, z) \notin b) \\
  \to \exists y(((x, y) \in c \wedge (y, z) \in a) \wedge ((x, y) \notin c \vee (y, z) \notin b))
\end{multline*}
が共に成り立つ.
またThm \ref{aw1bvc1t1awb1v1awc1}より
\begin{multline*}
\tag{19}
  ((x, y) \in a \wedge (y, z) \in c) \wedge ((x, y) \notin b \vee (y, z) \notin c) \\
  \to (((x, y) \in a \wedge (y, z) \in c) \wedge (x, y) \notin b) \vee (((x, y) \in a \wedge (y, z) \in c) \wedge (y, z) \notin c), 
\end{multline*}
\begin{multline*}
\tag{20}
  ((x, y) \in c \wedge (y, z) \in a) \wedge ((x, y) \notin c \vee (y, z) \notin b) \\
  \to (((x, y) \in c \wedge (y, z) \in a) \wedge (x, y) \notin c) \vee (((x, y) \in c \wedge (y, z) \in a) \wedge (y, z) \notin b)
\end{multline*}
が共に成り立つ.
またThm \ref{awblbwa}より
\[
  (x, y) \in c \wedge (y, z) \in a \leftrightarrow (y, z) \in a \wedge (x, y) \in c
\]
が成り立つから, 推論法則 \ref{dedaddeqw}により
\[
\tag{21}
  ((x, y) \in c \wedge (y, z) \in a) \wedge (x, y) \notin c \leftrightarrow ((y, z) \in a \wedge (x, y) \in c) \wedge (x, y) \notin c
\]
が成り立つ.
またThm \ref{1awb1wclaw1bwc1}より
\begin{align*}
  \tag{22}
  ((x, y) \in a \wedge (y, z) \in c) \wedge (y, z) \notin c &\leftrightarrow (x, y) \in a \wedge ((y, z) \in c \wedge (y, z) \notin c), \\
  \mbox{} \\
  \tag{23}
  ((y, z) \in a \wedge (x, y) \in c) \wedge (x, y) \notin c &\leftrightarrow (y, z) \in a \wedge ((x, y) \in c \wedge (x, y) \notin c)
\end{align*}
が共に成り立つ.
またThm \ref{n1awna1}より
\[
  \neg ((y, z) \in c \wedge (y, z) \notin c), ~~
  \neg ((x, y) \in c \wedge (x, y) \notin c)
\]
が共に成り立つから, 推論法則 \ref{dednw}により
\[
  \neg ((x, y) \in a \wedge ((y, z) \in c \wedge (y, z) \notin c)), ~~
  \neg ((y, z) \in a \wedge ((x, y) \in c \wedge (x, y) \notin c))
\]
が共に成り立つ.
そこでこの前者と(22), 後者と(23)から, それぞれ推論法則 \ref{dedeqfund}によって
\begin{align*}
  \tag{24}
  &\neg (((x, y) \in a \wedge (y, z) \in c) \wedge (y, z) \notin c), \\
  \mbox{} \\
  \tag{25}
  &\neg (((y, z) \in a \wedge (x, y) \in c) \wedge (x, y) \notin c)
\end{align*}
が成り立つ.
この(24)から, 推論法則 \ref{dedavbtbtrue2}により
\begin{multline*}
\tag{26}
  (((x, y) \in a \wedge (y, z) \in c) \wedge (x, y) \notin b) \vee (((x, y) \in a \wedge (y, z) \in c) \wedge (y, z) \notin c) \\
  \to ((x, y) \in a \wedge (y, z) \in c) \wedge (x, y) \notin b
\end{multline*}
が成り立つ.
また(25)と(21)から, 推論法則 \ref{dedeqfund}によって
\[
  \neg (((x, y) \in c \wedge (y, z) \in a) \wedge (x, y) \notin c)
\]
が成り立つから, 同じく推論法則 \ref{dedavbtbtrue2}により
\begin{multline*}
\tag{27}
  (((x, y) \in c \wedge (y, z) \in a) \wedge (x, y) \notin c) \vee (((x, y) \in c \wedge (y, z) \in a) \wedge (y, z) \notin b) \\
  \to ((x, y) \in c \wedge (y, z) \in a) \wedge (y, z) \notin b
\end{multline*}
が成り立つ.
またThm \ref{1awb1wctaw1bwc1}より
\begin{align*}
  \tag{28}
  ((x, y) \in a \wedge (y, z) \in c) \wedge (x, y) \notin b &\to (x, y) \in a \wedge ((y, z) \in c \wedge (x, y) \notin b), \\
  \mbox{} \\
  \tag{29}
  ((x, y) \in c \wedge (y, z) \in a) \wedge (y, z) \notin b &\to (x, y) \in c \wedge ((y, z) \in a \wedge (y, z) \notin b)
\end{align*}
が共に成り立つ.
またThm \ref{awbtbwa}より
\[
  (y, z) \in c \wedge (x, y) \notin b \to (x, y) \notin b \wedge (y, z) \in c
\]
が成り立つから, 推論法則 \ref{dedaddw}により
\[
\tag{30}
  (x, y) \in a \wedge ((y, z) \in c \wedge (x, y) \notin b) \to (x, y) \in a \wedge ((x, y) \notin b \wedge (y, z) \in c)
\]
が成り立つ.
またThm \ref{aw1bwc1t1awb1wc}より
\[
\tag{31}
  (x, y) \in a \wedge ((x, y) \notin b \wedge (y, z) \in c) \to ((x, y) \in a \wedge (x, y) \notin b) \wedge (y, z) \in c
\]
が成り立つ.
また定理 \ref{sthm-basis}と推論法則 \ref{dedequiv}により
\begin{align*}
  (x, y) \in a \wedge (x, y) \notin b &\to (x, y) \in a - b, \\
  \mbox{} \\
  (y, z) \in a \wedge (y, z) \notin b &\to (y, z) \in a - b
\end{align*}
が共に成り立つから, 推論法則 \ref{dedaddw}により
\begin{align*}
  \tag{32}
  ((x, y) \in a \wedge (x, y) \notin b) \wedge (y, z) \in c &\to (x, y) \in a - b \wedge (y, z) \in c, \\
  \mbox{} \\
  \tag{33}
  (x, y) \in c \wedge ((y, z) \in a \wedge (y, z) \notin b) &\to (x, y) \in c \wedge (y, z) \in a - b
\end{align*}
が共に成り立つ.
そこで(19), (26), (28), (30), (31), (32)から, 推論法則 \ref{dedmmp}によって
\[
  ((x, y) \in a \wedge (y, z) \in c) \wedge ((x, y) \notin b \vee (y, z) \notin c) 
  \to (x, y) \in a - b \wedge (y, z) \in c
\]
が成り立ち, 
(20), (27), (29), (33)から, 同じく推論法則 \ref{dedmmp}によって
\[
  ((x, y) \in c \wedge (y, z) \in a) \wedge ((x, y) \notin c \vee (y, z) \notin b) 
  \to (x, y) \in c \wedge (y, z) \in a - b
\]
が成り立つことがわかる.
いま$y$は定数でないから, これらから推論法則 \ref{dedalltquansepconst}により
\begin{align*}
  \tag{34}
  \exists y(((x, y) \in a \wedge (y, z) \in c) \wedge ((x, y) \notin b \vee (y, z) \notin c)) 
  &\to \exists y((x, y) \in a - b \wedge (y, z) \in c), \\
  \mbox{} \\
  \tag{35}
  \exists y(((x, y) \in c \wedge (y, z) \in a) \wedge ((x, y) \notin c \vee (y, z) \notin b)) 
  &\to \exists y((x, y) \in c \wedge (y, z) \in a - b)
\end{align*}
が共に成り立つ.
また$y$は$a$及び$b$の中に自由変数として現れないから, 
変数法則 \ref{val-}により, $y$は$a - b$の中にも自由変数として現れない.
このことと, $y$が$x$とも$z$とも異なり, $c$の中にも自由変数として現れないことから, 
定理 \ref{sthmpairincompeq}と推論法則 \ref{dedequiv}により
\begin{align*}
  \tag{36}
  \exists y((x, y) \in a - b \wedge (y, z) \in c) &\to (x, z) \in c \circ (a - b), \\
  \mbox{} \\
  \tag{37}
  \exists y((x, y) \in c \wedge (y, z) \in a - b) &\to (x, z) \in (a - b) \circ c
\end{align*}
が共に成り立つ.
以上の(1), (15), (17), (34), (36)から, 推論法則 \ref{dedmmp}によって
\[
\tag{38}
  (x, z) \in (c \circ a) - (c \circ b) \to (x, z) \in c \circ (a - b)
\]
が成り立ち, 
(2), (16), (18), (35), (37)から, 同じく推論法則 \ref{dedmmp}によって
\[
\tag{39}
  (x, z) \in (a \circ c) - (b \circ c) \to (x, z) \in (a - b) \circ c
\]
が成り立つことがわかる.
さていま定理 \ref{sthm-graph}, \ref{sthmcompgraph}によってわかるように, 
$(c \circ a) - (c \circ b)$と$(a \circ c) - (b \circ c)$は
共にグラフである.
また$x$と$z$は共に$a$, $b$, $c$のいずれの記号列の中にも自由変数として現れないから, 
変数法則 \ref{val-}, \ref{valcomp}により, これらは共に
$(c \circ a) - (c \circ b)$, $c \circ (a - b)$, $(a \circ c) - (b \circ c)$, $(a - b) \circ c$の
いずれの記号列の中にも自由変数として現れない.
また$x$と$z$は互いに異なり, 共に定数でない.
これらのことと, (38), (39)が共に成り立つことから, 
定理 \ref{sthmgraphpairsubset}により
\[
  (c \circ a) - (c \circ b) \subset c \circ (a - b), ~~
  (a \circ c) - (b \circ c) \subset (a - b) \circ c
\]
が共に成り立つ.
\halmos




\mathstrut
\begin{thm}
\label{sthmemptycomp}%定理
$a$と$b$を集合とするとき, 
\[
  b \circ a = \phi \leftrightarrow {\rm pr}_{2}\langle a \rangle \cap {\rm pr}_{1}\langle b \rangle = \phi
\]
が成り立つ.
またこのことから, 次の1), 2)が成り立つ.

1)
$b \circ a$が空ならば, ${\rm pr}_{2}\langle a \rangle \cap {\rm pr}_{1}\langle b \rangle$は空である.
逆に${\rm pr}_{2}\langle a \rangle \cap {\rm pr}_{1}\langle b \rangle$が空ならば, $b \circ a$は空である.

2)
$b \circ \phi$と$\phi \circ a$は共に空である.
\end{thm}


\noindent{\bf 証明}
~$y$を$a$及び$b$の中に自由変数として現れない文字とする.
このとき変数法則 \ref{valcomp}により, $y$は$b \circ a$の中に自由変数として現れない.
そこで$\tau_{y}(y \in b \circ a)$を$T$と書けば, $T$は集合であり, 
定理 \ref{sthmelm&empty}と推論法則 \ref{dedequiv}により
\[
\tag{1}
  b \circ a \neq \phi \to T \in b \circ a
\]
が成り立つ.
また$T$の定義から, 変数法則 \ref{valtau}により, $y$は$T$の中に自由変数として現れない.
このことと, $y$が$a$及び$b$の中にも自由変数として現れないことから, 
定理 \ref{sthmcompelement}と推論法則 \ref{dedequiv}により
\[
  T \in b \circ a \to {\rm Pair}(T) \wedge \exists y(({\rm pr}_{1}(T), y) \in a \wedge (y, {\rm pr}_{2}(T)) \in b)
\]
が成り立つ.
そこで特に, 推論法則 \ref{dedprewedge}により
\[
  T \in b \circ a \to \exists y(({\rm pr}_{1}(T), y) \in a \wedge (y, {\rm pr}_{2}(T)) \in b)
\]
が成り立つ.
ここで$\tau_{y}(({\rm pr}_{1}(T), y) \in a \wedge (y, {\rm pr}_{2}(T)) \in b)$を$U$と書けば, 
$U$は集合であり, 定義から上記の記号列は
\[
  T \in b \circ a \to (U|y)(({\rm pr}_{1}(T), y) \in a \wedge (y, {\rm pr}_{2}(T)) \in b)
\]
と同じである.
また上述のように$y$は$T$の中に自由変数として現れないから, 
変数法則 \ref{valpr}により, $y$は${\rm pr}_{1}(T)$及び${\rm pr}_{2}(T)$の中に
自由変数として現れない.
このことと, $y$が$a$及び$b$の中にも自由変数として現れないことから, 
代入法則 \ref{substfree}, \ref{substfund}, \ref{substwedge}, \ref{substpair}により, 
上記の記号列は
\[
\tag{2}
  T \in b \circ a \to ({\rm pr}_{1}(T), U) \in a \wedge (U, {\rm pr}_{2}(T)) \in b
\]
と一致する.
よってこれが定理となる.
また定理 \ref{sthmpairelementinprset}より
\begin{align*}
  ({\rm pr}_{1}(T), U) \in a &\to {\rm pr}_{1}(T) \in {\rm pr}_{1}\langle a \rangle \wedge U \in {\rm pr}_{2}\langle a \rangle, \\
  \mbox{} \\
  (U, {\rm pr}_{2}(T)) \in b &\to U \in {\rm pr}_{1}\langle b \rangle \wedge {\rm pr}_{2}(T) \in {\rm pr}_{2}\langle b \rangle
\end{align*}
が共に成り立つから, 推論法則 \ref{dedprewedge}により
\begin{align*}
  ({\rm pr}_{1}(T), U) \in a &\to U \in {\rm pr}_{2}\langle a \rangle, \\
  \mbox{} \\
  (U, {\rm pr}_{2}(T)) \in b &\to U \in {\rm pr}_{1}\langle b \rangle
\end{align*}
が共に成り立つ.
そこでこれらから, 推論法則 \ref{dedfromaddw}により
\[
\tag{3}
  ({\rm pr}_{1}(T), U) \in a \wedge (U, {\rm pr}_{2}(T)) \in b 
  \to U \in {\rm pr}_{2}\langle a \rangle \wedge U \in {\rm pr}_{1}\langle b \rangle
\]
が成り立つ.
また定理 \ref{sthmcapelement}と推論法則 \ref{dedequiv}により
\[
\tag{4}
  U \in {\rm pr}_{2}\langle a \rangle \wedge U \in {\rm pr}_{1}\langle b \rangle 
  \to U \in {\rm pr}_{2}\langle a \rangle \cap {\rm pr}_{1}\langle b \rangle
\]
が成り立つ.
また定理 \ref{sthmnotemptyeqexin}より
\[
\tag{5}
  U \in {\rm pr}_{2}\langle a \rangle \cap {\rm pr}_{1}\langle b \rangle 
  \to {\rm pr}_{2}\langle a \rangle \cap {\rm pr}_{1}\langle b \rangle \neq \phi
\]
が成り立つ.
そこで(1)---(5)から, 推論法則 \ref{dedmmp}によって
\[
\tag{6}
  b \circ a \neq \phi \to {\rm pr}_{2}\langle a \rangle \cap {\rm pr}_{1}\langle b \rangle \neq \phi
\]
が成り立つことがわかる.
またいま$y$が$a$及び$b$の中に自由変数として現れないことから, 
変数法則 \ref{valcap}, \ref{valprset}により, 
$y$は${\rm pr}_{2}\langle a \rangle \cap {\rm pr}_{1}\langle b \rangle$の中に
自由変数として現れない.
そこで$\tau_{y}(y \in {\rm pr}_{2}\langle a \rangle \cap {\rm pr}_{1}\langle b \rangle)$を
$V$と書けば, $V$は集合であり, 定理 \ref{sthmelm&empty}と推論法則 \ref{dedequiv}により
\[
\tag{7}
  {\rm pr}_{2}\langle a \rangle \cap {\rm pr}_{1}\langle b \rangle \neq \phi 
  \to V \in {\rm pr}_{2}\langle a \rangle \cap {\rm pr}_{1}\langle b \rangle
\]
が成り立つ.
また定理 \ref{sthmcapelement}と推論法則 \ref{dedequiv}により
\[
\tag{8}
  V \in {\rm pr}_{2}\langle a \rangle \cap {\rm pr}_{1}\langle b \rangle 
  \to V \in {\rm pr}_{2}\langle a \rangle \wedge V \in {\rm pr}_{1}\langle b \rangle
\]
が成り立つ.
さていま$x$を$y$と異なり, $a$及び$b$の中に自由変数として現れない文字とする.
このとき変数法則 \ref{valfund}, \ref{valtau}, \ref{valcap}, \ref{valprset}によってわかるように, 
$x$と$y$は共に$V$の中に自由変数として現れない.
このことと, $x$と$y$が共に$a$及び$b$の中にも自由変数として現れないことから, 
定理 \ref{sthmprsetelement}と推論法則 \ref{dedequiv}により
\[
  V \in {\rm pr}_{2}\langle a \rangle \to \exists x((x, V) \in a), ~~
  V \in {\rm pr}_{1}\langle b \rangle \to \exists y((V, y) \in b)
\]
が共に成り立つ.
ここで$\tau_{x}((x, V) \in a)$, $\tau_{y}((V, y) \in b)$をそれぞれ
$X$, $Y$と書けば, これらは共に集合であり, 定義から上記の記号列は
それぞれ
\[
  V \in {\rm pr}_{2}\langle a \rangle \to (X|x)((x, V) \in a), ~~
  V \in {\rm pr}_{1}\langle b \rangle \to (Y|y)((V, y) \in b)
\]
と同じである.
またいま述べたように, $x$と$y$は共に$a$, $b$, $V$のいずれの記号列の中にも
自由変数として現れないから, 
代入法則 \ref{substfree}, \ref{substfund}, \ref{substpair}により, 
これらの記号列はそれぞれ
\[
  V \in {\rm pr}_{2}\langle a \rangle \to (X, V) \in a, ~~
  V \in {\rm pr}_{1}\langle b \rangle \to (V, Y) \in b
\]
と一致する.
よってこれらが共に定理となる.
そこで推論法則 \ref{dedfromaddw}により, 
\[
\tag{9}
  V \in {\rm pr}_{2}\langle a \rangle \wedge V \in {\rm pr}_{1}\langle b \rangle 
  \to (X, V) \in a \wedge (V, Y) \in b
\]
が成り立つ.
また定理 \ref{sthmpairincompt}より
\[
\tag{10}
  (X, V) \in a \wedge (V, Y) \in b \to (X, Y) \in b \circ a
\]
が成り立つ.
また定理 \ref{sthmnotemptyeqexin}より
\[
\tag{11}
  (X, Y) \in b \circ a \to b \circ a \neq \phi
\]
が成り立つ.
そこで(7)---(11)から, 推論法則 \ref{dedmmp}によって
\[
\tag{12}
  {\rm pr}_{2}\langle a \rangle \cap {\rm pr}_{1}\langle b \rangle \neq \phi 
  \to b \circ a \neq \phi
\]
が成り立つことがわかる.
そこで(6), (12)から, 推論法則 \ref{dedequiv}によって
\[
  b \circ a \neq \phi \leftrightarrow {\rm pr}_{2}\langle a \rangle \cap {\rm pr}_{1}\langle b \rangle \neq \phi
\]
が成り立ち, これから推論法則 \ref{dedeqcp}によって
\[
\tag{13}
  b \circ a = \phi \leftrightarrow {\rm pr}_{2}\langle a \rangle \cap {\rm pr}_{1}\langle b \rangle = \phi
\]
が成り立つ.

\noindent
1)
いま示したように(13)が成り立つから, 1)が成り立つことはこれと推論法則 \ref{dedeqfund}から明らか.

\noindent
2)
定理 \ref{sthmemptyprset}より
${\rm pr}_{2}\langle \phi \rangle = \phi$と${\rm pr}_{1}\langle \phi \rangle = \phi$が共に成り立つから, 
定理 \ref{sthmcap=}により
\[
  {\rm pr}_{2}\langle \phi \rangle \cap {\rm pr}_{1}\langle b \rangle = \phi \cap {\rm pr}_{1}\langle b \rangle, ~~
  {\rm pr}_{2}\langle a \rangle \cap {\rm pr}_{1}\langle \phi \rangle = {\rm pr}_{2}\langle a \rangle \cap \phi
\]
が共に成り立つ.
また定理 \ref{sthmcapempty}より
\[
  \phi \cap {\rm pr}_{1}\langle b \rangle = \phi, ~~
  {\rm pr}_{2}\langle a \rangle \cap \phi = \phi
\]
が共に成り立つ.
そこでこれらから, 推論法則 \ref{ded=trans}によって
\[
  {\rm pr}_{2}\langle \phi \rangle \cap {\rm pr}_{1}\langle b \rangle = \phi, ~~
  {\rm pr}_{2}\langle a \rangle \cap {\rm pr}_{1}\langle \phi \rangle = \phi
\]
が共に成り立つ.
故に1)により, $b \circ \phi = \phi$と$\phi \circ a = \phi$が共に成り立つ.
\halmos




\mathstrut
\begin{thm}
\label{sthmproductcomp}%定理
$a$, $b$, $c$を集合とするとき, 
\[
  c \circ (a \times b) = a \times c[b], ~~
  (a \times b) \circ c = c^{-1}[a] \times b
\]
が成り立つ.
\end{thm}


\noindent{\bf 証明}
~$x$, $y$, $z$を, どの二つも互いに異なり, どの一つも$a$, $b$, $c$のいずれの記号列の中にも
自由変数として現れない, 定数でない文字とする.
このとき変数法則 \ref{valproduct}により, $y$は$a \times b$の中にも
自由変数として現れないから, 定理 \ref{sthmpairincompeq}より
\begin{align*}
  \tag{1}
  (x, z) \in c \circ (a \times b) &\leftrightarrow \exists y((x, y) \in a \times b \wedge (y, z) \in c), \\
  \mbox{} \\
  \tag{2}
  (x, z) \in (a \times b) \circ c &\leftrightarrow \exists y((x, y) \in c \wedge (y, z) \in a \times b)
\end{align*}
が共に成り立つ.
また定理 \ref{sthmpairinproduct}より
\[
  (x, y) \in a \times b \leftrightarrow x \in a \wedge y \in b, ~~
  (y, z) \in a \times b \leftrightarrow y \in a \wedge z \in b
\]
が共に成り立つから, 推論法則 \ref{dedaddeqw}により
\begin{align*}
  \tag{3}
  (x, y) \in a \times b \wedge (y, z) \in c &\leftrightarrow (x \in a \wedge y \in b) \wedge (y, z) \in c, \\
  \mbox{} \\
  \tag{4}
  (x, y) \in c \wedge (y, z) \in a \times b &\leftrightarrow (x, y) \in c \wedge (y \in a \wedge z \in b)
\end{align*}
が共に成り立つ.
またThm \ref{1awb1wclaw1bwc1}より
\[
\tag{5}
  (x \in a \wedge y \in b) \wedge (y, z) \in c \leftrightarrow x \in a \wedge (y \in b \wedge (y, z) \in c)
\]
が成り立ち, Thm \ref{1awb1wclaw1bwc1}と推論法則 \ref{dedeqch}により
\[
\tag{6}
  (x, y) \in c \wedge (y \in a \wedge z \in b) \leftrightarrow ((x, y) \in c \wedge y \in a) \wedge z \in b
\]
が成り立つ.
また定理 \ref{sthmpairininv}と推論法則 \ref{dedeqch}により
$(x, y) \in c \leftrightarrow (y, x) \in c^{-1}$が成り立つから, 
推論法則 \ref{dedaddeqw}により
\[
\tag{7}
  (x, y) \in c \wedge y \in a \leftrightarrow (y, x) \in c^{-1} \wedge y \in a
\]
が成り立つ.
またThm \ref{awblbwa}より
\[
\tag{8}
  (y, x) \in c^{-1} \wedge y \in a \leftrightarrow y \in a \wedge (y, x) \in c^{-1}
\]
が成り立つ.
そこで(7), (8)から, 推論法則 \ref{dedeqtrans}によって
\[
  (x, y) \in c \wedge y \in a \leftrightarrow y \in a \wedge (y, x) \in c^{-1}
\]
が成り立ち, これから推論法則 \ref{dedaddeqw}によって
\[
\tag{9}
  ((x, y) \in c \wedge y \in a) \wedge z \in b \leftrightarrow (y \in a \wedge (y, x) \in c^{-1}) \wedge z \in b
\]
が成り立つ.
そこで(3), (5)から, 推論法則 \ref{dedeqtrans}によって
\[
  (x, y) \in a \times b \wedge (y, z) \in c \leftrightarrow x \in a \wedge (y \in b \wedge (y, z) \in c)
\]
が成り立ち, (4), (6), (9)から, 同じく推論法則 \ref{dedeqtrans}によって
\[
  (x, y) \in c \wedge (y, z) \in a \times b \leftrightarrow (y \in a \wedge (y, x) \in c^{-1}) \wedge z \in b
\]
が成り立つことがわかる.
いま$y$は定数でないので, これらから推論法則 \ref{dedalleqquansepconst}によって
\begin{align*}
  \tag{10}
  \exists y((x, y) \in a \times b \wedge (y, z) \in c) 
  &\leftrightarrow \exists y(x \in a \wedge (y \in b \wedge (y, z) \in c)), \\
  \mbox{} \\
  \tag{11}
  \exists y((x, y) \in c \wedge (y, z) \in a \times b) 
  &\leftrightarrow \exists y((y \in a \wedge (y, x) \in c^{-1}) \wedge z \in b)
\end{align*}
が共に成り立つ.
また$y$は$x$とも$z$とも異なり, $a$及び$b$の中に自由変数として現れないから, 
変数法則 \ref{valfund}により, $y$は$x \in a$及び$z \in b$の中に自由変数として現れない.
そこでThm \ref{thmexwrfree}より
\begin{align*}
  \tag{12}
  \exists y(x \in a \wedge (y \in b \wedge (y, z) \in c)) 
  &\leftrightarrow x \in a \wedge \exists y(y \in b \wedge (y, z) \in c), \\
  \mbox{} \\
  \tag{13}
  \exists y((y \in a \wedge (y, x) \in c^{-1}) \wedge z \in b) 
  &\leftrightarrow \exists y(y \in a \wedge (y, x) \in c^{-1}) \wedge z \in b
\end{align*}
が共に成り立つ.
また$y$は$c$の中に自由変数として現れず, 従って変数法則 \ref{valinv}により, 
$y$は$c^{-1}$の中にも自由変数として現れない.
このことと, $y$が$x$とも$z$とも異なり, $a$及び$b$の中にも自由変数として現れないことから, 
定理 \ref{sthmvaluesetelement}と推論法則 \ref{dedeqch}により
\begin{align*}
  \exists y(y \in b \wedge (y, z) \in c) &\leftrightarrow z \in c[b], \\
  \mbox{} \\
  \exists y(y \in a \wedge (y, x) \in c^{-1}) &\leftrightarrow x \in c^{-1}[a]
\end{align*}
が共に成り立つ.
そこでこれらから, 推論法則 \ref{dedaddeqw}によって
\begin{align*}
  \tag{14}
  x \in a \wedge \exists y(y \in b \wedge (y, z) \in c) &\leftrightarrow x \in a \wedge z \in c[b], \\
  \mbox{} \\
  \tag{15}
  \exists y(y \in a \wedge (y, x) \in c^{-1}) \wedge z \in b &\leftrightarrow x \in c^{-1}[a] \wedge z \in b
\end{align*}
が共に成り立つ.
また定理 \ref{sthmpairinproduct}と推論法則 \ref{dedeqch}により
\begin{align*}
  \tag{16}
  x \in a \wedge z \in c[b] &\leftrightarrow (x, z) \in a \times c[b], \\
  \mbox{} \\
  \tag{17}
  x \in c^{-1}[a] \wedge z \in b &\leftrightarrow (x, z) \in c^{-1}[a] \times b
\end{align*}
が共に成り立つ.
以上の(1), (10), (12), (14), (16)から, 推論法則 \ref{dedeqtrans}によって
\[
\tag{18}
  (x, z) \in c \circ (a \times b) \leftrightarrow (x, z) \in a \times c[b]
\]
が成り立ち, 
(2), (11), (13), (15), (17)から, 同じく推論法則 \ref{dedeqtrans}によって
\[
\tag{19}
  (x, z) \in (a \times b) \circ c \leftrightarrow (x, z) \in c^{-1}[a] \times b
\]
が成り立つことがわかる.
いま定理 \ref{sthmproductgraph}, \ref{sthmcompgraph}によってわかるように, 
$c \circ (a \times b)$, $a \times c[b]$, $(a \times b) \circ c$, $c^{-1}[a] \times b$は
いずれもグラフである.
また$x$と$z$は共に$a$, $b$, $c$のいずれの記号列の中にも自由変数として現れないから, 
変数法則 \ref{valproduct}, \ref{valvalueset}, \ref{valinv}, \ref{valcomp}によってわかるように, 
これらは共に$c \circ (a \times b)$, $a \times c[b]$, $(a \times b) \circ c$, $c^{-1}[a] \times b$の
いずれの記号列の中にも自由変数として現れない.
また$x$と$z$は互いに異なり, 共に定数でない.
そこでこれらのことと, (18), (19)が共に成り立つことから, 
定理 \ref{sthmgraphpair=}により
\[
  c \circ (a \times b) = a \times c[b], ~~
  (a \times b) \circ c = c^{-1}[a] \times b
\]
が共に成り立つ.
\halmos




\mathstrut
\begin{thm}
\label{sthmprsetcomp}%定理
$a$と$b$を集合とするとき, 
\[
  {\rm pr}_{1}\langle b \circ a \rangle = a^{-1}[{\rm pr}_{1}\langle b \rangle], ~~
  {\rm pr}_{2}\langle b \circ a \rangle = b[{\rm pr}_{2}\langle a \rangle]
\]
が成り立つ.
\end{thm}


\noindent{\bf 証明}
~$x$と$z$を, 互いに異なり, 共に$a$及び$b$の中に自由変数として現れない, 
定数でない文字とする.
このとき変数法則 \ref{valcomp}により, $x$と$z$は共に$b \circ a$の中に自由変数として現れないから, 
定理 \ref{sthmprsetelement}より
\begin{align*}
  \tag{1}
  x \in {\rm pr}_{1}\langle b \circ a \rangle &\leftrightarrow \exists z((x, z) \in b \circ a), \\
  \mbox{} \\
  \tag{2}
  z \in {\rm pr}_{2}\langle b \circ a \rangle &\leftrightarrow \exists x((x, z) \in b \circ a)
\end{align*}
が共に成り立つ.
また$y$を$x$とも$z$とも異なり, $a$及び$b$の中に自由変数として現れない, 
定数でない文字とすれば, 定理 \ref{sthmpairincompeq}より
\[
  (x, z) \in b \circ a \leftrightarrow \exists y((x, y) \in a \wedge (y, z) \in b)
\]
が成り立つから, $x$と$z$が共に定数でないことから, 推論法則 \ref{dedalleqquansepconst}により
\begin{align*}
  \tag{3}
  \exists z((x, z) \in b \circ a) &\leftrightarrow \exists z(\exists y((x, y) \in a \wedge (y, z) \in b)), \\
  \mbox{} \\
  \tag{4}
  \exists x((x, z) \in b \circ a) &\leftrightarrow \exists x(\exists y((x, y) \in a \wedge (y, z) \in b))
\end{align*}
が共に成り立つ.
またThm \ref{thmexch}より
\begin{align*}
  \tag{5}
  \exists z(\exists y((x, y) \in a \wedge (y, z) \in b)) 
  &\leftrightarrow \exists y(\exists z((x, y) \in a \wedge (y, z) \in b)), \\
  \mbox{} \\
  \tag{6}
  \exists x(\exists y((x, y) \in a \wedge (y, z) \in b)) 
  &\leftrightarrow \exists y(\exists x((x, y) \in a \wedge (y, z) \in b))
\end{align*}
が共に成り立つ.
また$x$と$z$は互いに異なり, 共に$y$と異なり, $a$及び$b$の中に自由変数として現れないから, 
変数法則 \ref{valfund}, \ref{valpair}によってわかるように, 
$z$は$(x, y) \in a$の中に自由変数として現れず, $x$は$(y, z) \in b$の中に自由変数として現れない.
そこでThm \ref{thmexwrfree}より
\begin{align*}
  \tag{7}
  \exists z((x, y) \in a \wedge (y, z) \in b) &\leftrightarrow (x, y) \in a \wedge \exists z((y, z) \in b), \\
  \mbox{} \\
  \tag{8}
  \exists x((x, y) \in a \wedge (y, z) \in b) &\leftrightarrow \exists x((x, y) \in a) \wedge (y, z) \in b
\end{align*}
が共に成り立つ.
また定理 \ref{sthmpairininv}と推論法則 \ref{dedeqch}により
\[
\tag{9}
  (x, y) \in a \leftrightarrow (y, x) \in a^{-1}
\]
が成り立つ.
また$x$と$z$は共に$y$と異なり, $a$及び$b$の中に自由変数として現れないから, 
定理 \ref{sthmprsetelement}と推論法則 \ref{dedeqch}により
\begin{align*}
  \tag{10}
  \exists z((y, z) \in b) &\leftrightarrow y \in {\rm pr}_{1}\langle b \rangle, \\
  \mbox{} \\
  \tag{11}
  \exists x((x, y) \in a) &\leftrightarrow y \in {\rm pr}_{2}\langle a \rangle
\end{align*}
が共に成り立つ.
そこで(9), (10)から, 推論法則 \ref{dedaddeqw}によって
\[
\tag{12}
  (x, y) \in a \wedge \exists z((y, z) \in b) 
  \leftrightarrow (y, x) \in a^{-1} \wedge y \in {\rm pr}_{1}\langle b \rangle
\]
が成り立ち, また(11)から, 同じく推論法則 \ref{dedaddeqw}によって
\[
\tag{13}
  \exists x((x, y) \in a) \wedge (y, z) \in b
  \leftrightarrow y \in {\rm pr}_{2}\langle a \rangle \wedge (y, z) \in b
\]
が成り立つ.
またThm \ref{awblbwa}より
\[
\tag{14}
  (y, x) \in a^{-1} \wedge y \in {\rm pr}_{1}\langle b \rangle 
  \leftrightarrow y \in {\rm pr}_{1}\langle b \rangle \wedge (y, x) \in a^{-1}
\]
が成り立つ.
そこで(7), (12), (14)から, 推論法則 \ref{dedeqtrans}によって
\[
  \exists z((x, y) \in a \wedge (y, z) \in b) 
  \leftrightarrow y \in {\rm pr}_{1}\langle b \rangle \wedge (y, x) \in a^{-1}
\]
が成り立ち, (8), (13)から, 同じく推論法則 \ref{dedeqtrans}によって
\[
  \exists x((x, y) \in a \wedge (y, z) \in b) 
  \leftrightarrow y \in {\rm pr}_{2}\langle a \rangle \wedge (y, z) \in b
\]
が成り立つ.
$y$は定数でないので, これらから推論法則 \ref{dedalleqquansepconst}により
\begin{align*}
  \tag{15}
  \exists y(\exists z((x, y) \in a \wedge (y, z) \in b)) 
  &\leftrightarrow \exists y(y \in {\rm pr}_{1}\langle b \rangle \wedge (y, x) \in a^{-1}), \\
  \mbox{} \\
  \tag{16}
  \exists y(\exists x((x, y) \in a \wedge (y, z) \in b)) 
  &\leftrightarrow \exists y(y \in {\rm pr}_{2}\langle a \rangle \wedge (y, z) \in b)
\end{align*}
が共に成り立つ.
またいま$y$は$a$及び$b$の中に自由変数として現れないから, 
変数法則 \ref{valprset}により, 
$y$は${\rm pr}_{1}\langle b \rangle$及び${\rm pr}_{2}\langle a \rangle$の中にも自由変数として現れず, 
変数法則 \ref{valinv}により, $y$は$a^{-1}$の中にも自由変数として現れない.
このことと, $y$が$x$とも$z$とも異なることから, 
定理 \ref{sthmvaluesetelement}と推論法則 \ref{dedeqch}により
\begin{align*}
  \tag{17}
  \exists y(y \in {\rm pr}_{1}\langle b \rangle \wedge (y, x) \in a^{-1}) 
  &\leftrightarrow x \in a^{-1}[{\rm pr}_{1}\langle b \rangle], \\
  \mbox{} \\
  \tag{18}
  \exists y(y \in {\rm pr}_{2}\langle a \rangle \wedge (y, z) \in b) 
  &\leftrightarrow z \in b[{\rm pr}_{2}\langle a \rangle]
\end{align*}
が共に成り立つ.
以上の(1), (3), (5), (15), (17)から, 推論法則 \ref{dedeqtrans}によって
\[
\tag{19}
  x \in {\rm pr}_{1}\langle b \circ a \rangle 
  \leftrightarrow x \in a^{-1}[{\rm pr}_{1}\langle b \rangle]
\]
が成り立ち, 
(2), (4), (6), (16), (18)から, 同じく推論法則 \ref{dedeqtrans}によって
\[
\tag{20}
  z \in {\rm pr}_{2}\langle b \circ a \rangle 
  \leftrightarrow z \in b[{\rm pr}_{2}\langle a \rangle]
\]
が成り立つことがわかる.
いま$x$と$z$は共に$a$及び$b$の中に自由変数として現れないから, 
変数法則 \ref{valprset}, \ref{valvalueset}, \ref{valinv}, \ref{valcomp}によってわかるように, 
$x$は${\rm pr}_{1}\langle b \circ a \rangle$及び$a^{-1}[{\rm pr}_{1}\langle b \rangle]$の中に
自由変数として現れず, 
$z$は${\rm pr}_{2}\langle b \circ a \rangle$及び$b[{\rm pr}_{2}\langle a \rangle]$の中に
自由変数として現れない.
また$x$と$z$は共に定数でない.
そこでこれらのことと, (19), (20)が共に成り立つことから, 
定理 \ref{sthmset=}により
${\rm pr}_{1}\langle b \circ a \rangle = a^{-1}[{\rm pr}_{1}\langle b \rangle]$と
${\rm pr}_{2}\langle b \circ a \rangle = b[{\rm pr}_{2}\langle a \rangle]$が共に成り立つ.
\halmos




\mathstrut
\begin{thm}
\label{sthmprsetcomp2}%定理
$a$と$b$を集合とするとき, 
\begin{align*}
  {\rm pr}_{1}\langle b \circ a \rangle \subset {\rm pr}_{1}\langle a \rangle&, ~~
  {\rm pr}_{2}\langle b \circ a \rangle \subset {\rm pr}_{2}\langle b \rangle, \\
  \mbox{} \\
  {\rm pr}_{2}\langle a \rangle \subset {\rm pr}_{1}\langle b \rangle 
  \to {\rm pr}_{1}\langle b \circ a \rangle = {\rm pr}_{1}\langle a \rangle&, ~~
  {\rm pr}_{1}\langle b \rangle \subset {\rm pr}_{2}\langle a \rangle 
  \to {\rm pr}_{2}\langle b \circ a \rangle = {\rm pr}_{2}\langle b \rangle
\end{align*}
が成り立つ.
またこの後の二つから, 次の($*$)が成り立つ: 

($*$) ~~${\rm pr}_{2}\langle a \rangle \subset {\rm pr}_{1}\langle b \rangle$が成り立つならば, 
        ${\rm pr}_{1}\langle b \circ a \rangle = {\rm pr}_{1}\langle a \rangle$が成り立つ.
        また${\rm pr}_{1}\langle b \rangle \subset {\rm pr}_{2}\langle a \rangle$が成り立つならば, 
        ${\rm pr}_{2}\langle b \circ a \rangle = {\rm pr}_{2}\langle b \rangle$が成り立つ.
\end{thm}


\noindent{\bf 証明}
~まずはじめの二つが定理となることを示す.
定理 \ref{sthmprsetcomp}より
\[
  {\rm pr}_{1}\langle b \circ a \rangle = a^{-1}[{\rm pr}_{1}\langle b \rangle], ~~
  {\rm pr}_{2}\langle b \circ a \rangle = b[{\rm pr}_{2}\langle a \rangle]
\]
が共に成り立つから, 定理 \ref{sthm=tsubseteq}により
\begin{align*}
  \tag{1}
  {\rm pr}_{1}\langle b \circ a \rangle \subset {\rm pr}_{1}\langle a \rangle
  &\leftrightarrow a^{-1}[{\rm pr}_{1}\langle b \rangle] \subset {\rm pr}_{1}\langle a \rangle, \\
  \mbox{} \\
  \tag{2}
  {\rm pr}_{2}\langle b \circ a \rangle \subset {\rm pr}_{2}\langle b \rangle
  &\leftrightarrow b[{\rm pr}_{2}\langle a \rangle] \subset {\rm pr}_{2}\langle b \rangle
\end{align*}
が共に成り立つ.
また定理 \ref{sthmvaluesetsubsetpr2set}より
\begin{align*}
  \tag{3}
  a^{-1}[{\rm pr}_{1}\langle b \rangle] &\subset {\rm pr}_{2}\langle a^{-1} \rangle, \\
  \mbox{} \\
  \tag{4}
  b[{\rm pr}_{2}\langle a \rangle] &\subset {\rm pr}_{2}\langle b \rangle
\end{align*}
が共に成り立つ.
そこで(2), (4)から, 推論法則 \ref{dedeqfund}によって
\[
  {\rm pr}_{2}\langle b \circ a \rangle \subset {\rm pr}_{2}\langle b \rangle
\]
が成り立つ.
また定理 \ref{sthmprsetinv}より
${\rm pr}_{2}\langle a^{-1} \rangle = {\rm pr}_{1}\langle a \rangle$が
成り立つから, 定理 \ref{sthm=tsubseteq}により
\[
\tag{5}
  a^{-1}[{\rm pr}_{1}\langle b \rangle] \subset {\rm pr}_{2}\langle a^{-1} \rangle 
  \leftrightarrow a^{-1}[{\rm pr}_{1}\langle b \rangle] \subset {\rm pr}_{1}\langle a \rangle
\]
が成り立つ.
そこで(3), (5)から, 推論法則 \ref{dedeqfund}によって
\[
  a^{-1}[{\rm pr}_{1}\langle b \rangle] \subset {\rm pr}_{1}\langle a \rangle
\]
が成り立ち, これと(1)から, やはり推論法則 \ref{dedeqfund}によって
\[
  {\rm pr}_{1}\langle b \circ a \rangle \subset {\rm pr}_{1}\langle a \rangle
\]
が成り立つ.

次に後の二つが定理となることを示す.
上述のように
\[
  {\rm pr}_{1}\langle b \circ a \rangle = a^{-1}[{\rm pr}_{1}\langle b \rangle], ~~
  {\rm pr}_{2}\langle b \circ a \rangle = b[{\rm pr}_{2}\langle a \rangle]
\]
が共に成り立つから, 推論法則 \ref{dedaddeq=}により
\begin{align*}
  {\rm pr}_{1}\langle b \circ a \rangle = {\rm pr}_{1}\langle a \rangle 
  &\leftrightarrow a^{-1}[{\rm pr}_{1}\langle b \rangle] = {\rm pr}_{1}\langle a \rangle, \\
  \mbox{} \\
  {\rm pr}_{2}\langle b \circ a \rangle = {\rm pr}_{2}\langle b \rangle
  &\leftrightarrow b[{\rm pr}_{2}\langle a \rangle] = {\rm pr}_{2}\langle b \rangle
\end{align*}
が共に成り立つ.
そこで特に推論法則 \ref{dedequiv}により, 
\begin{align*}
  \tag{6}
  a^{-1}[{\rm pr}_{1}\langle b \rangle] = {\rm pr}_{1}\langle a \rangle 
  &\to {\rm pr}_{1}\langle b \circ a \rangle = {\rm pr}_{1}\langle a \rangle, \\
  \mbox{} \\
  \tag{7}
  b[{\rm pr}_{2}\langle a \rangle] = {\rm pr}_{2}\langle b \rangle 
  &\to {\rm pr}_{2}\langle b \circ a \rangle = {\rm pr}_{2}\langle b \rangle
\end{align*}
が共に成り立つ.
また定理 \ref{sthmvalueset=pr2set}より
\begin{align*}
  \tag{8}
  {\rm pr}_{1}\langle a^{-1} \rangle \subset {\rm pr}_{1}\langle b \rangle 
  &\to a^{-1}[{\rm pr}_{1}\langle b \rangle] = {\rm pr}_{2}\langle a^{-1} \rangle, \\
  \mbox{} \\
  \tag{9}
  {\rm pr}_{1}\langle b \rangle \subset {\rm pr}_{2}\langle a \rangle 
  &\to b[{\rm pr}_{2}\langle a \rangle] = {\rm pr}_{2}\langle b \rangle
\end{align*}
が共に成り立つ.
そこで(9), (7)から, 推論法則 \ref{dedmmp}によって
\[
\tag{10}
  {\rm pr}_{1}\langle b \rangle \subset {\rm pr}_{2}\langle a \rangle 
  \to {\rm pr}_{2}\langle b \circ a \rangle = {\rm pr}_{2}\langle b \rangle
\]
が成り立つ.
また定理 \ref{sthmprsetinv}より
\[
  {\rm pr}_{1}\langle a^{-1} \rangle = {\rm pr}_{2}\langle a \rangle, ~~
  {\rm pr}_{2}\langle a^{-1} \rangle = {\rm pr}_{1}\langle a \rangle
\]
が共に成り立つから, この前者から, 定理 \ref{sthm=tsubseteq}により
\[
  {\rm pr}_{1}\langle a^{-1} \rangle \subset {\rm pr}_{1}\langle b \rangle 
  \leftrightarrow {\rm pr}_{2}\langle a \rangle \subset {\rm pr}_{1}\langle b \rangle
\]
が成り立ち, 後者から, 推論法則 \ref{dedaddeq=}により
\[
  a^{-1}[{\rm pr}_{1}\langle b \rangle] = {\rm pr}_{2}\langle a^{-1} \rangle 
  \leftrightarrow a^{-1}[{\rm pr}_{1}\langle b \rangle] = {\rm pr}_{1}\langle a \rangle
\]
が成り立つ.
そこでこれらから特に, 推論法則 \ref{dedequiv}により
\begin{align*}
  \tag{11}
  {\rm pr}_{2}\langle a \rangle \subset {\rm pr}_{1}\langle b \rangle 
  &\to {\rm pr}_{1}\langle a^{-1} \rangle \subset {\rm pr}_{1}\langle b \rangle , \\
  \mbox{} \\
  \tag{12}
  a^{-1}[{\rm pr}_{1}\langle b \rangle] = {\rm pr}_{2}\langle a^{-1} \rangle 
  &\to a^{-1}[{\rm pr}_{1}\langle b \rangle] = {\rm pr}_{1}\langle a \rangle
\end{align*}
が共に成り立つ.
そこで(11), (8), (12), (6)から, 推論法則 \ref{dedmmp}によって
\[
\tag{13}
  {\rm pr}_{2}\langle a \rangle \subset {\rm pr}_{1}\langle b \rangle 
  \to {\rm pr}_{1}\langle b \circ a \rangle = {\rm pr}_{1}\langle a \rangle
\]
が成り立つことがわかる.

($*$)が成り立つことは, (10), (13)が共に成り立つことから, 推論法則 \ref{dedmp}によって
明らかである.
\halmos




\mathstrut
\begin{thm}
\label{sthmvaluesetcomp}%定理
$a$, $b$, $c$を集合とするとき, 
\[
  (b \circ a)[c] = b[a[c]]
\]
が成り立つ.
\end{thm}


\noindent{\bf 証明}
~$x$, $y$, $z$を, どの二つも互いに異なり, どの一つも$a$, $b$, $c$のいずれの記号列の中にも
自由変数として現れない, 定数でない文字とする.
このとき変数法則 \ref{valcomp}により, $x$は$b \circ a$の中にも自由変数として現れないから, 
定理 \ref{sthmvaluesetelement}より
\[
\tag{1}
  z \in (b \circ a)[c] \leftrightarrow \exists x(x \in c \wedge (x, z) \in b \circ a)
\]
が成り立つ.
また$y$が$x$とも$z$とも異なり, $a$及び$b$の中に自由変数として現れないことから, 
定理 \ref{sthmpairincompeq}より
\[
  (x, z) \in b \circ a \leftrightarrow \exists y((x, y) \in a \wedge (y, z) \in b)
\]
が成り立つ.
そこで推論法則 \ref{dedaddeqw}により
\[
\tag{2}
  x \in c \wedge (x, z) \in b \circ a 
  \leftrightarrow x \in c \wedge \exists y((x, y) \in a \wedge (y, z) \in b)
\]
が成り立つ.
また$y$が$x$と異なり, $c$の中に自由変数として現れないことから, 
変数法則 \ref{valfund}により, $y$は$x \in c$の中に自由変数として現れないから, 
Thm \ref{thmexwrfree}と推論法則 \ref{dedeqch}により
\[
\tag{3}
  x \in c \wedge \exists y((x, y) \in a \wedge (y, z) \in b) 
  \leftrightarrow \exists y(x \in c \wedge ((x, y) \in a \wedge (y, z) \in b))
\]
が成り立つ.
またThm \ref{1awb1wclaw1bwc1}と推論法則 \ref{dedeqch}により
\[
  x \in c \wedge ((x, y) \in a \wedge (y, z) \in b) 
  \leftrightarrow (x \in c \wedge (x, y) \in a) \wedge (y, z) \in b
\]
が成り立つ.
$y$は定数でないので, これから推論法則 \ref{dedalleqquansepconst}により
\[
\tag{4}
  \exists y(x \in c \wedge ((x, y) \in a \wedge (y, z) \in b)) 
  \leftrightarrow \exists y((x \in c \wedge (x, y) \in a) \wedge (y, z) \in b)
\]
が成り立つ.
そこで(2), (3), (4)から, 推論法則 \ref{dedeqtrans}によって
\[
  x \in c \wedge (x, z) \in b \circ a 
  \leftrightarrow \exists y((x \in c \wedge (x, y) \in a) \wedge (y, z) \in b)
\]
が成り立つことがわかる.
$x$は定数でないので, これから推論法則 \ref{dedalleqquansepconst}により
\[
\tag{5}
  \exists x(x \in c \wedge (x, z) \in b \circ a) 
  \leftrightarrow \exists x(\exists y((x \in c \wedge (x, y) \in a) \wedge (y, z) \in b))
\]
が成り立つ.
またThm \ref{thmexch}より
\[
\tag{6}
  \exists x(\exists y((x \in c \wedge (x, y) \in a) \wedge (y, z) \in b)) 
  \leftrightarrow \exists y(\exists x((x \in c \wedge (x, y) \in a) \wedge (y, z) \in b))
\]
が成り立つ.
また$x$が$y$とも$z$とも異なり, $b$の中に自由変数として現れないことから, 
変数法則 \ref{valfund}, \ref{valpair}により, 
$x$は$(y, z) \in b$の中に自由変数として現れないから, Thm \ref{thmexwrfree}より
\[
\tag{7}
  \exists x((x \in c \wedge (x, y) \in a) \wedge (y, z) \in b) 
  \leftrightarrow \exists x(x \in c \wedge (x, y) \in a) \wedge (y, z) \in b
\]
が成り立つ.
また$x$が$y$と異なり, $a$及び$c$の中に自由変数として現れないことから, 
定理 \ref{sthmvaluesetelement}と推論法則 \ref{dedeqch}により
\[
  \exists x(x \in c \wedge (x, y) \in a) \leftrightarrow y \in a[c]
\]
が成り立つ.
そこで推論法則 \ref{dedaddeqw}により
\[
\tag{8}
  \exists x(x \in c \wedge (x, y) \in a) \wedge (y, z) \in b 
  \leftrightarrow y \in a[c] \wedge (y, z) \in b
\]
が成り立つ.
そこで(7), (8)から, 推論法則 \ref{dedeqtrans}によって
\[
  \exists x((x \in c \wedge (x, y) \in a) \wedge (y, z) \in b) 
  \leftrightarrow y \in a[c] \wedge (y, z) \in b
\]
が成り立つ.
$y$は定数でないので, これから推論法則 \ref{dedalleqquansepconst}により
\[
\tag{9}
  \exists y(\exists x((x \in c \wedge (x, y) \in a) \wedge (y, z) \in b)) 
  \leftrightarrow \exists y(y \in a[c] \wedge (y, z) \in b)
\]
が成り立つ.
また$y$は$a$及び$c$の中に自由変数として現れないから, 
変数法則 \ref{valvalueset}により, $y$は$a[c]$の中に自由変数として現れない.
このことと, $y$が$z$と異なり, $b$の中にも自由変数として現れないことから, 
定理 \ref{sthmvaluesetelement}と推論法則 \ref{dedeqch}により
\[
\tag{10}
  \exists y(y \in a[c] \wedge (y, z) \in b) \leftrightarrow z \in b[a[c]]
\]
が成り立つ.
以上の(1), (5), (6), (9), (10)から, 推論法則 \ref{dedeqtrans}によって
\[
\tag{11}
  z \in (b \circ a)[c] \leftrightarrow z \in b[a[c]]
\]
が成り立つことがわかる.
いま$z$は$a$, $b$, $c$のいずれの記号列の中にも自由変数として現れないから, 
変数法則 \ref{valvalueset}, \ref{valcomp}によってわかるように, 
$z$は$(b \circ a)[c]$及び$b[a[c]]$の中に自由変数として現れない.
また$z$は定数でない.
これらのことと, (11)が成り立つことから, 
定理 \ref{sthmset=}により
$(b \circ a)[c] = b[a[c]]$が成り立つ.
\halmos




\mathstrut
\begin{thm}
\label{sthmcompinv}%定理
$a$と$b$を集合とするとき, 
\[
  (b \circ a)^{-1} = a^{-1} \circ b^{-1}
\]
が成り立つ.
\end{thm}


\noindent{\bf 証明}
~$x$, $y$, $z$を, どの二つも互いに異なり, いずれも$a$及び$b$の中に自由変数として現れない, 
定数でない文字とする.
このとき定理 \ref{sthmpairininv}より
\[
\tag{1}
  (x, z) \in (b \circ a)^{-1} \leftrightarrow (z, x) \in b \circ a
\]
が成り立つ.
また$y$が$x$とも$z$とも異なり, $a$及び$b$の中に自由変数として現れないことから, 
定理 \ref{sthmpairincompeq}より
\[
\tag{2}
  (z, x) \in b \circ a \leftrightarrow \exists y((z, y) \in a \wedge (y, x) \in b)
\]
が成り立つ.
また定理 \ref{sthmpairininv}と推論法則 \ref{dedeqch}により
\[
  (z, y) \in a \leftrightarrow (y, z) \in a^{-1}, ~~
  (y, x) \in b \leftrightarrow (x, y) \in b^{-1}
\]
が共に成り立つから, 推論法則 \ref{dedaddeqw}により
\[
\tag{3}
  (z, y) \in a \wedge (y, x) \in b \leftrightarrow (y, z) \in a^{-1} \wedge (x, y) \in b^{-1}
\]
が成り立つ.
またThm \ref{awblbwa}より
\[
\tag{4}
  (y, z) \in a^{-1} \wedge (x, y) \in b^{-1} \leftrightarrow (x, y) \in b^{-1} \wedge (y, z) \in a^{-1}
\]
が成り立つ.
そこで(3), (4)から, 推論法則 \ref{dedeqtrans}によって
\[
  (z, y) \in a \wedge (y, x) \in b \leftrightarrow (x, y) \in b^{-1} \wedge (y, z) \in a^{-1}
\]
が成り立つ.
いま$y$は定数でないので, これから推論法則 \ref{dedalleqquansepconst}によって
\[
\tag{5}
  \exists y((z, y) \in a \wedge (y, x) \in b) \leftrightarrow \exists y((x, y) \in b^{-1} \wedge (y, z) \in a^{-1})
\]
が成り立つ.
また$y$は$a$及び$b$の中に自由変数として現れないから, 変数法則 \ref{valinv}により, 
$y$は$a^{-1}$及び$b^{-1}$の中に自由変数として現れない.
このことと, $y$が$x$とも$z$とも異なることから, 定理 \ref{sthmpairincompeq}と推論法則 \ref{dedeqch}により
\[
\tag{6}
  \exists y((x, y) \in b^{-1} \wedge (y, z) \in a^{-1}) \leftrightarrow (x, z) \in a^{-1} \circ b^{-1}
\]
が成り立つ.
そこで(1), (2), (5), (6)から, 推論法則 \ref{dedeqtrans}によって
\[
\tag{7}
  (x, z) \in (b \circ a)^{-1} \leftrightarrow (x, z) \in a^{-1} \circ b^{-1}
\]
が成り立つことがわかる.
さていま定理 \ref{sthminvgraph}, \ref{sthmcompgraph}により, 
$(b \circ a)^{-1}$と$a^{-1} \circ b^{-1}$は共にグラフである.
また$x$と$z$は共に$a$及び$b$の中に自由変数として現れないから, 
変数法則 \ref{valinv}, \ref{valcomp}により, 
これらは共に$(b \circ a)^{-1}$及び$a^{-1} \circ b^{-1}$の中に自由変数として現れない.
また$x$と$z$は互いに異なり, 共に定数でない.
そこでこれらのことと, (7)が成り立つことから, 
定理 \ref{sthmgraphpair=}によって
$(b \circ a)^{-1} = a^{-1} \circ b^{-1}$が成り立つ.
\halmos
%[合成]確認済



\mathstrut
\begin{defo}
\label{identity}%変形
$a$を記号列とする.
また$x$と$y$を共に$a$の中に自由変数として現れない文字とする.
このとき
\[
  \{(x, x)|x \in a\} \equiv \{(y, y)|y \in a\}
\]
が成り立つ.
\end{defo}


\noindent{\bf 証明}
~$x$と$y$が同じ文字である場合には明らかだから, 
$x$と$y$が異なる文字である場合を考える.
このとき変数法則 \ref{valpair}により, 
$y$は$(x, x)$の中に自由変数として現れない.
このことと, $x$と$y$が共に$a$の中に自由変数として現れないことから, 
代入法則 \ref{substosettrans}により
\[
  \{(x, x)|x \in a\} \equiv \{(y|x)((x, x))|y \in a\}
\]
が成り立つ.
代入法則 \ref{substpair}により$(y|x)((x, x))$は$(y, y)$と一致するから, 
従って本法則が成り立つ.
\halmos




\mathstrut
\begin{defi}
\label{defiden}%定義
$a$を記号列とする.
また$x$と$y$を共に$a$の中に自由変数として現れない文字とする.
このとき上記の変形法則 \ref{identity}によれば, 
$\{(x, x)|x \in a\}$と$\{(y, y)|y \in a\}$という二つの記号列は一致する.
$a$に対して定まるこの記号列を, ${\rm id}_{a}$と書き表す.
\end{defi}




\mathstrut
\begin{valu}
\label{validen}%変数
$a$を記号列とし, $x$を文字とする.
$x$が$a$の中に自由変数として現れなければ, 
$x$は${\rm id}_{a}$の中に自由変数として現れない.
\end{valu}


\noindent{\bf 証明}
~このとき定義から, ${\rm id}_{a}$は$\{(x, x)|x \in a\}$と同じである.
変数法則 \ref{valoset}により, $x$はこの中に自由変数として現れない.
\halmos




\mathstrut
\begin{subs}
\label{substiden}%代入
$a$と$b$を記号列とし, $x$を文字とするとき, 
\[
  (b|x)({\rm id}_{a}) \equiv {\rm id}_{(b|x)(a)}
\]
が成り立つ.
\end{subs}


\noindent{\bf 証明}
~$y$を$x$と異なり, $a$及び$b$の中に自由変数として現れない文字とする.
このとき定義から, ${\rm id}_{a}$は$\{(y, y)|y \in a\}$と同じである.
このことと, $y$が$x$と異なり, $b$の中に自由変数として現れないことから, 
代入法則 \ref{substoset}により, $(b|x)({\rm id}_{a})$は
$\{(b|x)((y, y))|y \in (b|x)(a)\}$と一致し, 代入法則 \ref{substpair}により, 
これは$\{(y, y)|y \in (b|x)(a)\}$と一致する.
また$y$が$a$及び$b$の中に自由変数として現れないことから, 
変数法則 \ref{valsubst}により, $y$は$(b|x)(a)$の中に自由変数として現れないから, 
定義から$\{(y, y)|y \in (b|x)(a)\}$は${\rm id}_{(b|x)(a)}$と同じである.
以上のことからわかるように, $(b|x)({\rm id}_{a})$は${\rm id}_{(b|x)(a)}$と一致する.
\halmos




\mathstrut
\begin{form}
\label{formiden}%構成
$a$が集合ならば, ${\rm id}_{a}$は集合である.
\end{form}


\noindent{\bf 証明}
~$x$を$a$の中に自由変数として現れない文字とすれば, 定義から${\rm id}_{a}$は
$\{(x, x)|x \in a\}$と同じである.
$a$が集合ならば, 構成法則 \ref{formfund}, \ref{formoset}, \ref{formpair}から直ちにわかるように, 
これは集合である.
\halmos




\mathstrut
\begin{thm}
\label{sthmidenelement}%定理
$a$と$b$を集合とするとき, 
\[
  b \in {\rm id}_{a} \leftrightarrow b = ({\rm pr}_{1}(b), {\rm pr}_{1}(b)) \wedge {\rm pr}_{1}(b) \in a
\]
が成り立つ.
\end{thm}


\noindent{\bf 証明}
~$x$を$a$及び$b$の中に自由変数として現れない文字とする.
このとき定義から, ${\rm id}_{a}$は$\{(x, x)|x \in a\}$と同じだから, 
定理 \ref{sthmosetbasis}より
\[
\tag{1}
  b \in {\rm id}_{a} \leftrightarrow \exists x(x \in a \wedge b = (x, x))
\]
が成り立つ.
ここで$\tau_{x}(x \in a \wedge b = (x, x))$を$T$と書けば, $T$は集合であり, 
定義から$\exists x(x \in a \wedge b = (x, x))$は
$(T|x)(x \in a \wedge b = (x, x))$と同じである.
また$x$が$a$及び$b$の中に自由変数として現れないことから, 
代入法則 \ref{substfree}, \ref{substfund}, \ref{substwedge}, \ref{substpair}により, 
この記号列は$T \in a \wedge b = (T, T)$と一致する.
以上のことから, 
\[
\tag{2}
  \exists x(x \in a \wedge b = (x, x)) \equiv T \in a \wedge b = (T, T)
\]
が成り立つことがわかる.
いまThm \ref{awbta}より
\[
  T \in a \wedge b = (T, T) \to b = (T, T)
\]
が成り立つから, 従って(2)により, 
\[
  \tag{3}
  \exists x(x \in a \wedge b = (x, x)) \to b = (T, T)
\]
が定理となる.
また定理 \ref{sthmpairpreq}と推論法則 \ref{dedequiv}により
\[
\tag{4}
  b = (T, T) \to 
  {\rm Pair}(b) \wedge (T = {\rm pr}_{1}(b) \wedge T = {\rm pr}_{2}(b))
\]
が成り立つ.
そこで特に, 推論法則 \ref{dedprewedge}によって
\[
  b = (T, T) \to T = {\rm pr}_{1}(b)
\]
が成り立つことがわかり, これから推論法則 \ref{dedaddw}によって
\[
\tag{5}
  T \in a \wedge b = (T, T) \to T \in a \wedge T = {\rm pr}_{1}(b)
\]
が成り立つ.
またThm \ref{awbtbwa}より
\[
\tag{6}
  T \in a \wedge T = {\rm pr}_{1}(b) \to T = {\rm pr}_{1}(b) \wedge T \in a
\]
が成り立つ.
また定理 \ref{sthm=&in}より
\[
\tag{7}
  T = {\rm pr}_{1}(b) \wedge T \in a \to {\rm pr}_{1}(b) \in a
\]
が成り立つ.
そこで(5), (6), (7)から, 推論法則 \ref{dedmmp}によって
\[
  T \in a \wedge b = (T, T) \to {\rm pr}_{1}(b) \in a
\]
が成り立つことがわかる.
(2)によれば, この記号列は
\[
\tag{8}
  \exists x(x \in a \wedge b = (x, x)) \to {\rm pr}_{1}(b) \in a
\]
と一致するから, 従ってこれが定理となる.
またThm \ref{awbtbwa}より
\[
\tag{9}
  T = {\rm pr}_{1}(b) \wedge T = {\rm pr}_{2}(b) 
  \to T = {\rm pr}_{2}(b) \wedge T = {\rm pr}_{1}(b)
\]
が成り立つ.
またThm \ref{x=yty=x}より
$T = {\rm pr}_{2}(b) \to {\rm pr}_{2}(b) = T$が成り立つから, 
推論法則 \ref{dedaddw}により
\[
\tag{10}
  T = {\rm pr}_{2}(b) \wedge T = {\rm pr}_{1}(b) 
  \to {\rm pr}_{2}(b) = T \wedge T = {\rm pr}_{1}(b)
\]
が成り立つ.
またThm \ref{x=ywy=ztx=z}より
\[
\tag{11}
  {\rm pr}_{2}(b) = T \wedge T = {\rm pr}_{1}(b) 
  \to {\rm pr}_{2}(b) = {\rm pr}_{1}(b)
\]
が成り立つ.
また定理 \ref{sthmpairweak}と推論法則 \ref{dedequiv}により
\[
\tag{12}
  {\rm pr}_{2}(b) = {\rm pr}_{1}(b) 
  \to ({\rm pr}_{1}(b), {\rm pr}_{2}(b)) = ({\rm pr}_{1}(b), {\rm pr}_{1}(b))
\]
が成り立つ.
そこで(9)---(12)から, 推論法則 \ref{dedmmp}によって
\[
\tag{13}
  T = {\rm pr}_{1}(b) \wedge T = {\rm pr}_{2}(b) 
  \to ({\rm pr}_{1}(b), {\rm pr}_{2}(b)) = ({\rm pr}_{1}(b), {\rm pr}_{1}(b))
\]
が成り立つことがわかる.
また定理 \ref{sthmbigpairpr}と推論法則 \ref{dedequiv}により
\[
\tag{14}
  {\rm Pair}(b) \to b = ({\rm pr}_{1}(b), {\rm pr}_{2}(b))
\]
が成り立つ.
そこで(13), (14)から, 推論法則 \ref{dedfromaddw}によって
\[
\tag{15}
  {\rm Pair}(b) \wedge (T = {\rm pr}_{1}(b) \wedge T = {\rm pr}_{2}(b)) 
  \to b = ({\rm pr}_{1}(b), {\rm pr}_{2}(b)) \wedge ({\rm pr}_{1}(b), {\rm pr}_{2}(b)) = ({\rm pr}_{1}(b), {\rm pr}_{1}(b))
\]
が成り立つ.
またThm \ref{x=ywy=ztx=z}より
\[
\tag{16}
  b = ({\rm pr}_{1}(b), {\rm pr}_{2}(b)) \wedge ({\rm pr}_{1}(b), {\rm pr}_{2}(b)) = ({\rm pr}_{1}(b), {\rm pr}_{1}(b)) 
  \to b = ({\rm pr}_{1}(b), {\rm pr}_{1}(b))
\]
が成り立つ.
そこで(3), (4), (15), (16)から, 推論法則 \ref{dedmmp}によって
\[
\tag{17}
  \exists x(x \in a \wedge b = (x, x)) \to b = ({\rm pr}_{1}(b), {\rm pr}_{1}(b))
\]
が成り立つことがわかる.
故に(17), (8)から, 推論法則 \ref{dedprewedge}によって
\[
\tag{18}
  \exists x(x \in a \wedge b = (x, x)) \to b = ({\rm pr}_{1}(b), {\rm pr}_{1}(b)) \wedge {\rm pr}_{1}(b) \in a
\]
が成り立つ.
またThm \ref{awbtbwa}より
\[
  b = ({\rm pr}_{1}(b), {\rm pr}_{1}(b)) \wedge {\rm pr}_{1}(b) \in a 
  \to {\rm pr}_{1}(b) \in a \wedge b = ({\rm pr}_{1}(b), {\rm pr}_{1}(b))
\]
が成り立つ.
ここで$x$が$a$及び$b$の中に自由変数として現れないことから, 
代入法則 \ref{substfree}, \ref{substfund}, \ref{substwedge}, \ref{substpair}により, 
上記の記号列は
\[
\tag{19}
  b = ({\rm pr}_{1}(b), {\rm pr}_{1}(b)) \wedge {\rm pr}_{1}(b) \in a 
  \to ({\rm pr}_{1}(b)|x)(x \in a \wedge b = (x, x))
\]
と一致する.
よってこれが定理となる.
またschema S4の適用により
\[
\tag{20}
  ({\rm pr}_{1}(b)|x)(x \in a \wedge b = (x, x)) 
  \to \exists x(x \in a \wedge b = (x, x))
\]
が成り立つ.
そこで(19), (20)から, 推論法則 \ref{dedmmp}によって
\[
\tag{21}
  b = ({\rm pr}_{1}(b), {\rm pr}_{1}(b)) \wedge {\rm pr}_{1}(b) \in a \to \exists x(x \in a \wedge b = (x, x))
\]
が成り立つ.
故に(18), (21)から, 推論法則 \ref{dedequiv}によって
\[
\tag{22}
  \exists x(x \in a \wedge b = (x, x)) \leftrightarrow b = ({\rm pr}_{1}(b), {\rm pr}_{1}(b)) \wedge {\rm pr}_{1}(b) \in a
\]
が成り立つことがわかり, 
(1)とこの(22)から, 推論法則 \ref{dedeqtrans}によって
\[
  b \in {\rm id}_{a} \leftrightarrow b = ({\rm pr}_{1}(b), {\rm pr}_{1}(b)) \wedge {\rm pr}_{1}(b) \in a
\]
が成り立つ.
\halmos




\mathstrut
\begin{thm}
\label{sthmpairiniden}%定理
$a$, $b$, $c$を集合とするとき, 
\[
  (b, c) \in {\rm id}_{a} \leftrightarrow b = c \wedge b \in a, ~~
  (b, c) \in {\rm id}_{a} \leftrightarrow b = c \wedge c \in a
\]
が成り立つ.
またこれらから, 次の($*$)が成り立つ: 

($*$) ~~$(b, c) \in {\rm id}_{a}$が成り立つならば, $b = c$, $b \in a$, $c \in a$がすべて成り立つ.
        また$b = c$と$b \in a$が共に成り立つならば, $(b, c) \in {\rm id}_{a}$が成り立つ.
        また$b = c$と$c \in a$が共に成り立つならば, $(b, c) \in {\rm id}_{a}$が成り立つ.
\end{thm}


\noindent{\bf 証明}
~定理 \ref{sthmidenelement}より
\[
\tag{1}
  (b, c) \in {\rm id}_{a} \leftrightarrow (b, c) = ({\rm pr}_{1}((b, c)), {\rm pr}_{1}((b, c))) \wedge {\rm pr}_{1}((b, c)) \in a
\]
が成り立つ.
また定理 \ref{sthmprpair}より${\rm pr}_{1}((b, c)) = b$が成り立つから, 
定理 \ref{sthmpair}により
\[
\tag{2}
  ({\rm pr}_{1}((b, c)), {\rm pr}_{1}((b, c))) = (b, b)
\]
が成り立ち, 定理 \ref{sthm=tineq}により
\[
\tag{3}
  {\rm pr}_{1}((b, c)) \in a \leftrightarrow b \in a
\]
が成り立つ.
そこで(2)から, 推論法則 \ref{dedaddeq=}により
\[
\tag{4}
  (b, c) = ({\rm pr}_{1}((b, c)), {\rm pr}_{1}((b, c))) \leftrightarrow (b, c) = (b, b)
\]
が成り立つ.
また定理 \ref{sthmpairweak}と推論法則 \ref{dedeqch}により
\[
\tag{5}
  (b, c) = (b, b) \leftrightarrow c = b
\]
が成り立つ.
またThm \ref{x=yly=x}より
\[
\tag{6}
  c = b \leftrightarrow b = c
\]
が成り立つ.
そこで(4), (5), (6)から, 推論法則 \ref{dedeqtrans}によって
\[
  (b, c) = ({\rm pr}_{1}((b, c)), {\rm pr}_{1}((b, c))) \leftrightarrow b = c
\]
が成り立つことがわかり, これと(3)から, 推論法則 \ref{dedaddeqw}により
\[
\tag{8}
  (b, c) = ({\rm pr}_{1}((b, c)), {\rm pr}_{1}((b, c))) \wedge {\rm pr}_{1}((b, c)) \in a 
  \leftrightarrow b = c \wedge b \in a
\]
が成り立つ.
故に(1), (8)から, 推論法則 \ref{dedeqtrans}によって
\[
\tag{9}
  (b, c) \in {\rm id}_{a} \leftrightarrow b = c \wedge b \in a
\]
が成り立つ.
また$x$を$a$の中に自由変数として現れない文字とすれば, 
Thm \ref{thmfroms5eq}より
\[
  b = c \wedge (b|x)(x \in a) \leftrightarrow b = c \wedge (c|x)(x \in a)
\]
が成り立つが, 代入法則 \ref{substfree}, \ref{substfund}によれば, この記号列は
\[
\tag{10}
  b = c \wedge b \in a \leftrightarrow b = c \wedge c \in a
\]
と一致するから, これが定理となる.
そこで(9), (10)から, 推論法則 \ref{dedeqtrans}によって
\[
\tag{11}
  (b, c) \in {\rm id}_{a} \leftrightarrow b = c \wedge c \in a
\]
が成り立つ.

さていま$(b, c) \in {\rm id}_{a}$が成り立つとする.
このとき(9), (11)と推論法則 \ref{dedeqfund}によって
$b = c \wedge b \in a$と$b = c \wedge c \in a$が共に成り立つから, 
推論法則 \ref{dedwedge}によって$b = c$, $b \in a$, $c \in a$がすべて成り立つことがわかる.
また$b = c$と$b \in a$が共に成り立つならば, 推論法則 \ref{dedwedge}によって
$b = c \wedge b \in a$が成り立つから, これと(9)から推論法則 \ref{dedeqfund}によって
$(b, c) \in {\rm id}_{a}$が成り立つ.
また$b = c$と$c \in a$が共に成り立つならば, 同様に推論法則 \ref{dedwedge}によって
$b = c \wedge c \in a$が成り立ち, これと(11)から, 推論法則 \ref{dedeqfund}によって
$(b, c) \in {\rm id}_{a}$が成り立つ.
以上で($*$)も成り立つことが示された.
\halmos




\mathstrut
\begin{thm}
\label{sthmpairiniden2}%定理
$a$と$b$を集合とするとき, 
\[
  (b, b) \in {\rm id}_{a} \leftrightarrow b \in a
\]
が成り立つ.
またこのことから, 特に次の($*$)が成り立つ: 

($*$) ~~$b \in a$が成り立つならば, $(b, b) \in {\rm id}_{a}$が成り立つ.
\end{thm}


\noindent{\bf 証明}
~定理 \ref{sthmpairiniden}より
\[
  (b, b) \in {\rm id}_{a} \leftrightarrow b = b \wedge b \in a
\]
が成り立つ.
またThm \ref{x=x}より$b = b$が成り立つから, 
推論法則 \ref{dedawblatrue2}により
\[
  b = b \wedge b \in a \leftrightarrow b \in a
\]
が成り立つ.
そこでこれらから, 推論法則 \ref{dedeqtrans}によって
$(b, b) \in {\rm id}_{a} \leftrightarrow b \in a$が成り立つ.
($*$)が成り立つことは, これと推論法則 \ref{dedeqfund}によって明らかである.
\halmos




\mathstrut
\begin{thm}
\label{sthmidengraph}%定理
$a$を集合とするとき, ${\rm id}_{a}$はグラフである.
また
\[
  {\rm id}_{a} \subset a \times a
\]
が成り立つ.
\end{thm}


\noindent{\bf 証明}
~$x$を$a$の中に自由変数として現れない, 定数でない文字とする.
このとき変数法則 \ref{valproduct}, \ref{validen}により, 
$x$は${\rm id}_{a}$及び$a \times a$の中に自由変数として現れない.
また定理 \ref{sthmidenelement}と推論法則 \ref{dedequiv}により
\[
\tag{1}
  x \in {\rm id}_{a} \to x = ({\rm pr}_{1}(x), {\rm pr}_{1}(x)) \wedge {\rm pr}_{1}(x) \in a
\]
が成り立つ.
またThm \ref{atawa}より
\[
\tag{2}
  {\rm pr}_{1}(x) \in a 
  \to {\rm pr}_{1}(x) \in a \wedge {\rm pr}_{1}(x) \in a
\]
が成り立つ.
また定理 \ref{sthmpairinproduct}と推論法則 \ref{dedequiv}により
\[
\tag{3}
  {\rm pr}_{1}(x) \in a \wedge {\rm pr}_{1}(x) \in a 
  \to ({\rm pr}_{1}(x), {\rm pr}_{1}(x)) \in a \times a
\]
が成り立つ.
そこで(2), (3)から, 推論法則 \ref{dedmmp}によって
\[
  {\rm pr}_{1}(x) \in a \to ({\rm pr}_{1}(x), {\rm pr}_{1}(x)) \in a \times a
\]
が成り立ち, これから推論法則 \ref{dedaddw}によって
\[
\tag{4}
  x = ({\rm pr}_{1}(x), {\rm pr}_{1}(x)) \wedge {\rm pr}_{1}(x) \in a 
  \to x = ({\rm pr}_{1}(x), {\rm pr}_{1}(x)) \wedge ({\rm pr}_{1}(x), {\rm pr}_{1}(x)) \in a \times a
\]
が成り立つ.
また定理 \ref{sthm=&in}より
\[
\tag{5}
  x = ({\rm pr}_{1}(x), {\rm pr}_{1}(x)) \wedge ({\rm pr}_{1}(x), {\rm pr}_{1}(x)) \in a \times a 
  \to x \in a \times a
\]
が成り立つ.
そこで(1), (4), (5)から, 推論法則 \ref{dedmmp}によって
\[
\tag{6}
  x \in {\rm id}_{a} \to x \in a \times a
\]
が成り立つことがわかる.
上述のように, $x$は定数でなく, ${\rm id}_{a}$及び$a \times a$の中に自由変数として現れないから, 
このことと(6)が成り立つことから, 定理 \ref{sthmsubsetconst}によって
${\rm id}_{a} \subset a \times a$が成り立つ.
故に定理 \ref{sthmproductsubsetgraph}により, ${\rm id}_{a}$はグラフである.
\halmos




\mathstrut
\begin{thm}
\label{sthmidensubset}%定理
$a$と$b$を集合とするとき, 
\[
  a \subset b \leftrightarrow {\rm id}_{a} \subset {\rm id}_{b}
\]
が成り立つ.
またこのことから, 次の($*$)が成り立つ: 

($*$) ~~$a \subset b$が成り立つならば, ${\rm id}_{a} \subset {\rm id}_{b}$が成り立つ.
        逆に${\rm id}_{a} \subset {\rm id}_{b}$が成り立つならば, $a \subset b$が成り立つ.
\end{thm}


\noindent{\bf 証明}
~$x$と$y$を互いに異なり, 共に$a$及び$b$の中に自由変数として現れない, 
定数でない文字とする.
このとき変数法則 \ref{validen}により, $x$と$y$は共に${\rm id}_{a}$及び${\rm id}_{b}$の中に
自由変数として現れない.
また定理 \ref{sthmsubsetbasis}より
\[
\tag{1}
  a \subset b \to (x \in a \to x \in b)
\]
が成り立つ.
またThm \ref{1atb1t1awctbwc1}より
\[
\tag{2}
  (x \in a \to x \in b) \to (x = y \wedge x \in a \to x = y \wedge x \in b)
\]
が成り立つ.
また定理 \ref{sthmpairiniden}と推論法則 \ref{dedequiv}により, 
\[
  (x, y) \in {\rm id}_{a} \to x = y \wedge x \in a, ~~
  x = y \wedge x \in b \to (x, y) \in {\rm id}_{b}
\]
が共に成り立つから, この前者から, 推論法則 \ref{dedaddf}によって
\[
\tag{3}
  (x = y \wedge x \in a \to x = y \wedge x \in b) \to ((x, y) \in {\rm id}_{a} \to x = y \wedge x \in b)
\]
が成り立ち, 後者から, 推論法則 \ref{dedaddb}によって
\[
\tag{4}
  ((x, y) \in {\rm id}_{a} \to x = y \wedge x \in b) \to ((x, y) \in {\rm id}_{a} \to (x, y) \in {\rm id}_{b})
\]
が成り立つ.
そこで(1)---(4)から, 推論法則 \ref{dedmmp}によって
\[
\tag{5}
  a \subset b \to ((x, y) \in {\rm id}_{a} \to (x, y) \in {\rm id}_{b})
\]
が成り立つことがわかる.
いま$x$と$y$は共に$a$及び$b$の中に自由変数として現れないから, 
変数法則 \ref{valsubset}により, これらは共に$a \subset b$の中に自由変数として現れない.
また$x$と$y$は共に定数でない.
そこでこれらのことと, (5)が成り立つことから, 推論法則 \ref{dedalltquansepfreeconst}によって
\[
\tag{6}
  a \subset b \to \forall x(\forall y((x, y) \in {\rm id}_{a} \to (x, y) \in {\rm id}_{b}))
\]
が成り立つことがわかる.
またいま定理 \ref{sthmidengraph}より, ${\rm id}_{a}$はグラフである.
また$x$と$y$は互いに異なり, 上述のように共に${\rm id}_{a}$及び${\rm id}_{b}$の中に自由変数として現れない.
そこでこれらのことから, 定理 \ref{sthmgraphpairsubset}により
\[
  {\rm id}_{a} \subset {\rm id}_{b} \leftrightarrow \forall x(\forall y((x, y) \in {\rm id}_{a} \to (x, y) \in {\rm id}_{b}))
\]
が成り立つ.
故に推論法則 \ref{dedequiv}により
\[
\tag{7}
  \forall x(\forall y((x, y) \in {\rm id}_{a} \to (x, y) \in {\rm id}_{b})) \to {\rm id}_{a} \subset {\rm id}_{b}
\]
が成り立つ.
そこで(6), (7)から, 推論法則 \ref{dedmmp}によって
\[
\tag{8}
  a \subset b \to {\rm id}_{a} \subset {\rm id}_{b}
\]
が成り立つ.
また定理 \ref{sthmsubsetbasis}より
\[
\tag{9}
  {\rm id}_{a} \subset {\rm id}_{b} \to ((x, x) \in {\rm id}_{a} \to (x, x) \in {\rm id}_{b})
\]
が成り立つ.
また定理 \ref{sthmpairiniden2}と推論法則 \ref{dedequiv}により
\[
  x \in a \to (x, x) \in {\rm id}_{a}, ~~
  (x, x) \in {\rm id}_{b} \to x \in b
\]
が共に成り立つから, この前者から, 推論法則 \ref{dedaddf}によって
\[
\tag{10}
  ((x, x) \in {\rm id}_{a} \to (x, x) \in {\rm id}_{b}) \to (x \in a \to (x, x) \in {\rm id}_{b})
\]
が成り立ち, 後者から, 推論法則 \ref{dedaddb}によって
\[
\tag{11}
  (x \in a \to (x, x) \in {\rm id}_{b}) \to (x \in a \to x \in b)
\]
が成り立つ.
そこで(9), (10), (11)から, 推論法則 \ref{dedmmp}によって
\[
\tag{12}
  {\rm id}_{a} \subset {\rm id}_{b} \to (x \in a \to x \in b)
\]
が成り立つことがわかる.
ここで上述のように$x$が${\rm id}_{a}$及び${\rm id}_{b}$の中に自由変数として現れないことから, 
変数法則 \ref{valsubset}により, $x$は${\rm id}_{a} \subset {\rm id}_{b}$の中に
自由変数として現れない.
また$x$は定数でない.
これらのことと, (12)が成り立つことから, 推論法則 \ref{dedalltquansepfreeconst}により
\[
  {\rm id}_{a} \subset {\rm id}_{b} \to \forall x(x \in a \to x \in b)
\]
が成り立つ.
$x$は$a$及び$b$の中に自由変数として現れないから, 
定義からこの記号列は
\[
\tag{13}
  {\rm id}_{a} \subset {\rm id}_{b} \to a \subset b
\]
と同じである.故にこれが定理となる.
(8), (13)から, 推論法則 \ref{dedequiv}によって
$a \subset b \leftrightarrow {\rm id}_{a} \subset {\rm id}_{b}$が成り立つ.
($*$)が成り立つことは, これと推論法則 \ref{dedeqfund}によって明らかである.
\halmos




\mathstrut
\begin{thm}
\label{sthmiden=}%定理
$a$と$b$を集合とするとき, 
\[
  a = b \leftrightarrow {\rm id}_{a} = {\rm id}_{b}
\]
が成り立つ.
またこのことから, 次の($*$)が成り立つ: 

($*$) ~~$a = b$が成り立つならば, ${\rm id}_{a} = {\rm id}_{b}$が成り立つ.
        逆に${\rm id}_{a} = {\rm id}_{b}$が成り立つならば, $a = b$が成り立つ.
\end{thm}


\noindent{\bf 証明}
~定理 \ref{sthmaxiom1}と推論法則 \ref{dedeqch}により
\[
\tag{1}
  a = b \leftrightarrow a \subset b \wedge b \subset a
\]
が成り立つ.
また定理 \ref{sthmidensubset}より
\[
  a \subset b \leftrightarrow {\rm id}_{a} \subset {\rm id}_{b}, ~~
  b \subset a \leftrightarrow {\rm id}_{b} \subset {\rm id}_{a}
\]
が共に成り立つから, 推論法則 \ref{dedaddeqw}により
\[
\tag{2}
  a \subset b \wedge b \subset a \leftrightarrow {\rm id}_{a} \subset {\rm id}_{b} \wedge {\rm id}_{b} \subset {\rm id}_{a}
\]
が成り立つ.
また定理 \ref{sthmaxiom1}より
\[
\tag{3}
  {\rm id}_{a} \subset {\rm id}_{b} \wedge {\rm id}_{b} \subset {\rm id}_{a} \leftrightarrow {\rm id}_{a} = {\rm id}_{b}
\]
が成り立つ.
そこで(1), (2), (3)から, 推論法則 \ref{dedeqtrans}によって
$a = b \leftrightarrow {\rm id}_{a} = {\rm id}_{b}$が成り立つことがわかる.
($*$)が成り立つことは, これと推論法則 \ref{dedeqfund}によって明らかである.
\halmos




\mathstrut
\begin{thm}
\label{sthmuopairiden}%定理
$a$と$b$を集合とするとき, 
\[
  {\rm id}_{\{a, b\}} = \{(a, a), (b, b)\}
\]
が成り立つ.
\end{thm}


\noindent{\bf 証明}
~$x$と$y$を, 互いに異なり, 共に$a$及び$b$の中に自由変数として現れない, 定数でない文字とする.
このとき変数法則 \ref{valnset}, \ref{valpair}, \ref{validen}からわかるように, 
$x$と$y$は共に${\rm id}_{\{a, b\}}$及び$\{(a, a), (b, b)\}$の中に自由変数として現れない.
そして定理 \ref{sthmpairiniden}より
\[
\tag{1}
  (x, y) \in {\rm id}_{\{a, b\}} \leftrightarrow x = y \wedge x \in \{a, b\}
\]
が成り立つ.
また定理 \ref{sthmuopairbasis}より
$x \in \{a, b\} \leftrightarrow x = a \vee x = b$が
成り立つから, 推論法則 \ref{dedaddeqw}により
\[
\tag{2}
  x = y \wedge x \in \{a, b\} \leftrightarrow x = y \wedge (x = a \vee x = b)
\]
が成り立つ.
またThm \ref{aw1bvc1l1awb1v1awc1}より
\[
\tag{3}
  x = y \wedge (x = a \vee x = b) \leftrightarrow (x = y \wedge x = a) \vee (x = y \wedge x = b)
\]
が成り立つ.
またThm \ref{x=yty=x}より$x = y \to y = x$が成り立つから, 
推論法則 \ref{dedaddw}により
\[
\tag{4}
  x = y \wedge x = a \to y = x \wedge x = a
\]
が成り立つ.
またThm \ref{awbta}より
\[
  y = x \wedge x = a \to x = a
\]
が成り立ち, Thm \ref{x=ywy=ztx=z}より
\[
  y = x \wedge x = a \to y = a
\]
が成り立つから, これらから推論法則 \ref{dedprewedge}によって
\[
\tag{5}
  y = x \wedge x = a \to x = a \wedge y = a
\]
が成り立つ.
そこで(4), (5)から, 推論法則 \ref{dedmmp}によって
\[
\tag{6}
  x = y \wedge x = a \to x = a \wedge y = a
\]
が成り立つ.
またThm \ref{x=yty=x}より$y = a \to a = y$が成り立つから, 
推論法則 \ref{dedaddw}により
\[
\tag{7}
  x = a \wedge y = a \to x = a \wedge a = y
\]
が成り立つ.
またThm \ref{x=ywy=ztx=z}より
\[
  x = a \wedge a = y \to x = y
\]
が成り立ち, Thm \ref{awbta}より
\[
  x = a \wedge a = y \to x = a
\]
が成り立つから, これらから推論法則 \ref{dedprewedge}によって
\[
\tag{8}
  x = a \wedge a = y \to x = y \wedge x = a
\]
が成り立つ.
そこで(7), (8)から, 推論法則 \ref{dedmmp}によって
\[
\tag{9}
  x = a \wedge y = a \to x = y \wedge x = a
\]
が成り立つ.
故に(6), (9)から, 推論法則 \ref{dedequiv}によって
\[
\tag{10}
  x = y \wedge x = a \leftrightarrow x = a \wedge y = a
\]
が成り立つ.
以上と全く同様にして, 
\[
\tag{11}
  x = y \wedge x = b \leftrightarrow x = b \wedge y = b
\]
も成り立つことがわかる.
また定理 \ref{sthmpair}と推論法則 \ref{dedeqch}により
\begin{align*}
  \tag{12}
  x = a \wedge y = a &\leftrightarrow (x, y) = (a, a), \\
  \mbox{} \\
  \tag{13}
  x = b \wedge y = b &\leftrightarrow (x, y) = (b, b)
\end{align*}
が成り立つ.
そこで(10)と(12), (11)と(13)から, それぞれ推論法則 \ref{dedeqtrans}によって
\[
  x = y \wedge x = a \leftrightarrow (x, y) = (a, a), ~~
  x = y \wedge x = b \leftrightarrow (x, y) = (b, b)
\]
が成り立つ.
故にこれらから, 推論法則 \ref{dedaddeqv}により
\[
\tag{14}
  (x = y \wedge x = a) \vee (x = y \wedge x = b) \leftrightarrow (x, y) = (a, a) \vee (x, y) = (b, b)
\]
が成り立つ.
また定理 \ref{sthmuopairbasis}と推論法則 \ref{dedeqch}により
\[
\tag{15}
  (x, y) = (a, a) \vee (x, y) = (b, b) \leftrightarrow (x, y) \in \{(a, a), (b, b)\}
\]
が成り立つ.
以上の(1), (2), (3), (14), (15)から, 推論法則 \ref{dedeqtrans}によって
\[
\tag{16}
  (x, y) \in {\rm id}_{\{a, b\}} \leftrightarrow (x, y) \in \{(a, a), (b, b)\}
\]
が成り立つことがわかる.
さていま定理 \ref{sthmidengraph}より, ${\rm id}_{\{a, b\}}$はグラフである.
また定理 \ref{sthmbigpairpair}より$(a, a)$と$(b, b)$は共に対だから, 
定理 \ref{sthmuopairgraph}により$\{(a, a), (b, b)\}$もグラフである.
またはじめに述べたように, $x$と$y$は共に${\rm id}_{\{a, b\}}$及び$\{(a, a), (b, b)\}$の中に
自由変数として現れない.
また$x$と$y$は互いに異なり, 共に定数でない.
以上のことと, (16)が成り立つことから, 定理 \ref{sthmgraphpair=}により
${\rm id}_{\{a, b\}} = \{(a, a), (b, b)\}$が成り立つ.
\halmos




\mathstrut
\begin{thm}
\label{sthmsingletoniden}%定理
$a$を集合とするとき, 
\[
  {\rm id}_{\{a\}} = \{(a, a)\}
\]
が成り立つ.
\end{thm}


\noindent{\bf 証明}
~定理 \ref{sthmuopairiden}より
${\rm id}_{\{a, a\}} = \{(a, a), (a, a)\}$が成り立つが, 
定義からこの記号列は${\rm id}_{\{a\}} = \{(a, a)\}$と同じだから, 
これが定理となる.
\halmos




\mathstrut
\begin{thm}
\label{sthmcupiden}%定理
$a$と$b$を集合とするとき, 
\[
  {\rm id}_{a \cup b} = {\rm id}_{a} \cup {\rm id}_{b}
\]
が成り立つ.
\end{thm}


\noindent{\bf 証明}
~$x$と$y$を, 互いに異なり, 共に$a$及び$b$の中に自由変数として現れない, 定数でない文字とする.
このとき変数法則 \ref{valcup}, \ref{validen}により, $x$と$y$は共に
${\rm id}_{a \cup b}$及び${\rm id}_{a} \cup {\rm id}_{b}$の中に自由変数として現れない.
また定理 \ref{sthmpairiniden}より
\[
\tag{1}
  (x, y) \in {\rm id}_{a \cup b} \leftrightarrow x = y \wedge x \in a \cup b
\]
が成り立つ.
また定理 \ref{sthmcupbasis}より
$x \in a \cup b \leftrightarrow x \in a \vee x \in b$が成り立つから, 
推論法則 \ref{dedaddeqw}により
\[
\tag{2}
  x = y \wedge x \in a \cup b \leftrightarrow x = y \wedge (x \in a \vee x \in b)
\]
が成り立つ.
またThm \ref{aw1bvc1l1awb1v1awc1}より
\[
\tag{3}
  x = y \wedge (x \in a \vee x \in b) \leftrightarrow (x = y \wedge x \in a) \vee (x = y \wedge x \in b)
\]
が成り立つ.
また定理 \ref{sthmpairiniden}と推論法則 \ref{dedeqch}により
\[
  x = y \wedge x \in a \leftrightarrow (x, y) \in {\rm id}_{a}, ~~
  x = y \wedge x \in b \leftrightarrow (x, y) \in {\rm id}_{b}
\]
が共に成り立つから, 推論法則 \ref{dedaddeqv}により
\[
\tag{4}
  (x = y \wedge x \in a) \vee (x = y \wedge x \in b) 
  \leftrightarrow (x, y) \in {\rm id}_{a} \vee (x, y) \in {\rm id}_{b}
\]
が成り立つ.
また定理 \ref{sthmcupbasis}と推論法則 \ref{dedeqch}により
\[
\tag{5}
  (x, y) \in {\rm id}_{a} \vee (x, y) \in {\rm id}_{b} 
  \leftrightarrow (x, y) \in {\rm id}_{a} \cup {\rm id}_{b}
\]
が成り立つ.
そこで(1)---(5)から, 推論法則 \ref{dedeqtrans}によって
\[
\tag{6}
  (x, y) \in {\rm id}_{a \cup b} \leftrightarrow (x, y) \in {\rm id}_{a} \cup {\rm id}_{b}
\]
が成り立つことがわかる.
さていま定理 \ref{sthmcupgraph}, \ref{sthmidengraph}より, 
${\rm id}_{a \cup b}$と${\rm id}_{a} \cup {\rm id}_{b}$は共にグラフである.
また上述のように, $x$と$y$は共にこれらの中に自由変数として現れない.
また$x$と$y$は互いに異なり, 共に定数でない.
これらのことと, (6)が成り立つことから, 
定理 \ref{sthmgraphpair=}により
${\rm id}_{a \cup b} = {\rm id}_{a} \cup {\rm id}_{b}$が成り立つ.
\halmos




\mathstrut
\begin{thm}
\label{sthmcapiden}%定理
$a$と$b$を集合とするとき, 
\[
  {\rm id}_{a \cap b} = {\rm id}_{a} \cap {\rm id}_{b}
\]
が成り立つ.
\end{thm}


\noindent{\bf 証明}
~$x$と$y$を, 互いに異なり, 共に$a$及び$b$の中に自由変数として現れない, 定数でない文字とする.
このとき変数法則 \ref{valcap}, \ref{validen}により, $x$と$y$は共に
${\rm id}_{a \cap b}$及び${\rm id}_{a} \cap {\rm id}_{b}$の中に自由変数として現れない.
また定理 \ref{sthmpairiniden}より
\[
\tag{1}
  (x, y) \in {\rm id}_{a \cap b} \leftrightarrow x = y \wedge x \in a \cap b
\]
が成り立つ.
また定理 \ref{sthmcapelement}より
$x \in a \cap b \leftrightarrow x \in a \wedge x \in b$が成り立つから, 
推論法則 \ref{dedaddeqw}により
\[
\tag{2}
  x = y \wedge x \in a \cap b \leftrightarrow x = y \wedge (x \in a \wedge x \in b)
\]
が成り立つ.
またThm \ref{aw1bwc1l1awb1w1awc1}より
\[
\tag{3}
  x = y \wedge (x \in a \wedge x \in b) \leftrightarrow (x = y \wedge x \in a) \wedge (x = y \wedge x \in b)
\]
が成り立つ.
また定理 \ref{sthmpairiniden}と推論法則 \ref{dedeqch}により
\[
  x = y \wedge x \in a \leftrightarrow (x, y) \in {\rm id}_{a}, ~~
  x = y \wedge x \in b \leftrightarrow (x, y) \in {\rm id}_{b}
\]
が共に成り立つから, 推論法則 \ref{dedaddeqw}により
\[
\tag{4}
  (x = y \wedge x \in a) \wedge (x = y \wedge x \in b) 
  \leftrightarrow (x, y) \in {\rm id}_{a} \wedge (x, y) \in {\rm id}_{b}
\]
が成り立つ.
また定理 \ref{sthmcapelement}と推論法則 \ref{dedeqch}により
\[
\tag{5}
  (x, y) \in {\rm id}_{a} \wedge (x, y) \in {\rm id}_{b} 
  \leftrightarrow (x, y) \in {\rm id}_{a} \cap {\rm id}_{b}
\]
が成り立つ.
そこで(1)---(5)から, 推論法則 \ref{dedeqtrans}によって
\[
\tag{6}
  (x, y) \in {\rm id}_{a \cap b} \leftrightarrow (x, y) \in {\rm id}_{a} \cap {\rm id}_{b}
\]
が成り立つことがわかる.
さていま定理 \ref{sthmcapgraph}, \ref{sthmidengraph}より, 
${\rm id}_{a \cap b}$と${\rm id}_{a} \cap {\rm id}_{b}$は共にグラフである.
また上述のように, $x$と$y$は共にこれらの中に自由変数として現れない.
また$x$と$y$は互いに異なり, 共に定数でない.
これらのことと, (6)が成り立つことから, 
定理 \ref{sthmgraphpair=}により
${\rm id}_{a \cap b} = {\rm id}_{a} \cap {\rm id}_{b}$が成り立つ.
\halmos




\mathstrut
\begin{thm}
\label{sthm-iden}%定理
$a$と$b$を集合とするとき, 
\[
  {\rm id}_{a - b} = {\rm id}_{a} - {\rm id}_{b}
\]
が成り立つ.
\end{thm}


\noindent{\bf 証明}
~$x$と$y$を, 互いに異なり, 共に$a$及び$b$の中に自由変数として現れない, 定数でない文字とする.
このとき変数法則 \ref{val-}, \ref{validen}により, $x$と$y$は共に
${\rm id}_{a - b}$及び${\rm id}_{a} - {\rm id}_{b}$の中に自由変数として現れない.
また定理 \ref{sthmpairiniden}より
\[
\tag{1}
  (x, y) \in {\rm id}_{a - b} \leftrightarrow x = y \wedge x \in a - b
\]
が成り立つ.
また定理 \ref{sthm-basis}より
$x \in a - b \leftrightarrow x \in a \wedge x \notin b$が成り立つから, 
推論法則 \ref{dedaddeqw}により
\[
\tag{2}
  x = y \wedge x \in a - b \leftrightarrow x = y \wedge (x \in a \wedge x \notin b)
\]
が成り立つ.
またThm \ref{1awb1wclaw1bwc1}と推論法則 \ref{dedeqch}により
\[
\tag{3}
  x = y \wedge (x \in a \wedge x \notin b) \leftrightarrow (x = y \wedge x \in a) \wedge x \notin b
\]
が成り立つ.
またThm \ref{awblbwa}より
$x = y \wedge x \in a \leftrightarrow x \in a \wedge x = y$が成り立つから, 
推論法則 \ref{dedaddeqw}により
\[
\tag{4}
  (x = y \wedge x \in a) \wedge x \neq y \leftrightarrow (x \in a \wedge x = y) \wedge x \neq y
\]
が成り立つ.
またThm \ref{1awb1wclaw1bwc1}より
\[
\tag{5}
  (x \in a \wedge x = y) \wedge x \neq y \leftrightarrow x \in a \wedge (x = y \wedge x \neq y)
\]
が成り立つ.
そこで(4), (5)から, 推論法則 \ref{dedeqtrans}によって
\[
\tag{6}
  (x = y \wedge x \in a) \wedge x \neq y \leftrightarrow x \in a \wedge (x = y \wedge x \neq y)
\]
が成り立つ.
ここでThm \ref{n1awna1}より$\neg (x = y \wedge x \neq y)$が成り立つから, 
推論法則 \ref{dednw}により
\[
  \neg (x \in a \wedge (x = y \wedge x \neq y))
\]
が成り立つ.
故にこれと(6)から, 推論法則 \ref{dedeqfund}により
\[
  \neg ((x = y \wedge x \in a) \wedge x \neq y)
\]
が成り立つ.
そこで推論法則 \ref{dedavblbtrue2}により
\[
  ((x = y \wedge x \in a) \wedge x \neq y) \vee ((x = y \wedge x \in a) \wedge x \notin b) 
  \leftrightarrow (x = y \wedge x \in a) \wedge x \notin b
\]
が成り立ち, これから推論法則 \ref{dedeqch}により
\[
\tag{7}
  (x = y \wedge x \in a) \wedge x \notin b 
  \leftrightarrow ((x = y \wedge x \in a) \wedge x \neq y) \vee ((x = y \wedge x \in a) \wedge x \notin b)
\]
が成り立つ.
またThm \ref{aw1bvc1l1awb1v1awc1}と推論法則 \ref{dedeqch}により
\[
\tag{8}
  ((x = y \wedge x \in a) \wedge x \neq y) \vee ((x = y \wedge x \in a) \wedge x \notin b) 
  \leftrightarrow (x = y \wedge x \in a) \wedge (x \neq y \vee x \notin b)
\]
が成り立つ.
またThm \ref{n1awb1lnavnb}と推論法則 \ref{dedeqch}により
$x \neq y \vee x \notin b \leftrightarrow \neg (x = y \wedge x \in b)$が成り立つから, 
推論法則 \ref{dedaddeqw}により
\[
\tag{9}
  (x = y \wedge x \in a) \wedge (x \neq y \vee x \notin b) 
  \leftrightarrow (x = y \wedge x \in a) \wedge \neg (x = y \wedge x \in b)
\]
が成り立つ.
また定理 \ref{sthmpairiniden}と推論法則 \ref{dedeqch}により
\begin{align*}
  \tag{10}
  x = y \wedge x \in a &\leftrightarrow (x, y) \in {\rm id}_{a}, \\
  \mbox{} \\
  \tag{11}
  x = y \wedge x \in b &\leftrightarrow (x, y) \in {\rm id}_{b}
\end{align*}
が共に成り立つから, この(11)から, 推論法則 \ref{dedeqcp}によって
\[
  \neg (x = y \wedge x \in b) \leftrightarrow (x, y) \notin {\rm id}_{b}
\]
が成り立ち, これと(10)から, 推論法則 \ref{dedaddeqw}によって
\[
\tag{12}
  (x = y \wedge x \in a) \wedge \neg (x = y \wedge x \in b) 
  \leftrightarrow (x, y) \in {\rm id}_{a} \wedge (x, y) \notin {\rm id}_{b}
\]
が成り立つ.
また定理 \ref{sthm-basis}と推論法則 \ref{dedeqch}により
\[
\tag{13}
  (x, y) \in {\rm id}_{a} \wedge (x, y) \notin {\rm id}_{b} 
  \leftrightarrow (x, y) \in {\rm id}_{a} - {\rm id}_{b}
\]
が成り立つ.
以上の(1), (2), (3), (7), (8), (9), (12), (13)から, 推論法則 \ref{dedeqtrans}によって
\[
\tag{14}
  (x, y) \in {\rm id}_{a - b} \leftrightarrow (x, y) \in {\rm id}_{a} - {\rm id}_{b}
\]
が成り立つことがわかる.
さていま定理 \ref{sthm-graph}, \ref{sthmidengraph}より, 
${\rm id}_{a - b}$と${\rm id}_{a} - {\rm id}_{b}$は共にグラフである.
またはじめに述べたように, $x$と$y$は共にこれらの中に自由変数として現れない.
また$x$と$y$は互いに異なり, 共に定数でない.
これらのことと, (14)が成り立つことから, 
定理 \ref{sthmgraphpair=}により
${\rm id}_{a - b} = {\rm id}_{a} - {\rm id}_{b}$が成り立つ.
\halmos




\mathstrut
\begin{thm}
\label{sthmemptyiden}%定理
$a$を集合とするとき, 
\[
  {\rm id}_{a} = \phi \leftrightarrow a = \phi
\]
が成り立つ.
またこのことから, 次の($*$)が成り立つ: 

($*$) ~~${\rm id}_{a}$が空ならば, $a$は空である.
        逆に$a$が空ならば, ${\rm id}_{a}$は空である.
        特に, ${\rm id}_{\phi}$は空である.
\end{thm}


\noindent{\bf 証明}
~$x$を$a$の中に自由変数として現れない文字とする.
このとき変数法則 \ref{validen}により, $x$は${\rm id}_{a}$の中に自由変数として現れない.
そこで$\tau_{x}(x \in {\rm id}_{a})$を$T$と書けば, $T$は集合であり, 
定理 \ref{sthmelm&empty}と推論法則 \ref{dedequiv}により
\[
\tag{1}
  {\rm id}_{a} \neq \phi \to T \in {\rm id}_{a}
\]
が成り立つ.
また定理 \ref{sthmidenelement}と推論法則 \ref{dedequiv}により
\[
  T \in {\rm id}_{a} \to T = ({\rm pr}_{1}(T), {\rm pr}_{1}(T)) \wedge {\rm pr}_{1}(T) \in a
\]
が成り立つから, 推論法則 \ref{dedprewedge}により
\[
\tag{2}
  T \in {\rm id}_{a} \to {\rm pr}_{1}(T) \in a
\]
が成り立つ.
また定理 \ref{sthmnotemptyeqexin}より
\[
\tag{3}
  {\rm pr}_{1}(T) \in a \to a \neq \phi
\]
が成り立つ.
そこで(1), (2), (3)から, 推論法則 \ref{dedmmp}によって
\[
\tag{4}
  {\rm id}_{a} \neq \phi \to a \neq \phi
\]
が成り立つことがわかる.
またいま$\tau_{x}(x \in a)$を$U$と書けば, $U$は集合であり, 
$x$が$a$の中に自由変数として現れないことから, 
定理 \ref{sthmelm&empty}と推論法則 \ref{dedequiv}により
\[
\tag{5}
  a \neq \phi \to U \in a
\]
が成り立つ.
また定理 \ref{sthmpairiniden2}と推論法則 \ref{dedequiv}により
\[
\tag{6}
  U \in a \to (U, U) \in {\rm id}_{a}
\]
が成り立つ.
また定理 \ref{sthmnotemptyeqexin}より
\[
\tag{7}
  (U, U) \in {\rm id}_{a} \to {\rm id}_{a} \neq \phi
\]
が成り立つ.
そこで(5), (6), (7)から, 推論法則 \ref{dedmmp}によって
\[
\tag{8}
  a \neq \phi \to {\rm id}_{a} \neq \phi
\]
が成り立つことがわかる.
故に(4), (8)から, 推論法則 \ref{dedequiv}によって
\[
  {\rm id}_{a} \neq \phi \leftrightarrow a \neq \phi
\]
が成り立ち, これから推論法則 \ref{dedeqcp}によって
\[
  {\rm id}_{a} = \phi \leftrightarrow a = \phi
\]
が成り立つ.
これと推論法則 \ref{dedeqfund}により, 
${\rm id}_{a}$が空ならば$a$は空であり, 
$a$が空ならば${\rm id}_{a}$は空であることがわかる.
特にThm \ref{x=x}より$\phi = \phi$が成り立つから, 
いま述べたことから, ${\rm id}_{\phi}$は空となる.
\halmos




\mathstrut
\begin{thm}
\label{sthmprsetiden}%定理
$a$を集合とするとき, 
\[
  {\rm pr}_{1}\langle {\rm id}_{a} \rangle = a, ~~
  {\rm pr}_{2}\langle {\rm id}_{a} \rangle = a
\]
が成り立つ.
\end{thm}


\noindent{\bf 証明}
~$x$と$y$を, 互いに異なり, 共に$a$の中に自由変数として現れない, 定数でない文字とする.
このとき変数法則 \ref{validen}により, $x$と$y$は共に${\rm id}_{a}$の中に自由変数として現れないから, 
定理 \ref{sthmprsetelement}より
\begin{align*}
  \tag{1}
  x \in {\rm pr}_{1}\langle {\rm id}_{a} \rangle &\leftrightarrow \exists y((x, y) \in {\rm id}_{a}), \\
  \mbox{} \\
  \tag{2}
  y \in {\rm pr}_{2}\langle {\rm id}_{a} \rangle &\leftrightarrow \exists x((x, y) \in {\rm id}_{a})
\end{align*}
が共に成り立つ.
また定理 \ref{sthmpairiniden}より
\[
  (x, y) \in {\rm id}_{a} \leftrightarrow x = y \wedge x \in a, ~~
  (x, y) \in {\rm id}_{a} \leftrightarrow x = y \wedge y \in a
\]
が共に成り立つから, 
$x$と$y$が共に定数でないことから, 推論法則 \ref{dedalleqquansepconst}により
\begin{align*}
  \tag{3}
  \exists y((x, y) \in {\rm id}_{a}) &\leftrightarrow \exists y(x = y \wedge x \in a), \\
  \mbox{} \\
  \tag{4}
  \exists x((x, y) \in {\rm id}_{a}) &\leftrightarrow \exists x(x = y \wedge y \in a)
\end{align*}
が共に成り立つ.
また$x$と$y$は互いに異なり, 共に$a$の中に自由変数として現れないから, 
変数法則 \ref{valfund}により, $y$は$x \in a$の中に自由変数として現れず, 
$x$は$y \in a$の中に自由変数として現れない.
そこでThm \ref{thmexwrfree}より
\begin{align*}
  \tag{5}
  \exists y(x = y \wedge x \in a) &\leftrightarrow \exists y(x = y) \wedge x \in a, \\
  \mbox{} \\
  \tag{6}
  \exists x(x = y \wedge y \in a) &\leftrightarrow \exists x(x = y) \wedge y \in a
\end{align*}
が共に成り立つ.
またThm \ref{x=x}より$x = x$と$y = y$が共に成り立つが, 
$x$と$y$が互いに異なることから, これらはそれぞれ
$(x|y)(x = y)$, $(y|x)(x = y)$と一致し, 
故にこれらが定理となる.
そこで推論法則 \ref{deds4}により, 
\[
  \exists y(x = y), ~~
  \exists x(x = y)
\]
が共に成り立つ.
そこで推論法則 \ref{dedawblatrue2}により
\begin{align*}
  \tag{7}
  \exists y(x = y) \wedge x \in a &\leftrightarrow x \in a, \\
  \mbox{} \\
  \tag{8}
  \exists x(x = y) \wedge y \in a &\leftrightarrow y \in a
\end{align*}
が共に成り立つ.
以上の(1), (3), (5), (7)から, 推論法則 \ref{dedeqtrans}によって
\[
\tag{9}
  x \in {\rm pr}_{1}\langle {\rm id}_{a} \rangle \leftrightarrow x \in a
\]
が成り立ち, (2), (4), (6), (8)から, 同じく推論法則 \ref{dedeqtrans}によって
\[
\tag{10}
  y \in {\rm pr}_{2}\langle {\rm id}_{a} \rangle \leftrightarrow y \in a
\]
が成り立つことがわかる.
いま$x$と$y$は共に$a$の中に自由変数として現れないから, 
変数法則 \ref{valprset}, \ref{validen}により, これらは共に
${\rm pr}_{1}\langle {\rm id}_{a} \rangle$及び${\rm pr}_{2}\langle {\rm id}_{a} \rangle$の中にも
自由変数として現れない.
また$x$と$y$は共に定数でない.
これらのことと, (9)と(10)が共に成り立つことから, 定理 \ref{sthmset=}により
\[
  {\rm pr}_{1}\langle {\rm id}_{a} \rangle = a, ~~
  {\rm pr}_{2}\langle {\rm id}_{a} \rangle = a
\]
が共に成り立つ.
\halmos




\mathstrut
\begin{thm}
\label{sthmvaluesetiden}%定理
$a$と$b$を集合とするとき, 
\[
  {\rm id}_{a}[b] = a \cap b
\]
が成り立つ.
\end{thm}


\noindent{\bf 証明}
~$x$と$y$を, 互いに異なり, 共に$a$及び$b$の中に自由変数として現れない, 定数でない文字とする.
このとき変数法則 \ref{validen}により, $x$は${\rm id}_{a}$の中にも自由変数として現れないから, 
定理 \ref{sthmvaluesetelement}より
\[
\tag{1}
  y \in {\rm id}_{a}[b] \leftrightarrow \exists x(x \in b \wedge (x, y) \in {\rm id}_{a})
\]
が成り立つ.
また定理 \ref{sthmpairiniden}より
\[
  (x, y) \in {\rm id}_{a} \leftrightarrow x = y \wedge y \in a
\]
が成り立つから, 推論法則 \ref{dedaddeqw}により
\[
\tag{2}
  x \in b \wedge (x, y) \in {\rm id}_{a} \leftrightarrow x \in b \wedge (x = y \wedge y \in a)
\]
が成り立つ.
またThm \ref{1awb1wclaw1bwc1}と推論法則 \ref{dedeqch}により
\[
\tag{3}
  x \in b \wedge (x = y \wedge y \in a) \leftrightarrow (x \in b \wedge x = y) \wedge y \in a
\]
が成り立つ.
またThm \ref{awblbwa}より
\[
\tag{4}
  x \in b \wedge x = y \leftrightarrow x = y \wedge x \in b
\]
が成り立つ.
またThm \ref{thmfroms5eq}より
\[
  x = y \wedge (x|x)(x \in b) \leftrightarrow x = y \wedge (y|x)(x \in b)
\]
が成り立つが, $x$が$b$の中に自由変数として現れないことから, 
代入法則 \ref{substsame}, \ref{substfree}, \ref{substfund}により, 
この記号列は
\[
\tag{5}
  x = y \wedge x \in b \leftrightarrow x = y \wedge y \in b
\]
と一致し, 故にこれが定理となる.
そこで(4), (5)から, 推論法則 \ref{dedeqtrans}によって
\[
  x \in b \wedge x = y \leftrightarrow x = y \wedge y \in b
\]
が成り立ち, これから推論法則 \ref{dedaddeqw}によって
\[
\tag{6}
  (x \in b \wedge x = y) \wedge y \in a \leftrightarrow (x = y \wedge y \in b) \wedge y \in a
\]
が成り立つ.
またThm \ref{1awb1wclaw1bwc1}より
\[
\tag{7}
  (x = y \wedge y \in b) \wedge y \in a \leftrightarrow x = y \wedge (y \in b \wedge y \in a)
\]
が成り立つ.
そこで(2), (3), (6), (7)から, 推論法則 \ref{dedeqtrans}によって
\[
  x \in b \wedge (x, y) \in {\rm id}_{a} \leftrightarrow x = y \wedge (y \in b \wedge y \in a)
\]
が成り立つことがわかる.
$x$は定数でないから, これから推論法則 \ref{dedalleqquansepconst}によって
\[
\tag{8}
  \exists x(x \in b \wedge (x, y) \in {\rm id}_{a}) \leftrightarrow \exists x(x = y \wedge (y \in b \wedge y \in a))
\]
が成り立つ.
また$x$が$y$と異なり, $a$及び$b$の中に自由変数として現れないことから, 
変数法則 \ref{valfund}, \ref{valwedge}によって, $x$が
$y \in b \wedge y \in a$の中に自由変数として現れないことがわかるから, 
Thm \ref{thmexwrfree}より
\[
\tag{9}
  \exists x(x = y \wedge (y \in b \wedge y \in a)) \leftrightarrow \exists x(x = y) \wedge (y \in b \wedge y \in a)
\]
が成り立つ.
またThm \ref{x=x}より$y = y$が成り立つが, $x$と$y$が互いに異なることから, 
この記号列は$(y|x)(x = y)$と一致し, 故にこれが定理となる.
そこで推論法則 \ref{deds4}により
$\exists x(x = y)$が成り立ち, 
これから推論法則 \ref{dedawblatrue2}によって
\[
\tag{10}
  \exists x(x = y) \wedge (y \in b \wedge y \in a) \leftrightarrow y \in b \wedge y \in a
\]
が成り立つ.
またThm \ref{awblbwa}より
\[
\tag{11}
  y \in b \wedge y \in a \leftrightarrow y \in a \wedge y \in b
\]
が成り立つ.
また定理 \ref{sthmcapelement}と推論法則 \ref{dedeqch}により
\[
\tag{12}
  y \in a \wedge y \in b \leftrightarrow y \in a \cap b
\]
が成り立つ.
そこで(1), (8)---(12)から, 推論法則 \ref{dedeqtrans}によって
\[
\tag{13}
  y \in {\rm id}_{a}[b] \leftrightarrow y \in a \cap b
\]
が成り立つことがわかる.
いま$y$は$a$及び$b$の中に自由変数として現れないから, 
変数法則 \ref{valcap}, \ref{valvalueset}, \ref{validen}によってわかるように, 
$y$は${\rm id}_{a}[b]$及び$a \cap b$の中に自由変数として現れない.
また$y$は定数でない.
これらのことと, (13)が成り立つことから, 
定理 \ref{sthmset=}により${\rm id}_{a}[b] = a \cap b$が成り立つ.
\halmos




\mathstrut
\begin{thm}
\label{sthmvaluesetiden2}%定理
$a$と$b$を集合とするとき, 
\[
  a \subset b \leftrightarrow {\rm id}_{a}[b] = a, ~~
  b \subset a \leftrightarrow {\rm id}_{a}[b] = b
\]
が成り立つ.
またこのことから, 次の1), 2)が成り立つ.

1)
$a \subset b$が成り立つならば, ${\rm id}_{a}[b] = a$が成り立つ.
逆に${\rm id}_{a}[b] = a$が成り立つならば, $a \subset b$が成り立つ.

2)
$b \subset a$が成り立つならば, ${\rm id}_{a}[b] = b$が成り立つ.
逆に${\rm id}_{a}[b] = b$が成り立つならば, $b \subset a$が成り立つ.
\end{thm}


\noindent{\bf 証明}
~定理 \ref{sthmcapsubset=}より
\begin{align*}
  \tag{1}
  a \subset b &\leftrightarrow a \cap b = a, \\
  \mbox{} \\
  \tag{2}
  b \subset a &\leftrightarrow b \cap a = b
\end{align*}
が共に成り立つ.
また定理 \ref{sthmcapch}より$b \cap a = a \cap b$が成り立つから, 
推論法則 \ref{dedaddeq=}により
\[
\tag{3}
  b \cap a = b \leftrightarrow a \cap b = b
\]
が成り立つ.
また定理 \ref{sthmvaluesetiden}と推論法則 \ref{ded=ch}により
$a \cap b = {\rm id}_{a}[b]$が成り立つから, 
再び推論法則 \ref{dedaddeq=}により
\begin{align*}
  \tag{4}
  a \cap b = a &\leftrightarrow {\rm id}_{a}[b] = a, \\
  \mbox{} \\
  \tag{5}
  a \cap b = b &\leftrightarrow {\rm id}_{a}[b] = b
\end{align*}
が共に成り立つ.
そこで(1), (4)から, 推論法則 \ref{dedeqtrans}によって
\[
  a \subset b \leftrightarrow {\rm id}_{a}[b] = a
\]
が成り立ち, (2), (3), (5)から, 同じく推論法則 \ref{dedeqtrans}によって
\[
  b \subset a \leftrightarrow {\rm id}_{a}[b] = b
\]
が成り立つことがわかる.
1), 2)が成り立つことは, これらと推論法則 \ref{dedeqfund}によって明らかである.
\halmos




\mathstrut
\begin{thm}
\label{sthmideninv}%定理
$a$を集合とするとき, 
\[
  {\rm id}_{a}^{-1} = {\rm id}_{a}
\]
が成り立つ.
\end{thm}


\noindent{\bf 証明}
~$x$と$y$を, 互いに異なり, 共に$a$の中に自由変数として現れない, 定数でない文字とする.
このとき変数法則 \ref{valinv}, \ref{validen}により, $x$と$y$は共に
${\rm id}_{a}$及び${\rm id}_{a}^{-1}$の中に自由変数として現れない.
そして定理 \ref{sthmpairininv}より
\[
\tag{1}
  (x, y) \in {\rm id}_{a}^{-1} \leftrightarrow (y, x) \in {\rm id}_{a}
\]
が成り立つ.
また定理 \ref{sthmpairiniden}より
\[
\tag{2}
  (y, x) \in {\rm id}_{a} \leftrightarrow y = x \wedge x \in a
\]
が成り立つ.
またThm \ref{x=yly=x}より$y = x \leftrightarrow x = y$が成り立つから, 
推論法則 \ref{dedaddeqw}により
\[
\tag{3}
  y = x \wedge x \in a \leftrightarrow x = y \wedge x \in a
\]
が成り立つ.
また定理 \ref{sthmpairiniden}と推論法則 \ref{dedeqch}により
\[
\tag{4}
  x = y \wedge x \in a \leftrightarrow (x, y) \in {\rm id}_{a}
\]
が成り立つ.
以上の(1)---(4)から, 推論法則 \ref{dedeqtrans}によって
\[
\tag{5}
  (x, y) \in {\rm id}_{a}^{-1} \leftrightarrow (x, y) \in {\rm id}_{a}
\]
が成り立つことがわかる.
さていま定理 \ref{sthminvgraph}, \ref{sthmidengraph}より, 
${\rm id}_{a}^{-1}$と${\rm id}_{a}$は共にグラフである.
また上述のように, $x$と$y$は共にこれらの中に自由変数として現れない.
また$x$と$y$は互いに異なり, 共に定数でない.
これらのことと, (5)が成り立つことから, 定理 \ref{sthmgraphpair=}により
${\rm id}_{a}^{-1} = {\rm id}_{a}$が成り立つ.
\halmos




\mathstrut
\begin{thm}
\label{sthmidencomp}%定理
$a$と$b$を集合とするとき, 
\[
  b \circ {\rm id}_{a} = b \cap (a \times {\rm pr}_{2}\langle b \rangle), ~~
  {\rm id}_{a} \circ b = b \cap ({\rm pr}_{1}\langle b \rangle \times a)
\]
が成り立つ.
\end{thm}


\noindent{\bf 証明}
~$x$, $y$, $z$を, どの二つも互いに異なり, いずれも$a$及び$b$の中に自由変数として現れない, 
定数でない文字とする.
このとき変数法則 \ref{validen}により, $y$は${\rm id}_{a}$の中にも自由変数として現れないから, 
定理 \ref{sthmpairincompeq}より
\begin{align*}
  \tag{1}
  (x, z) \in b \circ {\rm id}_{a} &\leftrightarrow \exists y((x, y) \in {\rm id}_{a} \wedge (y, z) \in b), \\
  \mbox{} \\
  \tag{2}
  (x, z) \in {\rm id}_{a} \circ b &\leftrightarrow \exists y((x, y) \in b \wedge (y, z) \in {\rm id}_{a})
\end{align*}
が共に成り立つ.
また定理 \ref{sthmpairiniden}より
\[
  (x, y) \in {\rm id}_{a} \leftrightarrow x = y \wedge x \in a, ~~
  (y, z) \in {\rm id}_{a} \leftrightarrow y = z \wedge z \in a
\]
が共に成り立つから, 推論法則 \ref{dedaddeqw}により
\begin{align*}
  \tag{3}
  (x, y) \in {\rm id}_{a} \wedge (y, z) \in b &\leftrightarrow (x = y \wedge x \in a) \wedge (y, z) \in b, \\
  \mbox{} \\
  \tag{4}
  (x, y) \in b \wedge (y, z) \in {\rm id}_{a} &\leftrightarrow (x, y) \in b \wedge (y = z \wedge z \in a)
\end{align*}
が共に成り立つ.
またThm \ref{1awb1wclaw1bwc1}より
\[
\tag{5}
  (x = y \wedge x \in a) \wedge (y, z) \in b \leftrightarrow x = y \wedge (x \in a \wedge (y, z) \in b)
\]
が成り立ち, Thm \ref{1awb1wclaw1bwc1}と推論法則 \ref{dedeqch}により
\[
\tag{6}
  (x, y) \in b \wedge (y = z \wedge z \in a) \leftrightarrow ((x, y) \in b \wedge y = z) \wedge z \in a
\]
が成り立つ.
またThm \ref{awblbwa}より
\[
  x \in a \wedge (y, z) \in b \leftrightarrow (y, z) \in b \wedge x \in a, ~~
  (x, y) \in b \wedge y = z \leftrightarrow y = z \wedge (x, y) \in b
\]
が共に成り立つから, 推論法則 \ref{dedaddeqw}により
\begin{align*}
  \tag{7}
  x = y \wedge (x \in a \wedge (y, z) \in b) &\leftrightarrow x = y \wedge ((y, z) \in b \wedge x \in a), \\
  \mbox{} \\
  \tag{8}
  ((x, y) \in b \wedge y = z) \wedge z \in a &\leftrightarrow (y = z \wedge (x, y) \in b) \wedge z \in a
\end{align*}
が共に成り立つ.
またThm \ref{1awb1wclaw1bwc1}と推論法則 \ref{dedeqch}により
\[
\tag{9}
  x = y \wedge ((y, z) \in b \wedge x \in a) \leftrightarrow (x = y \wedge (y, z) \in b) \wedge x \in a
\]
が成り立つ.
またThm \ref{thmfroms5eq}より
\begin{align*}
  x = y \wedge (x|x)((x, z) \in b) &\leftrightarrow x = y \wedge (y|x)((x, z) \in b), \\
  \mbox{} \\
  y = z \wedge (y|y)((x, y) \in b) &\leftrightarrow y = z \wedge (z|y)((x, y) \in b)
\end{align*}
が共に成り立つが, $x$, $y$, $z$はどの二つも互いに異なり, いずれも$b$の中に
自由変数として現れないから, 
代入法則 \ref{substsame}, \ref{substfree}, \ref{substfund}, \ref{substpair}により, 
これらの記号列はそれぞれ
\begin{align*}
  \tag{10}
  x = y \wedge (x, z) \in b &\leftrightarrow x = y \wedge (y, z) \in b, \\
  \mbox{} \\
  \tag{11}
  y = z \wedge (x, y) \in b &\leftrightarrow y = z \wedge (x, z) \in b
\end{align*}
と一致する.
よってこれらが共に定理となる.
特にこの(10)から, 推論法則 \ref{dedeqch}により
\[
\tag{12}
  x = y \wedge (y, z) \in b \leftrightarrow x = y \wedge (x, z) \in b
\]
が成り立つ.
そこで(12), (11)から, それぞれ推論法則 \ref{dedaddeqw}により
\begin{align*}
  \tag{13}
  (x = y \wedge (y, z) \in b) \wedge x \in a &\leftrightarrow (x = y \wedge (x, z) \in b) \wedge x \in a, \\
  \mbox{} \\
  \tag{14}
  (y = z \wedge (x, y) \in b) \wedge z \in a &\leftrightarrow (y = z \wedge (x, z) \in b) \wedge z \in a
\end{align*}
が成り立つ.
またThm \ref{1awb1wclaw1bwc1}より
\begin{align*}
  \tag{15}
  (x = y \wedge (x, z) \in b) \wedge x \in a &\leftrightarrow x = y \wedge ((x, z) \in b \wedge x \in a), \\
  \mbox{} \\
  \tag{16}
  (y = z \wedge (x, z) \in b) \wedge z \in a &\leftrightarrow y = z \wedge ((x, z) \in b \wedge z \in a)
\end{align*}
が共に成り立つ.
以上の(3), (5), (7), (9), (13), (15)から, 推論法則 \ref{dedeqtrans}によって
\[
  (x, y) \in {\rm id}_{a} \wedge (y, z) \in b \leftrightarrow x = y \wedge ((x, z) \in b \wedge x \in a)
\]
が成り立ち, (4), (6), (8), (14), (16)から, 同じく推論法則 \ref{dedeqtrans}によって
\[
  (x, y) \in b \wedge (y, z) \in {\rm id}_{a} \leftrightarrow y = z \wedge ((x, z) \in b \wedge z \in a)
\]
が成り立つことがわかる.
いま$y$は定数でないので, これらから推論法則 \ref{dedalleqquansepconst}によって
\begin{align*}
  \tag{17}
  \exists y((x, y) \in {\rm id}_{a} \wedge (y, z) \in b) &\leftrightarrow \exists y(x = y \wedge ((x, z) \in b \wedge x \in a)), \\
  \mbox{} \\
  \tag{18}
  \exists y((x, y) \in b \wedge (y, z) \in {\rm id}_{a}) &\leftrightarrow \exists y(y = z \wedge ((x, z) \in b \wedge z \in a))
\end{align*}
が共に成り立つ.
また$y$は$x$とも$z$とも異なり, $a$及び$b$の中に自由変数として現れないから, 
変数法則 \ref{valfund}, \ref{valwedge}, \ref{valpair}によってわかるように, 
$y$は$(x, z) \in b \wedge x \in a$及び$(x, z) \in b \wedge z \in a$の中に
自由変数として現れない.
そこでThm \ref{thmexwrfree}より
\begin{align*}
  \tag{19}
  \exists y(x = y \wedge ((x, z) \in b \wedge x \in a)) &\leftrightarrow \exists y(x = y) \wedge ((x, z) \in b \wedge x \in a), \\
  \mbox{} \\
  \tag{20}
  \exists y(y = z \wedge ((x, z) \in b \wedge z \in a)) &\leftrightarrow \exists y(y = z) \wedge ((x, z) \in b \wedge z \in a)
\end{align*}
が共に成り立つ.
またThm \ref{x=x}より$x = x$と$z = z$が共に成り立つが, 
$y$が$x$とも$z$とも異なることから, これらの記号列はそれぞれ$(x|y)(x = y)$, $(z|y)(y = z)$と一致し, 
従ってこれらが共に定理となる.
そこで推論法則 \ref{deds4}により
\[
  \exists y(x = y), ~~
  \exists y(y = z)
\]
が共に成り立ち, 従ってこれらから, 推論法則 \ref{dedawblatrue2}により
\begin{align*}
  \tag{21}
  \exists y(x = y) \wedge ((x, z) \in b \wedge x \in a) &\leftrightarrow (x, z) \in b \wedge x \in a, \\
  \mbox{} \\
  \tag{22}
  \exists y(y = z) \wedge ((x, z) \in b \wedge z \in a) &\leftrightarrow (x, z) \in b \wedge z \in a
\end{align*}
が共に成り立つ.
そこで(1), (17), (19), (21)から, 推論法則 \ref{dedeqtrans}によって
\[
\tag{23}
  (x, z) \in b \circ {\rm id}_{a} \leftrightarrow (x, z) \in b \wedge x \in a
\]
が成り立ち, (2), (18), (20), (22)から, 同じく推論法則 \ref{dedeqtrans}によって
\[
\tag{24}
  (x, z) \in {\rm id}_{a} \circ b \leftrightarrow (x, z) \in b \wedge z \in a
\]
が成り立つことがわかる.
また定理 \ref{sthmpairelementinprset}より
\[
  (x, z) \in b \to x \in {\rm pr}_{1}\langle b \rangle \wedge z \in {\rm pr}_{2}\langle b \rangle
\]
が成り立つから, 推論法則 \ref{dedprewedge}により
\[
  (x, z) \in b \to z \in {\rm pr}_{2}\langle b \rangle, ~~
  (x, z) \in b \to x \in {\rm pr}_{1}\langle b \rangle
\]
が共に成り立つ.
故に推論法則 \ref{dedawblatrue1}により, 
\[
  (x, z) \in b \wedge z \in {\rm pr}_{2}\langle b \rangle \leftrightarrow (x, z) \in b, ~~
  (x, z) \in b \wedge x \in {\rm pr}_{1}\langle b \rangle \leftrightarrow (x, z) \in b
\]
が共に成り立ち, これらから, 推論法則 \ref{dedeqch}により
\[
  (x, z) \in b \leftrightarrow (x, z) \in b \wedge z \in {\rm pr}_{2}\langle b \rangle, ~~
  (x, z) \in b \leftrightarrow (x, z) \in b \wedge x \in {\rm pr}_{1}\langle b \rangle
\]
が共に成り立つ.
そこで推論法則 \ref{dedaddeqw}により
\begin{align*}
  \tag{25}
  (x, z) \in b \wedge x \in a &\leftrightarrow ((x, z) \in b \wedge z \in {\rm pr}_{2}\langle b \rangle) \wedge x \in a, \\
  \mbox{} \\
  \tag{26}
  (x, z) \in b \wedge z \in a &\leftrightarrow ((x, z) \in b \wedge x \in {\rm pr}_{1}\langle b \rangle) \wedge z \in a
\end{align*}
が共に成り立つ.
またThm \ref{1awb1wclaw1bwc1}より
\begin{align*}
  \tag{27}
  ((x, z) \in b \wedge z \in {\rm pr}_{2}\langle b \rangle) \wedge x \in a 
  &\leftrightarrow (x, z) \in b \wedge (z \in {\rm pr}_{2}\langle b \rangle \wedge x \in a), \\
  \mbox{} \\
  \tag{28}
  ((x, z) \in b \wedge x \in {\rm pr}_{1}\langle b \rangle) \wedge z \in a 
  &\leftrightarrow (x, z) \in b \wedge (x \in {\rm pr}_{1}\langle b \rangle \wedge z \in a)
\end{align*}
が共に成り立つ.
またThm \ref{awblbwa}より
\[
  z \in {\rm pr}_{2}\langle b \rangle \wedge x \in a \leftrightarrow x \in a \wedge z \in {\rm pr}_{2}\langle b \rangle
\]
が成り立つから, 推論法則 \ref{dedaddeqw}により
\[
\tag{29}
  (x, z) \in b \wedge (z \in {\rm pr}_{2}\langle b \rangle \wedge x \in a) 
  \leftrightarrow (x, z) \in b \wedge (x \in a \wedge z \in {\rm pr}_{2}\langle b \rangle)
\]
が成り立つ.
また定理 \ref{sthmpairinproduct}と推論法則 \ref{dedeqch}により
\begin{align*}
  x \in a \wedge z \in {\rm pr}_{2}\langle b \rangle &\leftrightarrow (x, z) \in a \times {\rm pr}_{2}\langle b \rangle, \\
  \mbox{} \\
  x \in {\rm pr}_{1}\langle b \rangle \wedge z \in a &\leftrightarrow (x, z) \in {\rm pr}_{1}\langle b \rangle \times a
\end{align*}
が共に成り立つから, 推論法則 \ref{dedaddeqw}により
\begin{align*}
  \tag{30}
  (x, z) \in b \wedge (x \in a \wedge z \in {\rm pr}_{2}\langle b \rangle) 
  &\leftrightarrow (x, z) \in b \wedge (x, z) \in a \times {\rm pr}_{2}\langle b \rangle, \\
  \mbox{} \\
  \tag{31}
  (x, z) \in b \wedge (x \in {\rm pr}_{1}\langle b \rangle \wedge z \in a) 
  &\leftrightarrow (x, z) \in b \wedge (x, z) \in {\rm pr}_{1}\langle b \rangle \times a
\end{align*}
が共に成り立つ.
また定理 \ref{sthmcapelement}と推論法則 \ref{dedeqch}により
\begin{align*}
  \tag{32}
  (x, z) \in b \wedge (x, z) \in a \times {\rm pr}_{2}\langle b \rangle 
  &\leftrightarrow (x, z) \in b \cap (a \times {\rm pr}_{2}\langle b \rangle), \\
  \mbox{} \\
  \tag{33}
  (x, z) \in b \wedge (x, z) \in {\rm pr}_{1}\langle b \rangle \times a 
  &\leftrightarrow (x, z) \in b \cap ({\rm pr}_{1}\langle b \rangle \times a)
\end{align*}
が共に成り立つ.
以上の(23), (25), (27), (29), (30), (32)から, 推論法則 \ref{dedeqtrans}によって
\[
\tag{34}
  (x, z) \in b \circ {\rm id}_{a} \leftrightarrow (x, z) \in b \cap (a \times {\rm pr}_{2}\langle b \rangle)
\]
が成り立ち, (24), (26), (28), (31), (33)から, 同じく推論法則 \ref{dedeqtrans}によって
\[
\tag{35}
  (x, z) \in {\rm id}_{a} \circ b \leftrightarrow (x, z) \in b \cap ({\rm pr}_{1}\langle b \rangle \times a)
\]
が成り立つことがわかる.
さていま定理 \ref{sthmcapgraph}, \ref{sthmproductgraph}, \ref{sthmcompgraph}からわかるように, 
$b \circ {\rm id}_{a}$, $b \cap (a \times {\rm pr}_{2}\langle b \rangle)$, ${\rm id}_{a} \circ b$, 
$b \cap ({\rm pr}_{1}\langle b \rangle \times a)$はいずれもグラフである.
また$x$と$z$は共に$a$及び$b$の中に自由変数として現れないから, 
変数法則 \ref{valcap}, \ref{valproduct}, \ref{valprset}, \ref{valcomp}, \ref{validen}によってわかるように, 
$x$と$z$は共にこれら四つのいずれの記号列の中にも自由変数として現れない.
また$x$と$z$は互いに異なり, 共に定数でない.
これらのことと, (34), (35)が共に成り立つことから, 
定理 \ref{sthmgraphpair=}により
\[
  b \circ {\rm id}_{a} = b \cap (a \times {\rm pr}_{2}\langle b \rangle), ~~
  {\rm id}_{a} \circ b = b \cap ({\rm pr}_{1}\langle b \rangle \times a)
\]
が共に成り立つ.
\halmos




\mathstrut
\begin{thm}
\label{sthmidencomp2}%定理
$a$, $b$, $c$を集合とするとき, 
\[
  {\rm pr}_{2}\langle b \rangle \subset c \to b \circ {\rm id}_{a} = b \cap (a \times c), ~~
  {\rm pr}_{1}\langle b \rangle \subset c \to {\rm id}_{a} \circ b = b \cap (c \times a)
\]
が成り立つ.
またこれらから, 次の($*$)が成り立つ: 

($*$) ~~${\rm pr}_{2}\langle b \rangle \subset c$が成り立つならば, 
        $b \circ {\rm id}_{a} = b \cap (a \times c)$が成り立つ.
        また${\rm pr}_{1}\langle b \rangle \subset c$が成り立つならば, 
        ${\rm id}_{a} \circ b = b \cap (c \times a)$が成り立つ.
\end{thm}


\noindent{\bf 証明}
~はじめに
\begin{align*}
  \tag{1}
  b \cap (a \times c) &\subset a \times {\rm pr}_{2}\langle b \rangle, \\
  \mbox{} \\
  \tag{2}
  b \cap (c \times a) &\subset {\rm pr}_{1}\langle b \rangle \times a
\end{align*}
が共に成り立つことを示す.
$x$と$y$を, 互いに異なり, 共に$a$, $b$, $c$のいずれの記号列の中にも自由変数として現れない, 
定数でない文字とする.
このとき変数法則 \ref{valcap}, \ref{valproduct}, \ref{valprset}によってわかるように, 
$x$と$y$は共に$b \cap (a \times c)$, $a \times {\rm pr}_{2}\langle b \rangle$, 
$b \cap (c \times a)$, ${\rm pr}_{1}\langle b \rangle \times a$のいずれの記号列の中にも
自由変数として現れない.
また定理 \ref{sthmcapelement}と推論法則 \ref{dedequiv}により
\begin{align*}
  \tag{3}
  (x, y) \in b \cap (a \times c) &\to (x, y) \in b \wedge (x, y) \in a \times c, \\
  \mbox{} \\
  \tag{4}
  (x, y) \in b \cap (c \times a) &\to (x, y) \in b \wedge (x, y) \in c \times a
\end{align*}
が共に成り立つ.
また定理 \ref{sthmpairelementinprset}より
\[
  (x, y) \in b \to x \in {\rm pr}_{1}\langle b \rangle \wedge y \in {\rm pr}_{2}\langle b \rangle
\]
が成り立つから, 推論法則 \ref{dedprewedge}により
\begin{align*}
  \tag{5}
  (x, y) \in b &\to y \in {\rm pr}_{2}\langle b \rangle, \\
  \mbox{} \\
  \tag{6}
  (x, y) \in b &\to x \in {\rm pr}_{1}\langle b \rangle
\end{align*}
が共に成り立つ.
また定理 \ref{sthmpairinproduct}と推論法則 \ref{dedequiv}により
\[
  (x, y) \in a \times c \to x \in a \wedge y \in c, ~~
  (x, y) \in c \times a \to x \in c \wedge y \in a
\]
が共に成り立つから, 推論法則 \ref{dedprewedge}により
\begin{align*}
  \tag{7}
  (x, y) \in a \times c &\to x \in a, \\
  \mbox{} \\
  \tag{8}
  (x, y) \in c \times a &\to y \in a
\end{align*}
が共に成り立つ.
そこで(5)と(7), (6)と(8)から, それぞれ推論法則 \ref{dedfromaddw}によって
\begin{align*}
  \tag{9}
  (x, y) \in b \wedge (x, y) \in a \times c &\to y \in {\rm pr}_{2}\langle b \rangle \wedge x \in a, \\
  \mbox{} \\
  \tag{10}
  (x, y) \in b \wedge (x, y) \in c \times a &\to x \in {\rm pr}_{1}\langle b \rangle \wedge y \in a
\end{align*}
が成り立つ.
またThm \ref{awbtbwa}より
\[
\tag{11}
  y \in {\rm pr}_{2}\langle b \rangle \wedge x \in a \to x \in a \wedge y \in {\rm pr}_{2}\langle b \rangle
\]
が成り立つ.
また定理 \ref{sthmpairinproduct}と推論法則 \ref{dedequiv}により
\begin{align*}
  \tag{12}
  x \in a \wedge y \in {\rm pr}_{2}\langle b \rangle &\to (x, y) \in a \times {\rm pr}_{2}\langle b \rangle, \\
  \mbox{} \\
  \tag{13}
  x \in {\rm pr}_{1}\langle b \rangle \wedge y \in a &\to (x, y) \in {\rm pr}_{1}\langle b \rangle \times a
\end{align*}
が共に成り立つ.
そこで(3), (9), (11), (12)から, 推論法則 \ref{dedmmp}によって
\[
\tag{14}
  (x, y) \in b \cap (a \times c) \to (x, y) \in a \times {\rm pr}_{2}\langle b \rangle
\]
が成り立ち, (4), (10), (13)から, 同じく推論法則 \ref{dedmmp}によって
\[
\tag{15}
  (x, y) \in b \cap (c \times a) \to (x, y) \in {\rm pr}_{1}\langle b \rangle \times a
\]
が成り立つことがわかる.
いま定理 \ref{sthmcapgraph}, \ref{sthmproductgraph}により, 
$b \cap (a \times c)$と$b \cap (c \times a)$は共にグラフである.
また上述のように, $x$と$y$は共に
$b \cap (a \times c)$, $a \times {\rm pr}_{2}\langle b \rangle$, 
$b \cap (c \times a)$, ${\rm pr}_{1}\langle b \rangle \times a$のいずれの記号列の中にも
自由変数として現れない.
また$x$と$y$は互いに異なり, 共に定数でない.
これらのことと, (14), (15)が共に成り立つことから, 
定理 \ref{sthmgraphpairsubset}により(1), (2)が共に成り立つ.

さて次に
\[
  {\rm pr}_{2}\langle b \rangle \subset c \to b \circ {\rm id}_{a} = b \cap (a \times c), ~~
  {\rm pr}_{1}\langle b \rangle \subset c \to {\rm id}_{a} \circ b = b \cap (c \times a)
\]
が共に成り立つことを示す.
まず定理 \ref{sthmproductsubset}より
\begin{align*}
  \tag{16}
  {\rm pr}_{2}\langle b \rangle \subset c &\to a \times {\rm pr}_{2}\langle b \rangle \subset a \times c, \\
  \mbox{} \\
  \tag{17}
  {\rm pr}_{1}\langle b \rangle \subset c &\to {\rm pr}_{1}\langle b \rangle \times a \subset c \times a
\end{align*}
が共に成り立つ.
また定理 \ref{sthmcapsubset}より
\begin{align*}
  \tag{18}
  a \times {\rm pr}_{2}\langle b \rangle \subset a \times c 
  &\to b \cap (a \times {\rm pr}_{2}\langle b \rangle) \subset b \cap (a \times c), \\
  \mbox{} \\
  \tag{19}
  {\rm pr}_{1}\langle b \rangle \times a \subset c \times a 
  &\to b \cap ({\rm pr}_{1}\langle b \rangle \times a) \subset b \cap (c \times a)
\end{align*}
が共に成り立つ.
また定理 \ref{sthmcap}より
\[
  b \cap (a \times c) \subset b, ~~
  b \cap (c \times a) \subset b
\]
が共に成り立つから, これらと(1), (2)から, 定理 \ref{sthmcapdil}によって
\begin{align*}
  b \cap (a \times c) \subset b \cap (a \times {\rm pr}_{2}\langle b \rangle), \\
  \mbox{} \\
  b \cap (c \times a) \subset b \cap ({\rm pr}_{1}\langle b \rangle \times a)
\end{align*}
が共に成り立つことがわかる.
そこで推論法則 \ref{dedatawbtrue2}により
\begin{align*}
  \tag{20}
  b \cap (a \times {\rm pr}_{2}\langle b \rangle) \subset b \cap (a \times c) 
  &\to b \cap (a \times {\rm pr}_{2}\langle b \rangle) \subset b \cap (a \times c) 
  \wedge b \cap (a \times c) \subset b \cap (a \times {\rm pr}_{2}\langle b \rangle), \\
  \mbox{} \\
  \tag{21}
  b \cap ({\rm pr}_{1}\langle b \rangle \times a) \subset b \cap (c \times a) 
  &\to b \cap ({\rm pr}_{1}\langle b \rangle \times a) \subset b \cap (c \times a) 
  \wedge b \cap (c \times a) \subset b \cap ({\rm pr}_{1}\langle b \rangle \times a)
\end{align*}
が共に成り立つ.
また定理 \ref{sthmaxiom1}と推論法則 \ref{dedequiv}により
\begin{align*}
  \tag{22}
  b \cap (a \times {\rm pr}_{2}\langle b \rangle) \subset b \cap (a \times c) 
  \wedge b \cap (a \times c) \subset b \cap (a \times {\rm pr}_{2}\langle b \rangle) 
  &\to b \cap (a \times {\rm pr}_{2}\langle b \rangle) = b \cap (a \times c), \\
  \mbox{} \\
  \tag{23}
  b \cap ({\rm pr}_{1}\langle b \rangle \times a) \subset b \cap (c \times a) 
  \wedge b \cap (c \times a) \subset b \cap ({\rm pr}_{1}\langle b \rangle \times a) 
  &\to b \cap ({\rm pr}_{1}\langle b \rangle \times a) = b \cap (c \times a)
\end{align*}
が共に成り立つ.
また定理 \ref{sthmidencomp}より
\[
  b \circ {\rm id}_{a} = b \cap (a \times {\rm pr}_{2}\langle b \rangle), ~~
  {\rm id}_{a} \circ b = b \cap ({\rm pr}_{1}\langle b \rangle \times a)
\]
が共に成り立つから, 推論法則 \ref{dedaddeq=}により
\begin{align*}
  b \circ {\rm id}_{a} = b \cap (a \times c) 
  &\leftrightarrow b \cap (a \times {\rm pr}_{2}\langle b \rangle) = b \cap (a \times c), \\
  \mbox{} \\
  {\rm id}_{a} \circ b = b \cap (c \times a) 
  &\leftrightarrow b \cap ({\rm pr}_{1}\langle b \rangle \times a) = b \cap (c \times a)
\end{align*}
が共に成り立つ.
故に推論法則 \ref{dedequiv}により
\begin{align*}
  \tag{24}
  b \cap (a \times {\rm pr}_{2}\langle b \rangle) = b \cap (a \times c) 
  &\to b \circ {\rm id}_{a} = b \cap (a \times c), \\
  \mbox{} \\
  \tag{25}
  b \cap ({\rm pr}_{1}\langle b \rangle \times a) = b \cap (c \times a) 
  &\to {\rm id}_{a} \circ b = b \cap (c \times a)
\end{align*}
が共に成り立つ.
以上の(16), (18), (20), (22), (24)から, 推論法則 \ref{dedmmp}によって
\[
  {\rm pr}_{2}\langle b \rangle \subset c \to b \circ {\rm id}_{a} = b \cap (a \times c)
\]
が成り立ち, (17), (19), (21), (23), (25)から, 同じく推論法則 \ref{dedmmp}によって
\[
  {\rm pr}_{1}\langle b \rangle \subset c \to {\rm id}_{a} \circ b = b \cap (c \times a)
\]
が成り立つことがわかる.
($*$)が成り立つことは, これらと推論法則 \ref{dedmp}によって明らかである.
\halmos




\mathstrut
\begin{thm}
\label{sthmidencomp3}%定理
$a$と$b$を集合とするとき, 
\[
  {\rm Graph}(b) \wedge {\rm pr}_{1}\langle b \rangle \subset a
  \leftrightarrow b \circ {\rm id}_{a} = b, ~~
  {\rm Graph}(b) \wedge {\rm pr}_{2}\langle b \rangle \subset a
  \leftrightarrow {\rm id}_{a} \circ b = b
\]
が成り立つ.
またこれらから, 次の1), 2)が成り立つ.

1)
$b$がグラフで, かつ${\rm pr}_{1}\langle b \rangle \subset a$が成り立つならば, 
$b \circ {\rm id}_{a} = b$が成り立つ.
逆に$b \circ {\rm id}_{a} = b$が成り立つならば, 
$b$はグラフであり, かつ${\rm pr}_{1}\langle b \rangle \subset a$が成り立つ.

2)
$b$がグラフで, かつ${\rm pr}_{2}\langle b \rangle \subset a$が成り立つならば, 
${\rm id}_{a} \circ b = b$が成り立つ.
逆に${\rm id}_{a} \circ b = b$が成り立つならば, 
$b$はグラフであり, かつ${\rm pr}_{2}\langle b \rangle \subset a$が成り立つ.
\end{thm}


\noindent{\bf 証明}
~まず前半を示す.
推論法則 \ref{dedequiv}があるから, 
\begin{align*}
  \tag{1}
  &{\rm Graph}(b) \wedge {\rm pr}_{1}\langle b \rangle \subset a \to b \circ {\rm id}_{a} = b, \\
  \mbox{} \\
  \tag{2}
  &{\rm Graph}(b) \wedge {\rm pr}_{2}\langle b \rangle \subset a \to {\rm id}_{a} \circ b = b, \\
  \mbox{} \\
  \tag{3}
  &b \circ {\rm id}_{a} = b \to {\rm Graph}(b) \wedge {\rm pr}_{1}\langle b \rangle \subset a, \\
  \mbox{} \\
  \tag{4}
  &{\rm id}_{a} \circ b = b \to {\rm Graph}(b) \wedge {\rm pr}_{2}\langle b \rangle \subset a
\end{align*}
がすべて成り立つことを示せば良い.

(1)と(2)の証明: 
定理 \ref{sthmidencomp}より
\[
  b \circ {\rm id}_{a} = b \cap (a \times {\rm pr}_{2}\langle b \rangle), ~~
  {\rm id}_{a} \circ b = b \cap ({\rm pr}_{1}\langle b \rangle \times a)
\]
が共に成り立つから, 推論法則 \ref{deds1}により
\begin{align*}
  \tag{5}
  {\rm Graph}(b) \wedge {\rm pr}_{1}\langle b \rangle \subset a 
  &\to b \circ {\rm id}_{a} = b \cap (a \times {\rm pr}_{2}\langle b \rangle), \\
  \mbox{} \\
  \tag{6}
  {\rm Graph}(b) \wedge {\rm pr}_{2}\langle b \rangle \subset a 
  &\to {\rm id}_{a} \circ b = b \cap ({\rm pr}_{1}\langle b \rangle \times a)
\end{align*}
が共に成り立つ.
また定理 \ref{sthmgraphprset}と推論法則 \ref{dedequiv}により
\[
\tag{7}
  {\rm Graph}(b) \to b \subset {\rm pr}_{1}\langle b \rangle \times {\rm pr}_{2}\langle b \rangle
\]
が成り立つ.
また定理 \ref{sthmproductsubset}より
\begin{align*}
  \tag{8}
  {\rm pr}_{1}\langle b \rangle \subset a 
  &\to {\rm pr}_{1}\langle b \rangle \times {\rm pr}_{2}\langle b \rangle \subset a \times {\rm pr}_{2}\langle b \rangle, \\
  \mbox{} \\
  \tag{9}
  {\rm pr}_{2}\langle b \rangle \subset a 
  &\to {\rm pr}_{1}\langle b \rangle \times {\rm pr}_{2}\langle b \rangle \subset {\rm pr}_{1}\langle b \rangle \times a
\end{align*}
が共に成り立つ.
そこで(7)と(8), (7)と(9)から, それぞれ推論法則 \ref{dedfromaddw}によって
\begin{align*}
  \tag{10}
  {\rm Graph}(b) \wedge {\rm pr}_{1}\langle b \rangle \subset a 
  &\to b \subset {\rm pr}_{1}\langle b \rangle \times {\rm pr}_{2}\langle b \rangle 
  \wedge {\rm pr}_{1}\langle b \rangle \times {\rm pr}_{2}\langle b \rangle \subset a \times {\rm pr}_{2}\langle b \rangle, \\
  \mbox{} \\
  \tag{11}
  {\rm Graph}(b) \wedge {\rm pr}_{2}\langle b \rangle \subset a 
  &\to b \subset {\rm pr}_{1}\langle b \rangle \times {\rm pr}_{2}\langle b \rangle 
  \wedge {\rm pr}_{1}\langle b \rangle \times {\rm pr}_{2}\langle b \rangle \subset {\rm pr}_{1}\langle b \rangle \times a
\end{align*}
が成り立つ.
また定理 \ref{sthmsubsettrans}より
\begin{align*}
  \tag{12}
  b \subset {\rm pr}_{1}\langle b \rangle \times {\rm pr}_{2}\langle b \rangle 
  \wedge {\rm pr}_{1}\langle b \rangle \times {\rm pr}_{2}\langle b \rangle \subset a \times {\rm pr}_{2}\langle b \rangle 
  &\to b \subset a \times {\rm pr}_{2}\langle b \rangle, \\
  \mbox{} \\
  \tag{13}
  b \subset {\rm pr}_{1}\langle b \rangle \times {\rm pr}_{2}\langle b \rangle 
  \wedge {\rm pr}_{1}\langle b \rangle \times {\rm pr}_{2}\langle b \rangle \subset {\rm pr}_{1}\langle b \rangle \times a
  &\to b \subset {\rm pr}_{1}\langle b \rangle \times a
\end{align*}
が共に成り立つ.
また定理 \ref{sthmcapsubset=}と推論法則 \ref{dedequiv}により
\begin{align*}
  \tag{14}
  b \subset a \times {\rm pr}_{2}\langle b \rangle &\to b \cap (a \times {\rm pr}_{2}\langle b \rangle) = b, \\
  \mbox{} \\
  \tag{15}
  b \subset {\rm pr}_{1}\langle b \rangle \times a &\to b \cap ({\rm pr}_{1}\langle b \rangle \times a) = b
\end{align*}
が共に成り立つ.
そこで(10), (12), (14)から, 推論法則 \ref{dedmmp}によって
\[
\tag{16}
  {\rm Graph}(b) \wedge {\rm pr}_{1}\langle b \rangle \subset a \to b \cap (a \times {\rm pr}_{2}\langle b \rangle) = b
\]
が成り立ち, (11), (13), (15)から, 同じく推論法則 \ref{dedmmp}によって
\[
\tag{17}
  {\rm Graph}(b) \wedge {\rm pr}_{2}\langle b \rangle \subset a \to b \cap ({\rm pr}_{1}\langle b \rangle \times a) = b
\]
が成り立つことがわかる.
故に(5)と(16), (6)と(17)から, それぞれ推論法則 \ref{dedprewedge}によって
\begin{align*}
  \tag{18}
  {\rm Graph}(b) \wedge {\rm pr}_{1}\langle b \rangle \subset a 
  &\to b \circ {\rm id}_{a} = b \cap (a \times {\rm pr}_{2}\langle b \rangle) 
  \wedge b \cap (a \times {\rm pr}_{2}\langle b \rangle) = b, \\
  \mbox{} \\
  \tag{19}
  {\rm Graph}(b) \wedge {\rm pr}_{2}\langle b \rangle \subset a 
  &\to {\rm id}_{a} \circ b = b \cap ({\rm pr}_{1}\langle b \rangle \times a) 
  \wedge b \cap ({\rm pr}_{1}\langle b \rangle \times a) = b
\end{align*}
が成り立つ.
またThm \ref{x=ywy=ztx=z}より
\begin{align*}
  \tag{20}
  b \circ {\rm id}_{a} = b \cap (a \times {\rm pr}_{2}\langle b \rangle) 
  \wedge b \cap (a \times {\rm pr}_{2}\langle b \rangle) = b 
  &\to b \circ {\rm id}_{a} = b, \\
  \mbox{} \\
  \tag{21}
  {\rm id}_{a} \circ b = b \cap ({\rm pr}_{1}\langle b \rangle \times a) 
  \wedge b \cap ({\rm pr}_{1}\langle b \rangle \times a) = b 
  &\to {\rm id}_{a} \circ b = b
\end{align*}
が共に成り立つ.
そこで(18)と(20), (19)と(21)から, 推論法則 \ref{dedmmp}によってそれぞれ(1), (2)が成り立つ.

(3)と(4)の証明: 
定理 \ref{sthmgraph=}より
\begin{align*}
  b \circ {\rm id}_{a} = b &\to ({\rm Graph}(b \circ {\rm id}_{a}) \leftrightarrow {\rm Graph}(b)), \\
  \mbox{} \\
  {\rm id}_{a} \circ b = b &\to ({\rm Graph}({\rm id}_{a} \circ b) \leftrightarrow {\rm Graph}(b))
\end{align*}
が共に成り立つから, 推論法則 \ref{dedprewedge}により
\begin{align*}
  b \circ {\rm id}_{a} = b &\to ({\rm Graph}(b \circ {\rm id}_{a}) \to {\rm Graph}(b)), \\
  \mbox{} \\
  {\rm id}_{a} \circ b = b &\to ({\rm Graph}({\rm id}_{a} \circ b) \to {\rm Graph}(b))
\end{align*}
が共に成り立ち, これらから推論法則 \ref{dedch}により
\begin{align*}
  {\rm Graph}(b \circ {\rm id}_{a}) &\to (b \circ {\rm id}_{a} = b \to {\rm Graph}(b)), \\
  \mbox{} \\
  {\rm Graph}({\rm id}_{a} \circ b) &\to ({\rm id}_{a} \circ b = b \to {\rm Graph}(b))
\end{align*}
が共に成り立つ.
いま定理 \ref{sthmcompgraph}より${\rm Graph}(b \circ {\rm id}_{a})$と
${\rm Graph}({\rm id}_{a} \circ b)$が共に成り立つから, 
従って推論法則 \ref{dedmp}により, 
\begin{align*}
  \tag{22}
  b \circ {\rm id}_{a} = b &\to {\rm Graph}(b), \\
  \mbox{} \\
  \tag{23}
  {\rm id}_{a} \circ b = b &\to {\rm Graph}(b)
\end{align*}
が共に成り立つ.
また定理 \ref{sthm=tsubset}より
\begin{align*}
  \tag{24}
  b \circ {\rm id}_{a} = b &\to b \subset b \circ {\rm id}_{a}, \\
  \mbox{} \\
  \tag{25}
  {\rm id}_{a} \circ b = b &\to b \subset {\rm id}_{a} \circ b
\end{align*}
が共に成り立つ.
また定理 \ref{sthmprsetsubset}より
\begin{align*}
  \tag{26}
  b \subset b \circ {\rm id}_{a} &\to {\rm pr}_{1}\langle b \rangle \subset {\rm pr}_{1}\langle b \circ {\rm id}_{a} \rangle, \\
  \mbox{} \\
  \tag{27}
  b \subset {\rm id}_{a} \circ b &\to {\rm pr}_{2}\langle b \rangle \subset {\rm pr}_{2}\langle {\rm id}_{a} \circ b \rangle
\end{align*}
が共に成り立つ.
また定理 \ref{sthmprsetiden}より
\[
  {\rm pr}_{1}\langle {\rm id}_{a} \rangle = a, ~~
  {\rm pr}_{2}\langle {\rm id}_{a} \rangle = a
\]
が共に成り立ち, 定理 \ref{sthmprsetcomp2}より
\[
  {\rm pr}_{1}\langle b \circ {\rm id}_{a} \rangle \subset {\rm pr}_{1}\langle {\rm id}_{a} \rangle, ~~
  {\rm pr}_{2}\langle {\rm id}_{a} \circ b \rangle \subset {\rm pr}_{2}\langle {\rm id}_{a} \rangle
\]
が共に成り立つから, 定理 \ref{sthm=tsubseteq}により
\[
  {\rm pr}_{1}\langle b \circ {\rm id}_{a} \rangle \subset a, ~~
  {\rm pr}_{2}\langle {\rm id}_{a} \circ b \rangle \subset a
\]
が共に成り立つ.
故にこれらから, 推論法則 \ref{dedatawbtrue2}により
\begin{align*}
  \tag{28}
  {\rm pr}_{1}\langle b \rangle \subset {\rm pr}_{1}\langle b \circ {\rm id}_{a} \rangle 
  &\to {\rm pr}_{1}\langle b \rangle \subset {\rm pr}_{1}\langle b \circ {\rm id}_{a} \rangle 
  \wedge {\rm pr}_{1}\langle b \circ {\rm id}_{a} \rangle \subset a, \\
  \mbox{} \\
  \tag{29}
  {\rm pr}_{2}\langle b \rangle \subset {\rm pr}_{2}\langle {\rm id}_{a} \circ b \rangle 
  &\to {\rm pr}_{2}\langle b \rangle \subset {\rm pr}_{2}\langle {\rm id}_{a} \circ b \rangle 
  \wedge {\rm pr}_{2}\langle {\rm id}_{a} \circ b \rangle \subset a
\end{align*}
が共に成り立つ.
また定理 \ref{sthmsubsettrans}より
\begin{align*}
  \tag{30}
  {\rm pr}_{1}\langle b \rangle \subset {\rm pr}_{1}\langle b \circ {\rm id}_{a} \rangle 
  \wedge {\rm pr}_{1}\langle b \circ {\rm id}_{a} \rangle \subset a 
  &\to {\rm pr}_{1}\langle b \rangle \subset a, \\
  \mbox{} \\
  \tag{31}
  {\rm pr}_{2}\langle b \rangle \subset {\rm pr}_{2}\langle {\rm id}_{a} \circ b \rangle 
  \wedge {\rm pr}_{2}\langle {\rm id}_{a} \circ b \rangle \subset a 
  &\to {\rm pr}_{2}\langle b \rangle \subset a
\end{align*}
が共に成り立つ.
そこで(24), (26), (28), (30)から, 推論法則 \ref{dedmmp}によって
\[
\tag{32}
  b \circ {\rm id}_{a} = b \to {\rm pr}_{1}\langle b \rangle \subset a
\]
が成り立ち, (25), (27), (29), (31)から, 同じく推論法則 \ref{dedmmp}によって
\[
\tag{33}
  {\rm id}_{a} \circ b = b \to {\rm pr}_{2}\langle b \rangle \subset a
\]
が成り立つことがわかる.
故に(22)と(32), (23)と(33)から, 推論法則 \ref{dedprewedge}によってそれぞれ(3), (4)が成り立つ.

\noindent
1)
$b$がグラフで, かつ${\rm pr}_{1}\langle b \rangle \subset a$が成り立つならば, 
推論法則 \ref{dedwedge}により${\rm Graph}(b) \wedge {\rm pr}_{1}\langle b \rangle \subset a$が成り立つから, 
これと上に示した(1)から, 推論法則 \ref{dedmp}によって$b \circ {\rm id}_{a} = b$が成り立つ.
逆に$b \circ {\rm id}_{a} = b$が成り立つならば, 
これと上に示した(3)から, 推論法則 \ref{dedmp}によって
${\rm Graph}(b) \wedge {\rm pr}_{1}\langle b \rangle \subset a$が成り立つから, 
推論法則 \ref{dedwedge}により, $b$はグラフであり, かつ${\rm pr}_{1}\langle b \rangle \subset a$が成り立つ.

\noindent
2)
$b$がグラフで, かつ${\rm pr}_{2}\langle b \rangle \subset a$が成り立つならば, 
推論法則 \ref{dedwedge}により${\rm Graph}(b) \wedge {\rm pr}_{2}\langle b \rangle \subset a$が成り立つから, 
これと上に示した(2)から, 推論法則 \ref{dedmp}によって${\rm id}_{a} \circ b = b$が成り立つ.
逆に${\rm id}_{a} \circ b = b$が成り立つならば, 
これと上に示した(4)から, 推論法則 \ref{dedmp}によって
${\rm Graph}(b) \wedge {\rm pr}_{2}\langle b \rangle \subset a$が成り立つから, 
推論法則 \ref{dedwedge}により, $b$はグラフであり, かつ${\rm pr}_{2}\langle b \rangle \subset a$が成り立つ.
\halmos
%[id]確認済




























\newpage
\setcounter{defi}{0}
\section{グラフを与える関係}%%%%%%%%%%%%%%%%%%%%%%%%%%%%%%%%%%%%%%%%%%%%%%%%%%%%%%%%%%%%%%%




\mathstrut
\begin{defo}
\label{ggraph}%変形
$R$を記号列とし, $x$と$y$を互いに異なる文字とする.
また$z$と$w$を, 共に$x$とも$y$とも異なり, $R$の中に自由変数として現れない文字とする.
このとき
\[
  \exists z({\rm Graph}(z) \wedge \forall x(\forall y(R \leftrightarrow (x, y) \in z))) \equiv 
  \exists w({\rm Graph}(w) \wedge \forall x(\forall y(R \leftrightarrow (x, y) \in w)))
\]
が成り立つ.
\end{defo}


\noindent{\bf 証明}
~$z$と$w$が同じ文字であるときは明らか.
$z$と$w$が異なる文字であるとき, 仮定より$w$は$x$とも$y$とも異なり, 
$R$の中に自由変数として現れないから, 変数法則 \ref{valfund}, \ref{valwedge}, 
\ref{valequiv}, \ref{valquan}, \ref{valpair}, \ref{valgraph}からわかるように, 
$w$は${\rm Graph}(z) \wedge \forall x(\forall y(R \leftrightarrow (x, y) \in z))$の中にも
自由変数として現れない.
よって代入法則 \ref{substquantrans}により
\[
\tag{1}
  \exists z({\rm Graph}(z) \wedge \forall x(\forall y(R \leftrightarrow (x, y) \in z))) \equiv 
  \exists w((w|z)({\rm Graph}(z) \wedge \forall x(\forall y(R \leftrightarrow (x, y) \in z))))
\]
が成り立つ.
また代入法則 \ref{substwedge}, \ref{substgraph}により
\[
\tag{2}
  (w|z)({\rm Graph}(z) \wedge \forall x(\forall y(R \leftrightarrow (x, y) \in z))) \equiv 
  {\rm Graph}(w) \wedge (w|z)(\forall x(\forall y(R \leftrightarrow (x, y) \in z)))
\]
が成り立つ.
また$x$と$y$が共に$z$とも$w$とも異なることから, 
代入法則 \ref{substquan}により
\[
\tag{3}
  (w|z)(\forall x(\forall y(R \leftrightarrow (x, y) \in z))) \equiv 
  \forall x(\forall y((w|z)(R \leftrightarrow (x, y) \in z)))
\]
が成り立つ.
また$z$が$x$とも$y$とも異なり, $R$の中に自由変数として現れないことから, 
代入法則 \ref{substfree}, \ref{substfund}, \ref{substequiv}, \ref{substpair}により
\[
\tag{4}
  (w|z)(R \leftrightarrow (x, y) \in z) \equiv 
  R \leftrightarrow (x, y) \in z
\]
が成り立つ.
そこで(1)---(4)から, 
$\exists z({\rm Graph}(z) \wedge \forall x(\forall y(R \leftrightarrow (x, y) \in z)))$が
$\exists w({\rm Graph}(w) \wedge \forall x(\forall y(R \leftrightarrow (x, y) \in w)))$と
一致することがわかる.
\halmos




\mathstrut
\begin{defi}
\label{defggraph}%定義
$R$を記号列とし, $x$と$y$を互いに異なる文字とする.
また$z$と$w$を, 共に$x$とも$y$とも異なり, $R$の中に自由変数として現れない文字とする.
このとき上記の変形法則 \ref{ggraph}によれば, 
$\exists z({\rm Graph}(z) \wedge \forall x(\forall y(R \leftrightarrow (x, y) \in z)))$と
$\exists w({\rm Graph}(w) \wedge \forall x(\forall y(R \leftrightarrow (x, y) \in w)))$という
二つの記号列は一致する.
$R$, $x$, $y$に対して定まるこの記号列を, 
${\rm Graph}_{x, y}(R)$と書き表す.
\end{defi}




\mathstrut
\begin{valu}
\label{valggraph}%変数
$R$を記号列とし, $x$と$y$を互いに異なる文字とする.

1) $x$と$y$は共に${\rm Graph}_{x, y}(R)$の中に自由変数として現れない.

2) $z$を文字とする.
   $z$が$R$の中に自由変数として現れなければ, 
   $z$は${\rm Graph}_{x, y}(R)$の中にも自由変数として現れない.
\end{valu}


\noindent{\bf 証明}
~1)
$w$を$x$とも$y$とも異なり, $R$の中に自由変数として現れない文字とすれば, 
定義から${\rm Graph}_{x, y}(R)$は
$\exists w({\rm Graph}(w) \wedge \forall x(\forall y(R \leftrightarrow (x, y) \in w)))$である.
ここで$w$が$x$とも$y$とも異なることから, 
変数法則 \ref{valgraph}により, $x$と$y$は共に${\rm Graph}(w)$の中に自由変数として現れない.
また変数法則 \ref{valquan}により, $x$と$y$は共に
$\forall x(\forall y(R \leftrightarrow (x, y) \in w))$の中にも自由変数として現れない.
よって変数法則 \ref{valwedge}, \ref{valquan}により, $x$と$y$は共に
$\exists w({\rm Graph}(w) \wedge \forall x(\forall y(R \leftrightarrow (x, y) \in w)))$, 即ち
${\rm Graph}_{x, y}(R)$の中にも自由変数として現れない.

\noindent
2)
$z$が$x$または$y$と同じ文字である場合には1)により明らか.
$z$が$x$とも$y$とも異なる文字である場合は, 
定義から${\rm Graph}_{x, y}(R)$は
$\exists z({\rm Graph}(z) \wedge \forall x(\forall y(R \leftrightarrow (x, y) \in z)))$であるから, 
変数法則 \ref{valquan}により, $z$はこの中に自由変数として現れない.
\halmos




\mathstrut
\begin{subs}
\label{substggraphtrans}%代入
$R$を記号列とし, $x$と$y$を互いに異なる文字とする.

1) $z$を$y$と異なり, $R$の中に自由変数として現れない文字とする.
   このとき
   \[
     {\rm Graph}_{x, y}(R) \equiv {\rm Graph}_{z, y}((z|x)(R))
   \]
   が成り立つ.

2) $w$を$x$と異なり, $R$の中に自由変数として現れない文字とする.
   このとき
   \[
     {\rm Graph}_{x, y}(R) \equiv {\rm Graph}_{x, w}((w|y)(R))
   \]
   が成り立つ.

3) $z$と$w$を, 互いに異なり, それぞれ$y$, $x$と異なり, 
   共に$R$の中に自由変数として現れない文字とする.
   このとき
   \[
     {\rm Graph}_{x, y}(R) \equiv {\rm Graph}_{z, w}((w|y)((z|x)(R))), ~~
     {\rm Graph}_{x, y}(R) \equiv {\rm Graph}_{z, w}((z|x)((w|y)(R)))
   \]
   が成り立つ.
\end{subs}


\noindent{\bf 証明}
~




\mathstrut
\begin{subs}
\label{substggraph}%代入
$a$と$R$を記号列とし, $x$と$y$を互いに異なる文字とする.
また$z$を$x$とも$y$とも異なる文字とする.
$x$と$y$が共に$a$の中に自由変数として現れなければ, 
\[
  (a|z)({\rm Graph}_{x, y}(R)) \equiv {\rm Graph}_{x, y}((a|z)(R))
\]
が成り立つ.
\end{subs}


\noindent{\bf 証明}
~




\mathstrut
\begin{form}
\label{formggraph}%構成
$R$を関係式とし, $x$と$y$を互いに異なる文字とする.
このとき${\rm Graph}_{x, y}(R)$は関係式である.
\end{form}


\noindent{\bf 証明}
~




\mathstrut
\begin{defo}
\label{rgraph}%変形
$R$を記号列とし, $x$と$y$を互いに異なる文字とする.
また$z$と$w$を, 共に$x$とも$y$とも異なり, $R$の中に自由変数として現れない文字とする.
このとき
\[
  \tau_{z}({\rm Graph}(z) \wedge \forall x(\forall y(R \leftrightarrow (x, y) \in z))) \equiv
  \tau_{w}({\rm Graph}(w) \wedge \forall x(\forall y(R \leftrightarrow (x, y) \in w)))
\]
が成り立つ.
\end{defo}


\noindent{\bf 証明}
~$z$と$w$が同じ文字であるときは明らか.
$z$と$w$が異なる文字であるとき, 仮定より$w$は$x$とも$y$とも異なり, 
$R$の中に自由変数として現れないから, 変数法則 \ref{valfund}, \ref{valwedge}, 
\ref{valequiv}, \ref{valquan}, \ref{valpair}, \ref{valgraph}からわかるように, 
$w$は${\rm Graph}(z) \wedge \forall x(\forall y(R \leftrightarrow (x, y) \in z))$の中にも
自由変数として現れない.
よって代入法則 \ref{substtautrans}により
\[
\tag{1}
  \tau_{z}({\rm Graph}(z) \wedge \forall x(\forall y(R \leftrightarrow (x, y) \in z))) \equiv 
  \tau_{w}((w|z)({\rm Graph}(z) \wedge \forall x(\forall y(R \leftrightarrow (x, y) \in z))))
\]
が成り立つ.
また代入法則 \ref{substwedge}, \ref{substgraph}により
\[
\tag{2}
  (w|z)({\rm Graph}(z) \wedge \forall x(\forall y(R \leftrightarrow (x, y) \in z))) \equiv 
  {\rm Graph}(w) \wedge (w|z)(\forall x(\forall y(R \leftrightarrow (x, y) \in z)))
\]
が成り立つ.
また$x$と$y$が共に$z$とも$w$とも異なることから, 
代入法則 \ref{substquan}により
\[
\tag{3}
  (w|z)(\forall x(\forall y(R \leftrightarrow (x, y) \in z))) \equiv 
  \forall x(\forall y((w|z)(R \leftrightarrow (x, y) \in z)))
\]
が成り立つ.
また$z$が$x$とも$y$とも異なり, $R$の中に自由変数として現れないことから, 
代入法則 \ref{substfree}, \ref{substfund}, \ref{substequiv}, \ref{substpair}により
\[
\tag{4}
  (w|z)(R \leftrightarrow (x, y) \in z) \equiv 
  R \leftrightarrow (x, y) \in z
\]
が成り立つ.
そこで(1)---(4)から, 
$\tau_{z}({\rm Graph}(z) \wedge \forall x(\forall y(R \leftrightarrow (x, y) \in z)))$が
$\tau_{w}({\rm Graph}(w) \wedge \forall x(\forall y(R \leftrightarrow (x, y) \in w)))$と
一致することがわかる.
\halmos








\mathstrut
\begin{defi}
\label{defrgraph}%定義
$R$を記号列とし, $x$と$y$を互いに異なる文字とする.
また$z$と$w$を, 共に$x$とも$y$とも異なり, $R$の中に自由変数として現れない文字とする.
このとき上記の変形法則 \ref{ggraph}によれば, 
$\tau_{z}({\rm Graph}(z) \wedge \forall x(\forall y(R \leftrightarrow (x, y) \in z)))$と
$\tau_{w}({\rm Graph}(w) \wedge \forall x(\forall y(R \leftrightarrow (x, y) \in w)))$という
二つの記号列は一致する.
$R$, $x$, $y$に対して定まるこの記号列を, 
${\rm G}_{x, y}(R)$と書き表す.
\end{defi}






\newpage
\setcounter{defi}{0}
\section{函数}%%%%%%%%%%%%%%%%%%%%%%%%%%%%%%%%%%%%%%%%%%%%%%%%%%%%%%%%%%%%%%%%%%%%%%%%%%%%%%%%%%




この節では, 数学において重要な概念である函数(写像)を定義し, その諸性質について述べる.




\mathstrut
\begin{defo}
\label{function}%変形%確認済
$f$を記号列とする.
また$x$と$y$を, 互いに異なり, 共に$f$の中に自由変数として現れない文字とする.
同様に$z$と$w$を, 互いに異なり, 共に$f$の中に自由変数として現れない文字とする.
このとき
\[
  \forall x(!y((x, y) \in f)) \equiv \forall z(!w((z, w) \in f))
\]
が成り立つ.
\end{defo}


\noindent{\bf 証明}
~$u$と$v$を, 互いに異なり, 共に$x$, $y$, $z$, $w$のいずれとも異なり, 
$f$の中に自由変数として現れない文字とする.
このとき変数法則 \ref{valfund}, \ref{val!}, \ref{valpair}によってわかるように, 
$u$は$!y((x, y) \in f)$の中に自由変数として現れないから, 
代入法則 \ref{substquantrans}により
\[
\tag{1}
  \forall x(!y((x, y) \in f)) \equiv \forall u((u|x)(!y((x, y) \in f)))
\]
が成り立つ.
また$y$が$x$とも$u$とも異なることから, 代入法則 \ref{subst!}により
\[
\tag{2}
  (u|x)(!y((x, y) \in f)) \equiv \ !y((u|x)((x, y) \in f))
\]
が成り立つ.
また$x$が$y$と異なり, $f$の中に自由変数として現れないことから, 
代入法則 \ref{substfree}, \ref{substfund}, \ref{substpair}により
\[
\tag{3}
  (u|x)((x, y) \in f) \equiv (u, y) \in f
\]
が成り立つ.
そこで(1), (2), (3)から, 
\[
\tag{4}
  \forall x(!y((x, y) \in f)) \equiv \forall u(!y((u, y) \in f))
\]
が成り立つことがわかる.
また$v$が$y$とも$u$とも異なり, $f$の中に自由変数として現れないことから, 
変数法則 \ref{valfund}, \ref{valpair}によって$v$が
$(u, y) \in f$の中に自由変数として現れないことがわかるから, 
代入法則 \ref{subst!trans}により
\[
\tag{5}
  !y((u, y) \in f) \equiv \ !v((v|y)((u, y) \in f))
\]
が成り立つ.
また$y$が$u$と異なり, $f$の中に自由変数として現れないことから, 
代入法則 \ref{substfree}, \ref{substfund}, \ref{substpair}により
\[
\tag{6}
  (v|y)((u, y) \in f) \equiv (u, v) \in f
\]
が成り立つ.
そこで(5), (6)から, 
\[
\tag{7}
  !y((u, y) \in f) \equiv \ !v((u, v) \in f)
\]
が成り立つ.
故に(4), (7)から, 
\[
  \forall x(!y((x, y) \in f)) \equiv \forall u(!v((u, v) \in f))
\]
が成り立つことがわかる.
ここまでの議論と全く同様にして
\[
  \forall z(!w((z, w) \in f)) \equiv \forall u(!v((u, v) \in f))
\]
も成り立つから, 従って$\forall x(!y((x, y) \in f))$と$\forall z(!w((z, w) \in f))$という
二つの記号列は一致する.
\halmos




\mathstrut
\begin{defi}
\label{deffunc}%定義%確認済
$f$を記号列とする.
また$x$と$y$を, 互いに異なり, 共に$f$の中に自由変数として現れない文字とする.
同様に$z$と$w$を, 互いに異なり, 共に$f$の中に自由変数として現れない文字とする.
このとき上記の変形法則 \ref{function}によれば, 
${\rm Graph}(f) \wedge \forall x(!y((x, y) \in f))$と
${\rm Graph}(f) \wedge \forall z(!w((z, w) \in f))$という
二つの記号列は一致する.
$f$に対して定まるこの記号列を, ${\rm Func}(f)$と書き表す.
\end{defi}




\mathstrut
\begin{valu}
\label{valfunc}%変数%確認済
$f$を記号列とし, $x$を文字とする.
$x$が$f$の中に自由変数として現れなければ, 
$x$は${\rm Func}(f)$の中に自由変数として現れない.
\end{valu}


\noindent{\bf 証明}
~このとき, $y$を$x$と異なり, $f$の中に自由変数として現れない文字とすれば, 
定義から${\rm Func}(f)$は${\rm Graph}(f) \wedge \forall x(!y((x, y) \in f))$と同じである.
変数法則 \ref{valwedge}, \ref{valquan}, \ref{valgraph}によってわかるように, 
$x$はこの中に自由変数として現れない.
\halmos




\mathstrut
\begin{subs}
\label{substfunc}%代入%確認済
$a$と$f$を記号列とし, $x$を文字とするとき, 
\[
  (a|x)({\rm Func}(f)) \equiv {\rm Func}((a|x)(f))
\]
が成り立つ.
\end{subs}


\noindent{\bf 証明}
~$u$, $v$を, 互いに異なり, 共に$x$と異なり, $a$及び$f$の中に自由変数として現れない文字とする.
このとき定義から, ${\rm Func}(f)$は${\rm Graph}(f) \wedge \forall u(!v((u, v) \in f))$と同じだから, 
代入法則 \ref{substwedge}により
\[
\tag{1}
  (a|x)({\rm Func}(f)) \equiv (a|x)({\rm Graph}(f)) \wedge (a|x)(\forall u(!v((u, v) \in f)))
\]
が成り立つ.
また代入法則 \ref{substgraph}により
\[
\tag{2}
  (a|x)({\rm Graph}(f)) \equiv {\rm Graph}((a|x)(f))
\]
が成り立つ.
また$u$が$x$と異なり, $a$の中に自由変数として現れないことから, 
代入法則 \ref{substquan}により
\[
\tag{3}
  (a|x)(\forall u(!v((u, v) \in f))) \equiv \forall u((a|x)(!v((u, v) \in f)))
\]
が成り立つ.
また$v$も$x$と異なり, $a$の中に自由変数として現れないから, 
代入法則 \ref{subst!}により
\[
\tag{4}
  (a|x)(!v((u, v) \in f)) \equiv \ !v((a|x)((u, v) \in f))
\]
が成り立つ.
また$u$と$v$が共に$x$と異なることから, 変数法則 \ref{valpair}により
$x$は$(u, v)$の中に自由変数として現れないから, 代入法則 \ref{substfree}, \ref{substfund}により
\[
\tag{5}
  (a|x)((u, v) \in f) \equiv (u, v) \in (a|x)(f)
\]
が成り立つ.
以上の(1)---(5)から, $(a|x)({\rm Func}(f))$が
\[
  {\rm Graph}((a|x)(f)) \wedge \forall u(!v((u, v) \in (a|x)(f)))
\]
と一致することがわかる.
いま$u$と$v$は共に$a$及び$f$の中に自由変数として現れないから, 
変数法則 \ref{valsubst}により, $u$と$v$は共に$(a|x)(f)$の中に
自由変数として現れない.
このことと, $u$と$v$が互いに異なることから, 定義により
上記の記号列が${\rm Func}((a|x)(f))$と書き表される記号列であることがわかる.
故に本法則が成り立つ.
\halmos




\mathstrut
\begin{form}
\label{formfunc}%構成%確認済
$f$が集合ならば, ${\rm Func}(f)$は関係式である.
\end{form}


\noindent{\bf 証明}
~$x$と$y$を, 互いに異なり, 共に$f$の中に自由変数として現れない文字とすれば, 
定義から${\rm Func}(f)$は${\rm Graph}(f) \wedge \forall x(!y((x, y) \in f))$と同じである.
$f$が集合ならば, 
構成法則 \ref{formfund}, \ref{formwedge}, \ref{formquan}, \ref{form!}, \ref{formpair}, \ref{formgraph}に
よって直ちにわかるように, これは関係式である.
\halmos




\mathstrut
集合$f$に対して関係式${\rm Func}(f)$が定理となるとき, 
$f$は\textbf{函数}(function)である, あるいは, \textbf{写像}(mapping)であるという.




\mathstrut
\begin{defi}
\label{defonafunc}%定義%確認済
$a$と$f$を記号列とするとき, 
${\rm Func}(f) \wedge {\rm pr}_{1}\langle f \rangle = a$という記号列を, 
${\rm Func}(f; a)$と書き表す.
\end{defi}




\mathstrut
\begin{valu}
\label{valonafunc}%変数%確認済
$a$と$f$を記号列とし, $x$を文字とする.
$x$が$a$及び$f$の中に自由変数として現れなければ, 
$x$は${\rm Func}(f; a)$の中に自由変数として現れない.
\end{valu}


\noindent{\bf 証明}
~このとき変数法則\ref{valfunc}により, $x$は${\rm Func}(f)$の中に自由変数として現れない.
また変数法則 \ref{valfund}, \ref{valprset}により, $x$は${\rm pr}_{1}\langle f \rangle = a$の中にも
自由変数として現れない.
故に変数法則 \ref{valwedge}により, $x$は
${\rm Func}(f) \wedge {\rm pr}_{1}\langle f \rangle = a$, 即ち
${\rm Func}(f; a)$の中にも自由変数として現れない.
\halmos




\mathstrut
\begin{subs}
\label{substonafunc}%代入%確認済
$a$, $b$, $f$を記号列とし, $x$を文字とするとき, 
\[
  (b|x)({\rm Func}(f; a)) \equiv {\rm Func}((b|x)(f); (b|x)(a))
\]
が成り立つ.
\end{subs}


\noindent{\bf 証明}
~定義から, ${\rm Func}(f; a)$は${\rm Func}(f) \wedge {\rm pr}_{1}\langle f \rangle = a$であるから, 
代入法則 \ref{substfund}, \ref{substwedge}, \ref{substprset}, \ref{substfunc}によってわかるように, 
$(b|x)({\rm Func}(f; a))$は
\[
  {\rm Func}((b|x)(f)) \wedge {\rm pr}_{1}\langle (b|x)(f) \rangle = (b|x)(a)
\]
と一致する.
定義によれば, これは${\rm Func}((b|x)(f); (b|x)(a))$と書き表される記号列である.
\halmos




\mathstrut
\begin{form}
\label{formonafunc}%構成%確認済
$a$と$f$が集合ならば, ${\rm Func}(f; a)$は関係式である.
\end{form}


\noindent{\bf 証明}
~このとき構成法則 \ref{formfunc}により${\rm Func}(f)$は関係式であり, 
構成法則 \ref{formfund}, \ref{formprset}により${\rm pr}_{1}\langle f \rangle = a$も関係式であるから, 
構成法則 \ref{formwedge}により, ${\rm Func}(f) \wedge {\rm pr}_{1}\langle f \rangle = a$, 即ち
${\rm Func}(f; a)$も関係式である.
\halmos




\mathstrut
集合$a$及び$f$に対して関係式${\rm Func}(f; a)$が定理となるとき, 
$f$は\textbf{${\bm a}$で定義された函数}(\textbf{写像})である, 
\textbf{${\bm a}$上の函数}(\textbf{写像})である, 
\textbf{${\bm a}$における函数}(\textbf{写像})であるなどという.




\mathstrut
\begin{defi}
\label{defatbfunc}%定義%確認済
$a$, $b$, $f$を記号列とするとき, 
${\rm Func}(f; a) \wedge {\rm pr}_{2}\langle f \rangle \subset b$という
記号列を, ${\rm Func}(f; a; b)$と書き表す.
\end{defi}




\mathstrut
\begin{valu}
\label{valatbfunc}%変数%確認済
$a$, $b$, $f$を記号列とし, $x$を文字とする.
$x$が$a$, $b$, $f$のいずれの記号列の中にも自由変数として現れなければ, 
$x$は${\rm Func}(f; a; b)$の中に自由変数として現れない.
\end{valu}


\noindent{\bf 証明}
~このとき変数法則\ref{valonafunc}により, $x$は${\rm Func}(f; a)$の中に自由変数として現れない.
また変数法則 \ref{valsubset}, \ref{valprset}により, $x$は${\rm pr}_{2}\langle f \rangle \subset b$の中にも
自由変数として現れない.
故に変数法則 \ref{valwedge}により, $x$は
${\rm Func}(f; a) \wedge {\rm pr}_{2}\langle f \rangle \subset b$, 即ち
${\rm Func}(f; a; b)$の中にも自由変数として現れない.
\halmos




\mathstrut
\begin{subs}
\label{substatbfunc}%代入%確認済
$a$, $b$, $c$, $f$を記号列とし, $x$を文字とするとき, 
\[
  (c|x)({\rm Func}(f; a; b)) \equiv {\rm Func}((c|x)(f); (c|x)(a); (c|x)(b))
\]
が成り立つ.
\end{subs}


\noindent{\bf 証明}
~定義から, ${\rm Func}(f; a; b)$は${\rm Func}(f; a) \wedge {\rm pr}_{2}\langle f \rangle \subset b$であるから, 
代入法則 \ref{substwedge}, \ref{substsubset}, \ref{substprset}, \ref{substonafunc}によってわかるように, 
$(c|x)({\rm Func}(f; a; b))$は
\[
  {\rm Func}((c|x)(f); (c|x)(a)) \wedge {\rm pr}_{2}\langle (c|x)(f) \rangle \subset (c|x)(b)
\]
と一致する.
定義によれば, これは${\rm Func}((c|x)(f); (c|x)(a); (c|x)(b))$と書き表される記号列である.
\halmos




\mathstrut
\begin{form}
\label{formatbfunc}%構成%確認済
$a$, $b$, $f$が集合ならば, ${\rm Func}(f; a; b)$は関係式である.
\end{form}


\noindent{\bf 証明}
~このとき構成法則 \ref{formonafunc}により${\rm Func}(f; a)$は関係式であり, 
構成法則 \ref{formsubset}, \ref{formprset}により${\rm pr}_{2}\langle f \rangle \subset b$も関係式であるから, 
構成法則 \ref{formwedge}により, ${\rm Func}(f; a) \wedge {\rm pr}_{2}\langle f \rangle \subset b$, 即ち
${\rm Func}(f; a; b)$も関係式である.
\halmos




\mathstrut
集合$a$, $b$, $f$に対して関係式${\rm Func}(f; a; b)$が定理となるとき, 
$f$は\textbf{${\bm a}$から${\bm b}$への函数}(\textbf{写像})である, 
\textbf{${\bm a}$で定義され, ${\bm b}$に値を取る函数}(\textbf{写像})であるなどという.




\mathstrut
さてまずは定義から直ちに得られる次の定理を証明しておく.




\mathstrut
\begin{thm}
\label{sthmfuncbasis}%定理
$f$を集合とする.
また$x$と$y$を, 互いに異なり, 共に$f$の中に自由変数として現れない文字とする.

1)
このとき
\[
  {\rm Func}(f) \to {\rm Graph}(f), ~~
  {\rm Func}(f) \to \forall x(!y((x, y) \in f))
\]
が成り立つ.
また次の($*$)が成り立つ: 

($*$) ~~$f$が函数ならば, $f$はグラフであり, かつ$\forall x(!y((x, y) \in f))$が成り立つ.

2)
$a$を集合とするとき, 
\begin{align*}
  {\rm Func}(f; a) \to {\rm Func}(f)&, ~~
  {\rm Func}(f; a) \to {\rm Graph}(f), \\
  \mbox{} \\
  {\rm Func}(f; a) \to \forall x(!y((x, y) \in f))&, ~~
  {\rm Func}(f; a) \to {\rm pr}_{1}\langle f \rangle = a
\end{align*}
が成り立つ.
また次の($**$)が成り立つ: 

($**$) ~~$f$が$a$における函数ならば, $f$は函数である.
         また$f$はグラフであり, $\forall x(!y((x, y) \in f))$及び
         ${\rm pr}_{1}\langle f \rangle = a$が成り立つ.

3)
$a$と$b$を集合とするとき, 
\begin{align*}
  {\rm Func}(f; a; b) \to {\rm Func}(f; a)&, ~~
  {\rm Func}(f; a; b) \to {\rm Func}(f), \\
  \mbox{} \\
  {\rm Func}(f; a; b) \to {\rm Graph}(f)&, ~~
  {\rm Func}(f; a; b) \to \forall x(!y((x, y) \in f)), \\
  \mbox{} \\
  {\rm Func}(f; a; b) \to {\rm pr}_{1}\langle f \rangle = a&, ~~
  {\rm Func}(f; a; b) \to {\rm pr}_{2}\langle f \rangle \subset b
\end{align*}
が成り立つ.
また次の(${**}*$)が成り立つ: 

(${**}*$) ~~$f$が$a$から$b$への函数ならば, $f$は$a$における函数であり, 
            かつ$f$は函数である.
            また$f$はグラフであり, 
            $\forall x(!y((x, y) \in f))$, 
            ${\rm pr}_{1}\langle f \rangle = a$, 
            ${\rm pr}_{2}\langle f \rangle \subset b$が
            すべて成り立つ.
\end{thm}


\noindent{\bf 証明}
~1)
$x$と$y$に対する仮定及び定義より, ${\rm Func}(f)$は
${\rm Graph}(f) \wedge \forall x(!y((x, y) \in f))$と同じだから, 
1)が成り立つことはThm \ref{awbta}と推論法則 \ref{dedwedge}により明らか.

\noindent
2)
定義より${\rm Func}(f; a)$は${\rm Func}(f) \wedge {\rm pr}_{1}\langle f \rangle = a$であるから, 
2)の第一, 第四の記号列が共に定理となることはThm \ref{awbta}より明らか.
また1)より
\[
  {\rm Func}(f) \to {\rm Graph}(f), ~~
  {\rm Func}(f) \to \forall x(!y((x, y) \in f))
\]
が共に成り立つから, これらといま示した${\rm Func}(f; a) \to {\rm Func}(f)$から, 
推論法則 \ref{dedmmp}によって2)の第二, 第三の記号列も共に定理となることがわかる.
($**$)が成り立つことは, これらと推論法則 \ref{dedmp}により明らかである.

\noindent
3)
定義より${\rm Func}(f; a; b)$は${\rm Func}(f; a) \wedge {\rm pr}_{2}\langle f \rangle \subset b$であるから, 
3)の第一, 第六の記号列が共に定理となることはThm \ref{awbta}より明らか.
また2)より
\begin{align*}
  {\rm Func}(f; a) \to {\rm Func}(f)&, ~~
  {\rm Func}(f; a) \to {\rm Graph}(f), \\
  \mbox{} \\
  {\rm Func}(f; a) \to \forall x(!y((x, y) \in f))&, ~~
  {\rm Func}(f; a) \to {\rm pr}_{1}\langle f \rangle = a
\end{align*}
がすべて成り立つから, これらといま示した${\rm Func}(f; a; b) \to {\rm Func}(f; a)$から, 
推論法則 \ref{dedmmp}によって3)の残りの四つの記号列もすべて定理となることがわかる.
(${**}*$)が成り立つことは, これらと推論法則 \ref{dedmp}により明らかである.
\halmos




\mathstrut
\begin{thm}
\label{sthmfuncrelation}%定理
$a$と$f$を集合とするとき, 
\begin{align*}
  {\rm Func}(f) &\leftrightarrow {\rm Func}(f; {\rm pr}_{1}\langle f \rangle), \\
  \mbox{} \\
  {\rm Func}(f) &\leftrightarrow {\rm Func}(f; {\rm pr}_{1}\langle f \rangle; {\rm pr}_{2}\langle f \rangle), \\
  \mbox{} \\
  {\rm Func}(f; a) &\leftrightarrow {\rm Func}(f; a; {\rm pr}_{2}\langle f \rangle)
\end{align*}
が成り立つ.
またこれらから, 特に次の($*$)が成り立つ: 

($*$) ~~$f$が函数ならば, $f$は${\rm pr}_{1}\langle f \rangle$における函数であり, 
         かつ$f$は${\rm pr}_{1}\langle f \rangle$から${\rm pr}_{2}\langle f \rangle$への函数である.
         また$f$が$a$における函数ならば, $f$は$a$から${\rm pr}_{2}\langle f \rangle$への函数である.
\end{thm}


\noindent{\bf 証明}
~Thm \ref{x=x}より
${\rm pr}_{1}\langle f \rangle = {\rm pr}_{1}\langle f \rangle$が成り立つから, 
推論法則 \ref{dedeqch}, \ref{dedawblatrue2}により
\[
  {\rm Func}(f) \leftrightarrow {\rm Func}(f) \wedge {\rm pr}_{1}\langle f \rangle = {\rm pr}_{1}\langle f \rangle, 
\]
即ち
\[
\tag{1}
  {\rm Func}(f) \leftrightarrow {\rm Func}(f; {\rm pr}_{1}\langle f \rangle)
\]
が成り立つ.
また定理 \ref{sthmsubsetself}より
${\rm pr}_{2}\langle f \rangle \subset {\rm pr}_{2}\langle f \rangle$が成り立つから, 
同じく推論法則 \ref{dedeqch}, \ref{dedawblatrue2}により
\[
  {\rm Func}(f; a) \leftrightarrow {\rm Func}(f; a) \wedge {\rm pr}_{2}\langle f \rangle \subset {\rm pr}_{2}\langle f \rangle, 
\]
即ち
\[
\tag{2}
  {\rm Func}(f; a) \leftrightarrow {\rm Func}(f; a; {\rm pr}_{2}\langle f \rangle)
\]
が成り立つ.
ここで$a$は任意の集合で良いので, 特に
\[
\tag{3}
  {\rm Func}(f; {\rm pr}_{1}\langle f \rangle) \leftrightarrow {\rm Func}(f; {\rm pr}_{1}\langle f \rangle; {\rm pr}_{2}\langle f \rangle)
\]
も成り立つ.
故に(1), (3)から, 推論法則 \ref{dedeqtrans}によって
\[
\tag{4}
  {\rm Func}(f) \leftrightarrow {\rm Func}(f; {\rm pr}_{1}\langle f \rangle; {\rm pr}_{2}\langle f \rangle)
\]
が成り立つ.
($*$)が成り立つことは, (1), (2), (4)が成り立つことと推論法則 \ref{dedeqfund}によって明らかである.
\halmos




\mathstrut
\begin{thm}
\label{sthmpairinfunc}%定理
\mbox{}

1)
$f$, $t$, $u$, $v$を集合とするとき, 
\[
  {\rm Func}(f) \to ((t, u) \in f \wedge (t, v) \in f \to u = v)
\]
が成り立つ.
またこのことから, 次の($*$)が成り立つ: 

($*$) ~~$f$が函数ならば, $(t, u) \in f \wedge (t, v) \in f \to u = v$が成り立つ.
        故に, $f$が函数であるとき, $(t, u) \in f$と$(t, v) \in f$が共に成り立つならば, 
        $u = v$が成り立つ.

2)
$a$, $f$, $t$, $u$, $v$を集合とするとき, 
\[
  {\rm Func}(f; a) \to ((t, u) \in f \wedge (t, v) \in f \to u = v)
\]
が成り立つ.
またこのことから, 次の($**$)が成り立つ: 

($**$) ~~$f$が$a$における函数ならば, $(t, u) \in f \wedge (t, v) \in f \to u = v$が成り立つ.
         故に, $f$が$a$における函数であるとき, $(t, u) \in f$と$(t, v) \in f$が共に成り立つならば, 
         $u = v$が成り立つ.

3)
$a$, $b$, $f$, $t$, $u$, $v$を集合とするとき, 
\[
  {\rm Func}(f; a; b) \to ((t, u) \in f \wedge (t, v) \in f \to u = v)
\]
が成り立つ.
またこのことから, 次の(${**}*$)が成り立つ: 

(${**}*$) ~~$f$が$a$から$b$への函数ならば, $(t, u) \in f \wedge (t, v) \in f \to u = v$が成り立つ.
            故に, $f$が$a$から$b$への函数であるとき, $(t, u) \in f$と$(t, v) \in f$が共に成り立つならば, 
            $u = v$が成り立つ.
\end{thm}


\noindent{\bf 証明}
~1)
$x$を$f$の中に自由変数として現れない文字とする.
また$y$を, $x$と異なり, $f$及び$t$の中に自由変数として現れない文字とする.
このとき定理 \ref{sthmfuncbasis}より
\[
\tag{1}
  {\rm Func}(f) \to \forall x(!y((x, y) \in f))
\]
が成り立つ.
またThm \ref{thmallfund2}より
\[
  \forall x(!y((x, y) \in f)) \to (t|x)(!y((x, y) \in f))
\]
が成り立つ.
ここで$y$が$x$と異なり, $t$の中に自由変数として現れないことから, 
代入法則 \ref{subst!}により, 上記の記号列は
\[
  \forall x(!y((x, y) \in f)) \to \ !y((t|x)((x, y) \in f))
\]
と一致する.
また$x$が$y$と異なり, $f$の中に自由変数として現れないことから, 
代入法則 \ref{substfree}, \ref{substfund}, \ref{substpair}により, 
この記号列は
\[
\tag{2}
  \forall x(!y((x, y) \in f)) \to \ !y((t, y) \in f)
\]
と一致する.
よってこれが定理となる.
またThm \ref{thm!fund}より
\[
  !y((t, y) \in f) \to ((u|y)((t, y) \in f) \wedge (v|y)((t, y) \in f) \to u = v)
\]
が成り立つが, $y$が$f$及び$t$の中に自由変数として現れないことから, 
代入法則 \ref{substfree}, \ref{substfund}, \ref{substpair}により, この記号列は
\[
\tag{3}
  !y((t, y) \in f) \to ((t, u) \in f \wedge (t, v) \in f \to u = v)
\]
と一致し, 故にこれが定理となる.
そこで(1), (2), (3)から, 推論法則 \ref{dedmmp}によって
\[
\tag{4}
  {\rm Func}(f) \to ((t, u) \in f \wedge (t, v) \in f \to u = v)
\]
が成り立つことがわかる.
($*$)が成り立つことは, これと推論法則 \ref{dedmp}, \ref{dedwedge}によって明らかである.

\noindent
2)
定理 \ref{sthmfuncbasis}より${\rm Func}(f; a) \to {\rm Func}(f)$が成り立つから, 
これと(4)から, 推論法則 \ref{dedmmp}によって
\[
  {\rm Func}(f; a) \to ((t, u) \in f \wedge (t, v) \in f \to u = v)
\]
が成り立つ.
($**$)が成り立つことは, これと推論法則 \ref{dedmp}, \ref{dedwedge}によって明らかである.

\noindent
3)
定理 \ref{sthmfuncbasis}より${\rm Func}(f; a; b) \to {\rm Func}(f)$が成り立つから, 
これと(4)から, 推論法則 \ref{dedmmp}によって
\[
  {\rm Func}(f; a; b) \to ((t, u) \in f \wedge (t, v) \in f \to u = v)
\]
が成り立つ.
(${**}*$)が成り立つことは, これと推論法則 \ref{dedmp}, \ref{dedwedge}によって明らかである.
\halmos




\mathstrut
\begin{thm}
\label{sthmfunc=}%定理
\mbox{} 

1)
$a$, $b$, $f$, $g$を集合とするとき, 
\begin{align*}
  f = g &\to ({\rm Func}(f) \leftrightarrow {\rm Func}(g)), \\
  \mbox{} \\
  f = g &\to ({\rm Func}(f; a) \leftrightarrow {\rm Func}(g; a)), \\
  \mbox{} \\
  f = g &\to ({\rm Func}(f; a; b) \leftrightarrow {\rm Func}(g; a; b))
\end{align*}
が成り立つ.
またこれらから, 次の($*$)が成り立つ: 

($*$) ~~$f = g$が成り立つならば, 
        \[
          {\rm Func}(f) \leftrightarrow {\rm Func}(g), ~~
          {\rm Func}(f; a) \leftrightarrow {\rm Func}(g; a), ~~
          {\rm Func}(f; a; b) \leftrightarrow {\rm Func}(g; a; b)
        \]
        がすべて成り立つ.
        故に, $f = g$が成り立つとき, 
        $f$が函数ならば$g$は函数であり, 
        $f$が$a$における函数ならば$g$は$a$における函数であり, 
        $f$が$a$から$b$への函数ならば$g$は$a$から$b$への函数である.
        また$f = g$が成り立つとき, 
        $g$が函数ならば$f$は函数であり, 
        $g$が$a$における函数ならば$f$は$a$における函数であり, 
        $g$が$a$から$b$への函数ならば$f$は$a$から$b$への函数である.

2)
$a$, $b$, $c$, $f$を集合とするとき, 
\begin{align*}
  a = b &\to ({\rm Func}(f; a) \leftrightarrow {\rm Func}(f; b)), \\
  \mbox{} \\
  a = b &\to ({\rm Func}(f; a; c) \leftrightarrow {\rm Func}(f; b; c))
\end{align*}
が成り立つ.
またこれらから, 次の($**$)が成り立つ: 

($**$) ~~$a = b$が成り立つならば, 
         \[
           {\rm Func}(f; a) \leftrightarrow {\rm Func}(f; b), ~~
           {\rm Func}(f; a; c) \leftrightarrow {\rm Func}(f; b; c)
         \]
         が共に成り立つ.
         故に, $a = b$が成り立つとき, $f$が$a$における函数ならば$f$は$b$における函数であり, 
         $f$が$a$から$c$への函数ならば$f$は$b$から$c$への函数である.
         また$a = b$が成り立つとき, $f$が$b$における函数ならば$f$は$a$における函数であり, 
         $f$が$b$から$c$への函数ならば$f$は$a$から$c$への函数である.

3)
$a$, $b$, $c$, $f$を集合とするとき, 
\[
  b = c \to ({\rm Func}(f; a; b) \leftrightarrow {\rm Func}(f; a; c))
\]
が成り立つ.
またこのことから, 次の(${**}*$)が成り立つ: 

(${**}*$) ~~$b = c$が成り立つならば, 
            \[
              {\rm Func}(f; a; b) \leftrightarrow {\rm Func}(f; a; c)
            \]
            が成り立つ.
            故に, $b = c$が成り立つとき, $f$が$a$から$b$への函数ならば$f$は$a$から$c$への函数であり, 
            $f$が$a$から$c$への函数ならば$f$は$a$から$b$への函数である.
\end{thm}


\noindent{\bf 証明}
~1)
$x$を$a$及び$b$の中に自由変数として現れない文字とするとき, Thm \ref{thms5eq}より
\begin{align*}
  f = g &\to ((f|x)({\rm Func}(x)) \leftrightarrow (g|x)({\rm Func}(x))), \\
  \mbox{} \\
  f = g &\to ((f|x)({\rm Func}(x; a)) \leftrightarrow (g|x)({\rm Func}(x; a))), \\
  \mbox{} \\
  f = g &\to ((f|x)({\rm Func}(x; a; b)) \leftrightarrow (g|x)({\rm Func}(x; a; b)))
\end{align*}
がすべて成り立つが, 
代入法則 \ref{substfree}, \ref{substfunc}, \ref{substonafunc}, \ref{substatbfunc}によれば
これらの記号列はそれぞれ
\begin{align*}
  f = g &\to ({\rm Func}(f) \leftrightarrow {\rm Func}(g)), \\
  \mbox{} \\
  f = g &\to ({\rm Func}(f; a) \leftrightarrow {\rm Func}(g; a)), \\
  \mbox{} \\
  f = g &\to ({\rm Func}(f; a; b) \leftrightarrow {\rm Func}(g; a; b))
\end{align*}
と一致するから, これらがすべて定理となる.
($*$)が成り立つことは, これらと推論法則 \ref{dedmp}, \ref{dedeqfund}によって明らかである.

\noindent
2)
$y$を$c$及び$f$の中に自由変数として現れない文字とするとき, 
Thm \ref{thms5eq}より
\begin{align*}
  a = b &\to ((a|y)({\rm Func}(f; y)) \leftrightarrow (b|y)({\rm Func}(f; y))), \\
  \mbox{} \\
  a = b &\to ((a|y)({\rm Func}(f; y; c)) \leftrightarrow (b|y)({\rm Func}(f; y; c)))
\end{align*}
が共に成り立つが, 代入法則 \ref{substfree}, \ref{substonafunc}, \ref{substatbfunc}によれば
これらの記号列はそれぞれ
\begin{align*}
  a = b &\to ({\rm Func}(f; a) \leftrightarrow {\rm Func}(f; b)), \\
  \mbox{} \\
  a = b &\to ({\rm Func}(f; a; c) \leftrightarrow {\rm Func}(f; b; c))
\end{align*}
と一致するから, これらが共に定理となる.
($**$)が成り立つことは, これらと推論法則 \ref{dedmp}, \ref{dedeqfund}によって明らかである.

\noindent
3)
$z$を$a$及び$f$の中に自由変数として現れない文字とするとき, Thm \ref{thms5eq}より
\[
  b = c \to ((b|z)({\rm Func}(f; a; z)) \leftrightarrow (c|z)({\rm Func}(f; a; z)))
\]
が成り立つが, 代入法則 \ref{substfree}, \ref{substatbfunc}によればこの記号列は
\[
  b = c \to ({\rm Func}(f; a; b) \leftrightarrow {\rm Func}(f; a; c))
\]
と一致するから, これが定理となる.
(${**}*$)が成り立つことは, これと推論法則 \ref{dedmp}, \ref{dedeqfund}によって明らかである.
\halmos




\mathstrut
\begin{thm}
\label{sthmfuncdefset=}%定理
\mbox{} 

1)
$a$, $b$, $f$を集合とするとき, 
\[
  {\rm Func}(f; a) \wedge {\rm Func}(f; b) \to a = b
\]
が成り立つ.
またこのことから, 次の($*$)が成り立つ: 

($*$) ~~$f$が$a$における函数であり, かつ$b$における函数でもあるならば, 
        $a = b$が成り立つ.

2)
$a$, $b$, $c$, $d$, $f$を集合とするとき, 
\[
  {\rm Func}(f; a; b) \wedge {\rm Func}(f; c; d) \to a = c
\]
が成り立つ.
またこのことから, 次の($**$)が成り立つ: 

($**$) ~~$f$が$a$から$b$への函数であり, かつ$c$から$d$への函数でもあるならば, 
         $a = c$が成り立つ.
\end{thm}


\noindent{\bf 証明}
~1)
定理 \ref{sthmfuncbasis}より
\begin{align*}
  \tag{1}
  {\rm Func}(f; a) &\to {\rm pr}_{1}\langle f \rangle = a, \\
  \mbox{} \\
  \tag{2}
  {\rm Func}(f; b) &\to {\rm pr}_{1}\langle f \rangle = b
\end{align*}
が共に成り立つ.
またThm \ref{x=yty=x}より
\[
\tag{3}
  {\rm pr}_{1}\langle f \rangle = a \to a = {\rm pr}_{1}\langle f \rangle
\]
が成り立つ.
そこで(1), (3)から, 推論法則 \ref{dedmmp}によって
\[
  {\rm Func}(f; a) \to a = {\rm pr}_{1}\langle f \rangle
\]
が成り立ち, これと(2)から, 推論法則 \ref{dedfromaddw}により
\[
\tag{4}
  {\rm Func}(f; a) \wedge {\rm Func}(f; b) 
  \to a = {\rm pr}_{1}\langle f \rangle \wedge {\rm pr}_{1}\langle f \rangle = b
\]
が成り立つ.
またThm \ref{x=ywy=ztx=z}より
\[
\tag{5}
  a = {\rm pr}_{1}\langle f \rangle \wedge {\rm pr}_{1}\langle f \rangle = b \to a = b
\]
が成り立つ.
そこで(4), (5)から, 推論法則 \ref{dedmmp}によって
\[
  {\rm Func}(f; a) \wedge {\rm Func}(f; b) \to a = b
\]
が成り立つ.
($*$)が成り立つことは, これと推論法則 \ref{dedmp}, \ref{dedwedge}によって明らかである.

\noindent
2)
定理 \ref{sthmfuncbasis}より
\[
  {\rm Func}(f; a; b) \to {\rm Func}(f; a), ~~
  {\rm Func}(f; c; d) \to {\rm Func}(f; c)
\]
が共に成り立つから, 推論法則 \ref{dedfromaddw}により
\[
  {\rm Func}(f; a; b) \wedge {\rm Func}(f; c; d) \to {\rm Func}(f; a) \wedge {\rm Func}(f; c)
\]
が成り立つ.
また1)より
\[
  {\rm Func}(f; a) \wedge {\rm Func}(f; c) \to a = c
\]
が成り立つ.
故にこの二つの定理から, 推論法則 \ref{dedmmp}によって
\[
  {\rm Func}(f; a; b) \wedge {\rm Func}(f; c; d) \to a = c
\]
が成り立つ.
($**$)が成り立つことは, これと推論法則 \ref{dedmp}, \ref{dedwedge}によって明らかである.
\halmos




\mathstrut
\begin{thm}
\label{sthmatbfuncsubset}%定理
$a$, $b$, $c$, $f$を集合とするとき, 
\[
  b \subset c \to ({\rm Func}(f; a; b) \to {\rm Func}(f; a; c))
\]
が成り立つ.
またこのことから, 次の($*$)が成り立つ: 

($*$) ~~$b \subset c$が成り立つならば, ${\rm Func}(f; a; b) \to {\rm Func}(f; a; c)$が成り立つ.
        故に, $b \subset c$が成り立ち, かつ$f$が$a$から$b$への函数であれば, $f$は$a$から$c$への函数である.
\end{thm}


\noindent{\bf 証明}
~定理 \ref{sthmsubsettrans}より
\[
  {\rm pr}_{2}\langle f \rangle \subset b \wedge b \subset c \to {\rm pr}_{2}\langle f \rangle \subset c
\]
が成り立つから, 推論法則 \ref{dedtwch}により
\[
  {\rm pr}_{2}\langle f \rangle \subset b \to (b \subset c \to {\rm pr}_{2}\langle f \rangle \subset c)
\]
が成り立ち, これから推論法則 \ref{dedch}により
\[
\tag{1}
  b \subset c \to ({\rm pr}_{2}\langle f \rangle \subset b \to {\rm pr}_{2}\langle f \rangle \subset c)
\]
が成り立つ.
またThm \ref{1atb1t1awctbwc1}より
\[
  ({\rm pr}_{2}\langle f \rangle \subset b \to {\rm pr}_{2}\langle f \rangle \subset c) 
  \to ({\rm Func}(f; a) \wedge {\rm pr}_{2}\langle f \rangle \subset b \to {\rm Func}(f; a) \wedge {\rm pr}_{2}\langle f \rangle \subset c)
\]
が成り立つが, 定義よりこの記号列は
\[
\tag{2}
  ({\rm pr}_{2}\langle f \rangle \subset b \to {\rm pr}_{2}\langle f \rangle \subset c) 
  \to ({\rm Func}(f; a; b) \to {\rm Func}(f; a; c))
\]
と同じだから, これが定理となる.
故に(1), (2)から, 推論法則 \ref{dedmmp}によって
\[
  b \subset c \to ({\rm Func}(f; a; b) \to {\rm Func}(f; a; c))
\]
が成り立つ.
($*$)が成り立つことは, これと推論法則 \ref{dedmp}によって明らかである.
\halmos




\mathstrut
\begin{thm}
\label{sthmatbfuncbcapc}%定理
$a$, $b$, $c$, $f$を集合とするとき, 
\[
  {\rm Func}(f; a; b) \wedge {\rm Func}(f; a; c) \leftrightarrow {\rm Func}(f; a; b \cap c)
\]
が成り立つ.
またこのことから, 次の($*$)が成り立つ: 

($*$) ~~$f$が$a$から$b$への函数であり, かつ$a$から$c$への函数でもあるならば, 
        $f$は$a$から$b \cap c$への函数である.
        逆に$f$が$a$から$b \cap c$への函数ならば, $f$は$a$から$b$への函数であり, 
        かつ$a$から$c$への函数でもある.
\end{thm}


\noindent{\bf 証明}
~${\rm Func}(f; a; b) \equiv {\rm Func}(f; a) \wedge {\rm pr}_{2}\langle f \rangle \subset b$, 
${\rm Func}(f; a; c) \equiv {\rm Func}(f; a) \wedge {\rm pr}_{2}\langle f \rangle \subset c$なる定義から, 
Thm \ref{aw1bwc1l1awb1w1awc1}と推論法則 \ref{dedeqch}により
\[
\tag{1}
  {\rm Func}(f; a; b) \wedge {\rm Func}(f; a; c) 
  \leftrightarrow {\rm Func}(f; a) \wedge ({\rm pr}_{2}\langle f \rangle \subset b \wedge {\rm pr}_{2}\langle f \rangle \subset c)
\]
が成り立つ.
また定理 \ref{sthmcapdil}より
\[
  {\rm pr}_{2}\langle f \rangle \subset b \wedge {\rm pr}_{2}\langle f \rangle \subset c 
  \leftrightarrow {\rm pr}_{2}\langle f \rangle \subset b \cap c
\]
が成り立つから, 推論法則 \ref{dedaddeqw}により
\[
  {\rm Func}(f; a) \wedge ({\rm pr}_{2}\langle f \rangle \subset b \wedge {\rm pr}_{2}\langle f \rangle \subset c) 
  \leftrightarrow {\rm Func}(f; a) \wedge {\rm pr}_{2}\langle f \rangle \subset b \cap c
\]
が成り立つが, 定義よりこの記号列は
\[
\tag{2}
  {\rm Func}(f; a) \wedge ({\rm pr}_{2}\langle f \rangle \subset b \wedge {\rm pr}_{2}\langle f \rangle \subset c) 
  \leftrightarrow {\rm Func}(f; a; b \cap c)
\]
と同じだから, これが定理となる.
故に(1), (2)から, 推論法則 \ref{dedeqtrans}によって
\[
  {\rm Func}(f; a; b) \wedge {\rm Func}(f; a; c) \leftrightarrow {\rm Func}(f; a; b \cap c)
\]
が成り立つ.
($*$)が成り立つことは, これと推論法則 \ref{dedwedge}, \ref{dedeqfund}によって明らかである.
\halmos




\mathstrut
\begin{thm}
\label{sthmfuncsubset}%定理
$f$と$g$を集合とするとき, 
\[
  f \subset g \to ({\rm Func}(g) \to {\rm Func}(f))
\]
が成り立つ.
またこのことから, 次の($*$)が成り立つ: 

($*$) ~~$f \subset g$が成り立つならば, ${\rm Func}(g) \to {\rm Func}(f)$が成り立つ.
        故に, $f \subset g$が成り立ち, かつ$g$が函数ならば, $f$は函数である.
\end{thm}


\noindent{\bf 証明}
~$x$と$y$を, 互いに異なり, 共に$f$及び$g$の中に自由変数として現れない, 
定数でない文字とする.
このとき変数法則 \ref{valsubset}により, 
$x$と$y$は共に$f \subset g$の中に自由変数として現れない.
また定理 \ref{sthmsubsetbasis}より
\[
  f \subset g \to ((x, y) \in f \to (x, y) \in g)
\]
が成り立つ.
故に推論法則 \ref{dedalltquansepfreeconst}により
\[
\tag{1}
  f \subset g \to \forall x(\forall y((x, y) \in f \to (x, y) \in g))
\]
が成り立つ.
またThm \ref{thmallt!sep}より
\[
  \forall y((x, y) \in f \to (x, y) \in g) \to (!y((x, y) \in g) \to \ !y((x, y) \in f))
\]
が成り立つから, $x$が定数でないことから, 推論法則 \ref{dedalltquansepconst}により
\[
\tag{2}
  \forall x(\forall y((x, y) \in f \to (x, y) \in g)) \to \forall x(!y((x, y) \in g) \to \ !y((x, y) \in f))
\]
が成り立つ.
またThm \ref{thmalltallsep}より
\[
\tag{3}
  \forall x(!y((x, y) \in g) \to \ !y((x, y) \in f)) \to (\forall x(!y((x, y) \in g)) \to \forall x(!y((x, y) \in f)))
\]
が成り立つ.
そこで(1), (2), (3)から, 推論法則 \ref{dedmmp}によって
\[
  f \subset g \to (\forall x(!y((x, y) \in g)) \to \forall x(!y((x, y) \in f)))
\]
が成り立つことがわかる.
また定理 \ref{sthmgraphsubset}より
\[
  f \subset g \to ({\rm Graph}(g) \to {\rm Graph}(f))
\]
が成り立つ.
故にこの二つの定理から, 推論法則 \ref{dedprewedge}により
\[
\tag{4}
  f \subset g 
  \to ({\rm Graph}(g) \to {\rm Graph}(f)) \wedge (\forall x(!y((x, y) \in g)) \to \forall x(!y((x, y) \in f)))
\]
が成り立つ.
またThm \ref{1atb1w1ctd1t1awctbwd1}より
\begin{multline*}
  ({\rm Graph}(g) \to {\rm Graph}(f)) \wedge (\forall x(!y((x, y) \in g)) \to \forall x(!y((x, y) \in f))) \\
  \to ({\rm Graph}(g) \wedge \forall x(!y((x, y) \in g)) \to {\rm Graph}(f) \wedge \forall x(!y((x, y) \in f)))
\end{multline*}
が成り立つが, $x$と$y$が互いに異なり, 共に$f$及び$g$の中に自由変数として現れないことから, 
定義よりこの記号列は
\[
\tag{5}
  ({\rm Graph}(g) \to {\rm Graph}(f)) \wedge (\forall x(!y((x, y) \in g)) \to \forall x(!y((x, y) \in f))) 
  \to ({\rm Func}(g) \to {\rm Func}(f))
\]
と一致する.
よってこれが定理となる.
そこで(4), (5)から, 推論法則 \ref{dedmmp}によって
\[
  f \subset g \to ({\rm Func}(g) \to {\rm Func}(f))
\]
が成り立つ.
($*$)が成り立つことは, これと推論法則 \ref{dedmp}によって明らかである.
\halmos




\mathstrut
\begin{thm}
\label{sthmfuncsubset=eq}%定理
\mbox{}

1)
$a$, $f$, $g$を集合とするとき, 
\begin{align*}
  a \subset {\rm pr}_{1}\langle f \rangle \wedge {\rm Func}(g; a) &\to (f \subset g \leftrightarrow f = g), \\
  \mbox{} \\
  {\rm pr}_{1}\langle f \rangle = a \wedge {\rm Func}(g; a) &\to (f \subset g \leftrightarrow f = g), \\
  \mbox{} \\
  {\rm Func}(f; a) \wedge {\rm Func}(g; a) &\to (f \subset g \leftrightarrow f = g)
\end{align*}
が成り立つ.
またこれらから, 次の($*$)が成り立つ: 

($*$) ~~$g$が$a$における函数であるとする.
        また(i) $a \subset {\rm pr}_{1}\langle f \rangle$, 
        (ii) ${\rm pr}_{1}\langle f \rangle = a$, 
        (iii) $f$は$a$における函数である, のいずれかが成り立つとする.
        このとき$f \subset g \leftrightarrow f = g$が成り立つ.
        そこで特に, $g$が$a$における函数であり, 
        かつ上記の(i), (ii), (iii)のいずれかが成り立つとき, 
        $f \subset g$が成り立つならば, $f = g$が成り立つ.

2)
$a$, $b$, $c$, $f$, $g$を集合とするとき, 
\[
  {\rm Func}(f; a; b) \wedge {\rm Func}(g; a; c) \to (f \subset g \leftrightarrow f = g)
\]
が成り立つ.
またこのことから, 次の($**$)が成り立つ: 

($**$) ~~$f$が$a$から$b$への函数であり, $g$が$a$から$c$への函数であるとする.
         このとき$f \subset g \leftrightarrow f = g$が成り立つ.
         そこで特にこのとき, $f \subset g$が成り立つならば, $f = g$が成り立つ.
\end{thm}


\noindent{\bf 証明}
~1)
はじめに
\[
  a \subset {\rm pr}_{1}\langle f \rangle \wedge {\rm Func}(g; a) \to (f \subset g \leftrightarrow f = g)
\]
が成り立つことを証明する.
定理 \ref{sthmfuncbasis}より
\[
\tag{1}
  {\rm Func}(g; a) \to {\rm pr}_{1}\langle g \rangle = a
\]
が成り立つ.
また$x$と$y$を, 互いに異なり, 共に$a$, $f$, $g$のいずれの記号列の中にも
自由変数として現れない, 定数でない文字とするとき, 
定理 \ref{sthmpairelementinprset}より
\[
  (x, y) \in g \to x \in {\rm pr}_{1}\langle g \rangle \wedge y \in {\rm pr}_{2}\langle g \rangle
\]
が成り立つから, 推論法則 \ref{dedprewedge}により
\[
\tag{2}
  (x, y) \in g \to x \in {\rm pr}_{1}\langle g \rangle
\]
が成り立つ.
そこで(1), (2)から, 推論法則 \ref{dedfromaddw}により
\[
\tag{3}
  {\rm Func}(g; a) \wedge (x, y) \in g \to {\rm pr}_{1}\langle g \rangle = a \wedge x \in {\rm pr}_{1}\langle g \rangle
\]
が成り立つ.
また定理 \ref{sthm=&in}より
\[
  {\rm pr}_{1}\langle g \rangle = a \wedge x \in {\rm pr}_{1}\langle g \rangle \to x \in a
\]
が成り立つ.
故にこれと(3)から, 推論法則 \ref{dedmmp}によって
\[
  {\rm Func}(g; a) \wedge (x, y) \in g \to x \in a
\]
が成り立ち, これから推論法則 \ref{dedaddw}によって
\[
\tag{4}
  a \subset {\rm pr}_{1}\langle f \rangle \wedge ({\rm Func}(g; a) \wedge (x, y) \in g) 
  \to a \subset {\rm pr}_{1}\langle f \rangle \wedge x \in a
\]
が成り立つ.
また定理 \ref{sthmsubsetbasis}より
\[
  a \subset {\rm pr}_{1}\langle f \rangle \to (x \in a \to x \in {\rm pr}_{1}\langle f \rangle)
\]
が成り立つから, 推論法則 \ref{dedtwch}により
\[
\tag{5}
  a \subset {\rm pr}_{1}\langle f \rangle \wedge x \in a \to x \in {\rm pr}_{1}\langle f \rangle
\]
が成り立つ.
また$y$が$x$と異なり, $f$の中に自由変数として現れないことから, 
定理 \ref{sthmprsetelement}と推論法則 \ref{dedequiv}により
\[
  x \in {\rm pr}_{1}\langle f \rangle \to \exists y((x, y) \in f)
\]
が成り立つ.
ここで$\tau_{y}((x, y) \in f)$を$T$と書けば, $T$は集合であり, 
定義から上記の記号列は
\[
  x \in {\rm pr}_{1}\langle f \rangle \to (T|y)((x, y) \in f)
\]
と一致する.
また$y$が$x$と異なり, $f$の中に自由変数として現れないことから, 
代入法則 \ref{substfree}, \ref{substfund}, \ref{substpair}により, この記号列は
\[
\tag{6}
  x \in {\rm pr}_{1}\langle f \rangle \to (x, T) \in f
\]
と一致する.
よってこれが定理となる.
そこで(4), (5), (6)から, 推論法則 \ref{dedmmp}によって
\[
  a \subset {\rm pr}_{1}\langle f \rangle \wedge ({\rm Func}(g; a) \wedge (x, y) \in g) \to (x, T) \in f
\]
が成り立つことがわかる.
故に推論法則 \ref{dedaddw}により
\[
\tag{7}
  f \subset g \wedge (a \subset {\rm pr}_{1}\langle f \rangle \wedge ({\rm Func}(g; a) \wedge (x, y) \in g)) 
  \to f \subset g \wedge (x, T) \in f
\]
が成り立つ.
また定理 \ref{sthmsubsetbasis}より
\[
  f \subset g \to ((x, T) \in f \to (x, T) \in g)
\]
が成り立つから, 推論法則 \ref{dedtwch}により
\[
\tag{8}
  f \subset g \wedge (x, T) \in f \to (x, T) \in g
\]
が成り立つ.
そこで(7), (8)から, 推論法則 \ref{dedmmp}によって
\[
\tag{9}
  f \subset g \wedge (a \subset {\rm pr}_{1}\langle f \rangle \wedge ({\rm Func}(g; a) \wedge (x, y) \in g)) 
  \to (x, T) \in g
\]
が成り立つ.
またThm \ref{awbta}より
\[
  f \subset g \wedge (a \subset {\rm pr}_{1}\langle f \rangle \wedge ({\rm Func}(g; a) \wedge (x, y) \in g)) 
  \to a \subset {\rm pr}_{1}\langle f \rangle \wedge ({\rm Func}(g; a) \wedge (x, y) \in g)
\]
が成り立つから, 
これに推論法則 \ref{dedprewedge}を二回ずつ用いて
\begin{align*}
  \tag{10}
  f \subset g \wedge (a \subset {\rm pr}_{1}\langle f \rangle \wedge ({\rm Func}(g; a) \wedge (x, y) \in g)) 
  &\to (x, y) \in g, \\
  \mbox{} \\
  \tag{11}
  f \subset g \wedge (a \subset {\rm pr}_{1}\langle f \rangle \wedge ({\rm Func}(g; a) \wedge (x, y) \in g)) 
  &\to {\rm Func}(g; a)
\end{align*}
が共に成り立つことがわかる.
故に(9), (10), (11)から, 推論法則 \ref{dedprewedge}によって
\[
\tag{12}
  f \subset g \wedge (a \subset {\rm pr}_{1}\langle f \rangle \wedge ({\rm Func}(g; a) \wedge (x, y) \in g)) 
  \to {\rm Func}(g; a) \wedge ((x, y) \in g \wedge (x, T) \in g)
\]
が成り立つことがわかる.
また定理 \ref{sthmpairinfunc}より
\[
  {\rm Func}(g; a) \to ((x, y) \in g \wedge (x, T) \in g \to y = T)
\]
が成り立つから, 推論法則 \ref{dedtwch}により
\[
\tag{13}
  {\rm Func}(g; a) \wedge ((x, y) \in g \wedge (x, T) \in g) \to y = T
\]
が成り立つ.
また定理 \ref{sthmpairweak}と推論法則 \ref{dedequiv}により
\[
\tag{14}
  y = T \to (x, y) = (x, T)
\]
が成り立つ.
そこで(12), (13), (14)から, 推論法則 \ref{dedmmp}によって
\[
\tag{15}
  f \subset g \wedge (a \subset {\rm pr}_{1}\langle f \rangle \wedge ({\rm Func}(g; a) \wedge (x, y) \in g)) 
  \to (x, y) = (x, T)
\]
が成り立つことがわかる.
また(7)から, 推論法則 \ref{dedprewedge}により
\[
\tag{16}
  f \subset g \wedge (a \subset {\rm pr}_{1}\langle f \rangle \wedge ({\rm Func}(g; a) \wedge (x, y) \in g)) 
  \to (x, T) \in f
\]
が成り立つ.
そこで(15), (16)から, 推論法則 \ref{dedprewedge}により
\[
\tag{17}
  f \subset g \wedge (a \subset {\rm pr}_{1}\langle f \rangle \wedge ({\rm Func}(g; a) \wedge (x, y) \in g)) 
  \to (x, y) = (x, T) \wedge (x, T) \in f
\]
が成り立つ.
また定理 \ref{sthm=&in}より
\[
\tag{18}
  (x, y) = (x, T) \wedge (x, T) \in f \to (x, y) \in f
\]
が成り立つ.
またThm \ref{1awb1wctaw1bwc1}より
\begin{multline*}
\tag{19}
  (f \subset g \wedge (a \subset {\rm pr}_{1}\langle f \rangle \wedge {\rm Func}(g; a))) \wedge (x, y) \in g \\
  \to f \subset g \wedge ((a \subset {\rm pr}_{1}\langle f \rangle \wedge {\rm Func}(g; a)) \wedge (x, y) \in g)
\end{multline*}
が成り立つ.
同じくThm \ref{1awb1wctaw1bwc1}より
\[
  (a \subset {\rm pr}_{1}\langle f \rangle \wedge {\rm Func}(g; a)) \wedge (x, y) \in g 
  \to a \subset {\rm pr}_{1}\langle f \rangle \wedge ({\rm Func}(g; a) \wedge (x, y) \in g)
\]
が成り立つから, 推論法則 \ref{dedaddw}により
\begin{multline*}
\tag{20}
  f \subset g \wedge ((a \subset {\rm pr}_{1}\langle f \rangle \wedge {\rm Func}(g; a)) \wedge (x, y) \in g) \\
  \to f \subset g \wedge (a \subset {\rm pr}_{1}\langle f \rangle \wedge ({\rm Func}(g; a) \wedge (x, y) \in g))
\end{multline*}
が成り立つ.
そこで(19), (20), (17), (18)にこの順で推論法則 \ref{dedmmp}を適用していき, 
\[
  (f \subset g \wedge (a \subset {\rm pr}_{1}\langle f \rangle \wedge {\rm Func}(g; a))) \wedge (x, y) \in g 
  \to (x, y) \in f
\]
が成り立つことがわかる.
故に推論法則 \ref{dedtwch}により
\[
\tag{21}
  f \subset g \wedge (a \subset {\rm pr}_{1}\langle f \rangle \wedge {\rm Func}(g; a)) 
  \to ((x, y) \in g \to (x, y) \in f)
\]
が成り立つ.
ここで$x$と$y$が共に$a$, $f$, $g$のいずれの記号列の中にも自由変数として現れないことから, 
変数法則 \ref{valwedge}, \ref{valsubset}, \ref{valprset}, \ref{valonafunc}によって
わかるように, $x$と$y$は共に
$f \subset g \wedge (a \subset {\rm pr}_{1}\langle f \rangle \wedge {\rm Func}(g; a))$の
中に自由変数として現れない.
また$x$と$y$は共に定数でない.
以上のことと, (21)が成り立つことから, 推論法則 \ref{dedalltquansepfreeconst}によって
\[
\tag{22}
  f \subset g \wedge (a \subset {\rm pr}_{1}\langle f \rangle \wedge {\rm Func}(g; a)) 
  \to \forall x(\forall y((x, y) \in g \to (x, y) \in f))
\]
が成り立つことがわかる.
またThm \ref{awbta}より
\[
  f \subset g \wedge (a \subset {\rm pr}_{1}\langle f \rangle \wedge {\rm Func}(g; a)) 
  \to a \subset {\rm pr}_{1}\langle f \rangle \wedge {\rm Func}(g; a)
\]
が成り立つから, 推論法則 \ref{dedprewedge}により
\[
\tag{23}
  f \subset g \wedge (a \subset {\rm pr}_{1}\langle f \rangle \wedge {\rm Func}(g; a)) 
  \to {\rm Func}(g; a)
\]
が成り立つ.
また定理 \ref{sthmfuncbasis}より
\[
\tag{24}
  {\rm Func}(g; a) \to {\rm Graph}(g)
\]
が成り立つ.
そこで(23), (24)から, 推論法則 \ref{dedmmp}によって
\[
  f \subset g \wedge (a \subset {\rm pr}_{1}\langle f \rangle \wedge {\rm Func}(g; a)) 
  \to {\rm Graph}(g)
\]
が成り立つ.
故にこれと(22)から, 推論法則 \ref{dedprewedge}によって
\[
\tag{25}
  f \subset g \wedge (a \subset {\rm pr}_{1}\langle f \rangle \wedge {\rm Func}(g; a)) 
  \to {\rm Graph}(g) \wedge \forall x(\forall y((x, y) \in g \to (x, y) \in f))
\]
が成り立つ.
また$x$と$y$が互いに異なり, 共に$f$及び$g$の中に自由変数として現れないことから, 
定理 \ref{sthmgraphpairsubset}より
\[
  {\rm Graph}(g) \to (g \subset f \leftrightarrow \forall x(\forall y((x, y) \in g \to (x, y) \in f)))
\]
が成り立つ.
故に推論法則 \ref{dedprewedge}により
\[
  {\rm Graph}(g) \to (\forall x(\forall y((x, y) \in g \to (x, y) \in f)) \to g \subset f)
\]
が成り立ち, これから推論法則 \ref{dedtwch}により
\[
\tag{26}
  {\rm Graph}(g) \wedge \forall x(\forall y((x, y) \in g \to (x, y) \in f)) \to g \subset f
\]
が成り立つ.
そこで(25), (26)から, 推論法則 \ref{dedmmp}によって
\[
\tag{27}
  f \subset g \wedge (a \subset {\rm pr}_{1}\langle f \rangle \wedge {\rm Func}(g; a)) 
  \to g \subset f
\]
が成り立つ.
またThm \ref{awbta}より
\[
  f \subset g \wedge (a \subset {\rm pr}_{1}\langle f \rangle \wedge {\rm Func}(g; a)) 
  \to f \subset g
\]
が成り立つから, これと(27)から, 推論法則 \ref{dedprewedge}により
\[
\tag{28}
  f \subset g \wedge (a \subset {\rm pr}_{1}\langle f \rangle \wedge {\rm Func}(g; a)) 
  \to f \subset g \wedge g \subset f
\]
が成り立つ.
またThm \ref{awbtbwa}より
\[
\tag{29}
  (a \subset {\rm pr}_{1}\langle f \rangle \wedge {\rm Func}(g; a)) \wedge f \subset g 
  \to f \subset g \wedge (a \subset {\rm pr}_{1}\langle f \rangle \wedge {\rm Func}(g; a))
\]
が成り立つ.
また定理 \ref{sthmaxiom1}と推論法則 \ref{dedequiv}により
\[
\tag{30}
  f \subset g \wedge g \subset f \to f = g
\]
が成り立つ.
そこで(29), (28), (30)にこの順で推論法則 \ref{dedmmp}を適用していき, 
\[
  (a \subset {\rm pr}_{1}\langle f \rangle \wedge {\rm Func}(g; a)) \wedge f \subset g 
  \to f = g
\]
が成り立つことがわかる.
故に推論法則 \ref{dedtwch}により
\[
\tag{31}
  a \subset {\rm pr}_{1}\langle f \rangle \wedge {\rm Func}(g; a)
  \to (f \subset g \to f = g)
\]
が成り立つ.
また定理 \ref{sthm=tsubset}より$f = g \to f \subset g$が成り立つから, 
推論法則 \ref{dedatawbtrue2}により
\[
\tag{32}
  (f \subset g \to f = g) \to (f \subset g \leftrightarrow f = g)
\]
が成り立つ.
そこで(31), (32)から, 推論法則 \ref{dedmmp}によって
\[
\tag{33}
  a \subset {\rm pr}_{1}\langle f \rangle \wedge {\rm Func}(g; a) 
  \to (f \subset g \leftrightarrow f = g)
\]
が成り立つ.

次に
\begin{align*}
  {\rm pr}_{1}\langle f \rangle = a \wedge {\rm Func}(g; a) &\to (f \subset g \leftrightarrow f = g), \\
  \mbox{} \\
  {\rm Func}(f; a) \wedge {\rm Func}(g; a) &\to (f \subset g \leftrightarrow f = g)
\end{align*}
が共に成り立つことを証明する.
定理 \ref{sthmfuncbasis}より
\[
\tag{34}
  {\rm Func}(f; a) \to {\rm pr}_{1}\langle f \rangle = a
\]
が成り立つ.
また定理 \ref{sthm=tsubset}より
\[
\tag{35}
  {\rm pr}_{1}\langle f \rangle = a \to a \subset {\rm pr}_{1}\langle f \rangle
\]
が成り立つ.
そこで(34), (35)から, 推論法則 \ref{dedmmp}によって
\[
\tag{36}
  {\rm Func}(f; a) \to a \subset {\rm pr}_{1}\langle f \rangle
\]
が成り立つ.
故に(35), (36)から, それぞれ推論法則 \ref{dedaddw}によって
\begin{align*}
  {\rm pr}_{1}\langle f \rangle = a \wedge {\rm Func}(g; a) 
  &\to a \subset {\rm pr}_{1}\langle f \rangle \wedge {\rm Func}(g; a), \\
  \mbox{} \\
  {\rm Func}(f; a) \wedge {\rm Func}(g; a) 
  &\to a \subset {\rm pr}_{1}\langle f \rangle \wedge {\rm Func}(g; a)
\end{align*}
が成り立つ.
そしてこれらと(33)から, それぞれ推論法則 \ref{dedmmp}によって
\begin{align*}
  \tag{37}
  {\rm pr}_{1}\langle f \rangle = a \wedge {\rm Func}(g; a) &\to (f \subset g \leftrightarrow f = g), \\
  \mbox{} \\
  \tag{38}
  {\rm Func}(f; a) \wedge {\rm Func}(g; a) &\to (f \subset g \leftrightarrow f = g)
\end{align*}
が成り立つ.

最後に($*$)であるが, これが成り立つことは, (33), (37), (38)が成り立つことと, 
推論法則 \ref{dedmp}, \ref{dedwedge}, \ref{dedeqfund}によって明らかである.

\noindent
2)
定理 \ref{sthmfuncbasis}より
\[
  {\rm Func}(f; a; b) \to {\rm Func}(f; a), ~~
  {\rm Func}(g; a; c) \to {\rm Func}(g; a)
\]
が共に成り立つから, 推論法則 \ref{dedfromaddw}により
\[
  {\rm Func}(f; a; b) \wedge {\rm Func}(g; a; c) \to {\rm Func}(f; a) \wedge {\rm Func}(g; a)
\]
が成り立つ.
故にこれと(38)から, 推論法則 \ref{dedmmp}によって
\[
  {\rm Func}(f; a; b) \wedge {\rm Func}(g; a; c) \to (f \subset g \leftrightarrow f = g)
\]
が成り立つ.
($**$)が成り立つことは, これと推論法則 \ref{dedmp}, \ref{dedwedge}, \ref{dedeqfund}によって
明らかである.
\halmos




\mathstrut
\begin{thm}
\label{sthmatbfuncsubsetatimesb}%定理
$a$, $b$, $f$を集合とするとき, 
\[
  {\rm Func}(f; a; b) \to f \subset a \times b
\]
が成り立つ.
またこのことから, 次の($*$)が成り立つ: 

($*$) ~~$f$が$a$から$b$への函数ならば, $f \subset a \times b$が成り立つ.
\end{thm}


\noindent{\bf 証明}
~定理 \ref{sthmfuncbasis}より
\[
  {\rm Func}(f; a; b) \to {\rm Graph}(f)
\]
が成り立ち, 定理 \ref{sthmgraphprset}と推論法則 \ref{dedequiv}により
\[
  {\rm Graph}(f) \to f \subset {\rm pr}_{1}\langle f \rangle \times {\rm pr}_{2}\langle f \rangle
\]
が成り立つから, 推論法則 \ref{dedmmp}により
\[
\tag{1}
  {\rm Func}(f; a; b) \to f \subset {\rm pr}_{1}\langle f \rangle \times {\rm pr}_{2}\langle f \rangle
\]
が成り立つ.
また定理 \ref{sthmfuncbasis}より
\[
  {\rm Func}(f; a; b) \to {\rm pr}_{1}\langle f \rangle = a
\]
が成り立ち, 定理 \ref{sthm=tsubset}より
\[
  {\rm pr}_{1}\langle f \rangle = a \to {\rm pr}_{1}\langle f \rangle \subset a
\]
が成り立つから, 推論法則 \ref{dedmmp}により
\[
\tag{2}
  {\rm Func}(f; a; b) \to {\rm pr}_{1}\langle f \rangle \subset a
\]
が成り立つ.
また定理 \ref{sthmfuncbasis}より
\[
\tag{3}
  {\rm Func}(f; a; b) \to {\rm pr}_{2}\langle f \rangle \subset b
\]
が成り立つ.
そこで(2), (3)から, 推論法則 \ref{dedprewedge}により
\[
\tag{4}
  {\rm Func}(f; a; b) \to {\rm pr}_{1}\langle f \rangle \subset a \wedge {\rm pr}_{2}\langle f \rangle \subset b
\]
が成り立つ.
また定理 \ref{sthmproductsubset}より
\[
\tag{5}
  {\rm pr}_{1}\langle f \rangle \subset a \wedge {\rm pr}_{2}\langle f \rangle \subset b 
  \to {\rm pr}_{1}\langle f \rangle \times {\rm pr}_{2}\langle f \rangle \subset a \times b
\]
が成り立つ.
そこで(4), (5)から, 推論法則 \ref{dedmmp}によって
\[
  {\rm Func}(f; a; b) \to {\rm pr}_{1}\langle f \rangle \times {\rm pr}_{2}\langle f \rangle \subset a \times b
\]
が成り立つ.
故にこれと(1)から, 推論法則 \ref{dedprewedge}により
\[
\tag{6}
  {\rm Func}(f; a; b) 
  \to f \subset {\rm pr}_{1}\langle f \rangle \times {\rm pr}_{2}\langle f \rangle 
  \wedge {\rm pr}_{1}\langle f \rangle \times {\rm pr}_{2}\langle f \rangle \subset a \times b
\]
が成り立つ.
また定理 \ref{sthmsubsettrans}より
\[
\tag{7}
  f \subset {\rm pr}_{1}\langle f \rangle \times {\rm pr}_{2}\langle f \rangle 
  \wedge {\rm pr}_{1}\langle f \rangle \times {\rm pr}_{2}\langle f \rangle \subset a \times b 
  \to f \subset a \times b
\]
が成り立つ.
そこで(6), (7)から, 推論法則 \ref{dedmmp}によって
\[
  {\rm Func}(f; a; b) \to f \subset a \times b
\]
が成り立つ.
($*$)が成り立つことは, これと推論法則 \ref{dedmp}によって明らかである.
\halmos




\mathstrut
\begin{thm}
\label{sthmatbfunceq}%定理
$a$, $b$, $f$を集合とする.
また$x$と$y$を, 互いに異なり, 共に$a$及び$f$の中に自由変数として現れない文字とする.
このとき, 
\[
  {\rm Func}(f; a; b) \leftrightarrow 
  f \subset a \times b \wedge \forall x(x \in a \to \exists !y((x, y) \in f))
\]
が成り立つ.
\end{thm}


\noindent{\bf 証明}
~推論法則 \ref{dedequiv}があるから, 
\begin{align*}
  \tag{1}
  &{\rm Func}(f; a; b) \to f \subset a \times b \wedge \forall x(x \in a \to \exists !y((x, y) \in f)), \\
  \mbox{} \\
  \tag{2}
  &f \subset a \times b \wedge \forall x(x \in a \to \exists !y((x, y) \in f)) \to {\rm Func}(f; a; b)
\end{align*}
が共に成り立つことを示せば良い.

(1)の証明: 
まず定理 \ref{sthmatbfuncsubsetatimesb}より
\[
\tag{3}
  {\rm Func}(f; a; b) \to f \subset a \times b
\]
が成り立つ.
またいま$\tau_{x}(\neg (x \in a \to \exists !y((x, y) \in f)))$を$T$と書けば, 
$T$は集合であり, 仮定より$y$が$x$と異なり, $a$の中に自由変数として現れないことから, 
変数法則 \ref{valfund}, \ref{valtau}, \ref{valex!}によってわかるように, $y$は$T$の中に
自由変数として現れない.
そして定理 \ref{sthmfuncbasis}より${\rm Func}(f; a; b) \to {\rm pr}_{1}\langle f \rangle = a$が成り立つから, 
推論法則 \ref{dedaddw}により
\[
\tag{4}
  {\rm Func}(f; a; b) \wedge T \in a \to {\rm pr}_{1}\langle f \rangle = a \wedge T \in a
\]
が成り立つ.
また定理 \ref{sthm=&in}より
\[
\tag{5}
  {\rm pr}_{1}\langle f \rangle = a \wedge T \in a \to T \in {\rm pr}_{1}\langle f \rangle
\]
が成り立つ.
また仮定より$y$は$f$の中に自由変数として現れず, 上述のように$T$の中にも自由変数として現れないから, 
定理 \ref{sthmprsetelement}と推論法則 \ref{dedequiv}により
\[
\tag{6}
  T \in {\rm pr}_{1}\langle f \rangle \to \exists y((T, y) \in f)
\]
が成り立つ.
そこで(4), (5), (6)から, 推論法則 \ref{dedmmp}によって
\[
\tag{7}
  {\rm Func}(f; a; b) \wedge T \in a \to \exists y((T, y) \in f)
\]
が成り立つことがわかる.
またThm \ref{awbta}より
\[
\tag{8}
  {\rm Func}(f; a; b) \wedge T \in a \to {\rm Func}(f; a; b)
\]
が成り立つ.
また$x$と$y$が互いに異なり, 共に$f$の中に自由変数として現れないという仮定から, 
定理 \ref{sthmfuncbasis}より
\[
\tag{9}
  {\rm Func}(f; a; b) \to \forall x(!y((x, y) \in f))
\]
が成り立つ.
またThm \ref{thmallfund2}より
\[
  \forall x(!y((x, y) \in f)) \to (T|x)(!y((x, y) \in f))
\]
が成り立つ.
ここで上述のように$y$は$x$と異なり, $T$の中に自由変数として現れないから, 
代入法則 \ref{subst!}によれば, 上記の記号列は
\[
  \forall x(!y((x, y) \in f)) \to \ !y((T|x)((x, y) \in f))
\]
と一致する.
また$x$は$y$と異なり, $f$の中に自由変数として現れないから, 
代入法則 \ref{substfree}, \ref{substfund}, \ref{substpair}によれば, 
この記号列は
\[
\tag{10}
  \forall x(!y((x, y) \in f)) \to \ !y((T, y) \in f)
\]
と一致する.
よってこれが定理となる.
そこで(8), (9), (10)から, 推論法則 \ref{dedmmp}によって
\[
  {\rm Func}(f; a; b) \wedge T \in a \to \ !y((T, y) \in f)
\]
が成り立つことがわかる.
故にこれと(7)から, 推論法則 \ref{dedprewedge}により
\[
  {\rm Func}(f; a; b) \wedge T \in a \to \exists !y((T, y) \in f)
\]
が成り立ち, これから推論法則 \ref{dedtwch}により
\[
\tag{11}
  {\rm Func}(f; a; b) \to (T \in a \to \exists !y((T, y) \in f))
\]
が成り立つ.
また$T$の定義から, Thm \ref{thmallfund1}と推論法則 \ref{dedequiv}により
\[
  (T|x)(x \in a \to \exists !y((x, y) \in f)) \to \forall x(x \in a \to \exists !y((x, y) \in f))
\]
が成り立つ.
ここで$x$が$a$の中に自由変数として現れないことから, 
代入法則 \ref{substfree}, \ref{substfund}により, 上記の記号列は
\[
  (T \in a \to (T|x)(\exists !y((x, y) \in f))) \to \forall x(x \in a \to \exists !y((x, y) \in f))
\]
と一致する.
また$y$が$x$と異なり, 上述のように$T$の中に自由変数として現れないことから, 
代入法則 \ref{substex!}により, この記号列は
\[
  (T \in a \to \exists !y((T|x)((x, y) \in f))) \to \forall x(x \in a \to \exists !y((x, y) \in f))
\]
と一致する.
更に$x$が$y$と異なり, $f$の中に自由変数として現れないことから, 
代入法則 \ref{substfree}, \ref{substfund}, \ref{substpair}により, この記号列は
\[
\tag{12}
  (T \in a \to \exists !y((T, y) \in f)) \to \forall x(x \in a \to \exists !y((x, y) \in f))
\]
と一致する.
故にこれが定理となる.
そこで(11), (12)から, 推論法則 \ref{dedmmp}によって
\[
  {\rm Func}(f; a; b) \to \forall x(x \in a \to \exists !y((x, y) \in f))
\]
が成り立つ.
故にこれと(3)から, 推論法則 \ref{dedprewedge}によって(1)が成り立つ.

(2)の証明: 
Thm \ref{awbta}より
\[
\tag{13}
  f \subset a \times b \wedge \forall x(x \in a \to \exists !y((x, y) \in f)) 
  \to f \subset a \times b
\]
が成り立ち, 定理 \ref{sthmproductsubsetgraph}より
\[
  f \subset a \times b \to {\rm Graph}(f)
\]
が成り立つから, 推論法則 \ref{dedmmp}により
\[
\tag{14}
  f \subset a \times b \wedge \forall x(x \in a \to \exists !y((x, y) \in f)) 
  \to {\rm Graph}(f)
\]
が成り立つ.
また定理 \ref{sthmprsetminimality}より
\[
  f \subset a \times b \to {\rm pr}_{1}\langle f \rangle \subset a \wedge {\rm pr}_{2}\langle f \rangle \subset b
\]
が成り立つから, 推論法則 \ref{dedprewedge}により
\begin{align*}
  \tag{15}
  f \subset a \times b &\to {\rm pr}_{1}\langle f \rangle \subset a, \\
  \mbox{} \\
  \tag{16}
  f \subset a \times b &\to {\rm pr}_{2}\langle f \rangle \subset b
\end{align*}
が共に成り立つ.
そこで(13), (16)から, 推論法則 \ref{dedmmp}によって
\[
\tag{17}
  f \subset a \times b \wedge \forall x(x \in a \to \exists !y((x, y) \in f)) 
  \to {\rm pr}_{2}\langle f \rangle \subset b
\]
が成り立つ.
またいま$\tau_{x}(\neg (x \in a \to x \in {\rm pr}_{1}\langle f \rangle))$を$U$と書けば, 
$U$は集合である.
また$y$が$x$と異なり, $a$及び$f$の中に自由変数として現れないことから, 
変数法則 \ref{valfund}, \ref{valtau}, \ref{valprset}によってわかるように, 
$y$は$U$の中に自由変数として現れない.
そしてThm \ref{thmallfund2}より
\[
  \forall x(x \in a \to \exists !y((x, y) \in f)) \to (U|x)(x \in a \to \exists !y((x, y) \in f))
\]
が成り立つ.
ここで$x$が$a$の中に自由変数として現れないことから, 
代入法則 \ref{substfree}, \ref{substfund}により, 上記の記号列は
\[
  \forall x(x \in a \to \exists !y((x, y) \in f)) \to (U \in a \to (U|x)(\exists !y((x, y) \in f)))
\]
と一致する.
また$y$が$x$と異なり, 上述のように$U$の中に自由変数として現れないことから, 
代入法則 \ref{substex!}により, この記号列は
\[
  \forall x(x \in a \to \exists !y((x, y) \in f)) \to (U \in a \to \exists !y((U|x)((x, y) \in f)))
\]
と一致する.
更に$x$が$y$と異なり, $f$の中に自由変数として現れないことから, 
代入法則 \ref{substfree}, \ref{substfund}, \ref{substpair}により, 
この記号列は
\[
\tag{18}
  \forall x(x \in a \to \exists !y((x, y) \in f)) \to (U \in a \to \exists !y((U, y) \in f))
\]
と一致する.
故にこれが定理となる.
またThm \ref{awbta}より
\[
\tag{19}
  \exists !y((U, y) \in f) \to \exists y((U, y) \in f)
\]
が成り立つ.
また$y$は$f$の中に自由変数として現れず, 
上述のように$U$の中にも自由変数として現れないから, 
定理 \ref{sthmprsetelement}と推論法則 \ref{dedequiv}により
\[
\tag{20}
  \exists y((U, y) \in f) \to U \in {\rm pr}_{1}\langle f \rangle
\]
が成り立つ.
そこで(19), (20)から, 推論法則 \ref{dedmmp}によって
\[
  \exists !y((U, y) \in f) \to U \in {\rm pr}_{1}\langle f \rangle
\]
が成り立ち, これから推論法則 \ref{dedaddb}によって
\[
\tag{21}
  (U \in a \to \exists !y((U, y) \in f)) \to (U \in a \to U \in {\rm pr}_{1}\langle f \rangle)
\]
が成り立つ.
また$U$の定義から, Thm \ref{thmallfund1}と推論法則 \ref{dedequiv}により
\[
  (U|x)(x \in a \to x \in {\rm pr}_{1}\langle f \rangle) \to \forall x(x \in a \to x \in {\rm pr}_{1}\langle f \rangle)
\]
が成り立つ.
いま$x$は$f$の中に自由変数として現れないから, 変数法則 \ref{valprset}により, 
$x$は${\rm pr}_{1}\langle f \rangle$の中に自由変数として現れない.
また$x$は$a$の中にも自由変数として現れない.
そこで代入法則 \ref{substfree}, \ref{substfund}及び定義によれば, 
上記の記号列は
\[
\tag{22}
  (U \in a \to U \in {\rm pr}_{1}\langle f \rangle) \to a \subset {\rm pr}_{1}\langle f \rangle
\]
と一致する.
故にこれが定理となる.
そこで(18), (21), (22)から, 推論法則 \ref{dedmmp}によって
\[
  \forall x(x \in a \to \exists !y((x, y) \in f)) \to a \subset {\rm pr}_{1}\langle f \rangle
\]
が成り立つことがわかる.
故にこれと(15)から, 推論法則 \ref{dedfromaddw}により
\[
\tag{23}
  f \subset a \times b \wedge \forall x(x \in a \to \exists !y((x, y) \in f)) 
  \to {\rm pr}_{1}\langle f \rangle \subset a \wedge a \subset {\rm pr}_{1}\langle f \rangle
\]
が成り立つ.
また定理 \ref{sthmaxiom1}と推論法則 \ref{dedequiv}により
\[
\tag{24}
  {\rm pr}_{1}\langle f \rangle \subset a \wedge a \subset {\rm pr}_{1}\langle f \rangle 
  \to {\rm pr}_{1}\langle f \rangle = a
\]
が成り立つ.
そこで(23), (24)から, 推論法則 \ref{dedmmp}によって
\[
\tag{25}
  f \subset a \times b \wedge \forall x(x \in a \to \exists !y((x, y) \in f)) 
  \to {\rm pr}_{1}\langle f \rangle = a
\]
が成り立つ.
さていま$\tau_{x}(\neg !y((x, y) \in f))$を$X$と書き, 
$\tau_{y}(\neg ((X, y) \notin f))$を$Y$と書けば, 
これらは共に集合である.
また変数法則 \ref{valfund}, \ref{valtau}, \ref{val!}によってわかるように, 
$y$は$X$の中に自由変数として現れない.
そして定理 \ref{sthmsubsetbasis}より
\[
\tag{26}
  f \subset a \times b \to ((X, Y) \in f \to (X, Y) \in a \times b)
\]
が成り立つ.
また定理 \ref{sthmpairinproduct}と推論法則 \ref{dedequiv}により
\[
  (X, Y) \in a \times b \to X \in a \wedge Y \in b
\]
が成り立つから, 推論法則 \ref{dedprewedge}により
\[
  (X, Y) \in a \times b \to X \in a
\]
が成り立ち, これから推論法則 \ref{dedaddb}によって
\[
\tag{27}
  ((X, Y) \in f \to (X, Y) \in a \times b) \to ((X, Y) \in f \to X \in a)
\]
が成り立つ.
またThm \ref{1atb1t1nbtna1}より
\[
\tag{28}
  ((X, Y) \in f \to X \in a) \to (X \notin a \to (X, Y) \notin f)
\]
が成り立つ.
そこで(26), (27), (28)から, 推論法則 \ref{dedmmp}によって
\[
  f \subset a \times b \to (X \notin a \to (X, Y) \notin f)
\]
が成り立つことがわかり, これから推論法則 \ref{dedtwch}により
\[
\tag{29}
  f \subset a \times b \wedge X \notin a \to (X, Y) \notin f
\]
が成り立つ.
また$Y$の定義から, Thm \ref{thmallfund1}と推論法則 \ref{dedequiv}により
\[
  (Y|y)((X, y) \notin f) \to \forall y((X, y) \notin f)
\]
が成り立つが, $y$は$f$の中に自由変数として現れず, 
上述のように$X$の中にも自由変数として現れないから, 
代入法則 \ref{substfree}, \ref{substfund}, \ref{substpair}によれば, 
この記号列は
\[
\tag{30}
  (X, Y) \notin f \to \forall y((X, y) \notin f)
\]
と一致する.
よってこれが定理となる.
またThm \ref{thmnquant!}より
\[
\tag{31}
  \forall y((X, y) \notin f) \to \ !y((X, y) \in f)
\]
が成り立つ.
そこで(29), (30), (31)から, 推論法則 \ref{dedmmp}によって
\[
  f \subset a \times b \wedge X \notin a \to \ !y((X, y) \in f)
\]
が成り立つことがわかり, これから推論法則 \ref{dedtwch}により
\[
\tag{32}
  f \subset a \times b \to (X \notin a \to \ !y((X, y) \in f))
\]
が成り立つ.
またThm \ref{thmallfund2}より
\[
  \forall x(x \in a \to \exists !y((x, y) \in f)) \to (X|x)(x \in a \to \exists !y((x, y) \in f))
\]
が成り立つ.
ここで$x$が$a$の中に自由変数として現れないことから, 
代入法則 \ref{substfree}, \ref{substfund}により, 
上記の記号列は
\[
  \forall x(x \in a \to \exists !y((x, y) \in f)) \to (X \in a \to (X|x)(\exists !y((x, y) \in f)))
\]
と一致する.
また$y$が$x$と異なり, 上述のように$X$の中に自由変数として現れないことから, 
代入法則 \ref{substex!}により, この記号列は
\[
  \forall x(x \in a \to \exists !y((x, y) \in f)) \to (X \in a \to \exists !y((X|x)((x, y) \in f)))
\]
と一致する.
更に$x$が$y$と異なり, $f$の中に自由変数として現れないことから, 
代入法則 \ref{substfree}, \ref{substfund}, \ref{substpair}により, この記号列は
\[
\tag{33}
  \forall x(x \in a \to \exists !y((x, y) \in f)) \to (X \in a \to \exists !y((X, y) \in f))
\]
と一致する.
故にこれが定理となる.
またThm \ref{awbta}より
\[
  \exists !y((X, y) \in f) \to \ !y((X, y) \in f)
\]
が成り立つから, 推論法則 \ref{dedaddb}により
\[
\tag{34}
  (X \in a \to \exists !y((X, y) \in f)) \to (X \in a \to \ !y((X, y) \in f))
\]
が成り立つ.
そこで(33), (34)から, 推論法則 \ref{dedmmp}によって
\[
  \forall x(x \in a \to \exists !y((x, y) \in f)) \to (X \in a \to \ !y((X, y) \in f))
\]
が成り立ち, これと(32)から, 推論法則 \ref{dedfromaddw}によって
\[
\tag{35}
  f \subset a \times b \wedge \forall x(x \in a \to \exists !y((x, y) \in f)) 
  \to (X \notin a \to \ !y((X, y) \in f)) \wedge (X \in a \to \ !y((X, y) \in f))
\]
が成り立つ.
またThm \ref{1atc1w1btc1t1avbtc1}より
\[
  (X \notin a \to \ !y((X, y) \in f)) \wedge (X \in a \to \ !y((X, y) \in f)) 
  \to (X \notin a \vee X \in a \to \ !y((X, y) \in f))
\]
が成り立つから, 推論法則 \ref{dedch}により
\[
  X \notin a \vee X \in a \to 
  ((X \notin a \to \ !y((X, y) \in f)) \wedge (X \in a \to \ !y((X, y) \in f)) 
  \to \ !y((X, y) \in f))
\]
が成り立つ.
いまThm \ref{avna}と推論法則 \ref{dedvch}により
$X \notin a \vee X \in a$が成り立つことがわかるから, 従ってこれと上記の定理から, 
推論法則 \ref{dedmp}により
\[
\tag{36}
  (X \notin a \to \ !y((X, y) \in f)) \wedge (X \in a \to \ !y((X, y) \in f)) 
  \to \ !y((X, y) \in f)
\]
が成り立つ.
また$X$の定義から, Thm \ref{thmallfund1}と推論法則 \ref{dedequiv}により
\[
  (X|x)(!y((x, y) \in f)) \to \forall x(!y((x, y) \in f))
\]
が成り立つ.
ここで$y$が$x$と異なり, 上述のように$X$の中に自由変数として現れないことから, 
代入法則 \ref{subst!}により, 上記の記号列は
\[
  !y((X|x)((x, y) \in f)) \to \forall x(!y((x, y) \in f))
\]
と一致する.
また$x$が$y$と異なり, $f$の中に自由変数として現れないことから, 
代入法則 \ref{substfree}, \ref{substfund}, \ref{substpair}により, 
この記号列は
\[
\tag{37}
  !y((X, y) \in f) \to \forall x(!y((x, y) \in f))
\]
と一致する.
故にこれが定理となる.
そこで(35), (36), (37)から, 推論法則 \ref{dedmmp}によって
\[
  f \subset a \times b \wedge \forall x(x \in a \to \exists !y((x, y) \in f)) 
  \to \forall x(!y((x, y) \in f))
\]
が成り立つことがわかる.
故にこれと(14)から, 推論法則 \ref{dedprewedge}により
\[
  f \subset a \times b \wedge \forall x(x \in a \to \exists !y((x, y) \in f)) 
  \to {\rm Graph}(f) \wedge \forall x(!y((x, y) \in f)), 
\]
が成り立つ.
ここで$x$と$y$が互いに異なり, 共に$f$の中に自由変数として現れないことから, 
定義により上記の記号列は
\[
  f \subset a \times b \wedge \forall x(x \in a \to \exists !y((x, y) \in f)) 
  \to {\rm Func}(f)
\]
と一致する.
よってこれが定理となる.
故にこれと(25)から, 再び推論法則 \ref{dedprewedge}により, 
\[
  f \subset a \times b \wedge \forall x(x \in a \to \exists !y((x, y) \in f)) 
  \to {\rm Func}(f) \wedge {\rm pr}_{1}\langle f \rangle = a, 
\]
即ち
\[
  f \subset a \times b \wedge \forall x(x \in a \to \exists !y((x, y) \in f)) 
  \to {\rm Func}(f; a)
\]
が成り立つ.
そしてこれと(17)から, やはり推論法則 \ref{dedprewedge}により, 
\[
  f \subset a \times b \wedge \forall x(x \in a \to \exists !y((x, y) \in f)) 
  \to {\rm Func}(f; a) \wedge {\rm pr}_{2}\langle f \rangle \subset b, 
\]
即ち(2)が成り立つ.
\halmos




\mathstrut
\begin{thm}
\label{sthmsingletonfunc}%定理
$a$と$b$を集合とするとき, $\{(a, b)\}$は$\{a\}$から$\{b\}$への函数である.
故に$\{(a, b)\}$は$\{a\}$における函数であり, また$\{(a, b)\}$は函数である.
\end{thm}


\noindent{\bf 証明}
~$x$と$y$を, 互いに異なり, 共に$a$及び$b$の中に自由変数として現れない, 定数でない文字とする.
このとき変数法則 \ref{valnset}, \ref{valpair}によってわかるように, 
$x$と$y$は共に$\{(a, b)\}$の中に自由変数として現れない.
そして定理 \ref{sthmsingletonbasis}と推論法則 \ref{dedequiv}により
\[
\tag{1}
  (x, y) \in \{(a, b)\} \to (x, y) = (a, b)
\]
が成り立つ.
また定理 \ref{sthmpair}と推論法則 \ref{dedequiv}により
\[
  (x, y) = (a, b) \to x = a \wedge y = b
\]
が成り立つから, 推論法則 \ref{dedprewedge}により
\[
\tag{2}
  (x, y) = (a, b) \to y = b
\]
が成り立つ.
そこで(1), (2)から, 推論法則 \ref{dedmmp}によって
\[
  (x, y) \in \{(a, b)\} \to y = b
\]
が成り立つ.
いま$y$は定数でなく, $b$の中に自由変数として現れないから, 
従って推論法則 \ref{ded!thmconst}により
\[
  !y((x, y) \in \{(a, b)\})
\]
が成り立つ.
また$x$も定数でないので, これから推論法則 \ref{dedltthmquan}により
\[
\tag{3}
  \forall x(!y((x, y) \in \{(a, b)\}))
\]
が成り立つ.
また定理 \ref{sthmbigpairpair}より$(a, b)$は対だから, 
定理 \ref{sthmsingletongraph}により, $\{(a, b)\}$はグラフである.
故にこのことと(3)から, 推論法則 \ref{dedwedge}により
\[
  {\rm Graph}(\{(a, b)\}) \wedge \forall x(!y((x, y) \in \{(a, b)\}))
\]
が成り立つ.
いま$x$と$y$は互いに異なり, 上述のように共に$\{(a, b)\}$の中に自由変数として現れないから, 
定義によれば, 上記の記号列は
\[
\tag{4}
  {\rm Func}(\{(a, b)\})
\]
と同じである.
故にこれが定理となる.
即ち, $\{(a, b)\}$は函数である.
また定理 \ref{sthmsingletonprset}より
\begin{align*}
  \tag{5}
  {\rm pr}_{1}\langle \{(a, b)\} \rangle &= \{a\}, \\
  \mbox{} \\
  \tag{6}
  {\rm pr}_{2}\langle \{(a, b)\} \rangle &= \{b\}
\end{align*}
が共に成り立つ.
そこで(4), (5)から, 推論法則 \ref{dedwedge}により
\[
  {\rm Func}(\{(a, b)\}) \wedge {\rm pr}_{1}\langle \{(a, b)\} \rangle = \{a\}, 
\]
即ち
\[
\tag{7}
  {\rm Func}(\{(a, b)\}; \{a\})
\]
が成り立つ.
即ち, $\{(a, b)\}$は$\{a\}$における函数である.
また(6)から, 定理 \ref{sthm=tsubset}により
\[
  {\rm pr}_{2}\langle \{(a, b)\} \rangle \subset \{b\}
\]
が成り立つ.
そこでこれと(7)から, 推論法則 \ref{dedwedge}により
\[
  {\rm Func}(\{(a, b)\}; \{a\}) \wedge {\rm pr}_{2}\langle \{(a, b)\} \rangle \subset \{b\}, 
\]
即ち
\[
  {\rm Func}(\{(a, b)\}; \{a\}; \{b\})
\]
が成り立つ.
即ち, $\{(a, b)\}$は$\{a\}$から$\{b\}$への函数である.
\halmos




\mathstrut
\begin{thm}
\label{sthmcupfunc}%定理
\mbox{}

1)
$f$と$g$を集合とするとき, 
\[
  {\rm Func}(f \cup g) \to {\rm Func}(f) \wedge {\rm Func}(g)
\]
が成り立つ.
またこのことから, 次の($*$)が成り立つ: 

($*$) ~~$f \cup g$が函数ならば, $f$と$g$は共に函数である.

2)
$a$, $b$, $f$, $g$を集合とするとき, 
\[
  a \cap b = \phi \to ({\rm Func}(f; a) \wedge {\rm Func}(g; b) \to {\rm Func}(f \cup g; a \cup b))
\]
が成り立つ.
またこのことから, 次の($**$)が成り立つ: 

($**$) ~~$a \cap b$が空ならば, 
         \[
           {\rm Func}(f; a) \wedge {\rm Func}(g; b) \to {\rm Func}(f \cup g; a \cup b)
         \]
         が成り立つ.
         故に, $a \cap b$が空であり, $f$, $g$がそれぞれ$a$, $b$における函数ならば, 
         $f \cup g$は$a \cup b$における函数である.

3)
$a$, $b$, $c$, $d$, $f$, $g$を集合とするとき, 
\[
  a \cap c = \phi \to ({\rm Func}(f; a; b) \wedge {\rm Func}(g; c; d) \to {\rm Func}(f \cup g; a \cup c; b \cup d))
\]
が成り立つ.
またこのことから, 次の(${**}*$)が成り立つ: 

(${**}*$) ~~$a \cap c$が空ならば, 
            \[
              {\rm Func}(f; a; b) \wedge {\rm Func}(g; c; d) \to {\rm Func}(f \cup g; a \cup c; b \cup d)
            \]
            が成り立つ.
            故に, $a \cap c$が空であり, $f$, $g$がそれぞれ$a$から$b$, $c$から$d$への函数ならば, 
            $f \cup g$は$a \cup c$から$b \cup d$への函数である.
\end{thm}


\noindent{\bf 証明}
~1)
定理 \ref{sthmsubsetcup}より
\[
  f \subset f \cup g, ~~
  g \subset f \cup g
\]
が共に成り立つから, 定理 \ref{sthmfuncsubset}により
\[
  {\rm Func}(f \cup g) \to {\rm Func}(f), ~~
  {\rm Func}(f \cup g) \to {\rm Func}(g)
\]
が共に成り立つ.
故に推論法則 \ref{dedprewedge}により
\[
  {\rm Func}(f \cup g) \to {\rm Func}(f) \wedge {\rm Func}(g)
\]
が成り立つ.
($*$)が成り立つことは, これと推論法則 \ref{dedmp}, \ref{dedwedge}によって明らかである.

\noindent
2)
まずThm \ref{awbta}より
\[
\tag{1}
  a \cap b = \phi \wedge ({\rm Func}(f; a) \wedge {\rm Func}(g; b)) \to {\rm Func}(f; a) \wedge {\rm Func}(g; b)
\]
が成り立つ.
また定理 \ref{sthmfuncbasis}より
\[
  {\rm Func}(f; a) \to {\rm Graph}(f), ~~
  {\rm Func}(g; b) \to {\rm Graph}(g)
\]
が共に成り立つから, 推論法則 \ref{dedfromaddw}により
\[
\tag{2}
  {\rm Func}(f; a) \wedge {\rm Func}(g; b) \to {\rm Graph}(f) \wedge {\rm Graph}(g)
\]
が成り立つ.
また定理 \ref{sthmcupgraph}と推論法則 \ref{dedequiv}により
\[
\tag{3}
  {\rm Graph}(f) \wedge {\rm Graph}(g) \to {\rm Graph}(f \cup g)
\]
が成り立つ.
そこで(1), (2), (3)から, 推論法則 \ref{dedmmp}によって
\[
\tag{4}
  a \cap b = \phi \wedge ({\rm Func}(f; a) \wedge {\rm Func}(g; b)) \to {\rm Graph}(f \cup g)
\]
が成り立つことがわかる.
また定理 \ref{sthmfuncbasis}より
\[
  {\rm Func}(f; a) \to {\rm pr}_{1}\langle f \rangle = a, ~~
  {\rm Func}(g; b) \to {\rm pr}_{1}\langle g \rangle = b
\]
が共に成り立つから, 推論法則 \ref{dedfromaddw}により
\[
\tag{5}
  {\rm Func}(f; a) \wedge {\rm Func}(g; b) 
  \to {\rm pr}_{1}\langle f \rangle = a \wedge {\rm pr}_{1}\langle g \rangle = b
\]
が成り立つ.
また定理 \ref{sthmcup=}より
\[
\tag{6}
  {\rm pr}_{1}\langle f \rangle = a \wedge {\rm pr}_{1}\langle g \rangle = b 
  \to {\rm pr}_{1}\langle f \rangle \cup {\rm pr}_{1}\langle g \rangle = a \cup b
\]
が成り立つ.
また定理 \ref{sthmcupprset}より
\[
  {\rm pr}_{1}\langle f \cup g \rangle = {\rm pr}_{1}\langle f \rangle \cup {\rm pr}_{1}\langle g \rangle
\]
が成り立つから, 推論法則 \ref{dedaddeq=}により
\[
  {\rm pr}_{1}\langle f \cup g \rangle = a \cup b 
  \leftrightarrow {\rm pr}_{1}\langle f \rangle \cup {\rm pr}_{1}\langle g \rangle = a \cup b
\]
が成り立ち, 故に推論法則 \ref{dedequiv}により
\[
\tag{7}
  {\rm pr}_{1}\langle f \rangle \cup {\rm pr}_{1}\langle g \rangle = a \cup b 
  \to {\rm pr}_{1}\langle f \cup g \rangle = a \cup b
\]
が成り立つ.
そこで(1), (5), (6), (7)から, 推論法則 \ref{dedmmp}によって
\[
\tag{8}
  a \cap b = \phi \wedge ({\rm Func}(f; a) \wedge {\rm Func}(g; b)) \to {\rm pr}_{1}\langle f \cup g \rangle = a \cup b
\]
が成り立つことがわかる.
さてここで$x$と$y$を, 互いに異なり, 共に$a$, $b$, $f$, $g$のいずれの記号列の中にも
自由変数として現れない, 定数でない文字とする.
このとき定理 \ref{sthmpairelementinprset}より
\[
  (x, y) \in f \to x \in {\rm pr}_{1}\langle f \rangle \wedge y \in {\rm pr}_{2}\langle f \rangle
\]
が成り立つから, 推論法則 \ref{dedprewedge}により
\[
  (x, y) \in f \to x \in {\rm pr}_{1}\langle f \rangle
\]
が成り立つ.
故にこれと上述の定理${\rm Func}(f; a) \to {\rm pr}_{1}\langle f \rangle = a$とから, 
推論法則 \ref{dedfromaddw}によって
\[
\tag{9}
  {\rm Func}(f; a) \wedge (x, y) \in f \to {\rm pr}_{1}\langle f \rangle = a \wedge x \in {\rm pr}_{1}\langle f \rangle
\]
が成り立つ.
また定理 \ref{sthm=&in}より
\[
\tag{10}
  {\rm pr}_{1}\langle f \rangle = a \wedge x \in {\rm pr}_{1}\langle f \rangle \to x \in a
\]
が成り立つ.
そこで(9), (10)から, 推論法則 \ref{dedmmp}によって
\[
  {\rm Func}(f; a) \wedge (x, y) \in f \to x \in a
\]
が成り立ち, これから推論法則 \ref{dedtwch}によって
\[
\tag{11}
  {\rm Func}(f; a) \to ((x, y) \in f \to x \in a)
\]
が成り立つ.
またThm \ref{1atb1t1nbtna1}より
\[
\tag{12}
  ((x, y) \in f \to x \in a) \to (x \notin a \to (x, y) \notin f)
\]
が成り立つ.
そこで(11), (12)から, 推論法則 \ref{dedmmp}によって
\[
  {\rm Func}(f; a) \to (x \notin a \to (x, y) \notin f)
\]
が成り立つ.
故に推論法則 \ref{dedch}によって
\[
  x \notin a \to ({\rm Func}(f; a) \to (x, y) \notin f)
\]
が成り立ち, これから推論法則 \ref{dedtwch}によって
\[
\tag{13}
  x \notin a \wedge {\rm Func}(f; a) \to (x, y) \notin f
\]
が成り立つ.
また定理 \ref{sthmcupbasis}と推論法則 \ref{dedequiv}により
\[
\tag{14}
  (x, y) \in f \cup g \to (x, y) \in f \vee (x, y) \in g
\]
が成り立つ.
そこで(13), (14)から, 推論法則 \ref{dedfromaddw}によって
\[
  (x \notin a \wedge {\rm Func}(f; a)) \wedge (x, y) \in f \cup g \to (x, y) \notin f \wedge ((x, y) \in f \vee (x, y) \in g)
\]
が成り立つが, 定義によればこの記号列は
\[
\tag{15}
  (x \notin a \wedge {\rm Func}(f; a)) \wedge (x, y) \in f \cup g \to (x, y) \notin f \wedge ((x, y) \notin f \to (x, y) \in g)
\]
と同じだから, これが定理となる.
またThm \ref{at11atb1tb1}より
\[
  (x, y) \notin f \to (((x, y) \notin f \to (x, y) \in g) \to (x, y) \in g)
\]
が成り立つから, 推論法則 \ref{dedtwch}により
\[
\tag{16}
  (x, y) \notin f \wedge ((x, y) \notin f \to (x, y) \in g) \to (x, y) \in g
\]
が成り立つ.
そこで(15), (16)から, 推論法則 \ref{dedmmp}によって
\[
  (x \notin a \wedge {\rm Func}(f; a)) \wedge (x, y) \in f \cup g \to (x, y) \in g
\]
が成り立ち, これから推論法則 \ref{dedtwch}によって
\[
\tag{17}
  x \notin a \wedge {\rm Func}(f; a) \to ((x, y) \in f \cup g \to (x, y) \in g)
\]
が成り立つ.
さていま$y$は$x$と異なり, $a$及び$f$の中に自由変数として現れないから, 
変数法則 \ref{valfund}, \ref{valwedge}, \ref{valonafunc}によってわかるように, 
$y$は$x \notin a \wedge {\rm Func}(f; a)$の中に自由変数として現れない.
また$y$は定数でない.
このことと, (17)が成り立つことから, 推論法則 \ref{dedalltquansepfreeconst}により
\[
\tag{18}
  x \notin a \wedge {\rm Func}(f; a) \to \forall y((x, y) \in f \cup g \to (x, y) \in g)
\]
が成り立つ.
またThm \ref{thmallt!sep}より
\[
\tag{19}
  \forall y((x, y) \in f \cup g \to (x, y) \in g) \to (!y((x, y) \in g) \to \ !y((x, y) \in f \cup g))
\]
が成り立つ.
そこで(18), (19)から, 推論法則 \ref{dedmmp}によって
\[
  x \notin a \wedge {\rm Func}(f; a) \to (!y((x, y) \in g) \to \ !y((x, y) \in f \cup g))
\]
が成り立ち, これから推論法則 \ref{dedtwch}によって
\[
\tag{20}
  (x \notin a \wedge {\rm Func}(f; a)) \ \wedge \ !y((x, y) \in g) \to \ !y((x, y) \in f \cup g)%空白
\]
が成り立つ.
また$x$と$y$が互いに異なり, 共に$g$の中に自由変数として現れないことから, 定理 \ref{sthmfuncbasis}より
\[
  {\rm Func}(g; b) \to \forall x(!y((x, y) \in g))
\]
が成り立つ.
またThm \ref{thmallfund3}より
\[
  \forall x(!y((x, y) \in g)) \to \ !y((x, y) \in g)
\]
が成り立つ.
そこでこれらから, 推論法則 \ref{dedmmp}によって
\[
  {\rm Func}(g; b) \to \ !y((x, y) \in g)
\]
が成り立つ.
故に推論法則 \ref{dedaddw}により
\[
\tag{21}
  (x \notin a \wedge {\rm Func}(f; a)) \wedge {\rm Func}(g; b) \to (x \notin a \wedge {\rm Func}(f; a)) \ \wedge \ !y((x, y) \in g)%空白
\]
が成り立つ.
またThm \ref{aw1bwc1t1awb1wc}より
\[
\tag{22}
  x \notin a \wedge ({\rm Func}(f; a) \wedge {\rm Func}(g; b)) 
  \to (x \notin a \wedge {\rm Func}(f; a)) \wedge {\rm Func}(g; b)
\]
が成り立つ.
そこで(22), (21), (20)にこの順で推論法則 \ref{dedmmp}を適用していき, 
\[
\tag{23}
  x \notin a \wedge ({\rm Func}(f; a) \wedge {\rm Func}(g; b)) \to \ !y((x, y) \in f \cup g)
\]
が成り立つことがわかる.
いまこの(23)において, $a$と$b$, $f$と$g$を入れ替えれば, $x$と$y$のこれらの記号列に対しての
条件が同じであるから, 
\[
\tag{24}
  x \notin b \wedge ({\rm Func}(g; b) \wedge {\rm Func}(f; a)) \to \ !y((x, y) \in g \cup f)
\]
も成り立つことがわかる.
またThm \ref{awbtbwa}より
\[
  {\rm Func}(f; a) \wedge {\rm Func}(g; b) \to {\rm Func}(g; b) \wedge {\rm Func}(f; a)
\]
が成り立つから, 推論法則 \ref{dedaddw}により
\[
\tag{25}
  x \notin b \wedge ({\rm Func}(f; a) \wedge {\rm Func}(g; b)) 
  \to x \notin b \wedge ({\rm Func}(g; b) \wedge {\rm Func}(f; a))
\]
が成り立つ.
また定理 \ref{sthmcupch}より$g \cup f = f \cup g$が成り立つから, 
定理 \ref{sthm=tineq}により
\[
  (x, y) \in g \cup f \leftrightarrow (x, y) \in f \cup g
\]
が成り立つ.
$y$は定数でないので, これから推論法則 \ref{dedalleq!sepconst}により
\[
  !y((x, y) \in g \cup f) \leftrightarrow \ !y((x, y) \in f \cup g)
\]
が成り立つ.
故に推論法則 \ref{dedequiv}により
\[
\tag{26}
  !y((x, y) \in g \cup f) \to \ !y((x, y) \in f \cup g)
\]
が成り立つ.
そこで(25), (24), (26)にこの順で推論法則 \ref{dedmmp}を適用していき, 
\[
\tag{27}
  x \notin b \wedge ({\rm Func}(f; a) \wedge {\rm Func}(g; b)) \to \ !y((x, y) \in f \cup g)
\]
が成り立つことがわかる.
故に(23)と(27)から, それぞれ推論法則 \ref{dedtwch}によって
\begin{align*}
  x \notin a &\to ({\rm Func}(f; a) \wedge {\rm Func}(g; b) \to \ !y((x, y) \in f \cup g)), \\
  \mbox{} \\
  x \notin b &\to ({\rm Func}(f; a) \wedge {\rm Func}(g; b) \to \ !y((x, y) \in f \cup g))
\end{align*}
が成り立ち, これらから, 推論法則 \ref{deddil}により
\[
\tag{28}
  x \notin a \vee x \notin b \to ({\rm Func}(f; a) \wedge {\rm Func}(g; b) \to \ !y((x, y) \in f \cup g))
\]
が成り立つ.
また定理 \ref{sthmnotinempty}より
\[
\tag{29}
  a \cap b = \phi \to x \notin a \cap b
\]
が成り立つ.
また定理 \ref{sthmcapelement}と推論法則 \ref{dedequiv}により
\[
  x \in a \wedge x \in b \to x \in a \cap b
\]
が成り立つから, 推論法則 \ref{dedcp}により
\[
\tag{30}
  x \notin a \cap b \to \neg (x \in a \wedge x \in b)
\]
が成り立つ.
またThm \ref{n1awb1tnavnb}より
\[
\tag{31}
  \neg (x \in a \wedge x \in b) \to x \notin a \vee x \notin b
\]
が成り立つ.
そこで(29), (30), (31), (28)にこの順で推論法則 \ref{dedmmp}を適用していき, 
\[
  a \cap b = \phi \to ({\rm Func}(f; a) \wedge {\rm Func}(g; b) \to \ !y((x, y) \in f \cup g))
\]
が成り立つことがわかる.
故に推論法則 \ref{dedtwch}により
\[
\tag{32}
  a \cap b = \phi \wedge ({\rm Func}(f; a) \wedge {\rm Func}(g; b)) \to \ !y((x, y) \in f \cup g)
\]
が成り立つ.
さていま$x$は$a$, $b$, $f$, $g$のいずれの記号列の中にも自由変数として現れないから, 
変数法則 \ref{valfund}, \ref{valwedge}, \ref{valcap}, \ref{valempty}, \ref{valonafunc}によって
わかるように, $x$は
$a \cap b = \phi \wedge ({\rm Func}(f; a) \wedge {\rm Func}(g; b))$の中に自由変数として現れない.
また$x$は定数でない.
このことと, (32)が成り立つことから, 推論法則 \ref{dedalltquansepfreeconst}により
\[
  a \cap b = \phi \wedge ({\rm Func}(f; a) \wedge {\rm Func}(g; b)) \to \forall x(!y((x, y) \in f \cup g))
\]
が成り立つ.
故にこれと(4)から, 推論法則 \ref{dedprewedge}によって
\[
  a \cap b = \phi \wedge ({\rm Func}(f; a) \wedge {\rm Func}(g; b)) \to {\rm Graph}(f \cup g) \wedge \forall x(!y((x, y) \in f \cup g))
\]
が成り立つ.
いま$x$と$y$は共に$f$及び$g$の中に自由変数として現れないから, 
変数法則 \ref{valcup}により, これらは共に$f \cup g$の中に自由変数として現れない.
また$x$と$y$は互いに異なる.
そこで定義から, 上記の記号列は
\[
  a \cap b = \phi \wedge ({\rm Func}(f; a) \wedge {\rm Func}(g; b)) \to {\rm Func}(f \cup g)
\]
と同じである.
故にこれが定理となる.
そこでこれと(8)から, 再び推論法則 \ref{dedprewedge}によって, 
\[
  a \cap b = \phi \wedge ({\rm Func}(f; a) \wedge {\rm Func}(g; b)) 
  \to {\rm Func}(f \cup g) \wedge {\rm pr}_{1}\langle f \cup g \rangle = a \cup b, 
\]
即ち
\[
  a \cap b = \phi \wedge ({\rm Func}(f; a) \wedge {\rm Func}(g; b)) \to {\rm Func}(f \cup g; a \cup b)
\]
が成り立つ.
故に推論法則 \ref{dedtwch}により
\[
  a \cap b = \phi \to ({\rm Func}(f; a) \wedge {\rm Func}(g; b) \to {\rm Func}(f \cup g; a \cup b))
\]
が成り立つ.
($**$)が成り立つことは, これと推論法則 \ref{dedmp}, \ref{dedwedge}によって明らかである.

\noindent
3)
Thm \ref{awbta}より
\[
\tag{33}
  a \cap c = \phi \wedge ({\rm Func}(f; a; b) \wedge {\rm Func}(g; c; d)) \to {\rm Func}(f; a; b) \wedge {\rm Func}(g; c; d)
\]
が成り立つ.
また定理 \ref{sthmfuncbasis}より
\[
  {\rm Func}(f; a; b) \to {\rm Func}(f; a), ~~
  {\rm Func}(g; c; d) \to {\rm Func}(g; c)
\]
が共に成り立つから, 推論法則 \ref{dedfromaddw}により
\[
  {\rm Func}(f; a; b) \wedge {\rm Func}(g; c; d) \to {\rm Func}(f; a) \wedge {\rm Func}(g; c)
\]
が成り立ち, これから推論法則 \ref{dedaddw}によって
\[
\tag{34}
  a \cap c = \phi \wedge ({\rm Func}(f; a; b) \wedge {\rm Func}(g; c; d)) 
  \to a \cap c = \phi \wedge ({\rm Func}(f; a) \wedge {\rm Func}(g; c))
\]
が成り立つ.
また2)より
\[
  a \cap c = \phi \to ({\rm Func}(f; a) \wedge {\rm Func}(g; c) \to {\rm Func}(f \cup g; a \cup c))
\]
が成り立つから, 推論法則 \ref{dedtwch}により
\[
\tag{35}
  a \cap c = \phi \wedge ({\rm Func}(f; a) \wedge {\rm Func}(g; c)) \to {\rm Func}(f \cup g; a \cup c)
\]
が成り立つ.
そこで(34), (35)から, 推論法則 \ref{dedmmp}によって
\[
\tag{36}
  a \cap c = \phi \wedge ({\rm Func}(f; a; b) \wedge {\rm Func}(g; c; d)) \to {\rm Func}(f \cup g; a \cup c)
\]
が成り立つ.
また定理 \ref{sthmfuncbasis}より
\[
  {\rm Func}(f; a; b) \to {\rm pr}_{2}\langle f \rangle \subset b, ~~
  {\rm Func}(g; c; d) \to {\rm pr}_{2}\langle g \rangle \subset d
\]
が共に成り立つから, 推論法則 \ref{dedfromaddw}により
\[
\tag{37}
  {\rm Func}(f; a; b) \wedge {\rm Func}(g; c; d) 
  \to {\rm pr}_{2}\langle f \rangle \subset b \wedge {\rm pr}_{2}\langle g \rangle \subset d
\]
が成り立つ.
また定理 \ref{sthmcupsubset}より
\[
\tag{38}
  {\rm pr}_{2}\langle f \rangle \subset b \wedge {\rm pr}_{2}\langle g \rangle \subset d 
  \to {\rm pr}_{2}\langle f \rangle \cup {\rm pr}_{2}\langle g \rangle \subset b \cup d
\]
が成り立つ.
また定理 \ref{sthmcupprset}より
\[
  {\rm pr}_{2}\langle f \cup g \rangle = {\rm pr}_{2}\langle f \rangle \cup {\rm pr}_{2}\langle g \rangle
\]
が成り立つから, 定理 \ref{sthm=tsubseteq}により
\[
  {\rm pr}_{2}\langle f \cup g \rangle \subset b \cup d 
  \leftrightarrow {\rm pr}_{2}\langle f \rangle \cup {\rm pr}_{2}\langle g \rangle \subset b \cup d
\]
が成り立ち, これから推論法則 \ref{dedequiv}により
\[
\tag{39}
  {\rm pr}_{2}\langle f \rangle \cup {\rm pr}_{2}\langle g \rangle \subset b \cup d 
  \to {\rm pr}_{2}\langle f \cup g \rangle \subset b \cup d
\]
が成り立つ.
そこで(33), (37), (38), (39)から, 推論法則 \ref{dedmmp}によって
\[
  a \cap c = \phi \wedge ({\rm Func}(f; a; b) \wedge {\rm Func}(g; c; d)) \to {\rm pr}_{2}\langle f \cup g \rangle \subset b \cup d
\]
が成り立つことがわかる.
故にこれと(36)から, 推論法則 \ref{dedprewedge}によって, 
\[
  a \cap c = \phi \wedge ({\rm Func}(f; a; b) \wedge {\rm Func}(g; c; d)) 
  \to {\rm Func}(f \cup g; a \cup c) \wedge {\rm pr}_{2}\langle f \cup g \rangle \subset b \cup d, 
\]
即ち
\[
  a \cap c = \phi \wedge ({\rm Func}(f; a; b) \wedge {\rm Func}(g; c; d)) \to {\rm Func}(f \cup g; a \cup c; b \cup d)
\]
が成り立つ.
そこでこれから推論法則 \ref{dedtwch}によって
\[
  a \cap c = \phi \to ({\rm Func}(f; a; b) \wedge {\rm Func}(g; c; d) \to {\rm Func}(f \cup g; a \cup c; b \cup d))
\]
が成り立つ.
(${**}*$)が成り立つことは, これと推論法則 \ref{dedmp}, \ref{dedwedge}によって明らかである.
\halmos




\mathstrut
\begin{thm}
\label{sthmcapfunc}%定理
$f$と$g$を集合とするとき, 
\[
  {\rm Func}(f) \vee {\rm Func}(g) \to {\rm Func}(f \cap g)
\]
が成り立つ.
またこのことから, 次の($*$)が成り立つ: 

($*$) ~~$f$が函数ならば, $f \cap g$は函数である.
        また$g$が函数ならば, $f \cap g$は函数である.
\end{thm}


\noindent{\bf 証明}
~定理 \ref{sthmcap}より
\[
  f \cap g \subset f, ~~
  f \cap g \subset g
\]
が共に成り立つから, 定理 \ref{sthmfuncsubset}により
\[
  {\rm Func}(f) \to {\rm Func}(f \cap g), ~~
  {\rm Func}(g) \to {\rm Func}(f \cap g)
\]
が共に成り立つ.
故に推論法則 \ref{deddil}により
\[
  {\rm Func}(f) \vee {\rm Func}(g) \to {\rm Func}(f \cap g)
\]
が成り立つ.
($*$)が成り立つことは, これと推論法則 \ref{dedmp}, \ref{dedvee}によって明らかである.
\halmos




\mathstrut
\begin{thm}
\label{sthm-func}%定理
$f$と$g$を集合とするとき, 
\[
  {\rm Func}(f) \to {\rm Func}(f - g)
\]
が成り立つ.
またこのことから, 次の($*$)が成り立つ: 

($*$) ~~$f$が函数ならば, $f - g$は函数である.
\end{thm}


\noindent{\bf 証明}
~定理 \ref{sthma-bsubseta}より$f - g \subset f$が成り立つから, 
定理 \ref{sthmfuncsubset}により
\[
  {\rm Func}(f) \to {\rm Func}(f - g)
\]
が成り立つ.
($*$)が成り立つことは, これと推論法則 \ref{dedmp}によって明らかである.
\halmos




\mathstrut
\begin{thm}
\label{sthmemptyfunc}%定理
\mbox{}

1)
$f$を集合とするとき, 
\[
  {\rm Func}(f; \phi) \leftrightarrow f = \phi
\]
が成り立つ.
またこのことから, 次の($*$)が成り立つ: 

($*$) ~~$f$が$\phi$における函数ならば, $f$は空である.
        逆に$f$が空ならば, $f$は$\phi$における函数である.

2)
$b$と$f$を集合とするとき, 
\[
  {\rm Func}(f; \phi; b) \leftrightarrow f = \phi
\]
が成り立つ.
またこのことから, 次の($**$)が成り立つ: 

($**$) ~~$f$が$\phi$から$b$への函数ならば, $f$は空である.
         逆に$f$が空ならば, $f$は$\phi$から$b$への函数である.

3)
$a$と$f$を集合とするとき, 
\[
  {\rm Func}(f; a; \phi) \leftrightarrow a = \phi \wedge f = \phi
\]
が成り立つ.
またこのことから, 次の(${**}*$)が成り立つ: 

(${**}*$) ~~$f$が$a$から$\phi$への函数ならば, $a$と$f$は共に空である.
            逆に$a$と$f$が共に空ならば, $f$は$a$から$\phi$への函数である.

4)
$\phi$は$\phi$から$\phi$への函数である.
故に$\phi$は$\phi$における函数であり, また$\phi$は函数である.
\end{thm}


\noindent{\bf 証明}
~1)
定理 \ref{sthmfuncbasis}より
\[
  {\rm Func}(f; \phi) \to {\rm Graph}(f), ~~
  {\rm Func}(f; \phi) \to {\rm pr}_{1}\langle f \rangle = \phi
\]
が共に成り立つから, 推論法則 \ref{dedprewedge}により
\[
\tag{1}
  {\rm Func}(f; \phi) \to {\rm Graph}(f) \wedge {\rm pr}_{1}\langle f \rangle = \phi
\]
が成り立つ.
また定理 \ref{sthmemptyprset}と推論法則 \ref{dedequiv}により
\[
\tag{2}
  {\rm Graph}(f) \wedge {\rm pr}_{1}\langle f \rangle = \phi \to f = \phi
\]
が成り立つ.
故に(1), (2)から, 推論法則 \ref{dedmmp}によって
\[
\tag{3}
  {\rm Func}(f; \phi) \to f = \phi
\]
が成り立つ.
またやはり定理 \ref{sthmemptyprset}と推論法則 \ref{dedequiv}により
\[
  f = \phi \to {\rm Graph}(f) \wedge {\rm pr}_{1}\langle f \rangle = \phi
\]
が成り立つから, 推論法則 \ref{dedprewedge}により
\begin{align*}
  \tag{4}
  f = \phi &\to {\rm Graph}(f), \\
  \mbox{} \\
  \tag{5}
  f = \phi &\to {\rm pr}_{1}\langle f \rangle = \phi
\end{align*}
が共に成り立つ.
ここで$x$と$y$を, 互いに異なり, 共に$f$の中に自由変数として現れない, 
定数でない文字とする.
このとき変数法則 \ref{valfund}, \ref{valempty}により, 
$x$と$y$は共に$f = \phi$の中に自由変数として現れない.
また定理 \ref{sthmnotinempty}より
\[
  f = \phi \to (x, y) \notin f
\]
が成り立つ.
故に推論法則 \ref{dedalltquansepfreeconst}により, 
\[
\tag{6}
  f = \phi \to \forall x(\forall y((x, y) \notin f))
\]
が成り立つことがわかる.
またThm \ref{thmnquant!}より
\[
  \forall y((x, y) \notin f) \to \ !y((x, y) \in f)
\]
が成り立つから, これと$x$が定数でないことから, 推論法則 \ref{dedalltquansepconst}により
\[
\tag{7}
  \forall x(\forall y((x, y) \notin f)) \to \forall x(!y((x, y) \in f))
\]
が成り立つ.
そこで(6), (7)から, 推論法則 \ref{dedmmp}によって
\[
  f = \phi \to \forall x(!y((x, y) \in f))
\]
が成り立つ.
故にこれと(4)から, 推論法則 \ref{dedprewedge}によって
\[
  f = \phi \to {\rm Graph}(f) \wedge \forall x(!y((x, y) \in f))
\]
が成り立ち, これと(5)から, 再び推論法則 \ref{dedprewedge}によって
\[
  f = \phi \to ({\rm Graph}(f) \wedge \forall x(!y((x, y) \in f))) \wedge {\rm pr}_{1}\langle f \rangle = \phi
\]
が成り立つ.
いま$x$と$y$は互いに異なり, 共に$f$の中に自由変数として現れないから, 
定義によればこの記号列は
\[
\tag{8}
  f = \phi \to {\rm Func}(f; \phi)
\]
と同じである.
故にこれが定理となる.
従って(3), (8)から, 推論法則 \ref{dedequiv}により
\[
  {\rm Func}(f; \phi) \leftrightarrow f = \phi
\]
が成り立つ.
($*$)が成り立つことは, これと推論法則 \ref{dedeqfund}によって明らかである.

\noindent
2)
定理 \ref{sthmfuncbasis}より
\[
  {\rm Func}(f; \phi; b) \to {\rm Func}(f; \phi)
\]
が成り立つから, これと(3)から, 推論法則 \ref{dedmmp}によって
\[
\tag{9}
  {\rm Func}(f; \phi; b) \to f = \phi
\]
が成り立つ.
また定理 \ref{sthmemptyprset}と推論法則 \ref{dedequiv}により
\[
  f = \phi \to {\rm Graph}(f) \wedge {\rm pr}_{2}\langle f \rangle = \phi
\]
が成り立つから, 推論法則 \ref{dedprewedge}により
\[
\tag{10}
  f = \phi \to {\rm pr}_{2}\langle f \rangle = \phi
\]
が成り立つ.
また定理 \ref{sthmemptysubset}より
\[
\tag{11}
  {\rm pr}_{2}\langle f \rangle = \phi \to {\rm pr}_{2}\langle f \rangle \subset b
\]
が成り立つ.
故に(10), (11)から, 推論法則 \ref{dedmmp}によって
\[
  f = \phi \to {\rm pr}_{2}\langle f \rangle \subset b
\]
が成り立ち, これと(8)から, 推論法則 \ref{dedprewedge}によって
\[
  f = \phi \to {\rm Func}(f; \phi) \wedge {\rm pr}_{2}\langle f \rangle \subset b, 
\]
即ち
\[
\tag{12}
  f = \phi \to {\rm Func}(f; \phi; b)
\]
が成り立つ.
従って(9), (12)から, 推論法則 \ref{dedequiv}により
\[
  {\rm Func}(f; \phi; b) \leftrightarrow f = \phi
\]
が成り立つ.
($**$)が成り立つことは, これと推論法則 \ref{dedeqfund}によって明らかである.

\noindent
3)
定理 \ref{sthmfuncbasis}より
\[
  {\rm Func}(f; a; \phi) \to {\rm Graph}(f)
\]
が成り立ち, 定理 \ref{sthmgraphprset}と推論法則 \ref{dedequiv}により
\[
  {\rm Graph}(f) \to f \subset {\rm pr}_{1}\langle f \rangle \times {\rm pr}_{2}\langle f \rangle
\]
が成り立つから, 推論法則 \ref{dedmmp}によって
\[
\tag{13}
  {\rm Func}(f; a; \phi) \to f \subset {\rm pr}_{1}\langle f \rangle \times {\rm pr}_{2}\langle f \rangle
\]
が成り立つ.
また定理 \ref{sthmfuncbasis}より
\[
\tag{14}
  {\rm Func}(f; a; \phi) \to {\rm pr}_{2}\langle f \rangle \subset \phi
\]
が成り立つ.
また定理 \ref{sthmproductsubset}より
\[
\tag{15}
  {\rm pr}_{2}\langle f \rangle \subset \phi 
  \to {\rm pr}_{1}\langle f \rangle \times {\rm pr}_{2}\langle f \rangle \subset {\rm pr}_{1}\langle f \rangle \times \phi
\]
が成り立つ.
また定理 \ref{sthmemptyproduct}より${\rm pr}_{1}\langle f \rangle \times \phi = \phi$が
成り立つから, 定理 \ref{sthm=tsubseteq}により
\[
  {\rm pr}_{1}\langle f \rangle \times {\rm pr}_{2}\langle f \rangle \subset {\rm pr}_{1}\langle f \rangle \times \phi 
  \leftrightarrow {\rm pr}_{1}\langle f \rangle \times {\rm pr}_{2}\langle f \rangle \subset \phi
\]
が成り立ち, これから推論法則 \ref{dedequiv}により
\[
\tag{16}
  {\rm pr}_{1}\langle f \rangle \times {\rm pr}_{2}\langle f \rangle \subset {\rm pr}_{1}\langle f \rangle \times \phi 
  \to {\rm pr}_{1}\langle f \rangle \times {\rm pr}_{2}\langle f \rangle \subset \phi
\]
が成り立つ.
そこで(14), (15), (16)から, 推論法則 \ref{dedmmp}によって
\[
  {\rm Func}(f; a; \phi) \to {\rm pr}_{1}\langle f \rangle \times {\rm pr}_{2}\langle f \rangle \subset \phi
\]
が成り立つことがわかる.
故にこれと(13)から, 推論法則 \ref{dedprewedge}によって
\[
\tag{17}
  {\rm Func}(f; a; \phi) 
  \to f \subset {\rm pr}_{1}\langle f \rangle \times {\rm pr}_{2}\langle f \rangle 
  \wedge {\rm pr}_{1}\langle f \rangle \times {\rm pr}_{2}\langle f \rangle \subset \phi
\]
が成り立つ.
また定理 \ref{sthmsubsettrans}より
\[
\tag{18}
  f \subset {\rm pr}_{1}\langle f \rangle \times {\rm pr}_{2}\langle f \rangle 
  \wedge {\rm pr}_{1}\langle f \rangle \times {\rm pr}_{2}\langle f \rangle \subset \phi 
  \to f \subset \phi
\]
が成り立つ.
また定理 \ref{sthmemptysubset=eq}と推論法則 \ref{dedequiv}により
\[
\tag{19}
  f \subset \phi \to f = \phi
\]
が成り立つ.
そこで(17), (18), (19)から, 推論法則 \ref{dedmmp}によって
\[
\tag{20}
  {\rm Func}(f; a; \phi) \to f = \phi
\]
が成り立つことがわかる.
また定理 \ref{sthmfuncbasis}より
\[
  {\rm Func}(f; a; \phi) \to {\rm pr}_{1}\langle f \rangle = a
\]
が成り立ち, Thm \ref{x=yty=x}より
\[
  {\rm pr}_{1}\langle f \rangle = a \to a = {\rm pr}_{1}\langle f \rangle
\]
が成り立つから, 推論法則 \ref{dedmmp}によって
\[
\tag{21}
  {\rm Func}(f; a; \phi) \to a = {\rm pr}_{1}\langle f \rangle
\]
が成り立つ.
また(20)と(5)から, 推論法則 \ref{dedmmp}によって
\[
\tag{22}
  {\rm Func}(f; a; \phi) \to {\rm pr}_{1}\langle f \rangle = \phi
\]
が成り立つ.
故に(21), (22)から, 推論法則 \ref{dedprewedge}によって
\[
\tag{23}
  {\rm Func}(f; a; \phi) \to a = {\rm pr}_{1}\langle f \rangle \wedge {\rm pr}_{1}\langle f \rangle = \phi
\]
が成り立つ.
またThm \ref{x=ywy=ztx=z}より
\[
\tag{24}
  a = {\rm pr}_{1}\langle f \rangle \wedge {\rm pr}_{1}\langle f \rangle = \phi \to a = \phi
\]
が成り立つ.
そこで(23), (24)から, 推論法則 \ref{dedmmp}によって
\[
  {\rm Func}(f; a; \phi) \to a = \phi
\]
が成り立つ.
故にこれと(20)から, 推論法則 \ref{dedprewedge}によって
\[
\tag{25}
  {\rm Func}(f; a; \phi) \to a = \phi \wedge f = \phi
\]
が成り立つ.
また2)の証明の中で示したように(12)が成り立つから, 
特にそこで$b$を$\phi$とした
\[
  f = \phi \to {\rm Func}(f; \phi; \phi)
\]
が成り立つ.
故に推論法則 \ref{dedaddw}により
\[
\tag{26}
  a = \phi \wedge f = \phi \to a = \phi \wedge {\rm Func}(f; \phi; \phi)
\]
が成り立つ.
また定理 \ref{sthmfunc=}より
\[
  a = \phi \to ({\rm Func}(f; a; \phi) \leftrightarrow {\rm Func}(f; \phi; \phi))
\]
が成り立つから, 推論法則 \ref{dedprewedge}により
\[
  a = \phi \to ({\rm Func}(f; \phi; \phi) \to {\rm Func}(f; a; \phi))
\]
が成り立ち, これから推論法則 \ref{dedtwch}により
\[
\tag{27}
  a = \phi \wedge {\rm Func}(f; \phi; \phi) \to {\rm Func}(f; a; \phi)
\]
が成り立つ.
そこで(26), (27)から, 推論法則 \ref{dedmmp}によって
\[
\tag{28}
  a = \phi \wedge f = \phi \to {\rm Func}(f; a; \phi)
\]
が成り立つ.
従って(25), (28)から, 推論法則 \ref{dedequiv}により
\[
  {\rm Func}(f; a; \phi) \leftrightarrow a = \phi \wedge f = \phi
\]
が成り立つ.
(${**}*$)が成り立つことは, これと
推論法則 \ref{dedwedge}, \ref{dedeqfund}によって明らかである.

\noindent
4)
Thm \ref{x=x}より$\phi = \phi$が成り立つから, (${**}*$)により$\phi$が$\phi$から$\phi$への函数であることがわかる.
故に定理 \ref{sthmfuncbasis}により, $\phi$は$\phi$における函数であり, 
また$\phi$は函数である.
\halmos




\mathstrut
\begin{thm}
\label{sthmproductfunc}%定理
$a$と$b$を集合とする.

1)
このとき$a \times \{b\}$は$a$から$\{b\}$への函数である.
故に$a \times \{b\}$は$a$における函数であり, 
また$a \times \{b\}$は函数である.

2)
また$f$を集合とするとき, 
\[
  {\rm Func}(f; a; \{b\}) \leftrightarrow f = a \times \{b\}
\]
が成り立つ.
そこでこのことから特に, 次の($*$)が成り立つ: 

($*$) ~~$f$が$a$から$\{b\}$への函数ならば, $f = a \times \{b\}$が成り立つ.
\end{thm}


\noindent{\bf 証明}
~1)
$x$と$y$を, 互いに異なり, 共に$a$及び$b$の中に自由変数として現れない, 定数でない文字とする.
このとき変数法則 \ref{valnset}, \ref{valproduct}によってわかるように, 
$x$と$y$は共に$a \times \{b\}$の中に自由変数として現れない.
そして定理 \ref{sthmpairinproduct}と推論法則 \ref{dedequiv}により
\[
  (x, y) \in a \times \{b\} \to x \in a \wedge y \in \{b\}
\]
が成り立つから, 推論法則 \ref{dedprewedge}により
\[
\tag{1}
  (x, y) \in a \times \{b\} \to y \in \{b\}
\]
が成り立つ.
また定理 \ref{sthmsingletonbasis}と推論法則 \ref{dedequiv}により
\[
\tag{2}
  y \in \{b\} \to y = b
\]
が成り立つ.
そこで(1), (2)から, 推論法則 \ref{dedmmp}によって
\[
  (x, y) \in a \times \{b\} \to y = b
\]
が成り立つ.
いま$y$は定数でなく, $b$の中に自由変数として現れないから, 
従って推論法則 \ref{ded!thmconst}により
\[
  !y((x, y) \in a \times \{b\})
\]
が成り立つ.
また$x$も定数でないので, これから推論法則 \ref{dedltthmquan}により
\[
\tag{3}
  \forall x(!y((x, y) \in a \times \{b\}))
\]
が成り立つ.
また定理 \ref{sthmproductgraph}より, $a \times \{b\}$はグラフである.
故にこのことと(3)から, 推論法則 \ref{dedwedge}により
\[
  {\rm Graph}(a \times \{b\}) \wedge \forall x(!y((x, y) \in a \times \{b\}))
\]
が成り立つ.
いま$x$と$y$は互いに異なり, 上述のように共に$a \times \{b\}$の中に自由変数として現れないから, 
定義によれば, 上記の記号列は
\[
\tag{4}
  {\rm Func}(a \times \{b\})
\]
と同じである.
故にこれが定理となる.
即ち, $a \times \{b\}$は函数である.
また定理 \ref{sthmsunotempty}より$\{b\}$は空でないから, 
定理 \ref{sthmproductprset}により
\[
  {\rm pr}_{1}\langle a \times \{b\} \rangle = a
\]
が成り立つ.
そこでこれと(4)から, 推論法則 \ref{dedwedge}により
\[
  {\rm Func}(a \times \{b\}) \wedge {\rm pr}_{1}\langle a \times \{b\} \rangle = a, 
\]
即ち
\[
\tag{5}
  {\rm Func}(a \times \{b\}; a)
\]
が成り立つ.
即ち, $a \times \{b\}$は$a$における函数である.
また定理 \ref{sthmproductprset}より
\[
  {\rm pr}_{2}\langle a \times \{b\} \rangle \subset \{b\}
\]
が成り立つ.
そこでこれと(5)から, 推論法則 \ref{dedwedge}により
\[
  {\rm Func}(a \times \{b\}; a) \wedge {\rm pr}_{2}\langle a \times \{b\} \rangle \subset \{b\}, 
\]
即ち
\[
\tag{6}
  {\rm Func}(a \times \{b\}; a; \{b\})
\]
が成り立つ.
即ち, $a \times \{b\}$は$a$から$\{b\}$への函数である.

\noindent
2)
定理 \ref{sthmatbfuncsubsetatimesb}より
\[
  {\rm Func}(f; a; \{b\}) \to f \subset a \times \{b\}
\]
が成り立つから, 推論法則 \ref{dedatawbtrue1}により
\[
\tag{7}
  {\rm Func}(f; a; \{b\}) \to {\rm Func}(f; a; \{b\}) \wedge f \subset a \times \{b\}
\]
が成り立つ.
また(6)が成り立つことから, 推論法則 \ref{dedatawbtrue2}により
\[
\tag{8}
  {\rm Func}(f; a; \{b\}) \to {\rm Func}(f; a; \{b\}) \wedge {\rm Func}(a \times \{b\}; a; \{b\})
\]
が成り立つ.
また定理 \ref{sthmfuncsubset=eq}より
\[
  {\rm Func}(f; a; \{b\}) \wedge {\rm Func}(a \times \{b\}; a; \{b\}) 
  \to (f \subset a \times \{b\} \leftrightarrow f = a \times \{b\})
\]
が成り立つから, 推論法則 \ref{dedprewedge}により
\[
\tag{9}
  {\rm Func}(f; a; \{b\}) \wedge {\rm Func}(a \times \{b\}; a; \{b\}) 
  \to (f \subset a \times \{b\} \to f = a \times \{b\})
\]
が成り立つ.
そこで(8), (9)から, 推論法則 \ref{dedmmp}によって
\[
  {\rm Func}(f; a; \{b\}) \to (f \subset a \times \{b\} \to f = a \times \{b\})
\]
が成り立ち, 故にこれから推論法則 \ref{dedtwch}により
\[
  {\rm Func}(f; a; \{b\}) \wedge f \subset a \times \{b\} \to f = a \times \{b\}
\]
が成り立つ.
そこでこれと(7)から, 推論法則 \ref{dedmmp}によって
\[
\tag{10}
  {\rm Func}(f; a; \{b\}) \to f = a \times \{b\}
\]
が成り立つ.
また定理 \ref{sthmfunc=}より
\[
  f = a \times \{b\} \to ({\rm Func}(f; a; \{b\}) \leftrightarrow {\rm Func}(a \times \{b\}; a; \{b\}))
\]
が成り立つから, 推論法則 \ref{dedprewedge}により
\[
  f = a \times \{b\} \to ({\rm Func}(a \times \{b\}; a; \{b\}) \to {\rm Func}(f; a; \{b\}))
\]
が成り立ち, これから推論法則 \ref{dedch}により
\[
  {\rm Func}(a \times \{b\}; a; \{b\}) \to (f = a \times \{b\} \to {\rm Func}(f; a; \{b\}))
\]
が成り立つ.
故にこれと(6)から, 推論法則 \ref{dedmp}によって
\[
\tag{11}
  f = a \times \{b\} \to {\rm Func}(f; a; \{b\})
\]
が成り立つ.
そこで(10), (11)から, 推論法則 \ref{dedequiv}により
\[
  {\rm Func}(f; a; \{b\}) \leftrightarrow f = a \times \{b\}
\]
が成り立つ.
($*$)が成り立つことは, これと推論法則 \ref{dedeqfund}によって明らかである.
\halmos




\mathstrut
\begin{thm}
\label{sthmvaluesetfunc}%定理
$a$, $b$, $c$, $f$を集合とする.

1)
このとき
\[
  {\rm Func}(f; a) \to f[c] = f[a \cap c], ~~
  {\rm Func}(f; a; b) \to f[c] = f[a \cap c]
\]
が成り立つ.
またこれらから, 次の($*$)が成り立つ: 

($*$) ~~$f$が$a$における函数ならば, $f[c] = f[a \cap c]$が成り立つ.
        また$f$が$a$から$b$への函数ならば, $f[c] = f[a \cap c]$が成り立つ.

2)
またこのとき
\[
  {\rm Func}(f; a; b) \to f[c] \subset b
\]
が成り立つ.
またこのことから, 次の($**$)が成り立つ: 

($**$) ~~$f$が$a$から$b$への函数ならば, $f[c] \subset b$が成り立つ.

3)
またこのとき
\begin{align*}
  {\rm Func}(f; a) \to (a \subset c \to f[c] = {\rm pr}_{2}\langle f \rangle)&, ~~
  {\rm Func}(f; a) \to f[a] = {\rm pr}_{2}\langle f \rangle, \\
  \mbox{} \\
  {\rm Func}(f; a; b) \to (a \subset c \to f[c] = {\rm pr}_{2}\langle f \rangle)&, ~~
  {\rm Func}(f; a; b) \to f[a] = {\rm pr}_{2}\langle f \rangle
\end{align*}
が成り立つ.
またこれらから, 次の(${**}*$), (${**}{**}$)が成り立つ: 

(${**}*$) ~~$f$が$a$における函数ならば, $a \subset c \to f[c] = {\rm pr}_{2}\langle f \rangle$が成り立つ.
            故にこのとき$a \subset c$が成り立つならば, $f[c] = {\rm pr}_{2}\langle f \rangle$が成り立つ.
            特にこのとき, $f[a] = {\rm pr}_{2}\langle f \rangle$が成り立つ.

(${**}{**}$) ~~$f$が$a$から$b$への函数ならば, $a \subset c \to f[c] = {\rm pr}_{2}\langle f \rangle$が成り立つ.
               故にこのとき$a \subset c$が成り立つならば, $f[c] = {\rm pr}_{2}\langle f \rangle$が成り立つ.
               特にこのとき, $f[a] = {\rm pr}_{2}\langle f \rangle$が成り立つ.
\end{thm}


\noindent{\bf 証明}
~1)
定理 \ref{sthmfuncbasis}より
\[
\tag{1}
  {\rm Func}(f; a) \to {\rm pr}_{1}\langle f \rangle = a
\]
が成り立つ.
また定理 \ref{sthmcap=}より
\[
\tag{2}
  {\rm pr}_{1}\langle f \rangle = a \to c \cap {\rm pr}_{1}\langle f \rangle = c \cap a
\]
が成り立つ.
また定理 \ref{sthmcapch}より$c \cap a = a \cap c$が成り立つから, 
推論法則 \ref{dedaddeq=}により
\[
  c \cap {\rm pr}_{1}\langle f \rangle = c \cap a \leftrightarrow c \cap {\rm pr}_{1}\langle f \rangle = a \cap c
\]
が成り立ち, これから推論法則 \ref{dedequiv}により
\[
\tag{3}
  c \cap {\rm pr}_{1}\langle f \rangle = c \cap a \to c \cap {\rm pr}_{1}\langle f \rangle = a \cap c
\]
が成り立つ.
また定理 \ref{sthmvalueset=}より
\[
\tag{4}
  c \cap {\rm pr}_{1}\langle f \rangle = a \cap c \to f[c \cap {\rm pr}_{1}\langle f \rangle] = f[a \cap c]
\]
が成り立つ.
また定理 \ref{sthmvaluesetcappr1set}より
$f[c] = f[c \cap {\rm pr}_{1}\langle f \rangle]$が成り立つから, 
推論法則 \ref{dedaddeq=}により
\[
  f[c] = f[a \cap c] \leftrightarrow f[c \cap {\rm pr}_{1}\langle f \rangle] = f[a \cap c]
\]
が成り立ち, これから推論法則 \ref{dedequiv}により
\[
\tag{5}
  f[c \cap {\rm pr}_{1}\langle f \rangle] = f[a \cap c] \to f[c] = f[a \cap c]
\]
が成り立つ.
そこで(1)---(5)から, 推論法則 \ref{dedmmp}によって
\[
\tag{6}
  {\rm Func}(f; a) \to f[c] = f[a \cap c]
\]
が成り立つことがわかる.
また定理 \ref{sthmfuncbasis}より
\[
  {\rm Func}(f; a; b) \to {\rm Func}(f; a)
\]
が成り立つから, これと(6)から, 推論法則 \ref{dedmmp}によって
\[
\tag{7}
  {\rm Func}(f; a; b) \to f[c] = f[a \cap c]
\]
が成り立つ.
($*$)が成り立つことは, (6), (7)と推論法則 \ref{dedmp}によって明らかである.

\noindent
2)
定理 \ref{sthmfuncbasis}より
\[
\tag{8}
  {\rm Func}(f; a; b) \to {\rm pr}_{2}\langle f \rangle \subset b
\]
が成り立つ.
また定理 \ref{sthmvaluesetsubsetpr2set}より
$f[c] \subset {\rm pr}_{2}\langle f \rangle$が成り立つから, 
推論法則 \ref{dedatawbtrue2}により
\[
\tag{9}
  {\rm pr}_{2}\langle f \rangle \subset b 
  \to f[c] \subset {\rm pr}_{2}\langle f \rangle \wedge {\rm pr}_{2}\langle f \rangle \subset b
\]
が成り立つ.
また定理 \ref{sthmsubsettrans}より
\[
\tag{10}
  f[c] \subset {\rm pr}_{2}\langle f \rangle \wedge {\rm pr}_{2}\langle f \rangle \subset b \to f[c] \subset b
\]
が成り立つ.
そこで(8), (9), (10)から, 推論法則 \ref{dedmmp}によって
\[
  {\rm Func}(f; a; b) \to f[c] \subset b
\]
が成り立つ.
($**$)が成り立つことは, これと推論法則 \ref{dedmp}によって明らかである.

\noindent
3)
(1)から, 推論法則 \ref{dedaddw}により
\[
\tag{11}
  {\rm Func}(f; a) \wedge a \subset c \to {\rm pr}_{1}\langle f \rangle = a \wedge a \subset c
\]
が成り立つ.
また定理 \ref{sthm=&subset}より
\[
\tag{12}
  {\rm pr}_{1}\langle f \rangle = a \wedge a \subset c \to {\rm pr}_{1}\langle f \rangle \subset c
\]
が成り立つ.
また定理 \ref{sthmvalueset=pr2set}より
\[
\tag{13}
  {\rm pr}_{1}\langle f \rangle \subset c \to f[c] = {\rm pr}_{2}\langle f \rangle
\]
が成り立つ.
そこで(11), (12), (13)から, 推論法則 \ref{dedmmp}によって
\[
  {\rm Func}(f; a) \wedge a \subset c \to f[c] = {\rm pr}_{2}\langle f \rangle
\]
が成り立つことがわかる.
故にこれから, 推論法則 \ref{dedtwch}により
\[
\tag{14}
  {\rm Func}(f; a) \to (a \subset c \to f[c] = {\rm pr}_{2}\langle f \rangle)
\]
が成り立つ.
この(14)において, 特に$c$を$a$に置き換えた
\[
  {\rm Func}(f; a) \to (a \subset a \to f[a] = {\rm pr}_{2}\langle f \rangle)
\]
も成り立つから, 推論法則 \ref{dedch}により
\[
\tag{15}
  a \subset a \to ({\rm Func}(f; a) \to f[a] = {\rm pr}_{2}\langle f \rangle)
\]
が成り立つ.
いま定理 \ref{sthmsubsetself}より$a \subset a$が成り立つから, 
従ってこれと(15)から, 推論法則 \ref{dedmp}により
\[
\tag{16}
  {\rm Func}(f; a) \to f[a] = {\rm pr}_{2}\langle f \rangle
\]
が成り立つ.
また定理 \ref{sthmfuncbasis}より
\[
  {\rm Func}(f; a; b) \to {\rm Func}(f; a)
\]
が成り立つから, これと(14), (16)から, それぞれ推論法則 \ref{dedmmp}によって
\begin{align*}
  \tag{17}
  {\rm Func}(f; a; b) &\to (a \subset c \to f[c] = {\rm pr}_{2}\langle f \rangle), \\
  \mbox{} \\
  \tag{18}
  {\rm Func}(f; a; b) &\to f[a] = {\rm pr}_{2}\langle f \rangle
\end{align*}
が成り立つ.
(${**}*$)が成り立つことは, (14), (16)と推論法則 \ref{dedmp}によって明らかである.
また(${**}{**}$)が成り立つことは, (17), (18)と推論法則 \ref{dedmp}によって明らかである.
\halmos




\mathstrut
\begin{thm}
\label{sthmvaluesetinvfunc}%定理
$a$, $b$, $c$, $f$を集合とする.

1)
このとき
\[
  {\rm Func}(f; a; b) \to f^{-1}[c] = f^{-1}[b \cap c]
\]
が成り立つ.
またこのことから, 次の($*$)が成り立つ:

($*$) ~~$f$が$a$から$b$への函数ならば, $f^{-1}[c] = f^{-1}[b \cap c]$が成り立つ.

2)
またこのとき
\[
  {\rm Func}(f; a) \to f^{-1}[c] \subset a, ~~
  {\rm Func}(f; a; b) \to f^{-1}[c] \subset a
\]
が成り立つ.
またこれらから, 次の($**$)が成り立つ: 

($**$) ~~$f$が$a$における函数ならば, $f^{-1}[c] \subset a$が成り立つ.
         また$f$が$a$から$b$への函数ならば, $f^{-1}[c] \subset a$が成り立つ.

3)
またこのとき
\[
  {\rm Func}(f; a; b) \to (b \subset c \to f^{-1}[c] = a), ~~
  {\rm Func}(f; a; b) \to f^{-1}[b] = a
\]
が成り立つ.
またこれらから, 次の(${**}*$)が成り立つ: 

(${**}*$) ~~$f$が$a$から$b$への函数ならば, $b \subset c \to f^{-1}[c] = a$が成り立つ.
            故にこのとき, $b \subset c$が成り立つならば, $f^{-1}[c] = a$が成り立つ.
            特にこのとき, $f^{-1}[b] = a$が成り立つ.
\end{thm}


\noindent{\bf 証明}
~1)
定理 \ref{sthmfuncbasis}より
\[
\tag{1}
  {\rm Func}(f; a; b) \to {\rm pr}_{2}\langle f \rangle \subset b
\]
が成り立つ.
また定理 \ref{sthmprsetinv}より
${\rm pr}_{1}\langle f^{-1} \rangle = {\rm pr}_{2}\langle f \rangle$が成り立つから, 
定理 \ref{sthm=tsubseteq}により
\[
  {\rm pr}_{1}\langle f^{-1} \rangle \subset b \leftrightarrow {\rm pr}_{2}\langle f \rangle \subset b
\]
が成り立ち, これから推論法則 \ref{dedequiv}により
\[
\tag{2}
  {\rm pr}_{2}\langle f \rangle \subset b \to {\rm pr}_{1}\langle f^{-1} \rangle \subset b
\]
が成り立つ.
また定理 \ref{sthmcapsubset}より
\[
\tag{3}
  {\rm pr}_{1}\langle f^{-1} \rangle \subset b \to c \cap {\rm pr}_{1}\langle f^{-1} \rangle \subset c \cap b
\]
が成り立つ.
また定理 \ref{sthmcapch}より$c \cap b = b \cap c$が成り立つから, 
定理 \ref{sthm=tsubseteq}により
\[
  c \cap {\rm pr}_{1}\langle f^{-1} \rangle \subset c \cap b \leftrightarrow c \cap {\rm pr}_{1}\langle f^{-1} \rangle \subset b \cap c
\]
が成り立ち, これから推論法則 \ref{dedequiv}により
\[
\tag{4}
  c \cap {\rm pr}_{1}\langle f^{-1} \rangle \subset c \cap b \to c \cap {\rm pr}_{1}\langle f^{-1} \rangle \subset b \cap c
\]
が成り立つ.
また定理 \ref{sthmvaluesetsubset}より
\[
\tag{5}
  c \cap {\rm pr}_{1}\langle f^{-1} \rangle \subset b \cap c 
  \to f^{-1}[c \cap {\rm pr}_{1}\langle f^{-1} \rangle] \subset f^{-1}[b \cap c]
\]
が成り立つ.
また定理 \ref{sthmvaluesetcappr1set}より
$f^{-1}[c] = f^{-1}[c \cap {\rm pr}_{1}\langle f^{-1} \rangle]$が成り立つから, 
定理 \ref{sthm=tsubseteq}により
\[
  f^{-1}[c] \subset f^{-1}[b \cap c] 
  \leftrightarrow f^{-1}[c \cap {\rm pr}_{1}\langle f^{-1} \rangle] \subset f^{-1}[b \cap c]
\]
が成り立ち, これから推論法則 \ref{dedequiv}により
\[
\tag{6}
  f^{-1}[c \cap {\rm pr}_{1}\langle f^{-1} \rangle] \subset f^{-1}[b \cap c] \to f^{-1}[c] \subset f^{-1}[b \cap c]
\]
が成り立つ.
また定理 \ref{sthmcap}より$b \cap c \subset c$が成り立つから, 
定理 \ref{sthmvaluesetsubset}により$f^{-1}[b \cap c] \subset f^{-1}[c]$が成り立ち, 
これから推論法則 \ref{dedatawbtrue2}により
\[
\tag{7}
  f^{-1}[c] \subset f^{-1}[b \cap c] \to f^{-1}[c] \subset f^{-1}[b \cap c] \wedge f^{-1}[b \cap c] \subset f^{-1}[c]
\]
が成り立つ.
また定理 \ref{sthmaxiom1}と推論法則 \ref{dedequiv}により
\[
\tag{8}
  f^{-1}[c] \subset f^{-1}[b \cap c] \wedge f^{-1}[b \cap c] \subset f^{-1}[c] \to f^{-1}[c] = f^{-1}[b \cap c]
\]
が成り立つ.
以上の(1)---(8)から, 推論法則 \ref{dedmmp}によって
\[
  {\rm Func}(f; a; b) \to f^{-1}[c] = f^{-1}[b \cap c]
\]
が成り立つことがわかる.
($*$)が成り立つことは, これと推論法則 \ref{dedmp}によって明らかである.

\noindent
2)
定理 \ref{sthmfuncbasis}より
\[
\tag{9}
  {\rm Func}(f; a) \to {\rm pr}_{1}\langle f \rangle = a
\]
が成り立つ.
また定理 \ref{sthmprsetinv}より
${\rm pr}_{2}\langle f^{-1} \rangle = {\rm pr}_{1}\langle f \rangle$が成り立つから, 
推論法則 \ref{dedatawbtrue2}により
\[
\tag{10}
  {\rm pr}_{1}\langle f \rangle = a 
  \to {\rm pr}_{2}\langle f^{-1} \rangle = {\rm pr}_{1}\langle f \rangle \wedge {\rm pr}_{1}\langle f \rangle = a
\]
が成り立つ.
またThm \ref{x=ywy=ztx=z}より
\[
\tag{11}
  {\rm pr}_{2}\langle f^{-1} \rangle = {\rm pr}_{1}\langle f \rangle \wedge {\rm pr}_{1}\langle f \rangle = a 
  \to {\rm pr}_{2}\langle f^{-1} \rangle = a
\]
が成り立つ.
また定理 \ref{sthmvaluesetsubsetpr2set}より
$f^{-1}[c] \subset {\rm pr}_{2}\langle f^{-1} \rangle$が成り立つから, 
推論法則 \ref{dedatawbtrue2}により
\[
\tag{12}
  {\rm pr}_{2}\langle f^{-1} \rangle = a 
  \to {\rm pr}_{2}\langle f^{-1} \rangle = a \wedge f^{-1}[c] \subset {\rm pr}_{2}\langle f^{-1} \rangle
\]
が成り立つ.
また定理 \ref{sthm=&subset}より
\[
\tag{13}
  {\rm pr}_{2}\langle f^{-1} \rangle = a \wedge f^{-1}[c] \subset {\rm pr}_{2}\langle f^{-1} \rangle 
  \to f^{-1}[c] \subset a
\]
が成り立つ.
そこで(9)---(13)から, 推論法則 \ref{dedmmp}によって
\[
\tag{14}
  {\rm Func}(f; a) \to f^{-1}[c] \subset a
\]
が成り立つことがわかる.
また定理 \ref{sthmfuncbasis}より
\[
  {\rm Func}(f; a; b) \to {\rm Func}(f; a)
\]
が成り立つから, これと(14)から, 推論法則 \ref{dedmmp}によって
\[
\tag{15}
  {\rm Func}(f; a; b) \to f^{-1}[c] \subset a
\]
が成り立つ.
($**$)が成り立つことは, (14), (15)と推論法則 \ref{dedmp}によって明らかである.

\noindent
3)
(1), (2)から, 推論法則 \ref{dedmmp}によって
\[
  {\rm Func}(f; a; b) \to {\rm pr}_{1}\langle f^{-1} \rangle \subset b
\]
が成り立ち, 故に推論法則 \ref{dedaddw}によって
\[
\tag{16}
  {\rm Func}(f; a; b) \wedge b \subset c \to {\rm pr}_{1}\langle f^{-1} \rangle \subset b \wedge b \subset c
\]
が成り立つ.
また定理 \ref{sthmsubsettrans}より
\[
\tag{17}
  {\rm pr}_{1}\langle f^{-1} \rangle \subset b \wedge b \subset c \to {\rm pr}_{1}\langle f^{-1} \rangle \subset c
\]
が成り立つ.
また定理 \ref{sthmvalueset=pr2set}より
\[
\tag{18}
  {\rm pr}_{1}\langle f^{-1} \rangle \subset c \to f^{-1}[c] = {\rm pr}_{2}\langle f^{-1} \rangle
\]
が成り立つ.
また定理 \ref{sthmprsetinv}より
${\rm pr}_{2}\langle f^{-1} \rangle = {\rm pr}_{1}\langle f \rangle$が成り立つから, 
推論法則 \ref{dedaddeq=}により
\[
  f^{-1}[c] = {\rm pr}_{2}\langle f^{-1} \rangle \leftrightarrow f^{-1}[c] = {\rm pr}_{1}\langle f \rangle
\]
が成り立ち, これから推論法則 \ref{dedequiv}により
\[
\tag{19}
  f^{-1}[c] = {\rm pr}_{2}\langle f^{-1} \rangle \to f^{-1}[c] = {\rm pr}_{1}\langle f \rangle
\]
が成り立つ.
そこで(16)---(19)から, 推論法則 \ref{dedmmp}によって
\[
\tag{20}
  {\rm Func}(f; a; b) \wedge b \subset c \to f^{-1}[c] = {\rm pr}_{1}\langle f \rangle
\]
が成り立つことがわかる.
またThm \ref{awbta}より
\[
  {\rm Func}(f; a; b) \wedge b \subset c \to {\rm Func}(f; a; b)
\]
が成り立ち, 定理 \ref{sthmfuncbasis}より
\[
  {\rm Func}(f; a; b) \to {\rm pr}_{1}\langle f \rangle = a
\]
が成り立つから, これらから, 推論法則 \ref{dedmmp}によって
\[
\tag{21}
  {\rm Func}(f; a; b) \wedge b \subset c \to {\rm pr}_{1}\langle f \rangle = a
\]
が成り立つ.
故に(20), (21)から, 推論法則 \ref{dedprewedge}によって
\[
\tag{22}
  {\rm Func}(f; a; b) \wedge b \subset c 
  \to f^{-1}[c] = {\rm pr}_{1}\langle f \rangle \wedge {\rm pr}_{1}\langle f \rangle = a
\]
が成り立つ.
またThm \ref{x=ywy=ztx=z}より
\[
\tag{23}
  f^{-1}[c] = {\rm pr}_{1}\langle f \rangle \wedge {\rm pr}_{1}\langle f \rangle = a 
  \to f^{-1}[c] = a
\]
が成り立つ.
そこで(22), (23)から, 推論法則 \ref{dedmmp}によって
\[
  {\rm Func}(f; a; b) \wedge b \subset c \to f^{-1}[c] = a
\]
が成り立ち, これから推論法則 \ref{dedtwch}によって
\[
\tag{24}
  {\rm Func}(f; a; b) \to (b \subset c \to f^{-1}[c] = a)
\]
が成り立つ.
この(24)において, 特に$c$を$b$に置き換えた
\[
  {\rm Func}(f; a; b) \to (b \subset b \to f^{-1}[b] = a)
\]
も成り立つから, 推論法則 \ref{dedch}により
\[
\tag{25}
  b \subset b \to ({\rm Func}(f; a; b) \to f^{-1}[b] = a)
\]
が成り立つ.
いま定理 \ref{sthmsubsetself}より$b \subset b$が成り立つから, 
従ってこれと(25)から, 推論法則 \ref{dedmp}によって
\[
\tag{26}
  {\rm Func}(f; a; b) \to f^{-1}[b] = a
\]
が成り立つ.
(${**}*$)が成り立つことは, (24), (26)と推論法則 \ref{dedmp}によって明らかである.
\halmos




\mathstrut
\begin{thm}
\label{sthmvaluesetinvfuncsubset1}%定理
\mbox{}

1)
$a$, $b$, $f$を集合とするとき, 
\begin{align*}
  {\rm Func}(f; a) &\to (b \subset a \leftrightarrow b \subset f^{-1}[f[b]]), \\
  \mbox{} \\
  {\rm Func}(f; a) &\to a \cap b \subset f^{-1}[f[b]]
\end{align*}
が成り立つ.
またこれらから, 次の($*$)が成り立つ: 

($*$) ~~$f$が$a$における函数であるとき, 
        \[
          b \subset a \leftrightarrow b \subset f^{-1}[f[b]], ~~
          a \cap b \subset f^{-1}[f[b]]
        \]
        が共に成り立つ.
        故にこのとき, $b \subset a$が成り立つならば$b \subset f^{-1}[f[b]]$が成り立ち, 
        逆に$b \subset f^{-1}[f[b]]$が成り立つならば$b \subset a$が成り立つ.

2)
$a$, $b$, $c$, $f$を集合とするとき, 
\begin{align*}
  {\rm Func}(f; a; b) &\to (c \subset a \leftrightarrow c \subset f^{-1}[f[c]]), \\
  \mbox{} \\
  {\rm Func}(f; a; b) &\to a \cap c \subset f^{-1}[f[c]]
\end{align*}
が成り立つ.
またこれらから, 次の($**$)が成り立つ: 

($**$) ~~$f$が$a$から$b$への函数であるとき, 
         \[
           c \subset a \leftrightarrow c \subset f^{-1}[f[c]], ~~
           a \cap c \subset f^{-1}[f[c]]
         \]
         が共に成り立つ.
         故にこのとき, $c \subset a$が成り立つならば$c \subset f^{-1}[f[c]]$が成り立ち, 
         逆に$c \subset f^{-1}[f[c]]$が成り立つならば$c \subset a$が成り立つ.
\end{thm}


\noindent{\bf 証明}
~1)
まず前者から示す.
定理 \ref{sthmfuncbasis}より
\[
\tag{1}
  {\rm Func}(f; a) \to {\rm pr}_{1}\langle f \rangle = a
\]
が成り立つ.
また定理 \ref{sthm=tsubseteq}より
\[
\tag{2}
  {\rm pr}_{1}\langle f \rangle = a \to (b \subset {\rm pr}_{1}\langle f \rangle \leftrightarrow b \subset a)
\]
が成り立つ.
またThm \ref{1alb1t1bla1}より
\[
\tag{3}
  (b \subset {\rm pr}_{1}\langle f \rangle \leftrightarrow b \subset a) 
  \to (b \subset a \leftrightarrow b \subset {\rm pr}_{1}\langle f \rangle)
\]
が成り立つ.
また定理 \ref{sthmvaluesetinv}より
\[
  b \subset {\rm pr}_{1}\langle f \rangle \leftrightarrow b \subset f^{-1}[f[b]]
\]
が成り立つから, 推論法則 \ref{dedatawbtrue2}により
\[
\tag{4}
  (b \subset a \leftrightarrow b \subset {\rm pr}_{1}\langle f \rangle) 
  \to (b \subset a \leftrightarrow b \subset {\rm pr}_{1}\langle f \rangle) 
  \wedge (b \subset {\rm pr}_{1}\langle f \rangle \leftrightarrow b \subset f^{-1}[f[b]])
\]
が成り立つ.
またThm \ref{1alb1w1blc1t1alc1}より
\[
\tag{5}
  (b \subset a \leftrightarrow b \subset {\rm pr}_{1}\langle f \rangle) 
  \wedge (b \subset {\rm pr}_{1}\langle f \rangle \leftrightarrow b \subset f^{-1}[f[b]]) 
  \to (b \subset a \leftrightarrow b \subset f^{-1}[f[b]])
\]
が成り立つ.
そこで(1)---(5)から, 推論法則 \ref{dedmmp}によって
\[
\tag{6}
  {\rm Func}(f; a) \to (b \subset a \leftrightarrow b \subset f^{-1}[f[b]])
\]
が成り立つことがわかる.

次に後者を示す.
いま示したように(6)が成り立つから, そこで$b$を$a \cap b$に置き換えた
\[
  {\rm Func}(f; a) \to (a \cap b \subset a \leftrightarrow a \cap b \subset f^{-1}[f[a \cap b]])
\]
が成り立つ.
故に推論法則 \ref{dedprewedge}により
\[
  {\rm Func}(f; a) \to (a \cap b \subset a \to a \cap b \subset f^{-1}[f[a \cap b]])
\]
が成り立ち, これから推論法則 \ref{dedch}により
\[
\tag{7}
  a \cap b \subset a \to ({\rm Func}(f; a) \to a \cap b \subset f^{-1}[f[a \cap b]])
\]
が成り立つ.
いま定理 \ref{sthmcap}より$a \cap b \subset a$が成り立つから, 
従ってこれと(7)から, 推論法則 \ref{dedmp}により
\[
\tag{8}
  {\rm Func}(f; a) \to a \cap b \subset f^{-1}[f[a \cap b]]
\]
が成り立つ.
また同じく定理 \ref{sthmcap}より$a \cap b \subset b$が成り立つから, 
定理 \ref{sthmvaluesetsubset}により$f[a \cap b] \subset f[b]$が成り立ち, 
これから再び定理 \ref{sthmvaluesetsubset}により
\[
  f^{-1}[f[a \cap b]] \subset f^{-1}[f[b]]
\]
が成り立つ.
故に推論法則 \ref{dedatawbtrue2}により
\[
\tag{9}
  a \cap b \subset f^{-1}[f[a \cap b]] \to a \cap b \subset f^{-1}[f[a \cap b]] \wedge f^{-1}[f[a \cap b]] \subset f^{-1}[f[b]]
\]
が成り立つ.
また定理 \ref{sthmsubsettrans}より
\[
\tag{10}
  a \cap b \subset f^{-1}[f[a \cap b]] \wedge f^{-1}[f[a \cap b]] \subset f^{-1}[f[b]] \to a \cap b \subset f^{-1}[f[b]]
\]
が成り立つ.
そこで(8), (9), (10)から, 推論法則 \ref{dedmmp}によって
\[
\tag{11}
  {\rm Func}(f; a) \to a \cap b \subset f^{-1}[f[b]]
\]
が成り立つことがわかる.
($*$)が成り立つことは, (6), (11)が成り立つことと推論法則 \ref{dedmp}, \ref{dedeqfund}によって明らかである.

\noindent
2)
定理 \ref{sthmfuncbasis}より
\[
\tag{12}
  {\rm Func}(f; a; b) \to {\rm Func}(f; a)
\]
が成り立ち, 1)より
\begin{align*}
  \tag{13}
  {\rm Func}(f; a) &\to (c \subset a \leftrightarrow c \subset f^{-1}[f[c]]), \\
  \mbox{} \\
  \tag{14}
  {\rm Func}(f; a) &\to a \cap c \subset f^{-1}[f[c]]
\end{align*}
が共に成り立つから, (12)と(13), (12)と(14)から, それぞれ推論法則 \ref{dedmmp}によって
\begin{align*}
  {\rm Func}(f; a; b) &\to (c \subset a \leftrightarrow c \subset f^{-1}[f[c]]), \\
  \mbox{} \\
  {\rm Func}(f; a; b) &\to a \cap c \subset f^{-1}[f[c]]
\end{align*}
が成り立つ.
($**$)が成り立つことは, これらと推論法則 \ref{dedmp}, \ref{dedeqfund}によって明らかである.
\halmos




\mathstrut
\begin{thm}
\label{sthmvaluesetinvfuncsubset2}%定理
\mbox{}

1)
$c$と$f$を集合とするとき, 
\[
  {\rm Func}(f) \to f[f^{-1}[c]] \subset c
\]
が成り立つ.
またこのことから, 次の($*$)が成り立つ: 

($*$) ~~$f$が函数ならば, $f[f^{-1}[c]] \subset c$が成り立つ.

2)
$a$, $c$, $f$を集合とするとき, 
\[
  {\rm Func}(f; a) \to f[f^{-1}[c]] \subset c
\]
が成り立つ.
またこのことから, 次の($**$)が成り立つ: 

($**$) ~~$f$が$a$における函数ならば, $f[f^{-1}[c]] \subset c$が成り立つ.

3)
$a$, $b$, $c$, $f$を集合とするとき, 
\[
  {\rm Func}(f; a; b) \to f[f^{-1}[c]] \subset c
\]
が成り立つ.
またこのことから, 次の(${**}*$)が成り立つ: 

(${**}*$) ~~$f$が$a$から$b$への函数ならば, $f[f^{-1}[c]] \subset c$が成り立つ.
\end{thm}


\noindent{\bf 証明}
~1)
$x$と$y$を, 互いに異なり, 共に$c$及び$f$の中に自由変数として現れない, 
定数でない文字とする.
このとき変数法則 \ref{valvalueset}, \ref{valinv}により, $x$は$f^{-1}[c]$の中にも
自由変数として現れないから, 定理 \ref{sthmvaluesetelement}より
\[
\tag{1}
  y \in f[f^{-1}[c]] \leftrightarrow \exists x(x \in f^{-1}[c] \wedge (x, y) \in f)
\]
が成り立つ.
またいま$z$を$x$とも$y$とも異なり, $c$及び$f$の中に自由変数として現れない, 
定数でない文字とすれば, 変数法則 \ref{valinv}より$z$は$f^{-1}$の中にも自由変数として現れないから, 
同じく定理 \ref{sthmvaluesetelement}より
\[
\tag{2}
  x \in f^{-1}[c] \leftrightarrow \exists z(z \in c \wedge (z, x) \in f^{-1})
\]
が成り立つ.
また定理 \ref{sthmpairininv}より
\[
  (z, x) \in f^{-1} \leftrightarrow (x, z) \in f
\]
が成り立つから, 推論法則 \ref{dedaddeqw}により
\[
  z \in c \wedge (z, x) \in f^{-1} \leftrightarrow z \in c \wedge (x, z) \in f
\]
が成り立ち, これと$z$が定数でないことから, 推論法則 \ref{dedalleqquansepconst}により
\[
\tag{3}
  \exists z(z \in c \wedge (z, x) \in f^{-1}) \leftrightarrow \exists z(z \in c \wedge (x, z) \in f)
\]
が成り立つ.
そこで(2), (3)から, 推論法則 \ref{dedeqtrans}によって
\[
  x \in f^{-1}[c] \leftrightarrow \exists z(z \in c \wedge (x, z) \in f)
\]
が成り立ち, これから推論法則 \ref{dedaddeqw}により
\[
\tag{4}
  x \in f^{-1}[c] \wedge (x, y) \in f \leftrightarrow \exists z(z \in c \wedge (x, z) \in f) \wedge (x, y) \in f
\]
が成り立つ.
また$z$が$x$とも$y$とも異なり, $f$の中に自由変数として現れないことから, 
変数法則 \ref{valfund}, \ref{valpair}により$z$は$(x, y) \in f$の中に自由変数として現れないので, 
Thm \ref{thmexwrfree}と推論法則 \ref{dedeqch}により
\[
\tag{5}
  \exists z(z \in c \wedge (x, z) \in f) \wedge (x, y) \in f \leftrightarrow \exists z((z \in c \wedge (x, z) \in f) \wedge (x, y) \in f)
\]
が成り立つ.
またThm \ref{1awb1wclaw1bwc1}より
\[
  (z \in c \wedge (x, z) \in f) \wedge (x, y) \in f \leftrightarrow z \in c \wedge ((x, z) \in f \wedge (x, y) \in f)
\]
が成り立ち, これと$z$が定数でないことから, 推論法則 \ref{dedalleqquansepconst}により
\[
\tag{6}
  \exists z((z \in c \wedge (x, z) \in f) \wedge (x, y) \in f) \leftrightarrow \exists z(z \in c \wedge ((x, z) \in f \wedge (x, y) \in f))
\]
が成り立つ.
そこで(4), (5), (6)から, 推論法則 \ref{dedeqtrans}によって
\[
  x \in f^{-1}[c] \wedge (x, y) \in f \leftrightarrow \exists z(z \in c \wedge ((x, z) \in f \wedge (x, y) \in f))
\]
が成り立つことがわかる.
いま$x$は定数でないので, これから推論法則 \ref{dedalleqquansepconst}により
\[
\tag{7}
  \exists x(x \in f^{-1}[c] \wedge (x, y) \in f) 
  \leftrightarrow \exists x(\exists z(z \in c \wedge ((x, z) \in f \wedge (x, y) \in f)))
\]
が成り立つ.
またThm \ref{thmexch}より
\[
\tag{8}
  \exists x(\exists z(z \in c \wedge ((x, z) \in f \wedge (x, y) \in f))) 
  \leftrightarrow \exists z(\exists x(z \in c \wedge ((x, z) \in f \wedge (x, y) \in f)))
\]
が成り立つ.
また$x$が$z$と異なり, $c$の中に自由変数として現れないことから, 
変数法則 \ref{valfund}により$x$は$z \in c$の中に自由変数として現れないので, 
Thm \ref{thmexwrfree}より
\[
  \exists x(z \in c \wedge ((x, z) \in f \wedge (x, y) \in f)) \leftrightarrow z \in c \wedge \exists x((x, z) \in f \wedge (x, y) \in f)
\]
が成り立つ.
故にこれと$z$が定数でないことから, 推論法則 \ref{dedalleqquansepconst}により
\[
\tag{9}
  \exists z(\exists x(z \in c \wedge ((x, z) \in f \wedge (x, y) \in f))) 
  \leftrightarrow \exists z(z \in c \wedge \exists x((x, z) \in f \wedge (x, y) \in f))
\]
が成り立つ.
そこで(1), (7), (8), (9)から, 推論法則 \ref{dedeqtrans}によって
\[
  y \in f[f^{-1}[c]] \leftrightarrow \exists z(z \in c \wedge \exists x((x, z) \in f \wedge (x, y) \in f))
\]
が成り立つことがわかる.
故に推論法則 \ref{dedequiv}により
\[
  y \in f[f^{-1}[c]] \to \exists z(z \in c \wedge \exists x((x, z) \in f \wedge (x, y) \in f))
\]
が成り立ち, これから推論法則 \ref{dedaddw}により
\[
\tag{10}
  {\rm Func}(f) \wedge y \in f[f^{-1}[c]] \to {\rm Func}(f) \wedge \exists z(z \in c \wedge \exists x((x, z) \in f \wedge (x, y) \in f))
\]
が成り立つ.
また$z$が$f$の中に自由変数として現れないことから, 変数法則 \ref{valfunc}により
$z$は${\rm Func}(f)$の中に自由変数として現れないので, 
Thm \ref{thmexwrfree}と推論法則 \ref{dedequiv}により
\begin{multline*}
\tag{11}
  {\rm Func}(f) \wedge \exists z(z \in c \wedge \exists x((x, z) \in f \wedge (x, y) \in f)) \\
  \to \exists z({\rm Func}(f) \wedge (z \in c \wedge \exists x((x, z) \in f \wedge (x, y) \in f)))
\end{multline*}
が成り立つ.
また定理 \ref{sthmpairinfunc}より
\[
  {\rm Func}(f) \to ((x, z) \in f \wedge (x, y) \in f \to z = y)
\]
が成り立つ.
いま$x$は$f$の中に自由変数として現れないから, 変数法則 \ref{valfunc}により$x$は${\rm Func}(f)$の中に
自由変数として現れない.
また$x$は定数でない.
そこでこのことと上記の定理から, 推論法則 \ref{dedalltquansepfreeconst}により
\[
\tag{12}
  {\rm Func}(f) \to \forall x((x, z) \in f \wedge (x, y) \in f \to z = y)
\]
が成り立つ.
また$x$が$y$とも$z$とも異なることから, $x$は$z = y$の中に自由変数として現れないので, 
Thm \ref{thmalltexsepsfree}と推論法則 \ref{dedequiv}により
\[
\tag{13}
  \forall x((x, z) \in f \wedge (x, y) \in f \to z = y) 
  \to (\exists x((x, z) \in f \wedge (x, y) \in f) \to z = y)
\]
が成り立つ.
そこで(12), (13)から, 推論法則 \ref{dedmmp}によって
\[
  {\rm Func}(f) \to (\exists x((x, z) \in f \wedge (x, y) \in f) \to z = y)
\]
が成り立つ.
故に推論法則 \ref{dedtwch}により
\[
  {\rm Func}(f) \wedge \exists x((x, z) \in f \wedge (x, y) \in f) \to z = y
\]
が成り立ち, これから推論法則 \ref{dedaddw}により
\[
\tag{14}
  ({\rm Func}(f) \wedge \exists x((x, z) \in f \wedge (x, y) \in f)) \wedge z \in c \to z = y \wedge z \in c
\]
が成り立つ.
またThm \ref{awbtbwa}より
\[
  z \in c \wedge \exists x((x, z) \in f \wedge (x, y) \in f) \to \exists x((x, z) \in f \wedge (x, y) \in f) \wedge z \in c
\]
が成り立つから, 推論法則 \ref{dedaddw}により
\[
\tag{15}
  {\rm Func}(f) \wedge (z \in c \wedge \exists x((x, z) \in f \wedge (x, y) \in f)) 
  \to {\rm Func}(f) \wedge (\exists x((x, z) \in f \wedge (x, y) \in f) \wedge z \in c)
\]
が成り立つ.
またThm \ref{aw1bwc1t1awb1wc}より
\[
\tag{16}
  {\rm Func}(f) \wedge (\exists x((x, z) \in f \wedge (x, y) \in f) \wedge z \in c) 
  \to ({\rm Func}(f) \wedge \exists x((x, z) \in f \wedge (x, y) \in f)) \wedge z \in c
\]
が成り立つ.
また定理 \ref{sthm=&in}より
\[
\tag{17}
  z = y \wedge z \in c \to y \in c
\]
が成り立つ.
そこで(15), (16), (14), (17)にこの順で推論法則 \ref{dedmmp}を適用していき, 
\[
  {\rm Func}(f) \wedge (z \in c \wedge \exists x((x, z) \in f \wedge (x, y) \in f)) \to y \in c
\]
が成り立つことがわかる.
いま$z$は$y$と異なり, $c$の中に自由変数として現れないから, 
変数法則 \ref{valfund}により, $z$は$y \in c$の中に自由変数として現れない.
また$z$は定数でない.
そこでこのことと上記の定理から, 推論法則 \ref{dedalltquansepfreeconst}により
\[
\tag{18}
  \exists z({\rm Func}(f) \wedge (z \in c \wedge \exists x((x, z) \in f \wedge (x, y) \in f))) \to y \in c
\]
が成り立つ.
そこで(10), (11), (18)から, 推論法則 \ref{dedmmp}によって
\[
  {\rm Func}(f) \wedge y \in f[f^{-1}[c]] \to y \in c
\]
が成り立つことがわかる.
故に推論法則 \ref{dedtwch}により
\[
  {\rm Func}(f) \to (y \in f[f^{-1}[c]] \to y \in c)
\]
が成り立つ.
いま$y$は$f$の中に自由変数として現れないから, 変数法則 \ref{valfunc}により, 
$y$は${\rm Func}(f)$の中に自由変数として現れない.
また$y$は定数でない.
そこでこのことと上記の定理から, 推論法則 \ref{dedalltquansepfreeconst}により
\[
  {\rm Func}(f) \to \forall y(y \in f[f^{-1}[c]] \to y \in c)
\]
が成り立つ.
ここで$y$が$c$及び$f$の中に自由変数として現れないことから, 
変数法則 \ref{valvalueset}, \ref{valinv}により, $y$が$f[f^{-1}[c]]$の中にも自由変数として現れないことがわかるから, 
定義によれば上記の記号列は
\[
\tag{19}
  {\rm Func}(f) \to f[f^{-1}[c]] \subset c
\]
と同じである.
故にこれが定理となる.
($*$)が成り立つことは, これと推論法則 \ref{dedmp}によって明らかである.

\noindent
2)
定理 \ref{sthmfuncbasis}より${\rm Func}(f; a) \to {\rm Func}(f)$が成り立つから, 
これと(19)から, 推論法則 \ref{dedmmp}によって
\[
  {\rm Func}(f; a) \to f[f^{-1}[c]] \subset c
\]
が成り立つ.
($**$)が成り立つことは, これと推論法則 \ref{dedmp}によって明らかである.

\noindent
3)
定理 \ref{sthmfuncbasis}より${\rm Func}(f; a; b) \to {\rm Func}(f)$が成り立つから, 
これと(19)から, 推論法則 \ref{dedmmp}によって
\[
  {\rm Func}(f; a; b) \to f[f^{-1}[c]] \subset c
\]
が成り立つ.
(${**}*$)が成り立つことは, これと推論法則 \ref{dedmp}によって明らかである.
\halmos




\mathstrut
\begin{thm}
\label{sthmvaluesetinvfunc=}%定理
\mbox{}

1)
$c$と$f$を集合とするとき, 
\begin{align*}
  {\rm Func}(f) &\to (c \subset {\rm pr}_{2}\langle f \rangle \leftrightarrow f[f^{-1}[c]] = c), \\
  \mbox{} \\
  {\rm Func}(f) &\to f[f^{-1}[c]] = {\rm pr}_{2}\langle f \rangle \cap c
\end{align*}
が成り立つ.
またこれらから, 次の($*$)が成り立つ: 

($*$) ~~$f$が函数ならば, 
        \[
          c \subset {\rm pr}_{2}\langle f \rangle \leftrightarrow f[f^{-1}[c]] = c, ~~
          f[f^{-1}[c]] = {\rm pr}_{2}\langle f \rangle \cap c
        \]
        が共に成り立つ.
        故にこのとき, $c \subset {\rm pr}_{2}\langle f \rangle$が成り立つならば$f[f^{-1}[c]] = c$が成り立ち, 
        逆に$f[f^{-1}[c]] = c$が成り立つならば$c \subset {\rm pr}_{2}\langle f \rangle$が成り立つ.

2)
$a$, $c$, $f$を集合とするとき, 
\begin{align*}
  {\rm Func}(f; a) &\to (c \subset {\rm pr}_{2}\langle f \rangle \leftrightarrow f[f^{-1}[c]] = c), \\
  \mbox{} \\
  {\rm Func}(f; a) &\to f[f^{-1}[c]] = {\rm pr}_{2}\langle f \rangle \cap c
\end{align*}
が成り立つ.
またこれらから, 次の($**$)が成り立つ: 

($**$) ~~$f$が$a$における函数ならば, 
         \[
           c \subset {\rm pr}_{2}\langle f \rangle \leftrightarrow f[f^{-1}[c]] = c, ~~
           f[f^{-1}[c]] = {\rm pr}_{2}\langle f \rangle \cap c
         \]
         が共に成り立つ.
         故にこのとき, $c \subset {\rm pr}_{2}\langle f \rangle$が成り立つならば$f[f^{-1}[c]] = c$が成り立ち, 
         逆に$f[f^{-1}[c]] = c$が成り立つならば$c \subset {\rm pr}_{2}\langle f \rangle$が成り立つ.

3)
$a$, $b$, $c$, $f$を集合とするとき, 
\begin{align*}
  {\rm Func}(f; a; b) &\to (c \subset {\rm pr}_{2}\langle f \rangle \leftrightarrow f[f^{-1}[c]] = c), \\
  \mbox{} \\
  {\rm Func}(f; a; b) &\to f[f^{-1}[c]] = {\rm pr}_{2}\langle f \rangle \cap c
\end{align*}
が成り立つ.
またこれらから, 次の(${**}*$)が成り立つ: 

(${**}*$) ~~$f$が$a$から$b$への函数ならば, 
            \[
              c \subset {\rm pr}_{2}\langle f \rangle \leftrightarrow f[f^{-1}[c]] = c, ~~
              f[f^{-1}[c]] = {\rm pr}_{2}\langle f \rangle \cap c
            \]
            が共に成り立つ.
            故にこのとき, $c \subset {\rm pr}_{2}\langle f \rangle$が成り立つならば$f[f^{-1}[c]] = c$が成り立ち, 
            逆に$f[f^{-1}[c]] = c$が成り立つならば$c \subset {\rm pr}_{2}\langle f \rangle$が成り立つ.
\end{thm}


\noindent{\bf 証明}
~1)
まず前者から示す.
定理 \ref{sthmvaluesetinvfuncsubset2}より
\[
  {\rm Func}(f) \to f[f^{-1}[c]] \subset c
\]
が成り立ち, 定理 \ref{sthmvaluesetinv}と推論法則 \ref{dedequiv}により
\[
  c \subset {\rm pr}_{2}\langle f \rangle \to c \subset f[f^{-1}[c]]
\]
が成り立つから, 推論法則 \ref{dedfromaddw}により
\[
\tag{1}
  {\rm Func}(f) \wedge c \subset {\rm pr}_{2}\langle f \rangle \to f[f^{-1}[c]] \subset c \wedge c \subset f[f^{-1}[c]]
\]
が成り立つ.
また定理 \ref{sthmaxiom1}と推論法則 \ref{dedequiv}により
\[
\tag{2}
  f[f^{-1}[c]] \subset c \wedge c \subset f[f^{-1}[c]] \to f[f^{-1}[c]] = c
\]
が成り立つ.
そこで(1), (2)から, 推論法則 \ref{dedmmp}によって
\[
  {\rm Func}(f) \wedge c \subset {\rm pr}_{2}\langle f \rangle \to f[f^{-1}[c]] = c
\]
が成り立ち, これから推論法則 \ref{dedtwch}により
\[
\tag{3}
  {\rm Func}(f) \to (c \subset {\rm pr}_{2}\langle f \rangle \to f[f^{-1}[c]] = c)
\]
が成り立つ.
また定理 \ref{sthm=tsubset}より
\[
  f[f^{-1}[c]] = c \to c \subset f[f^{-1}[c]]
\]
が成り立ち, 定理 \ref{sthmvaluesetinv}と推論法則 \ref{dedequiv}により
\[
  c \subset f[f^{-1}[c]] \to c \subset {\rm pr}_{2}\langle f \rangle
\]
が成り立つから, 推論法則 \ref{dedmmp}により
\[
  f[f^{-1}[c]] = c \to c \subset {\rm pr}_{2}\langle f \rangle
\]
が成り立つ.
故に推論法則 \ref{dedatawbtrue2}により
\[
\tag{4}
  (c \subset {\rm pr}_{2}\langle f \rangle \to f[f^{-1}[c]] = c) 
  \to (c \subset {\rm pr}_{2}\langle f \rangle \leftrightarrow f[f^{-1}[c]] = c)
\]
が成り立つ.
そこで(3), (4)から, 推論法則 \ref{dedmmp}によって
\[
\tag{5}
  {\rm Func}(f) \to (c \subset {\rm pr}_{2}\langle f \rangle \leftrightarrow f[f^{-1}[c]] = c)
\]
が成り立つ.

次に後者を示す.
(3)から, 推論法則 \ref{dedch}により
\[
  c \subset {\rm pr}_{2}\langle f \rangle \to ({\rm Func}(f) \to f[f^{-1}[c]] = c)
\]
が成り立つ.
ここで$c$は任意の集合で良いので, この定理において$c$を${\rm pr}_{2}\langle f \rangle \cap c$に置き換えた
\[
\tag{6}
  {\rm pr}_{2}\langle f \rangle \cap c \subset {\rm pr}_{2}\langle f \rangle 
  \to ({\rm Func}(f) \to f[f^{-1}[{\rm pr}_{2}\langle f \rangle \cap c]] = {\rm pr}_{2}\langle f \rangle \cap c)
\]
も定理である.
いま定理 \ref{sthmcap}より
${\rm pr}_{2}\langle f \rangle \cap c \subset {\rm pr}_{2}\langle f \rangle$が成り立つから, 
従ってこれと(6)から, 推論法則 \ref{dedmp}により
\[
\tag{7}
  {\rm Func}(f) \to f[f^{-1}[{\rm pr}_{2}\langle f \rangle \cap c]] = {\rm pr}_{2}\langle f \rangle \cap c
\]
が成り立つ.
また定理 \ref{sthmfuncrelation}と推論法則 \ref{dedequiv}により
\[
\tag{8}
  {\rm Func}(f) \to {\rm Func}(f; {\rm pr}_{1}\langle f \rangle; {\rm pr}_{2}\langle f \rangle)
\]
が成り立つ.
また定理 \ref{sthmvaluesetinvfunc}より
\[
\tag{9}
  {\rm Func}(f; {\rm pr}_{1}\langle f \rangle; {\rm pr}_{2}\langle f \rangle) 
  \to f^{-1}[c] = f^{-1}[{\rm pr}_{2}\langle f \rangle \cap c]
\]
が成り立つ.
また定理 \ref{sthmvalueset=}より
\[
\tag{10}
  f^{-1}[c] = f^{-1}[{\rm pr}_{2}\langle f \rangle \cap c] 
  \to f[f^{-1}[c]] = f[f^{-1}[{\rm pr}_{2}\langle f \rangle \cap c]]
\]
が成り立つ.
またThm \ref{x=yt1x=zly=z1}より
\[
  f[f^{-1}[c]] = f[f^{-1}[{\rm pr}_{2}\langle f \rangle \cap c]] 
  \to (f[f^{-1}[c]] = {\rm pr}_{2}\langle f \rangle \cap c 
  \leftrightarrow f[f^{-1}[{\rm pr}_{2}\langle f \rangle \cap c]] = {\rm pr}_{2}\langle f \rangle \cap c)
\]
が成り立つから, 推論法則 \ref{dedprewedge}により
\[
\tag{11}
  f[f^{-1}[c]] = f[f^{-1}[{\rm pr}_{2}\langle f \rangle \cap c]] 
  \to (f[f^{-1}[{\rm pr}_{2}\langle f \rangle \cap c]] = {\rm pr}_{2}\langle f \rangle \cap c 
  \to f[f^{-1}[c]] = {\rm pr}_{2}\langle f \rangle \cap c)
\]
が成り立つ.
そこで(8)---(11)から, 推論法則 \ref{dedmmp}によって
\[
  {\rm Func}(f) \to (f[f^{-1}[{\rm pr}_{2}\langle f \rangle \cap c]] = {\rm pr}_{2}\langle f \rangle \cap c 
  \to f[f^{-1}[c]] = {\rm pr}_{2}\langle f \rangle \cap c)
\]
が成り立つことがわかる.
故に推論法則 \ref{deds2}により
\[
  ({\rm Func}(f) \to f[f^{-1}[{\rm pr}_{2}\langle f \rangle \cap c]] = {\rm pr}_{2}\langle f \rangle \cap c) 
  \to ({\rm Func}(f) \to f[f^{-1}[c]] = {\rm pr}_{2}\langle f \rangle \cap c)
\]
が成り立ち, これと(7)から, 推論法則 \ref{dedmp}によって
\[
\tag{12}
  {\rm Func}(f) \to f[f^{-1}[c]] = {\rm pr}_{2}\langle f \rangle \cap c
\]
が成り立つ.
($*$)が成り立つことは, (5), (12)と推論法則 \ref{dedmp}, \ref{dedeqfund}によって明らかである.

\noindent
2)
定理 \ref{sthmfuncbasis}より${\rm Func}(f; a) \to {\rm Func}(f)$が成り立つから, 
これと(5), (12)から, それぞれ推論法則 \ref{dedmmp}によって
\begin{align*}
  {\rm Func}(f; a) &\to (c \subset {\rm pr}_{2}\langle f \rangle \leftrightarrow f[f^{-1}[c]] = c), \\
  \mbox{} \\
  {\rm Func}(f; a) &\to f[f^{-1}[c]] = {\rm pr}_{2}\langle f \rangle \cap c
\end{align*}
が成り立つ.
($**$)が成り立つことは, これらと推論法則 \ref{dedmp}, \ref{dedeqfund}によって明らかである.

\noindent
3)
定理 \ref{sthmfuncbasis}より${\rm Func}(f; a; b) \to {\rm Func}(f)$が成り立つから, 
これと(5), (12)から, それぞれ推論法則 \ref{dedmmp}によって
\begin{align*}
  {\rm Func}(f; a; b) &\to (c \subset {\rm pr}_{2}\langle f \rangle \leftrightarrow f[f^{-1}[c]] = c), \\
  \mbox{} \\
  {\rm Func}(f; a; b) &\to f[f^{-1}[c]] = {\rm pr}_{2}\langle f \rangle \cap c
\end{align*}
が成り立つ.
(${**}*$)が成り立つことは, これらと推論法則 \ref{dedmp}, \ref{dedeqfund}によって明らかである.
\halmos




\mathstrut
\begin{thm}
\label{sthmcapvaluesetinvfunc}%定理
\mbox{}

1)
$c$, $d$, $f$を集合とするとき, 
\[
  {\rm Func}(f) \to f^{-1}[c \cap d] = f^{-1}[c] \cap f^{-1}[d]
\]
が成り立つ.
またこのことから, 次の($*$)が成り立つ: 

($*$) ~~$f$が函数ならば, $f^{-1}[c \cap d] = f^{-1}[c] \cap f^{-1}[d]$が成り立つ.

2)
$a$, $c$, $d$, $f$を集合とするとき, 
\[
  {\rm Func}(f; a) \to f^{-1}[c \cap d] = f^{-1}[c] \cap f^{-1}[d]
\]
が成り立つ.
またこのことから, 次の($**$)が成り立つ: 

($**$) ~~$f$が$a$における函数ならば, $f^{-1}[c \cap d] = f^{-1}[c] \cap f^{-1}[d]$が成り立つ.

3)
$a$, $b$, $c$, $d$, $f$を集合とするとき, 
\[
  {\rm Func}(f; a; b) \to f^{-1}[c \cap d] = f^{-1}[c] \cap f^{-1}[d]
\]
が成り立つ.
またこのことから, 次の(${**}*$)が成り立つ: 

(${**}*$) ~~$f$が$a$から$b$への函数ならば, $f^{-1}[c \cap d] = f^{-1}[c] \cap f^{-1}[d]$が成り立つ.
\end{thm}


\noindent{\bf 証明}
~1)
$x$を$c$, $d$, $f$のいずれの記号列の中にも自由変数として現れない, 定数でない文字とする.
このときまずThm \ref{awbta}より
\[
\tag{1}
  {\rm Func}(f) \wedge x \in f^{-1}[c] \cap f^{-1}[d] \to x \in f^{-1}[c] \cap f^{-1}[d]
\]
が成り立つ.
また定理 \ref{sthmcapelement}と推論法則 \ref{dedequiv}により
\[
\tag{2}
  x \in f^{-1}[c] \cap f^{-1}[d] \to x \in f^{-1}[c] \wedge x \in f^{-1}[d]
\]
が成り立つ.
またいま$y$を$x$と異なり, $c$, $d$, $f$のいずれの記号列の中にも自由変数として現れない文字とすれば, 
変数法則 \ref{valinv}より$y$は$f^{-1}$の中にも自由変数として現れないから, 
定理 \ref{sthmvaluesetelement}と推論法則 \ref{dedequiv}により
\[
  x \in f^{-1}[c] \to \exists y(y \in c \wedge (y, x) \in f^{-1}), ~~
  x \in f^{-1}[d] \to \exists y(y \in d \wedge (y, x) \in f^{-1})
\]
が共に成り立つ.
ここで$\tau_{y}(y \in c \wedge (y, x) \in f^{-1})$, $\tau_{y}(y \in d \wedge (y, x) \in f^{-1})$を
それぞれ$T$, $U$と書けば, これらは共に集合であり, 定義から上記の二つの記号列はそれぞれ
\[
  x \in f^{-1}[c] \to (T|y)(y \in c \wedge (y, x) \in f^{-1}), ~~
  x \in f^{-1}[d] \to (U|y)(y \in d \wedge (y, x) \in f^{-1})
\]
と同じである.
また$y$は$x$と異なり, $c$及び$d$の中に自由変数として現れず, 
上述のように$f^{-1}$の中にも自由変数として現れないから, 
代入法則 \ref{substfree}, \ref{substfund}, \ref{substwedge}, \ref{substpair}によれば, 
上記の二つの記号列はそれぞれ
\[
  x \in f^{-1}[c] \to T \in c \wedge (T, x) \in f^{-1}, ~~
  x \in f^{-1}[d] \to U \in d \wedge (U, x) \in f^{-1}
\]
と一致する.
よってこれらが共に定理となる.
そこで推論法則 \ref{dedprewedge}により
\begin{align*}
  x \in f^{-1}[c] \to T \in c&, ~~
  x \in f^{-1}[d] \to U \in d, \\
  \mbox{} \\
  x \in f^{-1}[c] \to (T, x) \in f^{-1}&, ~~
  x \in f^{-1}[d] \to (U, x) \in f^{-1}
\end{align*}
がすべて成り立ち, 故にこのはじめの二つ, 後の二つから, それぞれ推論法則 \ref{dedfromaddw}により
\begin{align*}
  \tag{3}
  x \in f^{-1}[c] \wedge x \in f^{-1}[d] &\to T \in c \wedge U \in d, \\
  \mbox{} \\
  \tag{4}
  x \in f^{-1}[c] \wedge x \in f^{-1}[d] &\to (T, x) \in f^{-1} \wedge (U, x) \in f^{-1}
\end{align*}
が成り立つ.
そこで(1), (2), (4)から, 推論法則 \ref{dedmmp}によって
\[
  {\rm Func}(f) \wedge x \in f^{-1}[c] \cap f^{-1}[d] \to (T, x) \in f^{-1} \wedge (U, x) \in f^{-1}
\]
が成り立つことがわかり, 故にこれから推論法則 \ref{dedprewedge}によって
\[
\tag{5}
  {\rm Func}(f) \wedge x \in f^{-1}[c] \cap f^{-1}[d] \to (U, x) \in f^{-1}
\]
が成り立つ.
また(1), (2), (3)から, 推論法則 \ref{dedmmp}によって
\[
\tag{6}
  {\rm Func}(f) \wedge x \in f^{-1}[c] \cap f^{-1}[d] \to T \in c \wedge U \in d
\]
が成り立つこともわかる.
また定理 \ref{sthmpairininv}と推論法則 \ref{dedequiv}により
\[
  (T, x) \in f^{-1} \to (x, T) \in f, ~~
  (U, x) \in f^{-1} \to (x, U) \in f
\]
が共に成り立つから, 推論法則 \ref{dedfromaddw}により
\[
\tag{7}
  (T, x) \in f^{-1} \wedge (U, x) \in f^{-1} \to (x, T) \in f \wedge (x, U) \in f
\]
が成り立つ.
そこで(2), (4), (7)から, 推論法則 \ref{dedmmp}によって
\[
  x \in f^{-1}[c] \cap f^{-1}[d] \to (x, T) \in f \wedge (x, U) \in f
\]
が成り立つことがわかる.
故に推論法則 \ref{dedaddw}により
\[
\tag{8}
  {\rm Func}(f) \wedge x \in f^{-1}[c] \cap f^{-1}[d] \to {\rm Func}(f) \wedge ((x, T) \in f \wedge (x, U) \in f)
\]
が成り立つ.
また定理 \ref{sthmpairinfunc}より
\[
  {\rm Func}(f) \to ((x, T) \in f \wedge (x, U) \in f \to T = U)
\]
が成り立つから, 推論法則 \ref{dedtwch}により
\[
  {\rm Func}(f) \wedge ((x, T) \in f \wedge (x, U) \in f) \to T = U
\]
が成り立つ.
そこでこれと(8)から, 推論法則 \ref{dedmmp}によって
\[
  {\rm Func}(f) \wedge x \in f^{-1}[c] \cap f^{-1}[d] \to T = U
\]
が成り立ち, これと(6)から, 推論法則 \ref{dedprewedge}によって
\[
\tag{9}
  {\rm Func}(f) \wedge x \in f^{-1}[c] \cap f^{-1}[d] \to T = U \wedge (T \in c \wedge U \in d)
\]
が成り立つ.
またThm \ref{aw1bwc1t1awb1wc}より
\[
\tag{10}
  T = U \wedge (T \in c \wedge U \in d) \to (T = U \wedge T \in c) \wedge U \in d
\]
が成り立つ.
また定理 \ref{sthm=&in}より
\[
  T = U \wedge T \in c \to U \in c
\]
が成り立つから, 推論法則 \ref{dedaddw}により
\[
\tag{11}
  (T = U \wedge T \in c) \wedge U \in d \to U \in c \wedge U \in d
\]
が成り立つ.
また定理 \ref{sthmcapelement}と推論法則 \ref{dedequiv}により
\[
\tag{12}
  U \in c \wedge U \in d \to U \in c \cap d
\]
が成り立つ.
そこで(9)---(12)から, 推論法則 \ref{dedmmp}によって
\[
  {\rm Func}(f) \wedge x \in f^{-1}[c] \cap f^{-1}[d] \to U \in c \cap d
\]
が成り立つことがわかる.
故にこれと(5)から, 推論法則 \ref{dedprewedge}により
\[
\tag{13}
  {\rm Func}(f) \wedge x \in f^{-1}[c] \cap f^{-1}[d] \to U \in c \cap d \wedge (U, x) \in f^{-1}
\]
が成り立つ.
また定理 \ref{sthmvaluesetbasis}より
\[
\tag{14}
  U \in c \cap d \wedge (U, x) \in f^{-1} \to x \in f^{-1}[c \cap d]
\]
が成り立つ.
そこで(13), (14)から, 推論法則 \ref{dedmmp}によって
\[
  {\rm Func}(f) \wedge x \in f^{-1}[c] \cap f^{-1}[d] \to x \in f^{-1}[c \cap d]
\]
が成り立ち, これから推論法則 \ref{dedtwch}により
\[
\tag{15}
  {\rm Func}(f) \to (x \in f^{-1}[c] \cap f^{-1}[d] \to x \in f^{-1}[c \cap d])
\]
が成り立つ.
さていま$x$は$f$の中に自由変数として現れないので, 変数法則 \ref{valfunc}により, 
$x$は${\rm Func}(f)$の中に自由変数として現れない.
また$x$は定数でない.
これらのことと, (15)が成り立つことから, 推論法則 \ref{dedalltquansepfreeconst}により
\[
  {\rm Func}(f) \to \forall x(x \in f^{-1}[c] \cap f^{-1}[d] \to x \in f^{-1}[c \cap d])
\]
が成り立つ.
ここで$x$が$c$, $d$, $f$のいずれの記号列の中にも自由変数として現れないことから, 
変数法則 \ref{valcap}, \ref{valvalueset}, \ref{valinv}によってわかるように, 
$x$は$f^{-1}[c] \cap f^{-1}[d]$及び$f^{-1}[c \cap d]$の中に自由変数として現れない.
故に定義から, 上記の記号列は
\[
\tag{16}
  {\rm Func}(f) \to f^{-1}[c] \cap f^{-1}[d] \subset f^{-1}[c \cap d]
\]
と同じである.
よってこれが定理となる.
また定理 \ref{sthmcapvalueset}より
\[
  f^{-1}[c \cap d] \subset f^{-1}[c] \cap f^{-1}[d]
\]
が成り立つから, 推論法則 \ref{dedatawbtrue2}により
\[
\tag{17}
  f^{-1}[c] \cap f^{-1}[d] \subset f^{-1}[c \cap d] 
  \to f^{-1}[c \cap d] \subset f^{-1}[c] \cap f^{-1}[d] \wedge f^{-1}[c] \cap f^{-1}[d] \subset f^{-1}[c \cap d]
\]
が成り立つ.
また定理 \ref{sthmaxiom1}と推論法則 \ref{dedequiv}により
\[
\tag{18}
  f^{-1}[c \cap d] \subset f^{-1}[c] \cap f^{-1}[d] \wedge f^{-1}[c] \cap f^{-1}[d] \subset f^{-1}[c \cap d] 
  \to f^{-1}[c \cap d] = f^{-1}[c] \cap f^{-1}[d]
\]
が成り立つ.
そこで(16), (17), (18)から, 推論法則 \ref{dedmmp}によって
\[
\tag{19}
  {\rm Func}(f) \to f^{-1}[c \cap d] = f^{-1}[c] \cap f^{-1}[d]
\]
が成り立つことがわかる.
($*$)が成り立つことは, これと推論法則 \ref{dedmp}によって明らかである.

\noindent
2)
定理 \ref{sthmfuncbasis}より${\rm Func}(f; a) \to {\rm Func}(f)$が成り立つから, 
これと(19)から, 推論法則 \ref{dedmmp}によって
\[
  {\rm Func}(f; a) \to f^{-1}[c \cap d] = f^{-1}[c] \cap f^{-1}[d]
\]
が成り立つ.
($**$)が成り立つことは, これと推論法則 \ref{dedmp}によって明らかである.

\noindent
3)
定理 \ref{sthmfuncbasis}より${\rm Func}(f; a; b) \to {\rm Func}(f)$が成り立つから, 
これと(19)から, 推論法則 \ref{dedmmp}によって
\[
  {\rm Func}(f; a; b) \to f^{-1}[c \cap d] = f^{-1}[c] \cap f^{-1}[d]
\]
が成り立つ.
(${**}*$)が成り立つことは, これと推論法則 \ref{dedmp}によって明らかである.
\halmos




\mathstrut
\begin{thm}
\label{sthm-valuesetinvfunc}%定理
\mbox{}

1)
$c$, $d$, $f$を集合とするとき, 
\[
  {\rm Func}(f) \to f^{-1}[c - d] = f^{-1}[c] - f^{-1}[d]
\]
が成り立つ.
またこのことから, 次の($*$)が成り立つ: 

($*$) ~~$f$が函数ならば, $f^{-1}[c - d] = f^{-1}[c] - f^{-1}[d]$が成り立つ.

2)
$a$, $c$, $d$, $f$を集合とするとき, 
\[
  {\rm Func}(f; a) \to f^{-1}[c - d] = f^{-1}[c] - f^{-1}[d]
\]
が成り立つ.
またこのことから, 次の($**$)が成り立つ: 

($**$) ~~$f$が$a$における函数ならば, $f^{-1}[c - d] = f^{-1}[c] - f^{-1}[d]$が成り立つ.

3)
$a$, $b$, $c$, $d$, $f$を集合とするとき, 
\begin{align*}
  {\rm Func}(f; a; b) &\to f^{-1}[c - d] = f^{-1}[c] - f^{-1}[d], \\
  \mbox{} \\
  {\rm Func}(f; a; b) &\to f^{-1}[b - c] = a - f^{-1}[c]
\end{align*}
が成り立つ.
またこれらから, 次の(${**}*$)が成り立つ: 

(${**}*$) ~~$f$が$a$から$b$への函数ならば, $f^{-1}[c - d] = f^{-1}[c] - f^{-1}[d]$が成り立つ.
            特にこのとき, $f^{-1}[b - c] = a - f^{-1}[c]$が成り立つ.
\end{thm}


\noindent{\bf 証明}
~1)
定理 \ref{sthmcapvaluesetinvfunc}より
\[
\tag{1}
  {\rm Func}(f) \to f^{-1}[(c - d) \cap d] = f^{-1}[c - d] \cap f^{-1}[d]
\]
が成り立つ.
またThm \ref{x=yty=x}より
\[
\tag{2}
  f^{-1}[(c - d) \cap d] = f^{-1}[c - d] \cap f^{-1}[d] \to f^{-1}[c - d] \cap f^{-1}[d] = f^{-1}[(c - d) \cap d]
\]
が成り立つ.
また定理 \ref{sthmcap-}より
\[
\tag{3}
  (c - d) \cap d = (c \cap d) - d
\]
が成り立つ.
また定理 \ref{sthmcap}より$c \cap d \subset d$が成り立ち, 
定理 \ref{sthm-subset=}より
$c \cap d \subset d \leftrightarrow (c \cap d) - d = \phi$が成り立つから, 
推論法則 \ref{dedeqfund}により
\[
\tag{4}
  (c \cap d) - d = \phi
\]
が成り立つ.
そこで(3), (4)から, 推論法則 \ref{ded=trans}によって
$(c - d) \cap d = \phi$が成り立ち, 故に定理 \ref{sthmvalueset=}により
\[
\tag{5}
  f^{-1}[(c - d) \cap d] = f^{-1}[\phi]
\]
が成り立つ.
いま定理 \ref{sthmvaluesetempty}より
$f^{-1}[\phi] = \phi$が成り立つから, 
従ってこれと(5)から, 推論法則 \ref{ded=trans}により
\[
  f^{-1}[(c - d) \cap d] = \phi
\]
が成り立つ.
故に推論法則 \ref{dedatawbtrue2}により
\[
\tag{6}
  f^{-1}[c - d] \cap f^{-1}[d] = f^{-1}[(c - d) \cap d] 
  \to f^{-1}[c - d] \cap f^{-1}[d] = f^{-1}[(c - d) \cap d] \wedge f^{-1}[(c - d) \cap d] = \phi
\]
が成り立つ.
またThm \ref{x=ywy=ztx=z}より
\[
\tag{7}
  f^{-1}[c - d] \cap f^{-1}[d] = f^{-1}[(c - d) \cap d] \wedge f^{-1}[(c - d) \cap d] = \phi \to f^{-1}[c - d] \cap f^{-1}[d] = \phi
\]
が成り立つ.
さてここで$x$を$c$, $d$, $f$のいずれの記号列の中にも自由変数として現れない, 
定数でない文字とする.
このとき定理 \ref{sthmnotinempty}より
\[
\tag{8}
  f^{-1}[c - d] \cap f^{-1}[d] = \phi \to x \notin f^{-1}[c - d] \cap f^{-1}[d]
\]
が成り立つ.
また定理 \ref{sthmcapelement}と推論法則 \ref{dedequiv}により
\[
  x \in f^{-1}[c - d] \wedge x \in f^{-1}[d] \to x \in f^{-1}[c - d] \cap f^{-1}[d]
\]
が成り立つから, 推論法則 \ref{dedcp}により
\[
\tag{9}
  x \notin f^{-1}[c - d] \cap f^{-1}[d] \to \neg (x \in f^{-1}[c - d] \wedge x \in f^{-1}[d])
\]
が成り立つ.
またThm \ref{n1awb1tnavnb}より
\[
\tag{10}
  \neg (x \in f^{-1}[c - d] \wedge x \in f^{-1}[d]) \to x \notin f^{-1}[c - d] \vee x \notin f^{-1}[d]
\]
が成り立つ.
またThm \ref{1atb1tnavb}より
\[
\tag{11}
  x \notin f^{-1}[c - d] \vee x \notin f^{-1}[d] \to (x \in f^{-1}[c - d] \to x \notin f^{-1}[d])
\]
が成り立つ.
そこで(1), (2), (6)---(11)から, 推論法則 \ref{dedmmp}によって
\[
  {\rm Func}(f) \to (x \in f^{-1}[c - d] \to x \notin f^{-1}[d])
\]
が成り立つことがわかる.
故に推論法則 \ref{dedtwch}により
\[
\tag{12}
  {\rm Func}(f) \wedge x \in f^{-1}[c - d] \to x \notin f^{-1}[d]
\]
が成り立つ.
またThm \ref{awbta}より
\[
\tag{13}
  {\rm Func}(f) \wedge x \in f^{-1}[c - d] \to x \in f^{-1}[c - d]
\]
が成り立つ.
また定理 \ref{sthma-bsubseta}より$c - d \subset c$が成り立つから, 
定理 \ref{sthmvaluesetsubset}により
$f^{-1}[c - d] \subset f^{-1}[c]$が成り立つ.
故に定理 \ref{sthmsubsetbasis}により
\[
\tag{14}
  x \in f^{-1}[c - d] \to x \in f^{-1}[c]
\]
が成り立つ.
そこで(13), (14)から, 推論法則 \ref{dedmmp}によって
\[
  {\rm Func}(f) \wedge x \in f^{-1}[c - d] \to x \in f^{-1}[c]
\]
が成り立つ.
従ってこれと(12)から, 推論法則 \ref{dedprewedge}により
\[
\tag{15}
  {\rm Func}(f) \wedge x \in f^{-1}[c - d] \to x \in f^{-1}[c] \wedge x \notin f^{-1}[d]
\]
が成り立つ.
また定理 \ref{sthm-basis}と推論法則 \ref{dedequiv}により
\[
\tag{16}
  x \in f^{-1}[c] \wedge x \notin f^{-1}[d] \to x \in f^{-1}[c] - f^{-1}[d]
\]
が成り立つ.
そこで(15), (16)から, 推論法則 \ref{dedmmp}によって
\[
  {\rm Func}(f) \wedge x \in f^{-1}[c - d] \to x \in f^{-1}[c] - f^{-1}[d]
\]
が成り立ち, 故に推論法則 \ref{dedtwch}により
\[
\tag{17}
  {\rm Func}(f) \to (x \in f^{-1}[c - d] \to x \in f^{-1}[c] - f^{-1}[d])
\]
が成り立つ.
さていま$x$は$f$の中に自由変数として現れないから, 変数法則 \ref{valfunc}により, 
$x$は${\rm Func}(f)$の中に自由変数として現れない.
また$x$は定数でない.
これらのことと, (17)が成り立つことから, 推論法則 \ref{dedalltquansepfreeconst}により
\[
  {\rm Func}(f) \to \forall x(x \in f^{-1}[c - d] \to x \in f^{-1}[c] - f^{-1}[d])
\]
が成り立つ.
また$x$は$c$, $d$, $f$のいずれの記号列の中にも自由変数として現れないから, 
変数法則 \ref{val-}, \ref{valvalueset}, \ref{valinv}によれば, $x$は
$f^{-1}[c - d]$及び$f^{-1}[c] - f^{-1}[d]$の中に自由変数として現れない.
故に定義から, 上記の記号列は
\[
\tag{18}
  {\rm Func}(f) \to f^{-1}[c - d] \subset f^{-1}[c] - f^{-1}[d]
\]
と同じである.
従ってこれが定理となる.
また定理 \ref{sthm-valueset}より
$f^{-1}[c] - f^{-1}[d] \subset f^{-1}[c - d]$が成り立つから, 
推論法則 \ref{dedatawbtrue2}により
\[
\tag{19}
  f^{-1}[c - d] \subset f^{-1}[c] - f^{-1}[d] 
  \to f^{-1}[c - d] \subset f^{-1}[c] - f^{-1}[d] \wedge f^{-1}[c] - f^{-1}[d] \subset f^{-1}[c - d]
\]
が成り立つ.
また定理 \ref{sthmaxiom1}と推論法則 \ref{dedequiv}により
\[
\tag{20}
  f^{-1}[c - d] \subset f^{-1}[c] - f^{-1}[d] \wedge f^{-1}[c] - f^{-1}[d] \subset f^{-1}[c - d] 
  \to f^{-1}[c - d] = f^{-1}[c] - f^{-1}[d]
\]
が成り立つ.
そこで(18), (19), (20)から, 推論法則 \ref{dedmmp}によって
\[
\tag{21}
  {\rm Func}(f) \to f^{-1}[c - d] = f^{-1}[c] - f^{-1}[d]
\]
が成り立つことがわかる.
($*$)が成り立つことは, これと推論法則 \ref{dedmp}によって明らかである.

\noindent
2)
定理 \ref{sthmfuncbasis}より${\rm Func}(f; a) \to {\rm Func}(f)$が成り立つから, 
これと(21)から, 推論法則 \ref{dedmmp}によって
\[
  {\rm Func}(f; a) \to f^{-1}[c - d] = f^{-1}[c] - f^{-1}[d]
\]
が成り立つ.
($**$)が成り立つことは, これと推論法則 \ref{dedmp}によって明らかである.

\noindent
3)
定理 \ref{sthmfuncbasis}より${\rm Func}(f; a; b) \to {\rm Func}(f)$が成り立つから, 
これと(21)から, 推論法則 \ref{dedmmp}によって
\[
\tag{22}
  {\rm Func}(f; a; b) \to f^{-1}[c - d] = f^{-1}[c] - f^{-1}[d]
\]
が成り立つ.
そこで特に, この(22)において, $c$を$b$, $d$を$c$にそれぞれ置き換えて得られる記号列
\[
\tag{23}
  {\rm Func}(f; a; b) \to f^{-1}[b - c] = f^{-1}[b] - f^{-1}[c]
\]
も定理となる.
また定理 \ref{sthmvaluesetinvfunc}より
\[
  {\rm Func}(f; a; b) \to f^{-1}[b] = a
\]
が成り立ち, 定理 \ref{sthm-=}より
\[
  f^{-1}[b] = a \to f^{-1}[b] - f^{-1}[c] = a - f^{-1}[c]
\]
が成り立つから, これらから, 推論法則 \ref{dedmmp}によって
\[
  {\rm Func}(f; a; b) \to f^{-1}[b] - f^{-1}[c] = a - f^{-1}[c]
\]
が成り立つ.
故にこれと(23)から, 推論法則 \ref{dedprewedge}により
\[
\tag{24}
  {\rm Func}(f; a; b) \to f^{-1}[b - c] = f^{-1}[b] - f^{-1}[c] \wedge f^{-1}[b] - f^{-1}[c] = a - f^{-1}[c]
\]
が成り立つ.
またThm \ref{x=ywy=ztx=z}より
\[
\tag{25}
  f^{-1}[b - c] = f^{-1}[b] - f^{-1}[c] \wedge f^{-1}[b] - f^{-1}[c] = a - f^{-1}[c] \to f^{-1}[b - c] = a - f^{-1}[c]
\]
が成り立つ.
そこで(24), (25)から, 推論法則 \ref{dedmmp}によって
\[
\tag{26}
  {\rm Func}(f; a; b) \to f^{-1}[b - c] = a - f^{-1}[c]
\]
が成り立つ.
(${**}*$)が成り立つことは, (22), (26)が共に成り立つことと推論法則 \ref{dedmp}によって明らかである.
\halmos




\mathstrut
\begin{thm}
\label{sthmcompfunc}%定理
\mbox{}

1)
$f$と$g$を集合とするとき, 
\[
  {\rm Func}(f) \wedge {\rm Func}(g) \to {\rm Func}(g \circ f)
\]
が成り立つ.
またこのことから, 次の($*$)が成り立つ: 

($*$) ~~$f$と$g$が共に函数ならば, $g \circ f$は函数である.

2)
$a$, $b$, $f$, $g$を集合とするとき, 
\[
  {\rm Func}(f; a; b) \wedge {\rm Func}(g; b) \to {\rm Func}(g \circ f; a)
\]
が成り立つ.
またこのことから, 次の($**$)が成り立つ: 

($**$) ~~$f$が$a$から$b$への函数であり, $g$が$b$における函数ならば, 
         $g \circ f$は$a$における函数である.

3)
$a$, $b$, $c$, $f$, $g$を集合とするとき, 
\[
  {\rm Func}(f; a; b) \wedge {\rm Func}(g; b; c) \to {\rm Func}(g \circ f; a; c)
\]
が成り立つ.
またこのことから, 次の(${**}*$)が成り立つ: 

(${**}*$) ~~$f$が$a$から$b$への函数であり, $g$が$b$から$c$への函数ならば, 
            $g \circ f$は$a$から$c$への函数である.
\end{thm}


\noindent{\bf 証明}
~1)
$x$, $y$, $z$, $w$を, どの二つも互いに異なり, いずれも$f$及び$g$の中に
自由変数として現れない, 定数でない文字とする.
このとき定理 \ref{sthmpairincompeq}と推論法則 \ref{dedequiv}により
\[
  (x, z) \in g \circ f \to \exists y((x, y) \in f \wedge (y, z) \in g), ~~
  (x, w) \in g \circ f \to \exists y((x, y) \in f \wedge (y, w) \in g)
\]
が共に成り立つ.
ここで$\tau_{y}((x, y) \in f \wedge (y, z) \in g)$, $\tau_{y}((x, y) \in f \wedge (y, w) \in g)$を
それぞれ$T$, $U$と書けば, これらは共に集合であり, 
上記の二つの記号列はそれぞれ
\[
  (x, z) \in g \circ f \to (T|y)((x, y) \in f \wedge (y, z) \in g), ~~
  (x, w) \in g \circ f \to (U|y)((x, y) \in f \wedge (y, w) \in g)
\]
と一致する.
また$y$が$x$, $z$, $w$のいずれとも異なり, $f$及び$g$の中に自由変数として現れないことから, 
代入法則 \ref{substfree}, \ref{substfund}, \ref{substwedge}, \ref{substpair}により, 
これらの記号列はそれぞれ
\[
  (x, z) \in g \circ f \to (x, T) \in f \wedge (T, z) \in g, ~~
  (x, w) \in g \circ f \to (x, U) \in f \wedge (U, w) \in g
\]
と一致する.
よってこれらが共に定理となる.
そこで推論法則 \ref{dedprewedge}により
\begin{align*}
  (x, z) \in g \circ f \to (x, T) \in f&, ~~
  (x, w) \in g \circ f \to (x, U) \in f, \\
  \mbox{} \\
  (x, z) \in g \circ f \to (T, z) \in g&, ~~
  (x, w) \in g \circ f \to (U, w) \in g
\end{align*}
がすべて成り立つ.
故にこのはじめの二つ, あとの二つの定理から, それぞれ推論法則 \ref{dedfromaddw}によって
\begin{align*}
  \tag{1}
  (x, z) \in g \circ f \wedge (x, w) \in g \circ f &\to (x, T) \in f \wedge (x, U) \in f, \\
  \mbox{} \\
  \tag{2}
  (x, z) \in g \circ f \wedge (x, w) \in g \circ f &\to (T, z) \in g \wedge (U, w) \in g
\end{align*}
が成り立つ.
故に(1)から, 推論法則 \ref{dedaddw}によって
\[
\tag{3}
  {\rm Func}(f) \wedge ((x, z) \in g \circ f \wedge (x, w) \in g \circ f) 
  \to {\rm Func}(f) \wedge ((x, T) \in f \wedge (x, U) \in f)
\]
が成り立つ.
また定理 \ref{sthmpairinfunc}より
\[
  {\rm Func}(f) \to ((x, T) \in f \wedge (x, U) \in f \to T = U)
\]
が成り立つから, 推論法則 \ref{dedtwch}により
\[
\tag{4}
  {\rm Func}(f) \wedge ((x, T) \in f \wedge (x, U) \in f) \to T = U
\]
が成り立つ.
また定理 \ref{sthmpairweak}と推論法則 \ref{dedequiv}により
\[
\tag{5}
  T = U \to (T, z) = (U, z)
\]
が成り立つ.
そこで(3), (4), (5)から, 推論法則 \ref{dedmmp}によって
\[
\tag{6}
  {\rm Func}(f) \wedge ((x, z) \in g \circ f \wedge (x, w) \in g \circ f) \to (T, z) = (U, z)
\]
が成り立つことがわかる.
またThm \ref{awbta}より
\[
  {\rm Func}(f) \wedge ((x, z) \in g \circ f \wedge (x, w) \in g \circ f) 
  \to (x, z) \in g \circ f \wedge (x, w) \in g \circ f
\]
が成り立つから, これと(2)から, 推論法則 \ref{dedmmp}によって
\[
  {\rm Func}(f) \wedge ((x, z) \in g \circ f \wedge (x, w) \in g \circ f) 
  \to (T, z) \in g \wedge (U, w) \in g
\]
が成り立つ.
故にこれと(6)から, 推論法則 \ref{dedprewedge}によって
\[
\tag{7}
  {\rm Func}(f) \wedge ((x, z) \in g \circ f \wedge (x, w) \in g \circ f) 
  \to (T, z) = (U, z) \wedge ((T, z) \in g \wedge (U, w) \in g)
\]
が成り立つ.
またThm \ref{aw1bwc1t1awb1wc}より
\[
\tag{8}
  (T, z) = (U, z) \wedge ((T, z) \in g \wedge (U, w) \in g) 
  \to ((T, z) = (U, z) \wedge (T, z) \in g) \wedge (U, w) \in g
\]
が成り立つ.
また定理 \ref{sthm=&in}より
\[
  (T, z) = (U, z) \wedge (T, z) \in g \to (U, z) \in g
\]
が成り立つから, 推論法則 \ref{dedaddw}により
\[
\tag{9}
  ((T, z) = (U, z) \wedge (T, z) \in g) \wedge (U, w) \in g \to (U, z) \in g \wedge (U, w) \in g
\]
が成り立つ.
そこで(7), (8), (9)から, 推論法則 \ref{dedmmp}によって
\[
  {\rm Func}(f) \wedge ((x, z) \in g \circ f \wedge (x, w) \in g \circ f) 
  \to (U, z) \in g \wedge (U, w) \in g
\]
が成り立つことがわかる.
故にこれから, 推論法則 \ref{dedaddw}によって
\[
\tag{10}
  {\rm Func}(g) \wedge ({\rm Func}(f) \wedge ((x, z) \in g \circ f \wedge (x, w) \in g \circ f)) 
  \to {\rm Func}(g) \wedge ((U, z) \in g \wedge (U, w) \in g)
\]
が成り立つ.
またThm \ref{awbtbwa}より
\[
  {\rm Func}(f) \wedge {\rm Func}(g) \to {\rm Func}(g) \wedge {\rm Func}(f)
\]
が成り立つから, 推論法則 \ref{dedaddw}により
\begin{multline*}
\tag{11}
  ({\rm Func}(f) \wedge {\rm Func}(g)) \wedge ((x, z) \in g \circ f \wedge (x, w) \in g \circ f) \\
  \to ({\rm Func}(g) \wedge {\rm Func}(f)) \wedge ((x, z) \in g \circ f \wedge (x, w) \in g \circ f)
\end{multline*}
が成り立つ.
またThm \ref{1awb1wctaw1bwc1}より
\begin{multline*}
\tag{12}
  ({\rm Func}(g) \wedge {\rm Func}(f)) \wedge ((x, z) \in g \circ f \wedge (x, w) \in g \circ f) \\
  \to {\rm Func}(g) \wedge ({\rm Func}(f) \wedge ((x, z) \in g \circ f \wedge (x, w) \in g \circ f))
\end{multline*}
が成り立つ.
また定理 \ref{sthmpairinfunc}より
\[
  {\rm Func}(g) \to ((U, z) \in g \wedge (U, w) \in g \to z = w)
\]
が成り立つから, 推論法則 \ref{dedtwch}により
\[
\tag{13}
  {\rm Func}(g) \wedge ((U, z) \in g \wedge (U, w) \in g) \to z = w
\]
が成り立つ.
そこで(11), (12), (10), (13)にこの順で推論法則 \ref{dedmmp}を適用していき, 
\[
  ({\rm Func}(f) \wedge {\rm Func}(g)) \wedge ((x, z) \in g \circ f \wedge (x, w) \in g \circ f) \to z = w
\]
が成り立つことがわかる.
故にこれから, 推論法則 \ref{dedtwch}により
\[
\tag{14}
  {\rm Func}(f) \wedge {\rm Func}(g) \to ((x, z) \in g \circ f \wedge (x, w) \in g \circ f \to z = w)
\]
が成り立つ.
さていま$x$, $z$, $w$はいずれも$f$及び$g$の中に自由変数として現れないから, 
変数法則 \ref{valwedge}, \ref{valfunc}により, これらはいずれも${\rm Func}(f) \wedge {\rm Func}(g)$の中に
自由変数として現れない.
またこれらはいずれも定数でない.
故に(14)から, 推論法則 \ref{dedalltquansepfreeconst}により
\[
  {\rm Func}(f) \wedge {\rm Func}(g) \to \forall x(\forall z(\forall w((x, z) \in g \circ f \wedge (x, w) \in g \circ f \to z = w)))
\]
が成り立つことがわかる.
また$y$は$f$及び$g$の中に自由変数として現れないから, 
変数法則 \ref{valcomp}により, $y$は$g \circ f$の中に自由変数として現れない.
このことと, $y$が$x$と異なることから, 
代入法則 \ref{substfree}, \ref{substfund}, \ref{substpair}により, 上記の記号列は
\[
  {\rm Func}(f) \wedge {\rm Func}(g) \to \forall x(\forall z(\forall w((z|y)((x, y) \in g \circ f) \wedge (w|y)((x, y) \in g \circ f) \to z = w)))
\]
と一致する.
また$z$と$w$は共に$x$とも$y$とも異なり, $f$及び$g$の中に自由変数として現れないから, 
変数法則 \ref{valfund}, \ref{valpair}, \ref{valcomp}によってわかるように, 
$z$と$w$は共に$(x, y) \in g \circ f$の中に自由変数として現れない.
また$z$と$w$は互いに異なる.
これらのことから, 定義より上記の記号列は
\[
\tag{15}
  {\rm Func}(f) \wedge {\rm Func}(g) \to \forall x(!y((x, y) \in g \circ f))
\]
と一致する.
よってこれが定理となる.
また定理 \ref{sthmcompgraph}より$g \circ f$はグラフだから, 
推論法則 \ref{dedatawbtrue2}により
\[
  \forall x(!y((x, y) \in g \circ f)) \to {\rm Graph}(g \circ f) \wedge \forall x(!y((x, y) \in g \circ f))
\]
が成り立つ.
ここで$x$と$y$は共に$f$及び$g$の中に自由変数として現れないから, 
変数法則 \ref{valcomp}により, $x$と$y$は共に$g \circ f$の中に自由変数として現れない.
また$x$と$y$は互いに異なる.
故に定義から, 上記の記号列は
\[
\tag{16}
  \forall x(!y((x, y) \in g \circ f)) \to {\rm Func}(g \circ f)
\]
と一致し, 従ってこれが定理となる.
そこで(15), (16)から, 推論法則 \ref{dedmmp}によって
\[
\tag{17}
  {\rm Func}(f) \wedge {\rm Func}(g) \to {\rm Func}(g \circ f)
\]
が成り立つ.
($*$)が成り立つことは, これと推論法則 \ref{dedmp}, \ref{dedwedge}によって明らかである.

\noindent
2)
定理 \ref{sthmfuncbasis}より
\[
  {\rm Func}(f; a; b) \to {\rm Func}(f), ~~
  {\rm Func}(g; b) \to {\rm Func}(g)
\]
が共に成り立つから, 推論法則 \ref{dedfromaddw}により
\[
  {\rm Func}(f; a; b) \wedge {\rm Func}(g; b) \to {\rm Func}(f) \wedge {\rm Func}(g)
\]
が成り立つ.
そこでこれと(17)から, 推論法則 \ref{dedmmp}によって
\[
\tag{18}
  {\rm Func}(f; a; b) \wedge {\rm Func}(g; b) \to {\rm Func}(g \circ f)
\]
が成り立つ.
また定理 \ref{sthmfuncbasis}より
\begin{align*}
  \tag{19}
  {\rm Func}(f; a; b) &\to {\rm pr}_{2}\langle f \rangle \subset b, \\
  \mbox{} \\
  \tag{20}
  {\rm Func}(g; b) &\to {\rm pr}_{1}\langle g \rangle = b
\end{align*}
が共に成り立つ.
また定理 \ref{sthm=tsubset}より
\[
  {\rm pr}_{1}\langle g \rangle = b \to b \subset {\rm pr}_{1}\langle g \rangle
\]
が成り立つ.
故にこれと(20)から, 推論法則 \ref{dedmmp}によって
\[
  {\rm Func}(g; b) \to b \subset {\rm pr}_{1}\langle g \rangle
\]
が成り立ち, これと(19)から, 推論法則 \ref{dedfromaddw}によって
\[
\tag{21}
  {\rm Func}(f; a; b) \wedge {\rm Func}(g; b) 
  \to {\rm pr}_{2}\langle f \rangle \subset b \wedge b \subset {\rm pr}_{1}\langle g \rangle
\]
が成り立つ.
また定理 \ref{sthmsubsettrans}より
\[
\tag{22}
  {\rm pr}_{2}\langle f \rangle \subset b \wedge b \subset {\rm pr}_{1}\langle g \rangle
  \to {\rm pr}_{2}\langle f \rangle \subset {\rm pr}_{1}\langle g \rangle
\]
が成り立つ.
また定理 \ref{sthmprsetcomp2}より
\[
\tag{23}
  {\rm pr}_{2}\langle f \rangle \subset {\rm pr}_{1}\langle g \rangle 
  \to {\rm pr}_{1}\langle g \circ f \rangle = {\rm pr}_{1}\langle f \rangle
\]
が成り立つ.
そこで(21), (22), (23)から, 推論法則 \ref{dedmmp}によって
\[
\tag{24}
  {\rm Func}(f; a; b) \wedge {\rm Func}(g; b) \to {\rm pr}_{1}\langle g \circ f \rangle = {\rm pr}_{1}\langle f \rangle
\]
が成り立つことがわかる.
またThm \ref{awbta}より
\[
  {\rm Func}(f; a; b) \wedge {\rm Func}(g; b) \to {\rm Func}(f; a; b)
\]
が成り立ち, 定理 \ref{sthmfuncbasis}より
\[
  {\rm Func}(f; a; b) \to {\rm pr}_{1}\langle f \rangle = a
\]
が成り立つから, 推論法則 \ref{dedmmp}によって
\[
  {\rm Func}(f; a; b) \wedge {\rm Func}(g; b) \to {\rm pr}_{1}\langle f \rangle = a
\]
が成り立つ.
故にこれと(24)から, 推論法則 \ref{dedprewedge}によって
\[
\tag{25}
  {\rm Func}(f; a; b) \wedge {\rm Func}(g; b) 
  \to {\rm pr}_{1}\langle g \circ f \rangle = {\rm pr}_{1}\langle f \rangle \wedge {\rm pr}_{1}\langle f \rangle = a
\]
が成り立つ.
またThm \ref{x=ywy=ztx=z}より
\[
\tag{26}
  {\rm pr}_{1}\langle g \circ f \rangle = {\rm pr}_{1}\langle f \rangle \wedge {\rm pr}_{1}\langle f \rangle = a 
  \to {\rm pr}_{1}\langle g \circ f \rangle = a
\]
が成り立つ.
そこで(25), (26)から, 推論法則 \ref{dedmmp}によって
\[
  {\rm Func}(f; a; b) \wedge {\rm Func}(g; b) \to {\rm pr}_{1}\langle g \circ f \rangle = a
\]
が成り立ち, これと(18)から, 推論法則 \ref{dedprewedge}によって
\[
  {\rm Func}(f; a; b) \wedge {\rm Func}(g; b) \to {\rm Func}(g \circ f) \wedge {\rm pr}_{1}\langle g \circ f \rangle = a, 
\]
即ち
\[
\tag{27}
  {\rm Func}(f; a; b) \wedge {\rm Func}(g; b) \to {\rm Func}(g \circ f; a)
\]
が成り立つ.
($**$)が成り立つことは, これと推論法則 \ref{dedmp}, \ref{dedwedge}によって明らかである.

\noindent
3)
定理 \ref{sthmfuncbasis}より${\rm Func}(g; b; c) \to {\rm Func}(g; b)$が成り立つから, 
推論法則 \ref{dedaddw}により
\[
  {\rm Func}(f; a; b) \wedge {\rm Func}(g; b; c) \to {\rm Func}(f; a; b) \wedge {\rm Func}(g; b)
\]
が成り立つ.
故にこれと(27)から, 推論法則 \ref{dedmmp}によって
\[
\tag{28}
  {\rm Func}(f; a; b) \wedge {\rm Func}(g; b; c) \to {\rm Func}(g \circ f; a)
\]
が成り立つ.
またThm \ref{awbta}より
\[
\tag{29}
  {\rm Func}(f; a; b) \wedge {\rm Func}(g; b; c) \to {\rm Func}(g; b; c)
\]
が成り立つ.
また定理 \ref{sthmfuncbasis}より
\[
\tag{30}
  {\rm Func}(g; b; c) \to {\rm pr}_{2}\langle g \rangle \subset c
\]
が成り立つ.
また定理 \ref{sthmprsetcomp2}より
\[
  {\rm pr}_{2}\langle g \circ f \rangle \subset {\rm pr}_{2}\langle g \rangle
\]
が成り立つから, 推論法則 \ref{dedatawbtrue2}により
\[
\tag{31}
  {\rm pr}_{2}\langle g \rangle \subset c 
  \to {\rm pr}_{2}\langle g \circ f \rangle \subset {\rm pr}_{2}\langle g \rangle \wedge {\rm pr}_{2}\langle g \rangle \subset c
\]
が成り立つ.
また定理 \ref{sthmsubsettrans}より
\[
\tag{32}
  {\rm pr}_{2}\langle g \circ f \rangle \subset {\rm pr}_{2}\langle g \rangle \wedge {\rm pr}_{2}\langle g \rangle \subset c 
  \to {\rm pr}_{2}\langle g \circ f \rangle \subset c
\]
が成り立つ.
そこで(29)---(32)から, 推論法則 \ref{dedmmp}によって
\[
  {\rm Func}(f; a; b) \wedge {\rm Func}(g; b; c) \to {\rm pr}_{2}\langle g \circ f \rangle \subset c
\]
が成り立つことがわかる.
故にこれと(28)から, 推論法則 \ref{dedprewedge}によって
\[
  {\rm Func}(f; a; b) \wedge {\rm Func}(g; b; c) \to {\rm Func}(g \circ f; a) \wedge {\rm pr}_{2}\langle g \circ f \rangle \subset c, 
\]
即ち
\[
  {\rm Func}(f; a; b) \wedge {\rm Func}(g; b; c) \to {\rm Func}(g \circ f; a; c)
\]
が成り立つ.
(${**}*$)が成り立つことは, これと推論法則 \ref{dedmp}, \ref{dedwedge}によって明らかである.
\halmos




\mathstrut
\begin{thm}
\label{sthmidenfunc}%定理
$a$を集合とするとき, ${\rm id}_{a}$は$a$から$a$への函数である.
故に${\rm id}_{a}$は$a$における函数であり, また${\rm id}_{a}$は函数である.
\end{thm}


\noindent{\bf 証明}
~$x$と$y$を, 互いに異なり, 共に$a$の中に自由変数として現れない, 定数でない文字とする.
このとき変数法則 \ref{validen}により, 
$x$と$y$は共に${\rm id}_{a}$の中に自由変数として現れない.
そして定理 \ref{sthmpairiniden}と推論法則 \ref{dedequiv}により
\[
  (x, y) \in {\rm id}_{a} \to x = y \wedge x \in a
\]
が成り立つから, 推論法則 \ref{dedprewedge}により
\[
\tag{1}
  (x, y) \in {\rm id}_{a} \to x = y
\]
が成り立つ.
またThm \ref{x=yty=x}より
\[
\tag{2}
  x = y \to y = x
\]
が成り立つ.
そこで(1), (2)から, 推論法則 \ref{dedmmp}によって
\[
  (x, y) \in {\rm id}_{a} \to y = x
\]
が成り立つ.
いま$y$は定数でなく, $x$と異なるから, 従って推論法則 \ref{ded!thmconst}により
\[
  !y((x, y) \in {\rm id}_{a})
\]
が成り立つ.
また$x$も定数でないので, これから推論法則 \ref{dedltthmquan}により
\[
\tag{3}
  \forall x(!y((x, y) \in {\rm id}_{a}))
\]
が成り立つ.
また定理 \ref{sthmidengraph}より, ${\rm id}_{a}$はグラフである.
故にこのことと(3)から, 推論法則 \ref{dedwedge}により
\[
  {\rm Graph}({\rm id}_{a}) \wedge \forall x(!y((x, y) \in {\rm id}_{a}))
\]
が成り立つ.
いま$x$と$y$は互いに異なり, 上述のように共に${\rm id}_{a}$の中に自由変数として現れないから, 
定義によれば, 上記の記号列は
\[
\tag{4}
  {\rm Func}({\rm id}_{a})
\]
と同じである.
故にこれが定理となる.
即ち, ${\rm id}_{a}$は函数である.
また定理 \ref{sthmprsetiden}より
\begin{align*}
  \tag{5}
  {\rm pr}_{1}\langle {\rm id}_{a} \rangle &= a, \\
  \mbox{} \\
  \tag{6}
  {\rm pr}_{2}\langle {\rm id}_{a} \rangle &= a
\end{align*}
が共に成り立つ.
そこで(4), (5)から, 推論法則 \ref{dedwedge}により
\[
  {\rm Func}({\rm id}_{a}) \wedge {\rm pr}_{1}\langle {\rm id}_{a} \rangle = a, 
\]
即ち
\[
\tag{7}
  {\rm Func}({\rm id}_{a}; a)
\]
が成り立つ.
即ち, ${\rm id}_{a}$は$a$における函数である.
また(6)から, 定理 \ref{sthm=tsubset}により
\[
  {\rm pr}_{2}\langle {\rm id}_{a} \rangle \subset a
\]
が成り立つ.
そこでこれと(7)から, 推論法則 \ref{dedwedge}により
\[
  {\rm Func}({\rm id}_{a}; a) \wedge {\rm pr}_{2}\langle {\rm id}_{a} \rangle \subset a, 
\]
即ち
\[
  {\rm Func}({\rm id}_{a}; a; a)
\]
が成り立つ.
即ち, ${\rm id}_{a}$は$a$から$a$への函数である.
\halmos




\mathstrut
$a$を集合とするとき, 上記の定理 \ref{sthmidenfunc}より, ${\rm id}_{a}$は$a$から$a$への函数である.
そこで${\rm id}_{a}$のことを, \textbf{${\bm a}$の(${\bm a}$上の, ${\bm a}$における)恒等函数(恒等写像)}
という.




\mathstrut
\begin{thm}
\label{sthmcompidenfunc}%定理
\mbox{}

1)
$a$, $b$, $c$, $f$を集合とするとき, 
\[
  {\rm Func}(f; a) \to (a \subset c \leftrightarrow f \circ {\rm id}_{c} = f), ~~
  {\rm Func}(f; a; b) \to (a \subset c \leftrightarrow f \circ {\rm id}_{c} = f)
\]
が成り立つ.
またこれらから, 次の${(*)}_{1}$, ${(*)}_{2}$が成り立つ: 

${(*)}_{1}$ ~~$f$が$a$における函数ならば, 
              \[
                a \subset c \leftrightarrow f \circ {\rm id}_{c} = f
              \]
              が成り立つ.
              故にこのとき, $a \subset c$が成り立つならば$f \circ {\rm id}_{c} = f$が成り立ち, 
              逆に$f \circ {\rm id}_{c} = f$が成り立つならば$a \subset c$が成り立つ.

${(*)}_{2}$ ~~$f$が$a$から$b$への函数ならば, 
              \[
                a \subset c \leftrightarrow f \circ {\rm id}_{c} = f
              \]
              が成り立つ.
              故にこのとき, $a \subset c$が成り立つならば$f \circ {\rm id}_{c} = f$が成り立ち, 
              逆に$f \circ {\rm id}_{c} = f$が成り立つならば$a \subset c$が成り立つ.

2)
$a$, $b$, $f$は1)と同じとするとき, 
\[
  {\rm Func}(f; a) \to f \circ {\rm id}_{a} = f, ~~
  {\rm Func}(f; a; b) \to f \circ {\rm id}_{a} = f
\]
が成り立つ.
またこれらから, 次の${(**)}_{1}$, ${(**)}_{2}$が成り立つ: 

${(**)}_{1}$ ~~$f$が$a$における函数ならば, $f \circ {\rm id}_{a} = f$が成り立つ.

${(**)}_{2}$ ~~$f$が$a$から$b$への函数ならば, $f \circ {\rm id}_{a} = f$が成り立つ.

3)
$a$, $b$, $f$は1)及び2)と同じとするとき, 
\[
  {\rm Func}(f; a; b) \to {\rm id}_{b} \circ f = f
\]
が成り立つ.
またこのことから, 次の(${**}*$)が成り立つ: 

(${**}*$) ~~$f$が$a$から$b$への函数ならば, ${\rm id}_{b} \circ f = f$が成り立つ.
\end{thm}%書き写しok


\noindent{\bf 証明}
~1)
定理 \ref{sthmfuncbasis}より
\begin{align*}
  {\rm Func}(f; a) &\to {\rm Graph}(f), \\
  \mbox{} \\
  \tag{1}
  {\rm Func}(f; a) &\to {\rm pr}_{1}\langle f \rangle = a
\end{align*}
が共に成り立つから, 推論法則 \ref{dedprewedge}により
\[
  {\rm Func}(f; a) \to {\rm Graph}(f) \wedge {\rm pr}_{1}\langle f \rangle = a
\]
が成り立つ.
故に推論法則 \ref{dedaddw}により
\[
\tag{2}
  {\rm Func}(f; a) \wedge a \subset c 
  \to ({\rm Graph}(f) \wedge {\rm pr}_{1}\langle f \rangle = a) \wedge a \subset c
\]
が成り立つ.
またThm \ref{1awb1wctaw1bwc1}より
\[
\tag{3}
  ({\rm Graph}(f) \wedge {\rm pr}_{1}\langle f \rangle = a) \wedge a \subset c 
  \to {\rm Graph}(f) \wedge ({\rm pr}_{1}\langle f \rangle = a \wedge a \subset c)
\]
が成り立つ.
また定理 \ref{sthm=&subset}より
\[
  {\rm pr}_{1}\langle f \rangle = a \wedge a \subset c \to {\rm pr}_{1}\langle f \rangle \subset c
\]
が成り立つから, 推論法則 \ref{dedaddw}により
\[
\tag{4}
  {\rm Graph}(f) \wedge ({\rm pr}_{1}\langle f \rangle = a \wedge a \subset c) 
  \to {\rm Graph}(f) \wedge {\rm pr}_{1}\langle f \rangle \subset c
\]
が成り立つ.
また定理 \ref{sthmidencomp3}と推論法則 \ref{dedequiv}により
\begin{align*}
  \tag{5}
  &{\rm Graph}(f) \wedge {\rm pr}_{1}\langle f \rangle \subset c \to f \circ {\rm id}_{c} = f, \\
  \mbox{} \\
  \tag{6}
  &f \circ {\rm id}_{c} = f \to {\rm Graph}(f) \wedge {\rm pr}_{1}\langle f \rangle \subset c
\end{align*}
が共に成り立つ.
そこで(2)---(5)から, 推論法則 \ref{dedmmp}によって
\[
  {\rm Func}(f; a) \wedge a \subset c \to f \circ {\rm id}_{c} = f
\]
が成り立つことがわかる.
故に推論法則 \ref{dedtwch}により
\[
\tag{7}
  {\rm Func}(f; a) \to (a \subset c \to f \circ {\rm id}_{c} = f)
\]
が成り立つ.
また(6)から, 推論法則 \ref{dedprewedge}により
\[
  f \circ {\rm id}_{c} = f \to {\rm pr}_{1}\langle f \rangle \subset c
\]
が成り立つ.
故にこれと(1)から, 推論法則 \ref{dedfromaddw}により
\[
\tag{8}
  {\rm Func}(f; a) \wedge f \circ {\rm id}_{c} = f 
  \to {\rm pr}_{1}\langle f \rangle = a \wedge {\rm pr}_{1}\langle f \rangle \subset c
\]
が成り立つ.
また定理 \ref{sthm=&subset}より
\[
\tag{9}
  {\rm pr}_{1}\langle f \rangle = a \wedge {\rm pr}_{1}\langle f \rangle \subset c \to a \subset c
\]
が成り立つ.
そこで(8), (9)から, 推論法則 \ref{dedmmp}によって
\[
  {\rm Func}(f; a) \wedge f \circ {\rm id}_{c} = f \to a \subset c
\]
が成り立つ.
故に推論法則 \ref{dedtwch}により
\[
  {\rm Func}(f; a) \to (f \circ {\rm id}_{c} = f \to a \subset c)
\]
が成り立つ.
そこでこれと(7)から, 推論法則 \ref{dedprewedge}により
\[
  {\rm Func}(f; a) 
  \to (a \subset c \to f \circ {\rm id}_{c} = f) \wedge (f \circ {\rm id}_{c} = f \to a \subset c), 
\]
即ち
\[
\tag{10}
  {\rm Func}(f; a) \to (a \subset c \leftrightarrow f \circ {\rm id}_{c} = f)
\]
が成り立つ.
${(*)}_{1}$が成り立つことは, これと推論法則 \ref{dedmp}, \ref{dedeqfund}によって明らかである.
また定理 \ref{sthmfuncbasis}より
\[
\tag{11}
  {\rm Func}(f; a; b) \to {\rm Func}(f; a)
\]
が成り立つから, これと(10)から, 推論法則 \ref{dedmmp}によって
\[
  {\rm Func}(f; a; b) \to (a \subset c \leftrightarrow f \circ {\rm id}_{c} = f)
\]
が成り立つ.
${(*)}_{2}$が成り立つことは, これと推論法則 \ref{dedmp}, \ref{dedeqfund}によって明らかである.

\noindent
2)
1)の記号の下で(7)が成り立つから, そこで$c$を$a$に置き換えた
\[
  {\rm Func}(f; a) \to (a \subset a \to f \circ {\rm id}_{a} = f)
\]
も成り立つ.
故に推論法則 \ref{dedch}により
\[
\tag{12}
  a \subset a \to ({\rm Func}(f; a) \to f \circ {\rm id}_{a} = f)
\]
が成り立つ.
いま定理 \ref{sthmsubsetself}より$a \subset a$が成り立つから, 
従ってこれと(12)から, 推論法則 \ref{dedmp}によって
\[
\tag{13}
  {\rm Func}(f; a) \to f \circ {\rm id}_{a} = f
\]
が成り立つ.
${(**)}_{1}$が成り立つことは, これと推論法則 \ref{dedmp}によって明らかである.
また(11), (13)から, 推論法則 \ref{dedmmp}によって
\[
  {\rm Func}(f; a; b) \to f \circ {\rm id}_{a} = f
\]
が成り立つ.
${(**)}_{2}$が成り立つことは, これと推論法則 \ref{dedmp}によって明らかである.

\noindent
3)
定理 \ref{sthmfuncbasis}より
\[
  {\rm Func}(f; a; b) \to {\rm Graph}(f), ~~
  {\rm Func}(f; a; b) \to {\rm pr}_{2}\langle f \rangle \subset b
\]
が共に成り立つから, 推論法則 \ref{dedprewedge}により
\[
\tag{14}
  {\rm Func}(f; a; b) \to {\rm Graph}(f) \wedge {\rm pr}_{2}\langle f \rangle \subset b
\]
が成り立つ.
また定理 \ref{sthmidencomp3}と推論法則 \ref{dedequiv}により
\[
\tag{15}
  {\rm Graph}(f) \wedge {\rm pr}_{2}\langle f \rangle \subset b \to {\rm id}_{b} \circ f = f
\]
が成り立つ.
そこで(14), (15)から, 推論法則 \ref{dedmmp}によって
\[
  {\rm Func}(f; a; b) \to {\rm id}_{b} \circ f = f
\]
が成り立つ.
(${**}*$)が成り立つことは, これと推論法則 \ref{dedmp}によって明らかである.
\halmos




\mathstrut
\begin{defo}
\label{functionvalue}%変形
$f$と$t$を記号列とし, $x$と$y$を共にこれらの中に自由変数として現れない文字とする.
このとき
\[
  \tau_{x}((t, x) \in f) \equiv \tau_{y}((t, y) \in f)
\]
が成り立つ.
\end{defo}


\noindent{\bf 証明}
~$x$と$y$が同じ文字であるときは明らかに成り立つから, 
$x$と$y$が異なる文字である場合を考える.
このとき変数法則 \ref{valfund}, \ref{valpair}によってわかるように, 
$y$は$(t, x) \in f$の中に自由変数として現れないから, 
代入法則 \ref{substtautrans}により
\[
  \tau_{x}((t, x) \in f) \equiv \tau_{y}((y|x)((t, x) \in f))
\]
が成り立つ.
また$x$が$f$及び$t$の中に自由変数として現れないことから, 
代入法則 \ref{substfree}, \ref{substfund}, \ref{substpair}により
\[
  (y|x)((t, x) \in f) \equiv (t, y) \in f
\]
が成り立つ.
これらのことから, $\tau_{x}((t, x) \in f)$と$\tau_{y}((t, y) \in f)$が
同一の記号列となることがわかる.
\halmos




\mathstrut
\begin{defi}
\label{deffuncval}%定義
$f$と$t$を記号列とし, $x$と$y$を共にこれらの中に自由変数として現れない文字とする.
このとき上記の変形法則 \ref{functionvalue}によれば, 
$\tau_{x}((t, x) \in f)$と$\tau_{y}((t, y) \in f)$という二つの記号列は一致する.
$f$と$t$に対して定まるこの記号列を, $(f)(t)$と書き表す.
誤解のおそれがなければ, これを$f(t)$と書くことが多い.
また文脈によっては, これを$f_{t}$と書くこともある.
\end{defi}




\mathstrut
\begin{valu}
\label{valfuncval}%変数
$f$と$t$を記号列とし, $x$を文字とする.
$x$が$f$及び$t$の中に自由変数として現れなければ, 
$x$は$f(t)$の中に自由変数として現れない.
\end{valu}


\noindent{\bf 証明}
~このとき定義から, $f(t)$は$\tau_{x}((t, x) \in f)$と同じである.
変数法則 \ref{valtau}によれば, $x$はこの中に自由変数として現れない.
\halmos




\mathstrut
\begin{subs}
\label{substfuncval}%代入
$f$, $t$, $u$を記号列とし, $x$を文字とする.
このとき
\[
  (u|x)(f(t)) \equiv (u|x)(f)((u|x)(t))
\]
が成り立つ.
\end{subs}


\noindent{\bf 証明}
~$y$を$x$と異なり, $f$, $t$, $u$のいずれの記号列の中にも自由変数として現れない文字とする.
このとき定義から, $f(t)$は$\tau_{y}((t, y) \in f)$と同じである.
このことと, $y$が$x$と異なり, $u$の中に自由変数として現れないことから, 
代入法則 \ref{substtau}により
\[
  (u|x)(f(t)) \equiv \tau_{y}((u|x)((t, y) \in f))
\]
が成り立つ.
また$x$が$y$と異なることと代入法則 \ref{substfund}, \ref{substpair}により
\[
  (u|x)((t, y) \in f) \equiv ((u|x)(t), y) \in (u|x)(f)
\]
が成り立つ.
故に$(u|x)(f(t))$は
\[
  \tau_{y}(((u|x)(t), y) \in (u|x)(f))
\]
と一致する.
いま$y$は$f$, $t$, $u$のいずれの記号列の中にも自由変数として現れないから, 
変数法則 \ref{valsubst}により, $y$は$(u|x)(t)$及び$(u|x)(f)$の中に自由変数として現れない.
そこで定義から, 上記の記号列は$(u|x)(f)((u|x)(t))$と同じである.
故に本法則が成り立つ.
\halmos




\mathstrut
\begin{form}
\label{formfuncval}%構成
$f$と$t$が集合ならば, $f(t)$は集合である.
\end{form}


\noindent{\bf 証明}
~$x$を$f$及び$t$の中に自由変数として現れない文字とすれば, 
定義から$f(t)$は$\tau_{x}((t, x) \in f)$と同じである.
$f$と$t$が共に集合であるとき, 
構成法則 \ref{formfund}, \ref{formpair}によって直ちにわかるように, 
これは集合である.
\halmos




\mathstrut
$f$と$t$を集合とするとき, 
集合$f(t)$を, $f$の$t$における(あるいは, $t$に対する)\textbf{値}という.




\mathstrut
\begin{thm}
\label{sthmfuncval}%定理
\mbox{}

1)
$f$と$t$を集合とするとき, 
\[
  t \in {\rm pr}_{1}\langle f \rangle \leftrightarrow (t, f(t)) \in f
\]
が成り立つ.
またこのことから, 次の($*$)が成り立つ: 

($*$) ~~$t \in {\rm pr}_{1}\langle f \rangle$が成り立つならば, $(t, f(t)) \in f$が成り立つ.
        逆に$(t, f(t)) \in f$が成り立つならば, $t \in {\rm pr}_{1}\langle f \rangle$が成り立つ.

2)
$a$, $f$, $t$を集合とするとき, 
\[
  {\rm Func}(f; a) \to (t \in a \leftrightarrow (t, f(t)) \in f)
\]
が成り立つ.
またこのことから, 次の($**$)が成り立つ: 

($**$) ~~$f$が$a$における函数ならば, 
         \[
           t \in a \leftrightarrow (t, f(t)) \in f
         \]
         が成り立つ.
         故にこのとき, $t \in a$が成り立つならば$(t, f(t)) \in f$が成り立ち, 
         逆に$(t, f(t)) \in f$が成り立つならば$t \in a$が成り立つ.

3)
$a$, $b$, $f$, $t$を集合とするとき, 
\[
  {\rm Func}(f; a; b) \to (t \in a \leftrightarrow (t, f(t)) \in f)
\]
が成り立つ.
またこのことから, 次の(${**}*$)が成り立つ: 

(${**}*$) ~~$f$が$a$から$b$への函数ならば, 
            \[
              t \in a \leftrightarrow (t, f(t)) \in f
            \]
            が成り立つ.
            故にこのとき, $t \in a$が成り立つならば$(t, f(t)) \in f$が成り立ち, 
            逆に$(t, f(t)) \in f$が成り立つならば$t \in a$が成り立つ.
\end{thm}


\noindent{\bf 証明}
~1)
$y$を$f$及び$t$の中に自由変数として現れない文字とすれば, 定理 \ref{sthmprsetelement}より
\[
  t \in {\rm pr}_{1}\langle f \rangle \leftrightarrow \exists y((t, y) \in f)
\]
が成り立つが, 定義からこの記号列は
\[
  t \in {\rm pr}_{1}\langle f \rangle \leftrightarrow (\tau_{y}((t, y) \in f)|y)((t, y) \in f)
\]
と一致し, 再び定義からこの記号列は
\[
  t \in {\rm pr}_{1}\langle f \rangle \leftrightarrow (f(t)|y)((t, y) \in f)
\]
と一致し, 更に代入法則 \ref{substfree}, \ref{substfund}, \ref{substpair}により, 
この記号列は
\[
  t \in {\rm pr}_{1}\langle f \rangle \leftrightarrow (t, f(t)) \in f
\]
と一致する.
故にこれが定理となる.
($*$)が成り立つことは, これと推論法則 \ref{dedeqfund}によって明らかである.

\noindent
2)
定理 \ref{sthmfuncbasis}より
\[
\tag{1}
  {\rm Func}(f; a) \to {\rm pr}_{1}\langle f \rangle = a
\]
が成り立つ.
また定理 \ref{sthm=tineq}より
\[
\tag{2}
  {\rm pr}_{1}\langle f \rangle = a \to (t \in {\rm pr}_{1}\langle f \rangle \leftrightarrow t \in a)
\]
が成り立つ.
またThm \ref{1alb1t1bla1}より
\[
\tag{3}
  (t \in {\rm pr}_{1}\langle f \rangle \leftrightarrow t \in a) 
  \to (t \in a \leftrightarrow t \in {\rm pr}_{1}\langle f \rangle)
\]
が成り立つ.
また1)より$t \in {\rm pr}_{1}\langle f \rangle \leftrightarrow (t, f(t)) \in f$が成り立つから, 
推論法則 \ref{dedatawbtrue2}により
\[
\tag{4}
  (t \in a \leftrightarrow t \in {\rm pr}_{1}\langle f \rangle) 
  \to (t \in a \leftrightarrow t \in {\rm pr}_{1}\langle f \rangle) 
  \wedge (t \in {\rm pr}_{1}\langle f \rangle \leftrightarrow (t, f(t)) \in f)
\]
が成り立つ.
またThm \ref{1alb1w1blc1t1alc1}より
\[
\tag{5}
  (t \in a \leftrightarrow t \in {\rm pr}_{1}\langle f \rangle) 
  \wedge (t \in {\rm pr}_{1}\langle f \rangle \leftrightarrow (t, f(t)) \in f) 
  \to (t \in a \leftrightarrow (t, f(t)) \in f)
\]
が成り立つ.
そこで(1)---(5)から, 推論法則 \ref{dedmmp}によって
\[
\tag{6}
  {\rm Func}(f; a) \to (t \in a \leftrightarrow (t, f(t)) \in f)
\]
が成り立つことがわかる.
($**$)が成り立つことは, これと推論法則 \ref{dedmp}, \ref{dedeqfund}によって明らかである.

\noindent
3)
定理 \ref{sthmfuncbasis}より${\rm Func}(f; a; b) \to {\rm Func}(f; a)$が成り立つから, 
これと(6)から, 推論法則 \ref{dedmmp}によって
\[
  {\rm Func}(f; a; b) \to (t \in a \leftrightarrow (t, f(t)) \in f)
\]
が成り立つ.
(${**}*$)が成り立つことは, これと推論法則 \ref{dedmp}, \ref{dedeqfund}によって明らかである.
\halmos




\mathstrut
\begin{thm}
\label{sthmfuncvalbasisprepare}%定理
$f$, $t$, $u$を集合とするとき, 
\[
  (t, u) \in f \to (t, f(t)) \in f
\]
が成り立つ.
またこのことから, 次の($*$)が成り立つ: 

($*$) ~~$(t, u) \in f$が成り立つならば, $(t, f(t)) \in f$が成り立つ.
\end{thm}


\noindent{\bf 証明}
~定理 \ref{sthmpairelementinprset}より
\[
  (t, u) \in f \to t \in {\rm pr}_{1}\langle f \rangle \wedge u \in {\rm pr}_{2}\langle f \rangle
\]
が成り立つから, 推論法則 \ref{dedprewedge}により
\[
\tag{1}
  (t, u) \in f \to t \in {\rm pr}_{1}\langle f \rangle
\]
が成り立つ.
また定理 \ref{sthmfuncval}と推論法則 \ref{dedequiv}により
\[
\tag{2}
  t \in {\rm pr}_{1}\langle f \rangle \to (t, f(t)) \in f
\]
が成り立つ.
そこで(1), (2)から, 推論法則 \ref{dedmmp}によって
\[
  (t, u) \in f \to (t, f(t)) \in f
\]
が成り立つ.
($*$)が成り立つことは, これと推論法則 \ref{dedmp}によって明らかである.
\halmos




\mathstrut
\begin{thm}
\label{sthmfuncvalbasis}%定理
\mbox{} 

1)
$f$, $t$, $u$を集合とするとき, 
\[
  {\rm Func}(f) \to ((t, u) \in f \leftrightarrow t \in {\rm pr}_{1}\langle f \rangle \wedge u = f(t))
\]
が成り立つ.
またこのことから, 次の($*$)が成り立つ: 

($*$) ~~$f$が函数ならば, 
        \[
          (t, u) \in f \leftrightarrow t \in {\rm pr}_{1}\langle f \rangle \wedge u = f(t)
        \]
        が成り立つ.
        故にこのとき, $(t, u) \in f$が成り立つならば, $t \in {\rm pr}_{1}\langle f \rangle$と
        $u = f(t)$が共に成り立ち, 
        逆に$t \in {\rm pr}_{1}\langle f \rangle$と$u = f(t)$が共に成り立つならば, 
        $(t, u) \in f$が成り立つ.

2)
$a$, $f$, $t$, $u$を集合とするとき, 
\[
  {\rm Func}(f; a) \to ((t, u) \in f \leftrightarrow t \in a \wedge u = f(t))
\]
が成り立つ.
またこのことから, 次の($**$)が成り立つ: 

($**$) ~~$f$が$a$における函数ならば, 
         \[
           (t, u) \in f \leftrightarrow t \in a \wedge u = f(t)
         \]
         が成り立つ.
         故にこのとき, $(t, u) \in f$が成り立つならば, $t \in a$と$u = f(t)$が共に成り立ち, 
         逆に$t \in a$と$u = f(t)$が共に成り立つならば, $(t, u) \in f$が成り立つ.

3)
$a$, $b$, $f$, $t$, $u$を集合とするとき, 
\[
  {\rm Func}(f; a; b) \to ((t, u) \in f \leftrightarrow t \in a \wedge u = f(t))
\]
が成り立つ.
またこのことから, 次の(${**}*$)が成り立つ: 

(${**}*$) ~~$f$が$a$から$b$への函数ならば, 
            \[
              (t, u) \in f \leftrightarrow t \in a \wedge u = f(t)
            \]
            が成り立つ.
            故にこのとき, $(t, u) \in f$が成り立つならば, $t \in a$と$u = f(t)$が共に成り立ち, 
            逆に$t \in a$と$u = f(t)$が共に成り立つならば, $(t, u) \in f$が成り立つ.
\end{thm}


\noindent{\bf 証明}
~1)
Thm \ref{awbta}より
\[
\tag{1}
  {\rm Func}(f) \wedge (t, u) \in f \to (t, u) \in f
\]
が成り立つ.
また定理 \ref{sthmpairelementinprset}より
\[
  (t, u) \in f \to t \in {\rm pr}_{1}\langle f \rangle \wedge u \in {\rm pr}_{2}\langle f \rangle
\]
が成り立つから, 推論法則 \ref{dedprewedge}により
\[
\tag{2}
  (t, u) \in f \to t \in {\rm pr}_{1}\langle f \rangle
\]
が成り立つ.
そこで(1), (2)から, 推論法則 \ref{dedmmp}によって
\[
\tag{3}
  {\rm Func}(f) \wedge (t, u) \in f \to t \in {\rm pr}_{1}\langle f \rangle
\]
が成り立つ.
また定理 \ref{sthmfuncvalbasisprepare}より$(t, u) \in f \to (t, f(t)) \in f$が成り立つから, 
推論法則 \ref{dedatawbtrue1}により
\[
  (t, u) \in f \to (t, u) \in f \wedge (t, f(t)) \in f
\]
が成り立ち, これから推論法則 \ref{dedaddw}により
\[
\tag{4}
  {\rm Func}(f) \wedge (t, u) \in f \to {\rm Func}(f) \wedge ((t, u) \in f \wedge (t, f(t)) \in f)
\]
が成り立つ.
また定理 \ref{sthmpairinfunc}より
\[
  {\rm Func}(f) \to ((t, u) \in f \wedge (t, f(t)) \in f \to u = f(t))
\]
が成り立つから, 推論法則 \ref{dedtwch}により
\[
\tag{5}
  {\rm Func}(f) \wedge ((t, u) \in f \wedge (t, f(t)) \in f) \to u = f(t)
\]
が成り立つ.
そこで(4), (5)から, 推論法則 \ref{dedmmp}によって
\[
  {\rm Func}(f) \wedge (t, u) \in f \to u = f(t)
\]
が成り立つ.
故にこれと(3)から, 推論法則 \ref{dedprewedge}により
\[
  {\rm Func}(f) \wedge (t, u) \in f \to t \in {\rm pr}_{1}\langle f \rangle \wedge u = f(t)
\]
が成り立ち, これから推論法則 \ref{dedtwch}により
\[
\tag{6}
  {\rm Func}(f) \to ((t, u) \in f \to t \in {\rm pr}_{1}\langle f \rangle \wedge u = f(t))
\]
が成り立つ.
また定理 \ref{sthmfuncval}と推論法則 \ref{dedequiv}により
\[
  t \in {\rm pr}_{1}\langle f \rangle \to (t, f(t)) \in f
\]
が成り立つから, 推論法則 \ref{dedaddw}により
\[
\tag{7}
  t \in {\rm pr}_{1}\langle f \rangle \wedge u = f(t) \to (t, f(t)) \in f \wedge u = f(t)
\]
が成り立つ.
また定理 \ref{sthmpairweak}と推論法則 \ref{dedequiv}により
\[
\tag{8}
  u = f(t) \to (t, u) = (t, f(t))
\]
が成り立つ.
また定理 \ref{sthm=tineq}より
\[
  (t, u) = (t, f(t)) \to ((t, u) \in f \leftrightarrow (t, f(t)) \in f)
\]
が成り立つから, 推論法則 \ref{dedprewedge}により
\[
\tag{9}
  (t, u) = (t, f(t)) \to ((t, f(t)) \in f \to (t, u) \in f)
\]
が成り立つ.
そこで(8), (9)から, 推論法則 \ref{dedmmp}によって
\[
  u = f(t) \to ((t, f(t)) \in f \to (t, u) \in f)
\]
が成り立つ.
故に推論法則 \ref{dedch}により
\[
  (t, f(t)) \in f \to (u = f(t) \to (t, u) \in f)
\]
が成り立ち, これから推論法則 \ref{dedtwch}により
\[
\tag{10}
  (t, f(t)) \in f \wedge u = f(t) \to (t, u) \in f
\]
が成り立つ.
そこで(7), (10)から, 推論法則 \ref{dedmmp}によって
\[
  t \in {\rm pr}_{1}\langle f \rangle \wedge u = f(t) \to (t, u) \in f
\]
が成り立つ.
故に推論法則 \ref{dedatawbtrue2}により
\[
\tag{11}
  ((t, u) \in f \to t \in {\rm pr}_{1}\langle f \rangle \wedge u = f(t)) 
  \to ((t, u) \in f \leftrightarrow t \in {\rm pr}_{1}\langle f \rangle \wedge u = f(t))
\]
が成り立つ.
そこで(6), (11)から, 推論法則 \ref{dedmmp}によって
\[
\tag{12}
  {\rm Func}(f) \to ((t, u) \in f \leftrightarrow t \in {\rm pr}_{1}\langle f \rangle \wedge u = f(t))
\]
が成り立つ.
($*$)が成り立つことは, これと推論法則 \ref{dedmp}, \ref{dedwedge}, \ref{dedeqfund}によって明らかである.

\noindent
2)
定理 \ref{sthm=tineq}より
\[
\tag{13}
  {\rm pr}_{1}\langle f \rangle = a \to (t \in {\rm pr}_{1}\langle f \rangle \leftrightarrow t \in a)
\]
が成り立つ.
またThm \ref{1alb1t1awclbwc1}より
\[
\tag{14}
  (t \in {\rm pr}_{1}\langle f \rangle \leftrightarrow t \in a) 
  \to (t \in {\rm pr}_{1}\langle f \rangle \wedge u = f(t) \leftrightarrow t \in a \wedge u = f(t))
\]
が成り立つ.
そこで(13), (14)から, 推論法則 \ref{dedmmp}によって
\[
  {\rm pr}_{1}\langle f \rangle = a \to (t \in {\rm pr}_{1}\langle f \rangle \wedge u = f(t) \leftrightarrow t \in a \wedge u = f(t))
\]
が成り立つ.
故にこれと(12)から, 推論法則 \ref{dedfromaddw}により
\begin{multline*}
  {\rm Func}(f) \wedge {\rm pr}_{1}\langle f \rangle = a \\
  \to ((t, u) \in f \leftrightarrow t \in {\rm pr}_{1}\langle f \rangle \wedge u = f(t)) 
  \wedge (t \in {\rm pr}_{1}\langle f \rangle \wedge u = f(t) \leftrightarrow t \in a \wedge u = f(t)), 
\end{multline*}
即ち
\[
\tag{15}
  {\rm Func}(f; a) 
  \to ((t, u) \in f \leftrightarrow t \in {\rm pr}_{1}\langle f \rangle \wedge u = f(t)) 
  \wedge (t \in {\rm pr}_{1}\langle f \rangle \wedge u = f(t) \leftrightarrow t \in a \wedge u = f(t))
\]
が成り立つ.
またThm \ref{1alb1w1blc1t1alc1}より
\begin{multline*}
\tag{16}
  ((t, u) \in f \leftrightarrow t \in {\rm pr}_{1}\langle f \rangle \wedge u = f(t)) 
  \wedge (t \in {\rm pr}_{1}\langle f \rangle \wedge u = f(t) \leftrightarrow t \in a \wedge u = f(t)) \\
  \to ((t, u) \in f \leftrightarrow t \in a \wedge u = f(t))
\end{multline*}
が成り立つ.
そこで(15), (16)から, 推論法則 \ref{dedmmp}によって
\[
\tag{17}
  {\rm Func}(f; a) \to ((t, u) \in f \leftrightarrow t \in a \wedge u = f(t))
\]
が成り立つ.
($**$)が成り立つことは, これと推論法則 \ref{dedmp}, \ref{dedwedge}, \ref{dedeqfund}によって明らかである.

\noindent
3)
定理 \ref{sthmfuncbasis}より${\rm Func}(f; a; b) \to {\rm Func}(f; a)$が成り立つから, 
これと(17)から, 推論法則 \ref{dedmmp}によって
\[
  {\rm Func}(f; a; b) \to ((t, u) \in f \leftrightarrow t \in a \wedge u = f(t))
\]
が成り立つ.
(${**}*$)が成り立つことは, これと推論法則 \ref{dedmp}, \ref{dedwedge}, \ref{dedeqfund}によって明らかである.
\halmos




\mathstrut
次の定理 \ref{sthmfuncvalbasispractical}は, 上記の定理 \ref{sthmfuncvalbasis}から直ちに得られる.
以下のいくつかの定理の証明においては, こちらを主に引用する.




\mathstrut
\begin{thm}
\label{sthmfuncvalbasispractical}%定理
\mbox{}

1)
$f$, $t$, $u$を集合とするとき, 
\[
  {\rm Func}(f) \to ((t, u) \in f \to u = f(t))
\]
が成り立つ.
またこのことから, 次の($*$)が成り立つ: 

($*$) ~~$f$が函数ならば, $(t, u) \in f \to u = f(t)$が成り立つ.

2)
$a$, $f$, $t$, $u$を集合とするとき, 
\[
  {\rm Func}(f; a) \to ((t, u) \in f \to u = f(t))
\]
が成り立つ.
またこのことから, 次の($**$)が成り立つ: 

($**$) ~~$f$が$a$における函数ならば, $(t, u) \in f \to u = f(t)$が成り立つ.

3)
$a$, $b$, $f$, $t$, $u$を集合とするとき, 
\[
  {\rm Func}(f; a; b) \to ((t, u) \in f \to u = f(t))
\]
が成り立つ.
またこのことから, 次の(${**}*$)が成り立つ: 

(${**}*$) ~~$f$が$a$から$b$への函数ならば, $(t, u) \in f \to u = f(t)$が成り立つ.
\end{thm}


\noindent{\bf 証明}
~1)
定理 \ref{sthmfuncvalbasis}より
\[
  {\rm Func}(f) \to ((t, u) \in f \leftrightarrow t \in {\rm pr}_{1}\langle f \rangle \wedge u = f(t))
\]
が成り立つから, 推論法則 \ref{dedprewedge}により
\[
\tag{1}
  {\rm Func}(f) \to ((t, u) \in f \to t \in {\rm pr}_{1}\langle f \rangle \wedge u = f(t))
\]
が成り立つ.
またThm \ref{awbta}より
\[
  t \in {\rm pr}_{1}\langle f \rangle \wedge u = f(t) \to u = f(t)
\]
が成り立つから, 推論法則 \ref{dedaddb}により
\[
\tag{2}
  ((t, u) \in f \to t \in {\rm pr}_{1}\langle f \rangle \wedge u = f(t)) \to ((t, u) \in f \to u = f(t))
\]
が成り立つ.
そこで(1), (2)から, 推論法則 \ref{dedmmp}によって
\[
\tag{3}
  {\rm Func}(f) \to ((t, u) \in f \to u = f(t))
\]
が成り立つ.
($*$)が成り立つことは, これと推論法則 \ref{dedmp}によって明らかである.

\noindent
2)
定理 \ref{sthmfuncbasis}より${\rm Func}(f; a) \to {\rm Func}(f)$が成り立つから, 
これと(3)から, 推論法則 \ref{dedmmp}によって
\[
  {\rm Func}(f; a) \to ((t, u) \in f \to u = f(t))
\]
が成り立つ.
($**$)が成り立つことは, これと推論法則 \ref{dedmp}によって明らかである.

\noindent
3)
定理 \ref{sthmfuncbasis}より${\rm Func}(f; a; b) \to {\rm Func}(f)$が成り立つから, 
これと(3)から, 推論法則 \ref{dedmmp}によって
\[
  {\rm Func}(f; a; b) \to ((t, u) \in f \to u = f(t))
\]
が成り立つ.
(${**}*$)が成り立つことは, これと推論法則 \ref{dedmp}によって明らかである.
\halmos




\mathstrut
\begin{thm}
\label{sthmfuncvalinprset}%定理
\mbox{}

1)
$f$と$t$を集合とするとき, 
\[
  t \in {\rm pr}_{1}\langle f \rangle \to f(t) \in {\rm pr}_{2}\langle f \rangle
\]
が成り立つ.
またこのことから, 次の($*$)が成り立つ: 

($*$) ~~$t \in {\rm pr}_{1}\langle f \rangle$が成り立つならば, 
        $f(t) \in {\rm pr}_{2}\langle f \rangle$が成り立つ.

2)
$a$, $f$, $t$を集合とするとき, 
\[
  {\rm Func}(f; a) \to (t \in a \to f(t) \in {\rm pr}_{2}\langle f \rangle)
\]
が成り立つ.
またこのことから, 次の($**$)が成り立つ: 

($**$) ~~$f$が$a$における函数ならば, 
         \[
           t \in a \to f(t) \in {\rm pr}_{2}\langle f \rangle
         \]
         が成り立つ.
         故にこのとき, $t \in a$が成り立つならば, $f(t) \in {\rm pr}_{2}\langle f \rangle$が成り立つ.

3)
$a$, $b$, $f$, $t$を集合とするとき, 
\[
  {\rm Func}(f; a; b) \to (t \in a \to f(t) \in b)
\]
が成り立つ.
またこのことから, 次の(${**}*$)が成り立つ: 

(${**}*$) ~~$f$が$a$から$b$への函数ならば, 
            \[
              t \in a \to f(t) \in b
            \]
            が成り立つ.
            故にこのとき, $t \in a$が成り立つならば, $f(t) \in b$が成り立つ.
\end{thm}


\noindent{\bf 証明}
~1)
定理 \ref{sthmfuncval}と推論法則 \ref{dedequiv}により
\[
\tag{1}
  t \in {\rm pr}_{1}\langle f \rangle \to (t, f(t)) \in f
\]
が成り立つ.
また定理 \ref{sthmpairelementinprset}より
\[
  (t, f(t)) \in f \to t \in {\rm pr}_{1}\langle f \rangle \wedge f(t) \in {\rm pr}_{2}\langle f \rangle
\]
が成り立つから, 推論法則 \ref{dedprewedge}により
\[
\tag{2}
  (t, f(t)) \in f \to f(t) \in {\rm pr}_{2}\langle f \rangle
\]
が成り立つ.
そこで(1), (2)から, 推論法則 \ref{dedmmp}によって
\[
\tag{3}
  t \in {\rm pr}_{1}\langle f \rangle \to f(t) \in {\rm pr}_{2}\langle f \rangle
\]
が成り立つ.
($*$)が成り立つことは, これと推論法則 \ref{dedmp}によって明らかである.

\noindent
2)
定理 \ref{sthmfuncbasis}より
\[
\tag{4}
  {\rm Func}(f; a) \to {\rm pr}_{1}\langle f \rangle = a
\]
が成り立つ.
また定理 \ref{sthm=tineq}より
\[
  {\rm pr}_{1}\langle f \rangle = a \to (t \in {\rm pr}_{1}\langle f \rangle \leftrightarrow t \in a)
\]
が成り立つから, 推論法則 \ref{dedprewedge}により
\[
\tag{5}
  {\rm pr}_{1}\langle f \rangle = a \to (t \in a \to t \in {\rm pr}_{1}\langle f \rangle)
\]
が成り立つ.
また(3)から, 推論法則 \ref{dedaddb}により
\[
\tag{6}
  (t \in a \to t \in {\rm pr}_{1}\langle f \rangle) \to (t \in a \to f(t) \in {\rm pr}_{2}\langle f \rangle)
\]
が成り立つ.
そこで(4), (5), (6)から, 推論法則 \ref{dedmmp}によって
\[
\tag{7}
  {\rm Func}(f; a) \to (t \in a \to f(t) \in {\rm pr}_{2}\langle f \rangle)
\]
が成り立つことがわかる.
($**$)が成り立つことは, これと推論法則 \ref{dedmp}によって明らかである.

\noindent
3)
定理 \ref{sthmsubsetbasis}より
\[
  {\rm pr}_{2}\langle f \rangle \subset b \to (f(t) \in {\rm pr}_{2}\langle f \rangle \to f(t) \in b)
\]
が成り立つから, これと(7)から, 推論法則 \ref{dedfromaddw}により
\[
  {\rm Func}(f; a) \wedge {\rm pr}_{2}\langle f \rangle \subset b 
  \to (t \in a \to f(t) \in {\rm pr}_{2}\langle f \rangle) \wedge (f(t) \in {\rm pr}_{2}\langle f \rangle \to f(t) \in b), 
\]
即ち
\[
\tag{8}
  {\rm Func}(f; a; b)
  \to (t \in a \to f(t) \in {\rm pr}_{2}\langle f \rangle) \wedge (f(t) \in {\rm pr}_{2}\langle f \rangle \to f(t) \in b)
\]
が成り立つ.
またThm \ref{1atb1t11btc1t1atc11}より
\[
  (t \in a \to f(t) \in {\rm pr}_{2}\langle f \rangle) 
  \to ((f(t) \in {\rm pr}_{2}\langle f \rangle \to f(t) \in b) \to (t \in a \to f(t) \in b))
\]
が成り立つから, 推論法則 \ref{dedtwch}により
\[
\tag{9}
  (t \in a \to f(t) \in {\rm pr}_{2}\langle f \rangle) \wedge (f(t) \in {\rm pr}_{2}\langle f \rangle \to f(t) \in b) 
  \to (t \in a \to f(t) \in b)
\]
が成り立つ.
そこで(8), (9)から, 推論法則 \ref{dedmmp}によって
\[
  {\rm Func}(f; a; b) \to (t \in a \to f(t) \in b)
\]
が成り立つ.
(${**}*$)が成り立つことは, これと推論法則 \ref{dedmp}によって明らかである.
\halmos




\mathstrut
\begin{thm}
\label{sthmfuncval=}%定理
\mbox{}

1)
$f$, $g$, $t$を集合とするとき, 
\[
  f = g \to f(t) = g(t)
\]
が成り立つ.
またこのことから, 次の($*$)が成り立つ: 

($*$) ~~$f = g$が成り立つならば, $f(t) = g(t)$が成り立つ.

2)
$f$, $t$, $u$を集合とするとき, 
\[
  t = u \to f(t) = f(u)
\]
が成り立つ.
またこのことから, 次の($**$)が成り立つ: 

($**$) ~~$t = u$が成り立つならば, $f(t) = f(u)$が成り立つ.
\end{thm}


\noindent{\bf 証明}
~1)
$x$を$t$の中に自由変数として現れない文字とするとき, 
Thm \ref{T=Ut1TV=UV1}より
\[
  f = g \to (f|x)(x(t)) = (g|x)(x(t))
\]
が成り立つが, 代入法則 \ref{substfree}, \ref{substfuncval}によればこの記号列は
\[
  f = g \to f(t) = g(t)
\]
と一致するから, これが定理となる.
($*$)が成り立つことは, これと推論法則 \ref{dedmp}によって明らかである.

\noindent
2)
$y$を$f$の中に自由変数として現れない文字とするとき, 
Thm \ref{T=Ut1TV=UV1}より
\[
  t = u \to (t|y)(f(y)) = (u|y)(f(y))
\]
が成り立つが, 代入法則 \ref{substfree}, \ref{substfuncval}によればこの記号列は
\[
  t = u \to f(t) = f(u)
\]
と一致するから, これが定理となる.
($**$)が成り立つことは, これと推論法則 \ref{dedmp}によって明らかである.
\halmos




\mathstrut
\begin{thm}
\label{sthmfuncfuncvalsubset}%定理
\mbox{} 

1)
$f$と$g$を集合とし, $x$をこれらの中に自由変数として現れない文字とするとき, 
\[
  f \subset g \to \forall x(x \in {\rm pr}_{1}\langle f \rangle \to (x, f(x)) \in g)
\]
が成り立つ.
またこのことから, 次の($*$)が成り立つ: 

($*$) ~~$f \subset g$が成り立つならば, 
        $\forall x(x \in {\rm pr}_{1}\langle f \rangle \to (x, f(x)) \in g)$が成り立つ.

2)
$f$, $g$, $x$は1)と同じとするとき, 
\[
  {\rm Func}(f) \to (f \subset g \leftrightarrow \forall x(x \in {\rm pr}_{1}\langle f \rangle \to (x, f(x)) \in g))
\]
が成り立つ.
またこのことから, 次の${(**)}_{1}$, ${(**)}_{2}$が成り立つ: 

${(**)}_{1}$ ~~$f$が函数ならば, 
               \[
                 f \subset g \leftrightarrow \forall x(x \in {\rm pr}_{1}\langle f \rangle \to (x, f(x)) \in g)
               \]
               が成り立つ.
               そこで特にこのとき, $\forall x(x \in {\rm pr}_{1}\langle f \rangle \to (x, f(x)) \in g)$が成り立つならば, 
               $f \subset g$が成り立つ.

${(**)}_{2}$ ~~$f$が函数であるとき, $x$が定数でなく, $x \in {\rm pr}_{1}\langle f \rangle \to (x, f(x)) \in g$が成り立つならば, 
               $f \subset g$が成り立つ.

3)
$a$, $f$, $g$を集合とし, $x$をこれらの中に自由変数として現れない文字とするとき, 
\[
  {\rm Func}(f; a) \to (f \subset g \leftrightarrow \forall x(x \in a \to (x, f(x)) \in g))
\]
が成り立つ.
またこのことから, 次の${({**}*)}_{1}$, ${({**}*)}_{2}$が成り立つ: 

${({**}*)}_{1}$ ~~$f$が$a$における函数ならば, 
                  \[
                    f \subset g \leftrightarrow \forall x(x \in a \to (x, f(x)) \in g)
                  \]
                  が成り立つ.
                  そこで特にこのとき, $\forall x(x \in a \to (x, f(x)) \in g)$が成り立つならば, 
                  $f \subset g$が成り立つ.

${({**}*)}_{2}$ ~~$f$が$a$における函数であるとき, $x$が定数でなく, $x \in a \to (x, f(x)) \in g$が成り立つならば, 
                  $f \subset g$が成り立つ.

4)
$a$, $f$, $g$, $x$は3)と同じとし, 更に$b$を集合とするとき, 
\[
  {\rm Func}(f; a; b) \to (f \subset g \leftrightarrow \forall x(x \in a \to (x, f(x)) \in g))
\]
が成り立つ.
またこのことから, 次の${({**}{**})}_{1}$, ${({**}{**})}_{2}$が成り立つ: 

${({**}{**})}_{1}$ ~~$f$が$a$から$b$への函数ならば, 
                     \[
                       f \subset g \leftrightarrow \forall x(x \in a \to (x, f(x)) \in g)
                     \]
                     が成り立つ.
                     そこで特にこのとき, $\forall x(x \in a \to (x, f(x)) \in g)$が成り立つならば, 
                     $f \subset g$が成り立つ.

${({**}{**})}_{2}$ ~~$f$が$a$から$b$への函数であるとき, $x$が定数でなく, $x \in a \to (x, f(x)) \in g$が成り立つならば, 
                     $f \subset g$が成り立つ.
\end{thm}


\noindent{\bf 証明}
~1)
$\tau_{x}(\neg (x \in {\rm pr}_{1}\langle f \rangle \to (x, f(x)) \in g))$を$T$と書けば, 
$T$は集合であり, 定理 \ref{sthmsubsetbasis}より
\[
\tag{1}
  f \subset g \to ((T, f(T)) \in f \to (T, f(T)) \in g)
\]
が成り立つ.
また定理 \ref{sthmfuncval}と推論法則 \ref{dedequiv}により
\[
  T \in {\rm pr}_{1}\langle f \rangle \to (T, f(T)) \in f
\]
が成り立つから, 推論法則 \ref{dedaddf}により
\[
\tag{2}
  ((T, f(T)) \in f \to (T, f(T)) \in g) \to (T \in {\rm pr}_{1}\langle f \rangle \to (T, f(T)) \in g)
\]
が成り立つ.
また$T$の定義から, Thm \ref{thmallfund1}と推論法則 \ref{dedequiv}により
\[
  (T|x)(x \in {\rm pr}_{1}\langle f \rangle \to (x, f(x)) \in g) 
  \to \forall x(x \in {\rm pr}_{1}\langle f \rangle \to (x, f(x)) \in g)
\]
が成り立つ.
いま$x$は$f$の中に自由変数として現れないから, 変数法則 \ref{valprset}により, 
$x$は${\rm pr}_{1}\langle f \rangle$の中にも自由変数として現れない.
また$x$は$g$の中にも自由変数として現れない.
故に代入法則 \ref{substfree}, \ref{substfund}, \ref{substpair}, \ref{substfuncval}によれば, 
上記の記号列は
\[
\tag{3}
  (T \in {\rm pr}_{1}\langle f \rangle \to (T, f(T)) \in g) 
  \to \forall x(x \in {\rm pr}_{1}\langle f \rangle \to (x, f(x)) \in g)
\]
と一致する.
従ってこれが定理となる.
そこで(1), (2), (3)から, 推論法則 \ref{dedmmp}によって
\[
\tag{4}
  f \subset g \to \forall x(x \in {\rm pr}_{1}\langle f \rangle \to (x, f(x)) \in g)
\]
が成り立つことがわかる.
($*$)が成り立つことは, これと推論法則 \ref{dedmp}によって明らかである.

\noindent
2)
$u$と$v$を, 互いに異なり, 共に$x$と異なり, $f$及び$g$の中に自由変数として現れない, 定数でない文字とする.
このときThm \ref{awbtbwa}より
\[
  {\rm Func}(f) \wedge \forall x(x \in {\rm pr}_{1}\langle f \rangle \to (x, f(x)) \in g) 
  \to \forall x(x \in {\rm pr}_{1}\langle f \rangle \to (x, f(x)) \in g) \wedge {\rm Func}(f)
\]
が成り立つから, 推論法則 \ref{dedaddw}により
\begin{multline*}
\tag{5}
  ({\rm Func}(f) \wedge \forall x(x \in {\rm pr}_{1}\langle f \rangle \to (x, f(x)) \in g)) \wedge (u, v) \in f \\
  \to (\forall x(x \in {\rm pr}_{1}\langle f \rangle \to (x, f(x)) \in g) \wedge {\rm Func}(f)) \wedge (u, v) \in f
\end{multline*}
が成り立つ.
またThm \ref{1awb1wctaw1bwc1}より
\begin{multline*}
\tag{6}
  (\forall x(x \in {\rm pr}_{1}\langle f \rangle \to (x, f(x)) \in g) \wedge {\rm Func}(f)) \wedge (u, v) \in f \\
  \to \forall x(x \in {\rm pr}_{1}\langle f \rangle \to (x, f(x)) \in g) \wedge ({\rm Func}(f) \wedge (u, v) \in f)
\end{multline*}
が成り立つ.
また定理 \ref{sthmfuncvalbasis}より
\[
  {\rm Func}(f) \to ((u, v) \in f \leftrightarrow u \in {\rm pr}_{1}\langle f \rangle \wedge v = f(u))
\]
が成り立つから, 推論法則 \ref{dedprewedge}により
\[
  {\rm Func}(f) \to ((u, v) \in f \to u \in {\rm pr}_{1}\langle f \rangle \wedge v = f(u))
\]
が成り立ち, これから推論法則 \ref{dedtwch}により
\[
  {\rm Func}(f) \wedge (u, v) \in f \to u \in {\rm pr}_{1}\langle f \rangle \wedge v = f(u)
\]
が成り立つ.
故に推論法則 \ref{dedaddw}により
\begin{multline*}
\tag{7}
  \forall x(x \in {\rm pr}_{1}\langle f \rangle \to (x, f(x)) \in g) \wedge ({\rm Func}(f) \wedge (u, v) \in f) \\
  \to \forall x(x \in {\rm pr}_{1}\langle f \rangle \to (x, f(x)) \in g) \wedge (u \in {\rm pr}_{1}\langle f \rangle \wedge v = f(u))
\end{multline*}
が成り立つ.
またThm \ref{thmallfund2}より
\[
  \forall x(x \in {\rm pr}_{1}\langle f \rangle \to (x, f(x)) \in g) 
  \to (u|x)(x \in {\rm pr}_{1}\langle f \rangle \to (x, f(x)) \in g)
\]
が成り立つが, $x$は$f$及び$g$の中に自由変数として現れず, 
既に述べたように${\rm pr}_{1}\langle f \rangle$の中にも自由変数として現れないから, 
代入法則 \ref{substfree}, \ref{substfund}, \ref{substpair}, \ref{substfuncval}により, 
この記号列は
\[
\tag{8}
  \forall x(x \in {\rm pr}_{1}\langle f \rangle \to (x, f(x)) \in g) 
  \to (u \in {\rm pr}_{1}\langle f \rangle \to (u, f(u)) \in g)
\]
と一致する.
従ってこれが定理となる.
またThm \ref{1atb1t1awctbwc1}より
\[
\tag{9}
  (u \in {\rm pr}_{1}\langle f \rangle \to (u, f(u)) \in g) 
  \to (u \in {\rm pr}_{1}\langle f \rangle \wedge v = f(u) \to (u, f(u)) \in g \wedge v = f(u))
\]
が成り立つ.
そこで(8), (9)から, 推論法則 \ref{dedmmp}によって
\[
  \forall x(x \in {\rm pr}_{1}\langle f \rangle \to (x, f(x)) \in g) 
  \to (u \in {\rm pr}_{1}\langle f \rangle \wedge v = f(u) \to (u, f(u)) \in g \wedge v = f(u))
\]
が成り立つ.
故に推論法則 \ref{dedtwch}により
\[
\tag{10}
  \forall x(x \in {\rm pr}_{1}\langle f \rangle \to (x, f(x)) \in g) \wedge (u \in {\rm pr}_{1}\langle f \rangle \wedge v = f(u)) 
  \to (u, f(u)) \in g \wedge v = f(u)
\]
が成り立つ.
また定理 \ref{sthmpairweak}と推論法則 \ref{dedequiv}により
\[
  v = f(u) \to (u, v) = (u, f(u))
\]
が成り立つから, 推論法則 \ref{dedaddw}により
\[
\tag{11}
  (u, f(u)) \in g \wedge v = f(u) \to (u, f(u)) \in g \wedge (u, v) = (u, f(u))
\]
が成り立つ.
またThm \ref{awbtbwa}より
\[
\tag{12}
  (u, f(u)) \in g \wedge (u, v) = (u, f(u)) \to (u, v) = (u, f(u)) \wedge (u, f(u)) \in g
\]
が成り立つ.
また定理 \ref{sthm=&in}より
\[
\tag{13}
  (u, v) = (u, f(u)) \wedge (u, f(u)) \in g \to (u, v) \in g
\]
が成り立つ.
そこで(5)---(7), (10)---(13)から, 推論法則 \ref{dedmmp}によって
\[
  ({\rm Func}(f) \wedge \forall x(x \in {\rm pr}_{1}\langle f \rangle \to (x, f(x)) \in g)) \wedge (u, v) \in f \to (u, v) \in g
\]
が成り立つことがわかる.
故に推論法則 \ref{dedtwch}により
\[
\tag{14}
  {\rm Func}(f) \wedge \forall x(x \in {\rm pr}_{1}\langle f \rangle \to (x, f(x)) \in g) \to ((u, v) \in f \to (u, v) \in g)
\]
が成り立つ.
さていま$u$と$v$は共に$x$と異なり, $f$及び$g$の中に自由変数として現れないから, 
変数法則 \ref{valfund}, \ref{valwedge}, \ref{valquan}, \ref{valpair}, \ref{valprset}, \ref{valfunc}, \ref{valfuncval}によって
わかるように, $u$と$v$は共に${\rm Func}(f) \wedge \forall x(x \in {\rm pr}_{1}\langle f \rangle \to (x, f(x)) \in g)$の中に
自由変数として現れない.
また$u$と$v$は共に定数でない.
これらのことと, (14)が成り立つことから, 推論法則 \ref{dedalltquansepfreeconst}によって
\[
\tag{15}
  {\rm Func}(f) \wedge \forall x(x \in {\rm pr}_{1}\langle f \rangle \to (x, f(x)) \in g) 
  \to \forall u(\forall v((u, v) \in f \to (u, v) \in g))
\]
が成り立つことがわかる.
またThm \ref{awbta}より
\[
  {\rm Func}(f) \wedge \forall x(x \in {\rm pr}_{1}\langle f \rangle \to (x, f(x)) \in g) \to {\rm Func}(f)
\]
が成り立ち, 定理 \ref{sthmfuncbasis}より
\[
  {\rm Func}(f) \to {\rm Graph}(f)
\]
が成り立つから, 推論法則 \ref{dedmmp}によって
\[
  {\rm Func}(f) \wedge \forall x(x \in {\rm pr}_{1}\langle f \rangle \to (x, f(x)) \in g) \to {\rm Graph}(f)
\]
が成り立つ.
故にこれと(15)から, 推論法則 \ref{dedprewedge}により
\[
\tag{16}
  {\rm Func}(f) \wedge \forall x(x \in {\rm pr}_{1}\langle f \rangle \to (x, f(x)) \in g) 
  \to {\rm Graph}(f) \wedge \forall u(\forall v((u, v) \in f \to (u, v) \in g))
\]
が成り立つ.
また$u$と$v$は互いに異なり, 共に$f$及び$g$の中に自由変数として現れないから, 
定理 \ref{sthmgraphpairsubset}より
\[
  {\rm Graph}(f) \to (f \subset g \leftrightarrow \forall u(\forall v((u, v) \in f \to (u, v) \in g)))
\]
が成り立つ.
故に推論法則 \ref{dedprewedge}により
\[
  {\rm Graph}(f) \to (\forall u(\forall v((u, v) \in f \to (u, v) \in g)) \to f \subset g)
\]
が成り立ち, これから推論法則 \ref{dedtwch}により
\[
\tag{17}
  {\rm Graph}(f) \wedge \forall u(\forall v((u, v) \in f \to (u, v) \in g)) \to f \subset g
\]
が成り立つ.
そこで(16), (17)から, 推論法則 \ref{dedmmp}によって
\[
  {\rm Func}(f) \wedge \forall x(x \in {\rm pr}_{1}\langle f \rangle \to (x, f(x)) \in g) \to f \subset g
\]
が成り立ち, これから推論法則 \ref{dedtwch}により
\[
\tag{18}
  {\rm Func}(f) \to (\forall x(x \in {\rm pr}_{1}\langle f \rangle \to (x, f(x)) \in g) \to f \subset g)
\]
が成り立つ.
また(4)が成り立つことから, 推論法則 \ref{dedatawbtrue2}により
\[
\tag{19}
  (\forall x(x \in {\rm pr}_{1}\langle f \rangle \to (x, f(x)) \in g) \to f \subset g) 
  \to (f \subset g \leftrightarrow \forall x(x \in {\rm pr}_{1}\langle f \rangle \to (x, f(x)) \in g))
\]
が成り立つ.
そこで(18), (19)から, 推論法則 \ref{dedmmp}によって
\[
\tag{20}
  {\rm Func}(f) \to (f \subset g \leftrightarrow \forall x(x \in {\rm pr}_{1}\langle f \rangle \to (x, f(x)) \in g))
\]
が成り立つ.
${(**)}_{1}$が成り立つことは, これと推論法則 \ref{dedmp}, \ref{dedeqfund}によって明らかである.
また${(**)}_{2}$が成り立つことは, 
これと推論法則 \ref{dedmp}, \ref{dedeqfund}, \ref{dedltthmquan}によって明らかである.

\noindent
3)
$\tau_{x}(\neg ((x \in {\rm pr}_{1}\langle f \rangle \to (x, f(x)) \in g) \leftrightarrow (x \in a \to (x, f(x)) \in g)))$を
$U$と書けば, $U$は集合であり, 定理 \ref{sthm=tineq}より
\[
\tag{21}
  {\rm pr}_{1}\langle f \rangle = a \to (U \in {\rm pr}_{1}\langle f \rangle \leftrightarrow U \in a)
\]
が成り立つ.
またThm \ref{1alb1t11atc1l1btc11}より
\[
\tag{22}
  (U \in {\rm pr}_{1}\langle f \rangle \leftrightarrow U \in a) 
  \to ((U \in {\rm pr}_{1}\langle f \rangle \to (U, f(U)) \in g) \leftrightarrow (U \in a \to (U, f(U)) \in g))
\]
が成り立つ.
また$U$の定義から, Thm \ref{thmallfund1}と推論法則 \ref{dedequiv}により
\begin{multline*}
  (U|x)((x \in {\rm pr}_{1}\langle f \rangle \to (x, f(x)) \in g) \leftrightarrow (x \in a \to (x, f(x)) \in g)) \\
  \to \forall x((x \in {\rm pr}_{1}\langle f \rangle \to (x, f(x)) \in g) \leftrightarrow (x \in a \to (x, f(x)) \in g))
\end{multline*}
が成り立つ.
いま$x$は$f$の中に自由変数として現れないから, 変数法則 \ref{valprset}により, 
$x$は${\rm pr}_{1}\langle f \rangle$の中にも自由変数として現れない.
また$x$は$a$及び$g$の中にも自由変数として現れない.
そこで代入法則 \ref{substfree}, \ref{substfund}, \ref{substequiv}, \ref{substpair}, \ref{substfuncval}により, 
上記の記号列は
\begin{multline*}
\tag{23}
  ((U \in {\rm pr}_{1}\langle f \rangle \to (U, f(U)) \in g) \leftrightarrow (U \in a \to (U, f(U)) \in g)) \\
  \to \forall x((x \in {\rm pr}_{1}\langle f \rangle \to (x, f(x)) \in g) \leftrightarrow (x \in a \to (x, f(x)) \in g))
\end{multline*}
と一致する.
故にこれが定理となる.
またThm \ref{thmalleqallsep}より
\begin{multline*}
\tag{24}
  \forall x((x \in {\rm pr}_{1}\langle f \rangle \to (x, f(x)) \in g) \leftrightarrow (x \in a \to (x, f(x)) \in g)) \\
  \to (\forall x(x \in {\rm pr}_{1}\langle f \rangle \to (x, f(x)) \in g) \leftrightarrow \forall x(x \in a \to (x, f(x)) \in g))
\end{multline*}
が成り立つ.
そこで(21)---(24)から, 推論法則 \ref{dedmmp}によって
\[
\tag{25}
  {\rm pr}_{1}\langle f \rangle = a 
  \to (\forall x(x \in {\rm pr}_{1}\langle f \rangle \to (x, f(x)) \in g) \leftrightarrow \forall x(x \in a \to (x, f(x)) \in g))
\]
が成り立つことがわかる.
また$x$は$f$及び$g$の中に自由変数として現れないから, 2)の証明において示したように(20)が成り立つ.
故に(20), (25)から, 推論法則 \ref{dedfromaddw}により
\begin{multline*}
  {\rm Func}(f) \wedge {\rm pr}_{1}\langle f \rangle = a 
  \to (f \subset g \leftrightarrow \forall x(x \in {\rm pr}_{1}\langle f \rangle \to (x, f(x)) \in g)) \\
  \wedge (\forall x(x \in {\rm pr}_{1}\langle f \rangle \to (x, f(x)) \in g) \leftrightarrow \forall x(x \in a \to (x, f(x)) \in g)), 
\end{multline*}
即ち
\begin{multline*}
\tag{26}
  {\rm Func}(f; a) 
  \to (f \subset g \leftrightarrow \forall x(x \in {\rm pr}_{1}\langle f \rangle \to (x, f(x)) \in g)) \\
  \wedge (\forall x(x \in {\rm pr}_{1}\langle f \rangle \to (x, f(x)) \in g) \leftrightarrow \forall x(x \in a \to (x, f(x)) \in g))
\end{multline*}
が成り立つ.
またThm \ref{1alb1w1blc1t1alc1}より
\begin{multline*}
\tag{27}
  (f \subset g \leftrightarrow \forall x(x \in {\rm pr}_{1}\langle f \rangle \to (x, f(x)) \in g)) \\
  \wedge (\forall x(x \in {\rm pr}_{1}\langle f \rangle \to (x, f(x)) \in g) \leftrightarrow \forall x(x \in a \to (x, f(x)) \in g)) \\
  \to (f \subset g \leftrightarrow \forall x(x \in a \to (x, f(x)) \in g))
\end{multline*}
が成り立つ.
そこで(26), (27)から, 推論法則 \ref{dedmmp}によって
\[
\tag{28}
  {\rm Func}(f; a) \to (f \subset g \leftrightarrow \forall x(x \in a \to (x, f(x)) \in g))
\]
が成り立つ.
${({**}*)}_{1}$が成り立つことは, これと推論法則 \ref{dedmp}, \ref{dedeqfund}によって明らかである.
また${({**}*)}_{2}$が成り立つことは, 
これと推論法則 \ref{dedmp}, \ref{dedeqfund}, \ref{dedltthmquan}によって明らかである.

\noindent
4)
定理 \ref{sthmfuncbasis}より${\rm Func}(f; a; b) \to {\rm Func}(f; a)$が成り立つから, 
これと(28)から, 推論法則 \ref{dedmmp}によって
\[
  {\rm Func}(f; a; b) \to (f \subset g \leftrightarrow \forall x(x \in a \to (x, f(x)) \in g))
\]
が成り立つ.
${({**}{**})}_{1}$が成り立つことは, これと推論法則 \ref{dedmp}, \ref{dedeqfund}によって明らかである.
また${({**}{**})}_{2}$が成り立つことは, 
これと推論法則 \ref{dedmp}, \ref{dedeqfund}, \ref{dedltthmquan}によって明らかである.
\halmos




\mathstrut
\begin{thm}
\label{sthmfuncfuncval=}%定理
\mbox{} 

1)
$f$と$g$を集合とし, $x$をこれらの中に自由変数として現れない文字とするとき, 
\[
  f = g \to {\rm pr}_{1}\langle f \rangle = {\rm pr}_{1}\langle g \rangle \wedge \forall x(x \in {\rm pr}_{1}\langle f \rangle \to f(x) = g(x))
\]
が成り立つ.
またこのことから, 特に次の($*$)が成り立つ: 

($*$) ~~$f = g$が成り立つならば, 
        $\forall x(x \in {\rm pr}_{1}\langle f \rangle \to f(x) = g(x))$が成り立つ.

2)
$f$, $g$, $x$は1)と同じとするとき, 
\[
  {\rm Func}(f) \wedge {\rm Func}(g) 
  \to (f = g 
  \leftrightarrow {\rm pr}_{1}\langle f \rangle = {\rm pr}_{1}\langle g \rangle 
  \wedge \forall x(x \in {\rm pr}_{1}\langle f \rangle \to f(x) = g(x)))
\]
が成り立つ.
またこのことから, 次の${(**)}_{1}$, ${(**)}_{2}$が成り立つ: 

${(**)}_{1}$ ~~$f$と$g$が共に函数ならば, 
               \[
                 f = g 
                 \leftrightarrow {\rm pr}_{1}\langle f \rangle = {\rm pr}_{1}\langle g \rangle 
                 \wedge \forall x(x \in {\rm pr}_{1}\langle f \rangle \to f(x) = g(x))
               \]
               が成り立つ.
               そこで特にこのとき, ${\rm pr}_{1}\langle f \rangle = {\rm pr}_{1}\langle g \rangle$と
               $\forall x(x \in {\rm pr}_{1}\langle f \rangle \to f(x) = g(x))$が共に成り立つならば, 
               $f = g$が成り立つ.

${(**)}_{2}$ ~~$f$と$g$が共に函数であるとき, $x$が定数でなく, ${\rm pr}_{1}\langle f \rangle = {\rm pr}_{1}\langle g \rangle$と
               $x \in {\rm pr}_{1}\langle f \rangle \to f(x) = g(x)$が共に成り立つならば, 
               $f = g$が成り立つ.

3)
$a$, $b$, $f$, $g$を集合とし, $x$を$a$, $f$, $g$の中に自由変数として現れない文字とするとき, 
\[
  {\rm Func}(f; a) \wedge {\rm Func}(g; b) \to (f = g \leftrightarrow a = b \wedge \forall x(x \in a \to f(x) = g(x)))
\]
が成り立つ.
またこのことから, 次の${({**}*)}_{1}$, ${({**}*)}_{2}$が成り立つ: 

${({**}*)}_{1}$ ~~$f$が$a$における函数であり, $g$が$b$における函数ならば, 
                  \[
                    f = g \leftrightarrow a = b \wedge \forall x(x \in a \to f(x) = g(x))
                  \]
                  が成り立つ.
                  そこで特にこのとき, $a = b$と$\forall x(x \in a \to f(x) = g(x))$が共に成り立つならば, 
                  $f = g$が成り立つ.

${({**}*)}_{2}$ ~~$f$が$a$における函数であり, $g$が$b$における函数であるとき, 
                  $x$が定数でなく, $a = b$と$x \in a \to f(x) = g(x)$が共に成り立つならば, 
                  $f = g$が成り立つ.

4)
$a$, $b$, $c$, $d$, $f$, $g$を集合とし, $x$を$a$, $f$, $g$の中に自由変数として現れない文字とするとき, 
\[
  {\rm Func}(f; a; b) \wedge {\rm Func}(g; c; d) \to (f = g \leftrightarrow a = c \wedge \forall x(x \in a \to f(x) = g(x)))
\]
が成り立つ.
またこのことから, 次の${({**}{**})}_{1}$, ${({**}{**})}_{2}$が成り立つ: 

${({**}{**})}_{1}$ ~~$f$が$a$から$b$への函数であり, $g$が$c$から$d$への函数ならば, 
                     \[
                       f = g \leftrightarrow a = c \wedge \forall x(x \in a \to f(x) = g(x))
                     \]
                     が成り立つ.
                     そこで特にこのとき, $a = c$と$\forall x(x \in a \to f(x) = g(x))$が共に成り立つならば, 
                     $f = g$が成り立つ.

${({**}{**})}_{2}$ ~~$f$が$a$から$b$への函数であり, $g$が$c$から$d$への函数であるとき, 
                     $x$が定数でなく, $a = c$と$x \in a \to f(x) = g(x)$が共に成り立つならば, 
                     $f = g$が成り立つ.
\end{thm}


\noindent{\bf 証明}
~1)
定理 \ref{sthmprset=}より
\[
\tag{1}
  f = g \to {\rm pr}_{1}\langle f \rangle = {\rm pr}_{1}\langle g \rangle
\]
が成り立つ.
また$\tau_{x}(\neg (x \in {\rm pr}_{1}\langle f \rangle \to f(x) = g(x)))$を$T$と書けば, 
$T$は集合であり, 定理 \ref{sthmfuncval=}より
\[
\tag{2}
  f = g \to f(T) = g(T)
\]
が成り立つ.
またschema S1の適用により, 
\[
\tag{3}
  f(T) = g(T) \to (T \in {\rm pr}_{1}\langle f \rangle \to f(T) = g(T))
\]
が成り立つ.
また$T$の定義から, Thm \ref{thmallfund1}と推論法則 \ref{dedequiv}により
\[
  (T|x)(x \in {\rm pr}_{1}\langle f \rangle \to f(x) = g(x)) 
  \to \forall x(x \in {\rm pr}_{1}\langle f \rangle \to f(x) = g(x))
\]
が成り立つ.
いま$x$は$f$の中に自由変数として現れないから, 変数法則 \ref{valprset}により, 
$x$は${\rm pr}_{1}\langle f \rangle$の中にも自由変数として現れない.
また$x$は$g$の中にも自由変数として現れない.
故に代入法則 \ref{substfree}, \ref{substfund}, \ref{substfuncval}によれば, 
上記の記号列は
\[
\tag{4}
  (T \in {\rm pr}_{1}\langle f \rangle \to f(T) = g(T)) 
  \to \forall x(x \in {\rm pr}_{1}\langle f \rangle \to f(x) = g(x))
\]
と一致する.
従ってこれが定理となる.
そこで(2), (3), (4)から, 推論法則 \ref{dedmmp}によって
\[
  f = g \to \forall x(x \in {\rm pr}_{1}\langle f \rangle \to f(x) = g(x))
\]
が成り立つことがわかる.
故にこれと(1)から, 推論法則 \ref{dedprewedge}により
\[
\tag{5}
  f = g 
  \to {\rm pr}_{1}\langle f \rangle = {\rm pr}_{1}\langle g \rangle 
  \wedge \forall x(x \in {\rm pr}_{1}\langle f \rangle \to f(x) = g(x))
\]
が成り立つ.
($*$)が成り立つことは, これと推論法則 \ref{dedmp}, \ref{dedwedge}によって明らかである.

\noindent
2)
$\tau_{x}(\neg (x \in {\rm pr}_{1}\langle f \rangle \to (x, f(x)) \in g))$を$U$と書けば, 
$U$は集合であり, Thm \ref{1awb1wctaw1bwc1}より
\begin{multline*}
  ({\rm pr}_{1}\langle f \rangle = {\rm pr}_{1}\langle g \rangle 
  \wedge \forall x(x \in {\rm pr}_{1}\langle f \rangle \to f(x) = g(x))) 
  \wedge U \in {\rm pr}_{1}\langle f \rangle \\
  \to {\rm pr}_{1}\langle f \rangle = {\rm pr}_{1}\langle g \rangle 
  \wedge (\forall x(x \in {\rm pr}_{1}\langle f \rangle \to f(x) = g(x)) 
  \wedge U \in {\rm pr}_{1}\langle f \rangle)
\end{multline*}
が成り立つ.
故に推論法則 \ref{dedprewedge}により
\[
\tag{6}
  ({\rm pr}_{1}\langle f \rangle = {\rm pr}_{1}\langle g \rangle 
  \wedge \forall x(x \in {\rm pr}_{1}\langle f \rangle \to f(x) = g(x))) 
  \wedge U \in {\rm pr}_{1}\langle f \rangle 
  \to {\rm pr}_{1}\langle f \rangle = {\rm pr}_{1}\langle g \rangle, 
\]
\begin{multline*}
\tag{7}
  ({\rm pr}_{1}\langle f \rangle = {\rm pr}_{1}\langle g \rangle 
  \wedge \forall x(x \in {\rm pr}_{1}\langle f \rangle \to f(x) = g(x))) 
  \wedge U \in {\rm pr}_{1}\langle f \rangle \\
  \to \forall x(x \in {\rm pr}_{1}\langle f \rangle \to f(x) = g(x)) 
  \wedge U \in {\rm pr}_{1}\langle f \rangle
\end{multline*}
が共に成り立つ.
またThm \ref{awbta}より
\[
  ({\rm pr}_{1}\langle f \rangle = {\rm pr}_{1}\langle g \rangle 
  \wedge \forall x(x \in {\rm pr}_{1}\langle f \rangle \to f(x) = g(x))) 
  \wedge U \in {\rm pr}_{1}\langle f \rangle 
  \to U \in {\rm pr}_{1}\langle f \rangle
\]
が成り立つ.
そこでこれと(6)から, 再び推論法則 \ref{dedprewedge}により
\[
\tag{8}
  ({\rm pr}_{1}\langle f \rangle = {\rm pr}_{1}\langle g \rangle 
  \wedge \forall x(x \in {\rm pr}_{1}\langle f \rangle \to f(x) = g(x))) 
  \wedge U \in {\rm pr}_{1}\langle f \rangle 
  \to {\rm pr}_{1}\langle f \rangle = {\rm pr}_{1}\langle g \rangle 
  \wedge U \in {\rm pr}_{1}\langle f \rangle
\]
が成り立つ.
また定理 \ref{sthm=&in}より
\[
\tag{9}
  {\rm pr}_{1}\langle f \rangle = {\rm pr}_{1}\langle g \rangle \wedge U \in {\rm pr}_{1}\langle f \rangle 
  \to U \in {\rm pr}_{1}\langle g \rangle
\]
が成り立つ.
そこで(8), (9)から, 推論法則 \ref{dedmmp}によって
\[
\tag{10}
  ({\rm pr}_{1}\langle f \rangle = {\rm pr}_{1}\langle g \rangle 
  \wedge \forall x(x \in {\rm pr}_{1}\langle f \rangle \to f(x) = g(x))) 
  \wedge U \in {\rm pr}_{1}\langle f \rangle 
  \to U \in {\rm pr}_{1}\langle g \rangle
\]
が成り立つ.
またThm \ref{thmallfund2}より
\[
  \forall x(x \in {\rm pr}_{1}\langle f \rangle \to f(x) = g(x)) 
  \to (U|x)(x \in {\rm pr}_{1}\langle f \rangle \to f(x) = g(x))
\]
が成り立つが, 上述のように$x$は$f$, $g$, ${\rm pr}_{1}\langle f \rangle$の
いずれの記号列の中にも自由変数として現れないから, 
代入法則 \ref{substfree}, \ref{substfund}, \ref{substfuncval}により, 上記の記号列は
\[
  \forall x(x \in {\rm pr}_{1}\langle f \rangle \to f(x) = g(x)) 
  \to (U \in {\rm pr}_{1}\langle f \rangle \to f(U) = g(U))
\]
と一致し, これが定理となる.
故に推論法則 \ref{dedtwch}により
\[
\tag{11}
  \forall x(x \in {\rm pr}_{1}\langle f \rangle \to f(x) = g(x)) 
  \wedge U \in {\rm pr}_{1}\langle f \rangle 
  \to f(U) = g(U)
\]
が成り立つ.
そこで(7), (11)から, 推論法則 \ref{dedmmp}によって
\[
  ({\rm pr}_{1}\langle f \rangle = {\rm pr}_{1}\langle g \rangle 
  \wedge \forall x(x \in {\rm pr}_{1}\langle f \rangle \to f(x) = g(x))) 
  \wedge U \in {\rm pr}_{1}\langle f \rangle 
  \to f(U) = g(U)
\]
が成り立つ.
故にこれと(10)から, 推論法則 \ref{dedprewedge}により
\[
  ({\rm pr}_{1}\langle f \rangle = {\rm pr}_{1}\langle g \rangle 
  \wedge \forall x(x \in {\rm pr}_{1}\langle f \rangle \to f(x) = g(x))) 
  \wedge U \in {\rm pr}_{1}\langle f \rangle 
  \to U \in {\rm pr}_{1}\langle g \rangle \wedge f(U) = g(U)
\]
が成り立ち, これから推論法則 \ref{dedaddw}により
\begin{multline*}
\tag{12}
  {\rm Func}(g) 
  \wedge (({\rm pr}_{1}\langle f \rangle = {\rm pr}_{1}\langle g \rangle 
  \wedge \forall x(x \in {\rm pr}_{1}\langle f \rangle \to f(x) = g(x))) 
  \wedge U \in {\rm pr}_{1}\langle f \rangle) \\
  \to {\rm Func}(g) \wedge (U \in {\rm pr}_{1}\langle g \rangle \wedge f(U) = g(U))
\end{multline*}
が成り立つ.
またThm \ref{1awb1wctaw1bwc1}より
\begin{multline*}
\tag{13}
  ({\rm Func}(g) 
  \wedge ({\rm pr}_{1}\langle f \rangle = {\rm pr}_{1}\langle g \rangle 
  \wedge \forall x(x \in {\rm pr}_{1}\langle f \rangle \to f(x) = g(x)))) 
  \wedge U \in {\rm pr}_{1}\langle f \rangle \\
  \to {\rm Func}(g) 
  \wedge (({\rm pr}_{1}\langle f \rangle = {\rm pr}_{1}\langle g \rangle 
  \wedge \forall x(x \in {\rm pr}_{1}\langle f \rangle \to f(x) = g(x))) 
  \wedge U \in {\rm pr}_{1}\langle f \rangle)
\end{multline*}
が成り立つ.
また定理 \ref{sthmfuncvalbasis}より
\[
  {\rm Func}(g) \to ((U, f(U)) \in g \leftrightarrow U \in {\rm pr}_{1}\langle g \rangle \wedge f(U) = g(U))
\]
が成り立つから, 推論法則 \ref{dedprewedge}により
\[
  {\rm Func}(g) \to (U \in {\rm pr}_{1}\langle g \rangle \wedge f(U) = g(U) \to (U, f(U)) \in g)
\]
が成り立ち, これから推論法則 \ref{dedtwch}により
\[
\tag{14}
  {\rm Func}(g) \wedge (U \in {\rm pr}_{1}\langle g \rangle \wedge f(U) = g(U)) \to (U, f(U)) \in g
\]
が成り立つ.
そこで(13), (12), (14)にこの順で推論法則 \ref{dedmmp}を適用していき, 
\[
  ({\rm Func}(g) 
  \wedge ({\rm pr}_{1}\langle f \rangle = {\rm pr}_{1}\langle g \rangle 
  \wedge \forall x(x \in {\rm pr}_{1}\langle f \rangle \to f(x) = g(x)))) 
  \wedge U \in {\rm pr}_{1}\langle f \rangle 
  \to (U, f(U)) \in g
\]
が成り立つことがわかる.
故に推論法則 \ref{dedtwch}により
\[
\tag{15}
  {\rm Func}(g) 
  \wedge ({\rm pr}_{1}\langle f \rangle = {\rm pr}_{1}\langle g \rangle 
  \wedge \forall x(x \in {\rm pr}_{1}\langle f \rangle \to f(x) = g(x))) 
  \to (U \in {\rm pr}_{1}\langle f \rangle \to (U, f(U)) \in g)
\]
が成り立つ.
また$U$の定義から, Thm \ref{thmallfund1}と推論法則 \ref{dedequiv}により
\[
  (U|x)(x \in {\rm pr}_{1}\langle f \rangle \to (x, f(x)) \in g) 
  \to \forall x(x \in {\rm pr}_{1}\langle f \rangle \to (x, f(x)) \in g)
\]
が成り立つが, 上述のように$x$は$f$, $g$, ${\rm pr}_{1}\langle f \rangle$のいずれの記号列の中にも
自由変数として現れないから, 
代入法則 \ref{substfree}, \ref{substfund}, \ref{substpair}, \ref{substfuncval}によれば, 
この記号列は
\[
\tag{16}
  (U \in {\rm pr}_{1}\langle f \rangle \to (U, f(U)) \in g) 
  \to \forall x(x \in {\rm pr}_{1}\langle f \rangle \to (x, f(x)) \in g)
\]
と一致する.
故にこれが定理となる.
そこで(15), (16)から, 推論法則 \ref{dedmmp}によって
\[
  {\rm Func}(g) 
  \wedge ({\rm pr}_{1}\langle f \rangle = {\rm pr}_{1}\langle g \rangle 
  \wedge \forall x(x \in {\rm pr}_{1}\langle f \rangle \to f(x) = g(x))) 
  \to \forall x(x \in {\rm pr}_{1}\langle f \rangle \to (x, f(x)) \in g)
\]
が成り立つ.
故に推論法則 \ref{dedaddw}により
\begin{multline*}
\tag{17}
  {\rm Func}(f) 
  \wedge ({\rm Func}(g) 
  \wedge ({\rm pr}_{1}\langle f \rangle = {\rm pr}_{1}\langle g \rangle 
  \wedge \forall x(x \in {\rm pr}_{1}\langle f \rangle \to f(x) = g(x)))) \\
  \to {\rm Func}(f) \wedge \forall x(x \in {\rm pr}_{1}\langle f \rangle \to (x, f(x)) \in g)
\end{multline*}
が成り立つ.
またThm \ref{1awb1wctaw1bwc1}より
\begin{multline*}
\tag{18}
  ({\rm Func}(f) 
  \wedge {\rm Func}(g)) 
  \wedge ({\rm pr}_{1}\langle f \rangle = {\rm pr}_{1}\langle g \rangle 
  \wedge \forall x(x \in {\rm pr}_{1}\langle f \rangle \to f(x) = g(x))) \\
  \to {\rm Func}(f) 
  \wedge ({\rm Func}(g) 
  \wedge ({\rm pr}_{1}\langle f \rangle = {\rm pr}_{1}\langle g \rangle 
  \wedge \forall x(x \in {\rm pr}_{1}\langle f \rangle \to f(x) = g(x))))
\end{multline*}
が成り立つ.
また$x$が$f$及び$g$の中に自由変数として現れないことから, 定理 \ref{sthmfuncfuncvalsubset}より
\[
  {\rm Func}(f) 
  \to (f \subset g \leftrightarrow \forall x(x \in {\rm pr}_{1}\langle f \rangle \to (x, f(x)) \in g))
\]
が成り立つ.
故に推論法則 \ref{dedprewedge}により
\[
  {\rm Func}(f) 
  \to (\forall x(x \in {\rm pr}_{1}\langle f \rangle \to (x, f(x)) \in g) \to f \subset g)
\]
が成り立ち, これから推論法則 \ref{dedtwch}により
\[
\tag{19}
  {\rm Func}(f) \wedge \forall x(x \in {\rm pr}_{1}\langle f \rangle \to (x, f(x)) \in g) 
  \to f \subset g
\]
が成り立つ.
そこで(18), (17), (19)にこの順で推論法則 \ref{dedmmp}を適用していき, 
\[
\tag{20}
  ({\rm Func}(f) 
  \wedge {\rm Func}(g)) 
  \wedge ({\rm pr}_{1}\langle f \rangle = {\rm pr}_{1}\langle g \rangle 
  \wedge \forall x(x \in {\rm pr}_{1}\langle f \rangle \to f(x) = g(x))) 
  \to f \subset g
\]
が成り立つことがわかる.
またThm \ref{awbta}より
\[
  ({\rm Func}(f) 
  \wedge {\rm Func}(g)) 
  \wedge ({\rm pr}_{1}\langle f \rangle = {\rm pr}_{1}\langle g \rangle 
  \wedge \forall x(x \in {\rm pr}_{1}\langle f \rangle \to f(x) = g(x))) 
  \to {\rm Func}(f) \wedge {\rm Func}(g), 
\]
\begin{multline*}
  ({\rm Func}(f) 
  \wedge {\rm Func}(g)) 
  \wedge ({\rm pr}_{1}\langle f \rangle = {\rm pr}_{1}\langle g \rangle 
  \wedge \forall x(x \in {\rm pr}_{1}\langle f \rangle \to f(x) = g(x))) \\
  \to {\rm pr}_{1}\langle f \rangle = {\rm pr}_{1}\langle g \rangle 
  \wedge \forall x(x \in {\rm pr}_{1}\langle f \rangle \to f(x) = g(x))
\end{multline*}
が共に成り立つから, それぞれから推論法則 \ref{dedprewedge}により, 
\begin{align*}
  ({\rm Func}(f) 
  \wedge {\rm Func}(g)) 
  \wedge ({\rm pr}_{1}\langle f \rangle = {\rm pr}_{1}\langle g \rangle 
  \wedge \forall x(x \in {\rm pr}_{1}\langle f \rangle \to f(x) = g(x))) 
  &\to {\rm Func}(g), \\
  \mbox{} \\
  ({\rm Func}(f) 
  \wedge {\rm Func}(g)) 
  \wedge ({\rm pr}_{1}\langle f \rangle = {\rm pr}_{1}\langle g \rangle 
  \wedge \forall x(x \in {\rm pr}_{1}\langle f \rangle \to f(x) = g(x))) 
  &\to {\rm pr}_{1}\langle f \rangle = {\rm pr}_{1}\langle g \rangle
\end{align*}
が成り立つ.
故にこれらから, 再び推論法則 \ref{dedprewedge}により
\begin{multline*}
\tag{21}
  ({\rm Func}(f) 
  \wedge {\rm Func}(g)) 
  \wedge ({\rm pr}_{1}\langle f \rangle = {\rm pr}_{1}\langle g \rangle 
  \wedge \forall x(x \in {\rm pr}_{1}\langle f \rangle \to f(x) = g(x))) \\
  \to {\rm pr}_{1}\langle f \rangle = {\rm pr}_{1}\langle g \rangle \wedge {\rm Func}(g)
\end{multline*}
が成り立つ.
また定理 \ref{sthmfuncrelation}と推論法則 \ref{dedequiv}により
${\rm Func}(g) \to {\rm Func}(g; {\rm pr}_{1}\langle g \rangle)$が成り立つから, 
推論法則 \ref{dedaddw}により
\[
\tag{22}
  {\rm pr}_{1}\langle f \rangle = {\rm pr}_{1}\langle g \rangle \wedge {\rm Func}(g) 
  \to {\rm pr}_{1}\langle f \rangle = {\rm pr}_{1}\langle g \rangle \wedge {\rm Func}(g; {\rm pr}_{1}\langle g \rangle)
\]
が成り立つ.
また定理 \ref{sthmfuncsubset=eq}より
\[
  {\rm pr}_{1}\langle f \rangle = {\rm pr}_{1}\langle g \rangle \wedge {\rm Func}(g; {\rm pr}_{1}\langle g \rangle) 
  \to (f \subset g \leftrightarrow f = g)
\]
が成り立つから, 推論法則 \ref{dedprewedge}により
\[
\tag{23}
  {\rm pr}_{1}\langle f \rangle = {\rm pr}_{1}\langle g \rangle \wedge {\rm Func}(g; {\rm pr}_{1}\langle g \rangle) 
  \to (f \subset g \to f = g)
\]
が成り立つ.
そこで(21), (22), (23)から, 推論法則 \ref{dedmmp}によって
\[
  ({\rm Func}(f) 
  \wedge {\rm Func}(g)) 
  \wedge ({\rm pr}_{1}\langle f \rangle = {\rm pr}_{1}\langle g \rangle 
  \wedge \forall x(x \in {\rm pr}_{1}\langle f \rangle \to f(x) = g(x))) 
  \to (f \subset g \to f = g)
\]
が成り立つことがわかる.
故にこれと(20)から, 推論法則 \ref{dedprewedge}により
\begin{multline*}
\tag{24}
  ({\rm Func}(f) 
  \wedge {\rm Func}(g)) 
  \wedge ({\rm pr}_{1}\langle f \rangle = {\rm pr}_{1}\langle g \rangle 
  \wedge \forall x(x \in {\rm pr}_{1}\langle f \rangle \to f(x) = g(x))) \\
  \to f \subset g \wedge (f \subset g \to f = g)
\end{multline*}
が成り立つ.
またThm \ref{at11atb1tb1}より
\[
  f \subset g \to ((f \subset g \to f = g) \to f = g)
\]
が成り立つから, 推論法則 \ref{dedtwch}により
\[
\tag{25}
  f \subset g \wedge (f \subset g \to f = g) \to f = g
\]
が成り立つ.
そこで(24), (25)から, 推論法則 \ref{dedmmp}によって
\[
  ({\rm Func}(f) 
  \wedge {\rm Func}(g)) 
  \wedge ({\rm pr}_{1}\langle f \rangle = {\rm pr}_{1}\langle g \rangle 
  \wedge \forall x(x \in {\rm pr}_{1}\langle f \rangle \to f(x) = g(x))) 
  \to f = g
\]
が成り立つ.
故に推論法則 \ref{dedtwch}により
\[
\tag{26}
  {\rm Func}(f) 
  \wedge {\rm Func}(g) 
  \to ({\rm pr}_{1}\langle f \rangle = {\rm pr}_{1}\langle g \rangle 
  \wedge \forall x(x \in {\rm pr}_{1}\langle f \rangle \to f(x) = g(x)) 
  \to f = g)
\]
が成り立つ.
また1)の証明において示したように(5)が成り立つから, 推論法則 \ref{dedatawbtrue2}により
\begin{multline*}
\tag{27}
  ({\rm pr}_{1}\langle f \rangle = {\rm pr}_{1}\langle g \rangle 
  \wedge \forall x(x \in {\rm pr}_{1}\langle f \rangle \to f(x) = g(x)) 
  \to f = g) \\
  \to (f = g 
  \leftrightarrow {\rm pr}_{1}\langle f \rangle = {\rm pr}_{1}\langle g \rangle 
  \wedge \forall x(x \in {\rm pr}_{1}\langle f \rangle \to f(x) = g(x)))
\end{multline*}
が成り立つ.
そこで(26), (27)から, 推論法則 \ref{dedmmp}によって
\[
\tag{28}
  {\rm Func}(f) 
  \wedge {\rm Func}(g) 
  \to (f = g 
  \leftrightarrow {\rm pr}_{1}\langle f \rangle = {\rm pr}_{1}\langle g \rangle 
  \wedge \forall x(x \in {\rm pr}_{1}\langle f \rangle \to f(x) = g(x)))
\]
が成り立つ.
${(**)}_{1}$が成り立つことは, これと推論法則 \ref{dedmp}, \ref{dedwedge}, \ref{dedeqfund}によって明らかである.
また${(**)}_{2}$が成り立つことは, これと推論法則 \ref{dedmp}, \ref{dedwedge}, \ref{dedeqfund}, \ref{dedltthmquan}によって
明らかである.

\noindent
3)
定理 \ref{sthmfuncbasis}より
\begin{align*}
  {\rm Func}(f; a) \to {\rm Func}(f)&, ~~
  {\rm Func}(g; b) \to {\rm Func}(g), \\
  \mbox{} \\
  {\rm Func}(f; a) \to {\rm pr}_{1}\langle f \rangle = a&, ~~
  {\rm Func}(g; b) \to {\rm pr}_{1}\langle g \rangle = b
\end{align*}
がすべて成り立つから, このはじめの二つから, 推論法則 \ref{dedfromaddw}によって
\[
\tag{29}
  {\rm Func}(f; a) \wedge {\rm Func}(g; b) \to {\rm Func}(f) \wedge {\rm Func}(g)
\]
が成り立ち, 後の二つから, 同じく推論法則 \ref{dedfromaddw}によって
\[
\tag{30}
  {\rm Func}(f; a) \wedge {\rm Func}(g; b) \to {\rm pr}_{1}\langle f \rangle = a \wedge {\rm pr}_{1}\langle g \rangle = b
\]
が成り立つ.
またいま$x$は$f$及び$g$の中に自由変数として現れないから, 2)の証明において示したように(28)が成り立つ.
故に(29), (28)から, 推論法則 \ref{dedmmp}によって
\[
\tag{31}
  {\rm Func}(f; a) \wedge {\rm Func}(g; b) 
  \to (f = g \leftrightarrow {\rm pr}_{1}\langle f \rangle = {\rm pr}_{1}\langle g \rangle 
  \wedge \forall x(x \in {\rm pr}_{1}\langle f \rangle \to f(x) = g(x)))
\]
が成り立つ.
またThm \ref{x=ywz=ut1x=zly=u1}より
\[
\tag{32}
  {\rm pr}_{1}\langle f \rangle = a \wedge {\rm pr}_{1}\langle g \rangle = b 
  \to ({\rm pr}_{1}\langle f \rangle = {\rm pr}_{1}\langle g \rangle \leftrightarrow a = b)
\]
が成り立つ.
またThm \ref{awbta}より
\[
\tag{33}
  {\rm pr}_{1}\langle f \rangle = a \wedge {\rm pr}_{1}\langle g \rangle = b 
  \to {\rm pr}_{1}\langle f \rangle = a
\]
が成り立つ.
またいま$\tau_{x}(\neg ((x \in {\rm pr}_{1}\langle f \rangle \to f(x) = g(x)) \leftrightarrow (x \in a \to f(x) = g(x))))$を
$V$と書けば, $V$は集合であり, 定理 \ref{sthm=tineq}より
\[
\tag{34}
  {\rm pr}_{1}\langle f \rangle = a 
  \to (V \in {\rm pr}_{1}\langle f \rangle \leftrightarrow V \in a)
\]
が成り立つ.
またThm \ref{1alb1t11atc1l1btc11}より
\[
\tag{35}
  (V \in {\rm pr}_{1}\langle f \rangle \leftrightarrow V \in a) 
  \to ((V \in {\rm pr}_{1}\langle f \rangle \to f(V) = g(V)) \leftrightarrow (V \in a \to f(V) = g(V)))
\]
が成り立つ.
また$V$の定義から, Thm \ref{thmallfund1}と推論法則 \ref{dedequiv}により
\begin{multline*}
  (V|x)((x \in {\rm pr}_{1}\langle f \rangle \to f(x) = g(x)) \leftrightarrow (x \in a \to f(x) = g(x))) \\
  \to \forall x((x \in {\rm pr}_{1}\langle f \rangle \to f(x) = g(x)) \leftrightarrow (x \in a \to f(x) = g(x)))
\end{multline*}
が成り立つ.
いま$x$は$f$の中に自由変数として現れないから, 変数法則 \ref{valprset}により, 
$x$は${\rm pr}_{1}\langle f \rangle$の中にも自由変数として現れない.
また$x$は$a$及び$g$の中にも自由変数として現れない.
故に代入法則 \ref{substfree}, \ref{substfund}, \ref{substequiv}, \ref{substfuncval}によれば, 
上記の記号列は
\begin{multline*}
\tag{36}
  ((V \in {\rm pr}_{1}\langle f \rangle \to f(V) = g(V)) \leftrightarrow (V \in a \to f(V) = g(V))) \\
  \to \forall x((x \in {\rm pr}_{1}\langle f \rangle \to f(x) = g(x)) \leftrightarrow (x \in a \to f(x) = g(x)))
\end{multline*}
と一致する.
従ってこれが定理となる.
またThm \ref{thmalleqallsep}より
\begin{multline*}
\tag{37}
  \forall x((x \in {\rm pr}_{1}\langle f \rangle \to f(x) = g(x)) \leftrightarrow (x \in a \to f(x) = g(x))) \\
  \to (\forall x(x \in {\rm pr}_{1}\langle f \rangle \to f(x) = g(x)) \leftrightarrow \forall x(x \in a \to f(x) = g(x)))
\end{multline*}
が成り立つ.
そこで(33)---(37)から, 推論法則 \ref{dedmmp}によって
\[
  {\rm pr}_{1}\langle f \rangle =a \wedge {\rm pr}_{1}\langle g \rangle = b 
  \to (\forall x(x \in {\rm pr}_{1}\langle f \rangle \to f(x) = g(x)) \leftrightarrow \forall x(x \in a \to f(x) = g(x)))
\]
が成り立つことがわかる.
故にこれと(32)から, 推論法則 \ref{dedprewedge}により
\begin{multline*}
\tag{38}
  {\rm pr}_{1}\langle f \rangle =a \wedge {\rm pr}_{1}\langle g \rangle = b \\
  \to ({\rm pr}_{1}\langle f \rangle = {\rm pr}_{1}\langle g \rangle \leftrightarrow a = b) 
  \wedge (\forall x(x \in {\rm pr}_{1}\langle f \rangle \to f(x) = g(x)) \leftrightarrow \forall x(x \in a \to f(x) = g(x)))
\end{multline*}
が成り立つ.
またThm \ref{1alb1w1cld1t1awclbwd1}より
\begin{multline*}
\tag{39}
  ({\rm pr}_{1}\langle f \rangle = {\rm pr}_{1}\langle g \rangle \leftrightarrow a = b) 
  \wedge (\forall x(x \in {\rm pr}_{1}\langle f \rangle \to f(x) = g(x)) \leftrightarrow \forall x(x \in a \to f(x) = g(x))) \\
  \to ({\rm pr}_{1}\langle f \rangle = {\rm pr}_{1}\langle g \rangle \wedge \forall x(x \in {\rm pr}_{1}\langle f \rangle \to f(x) = g(x)) 
  \leftrightarrow a = b \wedge \forall x(x \in a \to f(x) = g(x)))
\end{multline*}
が成り立つ.
そこで(30), (38), (39)から, 推論法則 \ref{dedmmp}によって
\begin{multline*}
  {\rm Func}(f; a) \wedge {\rm Func}(g; b) \\
  \to ({\rm pr}_{1}\langle f \rangle = {\rm pr}_{1}\langle g \rangle 
  \wedge \forall x(x \in {\rm pr}_{1}\langle f \rangle \to f(x) = g(x)) 
  \leftrightarrow a = b \wedge \forall x(x \in a \to f(x) = g(x)))
\end{multline*}
が成り立つことがわかる.
故にこれと(31)から, 推論法則 \ref{dedprewedge}によって
\begin{multline*}
\tag{40}
  {\rm Func}(f; a) \wedge {\rm Func}(g; b) \\
  \to (f = g \leftrightarrow {\rm pr}_{1}\langle f \rangle = {\rm pr}_{1}\langle g \rangle 
  \wedge \forall x(x \in {\rm pr}_{1}\langle f \rangle \to f(x) = g(x))) ~~~~~~~~~~~~~~~~~~~~~~~~~~~~~~~~~~ \\
  \wedge ({\rm pr}_{1}\langle f \rangle = {\rm pr}_{1}\langle g \rangle 
  \wedge \forall x(x \in {\rm pr}_{1}\langle f \rangle \to f(x) = g(x)) 
  \leftrightarrow a = b \wedge \forall x(x \in a \to f(x) = g(x)))
\end{multline*}
が成り立つ.
またThm \ref{1alb1w1blc1t1alc1}より
\begin{multline*}
\tag{41}
  (f = g \leftrightarrow {\rm pr}_{1}\langle f \rangle = {\rm pr}_{1}\langle g \rangle 
  \wedge \forall x(x \in {\rm pr}_{1}\langle f \rangle \to f(x) = g(x))) \\
  \wedge ({\rm pr}_{1}\langle f \rangle = {\rm pr}_{1}\langle g \rangle 
  \wedge \forall x(x \in {\rm pr}_{1}\langle f \rangle \to f(x) = g(x)) 
  \leftrightarrow a = b \wedge \forall x(x \in a \to f(x) = g(x))) \\
  \to (f = g \leftrightarrow a = b \wedge \forall x(x \in a \to f(x) = g(x)))
\end{multline*}
が成り立つ.
そこで(40), (41)から, 推論法則 \ref{dedmmp}によって
\[
  {\rm Func}(f; a) \wedge {\rm Func}(g; b) 
  \to (f = g \leftrightarrow a = b \wedge \forall x(x \in a \to f(x) = g(x)))
\]
が成り立つ.
${({**}*)}_{1}$が成り立つことは, これと
推論法則 \ref{dedmp}, \ref{dedwedge}, \ref{dedeqfund}によって明らかである.
また${({**}*)}_{2}$が成り立つことは, これと
推論法則 \ref{dedmp}, \ref{dedwedge}, \ref{dedeqfund}, \ref{dedltthmquan}によって明らかである.

\noindent
4)
定理 \ref{sthmfuncbasis}より
\[
  {\rm Func}(f; a; b) \to {\rm Func}(f; a), ~~
  {\rm Func}(g; c; d) \to {\rm Func}(g; c)
\]
が共に成り立つから, 推論法則 \ref{dedfromaddw}により
\[
\tag{42}
  {\rm Func}(f; a; b) \wedge {\rm Func}(g; c; d) \to {\rm Func}(f; a) \wedge {\rm Func}(g; c)
\]
が成り立つ.
また$x$が$a$, $f$, $g$のいずれの記号列の中にも自由変数として現れないことから, 
3)より
\[
\tag{43}
  {\rm Func}(f; a) \wedge {\rm Func}(g; c) 
  \to (f = g \leftrightarrow a = c \wedge \forall x(x \in a \to f(x) = g(x)))
\]
が成り立つ.
そこで(42), (43)から, 推論法則 \ref{dedmmp}によって
\[
  {\rm Func}(f; a; b) \wedge {\rm Func}(g; c; d) 
  \to (f = g \leftrightarrow a = c \wedge \forall x(x \in a \to f(x) = g(x)))
\]
が成り立つ.
${({**}{**})}_{1}$が成り立つことは, これと
推論法則 \ref{dedmp}, \ref{dedwedge}, \ref{dedeqfund}によって明らかである.
また${({**}{**})}_{2}$が成り立つことは, これと
推論法則 \ref{dedmp}, \ref{dedwedge}, \ref{dedeqfund}, \ref{dedltthmquan}によって明らかである.
\halmos




\mathstrut
\begin{thm}
\label{sthmfuncfuncval=2}%定理
\mbox{}

1)
$a$, $f$, $g$を集合とし, $x$をこれらの中に自由変数として現れない文字とするとき, 
\[
  {\rm Func}(f; a) \wedge {\rm Func}(g; a) \to (f = g \leftrightarrow \forall x(x \in a \to f(x) = g(x)))
\]
が成り立つ.
またこのことから, 次の${(*)}_{1}$, ${(*)}_{2}$が成り立つ: 

${(*)}_{1}$ ~~$f$と$g$が共に$a$における函数ならば, 
              \[
                f = g \leftrightarrow \forall x(x \in a \to f(x) = g(x))
              \]
              が成り立つ.
              そこで特にこのとき, $\forall x(x \in a \to f(x) = g(x))$が成り立つならば, 
              $f = g$が成り立つ.

${(*)}_{2}$ ~~$f$と$g$が共に$a$における函数であるとき, 
              $x$が定数でなく, $x \in a \to f(x) = g(x)$が成り立つならば, 
              $f = g$が成り立つ.

2)
$a$, $f$, $g$, $x$は1)と同じとし, 更に$b$と$c$を集合とするとき, 
\[
  {\rm Func}(f; a; b) \wedge {\rm Func}(g; a; c) \to (f = g \leftrightarrow \forall x(x \in a \to f(x) = g(x)))
\]
が成り立つ.
またこのことから, 次の${(**)}_{1}$, ${(**)}_{2}$が成り立つ: 

${(**)}_{1}$ ~~$f$が$a$から$b$への函数であり, $g$が$a$から$c$への函数ならば, 
               \[
                 f = g \leftrightarrow \forall x(x \in a \to f(x) = g(x))
               \]
               が成り立つ.
               そこで特にこのとき, $\forall x(x \in a \to f(x) = g(x))$が成り立つならば, 
               $f = g$が成り立つ.

${(**)}_{2}$ ~~$f$が$a$から$b$への函数であり, $g$が$a$から$c$への函数であるとき, 
               $x$が定数でなく, $x \in a \to f(x) = g(x)$が成り立つならば, 
               $f = g$が成り立つ.
\end{thm}


\noindent{\bf 証明}
~1)
$x$が$a$, $f$, $g$のいずれの記号列の中にも自由変数として現れないことから, 
定理 \ref{sthmfuncfuncval=}より
\[
\tag{1}
  {\rm Func}(f; a) \wedge {\rm Func}(g; a) \to (f = g \leftrightarrow a = a \wedge \forall x(x \in a \to f(x) = g(x)))
\]
が成り立つ.
またThm \ref{x=x}より$a = a$が成り立つから, 推論法則 \ref{dedawblatrue2}により
\[
  a = a \wedge \forall x(x \in a \to f(x) = g(x)) \leftrightarrow \forall x(x \in a \to f(x) = g(x))
\]
が成り立つ.
故に推論法則 \ref{dedatawbtrue2}により
\begin{multline*}
\tag{2}
  (f = g \leftrightarrow a = a \wedge \forall x(x \in a \to f(x) = g(x))) \\
  \to (f = g \leftrightarrow a = a \wedge \forall x(x \in a \to f(x) = g(x))) \\
  \wedge (a = a \wedge \forall x(x \in a \to f(x) = g(x)) \leftrightarrow \forall x(x \in a \to f(x) = g(x)))
\end{multline*}
が成り立つ.
またThm \ref{1alb1w1blc1t1alc1}より
\begin{multline*}
\tag{3}
  (f = g \leftrightarrow a = a \wedge \forall x(x \in a \to f(x) = g(x))) \\
  \wedge (a = a \wedge \forall x(x \in a \to f(x) = g(x)) \leftrightarrow \forall x(x \in a \to f(x) = g(x))) \\
  \to (f = g \leftrightarrow \forall x(x \in a \to f(x) = g(x)))
\end{multline*}
が成り立つ.
そこで(1), (2), (3)から, 推論法則 \ref{dedmmp}によって
\[
\tag{4}
  {\rm Func}(f; a) \wedge {\rm Func}(g; a) \to (f = g \leftrightarrow \forall x(x \in a \to f(x) = g(x)))
\]
が成り立つことがわかる.
${(*)}_{1}$が成り立つことは, これと
推論法則 \ref{dedmp}, \ref{dedwedge}, \ref{dedeqfund}によって明らかである.
また${(*)}_{2}$が成り立つことは, これと
推論法則 \ref{dedmp}, \ref{dedwedge}, \ref{dedeqfund}, \ref{dedltthmquan}によって明らかである.

\noindent
2)
定理 \ref{sthmfuncbasis}より
\[
  {\rm Func}(f; a; b) \to {\rm Func}(f; a), ~~
  {\rm Func}(g; a; c) \to {\rm Func}(g; a)
\]
が共に成り立つから, 推論法則 \ref{dedfromaddw}により
\[
\tag{5}
  {\rm Func}(f; a; b) \wedge {\rm Func}(g; a; c) \to {\rm Func}(f; a) \wedge {\rm Func}(g; a)
\]
が成り立つ.
また1)の証明において示したように(4)が成り立つ.
そこで(5), (4)から, 推論法則 \ref{dedmmp}によって
\[
  {\rm Func}(f; a; b) \wedge {\rm Func}(g; a; c) \to (f = g \leftrightarrow \forall x(x \in a \to f(x) = g(x)))
\]
が成り立つ.
${(**)}_{1}$が成り立つことは, これと
推論法則 \ref{dedmp}, \ref{dedwedge}, \ref{dedeqfund}によって明らかである.
また${(**)}_{2}$が成り立つことは, これと
推論法則 \ref{dedmp}, \ref{dedwedge}, \ref{dedeqfund}, \ref{dedltthmquan}によって明らかである.
\halmos




\mathstrut
\begin{thm}
\label{sthmsingletonfuncval}%定理
$a$と$b$を集合とするとき, 
\[
  \{(a, b)\}(a) = b
\]
が成り立つ.
\end{thm}


\noindent{\bf 証明}
~定理 \ref{sthmsingletonfunc}より$\{(a, b)\}$は函数である.
また定理 \ref{sthmsingletonfund}より$(a, b) \in \{(a, b)\}$が成り立つ.
そこで定理 \ref{sthmfuncvalbasis}により
$a \in {\rm pr}_{1}\langle \{(a, b)\} \rangle$と
$b = \{(a, b)\}(a)$が共に成り立つ.
故にこの後者から, 推論法則 \ref{ded=ch}により
$\{(a, b)\}(a) = b$が成り立つ.
\halmos




\mathstrut
\begin{thm}
\label{sthmcupfuncval}%定理
$f$, $g$, $t$を集合とするとき, 
\begin{align*}
  {\rm Func}(f \cup g) &\to (t \in {\rm pr}_{1}\langle f \rangle \to (f \cup g)(t) = f(t)), \\
  \mbox{} \\
  {\rm Func}(f \cup g) &\to (t \in {\rm pr}_{1}\langle g \rangle \to (f \cup g)(t) = g(t))
\end{align*}
が成り立つ.
またこのことから, 次の($*$)が成り立つ: 

($*$) ~~$f \cup g$が函数ならば, 
        \[
          t \in {\rm pr}_{1}\langle f \rangle \to (f \cup g)(t) = f(t), ~~
          t \in {\rm pr}_{1}\langle g \rangle \to (f \cup g)(t) = g(t)
        \]
        が共に成り立つ.
        故に, $f \cup g$が函数であるとき, $t \in {\rm pr}_{1}\langle f \rangle$が成り立つならば
        $(f \cup g)(t) = f(t)$が成り立ち, $t \in {\rm pr}_{1}\langle g \rangle$が成り立つならば
        $(f \cup g)(t) = g(t)$が成り立つ.
\end{thm}


\noindent{\bf 証明}
~定理 \ref{sthmfuncval}と推論法則 \ref{dedequiv}により
\[
  t \in {\rm pr}_{1}\langle f \rangle \to (t, f(t)) \in f, ~~
  t \in {\rm pr}_{1}\langle g \rangle \to (t, g(t)) \in g
\]
が共に成り立つから, 推論法則 \ref{dedaddw}により
\begin{align*}
  \tag{1}
  {\rm Func}(f \cup g) \wedge t \in {\rm pr}_{1}\langle f \rangle &\to {\rm Func}(f \cup g) \wedge (t, f(t)) \in f, \\
  \mbox{} \\
  \tag{2}
  {\rm Func}(f \cup g) \wedge t \in {\rm pr}_{1}\langle g \rangle &\to {\rm Func}(f \cup g) \wedge (t, g(t)) \in g
\end{align*}
が共に成り立つ.
また定理 \ref{sthmsubsetcup}より
$f \subset f \cup g$と$g \subset f \cup g$が共に成り立つから, 
定理 \ref{sthmsubsetbasis}により
\[
  (t, f(t)) \in f \to (t, f(t)) \in f \cup g, ~~
  (t, g(t)) \in g \to (t, g(t)) \in f \cup g
\]
が共に成り立つ.
故に推論法則 \ref{dedaddw}により
\begin{align*}
  \tag{3}
  {\rm Func}(f \cup g) \wedge (t, f(t)) \in f &\to {\rm Func}(f \cup g) \wedge (t, f(t)) \in f \cup g, \\
  \mbox{} \\
  \tag{4}
  {\rm Func}(f \cup g) \wedge (t, g(t)) \in g &\to {\rm Func}(f \cup g) \wedge (t, g(t)) \in f \cup g
\end{align*}
が共に成り立つ.
また定理 \ref{sthmfuncvalbasispractical}より
\begin{align*}
  {\rm Func}(f \cup g) &\to ((t, f(t)) \in f \cup g \to f(t) = (f \cup g)(t)), \\
  \mbox{} \\
  {\rm Func}(f \cup g) &\to ((t, g(t)) \in f \cup g \to g(t) = (f \cup g)(t))
\end{align*}
が共に成り立つから, 推論法則 \ref{dedtwch}により
\begin{align*}
  \tag{5}
  {\rm Func}(f \cup g) \wedge (t, f(t)) \in f \cup g &\to f(t) = (f \cup g)(t), \\
  \mbox{} \\
  \tag{6}
  {\rm Func}(f \cup g) \wedge (t, g(t)) \in f \cup g &\to g(t) = (f \cup g)(t)
\end{align*}
が共に成り立つ.
またThm \ref{x=yty=x}より
\begin{align*}
  \tag{7}
  f(t) = (f \cup g)(t) &\to (f \cup g)(t) = f(t), \\
  \mbox{} \\
  \tag{8}
  g(t) = (f \cup g)(t) &\to (f \cup g)(t) = g(t)
\end{align*}
が共に成り立つ.
そこで(1), (3), (5), (7)から, 推論法則 \ref{dedmmp}によって
\[
  {\rm Func}(f \cup g) \wedge t \in {\rm pr}_{1}\langle f \rangle \to (f \cup g)(t) = f(t)
\]
が成り立ち, 
(2), (4), (6), (8)から, 同じく推論法則 \ref{dedmmp}によって
\[
  {\rm Func}(f \cup g) \wedge t \in {\rm pr}_{1}\langle g \rangle \to (f \cup g)(t) = g(t)
\]
が成り立つことがわかる.
故にこれらから, 推論法則 \ref{dedtwch}により
\begin{align*}
  {\rm Func}(f \cup g) &\to (t \in {\rm pr}_{1}\langle f \rangle \to (f \cup g)(t) = f(t)), \\
  \mbox{} \\
  {\rm Func}(f \cup g) &\to (t \in {\rm pr}_{1}\langle g \rangle \to (f \cup g)(t) = g(t))
\end{align*}
が共に成り立つ.
($*$)が成り立つことは, これらと推論法則 \ref{dedmp}によって明らかである.
\halmos




\mathstrut
\begin{thm}
\label{sthmcupfuncval2}%定理
\mbox{} 

1)
$a$, $b$, $f$, $g$, $t$を集合とするとき, 
\begin{align*}
  a \cap b = \phi &\to ({\rm Func}(f; a) \wedge {\rm Func}(g; b) \to (t \in a \to (f \cup g)(t) = f(t))), \\
  \mbox{} \\
  a \cap b = \phi &\to ({\rm Func}(f; a) \wedge {\rm Func}(g; b) \to (t \in b \to (f \cup g)(t) = g(t)))
\end{align*}
が成り立つ.
またこれらから, 次の($*$)が成り立つ:

($*$) ~~$a \cap b$が空ならば, 
        \begin{align*}
          {\rm Func}(f; a) \wedge {\rm Func}(g; b) &\to (t \in a \to (f \cup g)(t) = f(t)), \\
          \mbox{} \\
          {\rm Func}(f; a) \wedge {\rm Func}(g; b) &\to (t \in b \to (f \cup g)(t) = g(t))
        \end{align*}
        が共に成り立つ.
        故にこのとき, $f$が$a$における函数であり, かつ$g$が$b$における函数であるならば, 
        \[
          t \in a \to (f \cup g)(t) = f(t), ~~
          t \in b \to (f \cup g)(t) = g(t)
        \]
        が共に成り立つ.
        そこで上記の仮定に加え, $t \in a$が成り立つならば$(f \cup g)(t) = f(t)$が成り立ち, 
        $t \in b$が成り立つならば$(f \cup g)(t) = g(t)$が成り立つ.

2)
$a$, $b$, $c$, $d$, $f$, $g$, $t$を集合とするとき, 
\begin{align*}
  a \cap c = \phi &\to ({\rm Func}(f; a; b) \wedge {\rm Func}(g; c; d) \to (t \in a \to (f \cup g)(t) = f(t))), \\
  \mbox{} \\
  a \cap c = \phi &\to ({\rm Func}(f; a; b) \wedge {\rm Func}(g; c; d) \to (t \in c \to (f \cup g)(t) = g(t)))
\end{align*}
が成り立つ.
またこれらから, 次の($**$)が成り立つ:

($**$) ~~$a \cap c$が空ならば, 
         \begin{align*}
           {\rm Func}(f; a; b) \wedge {\rm Func}(g; c; d) &\to (t \in a \to (f \cup g)(t) = f(t)), \\
           \mbox{} \\
           {\rm Func}(f; a; b) \wedge {\rm Func}(g; c; d) &\to (t \in c \to (f \cup g)(t) = g(t))
         \end{align*}
         が共に成り立つ.
         故にこのとき, $f$が$a$から$b$への函数であり, かつ$g$が$c$から$d$への函数であるならば, 
         \[
           t \in a \to (f \cup g)(t) = f(t), ~~
           t \in c \to (f \cup g)(t) = g(t)
         \]
         が共に成り立つ.
         そこで上記の仮定に加え, $t \in a$が成り立つならば$(f \cup g)(t) = f(t)$が成り立ち, 
         $t \in c$が成り立つならば$(f \cup g)(t) = g(t)$が成り立つ.
\end{thm}


\noindent{\bf 証明}
~1)
定理 \ref{sthmcupfunc}より
\[
  a \cap b = \phi \to ({\rm Func}(f; a) \wedge {\rm Func}(g; b) \to {\rm Func}(f \cup g; a \cup b))
\]
が成り立つから, 推論法則 \ref{dedtwch}により
\[
\tag{1}
  a \cap b = \phi \wedge ({\rm Func}(f; a) \wedge {\rm Func}(g; b)) \to {\rm Func}(f \cup g; a \cup b)
\]
が成り立つ.
また定理 \ref{sthmfuncbasis}より
\[
\tag{2}
  {\rm Func}(f \cup g; a \cup b) \to {\rm Func}(f \cup g)
\]
が成り立つ.
また定理 \ref{sthmcupfuncval}より
\begin{align*}
  \tag{3}
  {\rm Func}(f \cup g) &\to (t \in {\rm pr}_{1}\langle f \rangle \to (f \cup g)(t) = f(t)), \\
  \mbox{} \\
  \tag{4}
  {\rm Func}(f \cup g) &\to (t \in {\rm pr}_{1}\langle g \rangle \to (f \cup g)(t) = g(t))
\end{align*}
が共に成り立つ.
そこで(1), (2), (3)から, 推論法則 \ref{dedmmp}によって
\[
\tag{5}
  a \cap b = \phi \wedge ({\rm Func}(f; a) \wedge {\rm Func}(g; b)) 
  \to (t \in {\rm pr}_{1}\langle f \rangle \to (f \cup g)(t) = f(t))
\]
が成り立ち, 
(1), (2), (4)から, 同じく推論法則 \ref{dedmmp}によって
\[
\tag{6}
  a \cap b = \phi \wedge ({\rm Func}(f; a) \wedge {\rm Func}(g; b)) 
  \to (t \in {\rm pr}_{1}\langle g \rangle \to (f \cup g)(t) = g(t))
\]
が成り立つことがわかる.
またThm \ref{awbta}より
\[
  a \cap b = \phi \wedge ({\rm Func}(f; a) \wedge {\rm Func}(g; b)) 
  \to {\rm Func}(f; a) \wedge {\rm Func}(g; b)
\]
が成り立つから, 推論法則 \ref{dedprewedge}により
\begin{align*}
  \tag{7}
  a \cap b = \phi \wedge ({\rm Func}(f; a) \wedge {\rm Func}(g; b)) &\to {\rm Func}(f; a), \\
  \mbox{} \\
  \tag{8}
  a \cap b = \phi \wedge ({\rm Func}(f; a) \wedge {\rm Func}(g; b)) &\to {\rm Func}(g; b)
\end{align*}
が共に成り立つ.
また定理 \ref{sthmfuncbasis}より
\begin{align*}
  \tag{9}
  {\rm Func}(f; a) &\to {\rm pr}_{1}\langle f \rangle = a, \\
  \mbox{} \\
  \tag{10}
  {\rm Func}(g; b) &\to {\rm pr}_{1}\langle g \rangle = b
\end{align*}
が共に成り立つ.
また定理 \ref{sthm=tineq}より
\[
  {\rm pr}_{1}\langle f \rangle = a \to (t \in {\rm pr}_{1}\langle f \rangle \leftrightarrow t \in a), ~~
  {\rm pr}_{1}\langle g \rangle = b \to (t \in {\rm pr}_{1}\langle g \rangle \leftrightarrow t \in b)
\]
が共に成り立つから, 推論法則 \ref{dedprewedge}により
\begin{align*}
  \tag{11}
  {\rm pr}_{1}\langle f \rangle = a &\to (t \in a \to t \in {\rm pr}_{1}\langle f \rangle), \\
  \mbox{} \\
  \tag{12}
  {\rm pr}_{1}\langle g \rangle = b &\to (t \in b \to t \in {\rm pr}_{1}\langle g \rangle)
\end{align*}
が共に成り立つ.
またThm \ref{1atb1t11btc1t1atc11}より
\begin{align*}
  \tag{13}
  (t \in a \to t \in {\rm pr}_{1}\langle f \rangle) 
  &\to ((t \in {\rm pr}_{1}\langle f \rangle \to (f \cup g)(t) = f(t)) \to (t \in a \to (f \cup g)(t) = f(t))), \\
  \mbox{} \\
  \tag{14}
  (t \in b \to t \in {\rm pr}_{1}\langle g \rangle) 
  &\to ((t \in {\rm pr}_{1}\langle g \rangle \to (f \cup g)(t) = g(t)) \to (t \in b \to (f \cup g)(t) = g(t)))
\end{align*}
が共に成り立つ.
そこで(7), (9), (11), (13)から, 推論法則 \ref{dedmmp}によって
\begin{multline*}
  a \cap b = \phi \wedge ({\rm Func}(f; a) \wedge {\rm Func}(g; b)) \\
  \to ((t \in {\rm pr}_{1}\langle f \rangle \to (f \cup g)(t) = f(t)) \to (t \in a \to (f \cup g)(t) = f(t)))
\end{multline*}
が成り立ち, 
(8), (10), (12), (14)から, 同じく推論法則 \ref{dedmmp}によって
\begin{multline*}
  a \cap b = \phi \wedge ({\rm Func}(f; a) \wedge {\rm Func}(g; b)) \\
  \to ((t \in {\rm pr}_{1}\langle g \rangle \to (f \cup g)(t) = g(t)) \to (t \in b \to (f \cup g)(t) = g(t)))
\end{multline*}
が成り立つことがわかる.
故にこれらから, それぞれ推論法則 \ref{deds2}によって
\begin{multline*}
\tag{15}
  (a \cap b = \phi \wedge ({\rm Func}(f; a) \wedge {\rm Func}(g; b)) 
  \to (t \in {\rm pr}_{1}\langle f \rangle \to (f \cup g)(t) = f(t))) \\
  \to (a \cap b = \phi \wedge ({\rm Func}(f; a) \wedge {\rm Func}(g; b)) 
  \to (t \in a \to (f \cup g)(t) = f(t))), 
\end{multline*}
\begin{multline*}
\tag{16}
  (a \cap b = \phi \wedge ({\rm Func}(f; a) \wedge {\rm Func}(g; b)) 
  \to (t \in {\rm pr}_{1}\langle g \rangle \to (f \cup g)(t) = g(t))) \\
  \to (a \cap b = \phi \wedge ({\rm Func}(f; a) \wedge {\rm Func}(g; b)) 
  \to (t \in b \to (f \cup g)(t) = g(t)))
\end{multline*}
が成り立つ.
そこで(5), (15)から, 推論法則 \ref{dedmp}によって
\[
  a \cap b = \phi \wedge ({\rm Func}(f; a) \wedge {\rm Func}(g; b)) 
  \to (t \in a \to (f \cup g)(t) = f(t))
\]
が成り立ち, 
(6), (16)から, 同じく推論法則 \ref{dedmp}によって
\[
  a \cap b = \phi \wedge ({\rm Func}(f; a) \wedge {\rm Func}(g; b)) 
  \to (t \in b \to (f \cup g)(t) = g(t))
\]
が成り立つ.
故にこれらから, 推論法則 \ref{dedtwch}により
\begin{align*}
  a \cap b = \phi 
  &\to ({\rm Func}(f; a) \wedge {\rm Func}(g; b) \to (t \in a \to (f \cup g)(t) = f(t))), \\
  \mbox{} \\
  a \cap b = \phi 
  &\to ({\rm Func}(f; a) \wedge {\rm Func}(g; b) \to (t \in b \to (f \cup g)(t) = g(t)))
\end{align*}
が共に成り立つ.
($*$)が成り立つことは, これらと推論法則 \ref{dedmp}, \ref{dedwedge}によって明らかである.

\noindent
2)
1)より
\begin{align*}
  \tag{17}
  a \cap c = \phi 
  &\to ({\rm Func}(f; a) \wedge {\rm Func}(g; c) \to (t \in a \to (f \cup g)(t) = f(t))), \\
  \mbox{} \\
  \tag{18}
  a \cap c = \phi 
  &\to ({\rm Func}(f; a) \wedge {\rm Func}(g; c) \to (t \in c \to (f \cup g)(t) = g(t)))
\end{align*}
が共に成り立つ.
また定理 \ref{sthmfuncbasis}より
\[
  {\rm Func}(f; a; b) \to {\rm Func}(f; a), ~~
  {\rm Func}(g; c; d) \to {\rm Func}(g; c)
\]
が共に成り立つから, 推論法則 \ref{dedfromaddw}により
\[
  {\rm Func}(f; a; b) \wedge {\rm Func}(g; c; d) \to {\rm Func}(f; a) \wedge {\rm Func}(g; c)
\]
が成り立ち, これから推論法則 \ref{dedaddf}により
\begin{multline*}
\tag{19}
  ({\rm Func}(f; a) \wedge {\rm Func}(g; c) \to (t \in a \to (f \cup g)(t) = f(t))) \\
  \to ({\rm Func}(f; a; b) \wedge {\rm Func}(g; c; d) \to (t \in a \to (f \cup g)(t) = f(t))), 
\end{multline*}
\begin{multline*}
\tag{20}
  ({\rm Func}(f; a) \wedge {\rm Func}(g; c) \to (t \in c \to (f \cup g)(t) = g(t))) \\
  \to ({\rm Func}(f; a; b) \wedge {\rm Func}(g; c; d) \to (t \in c \to (f \cup g)(t) = g(t)))
\end{multline*}
が共に成り立つ.
そこで(17), (19)から, 推論法則 \ref{dedmmp}によって
\[
  a \cap c = \phi 
  \to ({\rm Func}(f; a; b) \wedge {\rm Func}(g; c; d) \to (t \in a \to (f \cup g)(t) = f(t)))
\]
が成り立ち, (18), (20)から, 同じく推論法則 \ref{dedmmp}によって
\[
  a \cap c = \phi 
  \to ({\rm Func}(f; a; b) \wedge {\rm Func}(g; c; d) \to (t \in c \to (f \cup g)(t) = g(t)))
\]
が成り立つ.
($**$)が成り立つことは, これらと推論法則 \ref{dedmp}, \ref{dedwedge}によって明らかである.
\halmos




\mathstrut
\begin{thm}
\label{sthmcapfuncval}%定理
$f$, $g$, $t$を集合とするとき, 
\begin{align*}
  {\rm Func}(f) &\to (t \in {\rm pr}_{1}\langle f \cap g \rangle \to (f \cap g)(t) = f(t)), \\
  \mbox{} \\
  {\rm Func}(g) &\to (t \in {\rm pr}_{1}\langle f \cap g \rangle \to (f \cap g)(t) = g(t))
\end{align*}
が成り立つ.
またこのことから, 次の($*$)が成り立つ: 

($*$) ~~$f$が函数ならば, 
        \[
          t \in {\rm pr}_{1}\langle f \cap g \rangle \to (f \cap g)(t) = f(t)
        \]
        が成り立つ.
        故にこのとき, $t \in {\rm pr}_{1}\langle f \cap g \rangle$が成り立つならば, 
        $(f \cap g)(t) = f(t)$が成り立つ.
        同様に, $g$が函数ならば, 
        \[
          t \in {\rm pr}_{1}\langle f \cap g \rangle \to (f \cap g)(t) = g(t)
        \]
        が成り立つ.
        故にこのとき, $t \in {\rm pr}_{1}\langle f \cap g \rangle$が成り立つならば, 
        $(f \cap g)(t) = g(t)$が成り立つ.
\end{thm}


\noindent{\bf 証明}
~定理 \ref{sthmfuncval}と推論法則 \ref{dedequiv}により
\[
\tag{1}
  t \in {\rm pr}_{1}\langle f \cap g \rangle \to (t, (f \cap g)(t)) \in f \cap g
\]
が成り立つ.
また定理 \ref{sthmcap}より
$f \cap g \subset f$と$f \cap g \subset g$が共に成り立つから, 
定理 \ref{sthmsubsetbasis}により
\begin{align*}
  \tag{2}
  (t, (f \cap g)(t)) \in f \cap g &\to (t, (f \cap g)(t)) \in f, \\
  \mbox{} \\
  \tag{3}
  (t, (f \cap g)(t)) \in f \cap g &\to (t, (f \cap g)(t)) \in g
\end{align*}
が共に成り立つ.
そこで(1)と(2), (1)と(3)から, それぞれ推論法則 \ref{dedmmp}によって
\[
  t \in {\rm pr}_{1}\langle f \cap g \rangle \to (t, (f \cap g)(t)) \in f, ~~
  t \in {\rm pr}_{1}\langle f \cap g \rangle \to (t, (f \cap g)(t)) \in g
\]
が成り立つ.
故にこれらから, それぞれ推論法則 \ref{dedaddw}により
\begin{align*}
  \tag{4}
  {\rm Func}(f) \wedge t \in {\rm pr}_{1}\langle f \cap g \rangle &\to {\rm Func}(f) \wedge (t, (f \cap g)(t)) \in f, \\
  \mbox{} \\
  \tag{5}
  {\rm Func}(g) \wedge t \in {\rm pr}_{1}\langle f \cap g \rangle &\to {\rm Func}(g) \wedge (t, (f \cap g)(t)) \in g
\end{align*}
が成り立つ.
また定理 \ref{sthmfuncvalbasispractical}より
\begin{align*}
  {\rm Func}(f) &\to ((t, (f \cap g)(t)) \in f \to (f \cap g)(t) = f(t)), \\
  \mbox{} \\
  {\rm Func}(g) &\to ((t, (f \cap g)(t)) \in g \to (f \cap g)(t) = g(t))
\end{align*}
が共に成り立つから, 推論法則 \ref{dedtwch}により
\begin{align*}
  \tag{6}
  {\rm Func}(f) \wedge (t, (f \cap g)(t)) \in f &\to (f \cap g)(t) = f(t), \\
  \mbox{} \\
  \tag{7}
  {\rm Func}(g) \wedge (t, (f \cap g)(t)) \in g &\to (f \cap g)(t) = g(t)
\end{align*}
が共に成り立つ.
そこで(4), (6)から, 推論法則 \ref{dedmmp}によって
\[
  {\rm Func}(f) \wedge t \in {\rm pr}_{1}\langle f \cap g \rangle \to (f \cap g)(t) = f(t)
\]
が成り立ち, (5), (7)から, 同じく推論法則 \ref{dedmmp}によって
\[
  {\rm Func}(g) \wedge t \in {\rm pr}_{1}\langle f \cap g \rangle \to (f \cap g)(t) = g(t)
\]
が成り立つ.
故にこれらから, それぞれ推論法則 \ref{dedtwch}により
\begin{align*}
  {\rm Func}(f) &\to (t \in {\rm pr}_{1}\langle f \cap g \rangle \to (f \cap g)(t) = f(t)), \\
  \mbox{} \\
  {\rm Func}(g) &\to (t \in {\rm pr}_{1}\langle f \cap g \rangle \to (f \cap g)(t) = g(t))
\end{align*}
が成り立つ.
($*$)が成り立つことは, これらと推論法則 \ref{dedmp}によって明らかである.
\halmos




\mathstrut
\begin{thm}
\label{sthm-funcval}%定理
$f$, $g$, $t$を集合とするとき, 
\[
  {\rm Func}(f) \to (t \in {\rm pr}_{1}\langle f - g \rangle \to (f - g)(t) = f(t))
\]
が成り立つ.
またこのことから, 次の($*$)が成り立つ: 

($*$) ~~$f$が函数ならば, 
        \[
          t \in {\rm pr}_{1}\langle f - g \rangle \to (f - g)(t) = f(t)
        \]
        が成り立つ.
        故にこのとき, $t \in {\rm pr}_{1}\langle f - g \rangle$が成り立つならば, 
        $(f - g)(t) = f(t)$が成り立つ.
\end{thm}


\noindent{\bf 証明}
~定理 \ref{sthmfuncval}と推論法則 \ref{dedequiv}により
\[
\tag{1}
  t \in {\rm pr}_{1}\langle f - g \rangle \to (t, (f - g)(t)) \in f - g
\]
が成り立つ.
また定理 \ref{sthma-bsubseta}より$f - g \subset f$が成り立つから, 
定理 \ref{sthmsubsetbasis}により
\[
\tag{2}
  (t, (f - g)(t)) \in f - g \to (t, (f - g)(t)) \in f
\]
が成り立つ.
そこで(1), (2)から, 推論法則 \ref{dedmmp}によって
\[
  t \in {\rm pr}_{1}\langle f - g \rangle \to (t, (f - g)(t)) \in f
\]
が成り立ち, これから推論法則 \ref{dedaddw}により
\[
\tag{3}
  {\rm Func}(f) \wedge t \in {\rm pr}_{1}\langle f - g \rangle \to {\rm Func}(f) \wedge (t, (f - g)(t)) \in f
\]
が成り立つ.
また定理 \ref{sthmfuncvalbasispractical}より
\[
  {\rm Func}(f) \to ((t, (f - g)(t)) \in f \to (f - g)(t) = f(t))
\]
が成り立つから, 推論法則 \ref{dedtwch}により
\[
\tag{4}
  {\rm Func}(f) \wedge (t, (f - g)(t)) \in f \to (f - g)(t) = f(t)
\]
が成り立つ.
そこで(3), (4)から, 推論法則 \ref{dedmmp}によって
\[
  {\rm Func}(f) \wedge t \in {\rm pr}_{1}\langle f - g \rangle \to (f - g)(t) = f(t)
\]
が成り立ち, これから推論法則 \ref{dedtwch}により
\[
  {\rm Func}(f) \to (t \in {\rm pr}_{1}\langle f - g \rangle \to (f - g)(t) = f(t))
\]
が成り立つ.
($*$)が成り立つことは, これと推論法則 \ref{dedmp}によって明らかである.
\halmos




\mathstrut
\begin{thm}
\label{sthmproductfuncval}%定理
$a$, $b$, $t$を集合とするとき, 
\[
  t \in a \to (a \times \{b\})(t) = b
\]
が成り立つ.
またこのことから, 次の($*$)が成り立つ: 

($*$) ~~$t \in a$が成り立つならば, $(a \times \{b\})(t) = b$が成り立つ.
\end{thm}


\noindent{\bf 証明}
~定理 \ref{sthmsingletonfund}より$b \in \{b\}$が成り立つから, 
推論法則 \ref{dedatawbtrue2}により
\[
\tag{1}
  t \in a \to t \in a \wedge b \in \{b\}
\]
が成り立つ.
また定理 \ref{sthmpairinproduct}と推論法則 \ref{dedequiv}により
\[
\tag{2}
  t \in a \wedge b \in \{b\} \to (t, b) \in a \times \{b\}
\]
が成り立つ.
また定理 \ref{sthmproductfunc}より$a \times \{b\}$は函数だから, 
定理 \ref{sthmfuncvalbasispractical}により
\[
\tag{3}
  (t, b) \in a \times \{b\} \to b = (a \times \{b\})(t)
\]
が成り立つ.
またThm \ref{x=yty=x}より
\[
\tag{4}
  b = (a \times \{b\})(t) \to (a \times \{b\})(t) = b
\]
が成り立つ.
そこで(1)---(4)から, 推論法則 \ref{dedmmp}によって
\[
  t \in a \to (a \times \{b\})(t) = b
\]
が成り立つことがわかる.
($*$)が成り立つことは, これと推論法則 \ref{dedmp}によって明らかである.
\halmos




\mathstrut
\begin{thm}
\label{sthmfuncvalinvalueset}%定理
\mbox{}

1)
$c$, $f$, $t$を集合とするとき, 
\begin{align*}
  &t \in {\rm pr}_{1}\langle f \rangle \cap c \to f(t) \in f[c], \\
  \mbox{} \\
  &c \subset {\rm pr}_{1}\langle f \rangle \to (t \in c \to f(t) \in f[c])
\end{align*}
が成り立つ.
またこれらから, 次の${(*)}_{1}$, ${(*)}_{2}$が成り立つ: 

${(*)}_{1}$ ~~$t \in {\rm pr}_{1}\langle f \rangle \cap c$が成り立つならば, 
              $f(t) \in f[c]$が成り立つ.

${(*)}_{2}$ ~~$c \subset {\rm pr}_{1}\langle f \rangle$が成り立つならば, 
              $t \in c \to f(t) \in f[c]$が成り立つ.
              故にこのとき, $t \in c$が成り立つならば, $f(t) \in f[c]$が成り立つ.

2)
$a$, $c$, $f$, $t$を集合とするとき, 
\begin{align*}
  {\rm Func}(f; a) &\to (t \in a \cap c \to f(t) \in f[c]), \\
  \mbox{} \\
  {\rm Func}(f; a) &\to (c \subset a \to (t \in c \to f(t) \in f[c]))
\end{align*}
が成り立つ.
またこれらから, 次の${(**)}_{1}$, ${(**)}_{2}$が成り立つ: 

${(**)}_{1}$ ~~$f$が$a$における函数ならば, 
               \[
                 t \in a \cap c \to f(t) \in f[c]
               \]
               が成り立つ.
               故にこのとき, $t \in a \cap c$が成り立つならば, 
               $f(t) \in f[c]$が成り立つ.

${(**)}_{2}$ ~~$f$が$a$における函数ならば, 
               \[
                 c \subset a \to (t \in c \to f(t) \in f[c])
               \]
               が成り立つ.
               故にこのとき, $c \subset a$が成り立つならば, 
               $t \in c \to f(t) \in f[c]$が成り立つ.
               そこで更にこのとき, $t \in c$が成り立つならば, $f(t) \in f[c]$が成り立つ.

3)
$a$, $b$, $c$, $f$, $t$を集合とするとき, 
\begin{align*}
  {\rm Func}(f; a; b) &\to (t \in a \cap c \to f(t) \in f[c]), \\
  \mbox{} \\
  {\rm Func}(f; a; b) &\to (c \subset a \to (t \in c \to f(t) \in f[c]))
\end{align*}
が成り立つ.
またこれらから, 次の${({**}*)}_{1}$, ${({**}*)}_{2}$が成り立つ: 

${({**}*)}_{1}$ ~~$f$が$a$から$b$への函数ならば, 
                  \[
                    t \in a \cap c \to f(t) \in f[c]
                  \]
                  が成り立つ.
                  故にこのとき, $t \in a \cap c$が成り立つならば, 
                  $f(t) \in f[c]$が成り立つ.

${({**}*)}_{2}$ ~~$f$が$a$から$b$への函数ならば, 
                  \[
                    c \subset a \to (t \in c \to f(t) \in f[c])
                  \]
                  が成り立つ.
                  故にこのとき, $c \subset a$が成り立つならば, 
                  $t \in c \to f(t) \in f[c]$が成り立つ.
                  そこで更にこのとき, $t \in c$が成り立つならば, $f(t) \in f[c]$が成り立つ.
\end{thm}


\noindent{\bf 証明}
~1)
はじめに前者が成り立つことを示す.
定理 \ref{sthmcapelement}と推論法則 \ref{dedequiv}により
\[
\tag{1}
  t \in {\rm pr}_{1}\langle f \rangle \cap c \to t \in {\rm pr}_{1}\langle f \rangle \wedge t \in c
\]
が成り立つ.
また定理 \ref{sthmfuncval}と推論法則 \ref{dedequiv}により
\[
  t \in {\rm pr}_{1}\langle f \rangle \to (t, f(t)) \in f
\]
が成り立つから, 推論法則 \ref{dedaddw}により
\[
\tag{2}
  t \in {\rm pr}_{1}\langle f \rangle \wedge t \in c \to (t, f(t)) \in f \wedge t \in c
\]
が成り立つ.
またThm \ref{awbtbwa}より
\[
\tag{3}
  (t, f(t)) \in f \wedge t \in c \to t \in c \wedge (t, f(t)) \in f
\]
が成り立つ.
また定理 \ref{sthmvaluesetbasis}より
\[
\tag{4}
  t \in c \wedge (t, f(t)) \in f \to f(t) \in f[c]
\]
が成り立つ.
そこで(1)---(4)から, 推論法則 \ref{dedmmp}によって
\[
\tag{5}
  t \in {\rm pr}_{1}\langle f \rangle \cap c \to f(t) \in f[c]
\]
が成り立つことがわかる.
${(*)}_{1}$が成り立つことは, これと推論法則 \ref{dedmp}によって明らかである.

次に後者が成り立つことを示す.
定理 \ref{sthmcapsubset=}と推論法則 \ref{dedequiv}により
\[
\tag{6}
  c \subset {\rm pr}_{1}\langle f \rangle \to c \cap {\rm pr}_{1}\langle f \rangle = c
\]
が成り立つ.
また定理 \ref{sthmcapch}より
$c \cap {\rm pr}_{1}\langle f \rangle = {\rm pr}_{1}\langle f \rangle \cap c$が成り立つから, 
推論法則 \ref{dedaddeq=}により
\[
  c \cap {\rm pr}_{1}\langle f \rangle = c \leftrightarrow {\rm pr}_{1}\langle f \rangle \cap c = c
\]
が成り立ち, これから推論法則 \ref{dedequiv}により
\[
\tag{7}
  c \cap {\rm pr}_{1}\langle f \rangle = c \to {\rm pr}_{1}\langle f \rangle \cap c = c
\]
が成り立つ.
また定理 \ref{sthm=tineq}より
\[
  {\rm pr}_{1}\langle f \rangle \cap c = c 
  \to (t \in {\rm pr}_{1}\langle f \rangle \cap c \leftrightarrow t \in c)
\]
が成り立つから, 推論法則 \ref{dedprewedge}により
\[
\tag{8}
  {\rm pr}_{1}\langle f \rangle \cap c = c 
  \to (t \in c \to t \in {\rm pr}_{1}\langle f \rangle \cap c)
\]
が成り立つ.
また(5)から, 推論法則 \ref{dedaddb}により
\[
\tag{9}
  (t \in c \to t \in {\rm pr}_{1}\langle f \rangle \cap c) \to (t \in c \to f(t) \in f[c])
\]
が成り立つ.
そこで(6)---(9)から, 推論法則 \ref{dedmmp}によって
\[
\tag{10}
  c \subset {\rm pr}_{1}\langle f \rangle \to (t \in c \to f(t) \in f[c])
\]
が成り立つことがわかる.
${(*)}_{2}$が成り立つことは, これと推論法則 \ref{dedmp}によって明らかである.

\noindent
2)
まず前者が成り立つことを示す.
定理 \ref{sthmfuncbasis}より
\[
\tag{11}
  {\rm Func}(f; a) \to {\rm pr}_{1}\langle f \rangle = a
\]
が成り立つ.
またいま$x$を$c$, $f$, $t$のいずれの記号列の中にも自由変数として現れない文字とすれば, 
schema S5の適用により
\[
  {\rm pr}_{1}\langle f \rangle = a 
  \to (({\rm pr}_{1}\langle f \rangle|x)(t \in x \cap c \to f(t) \in f[c]) \to (a|x)(t \in x \cap c \to f(t) \in f[c]))
\]
が成り立つが, 変数法則 \ref{valfund}, \ref{valvalueset}, \ref{valfuncval}により
$x$は$f(t) \in f[c]$の中にも自由変数として現れないから, 
代入法則 \ref{substfree}, \ref{substfund}, \ref{substcap}によれば, この記号列は
\[
\tag{12}
  {\rm pr}_{1}\langle f \rangle = a 
  \to ((t \in {\rm pr}_{1}\langle f \rangle \cap c \to f(t) \in f[c]) \to (t \in a \cap c \to f(t) \in f[c]))
\]
と一致する.
故にこれが定理となる.
そこで(11), (12)から, 推論法則 \ref{dedmmp}によって
\[
  {\rm Func}(f; a) 
  \to ((t \in {\rm pr}_{1}\langle f \rangle \cap c \to f(t) \in f[c]) \to (t \in a \cap c \to f(t) \in f[c]))
\]
が成り立つ.
故に推論法則 \ref{dedch}により
\[
  (t \in {\rm pr}_{1}\langle f \rangle \cap c \to f(t) \in f[c]) 
  \to ({\rm Func}(f; a) \to (t \in a \cap c \to f(t) \in f[c]))
\]
が成り立つ.
そこでこれと(5)から, 推論法則 \ref{dedmp}によって
\[
\tag{13}
  {\rm Func}(f; a) \to (t \in a \cap c \to f(t) \in f[c])
\]
が成り立つ.
${(**)}_{1}$が成り立つことは, これと推論法則 \ref{dedmp}によって明らかである.

次に後者が成り立つことを示す.
$x$は上と同じとするとき, schema S5の適用により
\[
  {\rm pr}_{1}\langle f \rangle = a 
  \to (({\rm pr}_{1}\langle f \rangle|x)(c \subset x \to (t \in c \to f(t) \in f[c])) \to (a|x)(c \subset x \to (t \in c \to f(t) \in f[c])))
\]
が成り立つが, 上述のように$x$は$c$, $t$, $f(t) \in f[c]$のいずれの記号列の中にも
自由変数として現れないから, 変数法則 \ref{valfund}により$x$は
$t \in c \to f(t) \in f[c]$の中にも自由変数として現れず, 
従って代入法則 \ref{substfree}, \ref{substfund}, \ref{substsubset}によれば, 
この記号列は
\[
\tag{14}
  {\rm pr}_{1}\langle f \rangle = a 
  \to ((c \subset {\rm pr}_{1}\langle f \rangle \to (t \in c \to f(t) \in f[c])) \to (c \subset a \to (t \in c \to f(t) \in f[c])))
\]
と一致する.
故にこれが定理となる.
そこで(11), (14)から, 推論法則 \ref{dedmmp}によって
\[
  {\rm Func}(f; a) 
  \to ((c \subset {\rm pr}_{1}\langle f \rangle \to (t \in c \to f(t) \in f[c])) \to (c \subset a \to (t \in c \to f(t) \in f[c])))
\]
が成り立つ.
故に推論法則 \ref{dedch}により
\[
  (c \subset {\rm pr}_{1}\langle f \rangle \to (t \in c \to f(t) \in f[c])) 
  \to ({\rm Func}(f; a) \to (c \subset a \to (t \in c \to f(t) \in f[c])))
\]
が成り立つ.
そこでこれと(10)から, 推論法則 \ref{dedmp}によって
\[
\tag{15}
  {\rm Func}(f; a) \to (c \subset a \to (t \in c \to f(t) \in f[c]))
\]
が成り立つ.
${(**)}_{2}$が成り立つことは, これと推論法則 \ref{dedmp}によって明らかである.

\noindent
3)
定理 \ref{sthmfuncbasis}より
${\rm Func}(f; a; b) \to {\rm Func}(f; a)$が成り立つから, 
これと(13), (15)から, それぞれ推論法則 \ref{dedmmp}によって
\begin{align*}
  \tag{16}
  {\rm Func}(f; a; b) &\to (t \in a \cap c \to f(t) \in f[c]), \\
  \mbox{} \\
  \tag{17}
  {\rm Func}(f; a; b) &\to (c \subset a \to (t \in c \to f(t) \in f[c]))
\end{align*}
が成り立つ.
${({**}*)}_{1}$, ${({**}*)}_{2}$が成り立つことは, 
それぞれ(16), (17)と推論法則 \ref{dedmp}によって明らかである.
\halmos




\mathstrut
\begin{thm}
\label{sthmvaluesetfuncval}%定理
\mbox{}

1)
$c$と$f$を集合とし, $x$をこれらの中に自由変数として現れない文字とする.
このとき
\[
  \{f(x)|x \in {\rm pr}_{1}\langle f \rangle \cap c\} \subset f[c]
\]
が成り立つ.

2)
$c$, $f$, $x$は1)と同じとするとき, 
\[
  {\rm Func}(f) \to f[c] = \{f(x)|x \in {\rm pr}_{1}\langle f \rangle \cap c\}
\]
が成り立つ.
またこのことから, 次の($*$)が成り立つ: 

($*$) ~~$f$が函数ならば, 
        $f[c] = \{f(x)|x \in {\rm pr}_{1}\langle f \rangle \cap c\}$が成り立つ.

3)
$a$, $c$, $f$を集合とし, $x$をこれらの中に自由変数として現れない文字とする.
このとき
\[
  {\rm Func}(f; a) \to f[c] = \{f(x)|x \in a \cap c\}
\]
が成り立つ.
またこのことから, 次の($**$)が成り立つ: 

($**$) ~~$f$が$a$における函数ならば, 
         $f[c] = \{f(x)|x \in a \cap c\}$が成り立つ.

4)
$a$, $c$, $f$, $x$は3)と同じとし, 更に$b$を集合とする.
このとき
\begin{align*}
  {\rm Func}(f; a; b) \to f[c] = \{f(x)|x \in a \cap c\}
\end{align*}
が成り立つ.
またこのことから, 次の(${**}*$)が成り立つ: 

(${**}*$) ~~$f$が$a$から$b$への函数ならば, 
            $f[c] = \{f(x)|x \in a \cap c\}$が成り立つ.
\end{thm}


\noindent{\bf 証明}
~1)
$\tau_{x}(\neg (x \in {\rm pr}_{1}\langle f \rangle \cap c \to f(x) \in f[c]))$を$T$と書けば, 
$T$は集合であり, 定理 \ref{sthmfuncvalinvalueset}より
\[
  T \in {\rm pr}_{1}\langle f \rangle \cap c \to f(T) \in f[c]
\]
が成り立つ.
ここで$x$が$c$及び$f$の中に自由変数として現れないことから, 
変数法則 \ref{valcap}, \ref{valprset}, \ref{valvalueset}により, 
$x$は${\rm pr}_{1}\langle f \rangle \cap c$及び$f[c]$の中にも自由変数として現れない.
故に代入法則 \ref{substfree}, \ref{substfund}, \ref{substfuncval}によれば, 
上記の記号列は
\[
  (T|x)(x \in {\rm pr}_{1}\langle f \rangle \cap c \to f(x) \in f[c])
\]
と一致する.
よってこれが定理となる.
そこで$T$の定義から, 推論法則 \ref{dedallfund}により
\[
  \forall x(x \in {\rm pr}_{1}\langle f \rangle \cap c \to f(x) \in f[c])
\]
が成り立つ.
上述のように$x$は${\rm pr}_{1}\langle f \rangle \cap c$及び$f[c]$の中に自由変数として現れないから, 
従って定理 \ref{sthmosetsubsetb}により
\[
\tag{1}
  \{f(x)|x \in {\rm pr}_{1}\langle f \rangle \cap c\} \subset f[c]
\]
が成り立つ.

\noindent
2)
$y$を$x$と異なり, $c$及び$f$の中に自由変数として現れない, 定数でない文字とする.
このとき, $x$が$y$と異なり, $c$及び$f$の中に自由変数として現れないことから, 
定理 \ref{sthmvaluesetelement}と推論法則 \ref{dedequiv}により
\[
  y \in f[c] \to \exists x(x \in c \wedge (x, y) \in f)
\]
が成り立つ.
ここで$\tau_{x}(x \in c \wedge (x, y) \in f)$を$U$と書けば, $U$は集合であり, 
定義から上記の記号列は
\[
  y \in f[c] \to (U|x)(x \in c \wedge (x, y) \in f)
\]
と同じである.
またいま述べたように, $x$は$y$と異なり, $c$及び$f$の中に自由変数として現れないから, 
代入法則 \ref{substfree}, \ref{substfund}, \ref{substwedge}, \ref{substpair}により, 
この記号列は
\[
\tag{2}
  y \in f[c] \to U \in c \wedge (U, y) \in f
\]
と一致する.
よってこれが定理となる.
また定理 \ref{sthmpairelementinprset}より
\[
  (U, y) \in f \to U \in {\rm pr}_{1}\langle f \rangle \wedge y \in {\rm pr}_{2}\langle f \rangle
\]
が成り立つから, 推論法則 \ref{dedprewedge}により
\[
  (U, y) \in f \to U \in {\rm pr}_{1}\langle f \rangle
\]
が成り立つ.
そこで推論法則 \ref{dedatawbtrue1}により
\[
  (U, y) \in f \to U \in {\rm pr}_{1}\langle f \rangle \wedge (U, y) \in f
\]
が成り立ち, これから推論法則 \ref{dedaddw}により
\[
\tag{3}
  U \in c \wedge (U, y) \in f \to U \in c \wedge (U \in {\rm pr}_{1}\langle f \rangle \wedge (U, y) \in f)
\]
が成り立つ.
またThm \ref{aw1bwc1t1awb1wc}より
\[
\tag{4}
  U \in c \wedge (U \in {\rm pr}_{1}\langle f \rangle \wedge (U, y) \in f) 
  \to (U \in c \wedge U \in {\rm pr}_{1}\langle f \rangle) \wedge (U, y) \in f
\]
が成り立つ.
またThm \ref{awbtbwa}より
\[
\tag{5}
  U \in c \wedge U \in {\rm pr}_{1}\langle f \rangle \to U \in {\rm pr}_{1}\langle f \rangle \wedge U \in c
\]
が成り立つ.
また定理 \ref{sthmcapelement}と推論法則 \ref{dedequiv}により
\[
\tag{6}
  U \in {\rm pr}_{1}\langle f \rangle \wedge U \in c \to U \in {\rm pr}_{1}\langle f \rangle \cap c
\]
が成り立つ.
そこで(5), (6)から, 推論法則 \ref{dedmmp}によって
\[
  U \in c \wedge U \in {\rm pr}_{1}\langle f \rangle \to U \in {\rm pr}_{1}\langle f \rangle \cap c
\]
が成り立つ.
故に推論法則 \ref{dedaddw}により
\[
\tag{7}
  (U \in c \wedge U \in {\rm pr}_{1}\langle f \rangle) \wedge (U, y) \in f 
  \to U \in {\rm pr}_{1}\langle f \rangle \cap c \wedge (U, y) \in f
\]
が成り立つ.
そこで(2), (3), (4), (7)から, 推論法則 \ref{dedmmp}によって
\[
  y \in f[c] \to U \in {\rm pr}_{1}\langle f \rangle \cap c \wedge (U, y) \in f
\]
が成り立つことがわかる.
故に推論法則 \ref{dedaddw}により
\[
\tag{8}
  {\rm Func}(f) \wedge y \in f[c] 
  \to {\rm Func}(f) \wedge (U \in {\rm pr}_{1}\langle f \rangle \cap c \wedge (U, y) \in f)
\]
が成り立つ.
またThm \ref{aw1bwc1t1awb1wc}より
\[
\tag{9}
  {\rm Func}(f) \wedge (U \in {\rm pr}_{1}\langle f \rangle \cap c \wedge (U, y) \in f) 
  \to ({\rm Func}(f) \wedge U \in {\rm pr}_{1}\langle f \rangle \cap c) \wedge (U, y) \in f
\]
が成り立つ.
またThm \ref{awbtbwa}より
\[
  {\rm Func}(f) \wedge U \in {\rm pr}_{1}\langle f \rangle \cap c 
  \to U \in {\rm pr}_{1}\langle f \rangle \cap c \wedge {\rm Func}(f)
\]
が成り立つから, 推論法則 \ref{dedaddw}により
\[
\tag{10}
  ({\rm Func}(f) \wedge U \in {\rm pr}_{1}\langle f \rangle \cap c) \wedge (U, y) \in f 
  \to (U \in {\rm pr}_{1}\langle f \rangle \cap c \wedge {\rm Func}(f)) \wedge (U, y) \in f
\]
が成り立つ.
またThm \ref{1awb1wctaw1bwc1}より
\[
\tag{11}
  (U \in {\rm pr}_{1}\langle f \rangle \cap c \wedge {\rm Func}(f)) \wedge (U, y) \in f 
  \to U \in {\rm pr}_{1}\langle f \rangle \cap c \wedge ({\rm Func}(f) \wedge (U, y) \in f)
\]
が成り立つ.
また定理 \ref{sthmfuncvalbasispractical}より
\[
  {\rm Func}(f) \to ((U, y) \in f \to y = f(U))
\]
が成り立つから, 推論法則 \ref{dedtwch}により
\[
  {\rm Func}(f) \wedge (U, y) \in f \to y = f(U)
\]
が成り立ち, これから推論法則 \ref{dedaddw}により
\[
\tag{12}
  U \in {\rm pr}_{1}\langle f \rangle \cap c \wedge ({\rm Func}(f) \wedge (U, y) \in f) 
  \to U \in {\rm pr}_{1}\langle f \rangle \cap c \wedge y = f(U)
\]
が成り立つ.
またschema S4の適用により
\[
  (U|x)(x \in {\rm pr}_{1}\langle f \rangle \cap c \wedge y = f(x)) 
  \to \exists x(x \in {\rm pr}_{1}\langle f \rangle \cap c \wedge y = f(x))
\]
が成り立つが, $x$は$y$と異なり, $f$の中に自由変数として現れず, 
1)の証明において述べたように${\rm pr}_{1}\langle f \rangle \cap c$の中にも自由変数として現れないから, 
代入法則 \ref{substfree}, \ref{substfund}, \ref{substwedge}, \ref{substfuncval}によれば, この記号列は
\[
\tag{13}
  U \in {\rm pr}_{1}\langle f \rangle \cap c \wedge y = f(U) 
  \to \exists x(x \in {\rm pr}_{1}\langle f \rangle \cap c \wedge y = f(x))
\]
と一致する.
よってこれが定理となる.
またいま述べたように, $x$は$y$と異なり, ${\rm pr}_{1}\langle f \rangle \cap c$の中に
自由変数として現れないから, 
定理 \ref{sthmosetbasis}と推論法則 \ref{dedequiv}により
\[
\tag{14}
  \exists x(x \in {\rm pr}_{1}\langle f \rangle \cap c \wedge y = f(x)) 
  \to y \in \{f(x)|x \in {\rm pr}_{1}\langle f \rangle \cap c\}
\]
が成り立つ.
そこで(8)---(14)から, 推論法則 \ref{dedmmp}によって
\[
  {\rm Func}(f) \wedge y \in f[c] 
  \to y \in \{f(x)|x \in {\rm pr}_{1}\langle f \rangle \cap c\}
\]
が成り立つことがわかる.
故に推論法則 \ref{dedtwch}により
\[
\tag{15}
  {\rm Func}(f) \to (y \in f[c] \to y \in \{f(x)|x \in {\rm pr}_{1}\langle f \rangle \cap c\})
\]
が成り立つ.
さていま$y$は$f$の中に自由変数として現れないから, 変数法則 \ref{valfunc}により, 
$y$は${\rm Func}(f)$の中にも自由変数として現れない.
また$y$は定数でない.
これらのことと, (15)が成り立つことから, 推論法則 \ref{dedalltquansepfreeconst}により
\[
  {\rm Func}(f) \to \forall y(y \in f[c] \to y \in \{f(x)|x \in {\rm pr}_{1}\langle f \rangle \cap c\})
\]
が成り立つ.
また$y$は$x$と異なり, $c$及び$f$の中に自由変数として現れないから, 
変数法則 \ref{valvalueset}により, $y$は$f[c]$の中に自由変数として現れず, 
変数法則 \ref{valoset}, \ref{valcap}, \ref{valprset}, \ref{valfuncval}により, 
$y$は$\{f(x)|x \in {\rm pr}_{1}\langle f \rangle \cap c\}$の中にも自由変数として現れない.
故に定義から, 上記の記号列は
\[
\tag{16}
  {\rm Func}(f) \to f[c] \subset \{f(x)|x \in {\rm pr}_{1}\langle f \rangle \cap c\}
\]
と同じである.
よってこれが定理となる.
また(1)が成り立つことから, 推論法則 \ref{dedatawbtrue2}により
\[
\tag{17}
  f[c] \subset \{f(x)|x \in {\rm pr}_{1}\langle f \rangle \cap c\} 
  \to f[c] \subset \{f(x)|x \in {\rm pr}_{1}\langle f \rangle \cap c\} 
  \wedge \{f(x)|x \in {\rm pr}_{1}\langle f \rangle \cap c\} \subset f[c]
\]
が成り立つ.
また定理 \ref{sthmaxiom1}と推論法則 \ref{dedequiv}により
\[
\tag{18}
  f[c] \subset \{f(x)|x \in {\rm pr}_{1}\langle f \rangle \cap c\} 
  \wedge \{f(x)|x \in {\rm pr}_{1}\langle f \rangle \cap c\} \subset f[c] 
  \to f[c] = \{f(x)|x \in {\rm pr}_{1}\langle f \rangle \cap c\}
\]
が成り立つ.
そこで(16), (17), (18)から, 推論法則 \ref{dedmmp}によって
\[
\tag{19}
  {\rm Func}(f) \to f[c] = \{f(x)|x \in {\rm pr}_{1}\langle f \rangle \cap c\}
\]
が成り立つことがわかる.
($*$)が成り立つことは, これと推論法則 \ref{dedmp}によって明らかである.

\noindent
3)
定理 \ref{sthmcap=}より
\[
\tag{20}
  {\rm pr}_{1}\langle f \rangle = a \to {\rm pr}_{1}\langle f \rangle \cap c = a \cap c
\]
が成り立つ.
また$x$が$a$, $c$, $f$のいずれの記号列の中にも自由変数として現れないことから, 
変数法則 \ref{valcap}, \ref{valprset}により, $x$は${\rm pr}_{1}\langle f \rangle \cap c$及び
$a \cap c$の中に自由変数として現れないから, 定理 \ref{sthmoset=}より
\[
\tag{21}
  {\rm pr}_{1}\langle f \rangle \cap c = a \cap c 
  \to \{f(x)|x \in {\rm pr}_{1}\langle f \rangle \cap c\} = \{f(x)|x \in a \cap c\}
\]
が成り立つ.
そこで(20), (21)から, 推論法則 \ref{dedmmp}によって
\[
\tag{22}
  {\rm pr}_{1}\langle f \rangle = a 
  \to \{f(x)|x \in {\rm pr}_{1}\langle f \rangle \cap c\} = \{f(x)|x \in a \cap c\}
\]
が成り立つ.
また$x$が$c$及び$f$の中に自由変数として現れないことから, 上で示したように(19)が成り立つ.
故に(19), (22)から, 推論法則 \ref{dedfromaddw}により
\[
  {\rm Func}(f) \wedge {\rm pr}_{1}\langle f \rangle = a 
  \to f[c] = \{f(x)|x \in {\rm pr}_{1}\langle f \rangle \cap c\} 
  \wedge \{f(x)|x \in {\rm pr}_{1}\langle f \rangle \cap c\} = \{f(x)|x \in a \cap c\}, 
\]
即ち
\[
\tag{23}
  {\rm Func}(f; a) 
  \to f[c] = \{f(x)|x \in {\rm pr}_{1}\langle f \rangle \cap c\} 
  \wedge \{f(x)|x \in {\rm pr}_{1}\langle f \rangle \cap c\} = \{f(x)|x \in a \cap c\}
\]
が成り立つ.
またThm \ref{x=ywy=ztx=z}より
\[
\tag{24}
  f[c] = \{f(x)|x \in {\rm pr}_{1}\langle f \rangle \cap c\} 
  \wedge \{f(x)|x \in {\rm pr}_{1}\langle f \rangle \cap c\} = \{f(x)|x \in a \cap c\} 
  \to f[c] = \{f(x)|x \in a \cap c\}
\]
が成り立つ.
そこで(23), (24)から, 推論法則 \ref{dedmmp}によって
\[
\tag{25}
  {\rm Func}(f; a) \to f[c] = \{f(x)|x \in a \cap c\}
\]
が成り立つ.
($**$)が成り立つことは, これと推論法則 \ref{dedmp}によって明らかである.

\noindent
4)
定理 \ref{sthmfuncbasis}より${\rm Func}(f; a; b) \to {\rm Func}(f; a)$が成り立つから, 
これと(25)から, 推論法則 \ref{dedmmp}によって
\[
  {\rm Func}(f; a; b) \to f[c] = \{f(x)|x \in a \cap c\}
\]
が成り立つ.
(${**}*$)が成り立つことは, これと推論法則 \ref{dedmp}によって明らかである.
\halmos




\mathstrut
\begin{thm}
\label{sthmvaluesetfuncval2}%定理
\mbox{}

1)
$c$と$f$を集合とし, $x$をこれらの中に自由変数として現れない文字とする.
このとき
\[
  c \subset {\rm pr}_{1}\langle f \rangle \to \{f(x)|x \in c\} \subset f[c]
\]
が成り立つ.
またこのことから, 次の($*$)が成り立つ: 

($*$) ~~$c \subset {\rm pr}_{1}\langle f \rangle$が成り立つならば, 
        $\{f(x)|x \in c\} \subset f[c]$が成り立つ.

2)
$c$, $f$, $x$は1)と同じとするとき, 
\[
  {\rm Func}(f) \to (c \subset {\rm pr}_{1}\langle f \rangle \to f[c] = \{f(x)|x \in c\})
\]
が成り立つ.
またこのことから, 次の($**$)が成り立つ: 

($**$) ~~$f$が函数ならば, 
         \[
           c \subset {\rm pr}_{1}\langle f \rangle \to f[c] = \{f(x)|x \in c\}
         \]
         が成り立つ.
         故にこのとき, $c \subset {\rm pr}_{1}\langle f \rangle$が成り立つならば, 
         $f[c] = \{f(x)|x \in c\}$が成り立つ.

3)
$c$, $f$, $x$は1)及び2)と同じとし, 更に$a$を集合とする.
このとき
\[
  {\rm Func}(f; a) \to (c \subset a \to f[c] = \{f(x)|x \in c\})
\]
が成り立つ.
またこのことから, 次の(${**}*$)が成り立つ: 

(${**}*$) ~~$f$が$a$における函数ならば, 
            \[
              c \subset a \to f[c] = \{f(x)|x \in c\}
            \]
            が成り立つ.
            故にこのとき, $c \subset a$が成り立つならば, 
            $f[c] = \{f(x)|x \in c\}$が成り立つ.

4)
$a$, $c$, $f$, $x$は3)と同じとし, 更に$b$を集合とする.
このとき
\[
  {\rm Func}(f; a; b) \to (c \subset a \to f[c] = \{f(x)|x \in c\})
\]
が成り立つ.
またこのことから, 次の(${**}{**}$)が成り立つ: 

(${**}{**}$) ~~$f$が$a$から$b$への函数ならば, 
               \[
                 c \subset a \to f[c] = \{f(x)|x \in c\}
               \]
               が成り立つ.
               故にこのとき, $c \subset a$が成り立つならば, 
               $f[c] = \{f(x)|x \in c\}$が成り立つ.
\end{thm}


\noindent{\bf 証明}
~1)
定理 \ref{sthmcapsubset=}と推論法則 \ref{dedequiv}により
\[
\tag{1}
  c \subset {\rm pr}_{1}\langle f \rangle \to c \cap {\rm pr}_{1}\langle f \rangle = c
\]
が成り立つ.
また定理 \ref{sthmcapch}より
$c \cap {\rm pr}_{1}\langle f \rangle = {\rm pr}_{1}\langle f \rangle \cap c$が成り立つから, 
推論法則 \ref{dedaddeq=}により
\[
  c \cap {\rm pr}_{1}\langle f \rangle = c \leftrightarrow {\rm pr}_{1}\langle f \rangle \cap c = c
\]
が成り立つ.
故に推論法則 \ref{dedequiv}により
\[
\tag{2}
  c \cap {\rm pr}_{1}\langle f \rangle = c \to {\rm pr}_{1}\langle f \rangle \cap c = c
\]
が成り立つ.
またいま$x$は$c$及び$f$の中に自由変数として現れないから, 
変数法則 \ref{valcap}, \ref{valprset}により, $x$は
${\rm pr}_{1}\langle f \rangle \cap c$の中にも自由変数として現れない.
故に定理 \ref{sthmoset=}より
\[
\tag{3}
  {\rm pr}_{1}\langle f \rangle \cap c = c 
  \to \{f(x)|x \in {\rm pr}_{1}\langle f \rangle \cap c\} = \{f(x)|x \in c\}
\]
が成り立つ.
そこで(1), (2), (3)から, 推論法則 \ref{dedmmp}によって
\[
\tag{4}
  c \subset {\rm pr}_{1}\langle f \rangle \to \{f(x)|x \in {\rm pr}_{1}\langle f \rangle \cap c\} = \{f(x)|x \in c\}
\]
が成り立つことがわかる.
また定理 \ref{sthm=tsubseteq}より
\[
  \{f(x)|x \in {\rm pr}_{1}\langle f \rangle \cap c\} = \{f(x)|x \in c\} 
  \to (\{f(x)|x \in {\rm pr}_{1}\langle f \rangle \cap c\} \subset f[c] \leftrightarrow \{f(x)|x \in c\} \subset f[c])
\]
が成り立つから, 推論法則 \ref{dedprewedge}により
\[
\tag{5}
  \{f(x)|x \in {\rm pr}_{1}\langle f \rangle \cap c\} = \{f(x)|x \in c\} 
  \to (\{f(x)|x \in {\rm pr}_{1}\langle f \rangle \cap c\} \subset f[c] \to \{f(x)|x \in c\} \subset f[c])
\]
が成り立つ.
そこで(4), (5)から, 推論法則 \ref{dedmmp}によって
\[
  c \subset {\rm pr}_{1}\langle f \rangle 
  \to (\{f(x)|x \in {\rm pr}_{1}\langle f \rangle \cap c\} \subset f[c] \to \{f(x)|x \in c\} \subset f[c])
\]
が成り立つ.
故に推論法則 \ref{dedch}により
\[
\tag{6}
  \{f(x)|x \in {\rm pr}_{1}\langle f \rangle \cap c\} \subset f[c] 
  \to (c \subset {\rm pr}_{1}\langle f \rangle \to \{f(x)|x \in c\} \subset f[c])
\]
が成り立つ.
また$x$が$c$及び$f$の中に自由変数として現れないことから, 定理 \ref{sthmvaluesetfuncval}より
\[
  \{f(x)|x \in {\rm pr}_{1}\langle f \rangle \cap c\} \subset f[c]
\]
が成り立つ.
故にこれと(6)から, 推論法則 \ref{dedmp}によって
\[
  c \subset {\rm pr}_{1}\langle f \rangle \to \{f(x)|x \in c\} \subset f[c]
\]
が成り立つ.
($*$)が成り立つことは, これと推論法則 \ref{dedmp}によって明らかである.

\noindent
2)
$x$が$c$及び$f$の中に自由変数として現れないことから, 定理 \ref{sthmvaluesetfuncval}より
\[
  {\rm Func}(f) \to f[c] = \{f(x)|x \in {\rm pr}_{1}\langle f \rangle \cap c\}
\]
が成り立つ.
そこでこれと(4)から, 推論法則 \ref{dedfromaddw}により
\[
\tag{7}
  {\rm Func}(f) \wedge c \subset {\rm pr}_{1}\langle f \rangle 
  \to f[c] = \{f(x)|x \in {\rm pr}_{1}\langle f \rangle \cap c\} 
  \wedge \{f(x)|x \in {\rm pr}_{1}\langle f \rangle \cap c\} = \{f(x)|x \in c\}
\]
が成り立つ.
またThm \ref{x=ywy=ztx=z}より
\[
\tag{8}
  f[c] = \{f(x)|x \in {\rm pr}_{1}\langle f \rangle \cap c\} 
  \wedge \{f(x)|x \in {\rm pr}_{1}\langle f \rangle \cap c\} = \{f(x)|x \in c\} 
  \to f[c] = \{f(x)|x \in c\}
\]
が成り立つ.
そこで(7), (8)から, 推論法則 \ref{dedmmp}によって
\[
\tag{9}
  {\rm Func}(f) \wedge c \subset {\rm pr}_{1}\langle f \rangle 
  \to f[c] = \{f(x)|x \in c\}
\]
が成り立つ.
故に推論法則 \ref{dedtwch}により
\[
  {\rm Func}(f) \to (c \subset {\rm pr}_{1}\langle f \rangle \to f[c] = \{f(x)|x \in c\})
\]
が成り立つ.
($**$)が成り立つことは, これと推論法則 \ref{dedmp}によって明らかである.

\noindent
3)
${\rm Func}(f; a)$の定義から, Thm \ref{1awb1wctaw1bwc1}より
\[
\tag{10}
  {\rm Func}(f; a) \wedge c \subset a \to {\rm Func}(f) \wedge ({\rm pr}_{1}\langle f \rangle = a \wedge c \subset a)
\]
が成り立つ.
また定理 \ref{sthm=&subset}より
\[
  {\rm pr}_{1}\langle f \rangle = a \wedge c \subset a \to c \subset {\rm pr}_{1}\langle f \rangle
\]
が成り立つから, 推論法則 \ref{dedaddw}により
\[
\tag{11}
  {\rm Func}(f) \wedge ({\rm pr}_{1}\langle f \rangle = a \wedge c \subset a) 
  \to {\rm Func}(f) \wedge c \subset {\rm pr}_{1}\langle f \rangle
\]
が成り立つ.
また既に示したように(9)が成り立つ.
そこで(10), (11), (9)から, 推論法則 \ref{dedmmp}によって
\[
  {\rm Func}(f; a) \wedge c \subset a \to f[c] = \{f(x)|x \in c\}
\]
が成り立つことがわかる.
故に推論法則 \ref{dedtwch}により
\[
\tag{12}
  {\rm Func}(f; a) \to (c \subset a \to f[c] = \{f(x)|x \in c\})
\]
が成り立つ.
(${**}*$)が成り立つことは, これと推論法則 \ref{dedmp}によって明らかである.

\noindent
4)
定理 \ref{sthmfuncbasis}より
${\rm Func}(f; a; b) \to {\rm Func}(f; a)$が成り立つから, 
これと(12)から, 推論法則 \ref{dedmmp}によって
\[
  {\rm Func}(f; a; b) \to (c \subset a \to f[c] = \{f(x)|x \in c\})
\]
が成り立つ.
(${**}{**}$)が成り立つことは, これと推論法則 \ref{dedmp}によって明らかである.
\halmos




\mathstrut
\begin{thm}
\label{sthmvaluesetinvfuncvalelement}%定理
\mbox{}

1)
$c$, $f$, $t$を集合とするとき, 
\[
  t \in {\rm pr}_{1}\langle f \rangle \wedge f(t) \in c \to t \in f^{-1}[c]
\]
が成り立つ.
またこのことから, 次の($*$)が成り立つ: 

($*$) ~~$t \in {\rm pr}_{1}\langle f \rangle$と$f(t) \in c$が共に成り立つならば, 
        $t \in f^{-1}[c]$が成り立つ.

2)
$c$, $f$, $t$は1)と同じとするとき, 
\[
  {\rm Func}(f) \to (t \in f^{-1}[c] \leftrightarrow t \in {\rm pr}_{1}\langle f \rangle \wedge f(t) \in c)
\]
が成り立つ.
またこのことから, 次の($**$)が成り立つ: 

($**$) ~~$f$が函数ならば, 
         \[
           t \in f^{-1}[c] \leftrightarrow t \in {\rm pr}_{1}\langle f \rangle \wedge f(t) \in c
         \]
         が成り立つ.
         故にこのとき, $t \in f^{-1}[c]$が成り立つならば, 
         $t \in {\rm pr}_{1}\langle f \rangle$と$f(t) \in c$が共に成り立つ.

3)
$c$, $f$, $t$は1)及び2)と同じとし, 更に$a$を集合とする.
このとき
\[
  {\rm Func}(f; a) \to (t \in f^{-1}[c] \leftrightarrow t \in a \wedge f(t) \in c)
\]
が成り立つ.
またこのことから, 次の(${**}*$)が成り立つ: 

(${**}*$) ~~$f$が$a$における函数ならば, 
            \[
              t \in f^{-1}[c] \leftrightarrow t \in a \wedge f(t) \in c
            \]
            が成り立つ.
            故にこのとき, $t \in f^{-1}[c]$が成り立つならば$t \in a$と$f(t) \in c$が共に成り立ち, 
            逆に$t \in a$と$f(t) \in c$が共に成り立つならば$t \in f^{-1}[c]$が成り立つ.

4)
$a$, $c$, $f$, $t$は3)と同じとし, 更に$b$を集合とする.
このとき
\[
  {\rm Func}(f; a; b) \to (t \in f^{-1}[c] \leftrightarrow t \in a \wedge f(t) \in c)
\]
が成り立つ.
またこのことから, 次の(${**}{**}$)が成り立つ: 

(${**}{**}$) ~~$f$が$a$から$b$への函数ならば, 
               \[
                 t \in f^{-1}[c] \leftrightarrow t \in a \wedge f(t) \in c
               \]
               が成り立つ.
               故にこのとき, $t \in f^{-1}[c]$が成り立つならば$t \in a$と$f(t) \in c$が共に成り立ち, 
               逆に$t \in a$と$f(t) \in c$が共に成り立つならば$t \in f^{-1}[c]$が成り立つ.
\end{thm}


\noindent{\bf 証明}
~1)
定理 \ref{sthmfuncval}と推論法則 \ref{dedequiv}により
\[
\tag{1}
  t \in {\rm pr}_{1}\langle f \rangle \to (t, f(t)) \in f
\]
が成り立つ.
また定理 \ref{sthmpairininv}と推論法則 \ref{dedequiv}により
\[
\tag{2}
  (t, f(t)) \in f \to (f(t), t) \in f^{-1}
\]
が成り立つ.
そこで(1), (2)から, 推論法則 \ref{dedmmp}によって
\[
  t \in {\rm pr}_{1}\langle f \rangle \to (f(t), t) \in f^{-1}
\]
が成り立つ.
故に推論法則 \ref{dedaddw}により
\[
\tag{3}
  t \in {\rm pr}_{1}\langle f \rangle \wedge f(t) \in c \to (f(t), t) \in f^{-1} \wedge f(t) \in c
\]
が成り立つ.
またThm \ref{awbtbwa}より
\[
\tag{4}
  (f(t), t) \in f^{-1} \wedge f(t) \in c \to f(t) \in c \wedge (f(t), t) \in f^{-1}
\]
が成り立つ.
また定理 \ref{sthmvaluesetbasis}より
\[
\tag{5}
  f(t) \in c \wedge (f(t), t) \in f^{-1} \to t \in f^{-1}[c]
\]
が成り立つ.
そこで(3), (4), (5)から, 推論法則 \ref{dedmmp}によって
\[
\tag{6}
  t \in {\rm pr}_{1}\langle f \rangle \wedge f(t) \in c \to t \in f^{-1}[c]
\]
が成り立つことがわかる.
($*$)が成り立つことは, これと推論法則 \ref{dedmp}, \ref{dedwedge}によって明らかである.

\noindent
2)
$y$を$c$, $f$, $t$のいずれの記号列の中にも自由変数として現れない文字とする.
このとき変数法則 \ref{valinv}により, $y$は$f^{-1}$の中にも自由変数として現れないから, 
定理 \ref{sthmvaluesetelement}と推論法則 \ref{dedequiv}により
\[
  t \in f^{-1}[c] \to \exists y(y \in c \wedge (y, t) \in f^{-1})
\]
が成り立つ.
ここで$\tau_{y}(y \in c \wedge (y, t) \in f^{-1})$を$T$と書けば, $T$は集合であり, 
定義から上記の記号列は
\[
  t \in f^{-1}[c] \to (T|y)(y \in c \wedge (y, t) \in f^{-1})
\]
と同じである.
またいま述べたように, $y$は$c$, $t$, $f^{-1}$のいずれの記号列の中にも自由変数として現れないから, 
代入法則 \ref{substfree}, \ref{substfund}, \ref{substwedge}, \ref{substpair}によれば, 
この記号列は
\[
\tag{7}
  t \in f^{-1}[c] \to T \in c \wedge (T, t) \in f^{-1}
\]
と一致する.
故にこれが定理となる.
またThm \ref{awbtbwa}より
\[
\tag{8}
  T \in c \wedge (T, t) \in f^{-1} \to (T, t) \in f^{-1} \wedge T \in c
\]
が成り立つ.
また定理 \ref{sthmpairininv}と推論法則 \ref{dedequiv}により
\[
  (T, t) \in f^{-1} \to (t, T) \in f
\]
が成り立つから, 推論法則 \ref{dedaddw}により
\[
\tag{9}
  (T, t) \in f^{-1} \wedge T \in c \to (t, T) \in f \wedge T \in c
\]
が成り立つ.
そこで(7), (8), (9)から, 推論法則 \ref{dedmmp}によって
\[
  t \in f^{-1}[c] \to (t, T) \in f \wedge T \in c
\]
が成り立つことがわかる.
故に推論法則 \ref{dedaddw}により
\[
\tag{10}
  {\rm Func}(f) \wedge t \in f^{-1}[c] \to {\rm Func}(f) \wedge ((t, T) \in f \wedge T \in c)
\]
が成り立つ.
またThm \ref{aw1bwc1t1awb1wc}より
\[
\tag{11}
  {\rm Func}(f) \wedge ((t, T) \in f \wedge T \in c) 
  \to ({\rm Func}(f) \wedge (t, T) \in f) \wedge T \in c
\]
が成り立つ.
また定理 \ref{sthmfuncvalbasis}より
\[
  {\rm Func}(f) \to ((t, T) \in f \leftrightarrow t \in {\rm pr}_{1}\langle f \rangle \wedge T = f(t))
\]
が成り立つから, 推論法則 \ref{dedprewedge}により
\[
  {\rm Func}(f) \to ((t, T) \in f \to t \in {\rm pr}_{1}\langle f \rangle \wedge T = f(t))
\]
が成り立ち, これから推論法則 \ref{dedtwch}により
\[
  {\rm Func}(f) \wedge (t, T) \in f \to t \in {\rm pr}_{1}\langle f \rangle \wedge T = f(t)
\]
が成り立つ.
故に推論法則 \ref{dedaddw}により
\[
\tag{12}
  ({\rm Func}(f) \wedge (t, T) \in f) \wedge T \in c 
  \to (t \in {\rm pr}_{1}\langle f \rangle \wedge T = f(t)) \wedge T \in c
\]
が成り立つ.
またThm \ref{1awb1wctaw1bwc1}より
\[
\tag{13}
  (t \in {\rm pr}_{1}\langle f \rangle \wedge T = f(t)) \wedge T \in c 
  \to t \in {\rm pr}_{1}\langle f \rangle \wedge (T = f(t) \wedge T \in c)
\]
が成り立つ.
また定理 \ref{sthm=&in}より
\[
  T = f(t) \wedge T \in c \to f(t) \in c
\]
が成り立つから, 推論法則 \ref{dedaddw}により
\[
\tag{14}
  t \in {\rm pr}_{1}\langle f \rangle \wedge (T = f(t) \wedge T \in c) 
  \to t \in {\rm pr}_{1}\langle f \rangle \wedge f(t) \in c
\]
が成り立つ.
そこで(10)---(14)から, 推論法則 \ref{dedmmp}によって
\[
  {\rm Func}(f) \wedge t \in f^{-1}[c] \to t \in {\rm pr}_{1}\langle f \rangle \wedge f(t) \in c
\]
が成り立つことがわかる.
故に推論法則 \ref{dedtwch}により
\[
\tag{15}
  {\rm Func}(f) \to (t \in f^{-1}[c] \to t \in {\rm pr}_{1}\langle f \rangle \wedge f(t) \in c)
\]
が成り立つ.
また(6)が成り立つことから, 推論法則 \ref{dedatawbtrue2}により
\[
\tag{16}
  (t \in f^{-1}[c] \to t \in {\rm pr}_{1}\langle f \rangle \wedge f(t) \in c) 
  \to (t \in f^{-1}[c] \leftrightarrow t \in {\rm pr}_{1}\langle f \rangle \wedge f(t) \in c)
\]
が成り立つ.
そこで(15), (16)から, 推論法則 \ref{dedmmp}によって
\[
\tag{17}
  {\rm Func}(f) \to (t \in f^{-1}[c] \leftrightarrow t \in {\rm pr}_{1}\langle f \rangle \wedge f(t) \in c)
\]
が成り立つ.
($**$)が成り立つことは, これと推論法則 \ref{dedmp}, \ref{dedwedge}, \ref{dedeqfund}によって明らかである.

\noindent
3)
定理 \ref{sthm=tineq}より
\[
\tag{18}
  {\rm pr}_{1}\langle f \rangle = a \to (t \in {\rm pr}_{1}\langle f \rangle \leftrightarrow t \in a)
\]
が成り立つ.
またThm \ref{1alb1t1awclbwc1}より
\[
\tag{19}
  (t \in {\rm pr}_{1}\langle f \rangle \leftrightarrow t \in a) 
  \to (t \in {\rm pr}_{1}\langle f \rangle \wedge f(t) \in c \leftrightarrow t \in a \wedge f(t) \in c)
\]
が成り立つ.
そこで(18), (19)から, 推論法則 \ref{dedmmp}によって
\[
  {\rm pr}_{1}\langle f \rangle = a 
  \to (t \in {\rm pr}_{1}\langle f \rangle \wedge f(t) \in c \leftrightarrow t \in a \wedge f(t) \in c)
\]
が成り立つ.
故にこれと(17)から, 推論法則 \ref{dedfromaddw}により
\begin{multline*}
  {\rm Func}(f) \wedge {\rm pr}_{1}\langle f \rangle = a \\
  \to (t \in f^{-1}[c] \leftrightarrow t \in {\rm pr}_{1}\langle f \rangle \wedge f(t) \in c) 
  \wedge (t \in {\rm pr}_{1}\langle f \rangle \wedge f(t) \in c \leftrightarrow t \in a \wedge f(t) \in c), 
\end{multline*}
即ち
\[
\tag{20}
  {\rm Func}(f; a)
  \to (t \in f^{-1}[c] \leftrightarrow t \in {\rm pr}_{1}\langle f \rangle \wedge f(t) \in c) 
  \wedge (t \in {\rm pr}_{1}\langle f \rangle \wedge f(t) \in c \leftrightarrow t \in a \wedge f(t) \in c)
\]
が成り立つ.
またThm \ref{1alb1w1blc1t1alc1}より
\begin{multline*}
\tag{21}
  (t \in f^{-1}[c] \leftrightarrow t \in {\rm pr}_{1}\langle f \rangle \wedge f(t) \in c) 
  \wedge (t \in {\rm pr}_{1}\langle f \rangle \wedge f(t) \in c \leftrightarrow t \in a \wedge f(t) \in c) \\
  \to (t \in f^{-1}[c] \leftrightarrow t \in a \wedge f(t) \in c)
\end{multline*}
が成り立つ.
そこで(20), (21)から, 推論法則 \ref{dedmmp}によって
\[
\tag{22}
  {\rm Func}(f; a) \to (t \in f^{-1}[c] \leftrightarrow t \in a \wedge f(t) \in c)
\]
が成り立つ.
(${**}*$)が成り立つことは, これと推論法則 \ref{dedmp}, \ref{dedwedge}, \ref{dedeqfund}によって明らかである.

\noindent
4)
定理 \ref{sthmfuncbasis}より
${\rm Func}(f; a; b) \to {\rm Func}(f; a)$が成り立つから, これと(22)から, 
推論法則 \ref{dedmmp}によって
\[
  {\rm Func}(f; a; b) \to (t \in f^{-1}[c] \leftrightarrow t \in a \wedge f(t) \in c)
\]
が成り立つ.
(${**}{**}$)が成り立つことは, これと推論法則 \ref{dedmp}, \ref{dedwedge}, \ref{dedeqfund}によって明らかである.
\halmos




\mathstrut
\begin{thm}
\label{sthmvaluesetinvfuncval}%定理
\mbox{} 

1)
$c$と$f$を集合とし, $x$をこれらの中に自由変数として現れない文字とする.
このとき
\[
  \{x \in {\rm pr}_{1}\langle f \rangle|f(x) \in c\} \subset f^{-1}[c]
\]
が成り立つ.

2)
$c$, $f$, $x$は1)と同じとするとき, 
\[
  {\rm Func}(f) \to f^{-1}[c] = \{x \in {\rm pr}_{1}\langle f \rangle|f(x) \in c\}
\]
が成り立つ.
またこのことから, 次の($*$)が成り立つ: 

($*$) ~~$f$が函数ならば, 
        $f^{-1}[c] = \{x \in {\rm pr}_{1}\langle f \rangle|f(x) \in c\}$が成り立つ.

3)
$a$, $c$, $f$を集合とし, $x$をこれらの中に自由変数として現れない文字とする.
このとき
\[
  {\rm Func}(f; a) \to f^{-1}[c] = \{x \in a|f(x) \in c\}
\]
が成り立つ.
またこのことから, 次の($**$)が成り立つ: 

($**$) ~~$f$が$a$における函数ならば, 
         $f^{-1}[c] = \{x \in a|f(x) \in c\}$が成り立つ.

4)
$a$, $c$, $f$, $x$は3)と同じとし, 更に$b$を集合とする.
このとき
\[
  {\rm Func}(f; a; b) \to f^{-1}[c] = \{x \in a|f(x) \in c\}
\]
が成り立つ.
またこのことから, 次の(${**}*$)が成り立つ: 

(${**}*$) ~~$f$が$a$から$b$への函数ならば, 
            $f^{-1}[c] = \{x \in a|f(x) \in c\}$が成り立つ.
\end{thm}


\noindent{\bf 証明}
~1)
$x$が$f$の中に自由変数として現れないことから, 変数法則 \ref{valprset}により, 
$x$は${\rm pr}_{1}\langle f \rangle$の中に自由変数として現れない.
そこでいま$y$を$x$と異なり, $c$及び$f$の中に自由変数として現れない, 定数でない文字とすれば, 
定理 \ref{sthmssetbasis}より
\[
  y \in \{x \in {\rm pr}_{1}\langle f \rangle|f(x) \in c\} 
  \leftrightarrow y \in {\rm pr}_{1}\langle f \rangle \wedge (y|x)(f(x) \in c)
\]
が成り立つ.
ここで$x$が$c$及び$f$の中に自由変数として現れないことから, 
代入法則 \ref{substfree}, \ref{substfund}, \ref{substfuncval}により, 上記の記号列は
\[
\tag{1}
  y \in \{x \in {\rm pr}_{1}\langle f \rangle|f(x) \in c\} 
  \leftrightarrow y \in {\rm pr}_{1}\langle f \rangle \wedge f(y) \in c
\]
と一致する.
故にこれが定理となる.
そこで推論法則 \ref{dedequiv}により
\[
\tag{2}
  y \in \{x \in {\rm pr}_{1}\langle f \rangle|f(x) \in c\} 
  \to y \in {\rm pr}_{1}\langle f \rangle \wedge f(y) \in c
\]
が成り立つ.
また定理 \ref{sthmvaluesetinvfuncvalelement}より
\[
\tag{3}
  y \in {\rm pr}_{1}\langle f \rangle \wedge f(y) \in c 
  \to y \in f^{-1}[c]
\]
が成り立つ.
そこで(2), (3)から, 推論法則 \ref{dedmmp}によって
\[
\tag{4}
  y \in \{x \in {\rm pr}_{1}\langle f \rangle|f(x) \in c\} \to y \in f^{-1}[c]
\]
が成り立つ.
さていま$y$は$x$と異なり, $c$及び$f$の中に自由変数として現れないから, 
変数法則 \ref{valfund}, \ref{valsset}, \ref{valprset}, \ref{valfuncval}によってわかるように, 
$y$は$\{x \in {\rm pr}_{1}\langle f \rangle|f(x) \in c\}$の中に自由変数として現れず, 
変数法則 \ref{valvalueset}, \ref{valinv}によってわかるように, $y$は$f^{-1}[c]$の中にも自由変数として現れない.
また$y$は定数でない.
これらのことと, (4)が成り立つことから, 定理 \ref{sthmsubsetconst}により
\[
  \{x \in {\rm pr}_{1}\langle f \rangle|f(x) \in c\} \subset f^{-1}[c]
\]
が成り立つ.

\noindent
2)
$y$は上と同じとするとき, 定理 \ref{sthmvaluesetinvfuncvalelement}より
\[
\tag{5}
  {\rm Func}(f) \to (y \in f^{-1}[c] \leftrightarrow y \in {\rm pr}_{1}\langle f \rangle \wedge f(y) \in c)
\]
が成り立つ.
また示したように(1)が成り立つから, 推論法則 \ref{dedeqch}により
\[
  y \in {\rm pr}_{1}\langle f \rangle \wedge f(y) \in c 
  \leftrightarrow y \in \{x \in {\rm pr}_{1}\langle f \rangle|f(x) \in c\}
\]
が成り立つ.
故に推論法則 \ref{dedatawbtrue2}により
\begin{multline*}
\tag{6}
  (y \in f^{-1}[c] \leftrightarrow y \in {\rm pr}_{1}\langle f \rangle \wedge f(y) \in c) \\
  \to (y \in f^{-1}[c] \leftrightarrow y \in {\rm pr}_{1}\langle f \rangle \wedge f(y) \in c) 
  \wedge (y \in {\rm pr}_{1}\langle f \rangle \wedge f(y) \in c 
  \leftrightarrow y \in \{x \in {\rm pr}_{1}\langle f \rangle|f(x) \in c\})
\end{multline*}
が成り立つ.
またThm \ref{1alb1w1blc1t1alc1}より
\begin{multline*}
\tag{7}
  (y \in f^{-1}[c] \leftrightarrow y \in {\rm pr}_{1}\langle f \rangle \wedge f(y) \in c) 
  \wedge (y \in {\rm pr}_{1}\langle f \rangle \wedge f(y) \in c 
  \leftrightarrow y \in \{x \in {\rm pr}_{1}\langle f \rangle|f(x) \in c\}) \\
  \to (y \in f^{-1}[c] \leftrightarrow y \in \{x \in {\rm pr}_{1}\langle f \rangle|f(x) \in c\})
\end{multline*}
が成り立つ.
そこで(5), (6), (7)から, 推論法則 \ref{dedmmp}によって
\[
\tag{8}
  {\rm Func}(f) \to (y \in f^{-1}[c] \leftrightarrow y \in \{x \in {\rm pr}_{1}\langle f \rangle|f(x) \in c\})
\]
が成り立つことがわかる.
いま$y$は$f$の中に自由変数として現れないから, 変数法則 \ref{valfunc}により, 
$y$は${\rm Func}(f)$の中にも自由変数として現れない.
また$y$は定数でない.
故にこれらのことと, (8)が成り立つことから, 推論法則 \ref{dedalltquansepfreeconst}により
\[
\tag{9}
  {\rm Func}(f) \to \forall y(y \in f^{-1}[c] \leftrightarrow y \in \{x \in {\rm pr}_{1}\langle f \rangle|f(x) \in c\})
\]
が成り立つ.
また上述のように, $y$は$f^{-1}[c]$及び$\{x \in {\rm pr}_{1}\langle f \rangle|f(x) \in c\}$の中に自由変数として現れないから, 
定理 \ref{sthmset=}と推論法則 \ref{dedequiv}により
\[
\tag{10}
  \forall y(y \in f^{-1}[c] \leftrightarrow y \in \{x \in {\rm pr}_{1}\langle f \rangle|f(x) \in c\}) 
  \to f^{-1}[c] = \{x \in {\rm pr}_{1}\langle f \rangle|f(x) \in c\}
\]
が成り立つ.
そこで(9), (10)から, 推論法則 \ref{dedmmp}によって
\[
\tag{11}
  {\rm Func}(f) \to f^{-1}[c] = \{x \in {\rm pr}_{1}\langle f \rangle|f(x) \in c\}
\]
が成り立つ.
($*$)が成り立つことは, これと推論法則 \ref{dedmp}によって明らかである.

\noindent
3)
$x$は$f$の中に自由変数として現れないから, 変数法則 \ref{valprset}により, 
$x$は${\rm pr}_{1}\langle f \rangle$の中に自由変数として現れない.
また$x$は$a$の中にも自由変数として現れない.
そこで定理 \ref{sthmsset=}より
\[
\tag{12}
  {\rm pr}_{1}\langle f \rangle = a \to \{x \in {\rm pr}_{1}\langle f \rangle|f(x) \in c\} = \{x \in a|f(x) \in c\}
\]
が成り立つ.
また$x$は$c$及び$f$の中に自由変数として現れないから, 上で示したように(11)が成り立つ.
そこで(11), (12)から, 推論法則 \ref{dedfromaddw}により
\[
  {\rm Func}(f) \wedge {\rm pr}_{1}\langle f \rangle = a 
  \to f^{-1}[c] = \{x \in {\rm pr}_{1}\langle f \rangle|f(x) \in c\} 
  \wedge \{x \in {\rm pr}_{1}\langle f \rangle|f(x) \in c\} = \{x \in a|f(x) \in c\}, 
\]
即ち
\[
\tag{13}
  {\rm Func}(f; a) 
  \to f^{-1}[c] = \{x \in {\rm pr}_{1}\langle f \rangle|f(x) \in c\} 
  \wedge \{x \in {\rm pr}_{1}\langle f \rangle|f(x) \in c\} = \{x \in a|f(x) \in c\}
\]
が成り立つ.
またThm \ref{x=ywy=ztx=z}より
\begin{multline*}
\tag{14}
  f^{-1}[c] = \{x \in {\rm pr}_{1}\langle f \rangle|f(x) \in c\} 
  \wedge \{x \in {\rm pr}_{1}\langle f \rangle|f(x) \in c\} = \{x \in a|f(x) \in c\} \\
  \to f^{-1}[c] = \{x \in a|f(x) \in c\}
\end{multline*}
が成り立つ.
そこで(13), (14)から, 推論法則 \ref{dedmmp}によって
\[
\tag{15}
  {\rm Func}(f; a) \to f^{-1}[c] = \{x \in a|f(x) \in c\}
\]
が成り立つ.
($**$)が成り立つことは, これと推論法則 \ref{dedmp}によって明らかである.

\noindent
4)
定理 \ref{sthmfuncbasis}より${\rm Func}(f; a; b) \to {\rm Func}(f; a)$が成り立つから, 
これと(15)から, 推論法則 \ref{dedmmp}によって
\[
  {\rm Func}(f; a; b) \to f^{-1}[c] = \{x \in a|f(x) \in c\}
\]
が成り立つ.
(${**}*$)が成り立つことは, これと推論法則 \ref{dedmp}によって明らかである.
\halmos




\mathstrut
\begin{thm}
\label{sthmcompfuncval}%定理
\mbox{}

1)
$a$, $b$, $f$, $g$, $t$を集合とするとき, 
\[
  {\rm Func}(f; a; b) \wedge {\rm Func}(g; b) \to (t \in a \to (g \circ f)(t) = g(f(t)))
\]
が成り立つ.
またこのことから, 次の($*$)が成り立つ: 

($*$) ~~$f$が$a$から$b$への函数であり, $g$が$b$における函数ならば, 
        \[
          t \in a \to (g \circ f)(t) = g(f(t))
        \]
        が成り立つ.
        故にこのとき, $t \in a$が成り立つならば, 
        $(g \circ f)(t) = g(f(t))$が成り立つ.

2)
$a$, $b$, $f$, $g$, $t$は1)と同じとし, 更に$c$を集合とする.
このとき
\[
  {\rm Func}(f; a; b) \wedge {\rm Func}(g; b; c) \to (t \in a \to (g \circ f)(t) = g(f(t)))
\]
が成り立つ.
またこのことから, 次の($**$)が成り立つ: 

($**$) ~~$f$が$a$から$b$への函数であり, $g$が$b$から$c$への函数ならば, 
         \[
           t \in a \to (g \circ f)(t) = g(f(t))
         \]
         が成り立つ.
         故にこのとき, $t \in a$が成り立つならば, 
         $(g \circ f)(t) = g(f(t))$が成り立つ.
\end{thm}


\noindent{\bf 証明}
~1)
Thm \ref{awbta}より
\begin{align*}
  \tag{1}
  ({\rm Func}(f; a; b) \wedge {\rm Func}(g; b)) \wedge t \in a &\to {\rm Func}(f; a; b) \wedge {\rm Func}(g; b), \\
  \mbox{} \\
  \tag{2}
  ({\rm Func}(f; a; b) \wedge {\rm Func}(g; b)) \wedge t \in a &\to t \in a
\end{align*}
が共に成り立つ.
故にこの(1)から, 推論法則 \ref{dedprewedge}により
\begin{align*}
  \tag{3}
  ({\rm Func}(f; a; b) \wedge {\rm Func}(g; b)) \wedge t \in a &\to {\rm Func}(f; a; b), \\
  \mbox{} \\
  \tag{4}
  ({\rm Func}(f; a; b) \wedge {\rm Func}(g; b)) \wedge t \in a &\to {\rm Func}(g; b)
\end{align*}
が共に成り立つ.
そこで(3)と(2)から, 再び推論法則 \ref{dedprewedge}により
\[
\tag{5}
  ({\rm Func}(f; a; b) \wedge {\rm Func}(g; b)) \wedge t \in a \to {\rm Func}(f; a; b) \wedge t \in a
\]
が成り立つ.
また定理 \ref{sthmfuncval}より
\[
  {\rm Func}(f; a; b) \to (t \in a \leftrightarrow (t, f(t)) \in f)
\]
が成り立つから, やはり推論法則 \ref{dedprewedge}により
\[
  {\rm Func}(f; a; b) \to (t \in a \to (t, f(t)) \in f)
\]
が成り立ち, これから推論法則 \ref{dedtwch}により
\[
  {\rm Func}(f; a; b) \wedge t \in a \to (t, f(t)) \in f
\]
が成り立つ.
そこでこれと(5)から, 推論法則 \ref{dedmmp}によって
\[
\tag{6}
  ({\rm Func}(f; a; b) \wedge {\rm Func}(g; b)) \wedge t \in a \to (t, f(t)) \in f
\]
が成り立つ.
また定理 \ref{sthmfuncvalinprset}より
\[
  {\rm Func}(f; a; b) \to (t \in a \to f(t) \in b)
\]
が成り立つから, 推論法則 \ref{dedtwch}により
\[
  {\rm Func}(f; a; b) \wedge t \in a \to f(t) \in b
\]
が成り立つ.
そこでこれと(5)から, 推論法則 \ref{dedmmp}によって
\[
  ({\rm Func}(f; a; b) \wedge {\rm Func}(g; b)) \wedge t \in a \to f(t) \in b
\]
が成り立つ.
故にこれと(4)から, 推論法則 \ref{dedprewedge}により
\[
\tag{7}
  ({\rm Func}(f; a; b) \wedge {\rm Func}(g; b)) \wedge t \in a \to {\rm Func}(g; b) \wedge f(t) \in b
\]
が成り立つ.
また定理 \ref{sthmfuncval}より
\[
  {\rm Func}(g; b) \to (f(t) \in b \leftrightarrow (f(t), g(f(t))) \in g)
\]
が成り立つから, 推論法則 \ref{dedprewedge}により
\[
  {\rm Func}(g; b) \to (f(t) \in b \to (f(t), g(f(t))) \in g)
\]
が成り立ち, これから推論法則 \ref{dedtwch}により
\[
  {\rm Func}(g; b) \wedge f(t) \in b \to (f(t), g(f(t))) \in g
\]
が成り立つ.
そこでこれと(7)から, 推論法則 \ref{dedmmp}によって
\[
  ({\rm Func}(f; a; b) \wedge {\rm Func}(g; b)) \wedge t \in a \to (f(t), g(f(t))) \in g
\]
が成り立つ.
故にこれと(6)から, 推論法則 \ref{dedprewedge}により
\[
\tag{8}
  ({\rm Func}(f; a; b) \wedge {\rm Func}(g; b)) \wedge t \in a \to (t, f(t)) \in f \wedge (f(t), g(f(t))) \in g
\]
が成り立つ.
また定理 \ref{sthmpairincompt}より
\[
  (t, f(t)) \in f \wedge (f(t), g(f(t))) \in g \to (t, g(f(t))) \in g \circ f
\]
が成り立つ.
そこでこれと(8)から, 推論法則 \ref{dedmmp}によって
\[
\tag{9}
  ({\rm Func}(f; a; b) \wedge {\rm Func}(g; b)) \wedge t \in a \to (t, g(f(t))) \in g \circ f
\]
が成り立つ.
また定理 \ref{sthmcompfunc}より
\[
  {\rm Func}(f; a; b) \wedge {\rm Func}(g; b) \to {\rm Func}(g \circ f; a)
\]
が成り立つから, これと(1)から, 推論法則 \ref{dedmmp}によって
\[
  ({\rm Func}(f; a; b) \wedge {\rm Func}(g; b)) \wedge t \in a \to {\rm Func}(g \circ f; a)
\]
が成り立つ.
そこでこれと(9)から, 推論法則 \ref{dedprewedge}により
\[
\tag{10}
  ({\rm Func}(f; a; b) \wedge {\rm Func}(g; b)) \wedge t \in a \to {\rm Func}(g \circ f; a) \wedge (t, g(f(t))) \in g \circ f
\]
が成り立つ.
また定理 \ref{sthmfuncvalbasispractical}より
\[
  {\rm Func}(g \circ f; a) \to ((t, g(f(t))) \in g \circ f \to g(f(t)) = (g \circ f)(t))
\]
が成り立つから, 推論法則 \ref{dedtwch}により
\[
\tag{11}
  {\rm Func}(g \circ f; a) \wedge (t, g(f(t))) \in g \circ f \to g(f(t)) = (g \circ f)(t)
\]
が成り立つ.
またThm \ref{x=yty=x}より
\[
\tag{12}
  g(f(t)) = (g \circ f)(t) \to (g \circ f)(t) = g(f(t))
\]
が成り立つ.
そこで(10), (11), (12)から, 推論法則 \ref{dedmmp}によって
\[
  ({\rm Func}(f; a; b) \wedge {\rm Func}(g; b)) \wedge t \in a \to (g \circ f)(t) = g(f(t))
\]
が成り立つことがわかる.
故にこれから, 推論法則 \ref{dedtwch}により
\[
\tag{13}
  {\rm Func}(f; a; b) \wedge {\rm Func}(g; b) \to (t \in a \to (g \circ f)(t) = g(f(t)))
\]
が成り立つ.
($*$)が成り立つことは, これと推論法則 \ref{dedmp}, \ref{dedwedge}によって明らかである.

\noindent
2)
定理 \ref{sthmfuncbasis}より${\rm Func}(g; b; c) \to {\rm Func}(g; b)$が成り立つから, 
推論法則 \ref{dedaddw}により
\[
  {\rm Func}(f; a; b) \wedge {\rm Func}(g; b; c) \to {\rm Func}(f; a; b) \wedge {\rm Func}(g; b)
\]
が成り立つ.
そこでこれと(13)から, 推論法則 \ref{dedmmp}によって
\[
  {\rm Func}(f; a; b) \wedge {\rm Func}(g; b; c) \to (t \in a \to (g \circ f)(t) = g(f(t)))
\]
が成り立つ.
($**$)が成り立つことは, これと推論法則 \ref{dedmp}, \ref{dedwedge}によって明らかである.
\halmos




\mathstrut
\begin{thm}
\label{sthmidenfuncval}%定理
$a$と$t$を集合とするとき, 
\[
  t \in a \to {\rm id}_{a}(t) = t
\]
が成り立つ.
またこのことから, 次の($*$)が成り立つ: 

($*$) ~~$t \in a$が成り立つならば, ${\rm id}_{a}(t) = t$が成り立つ.
\end{thm}


\noindent{\bf 証明}
~定理 \ref{sthmpairiniden2}と推論法則 \ref{dedequiv}により
\[
\tag{1}
  t \in a \to (t, t) \in {\rm id}_{a}
\]
が成り立つ.
また定理 \ref{sthmidenfunc}より${\rm id}_{a}$は函数だから, 
定理 \ref{sthmfuncvalbasispractical}により
\[
\tag{2}
  (t, t) \in {\rm id}_{a} \to t = {\rm id}_{a}(t)
\]
が成り立つ.
またThm \ref{x=yty=x}より
\[
\tag{3}
  t = {\rm id}_{a}(t) \to {\rm id}_{a}(t) = t
\]
が成り立つ.
そこで(1), (2), (3)から, 推論法則 \ref{dedmmp}によって
\[
  t \in a \to {\rm id}_{a}(t) = t
\]
が成り立つことがわかる.
($*$)が成り立つことは, これと推論法則 \ref{dedmp}によって明らかである.
\halmos
%$f(t)$確認済



\mathstrut
\begin{defi}
\label{defrest}%定義
$a$と$f$を記号列とするとき, $f \cap (a \times {\rm pr}_{2}\langle f \rangle)$という記号列を, 
$(f)|_{a}$と書き表す(括弧は適宜省略する).
\end{defi}




\mathstrut
\begin{valu}
\label{valrest}%変数
$a$と$f$を記号列とし, $x$を文字とする.
$x$が$a$及び$f$の中に自由変数として現れなければ, $x$は$f|_{a}$の中に自由変数として現れない.
\end{valu}


\noindent{\bf 証明}
~定義より$f|_{a}$は$f \cap (a \times {\rm pr}_{2}\langle f \rangle)$であるから, 
変数法則 \ref{valcap}, \ref{valproduct}, \ref{valprset}によってわかるように, 
$x$はこの中に自由変数として現れない.
\halmos




\mathstrut
\begin{subs}
\label{substrest}%代入
$a$, $b$, $f$を記号列とし, $x$を文字とするとき, 
\[
  (b|x)(f|_{a}) \equiv (b|x)(f)|_{(b|x)(a)}
\]
が成り立つ.
\end{subs}


\noindent{\bf 証明}
~定義より$f|_{a}$は$f \cap (a \times {\rm pr}_{2}\langle f \rangle)$であるから, 
代入法則 \ref{substcap}, \ref{substproduct}, \ref{substprset}によれば, $(b|x)(f|_{a})$は
\[
  (b|x)(f) \cap ((b|x)(a) \times {\rm pr}_{2}\langle (b|x)(f) \rangle)
\]
と一致する.
再び定義によれば, これは$(b|x)(f)|_{(b|x)(a)}$と書き表される記号列である.
\halmos




\mathstrut
\begin{form}
\label{formrest}%構成
$a$と$f$が集合ならば, $f|_{a}$は集合である.
\end{form}


\noindent{\bf 証明}
~定義より$f|_{a}$は$f \cap (a \times {\rm pr}_{2}\langle f \rangle)$である.
$a$と$f$が集合ならば, 構成法則 \ref{formcap}, \ref{formproduct}, \ref{formprset}によって
直ちに分かるように, これは集合である.
\halmos




\mathstrut
$a$と$f$が集合であるとき, 集合$f|_{a}$を, \textbf{${\bm f}$の${\bm a}$への制限}という.




\mathstrut
\begin{thm}
\label{sthmrestbasis}%定理
$a$と$f$を集合とするとき, $f|_{a}$はグラフである.
また
\[
  f|_{a} \subset f, ~~
  f|_{a} \subset a \times {\rm pr}_{2}\langle f \rangle
\]
が成り立つ.
\end{thm}


\noindent{\bf 証明}
~定義より$f|_{a}$は$f \cap (a \times {\rm pr}_{2}\langle f \rangle)$であるから, 
定理 \ref{sthmcap}より
\[
  f|_{a} \subset f, ~~
  f|_{a} \subset a \times {\rm pr}_{2}\langle f \rangle
\]
が共に成り立つ.
そこでこの後者から, 定理 \ref{sthmproductsubsetgraph}により, $f|_{a}$がグラフであることがわかる.
\halmos




\mathstrut
\begin{thm}
\label{sthmpairinrest}%定理
$a$, $f$, $t$, $u$を集合とするとき, 
\[
  (t, u) \in f|_{a} \leftrightarrow t \in a \wedge (t, u) \in f
\]
が成り立つ.
またこのことから, 次の($*$)が成り立つ: 

($*$) ~~$(t, u) \in f|_{a}$が成り立つならば, $t \in a$と$(t, u) \in f$が共に成り立つ.
        逆に$t \in a$と$(t, u) \in f$が共に成り立つならば, $(t, u) \in f|_{a}$が成り立つ.
\end{thm}


\noindent{\bf 証明}
~定義より$f|_{a}$は$f \cap (a \times {\rm pr}_{2}\langle f \rangle)$だから, 
定理 \ref{sthmcapelement}より
\[
\tag{1}
  (t, u) \in f|_{a} \leftrightarrow (t, u) \in f \wedge (t, u) \in a \times {\rm pr}_{2}\langle f \rangle
\]
が成り立つ.
また定理 \ref{sthmpairinproduct}より
\[
\tag{2}
  (t, u) \in a \times {\rm pr}_{2}\langle f \rangle \leftrightarrow t \in a \wedge u \in {\rm pr}_{2}\langle f \rangle
\]
が成り立つ.
またThm \ref{awblbwa}より
\[
\tag{3}
  t \in a \wedge u \in {\rm pr}_{2}\langle f \rangle \leftrightarrow u \in {\rm pr}_{2}\langle f \rangle \wedge t \in a
\]
が成り立つ.
そこで(2), (3)から, 推論法則 \ref{dedeqtrans}によって
\[
  (t, u) \in a \times {\rm pr}_{2}\langle f \rangle \leftrightarrow u \in {\rm pr}_{2}\langle f \rangle \wedge t \in a
\]
が成り立つ.
故に推論法則 \ref{dedaddeqw}により
\[
\tag{4}
  (t, u) \in f \wedge (t, u) \in a \times {\rm pr}_{2}\langle f \rangle 
  \leftrightarrow (t, u) \in f \wedge (u \in {\rm pr}_{2}\langle f \rangle \wedge t \in a)
\]
が成り立つ.
またThm \ref{1awb1wclaw1bwc1}と推論法則 \ref{dedeqch}により
\[
\tag{5}
  (t, u) \in f \wedge (u \in {\rm pr}_{2}\langle f \rangle \wedge t \in a) 
  \leftrightarrow ((t, u) \in f \wedge u \in {\rm pr}_{2}\langle f \rangle) \wedge t \in a
\]
が成り立つ.
また定理 \ref{sthmpairelementinprset}より
\[
  (t, u) \in f \to t \in {\rm pr}_{1}\langle f \rangle \wedge u \in {\rm pr}_{2}\langle f \rangle
\]
が成り立つから, 推論法則 \ref{dedprewedge}により
\[
  (t, u) \in f \to u \in {\rm pr}_{2}\langle f \rangle
\]
が成り立つ.
故に推論法則 \ref{dedawblatrue1}により
\[
  (t, u) \in f \wedge u \in {\rm pr}_{2}\langle f \rangle \leftrightarrow (t, u) \in f
\]
が成り立つ.
故に推論法則 \ref{dedaddeqw}により
\[
\tag{6}
  ((t, u) \in f \wedge u \in {\rm pr}_{2}\langle f \rangle) \wedge t \in a \leftrightarrow (t, u) \in f \wedge t \in a
\]
が成り立つ.
またThm \ref{awblbwa}より
\[
\tag{7}
  (t, u) \in f \wedge t \in a \leftrightarrow t \in a \wedge (t, u) \in f
\]
が成り立つ.
そこで(1), (4)---(7)から, 推論法則 \ref{dedeqtrans}によって
\[
  (t, u) \in f|_{a} \leftrightarrow t \in a \wedge (t, u) \in f
\]
が成り立つことがわかる.
($*$)が成り立つことは, これと推論法則 \ref{dedwedge}, \ref{dedeqfund}によって明らかである.
\halmos




\mathstrut
\begin{thm}
\label{sthmrestanotherdef}%定理
\mbox{}

1)
$a$と$f$を集合とするとき, 
\[
  f|_{a} = f \cap (a \times f[a])
\]
が成り立つ.

2)
$a$と$f$は1)と同じとし, 更に$b$を集合とする.
このとき
\[
  f \cap (a \times b) \subset f|_{a}
\]
が成り立つ.

3)
$a$, $b$, $f$は2)と同じとするとき, 
\[
  f[a] \subset b \leftrightarrow f|_{a} = f \cap (a \times b)
\]
が成り立つ.
またこのことから, 次の($*$)が成り立つ: 

($*$) ~~$f[a] \subset b$が成り立つならば, 
        $f|_{a} = f \cap (a \times b)$が成り立つ.
        逆に$f|_{a} = f \cap (a \times b)$が成り立つならば, 
        $f[a] \subset b$が成り立つ.

4)
$a$, $b$, $f$は2)及び3)と同じとするとき, 
\[
  {\rm pr}_{2}\langle f \rangle \subset b \to f|_{a} = f \cap (a \times b)
\]
が成り立つ.
またこのことから, 次の($**$)が成り立つ: 

($**$) ~~${\rm pr}_{2}\langle f \rangle \subset b$が成り立つならば, 
         $f|_{a} = f \cap (a \times b)$が成り立つ.
\end{thm}


\noindent{\bf 証明}
~1)
$x$と$y$を, 互いに異なり, 共に$a$及び$f$の中に自由変数として現れない, 
定数でない文字とする.
このとき変数法則 \ref{valcap}, \ref{valproduct}, \ref{valvalueset}, \ref{valrest}によってわかるように, 
$x$と$y$は共に$f|_{a}$及び$f \cap (a \times f[a])$の中に自由変数として現れない.
また定理 \ref{sthmpairinrest}より
\[
\tag{1}
  (x, y) \in f|_{a} \leftrightarrow x \in a \wedge (x, y) \in f
\]
が成り立つ.
また定理 \ref{sthmvaluesetbasis}より
\[
  x \in a \wedge (x, y) \in f \to y \in f[a]
\]
が成り立つから, 推論法則 \ref{dedawblatrue1}により
\[
  (x \in a \wedge (x, y) \in f) \wedge y \in f[a] \leftrightarrow x \in a \wedge (x, y) \in f
\]
が成り立ち, これから推論法則 \ref{dedeqch}により
\[
\tag{2}
  x \in a \wedge (x, y) \in f \leftrightarrow (x \in a \wedge (x, y) \in f) \wedge y \in f[a]
\]
が成り立つ.
またThm \ref{awblbwa}より
\[
  x \in a \wedge (x, y) \in f \leftrightarrow (x, y) \in f \wedge x \in a
\]
が成り立つから, 推論法則 \ref{dedaddeqw}により
\[
\tag{3}
  (x \in a \wedge (x, y) \in f) \wedge y \in f[a] \leftrightarrow ((x, y) \in f \wedge x \in a) \wedge y \in f[a]
\]
が成り立つ.
またThm \ref{1awb1wclaw1bwc1}より
\[
\tag{4}
  ((x, y) \in f \wedge x \in a) \wedge y \in f[a] \leftrightarrow (x, y) \in f \wedge (x \in a \wedge y \in f[a])
\]
が成り立つ.
また定理 \ref{sthmpairinproduct}と推論法則 \ref{dedeqch}により
\[
  x \in a \wedge y \in f[a] \leftrightarrow (x, y) \in a \times f[a]
\]
が成り立つから, 推論法則 \ref{dedaddeqw}により
\[
\tag{5}
  (x, y) \in f \wedge (x \in a \wedge y \in f[a]) \leftrightarrow (x, y) \in f \wedge (x, y) \in a \times f[a]
\]
が成り立つ.
また定理 \ref{sthmcapelement}と推論法則 \ref{dedeqch}により
\[
\tag{6}
  (x, y) \in f \wedge (x, y) \in a \times f[a] \leftrightarrow (x, y) \in f \cap (a \times f[a])
\]
が成り立つ.
そこで(1)---(6)から, 推論法則 \ref{dedeqtrans}によって
\[
\tag{7}
  (x, y) \in f|_{a} \leftrightarrow (x, y) \in f \cap (a \times f[a])
\]
が成り立つことがわかる.
さていま定理 \ref{sthmrestbasis}より$f|_{a}$はグラフである.
また定理 \ref{sthmproductgraph}より$a \times f[a]$はグラフだから, 
定理 \ref{sthmcapgraph}により$f \cap (a \times f[a])$もグラフである.
また上述のように, $x$と$y$は互いに異なり, 共に$f|_{a}$及び$f \cap (a \times f[a])$の中に自由変数として現れず, 
共に定数でない.
これらのことと, (7)が成り立つことから, 定理 \ref{sthmgraphpair=}により
\[
\tag{8}
  f|_{a} = f \cap (a \times f[a])
\]
が成り立つ.

\noindent
2)
$z$と$w$を, 互いに異なり, 共に$a$, $b$, $f$のいずれの記号列の中にも自由変数として現れない, 
定数でない文字とする.
このとき変数法則 \ref{valcap}, \ref{valproduct}, \ref{valrest}によってわかるように, 
$z$と$w$は共に$f \cap (a \times b)$及び$f|_{a}$の中に自由変数として現れない.
また定理 \ref{sthmcapelement}と推論法則 \ref{dedequiv}により
\[
\tag{9}
  (z, w) \in f \cap (a \times b) \to (z, w) \in f \wedge (z, w) \in a \times b
\]
が成り立つ.
また定理 \ref{sthmpairinproduct}と推論法則 \ref{dedequiv}により
\[
  (z, w) \in a \times b \to z \in a \wedge w \in b
\]
が成り立つから, 推論法則 \ref{dedprewedge}により
\[
  (z, w) \in a \times b \to z \in a
\]
が成り立つ.
故に推論法則 \ref{dedaddw}により
\[
\tag{10}
  (z, w) \in f \wedge (z, w) \in a \times b \to (z, w) \in f \wedge z \in a
\]
が成り立つ.
またThm \ref{awbtbwa}より
\[
\tag{11}
  (z, w) \in f \wedge z \in a \to z \in a \wedge (z, w) \in f
\]
が成り立つ.
また定理 \ref{sthmpairinrest}と推論法則 \ref{dedequiv}により
\[
\tag{12}
  z \in a \wedge (z, w) \in f \to (z, w) \in f|_{a}
\]
が成り立つ.
そこで(9)---(12)から, 推論法則 \ref{dedmmp}によって
\[
\tag{13}
  (z, w) \in f \cap (a \times b) \to (z, w) \in f|_{a}
\]
が成り立つことがわかる.
さていま定理 \ref{sthmproductgraph}より$a \times b$はグラフだから, 
定理 \ref{sthmcapgraph}により, $f \cap (a \times b)$はグラフである.
また上述のように, $z$と$w$は互いに異なり, 共に$f \cap (a \times b)$及び$f|_{a}$の中に自由変数として現れず, 
共に定数でない.
これらのことと, (13)が成り立つことから, 定理 \ref{sthmgraphpairsubset}により
\[
\tag{14}
  f \cap (a \times b) \subset f|_{a}
\]
が成り立つ.

\noindent
3)
定理 \ref{sthmproductsubset}より
\[
\tag{15}
  f[a] \subset b \to a \times f[a] \subset a \times b
\]
が成り立つ.
また定理 \ref{sthmcapsubset}より
\[
\tag{16}
  a \times f[a] \subset a \times b \to f \cap (a \times f[a]) \subset f \cap (a \times b)
\]
が成り立つ.
また示したように(8)が成り立つから, 定理 \ref{sthm=tsubseteq}により
\[
  f|_{a} \subset f \cap (a \times b) \leftrightarrow f \cap (a \times f[a]) \subset f \cap (a \times b)
\]
が成り立つ.
故に推論法則 \ref{dedequiv}により
\[
\tag{17}
  f \cap (a \times f[a]) \subset f \cap (a \times b) \to f|_{a} \subset f \cap (a \times b)
\]
が成り立つ.
また示したように(14)が成り立つから, 推論法則 \ref{dedatawbtrue2}により
\[
\tag{18}
  f|_{a} \subset f \cap (a \times b) \to f|_{a} \subset f \cap (a \times b) \wedge f \cap (a \times b) \subset f|_{a}
\]
が成り立つ.
また定理 \ref{sthmaxiom1}と推論法則 \ref{dedequiv}により
\[
\tag{19}
  f|_{a} \subset f \cap (a \times b) \wedge f \cap (a \times b) \subset f|_{a} \to f|_{a} = f \cap (a \times b)
\]
が成り立つ.
そこで(15)---(19)から, 推論法則 \ref{dedmmp}によって
\[
\tag{20}
  f[a] \subset b \to f|_{a} = f \cap (a \times b)
\]
が成り立つことがわかる.
さていま$z$と$w$を2)の証明と同じ文字とする.
このとき, $z$は$w$と異なり, $a$及び$f$の中に自由変数として現れないから, 
定理 \ref{sthmvaluesetelement}と推論法則 \ref{dedequiv}により
\[
  w \in f[a] \to \exists z(z \in a \wedge (z, w) \in f)
\]
が成り立つ.
ここで$\tau_{z}(z \in a \wedge (z, w) \in f)$を$T$と書けば, $T$は集合であり, 
定義から上記の記号列は
\[
  w \in f[a] \to (T|z)(z \in a \wedge (z, w) \in f)
\]
と同じである.
また$z$が$w$と異なり, $a$及び$f$の中に自由変数として現れないことから, 
代入法則 \ref{substfree}, \ref{substfund}, \ref{substwedge}, \ref{substpair}により, 
この記号列は
\[
\tag{21}
  w \in f[a] \to T \in a \wedge (T, w) \in f
\]
と一致する.
よってこれが定理となる.
また定理 \ref{sthmpairinrest}と推論法則 \ref{dedequiv}により
\[
\tag{22}
  T \in a \wedge (T, w) \in f \to (T, w) \in f|_{a}
\]
が成り立つ.
そこで(21), (22)から, 推論法則 \ref{dedmmp}によって
\[
  w \in f[a] \to (T, w) \in f|_{a}
\]
が成り立つ.
故に推論法則 \ref{dedaddw}により
\[
\tag{23}
  f|_{a} = f \cap (a \times b) \wedge w \in f[a] \to f|_{a} = f \cap (a \times b) \wedge (T, w) \in f|_{a}
\]
が成り立つ.
また定理 \ref{sthm=&in}より
\[
\tag{24}
  f|_{a} = f \cap (a \times b) \wedge (T, w) \in f|_{a} \to (T, w) \in f \cap (a \times b)
\]
が成り立つ.
また定理 \ref{sthmcapelement}と推論法則 \ref{dedequiv}により
\[
  (T, w) \in f \cap (a \times b) \to (T, w) \in f \wedge (T, w) \in a \times b
\]
が成り立つから, 推論法則 \ref{dedprewedge}により
\[
\tag{25}
  (T, w) \in f \cap (a \times b) \to (T, w) \in a \times b
\]
が成り立つ.
また定理 \ref{sthmpairinproduct}と推論法則 \ref{dedequiv}により
\[
  (T, w) \in a \times b \to T \in a \wedge w \in b
\]
が成り立つから, 推論法則 \ref{dedprewedge}により
\[
\tag{26}
  (T, w) \in a \times b \to w \in b
\]
が成り立つ.
そこで(23)---(26)から, 推論法則 \ref{dedmmp}によって
\[
  f|_{a} = f \cap (a \times b) \wedge w \in f[a] \to w \in b
\]
が成り立つことがわかる.
故に推論法則 \ref{dedtwch}により
\[
\tag{27}
  f|_{a} = f \cap (a \times b) \to (w \in f[a] \to w \in b)
\]
が成り立つ.
さて上述のように, $w$は$f|_{a}$及び$f \cap (a \times b)$の中に自由変数として現れないから, 
変数法則 \ref{valfund}により, $w$は$f|_{a} = f \cap (a \times b)$の中に自由変数として現れない.
また$w$は定数でない.
これらのことと, (27)が成り立つことから, 推論法則 \ref{dedalltquansepfreeconst}により
\[
  f|_{a} = f \cap (a \times b) \to \forall w(w \in f[a] \to w \in b)
\]
が成り立つ.
また$w$は$a$及び$f$の中に自由変数として現れないから, 
変数法則 \ref{valvalueset}により, $w$は$f[a]$の中に自由変数として現れない.
更に, $w$は$b$の中にも自由変数として現れない.
故に定義から, 上記の記号列は
\[
\tag{28}
  f|_{a} = f \cap (a \times b) \to f[a] \subset b
\]
と同じである.
よってこれが定理となる.
そこで(20), (28)から, 推論法則 \ref{dedequiv}により
\[
  f[a] \subset b \leftrightarrow f|_{a} = f \cap (a \times b)
\]
が成り立つ.
($*$)が成り立つことは, これと推論法則 \ref{dedeqfund}によって明らかである.

\noindent
4)
定理 \ref{sthmvaluesetsubsetpr2set}より
$f[a] \subset {\rm pr}_{2}\langle f \rangle$が成り立つから, 
推論法則 \ref{dedatawbtrue2}により
\[
\tag{29}
  {\rm pr}_{2}\langle f \rangle \subset b 
  \to f[a] \subset {\rm pr}_{2}\langle f \rangle \wedge {\rm pr}_{2}\langle f \rangle \subset b
\]
が成り立つ.
また定理 \ref{sthmsubsettrans}より
\[
\tag{30}
  f[a] \subset {\rm pr}_{2}\langle f \rangle \wedge {\rm pr}_{2}\langle f \rangle \subset b 
  \to f[a] \subset b
\]
が成り立つ.
また示したように(20)が成り立つ.
そこで(29), (30), (20)から, 推論法則 \ref{dedmmp}によって
\[
  {\rm pr}_{2}\langle f \rangle \subset b \to f|_{a} = f \cap (a \times b)
\]
が成り立つことがわかる.
($**$)が成り立つことは, これと推論法則 \ref{dedmp}によって明らかである.
\halmos




\mathstrut
\begin{thm}
\label{sthmrestcappr1set}%定理
$a$と$f$を集合とするとき, 
\[
  f|_{a} = f|_{{\rm pr}_{1}\langle f \rangle \cap a}
\]
が成り立つ.
\end{thm}


\noindent{\bf 証明}
~$x$と$y$を, 互いに異なり, 共に$a$及び$f$の中に自由変数として現れない, 定数でない文字とする.
このとき変数法則 \ref{valcap}, \ref{valprset}, \ref{valrest}によってわかるように, 
$x$と$y$は共に$f|_{a}$及び$f|_{{\rm pr}_{1}\langle f \rangle \cap a}$の中に自由変数として現れない.
また定理 \ref{sthmpairinrest}より
\[
\tag{1}
  (x, y) \in f|_{a} \leftrightarrow x \in a \wedge (x, y) \in f
\]
が成り立つ.
また定理 \ref{sthmpairelementinprset}より
\[
  (x, y) \in f \to x \in {\rm pr}_{1}\langle f \rangle \wedge y \in {\rm pr}_{2}\langle f \rangle
\]
が成り立つから, 推論法則 \ref{dedprewedge}により
\[
  (x, y) \in f \to x \in {\rm pr}_{1}\langle f \rangle
\]
が成り立つ.
故に推論法則 \ref{dedawblatrue1}により
\[
  (x, y) \in f \wedge x \in {\rm pr}_{1}\langle f \rangle \leftrightarrow (x, y) \in f
\]
が成り立ち, これから推論法則 \ref{dedeqch}により
\[
  (x, y) \in f \leftrightarrow (x, y) \in f \wedge x \in {\rm pr}_{1}\langle f \rangle
\]
が成り立つ.
故に推論法則 \ref{dedaddeqw}により
\[
\tag{2}
  x \in a \wedge (x, y) \in f \leftrightarrow x \in a \wedge ((x, y) \in f \wedge x \in {\rm pr}_{1}\langle f \rangle)
\]
が成り立つ.
またThm \ref{awblbwa}より
\[
\tag{3}
  x \in a \wedge ((x, y) \in f \wedge x \in {\rm pr}_{1}\langle f \rangle) 
  \leftrightarrow ((x, y) \in f \wedge x \in {\rm pr}_{1}\langle f \rangle) \wedge x \in a
\]
が成り立つ.
またThm \ref{1awb1wclaw1bwc1}より
\[
\tag{4}
  ((x, y) \in f \wedge x \in {\rm pr}_{1}\langle f \rangle) \wedge x \in a 
  \leftrightarrow (x, y) \in f \wedge (x \in {\rm pr}_{1}\langle f \rangle \wedge x \in a)
\]
が成り立つ.
また定理 \ref{sthmcapelement}と推論法則 \ref{dedeqch}により
\[
  x \in {\rm pr}_{1}\langle f \rangle \wedge x \in a \leftrightarrow x \in {\rm pr}_{1}\langle f \rangle \cap a
\]
が成り立つから, 推論法則 \ref{dedaddeqw}により
\[
\tag{5}
  (x, y) \in f \wedge (x \in {\rm pr}_{1}\langle f \rangle \wedge x \in a) 
  \leftrightarrow (x, y) \in f \wedge x \in {\rm pr}_{1}\langle f \rangle \cap a
\]
が成り立つ.
またThm \ref{awblbwa}より
\[
\tag{6}
  (x, y) \in f \wedge x \in {\rm pr}_{1}\langle f \rangle \cap a 
  \leftrightarrow x \in {\rm pr}_{1}\langle f \rangle \cap a \wedge (x, y) \in f
\]
が成り立つ.
また定理 \ref{sthmpairinrest}と推論法則 \ref{dedeqch}により
\[
\tag{7}
  x \in {\rm pr}_{1}\langle f \rangle \cap a \wedge (x, y) \in f 
  \leftrightarrow (x, y) \in f|_{{\rm pr}_{1}\langle f \rangle \cap a}
\]
が成り立つ.
そこで(1)---(7)から, 推論法則 \ref{dedeqtrans}によって
\[
\tag{8}
  (x, y) \in f|_{a} \leftrightarrow (x, y) \in f|_{{\rm pr}_{1}\langle f \rangle \cap a}
\]
が成り立つことがわかる.
さていま定理 \ref{sthmrestbasis}より, $f|_{a}$と$f|_{{\rm pr}_{1}\langle f \rangle \cap a}$は共にグラフである.
また上述のように, $x$と$y$は互いに異なり, 共に定数でなく, 共にこの二つの記号列の中に自由変数として現れない.
これらのことと, (8)が成り立つことから, 定理 \ref{sthmgraphpair=}により
\[
  f|_{a} = f|_{{\rm pr}_{1}\langle f \rangle \cap a}
\]
が成り立つ.
\halmos




\mathstrut
\begin{thm}
\label{sthmrestsubset}%定理
\mbox{}

1)
$a$, $f$, $g$を集合とするとき, 
\[
  f \subset g \to f|_{a} \subset g|_{a}
\]
が成り立つ.
またこのことから, 次の($*$)が成り立つ: 

($*$) ~~$f \subset g$が成り立つならば, $f|_{a} \subset g|_{a}$が成り立つ.

2)
$a$, $b$, $f$を集合とするとき, 
\[
  a \subset b \to f|_{a} \subset f|_{b}
\]
が成り立つ.
またこのことから, 次の($**$)が成り立つ: 

($**$) ~~$a \subset b$が成り立つならば, $f|_{a} \subset f|_{b}$が成り立つ.
\end{thm}


\noindent{\bf 証明}
~1)
定理 \ref{sthmcapsubset}より
\[
  f \subset g \to f \cap (a \times {\rm pr}_{2}\langle f \rangle) \subset g \cap (a \times {\rm pr}_{2}\langle f \rangle), 
\]
即ち
\[
\tag{1}
  f \subset g \to f|_{a} \subset g \cap (a \times {\rm pr}_{2}\langle f \rangle)
\]
が成り立つ.
また定理 \ref{sthmrestanotherdef}より
\[
  g \cap (a \times {\rm pr}_{2}\langle f \rangle) \subset g|_{a}
\]
が成り立つから, 推論法則 \ref{dedatawbtrue2}により
\[
\tag{2}
  f|_{a} \subset g \cap (a \times {\rm pr}_{2}\langle f \rangle) 
  \to f|_{a} \subset g \cap (a \times {\rm pr}_{2}\langle f \rangle) 
  \wedge g \cap (a \times {\rm pr}_{2}\langle f \rangle) \subset g|_{a}
\]
が成り立つ.
また定理 \ref{sthmsubsettrans}より
\[
\tag{3}
  f|_{a} \subset g \cap (a \times {\rm pr}_{2}\langle f \rangle) 
  \wedge g \cap (a \times {\rm pr}_{2}\langle f \rangle) \subset g|_{a} 
  \to f|_{a} \subset g|_{a}
\]
が成り立つ.
そこで(1), (2), (3)から, 推論法則 \ref{dedmmp}によって
\[
  f \subset g \to f|_{a} \subset g|_{a}
\]
が成り立つことがわかる.
($*$)が成り立つことは, これと推論法則 \ref{dedmp}によって明らかである.

\noindent
2)
定理 \ref{sthmproductsubset}より
\[
\tag{4}
  a \subset b \to a \times {\rm pr}_{2}\langle f \rangle \subset b \times {\rm pr}_{2}\langle f \rangle
\]
が成り立つ.
また定理 \ref{sthmcapsubset}より
\[
  a \times {\rm pr}_{2}\langle f \rangle \subset b \times {\rm pr}_{2}\langle f \rangle 
  \to f \cap (a \times {\rm pr}_{2}\langle f \rangle) \subset f \cap (b \times {\rm pr}_{2}\langle f \rangle), 
\]
即ち
\[
\tag{5}
  a \times {\rm pr}_{2}\langle f \rangle \subset b \times {\rm pr}_{2}\langle f \rangle 
  \to f|_{a} \subset f|_{b}
\]
が成り立つ.
そこで(4), (5)から, 推論法則 \ref{dedmmp}によって
\[
  a \subset b \to f|_{a} \subset f|_{b}
\]
が成り立つ.
($**$)が成り立つことは, これと推論法則 \ref{dedmp}によって明らかである.
\halmos




\mathstrut
\begin{thm}
\label{sthmrest=}%定理
\mbox{}

1)
$a$, $f$, $g$を集合とするとき, 
\[
  f = g \to f|_{a} = g|_{a}
\]
が成り立つ.
またこのことから, 次の($*$)が成り立つ: 

($*$) ~~$f = g$が成り立つならば, $f|_{a} = g|_{a}$が成り立つ.

2)
$a$, $b$, $f$を集合とするとき, 
\[
  a = b \to f|_{a} = f|_{b}
\]
が成り立つ.
またこのことから, 次の($**$)が成り立つ: 

($**$) ~~$a = b$が成り立つならば, $f|_{a} = f|_{b}$が成り立つ.
\end{thm}


\noindent{\bf 証明}
~1)
$x$を$a$の中に自由変数として現れない文字とする.
このときThm \ref{T=Ut1TV=UV1}より
\[
  f = g \to (f|x)(x|_{a}) = (g|x)(x|_{a})
\]
が成り立つが, 代入法則 \ref{substfree}, \ref{substrest}によれば, この記号列は
\[
  f = g \to f|_{a} = g|_{a}
\]
と一致するから, これが定理となる.
($*$)が成り立つことは, これと推論法則 \ref{dedmp}によって明らかである.

\noindent
2)
$y$を$f$の中に自由変数として現れない文字とする.
このときThm \ref{T=Ut1TV=UV1}より
\[
  a = b \to (a|y)(f|_{y}) = (b|y)(f|_{y})
\]
が成り立つが, 代入法則 \ref{substfree}, \ref{substrest}によれば, この記号列は
\[
  a = b \to f|_{a} = f|_{b}
\]
と一致するから, これが定理となる.
($**$)が成り立つことは, これと推論法則 \ref{dedmp}によって明らかである.
\halmos




\mathstrut
\begin{thm}
\label{sthmrestrest}%定理
\mbox{}

1)
$a$, $b$, $f$を集合とするとき, 
\[
  (f|_{a})|_{b} = f|_{a \cap b}, ~~
  (f|_{a})|_{b} = (f|_{b})|_{a}
\]
が成り立つ.

2)
$a$, $b$, $f$は1)と同じとするとき, 
\[
  a \subset b \to (f|_{a})|_{b} = f|_{a}, ~~
  b \subset a \to (f|_{a})|_{b} = f|_{b}
\]
が成り立つ.
またこれらから, 次の($*$)が成り立つ: 

($*$) ~~$a \subset b$が成り立つならば, $(f|_{a})|_{b} = f|_{a}$が成り立つ.
        また$b \subset a$が成り立つならば, $(f|_{a})|_{b} = f|_{b}$が成り立つ.
\end{thm}


\noindent{\bf 証明}
~1)
まず前者の記号列が定理となることを示す.
$x$と$y$を, 互いに異なり, 共に$a$, $b$, $f$のいずれの記号列の中にも自由変数として現れない, 
定数でない文字とする.
このとき変数法則 \ref{valcap}, \ref{valrest}により, $x$と$y$は共に
$(f|_{a})|_{b}$及び$f|_{a \cap b}$の中に自由変数として現れない.
また定理 \ref{sthmpairinrest}より
\[
\tag{1}
  (x, y) \in (f|_{a})|_{b} \leftrightarrow x \in b \wedge (x, y) \in f|_{a}
\]
が成り立つ.
また同じく定理 \ref{sthmpairinrest}より
\[
  (x, y) \in f|_{a} \leftrightarrow x \in a \wedge (x, y) \in f
\]
が成り立つから, 推論法則 \ref{dedaddeqw}により
\[
\tag{2}
  x \in b \wedge (x, y) \in f|_{a} \leftrightarrow x \in b \wedge (x \in a \wedge (x, y) \in f)
\]
が成り立つ.
またThm \ref{1awb1wclaw1bwc1}と推論法則 \ref{dedeqch}により
\[
\tag{3}
  x \in b \wedge (x \in a \wedge (x, y) \in f) \leftrightarrow (x \in b \wedge x \in a) \wedge (x, y) \in f
\]
が成り立つ.
またThm \ref{awblbwa}より
\[
\tag{4}
  x \in b \wedge x \in a \leftrightarrow x \in a \wedge x \in b
\]
が成り立つ.
また定理 \ref{sthmcapelement}と推論法則 \ref{dedeqch}により
\[
\tag{5}
  x \in a \wedge x \in b \leftrightarrow x \in a \cap b
\]
が成り立つ.
そこで(4), (5)から, 推論法則 \ref{dedeqtrans}によって
\[
  x \in b \wedge x \in a \leftrightarrow x \in a \cap b
\]
が成り立つ.
故に推論法則 \ref{dedaddeqw}により
\[
\tag{6}
  (x \in b \wedge x \in a) \wedge (x, y) \in f \leftrightarrow x \in a \cap b \wedge (x, y) \in f
\]
が成り立つ.
また定理 \ref{sthmpairinrest}と推論法則 \ref{dedeqch}により
\[
\tag{7}
  x \in a \cap b \wedge (x, y) \in f \leftrightarrow (x, y) \in f|_{a \cap b}
\]
が成り立つ.
そこで(1), (2), (3), (6), (7)から, 推論法則 \ref{dedeqtrans}によって
\[
\tag{8}
  (x, y) \in (f|_{a})|_{b} \leftrightarrow (x, y) \in f|_{a \cap b}
\]
が成り立つことがわかる.
さていま定理 \ref{sthmrestbasis}より, $(f|_{a})|_{b}$と$f|_{a \cap b}$は共にグラフである.
また上述のように$x$と$y$は互いに異なり, 共に定数でなく, 共に$(f|_{a})|_{b}$及び$f|_{a \cap b}$の中に
自由変数として現れない.
これらのことと, (8)が成り立つことから, 定理 \ref{sthmgraphpair=}により
\[
\tag{9}
  (f|_{a})|_{b} = f|_{a \cap b}
\]
が成り立つ.

次に後者の記号列が定理となることを示す.
定理 \ref{sthmcapch}より$a \cap b = b \cap a$が成り立つから, 
定理 \ref{sthmrest=}により
\[
\tag{10}
  f|_{a \cap b} = f|_{b \cap a}
\]
が成り立つ.
またいま示したように(9)が成り立つから, (9)で$a$と$b$を入れ替えた
\[
  (f|_{b})|_{a} = f|_{b \cap a}
\]
も成り立つ.
故に推論法則 \ref{ded=ch}により
\[
\tag{11}
  f|_{b \cap a} = (f|_{b})|_{a}
\]
が成り立つ.
そこで(9), (10), (11)から, 推論法則 \ref{ded=trans}によって
\[
\tag{12}
  (f|_{a})|_{b} = (f|_{b})|_{a}
\]
が成り立つことがわかる.

\noindent
2)
まず前者の記号列が定理となることを示す.
定理 \ref{sthmcapsubset=}と推論法則 \ref{dedequiv}により
\[
\tag{13}
  a \subset b \to a \cap b = a
\]
が成り立つ.
また定理 \ref{sthmrest=}より
\[
\tag{14}
  a \cap b = a \to f|_{a \cap b} = f|_{a}
\]
が成り立つ.
また示したように(9)が成り立つから, 推論法則 \ref{dedatawbtrue2}により
\[
\tag{15}
  f|_{a \cap b} = f|_{a} \to (f|_{a})|_{b} = f|_{a \cap b} \wedge f|_{a \cap b} = f|_{a}
\]
が成り立つ.
またThm \ref{x=ywy=ztx=z}より
\[
\tag{16}
  (f|_{a})|_{b} = f|_{a \cap b} \wedge f|_{a \cap b} = f|_{a} \to (f|_{a})|_{b} = f|_{a}
\]
が成り立つ.
そこで(13)---(16)から, 推論法則 \ref{dedmmp}によって
\[
\tag{17}
  a \subset b \to (f|_{a})|_{b} = f|_{a}
\]
が成り立つことがわかる.

次に後者の記号列が定理となることを示す.
いま示したように(17)が成り立つから, (17)で$a$と$b$を入れ替えた
\[
\tag{18}
  b \subset a \to (f|_{b})|_{a} = f|_{b}
\]
も成り立つ.
また示したように(12)が成り立つから, 推論法則 \ref{dedatawbtrue2}により
\[
\tag{19}
  (f|_{b})|_{a} = f|_{b} \to (f|_{a})|_{b} = (f|_{b})|_{a} \wedge (f|_{b})|_{a} = f|_{b}
\]
が成り立つ.
またThm \ref{x=ywy=ztx=z}より
\[
\tag{20}
  (f|_{a})|_{b} = (f|_{b})|_{a} \wedge (f|_{b})|_{a} = f|_{b} 
  \to (f|_{a})|_{b} = f|_{b}
\]
が成り立つ.
そこで(18), (19), (20)から, 推論法則 \ref{dedmmp}によって
\[
\tag{21}
  b \subset a \to (f|_{a})|_{b} = f|_{b}
\]
が成り立つことがわかる.

さていま示したように(17), (21)が共に成り立つから, 
($*$)が成り立つことは, これらと推論法則 \ref{dedmp}によって明らかである.
\halmos




\mathstrut
\begin{thm}
\label{sthmcuprest}%定理
\mbox{}

1)
$a$, $f$, $g$を集合とするとき, 
\[
  (f \cup g)|_{a} = f|_{a} \cup g|_{a}
\]
が成り立つ.

2)
$a$, $b$, $f$を集合とするとき, 
\[
  f|_{a \cup b} = f|_{a} \cup f|_{b}
\]
が成り立つ.
\end{thm}


\noindent{\bf 証明}
~1)
定義より$(f \cup g)|_{a}$は
$(f \cup g) \cap (a \times {\rm pr}_{2}\langle f \cup g \rangle)$だから, 
定理 \ref{sthmsetdist}より
\[
\tag{1}
  (f \cup g)|_{a} = (f \cap (a \times {\rm pr}_{2}\langle f \cup g \rangle)) \cup (g \cap (a \times {\rm pr}_{2}\langle f \cup g \rangle))
\]
が成り立つ.
また定理 \ref{sthmsubsetcup}より$f \subset f \cup g$と$g \subset f \cup g$が共に成り立つから, 
定理 \ref{sthmprsetsubset}により
\[
  {\rm pr}_{2}\langle f \rangle \subset {\rm pr}_{2}\langle f \cup g \rangle, ~~
  {\rm pr}_{2}\langle g \rangle \subset {\rm pr}_{2}\langle f \cup g \rangle
\]
が共に成り立つ.
故に定理 \ref{sthmrestanotherdef}により
\[
  f|_{a} = f \cap (a \times {\rm pr}_{2}\langle f \cup g \rangle), ~~
  g|_{a} = g \cap (a \times {\rm pr}_{2}\langle f \cup g \rangle)
\]
が共に成り立つ.
故に推論法則 \ref{ded=ch}により
\[
  f \cap (a \times {\rm pr}_{2}\langle f \cup g \rangle) = f|_{a}, ~~
  g \cap (a \times {\rm pr}_{2}\langle f \cup g \rangle) = g|_{a}
\]
が共に成り立つ.
そこで定理 \ref{sthmcup=}により
\[
\tag{2}
  (f \cap (a \times {\rm pr}_{2}\langle f \cup g \rangle)) \cup (g \cap (a \times {\rm pr}_{2}\langle f \cup g \rangle)) 
  = f|_{a} \cup g|_{a}
\]
が成り立つ.
そこで(1), (2)から, 推論法則 \ref{ded=trans}によって
\[
  (f \cup g)|_{a} = f|_{a} \cup g|_{a}
\]
が成り立つ.

\noindent
2)
定理 \ref{sthmcupproduct}より
\[
  (a \cup b) \times {\rm pr}_{2}\langle f \rangle 
  = (a \times {\rm pr}_{2}\langle f \rangle) \cup (b \times {\rm pr}_{2}\langle f \rangle)
\]
が成り立つから, 定理 \ref{sthmcap=}により
\[
  f \cap ((a \cup b) \times {\rm pr}_{2}\langle f \rangle) 
  = f \cap ((a \times {\rm pr}_{2}\langle f \rangle) \cup (b \times {\rm pr}_{2}\langle f \rangle)), 
\]
即ち
\[
\tag{3}
  f|_{a \cup b} = f \cap ((a \times {\rm pr}_{2}\langle f \rangle) \cup (b \times {\rm pr}_{2}\langle f \rangle))
\]
が成り立つ.
また定理 \ref{sthmsetdist}より
\[
  f \cap ((a \times {\rm pr}_{2}\langle f \rangle) \cup (b \times {\rm pr}_{2}\langle f \rangle)) 
  = (f \cap (a \times {\rm pr}_{2}\langle f \rangle)) \cup (f \cap (b \times {\rm pr}_{2}\langle f \rangle)), 
\]
即ち
\[
\tag{4}
  f \cap ((a \times {\rm pr}_{2}\langle f \rangle) \cup (b \times {\rm pr}_{2}\langle f \rangle)) 
  = f|_{a} \cup f|_{b}
\]
が成り立つ.
そこで(3), (4)から, 推論法則 \ref{ded=trans}によって
\[
  f|_{a \cup b} = f|_{a} \cup f|_{b}
\]
が成り立つ.
\halmos




\mathstrut
\begin{thm}
\label{sthmcaprest}%定理
\mbox{}

1)
$a$, $f$, $g$を集合とするとき, 
\[
  (f \cap g)|_{a} = f|_{a} \cap g|_{a}, ~~
  (f \cap g)|_{a} = f|_{a} \cap g, ~~
  (f \cap g)|_{a} = f \cap g|_{a}
\]
が成り立つ.

2)
$a$, $b$, $f$を集合とするとき, 
\[
  f|_{a \cap b} = f|_{a} \cap f|_{b}
\]
が成り立つ.
\end{thm}


\noindent{\bf 証明}
~1)
$x$と$y$を, 互いに異なり, 共に$a$, $f$, $g$のいずれの記号列の中にも自由変数として現れない, 
定数でない文字とする.
このとき変数法則 \ref{valcap}, \ref{valrest}によってわかるように, $x$と$y$は共に
$(f \cap g)|_{a}$, $f|_{a} \cap g|_{a}$, $f|_{a} \cap g$, $f \cap g|_{a}$の
いずれの記号列の中にも自由変数として現れない.
また定理 \ref{sthmpairinrest}より
\[
\tag{1}
  (x, y) \in (f \cap g)|_{a} \leftrightarrow x \in a \wedge (x, y) \in f \cap g
\]
が成り立つ.
また定理 \ref{sthmcapelement}より
\[
  (x, y) \in f \cap g \leftrightarrow (x, y) \in f \wedge (x, y) \in g
\]
が成り立つから, 推論法則 \ref{dedaddeqw}により
\[
\tag{2}
  x \in a \wedge (x, y) \in f \cap g \leftrightarrow x \in a \wedge ((x, y) \in f \wedge (x, y) \in g)
\]
が成り立つ.
またThm \ref{aw1bwc1l1awb1w1awc1}より
\[
\tag{3}
  x \in a \wedge ((x, y) \in f \wedge (x, y) \in g) \leftrightarrow (x \in a \wedge (x, y) \in f) \wedge (x \in a \wedge (x, y) \in g)
\]
が成り立つ.
またThm \ref{1awb1wclaw1bwc1}と推論法則 \ref{dedeqch}により
\[
\tag{4}
  x \in a \wedge ((x, y) \in f \wedge (x, y) \in g) \leftrightarrow (x \in a \wedge (x, y) \in f) \wedge (x, y) \in g
\]
が成り立つ.
またThm \ref{awblbwa}より
\[
  x \in a \wedge (x, y) \in f \leftrightarrow (x, y) \in f \wedge x \in a
\]
が成り立つから, 推論法則 \ref{dedaddeqw}により
\[
\tag{5}
  (x \in a \wedge (x, y) \in f) \wedge (x, y) \in g \leftrightarrow ((x, y) \in f \wedge x \in a) \wedge (x, y) \in g
\]
が成り立つ.
またThm \ref{1awb1wclaw1bwc1}より
\[
\tag{6}
  ((x, y) \in f \wedge x \in a) \wedge (x, y) \in g \leftrightarrow (x, y) \in f \wedge (x \in a \wedge (x, y) \in g)
\]
が成り立つ.
また定理 \ref{sthmpairinrest}と推論法則 \ref{dedeqch}により
\begin{align*}
  \tag{7}
  x \in a \wedge (x, y) \in f &\leftrightarrow (x, y) \in f|_{a}, \\
  \mbox{} \\
  \tag{8}
  x \in a \wedge (x, y) \in g &\leftrightarrow (x, y) \in g|_{a}
\end{align*}
が共に成り立つ.
そこで(7), (8)から, 推論法則 \ref{dedaddeqw}により
\[
\tag{9}
  (x \in a \wedge (x, y) \in f) \wedge (x \in a \wedge (x, y) \in g) \leftrightarrow (x, y) \in f|_{a} \wedge (x, y) \in g|_{a}
\]
が成り立つ.
また(7)から, 同じく推論法則 \ref{dedaddeqw}により
\[
\tag{10}
  (x \in a \wedge (x, y) \in f) \wedge (x, y) \in g \leftrightarrow (x, y) \in f|_{a} \wedge (x, y) \in g
\]
が成り立つ.
また(8)から, やはり推論法則 \ref{dedaddeqw}により
\[
\tag{11}
  (x, y) \in f \wedge (x \in a \wedge (x, y) \in g) \leftrightarrow (x, y) \in f \wedge (x, y) \in g|_{a}
\]
が成り立つ.
また定理 \ref{sthmcapelement}と推論法則 \ref{dedeqch}により
\begin{align*}
  \tag{12}
  (x, y) \in f|_{a} \wedge (x, y) \in g|_{a} &\leftrightarrow (x, y) \in f|_{a} \cap g|_{a}, \\
  \mbox{} \\
  \tag{13}
  (x, y) \in f|_{a} \wedge (x, y) \in g &\leftrightarrow (x, y) \in f|_{a} \cap g, \\
  \mbox{} \\
  \tag{14}
  (x, y) \in f \wedge (x, y) \in g|_{a} &\leftrightarrow (x, y) \in f \cap g|_{a}
\end{align*}
がすべて成り立つ.
そこで(1), (2), (3), (9), (12)から, 推論法則 \ref{dedeqtrans}によって
\[
\tag{15}
  (x, y) \in (f \cap g)|_{a} \leftrightarrow (x, y) \in f|_{a} \cap g|_{a}
\]
が成り立つことがわかる.
また(1), (2), (4), (10), (13)から, 同じく推論法則 \ref{dedeqtrans}によって
\[
\tag{16}
  (x, y) \in (f \cap g)|_{a} \leftrightarrow (x, y) \in f|_{a} \cap g
\]
が成り立つことがわかる.
また(1), (2), (4), (5), (6), (11), (14)から, やはり推論法則 \ref{dedeqtrans}によって
\[
\tag{17}
  (x, y) \in (f \cap g)|_{a} \leftrightarrow (x, y) \in f \cap g|_{a}
\]
が成り立つことがわかる.
さていま定理 \ref{sthmrestbasis}より, $(f \cap g)|_{a}$, $f|_{a}$, $g|_{a}$はいずれもグラフである.
故に定理 \ref{sthmcapgraph}により, $f|_{a} \cap g|_{a}$, $f|_{a} \cap g$, $f \cap g|_{a}$も
すべてグラフである.
また上述のように, $x$と$y$は互いに異なり, 共に定数でなく, 共に
$(f \cap g)|_{a}$, $f|_{a} \cap g|_{a}$, $f|_{a} \cap g$, $f \cap g|_{a}$の
いずれの記号列の中にも自由変数として現れない.
これらのことと, (15), (16), (17)がすべて成り立つことから, 
定理 \ref{sthmgraphpair=}により
\[
  (f \cap g)|_{a} = f|_{a} \cap g|_{a}, ~~
  (f \cap g)|_{a} = f|_{a} \cap g, ~~
  (f \cap g)|_{a} = f \cap g|_{a}
\]
がすべて成り立つ.

\noindent
2)
定理 \ref{sthmcapproduct}より
\[
  (a \cap b) \times {\rm pr}_{2}\langle f \rangle 
  = (a \times {\rm pr}_{2}\langle f \rangle) \cap (b \times {\rm pr}_{2}\langle f \rangle)
\]
が成り立つから, 定理 \ref{sthmcap=}により
\[
  f \cap ((a \cap b) \times {\rm pr}_{2}\langle f \rangle) 
  = f \cap ((a \times {\rm pr}_{2}\langle f \rangle) \cap (b \times {\rm pr}_{2}\langle f \rangle)), 
\]
即ち
\[
\tag{18}
  f|_{a \cap b} = f \cap ((a \times {\rm pr}_{2}\langle f \rangle) \cap (b \times {\rm pr}_{2}\langle f \rangle))
\]
が成り立つ.
また定理 \ref{sthmcapdist}より
\[
  f \cap ((a \times {\rm pr}_{2}\langle f \rangle) \cap (b \times {\rm pr}_{2}\langle f \rangle)) 
  = (f \cap (a \times {\rm pr}_{2}\langle f \rangle)) \cap (f \cap (b \times {\rm pr}_{2}\langle f \rangle)), 
\]
即ち
\[
\tag{19}
  f \cap ((a \times {\rm pr}_{2}\langle f \rangle) \cap (b \times {\rm pr}_{2}\langle f \rangle)) 
  = f|_{a} \cap f|_{b}
\]
が成り立つ.
そこで(18), (19)から, 推論法則 \ref{ded=trans}によって
\[
  f|_{a \cap b} = f|_{a} \cap f|_{b}
\]
が成り立つ.
\halmos




\mathstrut
\begin{thm}
\label{sthm-rest}%定理
\mbox{}

1)
$a$, $f$, $g$を集合とするとき, 
\[
  (f - g)|_{a} = f|_{a} - g|_{a}, ~~
  (f - g)|_{a} = f|_{a} - g
\]
が成り立つ.

2)
$a$, $b$, $f$を集合とするとき, 
\[
  f|_{a - b} = f|_{a} - f|_{b}
\]
が成り立つ.
\end{thm}


\noindent{\bf 証明}
~1)
まず第二の記号列が定理となることから示す.
$x$と$y$を, 互いに異なり, 共に$a$, $f$, $g$のいずれの記号列の中にも自由変数として現れない, 
定数でない文字とする.
このとき変数法則 \ref{val-}, \ref{valrest}によってわかるように, $x$と$y$は共に
$(f - g)|_{a}$及び$f|_{a} - g$の中に自由変数として現れない.
また定理 \ref{sthmpairinrest}より
\[
\tag{1}
  (x, y) \in (f - g)|_{a} \leftrightarrow x \in a \wedge (x, y) \in f - g
\]
が成り立つ.
また定理 \ref{sthm-basis}より
\[
  (x, y) \in f - g \leftrightarrow (x, y) \in f \wedge (x, y) \notin g
\]
が成り立つから, 推論法則 \ref{dedaddeqw}により
\[
\tag{2}
  x \in a \wedge (x, y) \in f - g \leftrightarrow x \in a \wedge ((x, y) \in f \wedge (x, y) \notin g)
\]
が成り立つ.
またThm \ref{1awb1wclaw1bwc1}と推論法則 \ref{dedeqch}により
\[
\tag{3}
  x \in a \wedge ((x, y) \in f \wedge (x, y) \notin g) \leftrightarrow (x \in a \wedge (x, y) \in f) \wedge (x, y) \notin g
\]
が成り立つ.
また定理 \ref{sthmpairinrest}と推論法則 \ref{dedeqch}により
\[
  x \in a \wedge (x, y) \in f \leftrightarrow (x, y) \in f|_{a}
\]
が成り立つから, 推論法則 \ref{dedaddeqw}により
\[
\tag{4}
  (x \in a \wedge (x, y) \in f) \wedge (x, y) \notin g \leftrightarrow (x, y) \in f|_{a} \wedge (x, y) \notin g
\]
が成り立つ.
また定理 \ref{sthm-basis}と推論法則 \ref{dedeqch}により
\[
\tag{5}
  (x, y) \in f|_{a} \wedge (x, y) \notin g \leftrightarrow (x, y) \in f|_{a} - g
\]
が成り立つ.
そこで(1)---(5)から, 推論法則 \ref{dedeqtrans}によって
\[
\tag{6}
  (x, y) \in (f - g)|_{a} \leftrightarrow (x, y) \in f|_{a} - g
\]
が成り立つことがわかる.
さていま定理 \ref{sthmrestbasis}より, $(f - g)|_{a}$と$f|_{a}$は共にグラフである.
故にこの後者から, 定理 \ref{sthm-graph}により$f|_{a} - g$もグラフである.
また上述のように, $x$と$y$は互いに異なり, 共に定数でなく, 共に
$(f - g)|_{a}$及び$f|_{a} - g$の中に自由変数として現れない.
これらのことと, (6)が成り立つことから, 定理 \ref{sthmgraphpair=}により
\[
\tag{7}
  (f - g)|_{a} = f|_{a} - g
\]
が成り立つ.

次に第一の記号列が定理となることを示す.
定理 \ref{sthmcaprest}と推論法則 \ref{ded=ch}により
\[
  f|_{a} \cap g = (f \cap g)|_{a}
\]
が成り立ち, 定理 \ref{sthmcaprest}より
\[
  (f \cap g)|_{a} = f|_{a} \cap g|_{a}
\]
が成り立つから, 推論法則 \ref{ded=trans}により
\[
\tag{8}
  f|_{a} \cap g = f|_{a} \cap g|_{a}
\]
が成り立つ.
また定理 \ref{sthm=from-abs}より
\[
\tag{9}
  f|_{a} \cap g = f|_{a} \cap g|_{a} \leftrightarrow f|_{a} - g = f|_{a} - g|_{a}
\]
が成り立つ.
そこで(8), (9)から, 推論法則 \ref{dedeqfund}により
\[
  f|_{a} - g = f|_{a} - g|_{a}
\]
が成り立つ.
そこでこれと(7)から, 推論法則 \ref{ded=trans}により
\[
  (f - g)|_{a} = f|_{a} - g|_{a}
\]
が成り立つ.

\noindent
2)
定理 \ref{sthm-product}より
\[
  (a - b) \times {\rm pr}_{2}\langle f \rangle 
  = (a \times {\rm pr}_{2}\langle f \rangle) - (b \times {\rm pr}_{2}\langle f \rangle)
\]
が成り立つから, 定理 \ref{sthmcap=}により
\[
  f \cap ((a - b) \times {\rm pr}_{2}\langle f \rangle) 
  = f \cap ((a \times {\rm pr}_{2}\langle f \rangle) - (b \times {\rm pr}_{2}\langle f \rangle)), 
\]
即ち
\[
\tag{10}
  f|_{a - b} = f \cap ((a \times {\rm pr}_{2}\langle f \rangle) - (b \times {\rm pr}_{2}\langle f \rangle))
\]
が成り立つ.
また定理 \ref{sthmcap-}より
\[
  f \cap ((a \times {\rm pr}_{2}\langle f \rangle) - (b \times {\rm pr}_{2}\langle f \rangle)) 
  = (f \cap (a \times {\rm pr}_{2}\langle f \rangle)) - (f \cap (b \times {\rm pr}_{2}\langle f \rangle)), 
\]
即ち
\[
\tag{11}
  f \cap ((a \times {\rm pr}_{2}\langle f \rangle) - (b \times {\rm pr}_{2}\langle f \rangle)) 
  = f|_{a} - f|_{b}
\]
が成り立つ.
そこで(10), (11)から, 推論法則 \ref{ded=trans}によって
\[
  f|_{a - b} = f|_{a} - f|_{b}
\]
が成り立つ.
\halmos




\mathstrut
$f|_{a}$が空となるための条件は後で述べる(定理 \ref{sthmemptyrest}).




\mathstrut
\begin{thm}
\label{sthmproductrest}%定理
\mbox{}

1)
$a$, $b$, $c$を集合とするとき, 
\[
  (a \times b)|_{c} = (a \cap c) \times b
\]
が成り立つ.

2)
$a$, $b$, $c$は1)と同じとするとき, 
\[
  a \subset c \to (a \times b)|_{c} = a \times b, ~~
  c \subset a \to (a \times b)|_{c} = c \times b
\]
が成り立つ.
またこれらから, 次の($*$)が成り立つ: 

($*$) ~~$a \subset c$が成り立つならば, $(a \times b)|_{c} = a \times b$が成り立つ.
        また$c \subset a$が成り立つならば, $(a \times b)|_{c} = c \times b$が成り立つ.
\end{thm}


\noindent{\bf 証明}
~1)
定理 \ref{sthmproductprset}より${\rm pr}_{2}\langle a \times b \rangle \subset b$が
成り立つから, 定理 \ref{sthmrestanotherdef}により
\[
  (a \times b)|_{c} = (a \times b) \cap (c \times b)
\]
が成り立つ.
また定理 \ref{sthmcapproduct}と推論法則 \ref{ded=ch}により
\[
  (a \times b) \cap (c \times b) = (a \cap c) \times b
\]
が成り立つ.
故にこれらから, 推論法則 \ref{ded=trans}により
\[
\tag{1}
  (a \times b)|_{c} = (a \cap c) \times b
\]
が成り立つ.

\noindent
2)
定理 \ref{sthmcapsubset=}と推論法則 \ref{dedequiv}により
\begin{align*}
  \tag{2}
  a \subset c &\to a \cap c = a, \\
  \mbox{} \\
  \tag{3}
  c \subset a &\to c \cap a = c
\end{align*}
が共に成り立つ.
また定理 \ref{sthmcapch}より$c \cap a = a \cap c$が
成り立つから, 推論法則 \ref{dedaddeq=}により
\[
  c \cap a = c \leftrightarrow a \cap c = c
\]
が成り立つ.
故に推論法則 \ref{dedequiv}により
\[
  \tag{4}
  c \cap a = c \to a \cap c = c
\]
が成り立つ.
また定理 \ref{sthmproduct=}より
\begin{align*}
  \tag{5}
  a \cap c = a &\to (a \cap c) \times b = a \times b, \\
  \mbox{} \\
  \tag{6}
  a \cap c = c &\to (a \cap c) \times b = c \times b
\end{align*}
が共に成り立つ.
また示したように(1)が成り立つから, 推論法則 \ref{dedatawbtrue2}により
\begin{align*}
  \tag{7}
  (a \cap c) \times b = a \times b &\to (a \times b)|_{c} = (a \cap c) \times b \wedge (a \cap c) \times b = a \times b, \\
  \mbox{} \\
  \tag{8}
  (a \cap c) \times b = c \times b &\to (a \times b)|_{c} = (a \cap c) \times b \wedge (a \cap c) \times b = c \times b
\end{align*}
が共に成り立つ.
またThm \ref{x=ywy=ztx=z}より
\begin{align*}
  \tag{9}
  (a \times b)|_{c} = (a \cap c) \times b \wedge (a \cap c) \times b = a \times b &\to (a \times b)|_{c} = a \times b, \\
  \mbox{} \\
  \tag{10}
  (a \times b)|_{c} = (a \cap c) \times b \wedge (a \cap c) \times b = c \times b &\to (a \times b)|_{c} = c \times b
\end{align*}
が共に成り立つ.
そこで(2), (5), (7), (9)から, 推論法則 \ref{dedmmp}によって
\[
  a \subset c \to (a \times b)|_{c} = a \times b
\]
が成り立つことがわかる.
また(3), (4), (6), (8), (10)から, 同じく推論法則 \ref{dedmmp}によって
\[
  c \subset a \to (a \times b)|_{c} = c \times b
\]
が成り立つことがわかる.
($*$)が成り立つことは, これらと推論法則 \ref{dedmp}によって明らかである.
\halmos




\mathstrut
\begin{thm}
\label{sthmprsetrest}%定理
\mbox{}

1)
$a$と$f$を集合とするとき, 
\[
  {\rm pr}_{1}\langle f|_{a} \rangle = {\rm pr}_{1}\langle f \rangle \cap a, ~~
  {\rm pr}_{2}\langle f|_{a} \rangle = f[a]
\]
が成り立つ.

2)
$a$と$f$は1)と同じとするとき, 
\[
  {\rm pr}_{1}\langle f|_{a} \rangle \subset {\rm pr}_{1}\langle f \rangle, ~~
  {\rm pr}_{1}\langle f|_{a} \rangle \subset a, ~~
  {\rm pr}_{2}\langle f|_{a} \rangle \subset {\rm pr}_{2}\langle f \rangle
\]
が成り立つ.
\end{thm}


\noindent{\bf 証明}
~1)
まず前者の記号列が定理となることを示す.
$x$と$y$を, 互いに異なり, 共に$a$及び$f$の中に自由変数として現れない, 定数でない文字とする.
このとき変数法則 \ref{valrest}により$y$は$f|_{a}$の中に自由変数として現れないから, 
定理 \ref{sthmprsetelement}より
\[
\tag{1}
  x \in {\rm pr}_{1}\langle f|_{a} \rangle \leftrightarrow \exists y((x, y) \in f|_{a})
\]
が成り立つ.
また定理 \ref{sthmpairinrest}より
\[
\tag{2}
  (x, y) \in f|_{a} \leftrightarrow x \in a \wedge (x, y) \in f
\]
が成り立つから, これと$y$が定数でないことから, 
推論法則 \ref{dedalleqquansepconst}により
\[
\tag{3}
  \exists y((x, y) \in f|_{a}) \leftrightarrow \exists y(x \in a \wedge (x, y) \in f)
\]
が成り立つ.
また$y$が$x$と異なり, $a$の中に自由変数として現れないことから, 
変数法則 \ref{valfund}により$y$は$x \in a$の中に自由変数として現れないから, 
Thm \ref{thmexwrfree}より
\[
\tag{4}
  \exists y(x \in a \wedge (x, y) \in f) \leftrightarrow x \in a \wedge \exists y((x, y) \in f)
\]
が成り立つ.
また$y$が$x$と異なり, $f$の中に自由変数として現れないことから, 
定理 \ref{sthmprsetelement}と推論法則 \ref{dedeqch}により
\[
  \exists y((x, y) \in f) \leftrightarrow x \in {\rm pr}_{1}\langle f \rangle
\]
が成り立つ.
故に推論法則 \ref{dedaddeqw}により
\[
\tag{5}
  x \in a \wedge \exists y((x, y) \in f) \leftrightarrow x \in a \wedge x \in {\rm pr}_{1}\langle f \rangle
\]
が成り立つ.
またThm \ref{awblbwa}より
\[
\tag{6}
  x \in a \wedge x \in {\rm pr}_{1}\langle f \rangle \leftrightarrow x \in {\rm pr}_{1}\langle f \rangle \wedge x \in a
\]
が成り立つ.
また定理 \ref{sthmcapelement}と推論法則 \ref{dedeqch}により
\[
\tag{7}
  x \in {\rm pr}_{1}\langle f \rangle \wedge x \in a \leftrightarrow x \in {\rm pr}_{1}\langle f \rangle \cap a
\]
が成り立つ.
そこで(1), (3)---(7)から, 推論法則 \ref{dedeqtrans}によって
\[
\tag{8}
  x \in {\rm pr}_{1}\langle f|_{a} \rangle \leftrightarrow x \in {\rm pr}_{1}\langle f \rangle \cap a
\]
が成り立つことがわかる.
さていま$x$は$a$及び$f$の中に自由変数として現れないから, 
変数法則 \ref{valcap}, \ref{valprset}, \ref{valrest}により, 
$x$は${\rm pr}_{1}\langle f|_{a} \rangle$及び${\rm pr}_{1}\langle f \rangle \cap a$の中に
自由変数として現れない.
また$x$は定数でない.
これらのことと, (8)が成り立つことから, 定理 \ref{sthmset=}により
\[
\tag{9}
  {\rm pr}_{1}\langle f|_{a} \rangle = {\rm pr}_{1}\langle f \rangle \cap a
\]
が成り立つ.

次に後者の記号列が定理となることを示す.
$x$は$a$及び$f$の中に自由変数として現れないから, 変数法則 \ref{valrest}により, 
$x$は$f|_{a}$の中に自由変数として現れない.
このことと, $x$が$y$と異なることから, 定理 \ref{sthmprsetelement}より
\[
\tag{10}
  y \in {\rm pr}_{2}\langle f|_{a} \rangle \leftrightarrow \exists x((x, y) \in f|_{a})
\]
が成り立つ.
また示したように(2)が成り立つから, これと$x$が定数でないことから, 
推論法則 \ref{dedalleqquansepconst}により
\[
\tag{11}
  \exists x((x, y) \in f|_{a}) \leftrightarrow \exists x(x \in a \wedge (x, y) \in f)
\]
が成り立つ.
また$x$が$y$と異なり, $a$及び$f$の中に自由変数として現れないことから, 
定理 \ref{sthmvaluesetelement}と推論法則 \ref{dedeqch}により
\[
\tag{12}
  \exists x(x \in a \wedge (x, y) \in f) \leftrightarrow y \in f[a]
\]
が成り立つ.
そこで(10), (11), (12)から, 推論法則 \ref{dedeqtrans}によって
\[
\tag{13}
  y \in {\rm pr}_{2}\langle f|_{a} \rangle \leftrightarrow y \in f[a]
\]
が成り立つことがわかる.
さていま$y$は$a$及び$f$の中に自由変数として現れないから, 
変数法則 \ref{valprset}, \ref{valvalueset}, \ref{valrest}によってわかるように, 
$y$は${\rm pr}_{2}\langle f|_{a} \rangle$及び$f[a]$の中に自由変数として現れない.
また$y$は定数でない.
これらのことと, (13)が成り立つことから, 定理 \ref{sthmset=}により
\[
  {\rm pr}_{2}\langle f|_{a} \rangle = f[a]
\]
が成り立つ.

\noindent
2)
上で示したように(9)が成り立つ.
また定理 \ref{sthmcap}より
\begin{align*}
  \tag{14}
  {\rm pr}_{1}\langle f \rangle \cap a &\subset {\rm pr}_{1}\langle f \rangle, \\
  \mbox{} \\
  \tag{15}
  {\rm pr}_{1}\langle f \rangle \cap a &\subset a
\end{align*}
が共に成り立つ.
そこで(9)と(14), (9)と(15)から, それぞれ定理 \ref{sthm=tsubseteq}により
\[
  {\rm pr}_{1}\langle f|_{a} \rangle \subset {\rm pr}_{1}\langle f \rangle, ~~
  {\rm pr}_{1}\langle f|_{a} \rangle \subset a
\]
が成り立つ.
また定理 \ref{sthmrestbasis}より$f|_{a} \subset f$が成り立つから, 
定理 \ref{sthmprsetsubset}により
\[
  {\rm pr}_{2}\langle f|_{a} \rangle \subset {\rm pr}_{2}\langle f \rangle
\]
が成り立つ.
\halmos




\mathstrut
\begin{thm}
\label{sthmemptyrest}%定理
$a$と$f$を集合とするとき, 
\begin{align*}
  f|_{a} = \phi &\leftrightarrow {\rm pr}_{1}\langle f \rangle \cap a = \phi, \\
  \mbox{} \\
  f|_{a} = \phi &\leftrightarrow f[a] = \phi
\end{align*}
が成り立つ.
またこれらから, 次の($*$), ($**$)が成り立つ: 

($*$) ~~$f|_{a}$が空ならば, ${\rm pr}_{1}\langle f \rangle \cap a$と$f[a]$は共に空である.
        逆に${\rm pr}_{1}\langle f \rangle \cap a$が空ならば, $f|_{a}$は空である.
        また$f[a]$が空ならば, $f|_{a}$は空である.

($**$) ~~$f|_{\phi}$と$\phi|_{a}$は共に空である.
\end{thm}


\noindent{\bf 証明}
~定理 \ref{sthmrestbasis}より$f|_{a}$はグラフだから, 定理 \ref{sthmemptyprset}により
\begin{align*}
  \tag{1}
  f|_{a} = \phi &\leftrightarrow {\rm pr}_{1}\langle f|_{a} \rangle = \phi, \\
  \mbox{} \\
  \tag{2}
  f|_{a} = \phi &\leftrightarrow {\rm pr}_{2}\langle f|_{a} \rangle = \phi
\end{align*}
が共に成り立つ.
また定理 \ref{sthmprsetrest}より
\begin{align*}
  &{\rm pr}_{1}\langle f|_{a} \rangle = {\rm pr}_{1}\langle f \rangle \cap a, \\
  \mbox{} \\
  &{\rm pr}_{2}\langle f|_{a} \rangle = f[a]
\end{align*}
が共に成り立つから, それぞれから推論法則 \ref{dedaddeq=}により
\begin{align*}
  \tag{3}
  {\rm pr}_{1}\langle f|_{a} \rangle = \phi &\leftrightarrow {\rm pr}_{1}\langle f \rangle \cap a = \phi, \\
  \mbox{} \\
  \tag{4}
  {\rm pr}_{2}\langle f|_{a} \rangle = \phi &\leftrightarrow f[a] = \phi
\end{align*}
が成り立つ.
そこで(1)と(3), (2)と(4)から, それぞれ推論法則 \ref{dedeqtrans}によって
\begin{align*}
  f|_{a} = \phi &\leftrightarrow {\rm pr}_{1}\langle f \rangle \cap a = \phi, \\
  \mbox{} \\
  f|_{a} = \phi &\leftrightarrow f[a] = \phi
\end{align*}
が成り立つ.
($*$)が成り立つことは, これらと推論法則 \ref{dedeqfund}によって明らかである.

($**$)の証明: 
定理 \ref{sthmvaluesetempty}より$f[\phi]$と$\phi[a]$は共に空だから, 
($*$)により$f|_{\phi}$と$\phi|_{a}$は共に空である.
\halmos




\mathstrut
\begin{thm}
\label{sthmrestf=f}%定理
$a$と$f$を集合とするとき, 
\[
  f|_{a} = f \leftrightarrow {\rm Graph}(f) \wedge {\rm pr}_{1}\langle f \rangle \subset a
\]
が成り立つ.
またこのことから, 次の($*$), ($**$)が成り立つ: 

($*$) ~~$f|_{a} = f$が成り立つならば, $f$はグラフであり, ${\rm pr}_{1}\langle f \rangle \subset a$が成り立つ.
        逆に$f$がグラフであり, ${\rm pr}_{1}\langle f \rangle \subset a$が成り立つならば, $f|_{a} = f$が成り立つ.

($**$) ~~$f$がグラフならば, $f|_{{\rm pr}_{1}\langle f \rangle} = f$が成り立つ.
\end{thm}


\noindent{\bf 証明}
~定理 \ref{sthmgraph=}より
\[
  f|_{a} = f \to ({\rm Graph}(f|_{a}) \leftrightarrow {\rm Graph}(f))
\]
が成り立つから, 推論法則 \ref{dedprewedge}により
\[
  f|_{a} = f \to ({\rm Graph}(f|_{a}) \to {\rm Graph}(f))
\]
が成り立つ.
故に推論法則 \ref{dedch}により
\[
\tag{1}
  {\rm Graph}(f|_{a}) \to (f|_{a} = f \to {\rm Graph}(f))
\]
が成り立つ.
いま定理 \ref{sthmrestbasis}より$f|_{a}$はグラフだから, このことと(1)から, 
推論法則 \ref{dedmp}によって
\[
\tag{2}
  f|_{a} = f \to {\rm Graph}(f)
\]
が成り立つ.
また定理 \ref{sthmprset=}より
\[
\tag{3}
  f|_{a} = f \to {\rm pr}_{1}\langle f|_{a} \rangle = {\rm pr}_{1}\langle f \rangle
\]
が成り立つ.
また定理 \ref{sthmprsetrest}より
\[
  {\rm pr}_{1}\langle f|_{a} \rangle = {\rm pr}_{1}\langle f \rangle \cap a
\]
が成り立つから, 推論法則 \ref{dedaddeq=}により
\[
  {\rm pr}_{1}\langle f|_{a} \rangle = {\rm pr}_{1}\langle f \rangle 
  \leftrightarrow {\rm pr}_{1}\langle f \rangle \cap a = {\rm pr}_{1}\langle f \rangle
\]
が成り立つ.
故に推論法則 \ref{dedequiv}により
\[
\tag{4}
  {\rm pr}_{1}\langle f|_{a} \rangle = {\rm pr}_{1}\langle f \rangle 
  \to {\rm pr}_{1}\langle f \rangle \cap a = {\rm pr}_{1}\langle f \rangle
\]
が成り立つ.
また定理 \ref{sthmcapsubset=}と推論法則 \ref{dedequiv}により
\[
\tag{5}
  {\rm pr}_{1}\langle f \rangle \cap a = {\rm pr}_{1}\langle f \rangle 
  \to {\rm pr}_{1}\langle f \rangle \subset a
\]
が成り立つ.
そこで(3), (4), (5)から, 推論法則 \ref{dedmmp}によって
\[
  f|_{a} = f \to {\rm pr}_{1}\langle f \rangle \subset a
\]
が成り立つことがわかる.
故にこれと(2)から, 推論法則 \ref{dedprewedge}により
\[
\tag{6}
  f|_{a} = f \to {\rm Graph}(f) \wedge {\rm pr}_{1}\langle f \rangle \subset a
\]
が成り立つ.
また定理 \ref{sthmgraphprset}と推論法則 \ref{dedequiv}により
\[
\tag{7}
  {\rm Graph}(f) \to f \subset {\rm pr}_{1}\langle f \rangle \times {\rm pr}_{2}\langle f \rangle
\]
が成り立つ.
また定理 \ref{sthmproductsubset}より
\[
\tag{8}
  {\rm pr}_{1}\langle f \rangle \subset a 
  \to {\rm pr}_{1}\langle f \rangle \times {\rm pr}_{2}\langle f \rangle \subset a \times {\rm pr}_{2}\langle f \rangle
\]
が成り立つ.
そこで(7), (8)から, 推論法則 \ref{dedfromaddw}により
\[
\tag{9}
  {\rm Graph}(f) \wedge {\rm pr}_{1}\langle f \rangle \subset a 
  \to f \subset {\rm pr}_{1}\langle f \rangle \times {\rm pr}_{2}\langle f \rangle 
  \wedge {\rm pr}_{1}\langle f \rangle \times {\rm pr}_{2}\langle f \rangle \subset a \times {\rm pr}_{2}\langle f \rangle
\]
が成り立つ.
また定理 \ref{sthmsubsettrans}より
\[
\tag{10}
  f \subset {\rm pr}_{1}\langle f \rangle \times {\rm pr}_{2}\langle f \rangle 
  \wedge {\rm pr}_{1}\langle f \rangle \times {\rm pr}_{2}\langle f \rangle \subset a \times {\rm pr}_{2}\langle f \rangle 
  \to f \subset a \times {\rm pr}_{2}\langle f \rangle 
\]
が成り立つ.
また定理 \ref{sthmcapsubset=}と推論法則 \ref{dedequiv}により
\[
  f \subset a \times {\rm pr}_{2}\langle f \rangle 
  \to f \cap (a \times {\rm pr}_{2}\langle f \rangle) = f, 
\]
即ち
\[
\tag{11}
  f \subset a \times {\rm pr}_{2}\langle f \rangle \to f|_{a} = f
\]
が成り立つ.
そこで(9), (10), (11)から, 推論法則 \ref{dedmmp}によって
\[
  {\rm Graph}(f) \wedge {\rm pr}_{1}\langle f \rangle \subset a \to f|_{a} = f
\]
が成り立つことがわかる.
故にこれと(6)から, 推論法則 \ref{dedequiv}により
\[
  f|_{a} = f \leftrightarrow {\rm Graph}(f) \wedge {\rm pr}_{1}\langle f \rangle \subset a
\]
が成り立つ.
($*$)が成り立つことは, これと推論法則 \ref{dedwedge}, \ref{dedeqfund}によって明らかである.

($**$)の証明: 
定理 \ref{sthmsubsetself}より${\rm pr}_{1}\langle f \rangle \subset {\rm pr}_{1}\langle f \rangle$が
成り立つから, $f$がグラフならば, ($*$)により$f|_{{\rm pr}_{1}\langle f \rangle} = f$が成り立つ.
\halmos




\mathstrut
\begin{thm}
\label{sthmrestsubset2}%定理
\mbox{}

1)
$a$, $b$, $f$を集合とするとき, 
\[
  {\rm pr}_{1}\langle f \rangle \cap a \subset {\rm pr}_{1}\langle f \rangle \cap b \leftrightarrow f|_{a} \subset f|_{b}
\]
が成り立つ.
またこのことから, 次の($*$)が成り立つ: 

($*$) ~~${\rm pr}_{1}\langle f \rangle \cap a \subset {\rm pr}_{1}\langle f \rangle \cap b$が成り立つならば, 
        $f|_{a} \subset f|_{b}$が成り立つ.
        逆に$f|_{a} \subset f|_{b}$が成り立つならば, 
        ${\rm pr}_{1}\langle f \rangle \cap a \subset {\rm pr}_{1}\langle f \rangle \cap b$が成り立つ.

2)
$a$, $b$, $f$は1)と同じとするとき, 
\[
  a \subset {\rm pr}_{1}\langle f \rangle \to (a \subset b \leftrightarrow f|_{a} \subset f|_{b})
\]
が成り立つ.
またこのことから, 次の($**$)が成り立つ: 

($**$) ~~$a \subset {\rm pr}_{1}\langle f \rangle$が成り立つならば, 
         \[
           a \subset b \leftrightarrow f|_{a} \subset f|_{b}
         \]
         が成り立つ.
         故にこのとき特に, $f|_{a} \subset f|_{b}$が成り立つならば, $a \subset b$が成り立つ.
\end{thm}


\noindent{\bf 証明}
~1)
定理 \ref{sthmrestsubset}より
\[
\tag{1}
  {\rm pr}_{1}\langle f \rangle \cap a \subset {\rm pr}_{1}\langle f \rangle \cap b 
  \to f|_{{\rm pr}_{1}\langle f \rangle \cap a} \subset f|_{{\rm pr}_{1}\langle f \rangle \cap b}
\]
が成り立つ.
また定理 \ref{sthmrestcappr1set}より
\[
  f|_{a} = f|_{{\rm pr}_{1}\langle f \rangle \cap a}, ~~
  f|_{b} = f|_{{\rm pr}_{1}\langle f \rangle \cap b}
\]
が共に成り立つから, それぞれから定理 \ref{sthm=tsubseteq}により
\begin{align*}
  f|_{a} \subset f|_{{\rm pr}_{1}\langle f \rangle \cap b} 
  &\leftrightarrow f|_{{\rm pr}_{1}\langle f \rangle \cap a} \subset f|_{{\rm pr}_{1}\langle f \rangle \cap b}, \\
  \mbox{} \\
  f|_{a} \subset f|_{b} 
  &\leftrightarrow f|_{a} \subset f|_{{\rm pr}_{1}\langle f \rangle \cap b}
\end{align*}
が成り立つ.
故にそれぞれから推論法則 \ref{dedequiv}により
\begin{align*}
  \tag{2}
  f|_{{\rm pr}_{1}\langle f \rangle \cap a} \subset f|_{{\rm pr}_{1}\langle f \rangle \cap b} 
  &\to f|_{a} \subset f|_{{\rm pr}_{1}\langle f \rangle \cap b}, \\
  \mbox{} \\
  \tag{3}
  f|_{a} \subset f|_{{\rm pr}_{1}\langle f \rangle \cap b} 
  &\to f|_{a} \subset f|_{b}
\end{align*}
が成り立つ.
そこで(1), (2), (3)から, 推論法則 \ref{dedmmp}によって
\[
\tag{4}
  {\rm pr}_{1}\langle f \rangle \cap a \subset {\rm pr}_{1}\langle f \rangle \cap b \to f|_{a} \subset f|_{b}
\]
が成り立つことがわかる.
また定理 \ref{sthmprsetsubset}より
\[
\tag{5}
  f|_{a} \subset f|_{b} \to {\rm pr}_{1}\langle f|_{a} \rangle \subset {\rm pr}_{1}\langle f|_{b} \rangle
\]
が成り立つ.
また定理 \ref{sthmprsetrest}より
\[
  {\rm pr}_{1}\langle f|_{a} \rangle = {\rm pr}_{1}\langle f \rangle \cap a, ~~
  {\rm pr}_{1}\langle f|_{b} \rangle = {\rm pr}_{1}\langle f \rangle \cap b
\]
が共に成り立つから, それぞれから定理 \ref{sthm=tsubseteq}により
\begin{align*}
  {\rm pr}_{1}\langle f|_{a} \rangle \subset {\rm pr}_{1}\langle f|_{b} \rangle 
  &\leftrightarrow {\rm pr}_{1}\langle f \rangle \cap a \subset {\rm pr}_{1}\langle f|_{b} \rangle, \\
  \mbox{} \\
  {\rm pr}_{1}\langle f \rangle \cap a \subset {\rm pr}_{1}\langle f|_{b} \rangle 
  &\leftrightarrow {\rm pr}_{1}\langle f \rangle \cap a \subset {\rm pr}_{1}\langle f \rangle \cap b
\end{align*}
が成り立つ.
故にそれぞれから推論法則 \ref{dedequiv}により
\begin{align*}
  \tag{6}
  {\rm pr}_{1}\langle f|_{a} \rangle \subset {\rm pr}_{1}\langle f|_{b} \rangle 
  &\to {\rm pr}_{1}\langle f \rangle \cap a \subset {\rm pr}_{1}\langle f|_{b} \rangle, \\
  \mbox{} \\
  \tag{7}
  {\rm pr}_{1}\langle f \rangle \cap a \subset {\rm pr}_{1}\langle f|_{b} \rangle 
  &\to {\rm pr}_{1}\langle f \rangle \cap a \subset {\rm pr}_{1}\langle f \rangle \cap b
\end{align*}
が成り立つ.
そこで(5), (6), (7)から, 推論法則 \ref{dedmmp}によって
\[
\tag{8}
  f|_{a} \subset f|_{b} \to {\rm pr}_{1}\langle f \rangle \cap a \subset {\rm pr}_{1}\langle f \rangle \cap b
\]
が成り立つことがわかる.
従って(4), (8)から, 推論法則 \ref{dedequiv}により
\[
  {\rm pr}_{1}\langle f \rangle \cap a \subset {\rm pr}_{1}\langle f \rangle \cap b \leftrightarrow f|_{a} \subset f|_{b}
\]
が成り立つ.
($*$)が成り立つことは, これと推論法則 \ref{dedeqfund}によって明らかである.

\noindent
2)
定理 \ref{sthmcapsubset=}と推論法則 \ref{dedequiv}により
\[
\tag{9}
  a \subset {\rm pr}_{1}\langle f \rangle \to a \cap {\rm pr}_{1}\langle f \rangle = a
\]
が成り立つ.
また定理 \ref{sthmcapch}より
$a \cap {\rm pr}_{1}\langle f \rangle = {\rm pr}_{1}\langle f \rangle \cap a$が成り立つから, 
推論法則 \ref{dedaddeq=}により
\[
  a \cap {\rm pr}_{1}\langle f \rangle = a \leftrightarrow {\rm pr}_{1}\langle f \rangle \cap a = a
\]
が成り立つ.
故に推論法則 \ref{dedequiv}により
\[
\tag{10}
  a \cap {\rm pr}_{1}\langle f \rangle = a \to {\rm pr}_{1}\langle f \rangle \cap a = a
\]
が成り立つ.
そこで(9), (10)から, 推論法則 \ref{dedmmp}によって
\[
  a \subset {\rm pr}_{1}\langle f \rangle \to {\rm pr}_{1}\langle f \rangle \cap a = a
\]
が成り立つ.
故にこれと(8)から, 推論法則 \ref{dedfromaddw}により
\[
\tag{11}
  a \subset {\rm pr}_{1}\langle f \rangle \wedge f|_{a} \subset f|_{b} 
  \to {\rm pr}_{1}\langle f \rangle \cap a = a \wedge {\rm pr}_{1}\langle f \rangle \cap a \subset {\rm pr}_{1}\langle f \rangle \cap b
\]
が成り立つ.
また定理 \ref{sthm=&subset}より
\[
\tag{12}
  {\rm pr}_{1}\langle f \rangle \cap a = a \wedge {\rm pr}_{1}\langle f \rangle \cap a \subset {\rm pr}_{1}\langle f \rangle \cap b 
  \to a \subset {\rm pr}_{1}\langle f \rangle \cap b
\]
が成り立つ.
また定理 \ref{sthmcap}より
${\rm pr}_{1}\langle f \rangle \cap b \subset b$が成り立つから, 
推論法則 \ref{dedatawbtrue2}により
\[
\tag{13}
  a \subset {\rm pr}_{1}\langle f \rangle \cap b 
  \to a \subset {\rm pr}_{1}\langle f \rangle \cap b \wedge {\rm pr}_{1}\langle f \rangle \cap b \subset b
\]
が成り立つ.
また定理 \ref{sthmsubsettrans}より
\[
\tag{14}
  a \subset {\rm pr}_{1}\langle f \rangle \cap b \wedge {\rm pr}_{1}\langle f \rangle \cap b \subset b 
  \to a \subset b
\]
が成り立つ.
そこで(11)---(14)から, 推論法則 \ref{dedmmp}によって
\[
  a \subset {\rm pr}_{1}\langle f \rangle \wedge f|_{a} \subset f|_{b} \to a \subset b
\]
が成り立つことがわかる.
故に推論法則 \ref{dedtwch}により
\[
\tag{15}
  a \subset {\rm pr}_{1}\langle f \rangle \to (f|_{a} \subset f|_{b} \to a \subset b)
\]
が成り立つ.
また定理 \ref{sthmrestsubset}より
$a \subset b \to f|_{a} \subset f|_{b}$が成り立つから, 
推論法則 \ref{dedatawbtrue2}により
\[
  (f|_{a} \subset f|_{b} \to a \subset b) 
  \to (a \subset b \to f|_{a} \subset f|_{b}) \wedge (f|_{a} \subset f|_{b} \to a \subset b), 
\]
即ち
\[
\tag{16}
  (f|_{a} \subset f|_{b} \to a \subset b) 
  \to (a \subset b \leftrightarrow f|_{a} \subset f|_{b})
\]
が成り立つ.
そこで(15), (16)から, 推論法則 \ref{dedmmp}によって
\[
  a \subset {\rm pr}_{1}\langle f \rangle \to (a \subset b \leftrightarrow f|_{a} \subset f|_{b})
\]
が成り立つ.
($**$)が成り立つことは, これと推論法則 \ref{dedmp}, \ref{dedeqfund}によって明らかである.
\halmos




\mathstrut
\begin{thm}
\label{sthmrest=2}%定理
\mbox{}

1)
$a$, $b$, $f$を集合とするとき, 
\[
  {\rm pr}_{1}\langle f \rangle \cap a = {\rm pr}_{1}\langle f \rangle \cap b \leftrightarrow f|_{a} = f|_{b}
\]
が成り立つ.
またこのことから, 次の($*$)が成り立つ: 

($*$) ~~${\rm pr}_{1}\langle f \rangle \cap a = {\rm pr}_{1}\langle f \rangle \cap b$が成り立つならば, 
        $f|_{a} = f|_{b}$が成り立つ.
        逆に$f|_{a} = f|_{b}$が成り立つならば, 
        ${\rm pr}_{1}\langle f \rangle \cap a = {\rm pr}_{1}\langle f \rangle \cap b$が成り立つ.

2)
$a$, $b$, $f$は1)と同じとするとき, 
\[
  a \subset {\rm pr}_{1}\langle f \rangle \wedge b \subset {\rm pr}_{1}\langle f \rangle 
  \to (a = b \leftrightarrow f|_{a} = f|_{b})
\]
が成り立つ.
またこのことから, 次の($**$)が成り立つ: 

($**$) ~~$a \subset {\rm pr}_{1}\langle f \rangle$と$b \subset {\rm pr}_{1}\langle f \rangle$が共に成り立つならば, 
         \[
           a = b \leftrightarrow f|_{a} = f|_{b}
         \]
         が成り立つ.
         故にこのとき特に, $f|_{a} = f|_{b}$が成り立つならば, $a = b$が成り立つ.
\end{thm}


\noindent{\bf 証明}
~1)
定理 \ref{sthmaxiom1}と推論法則 \ref{dedeqch}により
\[
\tag{1}
  {\rm pr}_{1}\langle f \rangle \cap a = {\rm pr}_{1}\langle f \rangle \cap b 
  \leftrightarrow {\rm pr}_{1}\langle f \rangle \cap a \subset {\rm pr}_{1}\langle f \rangle \cap b 
  \wedge {\rm pr}_{1}\langle f \rangle \cap b \subset {\rm pr}_{1}\langle f \rangle \cap a
\]
が成り立つ.
また定理 \ref{sthmrestsubset2}より
\begin{align*}
  {\rm pr}_{1}\langle f \rangle \cap a \subset {\rm pr}_{1}\langle f \rangle \cap b 
  &\leftrightarrow f|_{a} \subset f|_{b}, \\
  \mbox{} \\
  {\rm pr}_{1}\langle f \rangle \cap b \subset {\rm pr}_{1}\langle f \rangle \cap a 
  &\leftrightarrow f|_{b} \subset f|_{a}
\end{align*}
が共に成り立つから, 推論法則 \ref{dedaddeqw}により
\[
\tag{2}
  {\rm pr}_{1}\langle f \rangle \cap a \subset {\rm pr}_{1}\langle f \rangle \cap b 
  \wedge {\rm pr}_{1}\langle f \rangle \cap b \subset {\rm pr}_{1}\langle f \rangle \cap a 
  \leftrightarrow f|_{a} \subset f|_{b} \wedge f|_{b} \subset f|_{a}
\]
が成り立つ.
また定理 \ref{sthmaxiom1}より
\[
\tag{3}
  f|_{a} \subset f|_{b} \wedge f|_{b} \subset f|_{a} \leftrightarrow f|_{a} = f|_{b}
\]
が成り立つ.
そこで(1), (2), (3)から, 推論法則 \ref{dedeqtrans}によって
\[
  {\rm pr}_{1}\langle f \rangle \cap a = {\rm pr}_{1}\langle f \rangle \cap b \leftrightarrow f|_{a} = f|_{b}
\]
が成り立つことがわかる.
($*$)が成り立つことは, これと推論法則 \ref{dedeqfund}によって明らかである.

\noindent
2)
定理 \ref{sthmrestsubset2}より
\begin{align*}
  a \subset {\rm pr}_{1}\langle f \rangle &\to (a \subset b \leftrightarrow f|_{a} \subset f|_{b}), \\
  \mbox{} \\
  b \subset {\rm pr}_{1}\langle f \rangle &\to (b \subset a \leftrightarrow f|_{b} \subset f|_{a})
\end{align*}
が共に成り立つから, 推論法則 \ref{dedfromaddw}により
\[
\tag{4}
  a \subset {\rm pr}_{1}\langle f \rangle \wedge b \subset {\rm pr}_{1}\langle f \rangle 
  \to (a \subset b \leftrightarrow f|_{a} \subset f|_{b}) \wedge (b \subset a \leftrightarrow f|_{b} \subset f|_{a})
\]
が成り立つ.
またThm \ref{1alb1w1cld1t1awclbwd1}より
\[
\tag{5}
  (a \subset b \leftrightarrow f|_{a} \subset f|_{b}) \wedge (b \subset a \leftrightarrow f|_{b} \subset f|_{a}) 
  \to (a \subset b \wedge b \subset a \leftrightarrow f|_{a} \subset f|_{b} \wedge f|_{b} \subset f|_{a})
\]
が成り立つ.
また定理 \ref{sthmaxiom1}より
\begin{align*}
  a \subset b \wedge b \subset a &\leftrightarrow a = b, \\
  \mbox{} \\
  f|_{a} \subset f|_{b} \wedge f|_{b} \subset f|_{a} &\leftrightarrow f|_{a} = f|_{b}
\end{align*}
が共に成り立つから, 推論法則 \ref{dedaddeqeq}により
\[
  (a \subset b \wedge b \subset a \leftrightarrow f|_{a} \subset f|_{b} \wedge f|_{b} \subset f|_{a}) 
  \leftrightarrow (a = b \leftrightarrow f|_{a} = f|_{b})
\]
が成り立つ.
故に推論法則 \ref{dedequiv}により
\[
\tag{6}
  (a \subset b \wedge b \subset a \leftrightarrow f|_{a} \subset f|_{b} \wedge f|_{b} \subset f|_{a}) 
  \to (a = b \leftrightarrow f|_{a} = f|_{b})
\]
が成り立つ.
そこで(4), (5), (6)から, 推論法則 \ref{dedmmp}によって
\[
  a \subset {\rm pr}_{1}\langle f \rangle \wedge b \subset {\rm pr}_{1}\langle f \rangle 
  \to (a = b \leftrightarrow f|_{a} = f|_{b})
\]
が成り立つことがわかる.
($**$)が成り立つことは, これと推論法則 \ref{dedmp}, \ref{dedwedge}, \ref{dedeqfund}によって明らかである.
\halmos




\mathstrut
\begin{thm}
\label{sthmvaluesetrest}%定理
\mbox{}

1)
$a$, $b$, $f$を集合とするとき, 
\[
  (f|_{a})[b] = f[a \cap b]
\]
が成り立つ.

2)
$a$, $b$, $f$は1)と同じとするとき, 
\[
  a \subset b \to (f|_{a})[b] = f[a], ~~
  b \subset a \to (f|_{a})[b] = f[b]
\]
が成り立つ.
またこれらから, 次の($*$)が成り立つ: 

($*$) ~~$a \subset b$が成り立つならば, $(f|_{a})[b] = f[a]$が成り立つ.
        また$b \subset a$が成り立つならば, $(f|_{a})[b] = f[b]$が成り立つ.
\end{thm}


\noindent{\bf 証明}
~1)
$x$と$y$を, 互いに異なり, 共に$a$, $b$, $f$のいずれの記号列の中にも自由変数として現れない, 
定数でない文字とする.
このとき変数法則 \ref{valrest}により, $x$は$f|_{a}$の中にも自由変数として現れないから, 
定理 \ref{sthmvaluesetelement}より
\[
\tag{1}
  y \in (f|_{a})[b] \leftrightarrow \exists x(x \in b \wedge (x, y) \in f|_{a})
\]
が成り立つ.
また定理 \ref{sthmpairinrest}より
\[
  (x, y) \in f|_{a} \leftrightarrow x \in a \wedge (x, y) \in f
\]
が成り立つから, 推論法則 \ref{dedaddeqw}により
\[
\tag{2}
  x \in b \wedge (x, y) \in f|_{a} \leftrightarrow x \in b \wedge (x \in a \wedge (x, y) \in f)
\]
が成り立つ.
またThm \ref{1awb1wclaw1bwc1}と推論法則 \ref{dedeqch}により
\[
\tag{3}
  x \in b \wedge (x \in a \wedge (x, y) \in f) \leftrightarrow (x \in b \wedge x \in a) \wedge (x, y) \in f
\]
が成り立つ.
またThm \ref{awblbwa}より
\[
\tag{4}
  x \in b \wedge x \in a \leftrightarrow x \in a \wedge x \in b
\]
が成り立つ.
また定理 \ref{sthmcapelement}と推論法則 \ref{dedeqch}により
\[
\tag{5}
  x \in a \wedge x \in b \leftrightarrow x \in a \cap b
\]
が成り立つ.
そこで(4), (5)から, 推論法則 \ref{dedeqtrans}によって
\[
  x \in b \wedge x \in a \leftrightarrow x \in a \cap b
\]
が成り立つ.
故に推論法則 \ref{dedaddeqw}により
\[
\tag{6}
  (x \in b \wedge x \in a) \wedge (x, y) \in f \leftrightarrow x \in a \cap b \wedge (x, y) \in f
\]
が成り立つ.
そこで(2), (3), (6)から, 推論法則 \ref{dedeqtrans}によって
\[
  x \in b \wedge (x, y) \in f|_{a} \leftrightarrow x \in a \cap b \wedge (x, y) \in f
\]
が成り立つことがわかる.
いま$x$は定数でないから, 従って推論法則 \ref{dedalleqquansepconst}により
\[
\tag{7}
  \exists x(x \in b \wedge (x, y) \in f|_{a}) \leftrightarrow \exists x(x \in a \cap b \wedge (x, y) \in f)
\]
が成り立つ.
またいま$x$は$a$及び$b$の中に自由変数として現れないから, 
変数法則 \ref{valcap}により, $x$は$a \cap b$の中に自由変数として現れない.
このことと, $x$が$y$と異なり, $f$の中にも自由変数として現れないことから, 
定理 \ref{sthmvaluesetelement}と推論法則 \ref{dedeqch}により
\[
\tag{8}
  \exists x(x \in a \cap b \wedge (x, y) \in f) \leftrightarrow y \in f[a \cap b]
\]
が成り立つ.
そこで(1), (7), (8)から, 推論法則 \ref{dedeqtrans}によって
\[
\tag{9}
  y \in (f|_{a})[b] \leftrightarrow y \in f[a \cap b]
\]
が成り立つことがわかる.
さていま$y$は$a$, $b$, $f$のいずれの記号列の中にも自由変数として現れないから, 
変数法則 \ref{valcap}, \ref{valvalueset}, \ref{valrest}によってわかるように, 
$y$は$(f|_{a})[b]$及び$f[a \cap b]$の中に自由変数として現れない.
また$y$は定数でない.
これらのことと, (9)が成り立つことから, 定理 \ref{sthmset=}により
\[
\tag{10}
  (f|_{a})[b] = f[a \cap b]
\]
が成り立つ.

\noindent
2)
定理 \ref{sthmcapsubset=}と推論法則 \ref{dedequiv}により
\begin{align*}
  \tag{11}
  a \subset b &\to a \cap b = a, \\
  \mbox{} \\
  \tag{12}
  b \subset a &\to b \cap a = b
\end{align*}
が共に成り立つ.
また定理 \ref{sthmcapch}より$b \cap a = a \cap b$が成り立つから, 
推論法則 \ref{dedaddeq=}により
\[
  b \cap a = b \leftrightarrow a \cap b = b
\]
が成り立つ.
故に推論法則 \ref{dedequiv}により
\[
\tag{13}
  b \cap a = b \to a \cap b = b
\]
が成り立つ.
また定理 \ref{sthmvalueset=}より
\begin{align*}
  \tag{14}
  a \cap b = a &\to f[a \cap b] = f[a], \\
  \mbox{} \\
  \tag{15}
  a \cap b = b &\to f[a \cap b] = f[b]
\end{align*}
が共に成り立つ.
また示したように(10)が成り立つから, 推論法則 \ref{dedatawbtrue2}により
\begin{align*}
  \tag{16}
  f[a \cap b] = f[a] &\to (f|_{a})[b] = f[a \cap b] \wedge f[a \cap b] = f[a], \\
  \mbox{} \\
  \tag{17}
  f[a \cap b] = f[b] &\to (f|_{a})[b] = f[a \cap b] \wedge f[a \cap b] = f[b]
\end{align*}
が共に成り立つ.
またThm \ref{x=ywy=ztx=z}より
\begin{align*}
  \tag{18}
  (f|_{a})[b] = f[a \cap b] \wedge f[a \cap b] = f[a] &\to (f|_{a})[b] = f[a], \\
  \mbox{} \\
  \tag{19}
  (f|_{a})[b] = f[a \cap b] \wedge f[a \cap b] = f[b] &\to (f|_{a})[b] = f[b]
\end{align*}
が共に成り立つ.
そこで(11), (14), (16), (18)から, 推論法則 \ref{dedmmp}によって
\[
  a \subset b \to (f|_{a})[b] = f[a]
\]
が成り立つことがわかる.
また(12), (13), (15), (17), (19)から, 同じく推論法則 \ref{dedmmp}によって
\[
  b \subset a \to (f|_{a})[b] = f[b]
\]
が成り立つことがわかる.
($*$)が成り立つことは, これらと推論法則 \ref{dedmp}によって明らかである.
\halmos




\mathstrut
\begin{thm}
\label{sthmcomprest}%定理
\mbox{}

1)
$a$, $f$, $g$を集合とするとき, 
\[
  (g \circ f)|_{a} = g \circ (f|_{a})
\]
が成り立つ.

2)
$a$, $f$, $g$は1)と同じとするとき, 
\[
  {\rm pr}_{2}\langle f \rangle \subset a \to (g|_{a}) \circ f = g \circ f
\]
が成り立つ.
またこのことから, 次の($*$)が成り立つ: 

($*$) ~~${\rm pr}_{2}\langle f \rangle \subset a$が成り立つならば, 
        $(g|_{a}) \circ f = g \circ f$が成り立つ.
\end{thm}


\noindent{\bf 証明}
~1)
$x$, $y$, $z$を, どの二つも互いに異なり, どの一つも$a$, $f$, $g$の
いずれの記号列の中にも自由変数として現れない, 定数でない文字とする.
このとき特に, 変数法則 \ref{valcomp}, \ref{valrest}により, $x$と$z$は共に
$(g \circ f)|_{a}$及び$g \circ (f|_{a})$の中に自由変数として現れない.
また定理 \ref{sthmpairinrest}より
\[
\tag{1}
  (x, z) \in (g \circ f)|_{a} \leftrightarrow x \in a \wedge (x, z) \in g \circ f
\]
が成り立つ.
また$y$が$x$とも$z$とも異なり, $f$及び$g$の中に自由変数として現れないことから, 
定理 \ref{sthmpairincompeq}より
\[
\tag{2}
  (x, z) \in g \circ f \leftrightarrow \exists y((x, y) \in f \wedge (y, z) \in g)
\]
が成り立つ.
故に推論法則 \ref{dedaddeqw}により
\[
\tag{3}
  x \in a \wedge (x, z) \in g \circ f 
  \leftrightarrow x \in a \wedge \exists y((x, y) \in f \wedge (y, z) \in g)
\]
が成り立つ.
また$y$が$x$と異なり, $a$の中に自由変数として現れないことから, 変数法則 \ref{valfund}により, 
$y$は$x \in a$の中に自由変数として現れないから, 
Thm \ref{thmexwrfree}と推論法則 \ref{dedeqch}により
\[
\tag{4}
  x \in a \wedge \exists y((x, y) \in f \wedge (y, z) \in g) 
  \leftrightarrow \exists y(x \in a \wedge ((x, y) \in f \wedge (y, z) \in g))
\]
が成り立つ.
またThm \ref{1awb1wclaw1bwc1}と推論法則 \ref{dedeqch}により
\[
\tag{5}
  x \in a \wedge ((x, y) \in f \wedge (y, z) \in g) \leftrightarrow (x \in a \wedge (x, y) \in f) \wedge (y, z) \in g
\]
が成り立つ.
また定理 \ref{sthmpairinrest}と推論法則 \ref{dedeqch}により
\[
  x \in a \wedge (x, y) \in f \leftrightarrow (x, y) \in f|_{a}
\]
が成り立つから, 推論法則 \ref{dedaddeqw}により
\[
\tag{6}
  (x \in a \wedge (x, y) \in f) \wedge (y, z) \in g \leftrightarrow (x, y) \in f|_{a} \wedge (y, z) \in g
\]
が成り立つ.
そこで(5), (6)から, 推論法則 \ref{dedeqtrans}によって
\[
  x \in a \wedge ((x, y) \in f \wedge (y, z) \in g) \leftrightarrow (x, y) \in f|_{a} \wedge (y, z) \in g
\]
が成り立つ.
いま$y$は定数でないので, これから推論法則 \ref{dedalleqquansepconst}により
\[
\tag{7}
  \exists y(x \in a \wedge ((x, y) \in f \wedge (y, z) \in g)) 
  \leftrightarrow \exists y((x, y) \in f|_{a} \wedge (y, z) \in g)
\]
が成り立つ.
また$y$は$a$及び$f$の中に自由変数として現れないから, 
変数法則 \ref{valrest}により, $y$は$f|_{a}$の中に自由変数として現れない.
このことと, $y$が$x$とも$z$とも異なり, $g$の中にも自由変数として現れないことから, 
定理 \ref{sthmpairincompeq}と推論法則 \ref{dedeqch}により
\[
\tag{8}
  \exists y((x, y) \in f|_{a} \wedge (y, z) \in g) \leftrightarrow (x, z) \in g \circ (f|_{a})
\]
が成り立つ.
そこで(1), (3), (4), (7), (8)から, 推論法則 \ref{dedeqtrans}によって
\[
\tag{9}
  (x, z) \in (g \circ f)|_{a} \leftrightarrow (x, z) \in g \circ (f|_{a})
\]
が成り立つことがわかる.
さていま定理 \ref{sthmrestbasis}より$(g \circ f)|_{a}$はグラフであり, 
定理 \ref{sthmcompgraph}より$g \circ (f|_{a})$もグラフである.
また上述のように, $x$と$z$は互いに異なり, 共に定数でなく, 共に$(g \circ f)|_{a}$及び
$g \circ (f|_{a})$の中に自由変数として現れない.
これらのことと, (9)が成り立つことから, 定理 \ref{sthmgraphpair=}により
\[
  (g \circ f)|_{a} = g \circ (f|_{a})
\]
が成り立つ.

\noindent
2)
$x$, $y$, $z$は上と同じとするとき, 示したように(2)が成り立つから, 推論法則 \ref{dedequiv}により
\[
  (x, z) \in g \circ f \to \exists y((x, y) \in f \wedge (y, z) \in g)
\]
が成り立つ.
ここで$\tau_{y}((x, y) \in f \wedge (y, z) \in g)$を$T$と書けば, $T$は集合であり, 
定義から上記の記号列は
\[
  (x, z) \in g \circ f \to (T|y)((x, y) \in f \wedge (y, z) \in g)
\]
と同じである.
また$y$が$x$とも$z$とも異なり, $f$及び$g$の中に自由変数として現れないことから, 
代入法則 \ref{substfree}, \ref{substfund}, \ref{substwedge}, \ref{substpair}により, 
この記号列は
\[
  (x, z) \in g \circ f \to (x, T) \in f \wedge (T, z) \in g
\]
と一致する.
よってこれが定理となる.
故に推論法則 \ref{dedaddw}により
\[
\tag{10}
  {\rm pr}_{2}\langle f \rangle \subset a \wedge (x, z) \in g \circ f 
  \to {\rm pr}_{2}\langle f \rangle \subset a \wedge ((x, T) \in f \wedge (T, z) \in g)
\]
が成り立つ.
またThm \ref{aw1bwc1t1awb1wc}より
\[
\tag{11}
  {\rm pr}_{2}\langle f \rangle \subset a \wedge ((x, T) \in f \wedge (T, z) \in g) 
  \to ({\rm pr}_{2}\langle f \rangle \subset a \wedge (x, T) \in f) \wedge (T, z) \in g
\]
が成り立つ.
また定理 \ref{sthmpairelementinprset}より
\[
  (x, T) \in f \to x \in {\rm pr}_{1}\langle f \rangle \wedge T \in {\rm pr}_{2}\langle f \rangle
\]
が成り立つから, 推論法則 \ref{dedprewedge}により
\[
  (x, T) \in f \to T \in {\rm pr}_{2}\langle f \rangle
\]
が成り立つ.
故に推論法則 \ref{dedaddw}により
\[
\tag{12}
  {\rm pr}_{2}\langle f \rangle \subset a \wedge (x, T) \in f 
  \to {\rm pr}_{2}\langle f \rangle \subset a \wedge T \in {\rm pr}_{2}\langle f \rangle
\]
が成り立つ.
また定理 \ref{sthmsubsetbasis}より
\[
  {\rm pr}_{2}\langle f \rangle \subset a \to (T \in {\rm pr}_{2}\langle f \rangle \to T \in a)
\]
が成り立つから, 推論法則 \ref{dedtwch}により
\[
\tag{13}
  {\rm pr}_{2}\langle f \rangle \subset a \wedge T \in {\rm pr}_{2}\langle f \rangle \to T \in a
\]
が成り立つ.
そこで(12), (13)から, 推論法則 \ref{dedmmp}によって
\[
\tag{14}
  {\rm pr}_{2}\langle f \rangle \subset a \wedge (x, T) \in f \to T \in a
\]
が成り立つ.
またThm \ref{awbta}より
\[
  {\rm pr}_{2}\langle f \rangle \subset a \wedge (x, T) \in f \to (x, T) \in f
\]
が成り立つ.
故にこれと(14)から, 推論法則 \ref{dedprewedge}により
\[
  {\rm pr}_{2}\langle f \rangle \subset a \wedge (x, T) \in f \to (x, T) \in f \wedge T \in a
\]
が成り立つ.
故に推論法則 \ref{dedaddw}により
\[
\tag{15}
  ({\rm pr}_{2}\langle f \rangle \subset a \wedge (x, T) \in f) \wedge (T, z) \in g 
  \to ((x, T) \in f \wedge T \in a) \wedge (T, z) \in g
\]
が成り立つ.
またThm \ref{1awb1wctaw1bwc1}より
\[
\tag{16}
  ((x, T) \in f \wedge T \in a) \wedge (T, z) \in g 
  \to (x, T) \in f \wedge (T \in a \wedge (T, z) \in g)
\]
が成り立つ.
また定理 \ref{sthmpairinrest}と推論法則 \ref{dedequiv}により
\[
  T \in a \wedge (T, z) \in g \to (T, z) \in g|_{a}
\]
が成り立つから, 推論法則 \ref{dedaddw}により
\[
\tag{17}
  (x, T) \in f \wedge (T \in a \wedge (T, z) \in g) \to (x, T) \in f \wedge (T, z) \in g|_{a}
\]
が成り立つ.
また定理 \ref{sthmpairincompt}より
\[
\tag{18}
  (x, T) \in f \wedge (T, z) \in g|_{a} \to (x, z) \in (g|_{a}) \circ f
\]
が成り立つ.
そこで(10), (11), (15)---(18)から, 推論法則 \ref{dedmmp}によって
\[
  {\rm pr}_{2}\langle f \rangle \subset a \wedge (x, z) \in g \circ f \to (x, z) \in (g|_{a}) \circ f
\]
が成り立つことがわかる.
故に推論法則 \ref{dedtwch}により
\[
\tag{19}
  {\rm pr}_{2}\langle f \rangle \subset a \to ((x, z) \in g \circ f \to (x, z) \in (g|_{a}) \circ f)
\]
が成り立つ.
さていま$x$と$z$は共に$a$及び$f$の中に自由変数として現れないから, 
変数法則 \ref{valsubset}, \ref{valprset}により, これらは共に
${\rm pr}_{2}\langle f \rangle \subset a$の中に自由変数として現れない.
また$x$と$z$は共に定数でない.
これらのことと, (19)が成り立つことから, 推論法則 \ref{dedalltquansepfreeconst}により
\[
\tag{20}
  {\rm pr}_{2}\langle f \rangle \subset a \to \forall x(\forall z((x, z) \in g \circ f \to (x, z) \in (g|_{a}) \circ f))
\]
が成り立つことがわかる.
またいま定理 \ref{sthmcompgraph}より$g \circ f$はグラフである.
また$x$と$z$は共に$a$, $f$, $g$のいずれの記号列の中にも自由変数として現れないから, 
変数法則 \ref{valcomp}, \ref{valrest}により, 
これらは共に$g \circ f$及び$(g|_{a}) \circ f$の中に自由変数として現れない.
また$x$と$z$は互いに異なる.
以上のことから, 定理 \ref{sthmgraphpairsubset}と推論法則 \ref{dedequiv}により
\[
\tag{21}
  \forall x(\forall z((x, z) \in g \circ f \to (x, z) \in (g|_{a}) \circ f)) 
  \to g \circ f \subset (g|_{a}) \circ f
\]
が成り立つ.
また定理 \ref{sthmrestbasis}より$g|_{a} \subset g$が成り立つから, 
定理 \ref{sthmcompsubset}により
\[
  (g|_{a}) \circ f \subset g \circ f
\]
が成り立つ.
故に推論法則 \ref{dedatawbtrue2}により
\[
\tag{22}
  g \circ f \subset (g|_{a}) \circ f 
  \to (g|_{a}) \circ f \subset g \circ f \wedge g \circ f \subset (g|_{a}) \circ f
\]
が成り立つ.
また定理 \ref{sthmaxiom1}と推論法則 \ref{dedequiv}により
\[
\tag{23}
  (g|_{a}) \circ f \subset g \circ f \wedge g \circ f \subset (g|_{a}) \circ f 
  \to (g|_{a}) \circ f = g \circ f
\]
が成り立つ.
そこで(20)---(23)から, 推論法則 \ref{dedmmp}によって
\[
  {\rm pr}_{2}\langle f \rangle \subset a \to (g|_{a}) \circ f = g \circ f
\]
が成り立つことがわかる.
($*$)が成り立つことは, これと推論法則 \ref{dedmp}によって明らかである.
\halmos




\mathstrut
\begin{thm}
\label{sthmidenrest}%定理
\mbox{}

1)
$a$と$b$を集合とするとき, 
\[
  ({\rm id}_{a})|_{b} = {\rm id}_{a \cap b}
\]
が成り立つ.

2)
$a$と$b$は1)と同じとするとき, 
\[
  a \subset b \leftrightarrow ({\rm id}_{a})|_{b} = {\rm id}_{a}, ~~
  b \subset a \leftrightarrow ({\rm id}_{a})|_{b} = {\rm id}_{b}
\]
が成り立つ.
またこれらから, 次の($*$)が成り立つ: 

($*$) ~~$a \subset b$が成り立つならば, $({\rm id}_{a})|_{b} = {\rm id}_{a}$が成り立つ.
        逆に$({\rm id}_{a})|_{b} = {\rm id}_{a}$が成り立つならば, $a \subset b$が成り立つ.
        また$b \subset a$が成り立つならば, $({\rm id}_{a})|_{b} = {\rm id}_{b}$が成り立つ.
        逆に$({\rm id}_{a})|_{b} = {\rm id}_{b}$が成り立つならば, $b \subset a$が成り立つ.
\end{thm}


\noindent{\bf 証明}
~1)
$x$と$y$を, 互いに異なり, 共に$a$及び$b$の中に自由変数として現れない, 定数でない文字とする.
このとき変数法則 \ref{valcap}, \ref{validen}, \ref{valrest}によってわかるように, 
$x$と$y$は共に$({\rm id}_{a})|_{b}$及び${\rm id}_{a \cap b}$の中に自由変数として現れない.
また定理 \ref{sthmpairinrest}より
\[
\tag{1}
  (x, y) \in ({\rm id}_{a})|_{b} \leftrightarrow x \in b \wedge (x, y) \in {\rm id}_{a}
\]
が成り立つ.
また定理 \ref{sthmpairiniden}より
\[
  (x, y) \in {\rm id}_{a} \leftrightarrow x = y \wedge x \in a
\]
が成り立つから, 推論法則 \ref{dedaddeqw}により
\[
\tag{2}
  x \in b \wedge (x, y) \in {\rm id}_{a} \leftrightarrow x \in b \wedge (x = y \wedge x \in a)
\]
が成り立つ.
またThm \ref{awblbwa}より
\[
\tag{3}
  x \in b \wedge (x = y \wedge x \in a) \leftrightarrow (x = y \wedge x \in a) \wedge x \in b
\]
が成り立つ.
またThm \ref{1awb1wclaw1bwc1}より
\[
\tag{4}
  (x = y \wedge x \in a) \wedge x \in b \leftrightarrow x = y \wedge (x \in a \wedge x \in b)
\]
が成り立つ.
また定理 \ref{sthmcapelement}と推論法則 \ref{dedeqch}により
\[
  x \in a \wedge x \in b \leftrightarrow x \in a \cap b
\]
が成り立つから, 推論法則 \ref{dedaddeqw}により
\[
\tag{5}
  x = y \wedge (x \in a \wedge x \in b) \leftrightarrow x = y \wedge x \in a \cap b
\]
が成り立つ.
また定理 \ref{sthmpairiniden}と推論法則 \ref{dedeqch}により
\[
\tag{6}
  x = y \wedge x \in a \cap b \leftrightarrow (x, y) \in {\rm id}_{a \cap b}
\]
が成り立つ.
そこで(1)---(6)から, 推論法則 \ref{dedeqtrans}によって
\[
\tag{7}
  (x, y) \in ({\rm id}_{a})|_{b} \leftrightarrow (x, y) \in {\rm id}_{a \cap b}
\]
が成り立つことがわかる.
さていま定理 \ref{sthmrestbasis}より$({\rm id}_{a})|_{b}$はグラフであり, 
定理 \ref{sthmidengraph}より${\rm id}_{a \cap b}$もグラフである.
また上述のように, $x$と$y$は互いに異なり, 共に定数でなく, 
共に$({\rm id}_{a})|_{b}$及び${\rm id}_{a \cap b}$の中に自由変数として現れない.
これらのことと, (7)が成り立つことから, 定理 \ref{sthmgraphpair=}により
\[
\tag{8}
  ({\rm id}_{a})|_{b} = {\rm id}_{a \cap b}
\]
が成り立つ.

\noindent
2)
定理 \ref{sthmcapsubset=}より
\begin{align*}
  \tag{9}
  a \subset b &\leftrightarrow a \cap b = a, \\
  \mbox{} \\
  \tag{10}
  b \subset a &\leftrightarrow b \cap a = b
\end{align*}
が共に成り立つ.
また定理 \ref{sthmcapch}より$b \cap a = a \cap b$が成り立つから, 
推論法則 \ref{dedaddeq=}により
\[
\tag{11}
  b \cap a = b \leftrightarrow a \cap b = b
\]
が成り立つ.
また定理 \ref{sthmiden=}より
\begin{align*}
  \tag{12}
  a \cap b = a &\leftrightarrow {\rm id}_{a \cap b} = {\rm id}_{a}, \\
  \mbox{} \\
  \tag{13}
  a \cap b = b &\leftrightarrow {\rm id}_{a \cap b} = {\rm id}_{b}
\end{align*}
が共に成り立つ.
また示したように(8)が成り立つから, 推論法則 \ref{ded=ch}により
\[
  {\rm id}_{a \cap b} = ({\rm id}_{a})|_{b}
\]
が成り立つ.
故に推論法則 \ref{dedaddeq=}により
\begin{align*}
  \tag{14}
  {\rm id}_{a \cap b} = {\rm id}_{a} &\leftrightarrow ({\rm id}_{a})|_{b} = {\rm id}_{a}, \\
  \mbox{} \\
  \tag{15}
  {\rm id}_{a \cap b} = {\rm id}_{b} &\leftrightarrow ({\rm id}_{a})|_{b} = {\rm id}_{b}
\end{align*}
が共に成り立つ.
そこで(9), (12), (14)から, 推論法則 \ref{dedeqtrans}によって
\[
  a \subset b \leftrightarrow ({\rm id}_{a})|_{b} = {\rm id}_{a}
\]
が成り立つことがわかる.
また(10), (11), (13), (15)から, 同じく推論法則 \ref{dedeqtrans}によって
\[
  b \subset a \leftrightarrow ({\rm id}_{a})|_{b} = {\rm id}_{b}
\]
が成り立つことがわかる.
($*$)が成り立つことは, これらと推論法則 \ref{dedeqfund}によって明らかである.
\halmos




\mathstrut
\begin{thm}
\label{sthmidenrest2}%定理
$a$と$f$を集合とするとき, 
\[
  f \circ {\rm id}_{a} = f|_{a}
\]
が成り立つ.
\end{thm}


\noindent{\bf 証明}
~定理 \ref{sthmidencomp}より
\[
  f \circ {\rm id}_{a} = f \cap (a \times {\rm pr}_{2}\langle f \rangle)
\]
が成り立つが, $f|_{a}$の定義よりこの記号列は
\[
  f \circ {\rm id}_{a} = f|_{a}
\]
と一致するから, これが定理となる.
\halmos




\mathstrut
\begin{thm}
\label{sthmfuncrest}%定理
\mbox{}

1)
$c$と$f$を集合とするとき, 
\[
  {\rm Func}(f) \to {\rm Func}(f|_{c})
\]
が成り立つ.
またこのことから, 次の($*$)が成り立つ: 

($*$) ~~$f$が函数ならば, $f|_{c}$は函数である.

2)
$c$, $f$は1)と同じとし, 更に$a$を集合とする.
このとき
\[
  {\rm Func}(f; a) \to {\rm Func}(f|_{c}; a \cap c)
\]
が成り立つ.
またこのことから, 次の($**$)が成り立つ: 

($**$) ~~$f$が$a$における函数ならば, $f|_{c}$は$a \cap c$における函数である.

3)
$a$, $c$, $f$は2)と同じとし, 更に$b$を集合とする.
このとき
\[
  {\rm Func}(f; a; b) \to {\rm Func}(f|_{c}; a \cap c; b)
\]
が成り立つ.
またこのことから, 次の(${**}*$)が成り立つ: 

(${**}*$) ~~$f$が$a$から$b$への函数ならば, $f|_{c}$は$a \cap c$から$b$への函数である.
\end{thm}


\noindent{\bf 証明}
~1)
定理 \ref{sthmrestbasis}より$f|_{c} \subset f$が成り立つから, 
定理 \ref{sthmfuncsubset}により
\[
\tag{1}
  {\rm Func}(f) \to {\rm Func}(f|_{c})
\]
が成り立つ.
($*$)が成り立つことは, これと推論法則 \ref{dedmp}によって明らかである.

\noindent
2)
定理 \ref{sthmcap=}より
\[
\tag{2}
  {\rm pr}_{1}\langle f \rangle = a \to {\rm pr}_{1}\langle f \rangle \cap c = a \cap c
\]
が成り立つ.
また定理 \ref{sthmprsetrest}より
\[
  {\rm pr}_{1}\langle f|_{c} \rangle = {\rm pr}_{1}\langle f \rangle \cap c
\]
が成り立つから, 推論法則 \ref{dedaddeq=}により
\[
  {\rm pr}_{1}\langle f|_{c} \rangle = a \cap c \leftrightarrow {\rm pr}_{1}\langle f \rangle \cap c = a \cap c
\]
が成り立つ.
故に推論法則 \ref{dedequiv}により
\[
\tag{3}
  {\rm pr}_{1}\langle f \rangle \cap c = a \cap c \to {\rm pr}_{1}\langle f|_{c} \rangle = a \cap c
\]
が成り立つ.
そこで(2), (3)から, 推論法則 \ref{dedmmp}によって
\[
  {\rm pr}_{1}\langle f \rangle = a \to {\rm pr}_{1}\langle f|_{c} \rangle = a \cap c
\]
が成り立つ.
そこでこれと(1)から, 推論法則 \ref{dedfromaddw}により
\[
  {\rm Func}(f) \wedge {\rm pr}_{1}\langle f \rangle = a 
  \to {\rm Func}(f|_{c}) \wedge {\rm pr}_{1}\langle f|_{c} \rangle = a \cap c, 
\]
即ち
\[
\tag{4}
  {\rm Func}(f; a) \to {\rm Func}(f|_{c}; a \cap c)
\]
が成り立つ.
($**$)が成り立つことは, これと推論法則 \ref{dedmp}によって明らかである.

\noindent
3)
定理 \ref{sthmprsetrest}より
${\rm pr}_{2}\langle f|_{c} \rangle \subset {\rm pr}_{2}\langle f \rangle$が
成り立つから, 推論法則 \ref{dedatawbtrue2}により
\[
\tag{5}
  {\rm pr}_{2}\langle f \rangle \subset b 
  \to {\rm pr}_{2}\langle f|_{c} \rangle \subset {\rm pr}_{2}\langle f \rangle 
  \wedge {\rm pr}_{2}\langle f \rangle \subset b
\]
が成り立つ.
また定理 \ref{sthmsubsettrans}より
\[
\tag{6}
  {\rm pr}_{2}\langle f|_{c} \rangle \subset {\rm pr}_{2}\langle f \rangle 
  \wedge {\rm pr}_{2}\langle f \rangle \subset b 
  \to {\rm pr}_{2}\langle f|_{c} \rangle \subset b
\]
が成り立つ.
そこで(5), (6)から, 推論法則 \ref{dedmmp}によって
\[
  {\rm pr}_{2}\langle f \rangle \subset b \to {\rm pr}_{2}\langle f|_{c} \rangle \subset b
\]
が成り立つ.
故にこれと(4)から, 推論法則 \ref{dedfromaddw}により
\[
  {\rm Func}(f; a) \wedge {\rm pr}_{2}\langle f \rangle \subset b 
  \to {\rm Func}(f|_{c}; a \cap c) \wedge {\rm pr}_{2}\langle f|_{c} \rangle \subset b, 
\]
即ち
\[
  {\rm Func}(f; a; b) \to {\rm Func}(f|_{c}; a \cap c; b)
\]
が成り立つ.
(${**}*$)が成り立つことは, これと推論法則 \ref{dedmp}によって明らかである.
\halmos




\mathstrut
\begin{thm}
\label{sthmfuncrest2}%定理
\mbox{}

1)
$a$, $c$, $f$を集合とするとき, 
\[
  {\rm Func}(f; a) \to (c \subset a \to {\rm Func}(f|_{c}; c))
\]
が成り立つ.
またこのことから, 次の($*$)が成り立つ: 

($*$) ~~$f$が$a$における函数ならば, 
        \[
          c \subset a \to {\rm Func}(f|_{c}; c)
        \]
        が成り立つ.
        故にこのとき, $c \subset a$が成り立つならば, $f|_{c}$は$c$における函数である.

2)
$a$, $c$, $f$は1)と同じとし, 更に$b$を集合とする.
このとき
\[
  {\rm Func}(f; a; b) \to (c \subset a \to {\rm Func}(f|_{c}; c; b))
\]
が成り立つ.
またこのことから, 次の($**$)が成り立つ: 

($**$) ~~$f$が$a$から$b$への函数ならば, 
         \[
           c \subset a \to {\rm Func}(f|_{c}; c; b)
         \]
         が成り立つ.
         故にこのとき, $c \subset a$が成り立つならば, $f|_{c}$は$c$から$b$への函数である.
\end{thm}


\noindent{\bf 証明}
~1)
定理 \ref{sthmfuncrest}より
\[
\tag{1}
  {\rm Func}(f; a) \to {\rm Func}(f|_{c}; a \cap c)
\]
が成り立つ.
また定理 \ref{sthmcapsubset=}と推論法則 \ref{dedequiv}により
\[
\tag{2}
  c \subset a \to c \cap a = c
\]
が成り立つ.
また定理 \ref{sthmcapch}より$c \cap a = a \cap c$が成り立つから, 
推論法則 \ref{dedaddeq=}により
\[
  c \cap a = c \leftrightarrow a \cap c = c
\]
が成り立つ.
故に推論法則 \ref{dedequiv}により
\[
\tag{3}
  c \cap a = c \to a \cap c = c
\]
が成り立つ.
そこで(2), (3)から, 推論法則 \ref{dedmmp}によって
\[
\tag{4}
  c \subset a \to a \cap c = c
\]
が成り立つ.
そこでこれと(1)から, 推論法則 \ref{dedfromaddw}により
\[
\tag{5}
  {\rm Func}(f; a) \wedge c \subset a \to {\rm Func}(f|_{c}; a \cap c) \wedge a \cap c = c
\]
が成り立つ.
また定理 \ref{sthmfunc=}より
\[
  a \cap c = c \to ({\rm Func}(f|_{c}; a \cap c) \leftrightarrow {\rm Func}(f|_{c}; c))
\]
が成り立つから, 推論法則 \ref{dedprewedge}により
\[
  a \cap c = c \to ({\rm Func}(f|_{c}; a \cap c) \to {\rm Func}(f|_{c}; c))
\]
が成り立つ.
故に推論法則 \ref{dedch}により
\[
  {\rm Func}(f|_{c}; a \cap c) \to (a \cap c = c \to {\rm Func}(f|_{c}; c))
\]
が成り立ち, これから推論法則 \ref{dedtwch}により
\[
\tag{6}
  {\rm Func}(f|_{c}; a \cap c) \wedge a \cap c = c \to {\rm Func}(f|_{c}; c)
\]
が成り立つ.
そこで(5), (6)から, 推論法則 \ref{dedmmp}によって
\[
  {\rm Func}(f; a) \wedge c \subset a \to {\rm Func}(f|_{c}; c)
\]
が成り立つ.
故に推論法則 \ref{dedtwch}により
\[
  {\rm Func}(f; a) \to (c \subset a \to {\rm Func}(f|_{c}; c))
\]
が成り立つ.
($*$)が成り立つことは, これと推論法則 \ref{dedmp}によって明らかである.

\noindent
2)
定理 \ref{sthmfuncrest}より
\[
  {\rm Func}(f; a; b) \to {\rm Func}(f|_{c}; a \cap c; b)
\]
が成り立つから, これと(4)から, 推論法則 \ref{dedfromaddw}により
\[
\tag{7}
  {\rm Func}(f; a; b) \wedge c \subset a \to {\rm Func}(f|_{c}; a \cap c; b) \wedge a \cap c = c
\]
が成り立つ.
また定理 \ref{sthmfunc=}より
\[
  a \cap c = c \to ({\rm Func}(f|_{c}; a \cap c; b) \leftrightarrow {\rm Func}(f|_{c}; c; b))
\]
が成り立つから, 推論法則 \ref{dedprewedge}により
\[
  a \cap c = c \to ({\rm Func}(f|_{c}; a \cap c; b) \to {\rm Func}(f|_{c}; c; b))
\]
が成り立つ.
故に推論法則 \ref{dedch}により
\[
  {\rm Func}(f|_{c}; a \cap c; b) \to (a \cap c = c \to {\rm Func}(f|_{c}; c; b))
\]
が成り立ち, これから推論法則 \ref{dedtwch}により
\[
\tag{8}
  {\rm Func}(f|_{c}; a \cap c; b) \wedge a \cap c = c \to {\rm Func}(f|_{c}; c; b)
\]
が成り立つ.
そこで(7), (8)から, 推論法則 \ref{dedmmp}によって
\[
  {\rm Func}(f; a; b) \wedge c \subset a \to {\rm Func}(f|_{c}; c; b)
\]
が成り立つ.
故に推論法則 \ref{dedtwch}により
\[
  {\rm Func}(f; a; b) \to (c \subset a \to {\rm Func}(f|_{c}; c; b))
\]
が成り立つ.
($**$)が成り立つことは, これと推論法則 \ref{dedmp}によって明らかである.
\halmos




\mathstrut
\begin{thm}
\label{sthmfuncrestcappr1set}%定理
\mbox{}

1)
$a$, $c$, $f$を集合とするとき, 
\[
  {\rm Func}(f; a) \to f|_{c} = f|_{a \cap c}
\]
が成り立つ.
またこのことから, 次の($*$)が成り立つ: 

($*$) ~~$f$が$a$における函数ならば, $f|_{c} = f|_{a \cap c}$が成り立つ.

2)
$a$, $c$, $f$は1)と同じとし, 更に$b$を集合とする.
このとき
\[
  {\rm Func}(f; a; b) \to f|_{c} = f|_{a \cap c}
\]
が成り立つ.
またこのことから, 次の($**$)が成り立つ: 

($**$) ~~$f$が$a$から$b$への函数ならば, $f|_{c} = f|_{a \cap c}$が成り立つ.
\end{thm}


\noindent{\bf 証明}
~1)
定理 \ref{sthmfuncbasis}より
\[
\tag{1}
  {\rm Func}(f; a) \to {\rm pr}_{1}\langle f \rangle = a
\]
が成り立つ.
また定理 \ref{sthmcap=}より
\[
\tag{2}
  {\rm pr}_{1}\langle f \rangle = a \to {\rm pr}_{1}\langle f \rangle \cap c = a \cap c
\]
が成り立つ.
また定理 \ref{sthmrest=}より
\[
\tag{3}
  {\rm pr}_{1}\langle f \rangle \cap c = a \cap c \to f|_{{\rm pr}_{1}\langle f \rangle \cap c} = f|_{a \cap c}
\]
が成り立つ.
また定理 \ref{sthmrestcappr1set}より
$f|_{c} = f|_{{\rm pr}_{1}\langle f \rangle \cap c}$が成り立つから, 
推論法則 \ref{dedaddeq=}により
\[
  f|_{c} = f|_{a \cap c} \leftrightarrow f|_{{\rm pr}_{1}\langle f \rangle \cap c} = f|_{a \cap c}
\]
が成り立つ.
故に推論法則 \ref{dedequiv}により
\[
\tag{4}
  f|_{{\rm pr}_{1}\langle f \rangle \cap c} = f|_{a \cap c} \to f|_{c} = f|_{a \cap c}
\]
が成り立つ.
そこで(1)---(4)から, 推論法則 \ref{dedmmp}によって
\[
\tag{5}
  {\rm Func}(f; a) \to f|_{c} = f|_{a \cap c}
\]
が成り立つことがわかる.
($*$)が成り立つことは, これと推論法則 \ref{dedmp}によって明らかである.

\noindent
2)
定理 \ref{sthmfuncbasis}より
${\rm Func}(f; a; b) \to {\rm Func}(f; a)$が成り立つから, これと(5)から, 
推論法則 \ref{dedmmp}によって
\[
  {\rm Func}(f; a; b) \to f|_{c} = f|_{a \cap c}
\]
が成り立つ.
($**$)が成り立つことは, これと推論法則 \ref{dedmp}によって明らかである.
\halmos




\mathstrut
\begin{thm}
\label{sthmfuncrestf=f}%定理
\mbox{}

1)
$c$と$f$を集合とするとき, 
\[
  {\rm Func}(f) \to ({\rm pr}_{1}\langle f \rangle \subset c \leftrightarrow f|_{c} = f)
\]
が成り立つ.
またこのことから, 次の${(*)}_{1}$, ${(*)}_{2}$が成り立つ: 

${(*)}_{1}$ ~~$f$が函数ならば, 
              \[
                {\rm pr}_{1}\langle f \rangle \subset c \leftrightarrow f|_{c} = f
              \]
              が成り立つ.
              故にこのとき特に, ${\rm pr}_{1}\langle f \rangle \subset c$が成り立つならば, 
              $f|_{c} = f$が成り立つ.

${(*)}_{2}$ ~~$f$が函数ならば, $f|_{{\rm pr}_{1}\langle f \rangle} = f$が成り立つ.

2)
$c$と$f$は1)と同じとし, 更に$a$を集合とする.
このとき
\[
  {\rm Func}(f; a) \to (a \subset c \leftrightarrow f|_{c} = f)
\]
が成り立つ.
またこのことから, 次の${(**)}_{1}$, ${(**)}_{2}$が成り立つ: 

${(**)}_{1}$ ~~$f$が$a$における函数ならば, 
               \[
                 a \subset c \leftrightarrow f|_{c} = f
               \]
               が成り立つ.
               故にこのとき, $a \subset c$が成り立つならば$f|_{c} = f$が成り立ち, 
               逆に$f|_{c} = f$が成り立つならば$a \subset c$が成り立つ.

${(**)}_{2}$ ~~$f$が$a$における函数ならば, $f|_{a} = f$が成り立つ.

3)
$a$, $c$, $f$は2)と同じとし, 更に$b$を集合とする.
このとき
\[
  {\rm Func}(f; a; b) \to (a \subset c \leftrightarrow f|_{c} = f)
\]
が成り立つ.
またこのことから, 次の${({**}*)}_{1}$, ${({**}*)}_{2}$が成り立つ: 

${({**}*)}_{1}$ ~~$f$が$a$から$b$への函数ならば, 
                  \[
                    a \subset c \leftrightarrow f|_{c} = f
                  \]
                  が成り立つ.
                  故にこのとき, $a \subset c$が成り立つならば$f|_{c} = f$が成り立ち, 
                  逆に$f|_{c} = f$が成り立つならば$a \subset c$が成り立つ.

${({**}*)}_{2}$ ~~$f$が$a$から$b$への函数ならば, $f|_{a} = f$が成り立つ.
\end{thm}


\noindent{\bf 証明}
~1)
定理 \ref{sthmfuncbasis}より
\[
\tag{1}
  {\rm Func}(f) \to {\rm Graph}(f)
\]
が成り立つ.
また定理 \ref{sthmrestf=f}より
\[
  f|_{c} = f \leftrightarrow {\rm Graph}(f) \wedge {\rm pr}_{1}\langle f \rangle \subset c
\]
が成り立つから, 推論法則 \ref{dedequiv}により
\begin{align*}
  &f|_{c} = f \to {\rm Graph}(f) \wedge {\rm pr}_{1}\langle f \rangle \subset c, \\
  \mbox{} \\
  &{\rm Graph}(f) \wedge {\rm pr}_{1}\langle f \rangle \subset c \to f|_{c} = f
\end{align*}
が共に成り立つ.
故にこの前者から, 推論法則 \ref{dedprewedge}により
\[
\tag{2}
  f|_{c} = f \to {\rm pr}_{1}\langle f \rangle \subset c
\]
が成り立ち, 後者から, 推論法則 \ref{dedtwch}により
\[
\tag{3}
  {\rm Graph}(f) \to ({\rm pr}_{1}\langle f \rangle \subset c \to f|_{c} = f)
\]
が成り立つ.
故にこの(2)から, 推論法則 \ref{dedatawbtrue2}により
\[
  ({\rm pr}_{1}\langle f \rangle \subset c \to f|_{c} = f) 
  \to ({\rm pr}_{1}\langle f \rangle \subset c \to f|_{c} = f) 
  \wedge (f|_{c} = f \to {\rm pr}_{1}\langle f \rangle \subset c), 
\]
即ち
\[
\tag{4}
  ({\rm pr}_{1}\langle f \rangle \subset c \to f|_{c} = f) 
  \to ({\rm pr}_{1}\langle f \rangle \subset c \leftrightarrow f|_{c} = f)
\]
が成り立つ.
そこで(1), (3), (4)から, 推論法則 \ref{dedmmp}によって
\[
\tag{5}
  {\rm Func}(f) \to ({\rm pr}_{1}\langle f \rangle \subset c \leftrightarrow f|_{c} = f)
\]
が成り立つことがわかる.
${(*)}_{1}$が成り立つことは, これと推論法則 \ref{dedmp}, \ref{dedeqfund}によって明らかである.

${(*)}_{2}$の証明: 
定理 \ref{sthmsubsetself}より
${\rm pr}_{1}\langle f \rangle \subset {\rm pr}_{1}\langle f \rangle$が成り立つ.
そこで$f$が函数ならば, ${(*)}_{1}$により$f|_{{\rm pr}_{1}\langle f \rangle} = f$が成り立つ.

\noindent
2)
Thm \ref{x=yty=x}より
\[
  {\rm pr}_{1}\langle f \rangle = a \to a = {\rm pr}_{1}\langle f \rangle
\]
が成り立つ.
また定理 \ref{sthm=tsubseteq}より
\[
  a = {\rm pr}_{1}\langle f \rangle 
  \to (a \subset c \leftrightarrow {\rm pr}_{1}\langle f \rangle \subset c)
\]
が成り立つ.
そこでこれらから, 推論法則 \ref{dedmmp}によって
\[
  {\rm pr}_{1}\langle f \rangle = a 
  \to (a \subset c \leftrightarrow {\rm pr}_{1}\langle f \rangle \subset c)
\]
が成り立つ.
故にこれと(5)から, 推論法則 \ref{dedfromaddw}により
\[
  {\rm Func}(f) \wedge {\rm pr}_{1}\langle f \rangle = a 
  \to ({\rm pr}_{1}\langle f \rangle \subset c \leftrightarrow f|_{c} = f) 
  \wedge (a \subset c \leftrightarrow {\rm pr}_{1}\langle f \rangle \subset c), 
\]
即ち
\[
\tag{6}
  {\rm Func}(f; a) 
  \to ({\rm pr}_{1}\langle f \rangle \subset c \leftrightarrow f|_{c} = f) 
  \wedge (a \subset c \leftrightarrow {\rm pr}_{1}\langle f \rangle \subset c)
\]
が成り立つ.
またThm \ref{awbtbwa}より
\[
\tag{7}
  ({\rm pr}_{1}\langle f \rangle \subset c \leftrightarrow f|_{c} = f) 
  \wedge (a \subset c \leftrightarrow {\rm pr}_{1}\langle f \rangle \subset c) 
  \to (a \subset c \leftrightarrow {\rm pr}_{1}\langle f \rangle \subset c) 
  \wedge ({\rm pr}_{1}\langle f \rangle \subset c \leftrightarrow f|_{c} = f)
\]
が成り立つ.
またThm \ref{1alb1w1blc1t1alc1}より
\[
\tag{8}
  (a \subset c \leftrightarrow {\rm pr}_{1}\langle f \rangle \subset c) 
  \wedge ({\rm pr}_{1}\langle f \rangle \subset c \leftrightarrow f|_{c} = f) 
  \to (a \subset c \leftrightarrow f|_{c} = f)
\]
が成り立つ.
そこで(6), (7), (8)から, 推論法則 \ref{dedmmp}によって
\[
\tag{9}
  {\rm Func}(f; a) \to (a \subset c \leftrightarrow f|_{c} = f)
\]
が成り立つことがわかる.
${(**)}_{1}$が成り立つことは, これと推論法則 \ref{dedmp}, \ref{dedeqfund}によって明らかである.

${(**)}_{2}$の証明: 
定理 \ref{sthmsubsetself}より$a \subset a$が成り立つ.
そこで$f$が$a$における函数ならば, ${(**)}_{1}$により$f|_{a} = f$が成り立つ.

\noindent
3)
定理 \ref{sthmfuncbasis}より
${\rm Func}(f; a; b) \to {\rm Func}(f; a)$が成り立つから, これと(9)から, 
推論法則 \ref{dedmmp}によって
\[
  {\rm Func}(f; a; b) \to (a \subset c \leftrightarrow f|_{c} = f)
\]
が成り立つ.
${({**}*)}_{1}$が成り立つことは, これと推論法則 \ref{dedmp}, \ref{dedeqfund}によって明らかである.

${({**}*)}_{2}$の証明: 
$f$が$a$から$b$への函数ならば, 定理 \ref{sthmfuncbasis}により$f$は$a$における函数である.
故に${(**)}_{2}$により$f|_{a} = f$が成り立つ.
\halmos




\mathstrut
\begin{thm}
\label{sthmfuncrestsubset2}%定理
\mbox{}

1)
$a$, $b$, $c$, $d$, $f$を集合とするとき, 
\begin{align*}
  {\rm Func}(f; a) &\to (a \cap c \subset a \cap d \leftrightarrow f|_{c} \subset f|_{d}), \\
  \mbox{} \\
  {\rm Func}(f; a; b) &\to (a \cap c \subset a \cap d \leftrightarrow f|_{c} \subset f|_{d})
\end{align*}
が成り立つ.
またこれらから, 次の${(*)}_{1}$, ${(*)}_{2}$が成り立つ: 

${(*)}_{1}$ ~~$f$が$a$における函数ならば, 
              \[
                a \cap c \subset a \cap d \leftrightarrow f|_{c} \subset f|_{d}
              \]
              が成り立つ.
              故にこのとき, $a \cap c \subset a \cap d$が成り立つならば$f|_{c} \subset f|_{d}$が成り立ち, 
              逆に$f|_{c} \subset f|_{d}$が成り立つならば$a \cap c \subset a \cap d$が成り立つ.

${(*)}_{2}$ ~~$f$が$a$から$b$への函数ならば, 
              \[
                a \cap c \subset a \cap d \leftrightarrow f|_{c} \subset f|_{d}
              \]
              が成り立つ.
              故にこのとき, $a \cap c \subset a \cap d$が成り立つならば$f|_{c} \subset f|_{d}$が成り立ち, 
              逆に$f|_{c} \subset f|_{d}$が成り立つならば$a \cap c \subset a \cap d$が成り立つ.

2)
$a$, $b$, $c$, $d$, $f$は1)と同じとするとき, 
\begin{align*}
  {\rm Func}(f; a) &\to (c \subset a \to (c \subset d \leftrightarrow f|_{c} \subset f|_{d})), \\
  \mbox{} \\
  {\rm Func}(f; a; b) &\to (c \subset a \to (c \subset d \leftrightarrow f|_{c} \subset f|_{d}))
\end{align*}
が成り立つ.
またこれらから, 次の${(**)}_{1}$, ${(**)}_{2}$が成り立つ: 

${(**)}_{1}$ ~~$f$が$a$における函数ならば, 
               \[
                 c \subset a \to (c \subset d \leftrightarrow f|_{c} \subset f|_{d})
               \]
               が成り立つ.
               故にこのとき, $c \subset a$が成り立つならば
               $c \subset d \leftrightarrow f|_{c} \subset f|_{d}$が成り立つ.
               そこで更にこのとき, $f|_{c} \subset f|_{d}$が成り立つならば$c \subset d$が成り立つ.

${(**)}_{2}$ ~~$f$が$a$から$b$への函数ならば, 
               \[
                 c \subset a \to (c \subset d \leftrightarrow f|_{c} \subset f|_{d})
               \]
               が成り立つ.
               故にこのとき, $c \subset a$が成り立つならば
               $c \subset d \leftrightarrow f|_{c} \subset f|_{d}$が成り立つ.
               そこで更にこのとき, $f|_{c} \subset f|_{d}$が成り立つならば$c \subset d$が成り立つ.
\end{thm}


\noindent{\bf 証明}
~1)
定理 \ref{sthmfuncbasis}より
\[
\tag{1}
  {\rm Func}(f; a) \to {\rm pr}_{1}\langle f \rangle = a
\]
が成り立つ.
またThm \ref{x=yty=x}より
\[
\tag{2}
  {\rm pr}_{1}\langle f \rangle = a \to a = {\rm pr}_{1}\langle f \rangle
\]
が成り立つ.
またいま$x$を$c$及び$d$の中に自由変数として現れない文字とするとき, 
Thm \ref{thms5eq}より
\[
  a = {\rm pr}_{1}\langle f \rangle 
  \to ((a|x)(x \cap c \subset x \cap d) \leftrightarrow ({\rm pr}_{1}\langle f \rangle|x)(x \cap c \subset x \cap d))
\]
が成り立つが, 代入法則 \ref{substfree}, \ref{substsubset}, \ref{substcap}によればこの記号列は
\[
\tag{3}
  a = {\rm pr}_{1}\langle f \rangle 
  \to (a \cap c \subset a \cap d \leftrightarrow {\rm pr}_{1}\langle f \rangle \cap c \subset {\rm pr}_{1}\langle f \rangle \cap d)
\]
と一致するから, これが定理となる.
また定理 \ref{sthmrestsubset2}より
\[
  {\rm pr}_{1}\langle f \rangle \cap c \subset {\rm pr}_{1}\langle f \rangle \cap d \leftrightarrow f|_{c} \subset f|_{d}
\]
が成り立つから, 推論法則 \ref{dedatawbtrue2}により
\begin{multline*}
\tag{4}
  (a \cap c \subset a \cap d \leftrightarrow {\rm pr}_{1}\langle f \rangle \cap c \subset {\rm pr}_{1}\langle f \rangle \cap d) \\
  \to (a \cap c \subset a \cap d \leftrightarrow {\rm pr}_{1}\langle f \rangle \cap c \subset {\rm pr}_{1}\langle f \rangle \cap d) 
  \wedge ({\rm pr}_{1}\langle f \rangle \cap c \subset {\rm pr}_{1}\langle f \rangle \cap d \leftrightarrow f|_{c} \subset f|_{d})
\end{multline*}
が成り立つ.
またThm \ref{1alb1w1blc1t1alc1}より
\begin{multline*}
\tag{5}
  (a \cap c \subset a \cap d \leftrightarrow {\rm pr}_{1}\langle f \rangle \cap c \subset {\rm pr}_{1}\langle f \rangle \cap d) 
  \wedge ({\rm pr}_{1}\langle f \rangle \cap c \subset {\rm pr}_{1}\langle f \rangle \cap d \leftrightarrow f|_{c} \subset f|_{d}) \\
  \to (a \cap c \subset a \cap d \leftrightarrow f|_{c} \subset f|_{d})
\end{multline*}
が成り立つ.
そこで(1)---(5)から, 推論法則 \ref{dedmmp}によって
\[
\tag{6}
  {\rm Func}(f; a) \to (a \cap c \subset a \cap d \leftrightarrow f|_{c} \subset f|_{d})
\]
が成り立つことがわかる.
${(*)}_{1}$が成り立つことは, これと推論法則 \ref{dedmp}, \ref{dedeqfund}によって明らかである.
また定理 \ref{sthmfuncbasis}より
\[
\tag{7}
  {\rm Func}(f; a; b) \to {\rm Func}(f; a)
\]
が成り立つから, これと(6)から, 推論法則 \ref{dedmmp}によって
\[
  {\rm Func}(f; a; b) \to (a \cap c \subset a \cap d \leftrightarrow f|_{c} \subset f|_{d})
\]
が成り立つ.
${(*)}_{2}$が成り立つことは, これと推論法則 \ref{dedmp}, \ref{dedeqfund}によって明らかである.

\noindent
2)
示したように(1)が成り立つから, これから推論法則 \ref{dedaddw}により
\[
\tag{8}
  {\rm Func}(f; a) \wedge c \subset a \to {\rm pr}_{1}\langle f \rangle = a \wedge c \subset a
\]
が成り立つ.
また定理 \ref{sthm=&subset}より
\[
\tag{9}
  {\rm pr}_{1}\langle f \rangle = a \wedge c \subset a \to c \subset {\rm pr}_{1}\langle f \rangle
\]
が成り立つ.
また定理 \ref{sthmrestsubset2}より
\[
\tag{10}
  c \subset {\rm pr}_{1}\langle f \rangle \to (c \subset d \leftrightarrow f|_{c} \subset f|_{d})
\]
が成り立つ.
そこで(8), (9), (10)から, 推論法則 \ref{dedmmp}によって
\[
  {\rm Func}(f; a) \wedge c \subset a \to (c \subset d \leftrightarrow f|_{c} \subset f|_{d})
\]
が成り立つことがわかる.
故に推論法則 \ref{dedtwch}により
\[
\tag{11}
  {\rm Func}(f; a) \to (c \subset a \to (c \subset d \leftrightarrow f|_{c} \subset f|_{d}))
\]
が成り立つ.
${(**)}_{1}$が成り立つことは, これと推論法則 \ref{dedmp}, \ref{dedeqfund}によって明らかである.
また示したように(7)が成り立つから, これと(11)から, 推論法則 \ref{dedmmp}によって
\[
  {\rm Func}(f; a; b) \to (c \subset a \to (c \subset d \leftrightarrow f|_{c} \subset f|_{d}))
\]
が成り立つ.
${(**)}_{2}$が成り立つことは, これと推論法則 \ref{dedmp}, \ref{dedeqfund}によって明らかである.
\halmos




\mathstrut
\begin{thm}
\label{sthmfuncrest=2}%定理
\mbox{}

1)
$a$, $b$, $c$, $d$, $f$を集合とするとき, 
\begin{align*}
  {\rm Func}(f; a) &\to (a \cap c = a \cap d \leftrightarrow f|_{c} = f|_{d}), \\
  \mbox{} \\
  {\rm Func}(f; a; b) &\to (a \cap c = a \cap d \leftrightarrow f|_{c} = f|_{d})
\end{align*}
が成り立つ.
またこれらから, 次の${(*)}_{1}$, ${(*)}_{2}$が成り立つ: 

${(*)}_{1}$ ~~$f$が$a$における函数ならば, 
              \[
                a \cap c = a \cap d \leftrightarrow f|_{c} = f|_{d}
              \]
              が成り立つ.
              故にこのとき, $a \cap c = a \cap d$が成り立つならば$f|_{c} = f|_{d}$が成り立ち, 
              逆に$f|_{c} = f|_{d}$が成り立つならば$a \cap c = a \cap d$が成り立つ.

${(*)}_{2}$ ~~$f$が$a$から$b$への函数ならば, 
              \[
                a \cap c = a \cap d \leftrightarrow f|_{c} = f|_{d}
              \]
              が成り立つ.
              故にこのとき, $a \cap c = a \cap d$が成り立つならば$f|_{c} = f|_{d}$が成り立ち, 
              逆に$f|_{c} = f|_{d}$が成り立つならば$a \cap c = a \cap d$が成り立つ.

2)
$a$, $b$, $c$, $d$, $f$は1)と同じとするとき, 
\begin{align*}
  {\rm Func}(f; a) &\to (c \subset a \wedge d \subset a \to (c = d \leftrightarrow f|_{c} = f|_{d})), \\
  \mbox{} \\
  {\rm Func}(f; a; b) &\to (c \subset a \wedge d \subset a \to (c = d \leftrightarrow f|_{c} = f|_{d}))
\end{align*}
が成り立つ.
またこれらから, 次の${(**)}_{1}$, ${(**)}_{2}$が成り立つ: 

${(**)}_{1}$ ~~$f$が$a$における函数ならば, 
               \[
                 c \subset a \wedge d \subset a \to (c = d \leftrightarrow f|_{c} = f|_{d})
               \]
               が成り立つ.
               故にこのとき, $c \subset a$と$d \subset a$が共に成り立つならば, 
               $c = d \leftrightarrow f|_{c} = f|_{d}$が成り立つ.
               そこで更にこのとき, $f|_{c} = f|_{d}$が成り立つならば, $c = d$が成り立つ.

${(**)}_{2}$ ~~$f$が$a$から$b$への函数ならば, 
               \[
                 c \subset a \wedge d \subset a \to (c = d \leftrightarrow f|_{c} = f|_{d})
               \]
               が成り立つ.
               故にこのとき, $c \subset a$と$d \subset a$が共に成り立つならば, 
               $c = d \leftrightarrow f|_{c} = f|_{d}$が成り立つ.
               そこで更にこのとき, $f|_{c} = f|_{d}$が成り立つならば, $c = d$が成り立つ.
\end{thm}


\noindent{\bf 証明}
~1)
定理 \ref{sthmfuncbasis}より
\[
\tag{1}
  {\rm Func}(f; a) \to {\rm pr}_{1}\langle f \rangle = a
\]
が成り立つ.
またThm \ref{x=yty=x}より
\[
\tag{2}
  {\rm pr}_{1}\langle f \rangle = a \to a = {\rm pr}_{1}\langle f \rangle
\]
が成り立つ.
またいま$x$を$c$及び$d$の中に自由変数として現れない文字とするとき, 
Thm \ref{thms5eq}より
\[
  a = {\rm pr}_{1}\langle f \rangle 
  \to ((a|x)(x \cap c = x \cap d) \leftrightarrow ({\rm pr}_{1}\langle f \rangle|x)(x \cap c = x \cap d))
\]
が成り立つが, 代入法則 \ref{substfree}, \ref{substfund}, \ref{substcap}によればこの記号列は
\[
\tag{3}
  a = {\rm pr}_{1}\langle f \rangle 
  \to (a \cap c = a \cap d \leftrightarrow {\rm pr}_{1}\langle f \rangle \cap c = {\rm pr}_{1}\langle f \rangle \cap d)
\]
と一致するから, これが定理となる.
また定理 \ref{sthmrest=2}より
\[
  {\rm pr}_{1}\langle f \rangle \cap c = {\rm pr}_{1}\langle f \rangle \cap d \leftrightarrow f|_{c} = f|_{d}
\]
が成り立つから, 推論法則 \ref{dedatawbtrue2}により
\begin{multline*}
\tag{4}
  (a \cap c = a \cap d \leftrightarrow {\rm pr}_{1}\langle f \rangle \cap c = {\rm pr}_{1}\langle f \rangle \cap d) \\
  \to (a \cap c = a \cap d \leftrightarrow {\rm pr}_{1}\langle f \rangle \cap c = {\rm pr}_{1}\langle f \rangle \cap d) 
  \wedge ({\rm pr}_{1}\langle f \rangle \cap c = {\rm pr}_{1}\langle f \rangle \cap d \leftrightarrow f|_{c} = f|_{d})
\end{multline*}
が成り立つ.
またThm \ref{1alb1w1blc1t1alc1}より
\begin{multline*}
\tag{5}
  (a \cap c = a \cap d \leftrightarrow {\rm pr}_{1}\langle f \rangle \cap c = {\rm pr}_{1}\langle f \rangle \cap d) 
  \wedge ({\rm pr}_{1}\langle f \rangle \cap c = {\rm pr}_{1}\langle f \rangle \cap d \leftrightarrow f|_{c} = f|_{d}) \\
  \to (a \cap c = a \cap d \leftrightarrow f|_{c} = f|_{d})
\end{multline*}
が成り立つ.
そこで(1)---(5)から, 推論法則 \ref{dedmmp}によって
\[
\tag{6}
  {\rm Func}(f; a) \to (a \cap c = a \cap d \leftrightarrow f|_{c} = f|_{d})
\]
が成り立つことがわかる.
${(*)}_{1}$が成り立つことは, これと推論法則 \ref{dedmp}, \ref{dedeqfund}によって明らかである.
また定理 \ref{sthmfuncbasis}より
\[
\tag{7}
  {\rm Func}(f; a; b) \to {\rm Func}(f; a)
\]
が成り立つから, これと(6)から, 推論法則 \ref{dedmmp}によって
\[
  {\rm Func}(f; a; b) \to (a \cap c = a \cap d \leftrightarrow f|_{c} = f|_{d})
\]
が成り立つ.
${(*)}_{2}$が成り立つことは, これと推論法則 \ref{dedmp}, \ref{dedeqfund}によって明らかである.

\noindent
2)
示したように(1)が成り立つから, これから推論法則 \ref{dedaddw}により
\[
\tag{8}
  {\rm Func}(f; a) \wedge (c \subset a \wedge d \subset a) \to {\rm pr}_{1}\langle f \rangle = a \wedge (c \subset a \wedge d \subset a)
\]
が成り立つ.
またThm \ref{aw1bwc1t1awb1w1awc1}より
\[
\tag{9}
  {\rm pr}_{1}\langle f \rangle = a \wedge (c \subset a \wedge d \subset a) 
  \to ({\rm pr}_{1}\langle f \rangle = a \wedge c \subset a) \wedge ({\rm pr}_{1}\langle f \rangle = a \wedge d \subset a)
\]
が成り立つ.
また定理 \ref{sthm=&subset}より
\[
  {\rm pr}_{1}\langle f \rangle = a \wedge c \subset a \to c \subset {\rm pr}_{1}\langle f \rangle, ~~
  {\rm pr}_{1}\langle f \rangle = a \wedge d \subset a \to d \subset {\rm pr}_{1}\langle f \rangle
\]
が共に成り立つから, 推論法則 \ref{dedfromaddw}により
\[
\tag{10}
  ({\rm pr}_{1}\langle f \rangle = a \wedge c \subset a) \wedge ({\rm pr}_{1}\langle f \rangle = a \wedge d \subset a) 
  \to c \subset {\rm pr}_{1}\langle f \rangle \wedge d \subset {\rm pr}_{1}\langle f \rangle
\]
が成り立つ.
また定理 \ref{sthmrest=2}より
\[
\tag{11}
  c \subset {\rm pr}_{1}\langle f \rangle \wedge d \subset {\rm pr}_{1}\langle f \rangle \to (c = d \leftrightarrow f|_{c} = f|_{d})
\]
が成り立つ.
そこで(8)---(11)から, 推論法則 \ref{dedmmp}によって
\[
  {\rm Func}(f; a) \wedge (c \subset a \wedge d \subset a) \to (c = d \leftrightarrow f|_{c} = f|_{d})
\]
が成り立つことがわかる.
故に推論法則 \ref{dedtwch}により
\[
\tag{12}
  {\rm Func}(f; a) \to (c \subset a \wedge d \subset a \to (c = d \leftrightarrow f|_{c} = f|_{d}))
\]
が成り立つ.
${(**)}_{1}$が成り立つことは, これと推論法則 \ref{dedmp}, \ref{dedwedge}, \ref{dedeqfund}によって明らかである.
また示したように(7)が成り立つから, これと(12)から, 推論法則 \ref{dedmmp}によって
\[
  {\rm Func}(f; a; b) \to (c \subset a \wedge d \subset a \to (c = d \leftrightarrow f|_{c} = f|_{d}))
\]
が成り立つ.
${(**)}_{2}$が成り立つことは, これと推論法則 \ref{dedmp}, \ref{dedwedge}, \ref{dedeqfund}によって明らかである.
\halmos




\mathstrut
\begin{thm}
\label{sthmfuncvalrest}%定理
\mbox{}

1)
$c$, $f$, $t$を集合とするとき, 
\[
  {\rm Func}(f) \to (t \in {\rm pr}_{1}\langle f \rangle \cap c \to (f|_{c})(t) = f(t))
\]
が成り立つ.
またこのことから, 次の($*$)が成り立つ: 

($*$) ~~$f$が函数ならば, 
        \[
          t \in {\rm pr}_{1}\langle f \rangle \cap c \to (f|_{c})(t) = f(t)
        \]
        が成り立つ.
        故にこのとき, $t \in {\rm pr}_{1}\langle f \rangle \cap c$が成り立つならば, 
        $(f|_{c})(t) = f(t)$が成り立つ.

2)
$c$, $f$, $t$は1)と同じとし, 更に$a$を集合とする.
このとき
\[
  {\rm Func}(f; a) \to (t \in a \cap c \to (f|_{c})(t) = f(t))
\]
が成り立つ.
またこのことから, 次の($**$)が成り立つ: 

($**$) ~~$f$が$a$における函数ならば, 
         \[
           t \in a \cap c \to (f|_{c})(t) = f(t)
         \]
         が成り立つ.
         故にこのとき, $t \in a \cap c$が成り立つならば, 
         $(f|_{c})(t) = f(t)$が成り立つ.

3)
$a$, $c$, $f$, $t$は2)と同じとし, 更に$b$を集合とする.
このとき
\[
  {\rm Func}(f; a; b) \to (t \in a \cap c \to (f|_{c})(t) = f(t))
\]
が成り立つ.
またこのことから, 次の(${**}*$)が成り立つ: 

(${**}*$) ~~$f$が$a$から$b$への函数ならば, 
            \[
              t \in a \cap c \to (f|_{c})(t) = f(t)
            \]
            が成り立つ.
            故にこのとき, $t \in a \cap c$が成り立つならば, 
            $(f|_{c})(t) = f(t)$が成り立つ.
\end{thm}


\noindent{\bf 証明}
~1)
定理 \ref{sthmprsetrest}より
\[
  {\rm pr}_{1}\langle f|_{c} \rangle = {\rm pr}_{1}\langle f \rangle \cap c
\]
が成り立つから, 定理 \ref{sthm=tineq}により
\[
  t \in {\rm pr}_{1}\langle f|_{c} \rangle \leftrightarrow t \in {\rm pr}_{1}\langle f \rangle \cap c
\]
が成り立つ.
故に推論法則 \ref{dedequiv}により
\[
\tag{1}
  t \in {\rm pr}_{1}\langle f \rangle \cap c \to t \in {\rm pr}_{1}\langle f|_{c} \rangle
\]
が成り立つ.
また定理 \ref{sthmfuncval}と推論法則 \ref{dedequiv}により
\[
\tag{2}
  t \in {\rm pr}_{1}\langle f|_{c} \rangle \to (t, (f|_{c})(t)) \in f|_{c}
\]
が成り立つ.
また定理 \ref{sthmrestbasis}より$f|_{c} \subset f$が成り立つから, 
定理 \ref{sthmsubsetbasis}により
\[
\tag{3}
  (t, (f|_{c})(t)) \in f|_{c} \to (t, (f|_{c})(t)) \in f
\]
が成り立つ.
そこで(1), (2), (3)から, 推論法則 \ref{dedmmp}によって
\[
  t \in {\rm pr}_{1}\langle f \rangle \cap c \to (t, (f|_{c})(t)) \in f
\]
が成り立つことがわかる.
故に推論法則 \ref{dedaddw}により
\[
\tag{4}
  {\rm Func}(f) \wedge t \in {\rm pr}_{1}\langle f \rangle \cap c \to {\rm Func}(f) \wedge (t, (f|_{c})(t)) \in f
\]
が成り立つ.
また定理 \ref{sthmfuncvalbasispractical}より
\[
  {\rm Func}(f) \to ((t, (f|_{c})(t)) \in f \to (f|_{c})(t) = f(t))
\]
が成り立つから, 推論法則 \ref{dedtwch}により
\[
\tag{5}
  {\rm Func}(f) \wedge (t, (f|_{c})(t)) \in f \to (f|_{c})(t) = f(t)
\]
が成り立つ.
そこで(4), (5)から, 推論法則 \ref{dedmmp}によって
\[
\tag{6}
  {\rm Func}(f) \wedge t \in {\rm pr}_{1}\langle f \rangle \cap c \to (f|_{c})(t) = f(t)
\]
が成り立つ.
故に推論法則 \ref{dedtwch}により
\[
  {\rm Func}(f) \to (t \in {\rm pr}_{1}\langle f \rangle \cap c \to (f|_{c})(t) = f(t))
\]
が成り立つ.
($*$)が成り立つことは, これと推論法則 \ref{dedmp}によって明らかである.

\noindent
2)
${\rm Func}(f; a)$の定義から, Thm \ref{1awb1wctaw1bwc1}より
\[
\tag{7}
  {\rm Func}(f; a) \wedge t \in a \cap c 
  \to {\rm Func}(f) \wedge ({\rm pr}_{1}\langle f \rangle = a \wedge t \in a \cap c)
\]
が成り立つ.
また定理 \ref{sthmcap=}より
\[
  {\rm pr}_{1}\langle f \rangle = a \to {\rm pr}_{1}\langle f \rangle \cap c = a \cap c
\]
が成り立つから, 推論法則 \ref{dedaddw}により
\[
\tag{8}
  {\rm pr}_{1}\langle f \rangle = a \wedge t \in a \cap c 
  \to {\rm pr}_{1}\langle f \rangle \cap c = a \cap c \wedge t \in a \cap c
\]
が成り立つ.
また定理 \ref{sthm=&in}より
\[
\tag{9}
  {\rm pr}_{1}\langle f \rangle \cap c = a \cap c \wedge t \in a \cap c 
  \to t \in {\rm pr}_{1}\langle f \rangle \cap c
\]
が成り立つ.
そこで(8), (9)から, 推論法則 \ref{dedmmp}によって
\[
  {\rm pr}_{1}\langle f \rangle = a \wedge t \in a \cap c 
  \to t \in {\rm pr}_{1}\langle f \rangle \cap c
\]
が成り立つ.
故に推論法則 \ref{dedaddw}により
\[
\tag{10}
  {\rm Func}(f) \wedge ({\rm pr}_{1}\langle f \rangle = a \wedge t \in a \cap c) 
  \to {\rm Func}(f) \wedge t \in {\rm pr}_{1}\langle f \rangle \cap c
\]
が成り立つ.
そこで(7), (10), (6)から, 推論法則 \ref{dedmmp}によって
\[
  {\rm Func}(f; a) \wedge t \in a \cap c \to (f|_{c})(t) = f(t)
\]
が成り立つことがわかる.
故に推論法則 \ref{dedtwch}により
\[
\tag{11}
  {\rm Func}(f; a) \to (t \in a \cap c \to (f|_{c})(t) = f(t))
\]
が成り立つ.
($**$)が成り立つことは, これと推論法則 \ref{dedmp}によって明らかである.

\noindent
3)
定理 \ref{sthmfuncbasis}より
${\rm Func}(f; a; b) \to {\rm Func}(f; a)$が成り立つから, これと(11)から, 推論法則 \ref{dedmmp}によって
\[
  {\rm Func}(f; a; b) \to (t \in a \cap c \to (f|_{c})(t) = f(t))
\]
が成り立つ.
(${**}*$)が成り立つことは, これと推論法則 \ref{dedmp}によって明らかである.
\halmos




\mathstrut
\begin{thm}
\label{sthmfuncvalrest2}%定理
\mbox{}

1)
$c$, $f$, $t$を集合とするとき, 
\[
  {\rm Func}(f) \to (c \subset {\rm pr}_{1}\langle f \rangle \to (t \in c \to (f|_{c})(t) = f(t)))
\]
が成り立つ.
またこのことから, 次の($*$)が成り立つ: 

($*$) ~~$f$が函数ならば, 
        \[
          c \subset {\rm pr}_{1}\langle f \rangle \to (t \in c \to (f|_{c})(t) = f(t))
        \]
        が成り立つ.
        故にこのとき, $c \subset {\rm pr}_{1}\langle f \rangle$が成り立つならば, 
        $t \in c \to (f|_{c})(t) = f(t)$が成り立つ.
        そこで更にこのとき, $t \in c$が成り立つならば, 
        $(f|_{c})(t) = f(t)$が成り立つ.

2)
$c$, $f$, $t$は1)と同じとし, 更に$a$を集合とする.
このとき
\[
  {\rm Func}(f; a) \to (c \subset a \to (t \in c \to (f|_{c})(t) = f(t)))
\]
が成り立つ.
またこのことから, 次の($**$)が成り立つ: 

($**$) ~~$f$が$a$における函数ならば, 
         \[
           c \subset a \to (t \in c \to (f|_{c})(t) = f(t))
         \]
         が成り立つ.
         故にこのとき, $c \subset a$が成り立つならば, 
         $t \in c \to (f|_{c})(t) = f(t)$が成り立つ.
         そこで更にこのとき, $t \in c$が成り立つならば, 
         $(f|_{c})(t) = f(t)$が成り立つ.

3)
$a$, $c$, $f$, $t$は2)と同じとし, 更に$b$を集合とする.
このとき
\[
  {\rm Func}(f; a; b) \to (c \subset a \to (t \in c \to (f|_{c})(t) = f(t)))
\]
が成り立つ.
またこのことから, 次の(${**}*$)が成り立つ: 

(${**}*$) ~~$f$が$a$から$b$への函数ならば, 
            \[
              c \subset a \to (t \in c \to (f|_{c})(t) = f(t))
            \]
            が成り立つ.
            故にこのとき, $c \subset a$が成り立つならば, 
            $t \in c \to (f|_{c})(t) = f(t)$が成り立つ.
            そこで更にこのとき, $t \in c$が成り立つならば, 
            $(f|_{c})(t) = f(t)$が成り立つ.
\end{thm}


\noindent{\bf 証明}
~1)
Thm \ref{1awb1wctaw1bwc1}より
\[
\tag{1}
  ({\rm Func}(f) \wedge c \subset {\rm pr}_{1}\langle f \rangle) \wedge t \in c 
  \to {\rm Func}(f) \wedge (c \subset {\rm pr}_{1}\langle f \rangle \wedge t \in c)
\]
が成り立つ.
また定理 \ref{sthmcapsubset=}と推論法則 \ref{dedequiv}により
\[
\tag{2}
  c \subset {\rm pr}_{1}\langle f \rangle \to c \cap {\rm pr}_{1}\langle f \rangle = c
\]
が成り立つ.
また定理 \ref{sthmcapch}より
$c \cap {\rm pr}_{1}\langle f \rangle = {\rm pr}_{1}\langle f \rangle \cap c$が成り立つから, 
推論法則 \ref{dedaddeq=}により
\[
  c \cap {\rm pr}_{1}\langle f \rangle = c \leftrightarrow {\rm pr}_{1}\langle f \rangle \cap c = c
\]
が成り立つ.
故に推論法則 \ref{dedequiv}により
\[
\tag{3}
  c \cap {\rm pr}_{1}\langle f \rangle = c \to {\rm pr}_{1}\langle f \rangle \cap c = c
\]
が成り立つ.
そこで(2), (3)から, 推論法則 \ref{dedmmp}によって
\[
  c \subset {\rm pr}_{1}\langle f \rangle \to {\rm pr}_{1}\langle f \rangle \cap c = c
\]
が成り立つ.
故に推論法則 \ref{dedaddw}により
\[
\tag{4}
  c \subset {\rm pr}_{1}\langle f \rangle \wedge t \in c \to {\rm pr}_{1}\langle f \rangle \cap c = c \wedge t \in c
\]
が成り立つ.
また定理 \ref{sthm=&in}より
\[
\tag{5}
  {\rm pr}_{1}\langle f \rangle \cap c = c \wedge t \in c \to t \in {\rm pr}_{1}\langle f \rangle \cap c
\]
が成り立つ.
そこで(4), (5)から, 推論法則 \ref{dedmmp}によって
\[
  c \subset {\rm pr}_{1}\langle f \rangle \wedge t \in c \to t \in {\rm pr}_{1}\langle f \rangle \cap c
\]
が成り立つ.
故に推論法則 \ref{dedaddw}により
\[
\tag{6}
  {\rm Func}(f) \wedge (c \subset {\rm pr}_{1}\langle f \rangle \wedge t \in c) 
  \to {\rm Func}(f) \wedge t \in {\rm pr}_{1}\langle f \rangle \cap c
\]
が成り立つ.
また定理 \ref{sthmfuncvalrest}より
\[
  {\rm Func}(f) \to (t \in {\rm pr}_{1}\langle f \rangle \cap c \to (f|_{c})(t) = f(t))
\]
が成り立つから, 推論法則 \ref{dedtwch}により
\[
\tag{7}
  {\rm Func}(f) \wedge t \in {\rm pr}_{1}\langle f \rangle \cap c \to (f|_{c})(t) = f(t)
\]
が成り立つ.
そこで(1), (6), (7)から, 推論法則 \ref{dedmmp}によって
\[
  ({\rm Func}(f) \wedge c \subset {\rm pr}_{1}\langle f \rangle) \wedge t \in c \to (f|_{c})(t) = f(t)
\]
が成り立つことがわかる.
故に推論法則 \ref{dedtwch}により
\[
\tag{8}
  {\rm Func}(f) \wedge c \subset {\rm pr}_{1}\langle f \rangle \to (t \in c \to (f|_{c})(t) = f(t))
\]
が成り立つ.
故に再び推論法則 \ref{dedtwch}により
\[
  {\rm Func}(f) \to (c \subset {\rm pr}_{1}\langle f \rangle \to (t \in c \to (f|_{c})(t) = f(t)))
\]
が成り立つ.
($*$)が成り立つことは, これと推論法則 \ref{dedmp}によって明らかである.

\noindent
2)
${\rm Func}(f; a)$の定義から, Thm \ref{1awb1wctaw1bwc1}より
\[
\tag{9}
  {\rm Func}(f; a) \wedge c \subset a \to {\rm Func}(f) \wedge ({\rm pr}_{1}\langle f \rangle = a \wedge c \subset a)
\]
が成り立つ.
また定理 \ref{sthm=&subset}より
\[
  {\rm pr}_{1}\langle f \rangle = a \wedge c \subset a \to c \subset {\rm pr}_{1}\langle f \rangle
\]
が成り立つから, 推論法則 \ref{dedaddw}により
\[
\tag{10}
  {\rm Func}(f) \wedge ({\rm pr}_{1}\langle f \rangle = a \wedge c \subset a) 
  \to {\rm Func}(f) \wedge c \subset {\rm pr}_{1}\langle f \rangle
\]
が成り立つ.
そこで(9), (10), (8)から, 推論法則 \ref{dedmmp}によって
\[
  {\rm Func}(f; a) \wedge c \subset a \to (t \in c \to (f|_{c})(t) = f(t))
\]
が成り立つことがわかる.
故に推論法則 \ref{dedtwch}により
\[
\tag{11}
  {\rm Func}(f; a) \to (c \subset a \to (t \in c \to (f|_{c})(t) = f(t)))
\]
が成り立つ.
($**$)が成り立つことは, これと推論法則 \ref{dedmp}によって明らかである.

\noindent
3)
定理 \ref{sthmfuncbasis}より${\rm Func}(f; a; b) \to {\rm Func}(f; a)$が成り立つから, 
これと(11)から, 推論法則 \ref{dedmmp}によって
\[
  {\rm Func}(f; a; b) \to (c \subset a \to (t \in c \to (f|_{c})(t) = f(t)))
\]
が成り立つ.
(${**}*$)が成り立つことは, これと推論法則 \ref{dedmp}によって明らかである.
\halmos
%ここまで確認済



\mathstrut
\begin{thm}
\label{sthmfuncvalrestf=g}%定理
\mbox{}

1)
$e$, $f$, $g$を集合とし, $x$をこれらの中に自由変数として現れない文字とする.
このとき
\[
  {\rm Func}(f) \wedge {\rm Func}(g) 
  \to (f|_{e} = g|_{e} 
  \leftrightarrow {\rm pr}_{1}\langle f \rangle \cap e = {\rm pr}_{1}\langle g \rangle \cap e 
  \wedge \forall x(x \in {\rm pr}_{1}\langle f \rangle \cap e \to f(x) = g(x)))
\]
が成り立つ.
またこのことから, 次の${(*)}_{1}$, ${(*)}_{2}$が成り立つ: 

${(*)}_{1}$ ~~$f$と$g$が共に函数ならば, 
              \[
                f|_{e} = g|_{e} 
                \leftrightarrow {\rm pr}_{1}\langle f \rangle \cap e = {\rm pr}_{1}\langle g \rangle \cap e 
                \wedge \forall x(x \in {\rm pr}_{1}\langle f \rangle \cap e \to f(x) = g(x))
              \]
              が成り立つ.
              故にこのとき, $f|_{e} = g|_{e}$が成り立つならば
              ${\rm pr}_{1}\langle f \rangle \cap e = {\rm pr}_{1}\langle g \rangle \cap e$と
              $\forall x(x \in {\rm pr}_{1}\langle f \rangle \cap e \to f(x) = g(x))$が共に成り立ち, 
              逆に${\rm pr}_{1}\langle f \rangle \cap e = {\rm pr}_{1}\langle g \rangle \cap e$と
              $\forall x(x \in {\rm pr}_{1}\langle f \rangle \cap e \to f(x) = g(x))$が共に成り立つならば
              $f|_{e} = g|_{e}$が成り立つ.

${(*)}_{2}$ ~~$f$と$g$が共に函数であるとき, $x$が定数でなく, 
              ${\rm pr}_{1}\langle f \rangle \cap e = {\rm pr}_{1}\langle g \rangle \cap e$と
              $x \in {\rm pr}_{1}\langle f \rangle \cap e \to f(x) = g(x)$が共に成り立つならば, 
              $f|_{e} = g|_{e}$が成り立つ.

2)
$e$, $f$, $g$は1)と同じとし, 更に$a$と$c$を集合とする.
また$x$を$a$, $e$, $f$, $g$の中に自由変数として現れない文字とする.
このとき
\[
  {\rm Func}(f; a) \wedge {\rm Func}(g; c) 
  \to (f|_{e} = g|_{e} 
  \leftrightarrow a \cap e = c \cap e \wedge \forall x(x \in a \cap e \to f(x) = g(x)))
\]
が成り立つ.
またこのことから, 次の${(**)}_{1}$, ${(**)}_{2}$が成り立つ: 

${(**)}_{1}$ ~~$f$が$a$における函数であり, $g$が$c$における函数ならば, 
               \[
                 f|_{e} = g|_{e} 
                 \leftrightarrow a \cap e = c \cap e \wedge \forall x(x \in a \cap e \to f(x) = g(x))
               \]
               が成り立つ.
               故にこのとき, $f|_{e} = g|_{e}$が成り立つならば
               $a \cap e = c \cap e$と
               $\forall x(x \in a \cap e \to f(x) = g(x))$が共に成り立ち, 
               逆に$a \cap e = c \cap e$と
               $\forall x(x \in a \cap e \to f(x) = g(x))$が共に成り立つならば
               $f|_{e} = g|_{e}$が成り立つ.

${(**)}_{2}$ ~~$f$が$a$における函数であり, $g$が$c$における函数であるとき, $x$が定数でなく, 
               $a \cap e = c \cap e$と$x \in a \cap e \to f(x) = g(x)$が共に成り立つならば, 
               $f|_{e} = g|_{e}$が成り立つ.

3)
$a$, $c$, $e$, $f$, $g$, $x$は2)と同じとし, 更に$b$と$d$を集合とする.
このとき
\[
  {\rm Func}(f; a; b) \wedge {\rm Func}(g; c; d) 
  \to (f|_{e} = g|_{e} 
  \leftrightarrow a \cap e = c \cap e \wedge \forall x(x \in a \cap e \to f(x) = g(x)))
\]
が成り立つ.
またこのことから, 次の${({**}*)}_{1}$, ${({**}*)}_{2}$が成り立つ: 

${({**}*)}_{1}$ ~~$f$が$a$から$b$への函数であり, $g$が$c$から$d$への函数ならば, 
                  \[
                    f|_{e} = g|_{e} 
                    \leftrightarrow a \cap e = c \cap e \wedge \forall x(x \in a \cap e \to f(x) = g(x))
                  \]
                  が成り立つ.
                  故にこのとき, $f|_{e} = g|_{e}$が成り立つならば
                  $a \cap e = c \cap e$と
                  $\forall x(x \in a \cap e \to f(x) = g(x))$が共に成り立ち, 
                  逆に$a \cap e = c \cap e$と
                  $\forall x(x \in a \cap e \to f(x) = g(x))$が共に成り立つならば
                  $f|_{e} = g|_{e}$が成り立つ.

${({**}*)}_{2}$ ~~$f$が$a$から$b$への函数であり, $g$が$c$から$d$への函数であるとき, $x$が定数でなく, 
                  $a \cap e = c \cap e$と$x \in a \cap e \to f(x) = g(x)$が共に成り立つならば, 
                  $f|_{e} = g|_{e}$が成り立つ.
\end{thm}


\noindent{\bf 証明}
~1)





\noindent
2)







\noindent
3)












\mathstrut
\begin{thm}
\label{sthmfuncvalrestf=g2}%定理%工事中
\mbox{}

1)
$e$, $f$, $g$を集合とし, $x$をこれらの中に自由変数として現れない文字とする.
このとき
\[
  {\rm Func}(f) \wedge {\rm Func}(g) 
  \to (e \subset {\rm pr}_{1}\langle f \rangle \wedge e \subset {\rm pr}_{1}\langle g \rangle 
  \to (f|_{e} = g|_{e} \leftrightarrow \forall x(x \in e \to f(x) = g(x))))
\]
が成り立つ.
またこのことから, 次の${(*)}_{1}$, ${(*)}_{2}$が成り立つ: 

${(*)}_{1}$ ~~$f$と$g$が共に函数ならば, 
              \[
                e \subset {\rm pr}_{1}\langle f \rangle \wedge e \subset {\rm pr}_{1}\langle g \rangle 
                \to (f|_{e} = g|_{e} \leftrightarrow \forall x(x \in e \to f(x) = g(x)))
              \]
              が成り立つ.
              故にこのとき, $e \subset {\rm pr}_{1}\langle f \rangle$と$e \subset {\rm pr}_{1}\langle g \rangle$が
              共に成り立つならば, 
              \[
                f|_{e} = g|_{e} \leftrightarrow \forall x(x \in e \to f(x) = g(x))
              \]
              が成り立つ.
              そこで更にこのとき, $f|_{e} = g|_{e}$が成り立つならば$\forall x(x \in e \to f(x) = g(x))$が成り立ち, 
              逆に$\forall x(x \in e \to f(x) = g(x))$が成り立つならば$f|_{e} = g|_{e}$が成り立つ.

${(*)}_{2}$ ~~$f$と$g$が共に函数であり, 
              $e \subset {\rm pr}_{1}\langle f \rangle$と$e \subset {\rm pr}_{1}\langle g \rangle$が
              共に成り立つとする.
              また$x$が定数でなく, $x \in e \to f(x) = g(x)$が成り立つとする.
              このとき$f|_{e} = g|_{e}$が成り立つ.

2)
$e$, $f$, $g$, $x$は1)と同じとし, 更に$a$と$c$を集合とする.
このとき
\[
  {\rm Func}(f; a) \wedge {\rm Func}(g; c) 
  \to (e \subset a \wedge e \subset c 
  \to (f|_{e} = g|_{e} \leftrightarrow \forall x(x \in e \to f(x) = g(x))))
\]
が成り立つ.
またこのことから, 次の${(**)}_{1}$, ${(**)}_{2}$が成り立つ: 

${(**)}_{1}$ ~~$f$が$a$における函数であり, $g$が$c$における函数ならば, 
               \[
                 e \subset a \wedge e \subset c 
                 \to (f|_{e} = g|_{e} \leftrightarrow \forall x(x \in e \to f(x) = g(x)))
               \]
               が成り立つ.
               故にこのとき, $e \subset a$と$e \subset c$が共に成り立つならば, 
               \[
                 f|_{e} = g|_{e} \leftrightarrow \forall x(x \in e \to f(x) = g(x))
               \]
               が成り立つ.
               そこで更にこのとき, $f|_{e} = g|_{e}$が成り立つならば$\forall x(x \in e \to f(x) = g(x))$が成り立ち, 
               逆に$\forall x(x \in e \to f(x) = g(x))$が成り立つならば$f|_{e} = g|_{e}$が成り立つ.

${(**)}_{2}$ ~~$f$が$a$における函数であり, $g$が$c$における函数であるとする.
               また$e \subset a$と$e \subset c$が共に成り立つとする.
               更に$x$が定数でなく, $x \in e \to f(x) = g(x)$が成り立つとする.
               このとき$f|_{e} = g|_{e}$が成り立つ.

3)
$a$, $c$, $e$, $f$, $g$, $x$は2)と同じとし, 更に$b$と$d$を集合とする.
このとき
\[
  {\rm Func}(f; a; b) \wedge {\rm Func}(g; c; d) 
  \to (e \subset a \wedge e \subset c 
  \to (f|_{e} = g|_{e} \leftrightarrow \forall x(x \in e \to f(x) = g(x))))
\]
が成り立つ.
またこのことから, 次の${({**}*)}_{1}$, ${({**}*)}_{2}$が成り立つ: 

${({**}*)}_{1}$ ~~$f$が$a$から$b$への函数であり, $g$が$c$から$d$への函数ならば, 
                  \[
                    e \subset a \wedge e \subset c 
                    \to (f|_{e} = g|_{e} \leftrightarrow \forall x(x \in e \to f(x) = g(x)))
                  \]
                  が成り立つ.
                  故にこのとき, $e \subset a$と$e \subset c$が共に成り立つならば, 
                  \[
                    f|_{e} = g|_{e} \leftrightarrow \forall x(x \in e \to f(x) = g(x))
                  \]
                  が成り立つ.
                  そこで更にこのとき, $f|_{e} = g|_{e}$が成り立つならば$\forall x(x \in e \to f(x) = g(x))$が成り立ち, 
                  逆に$\forall x(x \in e \to f(x) = g(x))$が成り立つならば$f|_{e} = g|_{e}$が成り立つ.

${({**}*)}_{2}$ ~~$f$が$a$から$b$への函数であり, $g$が$c$から$d$への函数であるとする.
                  また$e \subset a$と$e \subset c$が共に成り立つとする.
                  更に$x$が定数でなく, $x \in e \to f(x) = g(x)$が成り立つとする.
                  このとき$f|_{e} = g|_{e}$が成り立つ.
\end{thm}


\noindent{\bf 証明}
~1)







\noindent
2)










\noindent
3)




\mathstrut
\begin{thm}
\label{sthmfuncvalrestf=g3}%定理%工事中
\mbox{}

1)
$a$, $d$, $f$, $g$を集合とし, $x$をこれらの中に自由変数として現れない文字とする.
このとき
\[
  {\rm Func}(f; a) \wedge {\rm Func}(g; a) 
  \to (f|_{d} = g|_{d} \leftrightarrow \forall x(x \in a \cap d \to f(x) = g(x)))
\]
が成り立つ.
またこのことから, 次の${(*)}_{1}$, ${(*)}_{2}$が成り立つ: 

${(*)}_{1}$ ~~$f$と$g$が共に$a$における函数ならば, 
              \[
                f|_{d} = g|_{d} \leftrightarrow \forall x(x \in a \cap d \to f(x) = g(x))
              \]
              が成り立つ.
              故にこのとき特に, $\forall x(x \in a \cap d \to f(x) = g(x))$が成り立つならば, 
              $f|_{d} = g|_{d}$が成り立つ.

${(*)}_{2}$ ~~$f$と$g$が共に$a$における函数であるとする.
              また$x$が定数でなく, $x \in a \cap d \to f(x) = g(x)$が成り立つとする.
              このとき$f|_{d} = g|_{d}$が成り立つ.

2)
$a$, $d$, $f$, $g$, $x$は1)と同じとし, 更に$b$と$c$を集合とする.
このとき
\[
  {\rm Func}(f; a; b) \wedge {\rm Func}(g; a; c) 
  \to (f|_{d} = g|_{d} \leftrightarrow \forall x(x \in a \cap d \to f(x) = g(x)))
\]
が成り立つ.
またこのことから, 次の${(**)}_{1}$, ${(**)}_{2}$が成り立つ: 

${(**)}_{1}$ ~~$f$が$a$から$b$への函数であり, $g$が$a$から$c$への函数ならば, 
               \[
                 f|_{d} = g|_{d} \leftrightarrow \forall x(x \in a \cap d \to f(x) = g(x))
               \]
               が成り立つ.
               故にこのとき特に, $\forall x(x \in a \cap d \to f(x) = g(x))$が成り立つならば, 
               $f|_{d} = g|_{d}$が成り立つ.

${(**)}_{2}$ ~~$f$が$a$から$b$への函数であり, $g$が$a$から$c$への函数であるとする.
               また$x$が定数でなく, $x \in a \cap d \to f(x) = g(x)$が成り立つとする.
               このとき$f|_{d} = g|_{d}$が成り立つ.
\end{thm}


\noindent{\bf 証明}
~1)








\noindent
2)










\mathstrut
\begin{thm}
\label{sthmfuncvalrestf=g4}%定理%工事中
\mbox{}

1)
$a$, $d$, $f$, $g$を集合とし, $x$を$d$, $f$, $g$の中に自由変数として現れない文字とする.
このとき
\[
  {\rm Func}(f; a) \wedge {\rm Func}(g; a) 
  \to (d \subset a \to (f|_{d} = g|_{d} \leftrightarrow \forall x(x \in d \to f(x) = g(x))))
\]
が成り立つ.
またこのことから, 次の($*$)が成り立つ: 

($*$) ~~$f$と$g$が共に$a$における函数ならば, 
        \[
          d \subset a \to (f|_{d} = g|_{d} \leftrightarrow \forall x(x \in d \to f(x) = g(x)))
        \]
        が成り立つ.

2)
$a$, $d$, $f$, $g$, $x$は1)と同じとし, 更に$b$と$c$を集合とする.
このとき
\[
  {\rm Func}(f; a; b) \wedge {\rm Func}(g; a; c) 
  \to (d \subset a \to (f|_{d} = g|_{d} \leftrightarrow \forall x(x \in d \to f(x) = g(x))))
\]
が成り立つ.
またこのことから, 次の($**$)が成り立つ: 

($**$) ~~$f$が$a$から$b$への函数であり, $g$が$a$から$c$への函数ならば, 
         \[
           d \subset a \to (f|_{d} = g|_{d} \leftrightarrow \forall x(x \in d \to f(x) = g(x)))
         \]
         が成り立つ.
\end{thm}


\noindent{\bf 証明}
~








\mathstrut
\begin{defo}
\label{constant}%変形
$f$を記号列とする.
また$x$と$y$を, 互いに異なり, 共に$f$の中に自由変数として現れない文字とする.
同様に$z$と$w$を, 互いに異なり, 共に$f$の中に自由変数として現れない文字とする.
このとき
\[
  \forall x(\forall y(x \in {\rm pr}_{1}\langle f \rangle \wedge y \in {\rm pr}_{1}\langle f \rangle \to f(x) = f(y))) 
  \equiv \forall z(\forall w(z \in {\rm pr}_{1}\langle f \rangle \wedge w \in {\rm pr}_{1}\langle f \rangle \to f(z) = f(w)))
\]
が成り立つ.
\end{defo}


\noindent{\bf 証明}
~









\mathstrut
\begin{defi}
\label{defconst}%定義
$f$を記号列とする.
また$x$と$y$を, 互いに異なり, 共に$f$の中に自由変数として現れない文字とする.
同様に$z$と$w$を, 互いに異なり, 共に$f$の中に自由変数として現れない文字とする.
このとき上記の変形法則 \ref{constant}によれば, 
${\rm Func}(f) \wedge \forall x(\forall y(x \in {\rm pr}_{1}\langle f \rangle \wedge y \in {\rm pr}_{1}\langle f \rangle \to f(x) = f(y)))$と
${\rm Func}(f) \wedge \forall z(\forall w(z \in {\rm pr}_{1}\langle f \rangle \wedge w \in {\rm pr}_{1}\langle f \rangle \to f(z) = f(w)))$という
二つの記号列は一致する.
$f$に対して定まるこの記号列を, ${\rm Const}(f)$と書き表す.
\end{defi}




\mathstrut
\begin{defi}
\label{defonaconst}%定義
$a$と$f$を記号列とするとき, 
${\rm Const}(f) \wedge {\rm pr}_{1}\langle f \rangle = a$という記号列を, 
${\rm Const}(f; a)$と記す.
\end{defi}




\mathstrut
\begin{defi}
\label{defatbconst}%定義
$a$, $b$, $f$を記号列とするとき, 
${\rm Const}(f; a) \wedge {\rm pr}_{2}\langle f \rangle \subset b$という記号列を, 
${\rm Const}(f; a; b)$と記す.
\end{defi}




\newpage
\setcounter{defi}{0}
\section{対象式による函数の定義}%%%%%%%%%%%%%%%%%%%%%%%%%%%%%%%%%%%%%%%%%%%%%%%%%%%%%%%%%%%%%%%%%%%%%%%%%%%%%%%




\mathstrut
\begin{defi}
\label{deffamily}%定義
$a$と$T$を記号列とし, $x$を文字とするとき, 
記号列$\{(x, T)|x \in a\}$を以降$(T)_{x \in a}$とも書き表す.
\end{defi}




\mathstrut
\begin{valu}
\label{valfamily}%変数
$a$と$T$を記号列とし, $x$を文字とする.

1)
$x$は$(T)_{x \in a}$の中に自由変数として現れない.

2)
$y$を文字とする.
$y$が$a$及び$T$の中に自由変数として現れなければ, 
$y$は$(T)_{x \in a}$の中に自由変数として現れない.
\end{valu}


\noindent{\bf 証明}
~1)
定義より$(T)_{x \in a}$は$\{(x, T)|x \in a\}$である.
変数法則 \ref{valoset}によれば, $x$はこの中に自由変数として現れない.

\noindent
2)
$y$が$x$と同じ文字である場合には1)により2)が成り立つから, 
$y$が$x$と異なる文字である場合を考える.
このとき変数法則 \ref{valpair}により, $y$は$(x, T)$の中にも自由変数として現れないから, 
変数法則 \ref{valoset}により, $y$は$\{(x, T)|x \in a\}$, 即ち$(T)_{x \in a}$の中にも
自由変数として現れない.
\halmos%ok




\mathstrut
\begin{subs}
\label{substfamilytrans}%代入
$a$と$T$を記号列とし, $x$と$y$を文字とする.
$y$が$a$及び$T$の中に自由変数として現れなければ, 
\[
  (T)_{x \in a} \equiv ((y|x)(T))_{y \in (y|x)(a)}
\]
が成り立つ.
更にこのとき, $x$が$a$の中に自由変数として現れなければ, 
\[
  (T)_{x \in a} \equiv ((y|x)(T))_{y \in a}
\]
が成り立つ.
\end{subs}


\noindent{\bf 証明}
~$y$が$x$と同じ文字である場合には代入法則 \ref{substsame}によって明らかに成り立つから, 
$y$が$x$と異なる文字である場合を考える.
このとき変数法則 \ref{valpair}により, $y$は$(x, T)$の中にも自由変数として現れないから, 
代入法則 \ref{substosettrans}によれば, $(T)_{x \in a}$, 即ち$\{(x, T)|x \in a\}$は
\[
  \{(y|x)((x, T))|y \in (y|x)(a)\}
\]
と一致する.
更に代入法則 \ref{substpair}によれば, この記号列は
\[
  \{(y, (y|x)(T))|y \in (y|x)(a)\}
\]
と一致する.
定義によれば, これは$((y|x)(T))_{y \in (y|x)(a)}$と書き表される記号列である.
以上のことから, $(T)_{x \in a}$が$((y|x)(T))_{y \in (y|x)(a)}$と一致することがわかる.
更に$x$が$a$の中に自由変数として現れなければ, 代入法則 \ref{substfree}により, 
この記号列は$((y|x)(T))_{y \in a}$と一致する.
故に後半の主張も成り立つ.
\halmos%ok




\mathstrut
\begin{subs}
\label{substfamily}%代入
$a$, $b$, $T$を記号列とし, $x$と$y$を異なる文字とする.
$x$が$b$の中に自由変数として現れなければ, 
\[
  (b|y)((T)_{x \in a}) \equiv ((b|y)(T))_{x \in (b|y)(a)}
\]
が成り立つ.
\end{subs}


\noindent{\bf 証明}
~定義より$(T)_{x \in a}$は$\{(x, T)|x \in a\}$だから, $x$が$y$と異なり, 
$b$の中に自由変数として現れないという仮定から, 代入法則 \ref{substoset}により, 
$(b|y)((T)_{x \in a})$は
\[
  \{(b|y)((x, T))|x \in (b|y)(a)\}
\]
と一致する.
また$y$が$x$と異なることと代入法則 \ref{substpair}により, $(b|y)((x, T))$は
$(x, (b|y)(T))$と一致するから, 上記の記号列は
\[
  \{(x, (b|y)(T))|x \in (b|y)(a)\}
\]
と一致する.
定義によれば, これは$((b|y)(T))_{x \in (b|y)(a)}$と書き表される記号列である.
故に本法則が成り立つ.
\halmos%ok




\mathstrut
\begin{form}
\label{formfamily}%構成
$a$と$T$が集合で, $x$が文字ならば, $(T)_{x \in a}$は集合である.
\end{form}


\noindent{\bf 証明}
~定義より$(T)_{x \in a}$は$\{(x, T)|x \in a\}$である.
$a$と$T$が集合で, $x$が文字であるとき, 
構成法則 \ref{formfund}, \ref{formoset}, \ref{formpair}によって直ちに分かるように, 
これは集合である.
\halmos%ok




\mathstrut
\begin{thm}
\label{sthmpairinfamily}%定理
$a$, $T$, $t$, $u$を集合とし, $x$を$a$の中に自由変数として現れない文字とする.
このとき
\[
  (t, u) \in (T)_{x \in a} \leftrightarrow t \in a \wedge u = (t|x)(T)
\]
が成り立つ.
またこのことから, 次の($*$)が成り立つ: 

($*$) ~~$(t, u) \in (T)_{x \in a}$が成り立つならば, $t \in a$と$u = (t|x)(T)$が共に成り立つ.
        逆に$t \in a$と$u = (t|x)(T)$が共に成り立つならば, $(t, u) \in (T)_{x \in a}$が成り立つ.
\end{thm}


\noindent{\bf 証明}
~$y$を$a$, $T$, $t$, $u$のいずれの記号列の中にも自由変数として現れない, 定数でない文字とする.
このとき, $y$が$a$及び$T$の中に自由変数として現れないことと, $x$が$a$の中に自由変数として現れないことから, 
代入法則 \ref{substfamilytrans}により, $(T)_{x \in a}$は$((y|x)(T))_{y \in a}$, 
即ち$\{(y, (y|x)(T))|y \in a\}$と一致する.
また変数法則 \ref{valpair}により, $y$は$(t, u)$の中に自由変数として現れない.
これらのことと, $y$が$a$の中に自由変数として現れないことから, 定理 \ref{sthmosetbasis}により
\[
\tag{1}
  (t, u) \in (T)_{x \in a} \leftrightarrow \exists y(y \in a \wedge (t, u) \in (y, (y|x)(T)))
\]
が成り立つことがわかる.
また定理 \ref{sthmpair}より
\[
  (t, u) = (y, (y|x)(T)) \leftrightarrow t = y \wedge u = (y|x)(T)
\]
が成り立つから, 推論法則 \ref{dedaddeqw}により
\[
\tag{2}
  y \in a \wedge (t, u) = (y, (y|x)(T)) \leftrightarrow y \in a \wedge (t = y \wedge u = (y|x)(T))
\]
が成り立つ.
またThm \ref{1awb1wclaw1bwc1}と推論法則 \ref{dedeqch}により
\[
\tag{3}
  y \in a \wedge (t = y \wedge u = (y|x)(T)) \leftrightarrow (y \in a \wedge t = y) \wedge u = (y|x)(T)
\]
が成り立つ.
またThm \ref{awblbwa}より
\[
  y \in a \wedge t = y \leftrightarrow t = y \wedge y \in a
\]
が成り立つから, 推論法則 \ref{dedaddeqw}により
\[
\tag{4}
  (y \in a \wedge t = y) \wedge u = (y|x)(T) \leftrightarrow (t = y \wedge y \in a) \wedge u = (y|x)(T)
\]
が成り立つ.
またThm \ref{1awb1wclaw1bwc1}より
\[
\tag{5}
  (t = y \wedge y \in a) \wedge u = (y|x)(T) \leftrightarrow t = y \wedge (y \in a \wedge u = (y|x)(T))
\]
が成り立つ.
またThm \ref{thmfroms5eq}と推論法則 \ref{dedeqch}により
\[
  t = y \wedge (y|y)(y \in a \wedge u = (y|x)(T)) \leftrightarrow t = y \wedge (t|y)(y \in a \wedge u = (y|x)(T))
\]
が成り立つが, $y$は$a$及び$u$の中に自由変数として現れないから, 
代入法則 \ref{substsame}, \ref{substfree}, \ref{substfund}, \ref{substwedge}によれば, 
この記号列は
\[
  t = y \wedge (y \in a \wedge u = (y|x)(T)) \leftrightarrow t = y \wedge (t \in a \wedge u = (t|y)((y|x)(T)))
\]
と一致する.
また$y$は$T$の中にも自由変数として現れないから, 代入法則 \ref{substtrans}により, 
この記号列は
\[
\tag{6}
  t = y \wedge (y \in a \wedge u = (y|x)(T)) \leftrightarrow t = y \wedge (t \in a \wedge u = (t|x)(T))
\]
と一致する.
よってこれが定理となる.
そこで(2)---(6)から, 推論法則 \ref{dedeqtrans}によって
\[
  y \in a \wedge (t, u) = (y, (y|x)(T)) \leftrightarrow t = y \wedge (t \in a \wedge u = (t|x)(T))
\]
が成り立つことがわかる.
いま$y$は定数でないので, これから推論法則 \ref{dedalleqquansepconst}により
\[
\tag{7}
  \exists y(y \in a \wedge (t, u) = (y, (y|x)(T))) \leftrightarrow \exists y(t = y \wedge (t \in a \wedge u = (t|x)(T)))
\]
が成り立つ.
また$y$が$a$, $T$, $t$, $u$のいずれの記号列の中にも自由変数として現れないことから, 
変数法則 \ref{valfund}, \ref{valsubst}, \ref{valwedge}によってわかるように, 
$y$は$t \in a \wedge u = (t|x)(T)$の中に自由変数として現れないから, 
Thm \ref{thmexwrfree}より
\[
\tag{8}
  \exists y(t = y \wedge (t \in a \wedge u = (t|x)(T))) \leftrightarrow \exists y(t = y) \wedge (t \in a \wedge u = (t|x)(T))
\]
が成り立つ.
またThm \ref{x=x}より$t = t$が成り立つが, $y$が$t$の中に自由変数として現れないことから, 
代入法則 \ref{substfree}, \ref{substfund}によりこの記号列は$(t|y)(t = y)$と一致するので, 
これが定理となる.
故に推論法則 \ref{deds4}により
\[
  \exists y(t = y)
\]
が成り立つ.
そこで推論法則 \ref{dedawblatrue2}により
\[
\tag{9}
  \exists y(t = y) \wedge (t \in a \wedge u = (t|x)(T)) \leftrightarrow t \in a \wedge u = (t|x)(T)
\]
が成り立つ.
そこで(1), (7), (8), (9)から, 推論法則 \ref{dedeqtrans}によって
\[
  (t, u) \in (T)_{x \in a} \leftrightarrow t \in a \wedge u = (t|x)(T)
\]
が成り立つことがわかる.
($*$)が成り立つことは, これと推論法則 \ref{dedwedge}, \ref{dedeqfund}によって明らかである.
\halmos




\mathstrut
上記の定理から, 直ちに次の定理 \ref{sthmpairinfamily2}が得られる.




\mathstrut
\begin{thm}
\label{sthmpairinfamily2}%定理
\mbox{}

1)
$a$と$T$を集合とし, $x$を$a$の中に自由変数として現れない文字とする.
このとき
\[
  (x, T) \in (T)_{x \in a} \leftrightarrow x \in a
\]
が成り立つ.
またこのことから, 次の($*$)が成り立つ: 

($*$) ~~$(x, T) \in (T)_{x \in a}$が成り立つならば, $x \in a$が成り立つ.
        逆に$x \in a$が成り立つならば, $(x, T) \in (T)_{x \in a}$が成り立つ.

2)
$a$, $T$, $x$は1)と同じとし, 更に$t$を集合とする.
このとき
\[
  (t, (t|x)(T)) \in (T)_{x \in a} \leftrightarrow t \in a
\]
が成り立つ.
またこのことから, 次の($**$)が成り立つ: 

($**$) ~~$(t, (t|x)(T)) \in (T)_{x \in a}$が成り立つならば, $t \in a$が成り立つ.
         逆に$t \in a$が成り立つならば, $(t, (t|x)(T)) \in (T)_{x \in a}$が成り立つ.

3)
$a$, $T$, $x$は1)と同じとし, 更に$u$を集合とする.
このとき
\[
  (x, u) \in (T)_{x \in a} \leftrightarrow x \in a \wedge u = T
\]
が成り立つ.
またこのことから, 次の(${**}*$)が成り立つ: 

(${**}*$) ~~$(x, u) \in (T)_{x \in a}$が成り立つならば, $x \in a$と$u = T$が共に成り立つ.
            逆に$x \in a$と$u = T$が共に成り立つならば, $(x, u) \in (T)_{x \in a}$が成り立つ.
\end{thm}


\noindent{\bf 証明}
~1)






\noindent
2)






\noindent
3)











\mathstrut
\begin{thm}
\label{sthmgraphfamily}%定理
$a$と$T$を集合とし, $x$を$a$の中に自由変数として現れない文字とする.
このとき$(T)_{x \in a}$はグラフである.
\end{thm}


\noindent{\bf 証明}
~定義より$(T)_{x \in a}$は$\{(x, T)|x \in a\}$である.
いま$x$が$a$の中に自由変数として現れないから, 
定理 \ref{sthmobjectsetgraph2}によれば, これはグラフである.
\halmos




\mathstrut
\begin{thm}
\label{sthmprsetfamily}%定理
$a$と$T$を集合とし, $x$を$a$の中に自由変数として現れない文字とする.
このとき
\[
  {\rm pr}_{1}\langle (T)_{x \in a} \rangle = a, ~~
  {\rm pr}_{2}\langle (T)_{x \in a} \rangle = \{T|x \in a\}
\]
が成り立つ.
\end{thm}


\noindent{\bf 証明}
~はじめに前者が成り立つことを示す.
$y$と$z$を, 互いに異なり, 共に$a$及び$T$の中に自由変数として現れない, 定数でない文字とする.
このとき変数法則 \ref{valfamily}により, $y$と$z$は共に$(T)_{x \in a}$の中に自由変数として現れない.
このことと$y$と$z$が互いに異なることから, 定理 \ref{sthmprsetelement}より
\[
\tag{1}
  z \in {\rm pr}_{1}\langle (T)_{x \in a} \rangle \leftrightarrow \exists y((z, y) \in (T)_{x \in a})
\]
が成り立つ.
また$x$が$a$の中に自由変数として現れないことから, 定理 \ref{sthmpairinfamily}より
\[
\tag{2}
  (z, y) \in (T)_{x \in a} \leftrightarrow z \in a \wedge y = (z|x)(T)
\]
が成り立つ.
いま$y$は定数でないから, これから推論法則 \ref{dedalleqquansepconst}により
\[
\tag{3}
  \exists y((z, y) \in (T)_{x \in a}) \leftrightarrow \exists y(z \in a \wedge y = (z|x)(T))
\]
が成り立つ.
また$y$が$z$と異なり, $a$の中に自由変数として現れないことから, 
変数法則 \ref{valfund}により$y$は$z \in a$の中に自由変数として現れないから, 
Thm \ref{thmexwrfree}より
\[
\tag{4}
  \exists y(z \in a \wedge y = (z|x)(T)) \leftrightarrow z \in a \wedge \exists y(y = (z|x)(T))
\]
が成り立つ.
またThm \ref{x=x}より$(z|x)(T) = (z|x)(T)$が成り立つが, 
$y$が$z$と異なり, $T$の中に自由変数として現れないことから, 
変数法則 \ref{valsubst}により$y$は$(z|x)(T)$の中に自由変数として現れないから, 
代入法則 \ref{substfree}, \ref{substfund}によれば, この記号列は
$((z|x)(T)|y)(y = (z|x)(T))$と一致する.
故にこれが定理となる.
そこで推論法則 \ref{deds4}により
\[
  \exists y(y = (z|x)(T))
\]
が成り立つ.
故に推論法則 \ref{dedawblatrue2}により
\[
\tag{5}
  z \in a \wedge \exists y(y = (z|x)(T)) \leftrightarrow z \in a
\]
が成り立つ.
そこで(1), (3), (4), (5)から, 推論法則 \ref{dedeqtrans}によって
\[
\tag{6}
  z \in {\rm pr}_{1}\langle (T)_{x \in a} \rangle \leftrightarrow z \in a
\]
が成り立つことがわかる.
さて上述のように$z$は$(T)_{x \in a}$の中に自由変数として現れないから, 
変数法則 \ref{valprset}により, $z$は${\rm pr}_{1}\langle (T)_{x \in a} \rangle$の中に自由変数として現れない.
また$z$は$a$の中にも自由変数として現れない.
また$z$は定数でない.
これらのことと, (6)が成り立つことから, 定理 \ref{sthmset=}により
\[
  {\rm pr}_{1}\langle (T)_{x \in a} \rangle = a
\]
が成り立つ.

さて次に後者が成り立つことを示す.
$y$と$z$は上と同じとするとき, 上述のように$z$は$y$と異なり, $(T)_{x \in a}$の中に自由変数として現れないから, 
定理 \ref{sthmprsetelement}により
\[
\tag{7}
  y \in {\rm pr}_{2}\langle (T)_{x \in a} \rangle \leftrightarrow \exists z((z, y) \in (T)_{x \in a})
\]
が成り立つ.
また示したように(2)が成り立つから, 
これと$z$が定数でないことから, 推論法則 \ref{dedalleqquansepconst}により
\[
\tag{8}
  \exists z((z, y) \in (T)_{x \in a}) \leftrightarrow \exists z(z \in a \wedge y = (z|x)(T))
\]
が成り立つ.
また$z$が$y$と異なり, $a$の中に自由変数として現れないことから, 
定理 \ref{sthmosetbasis}と推論法則 \ref{dedeqch}により
\[
  \exists z(z \in a \wedge y = (z|x)(T)) \leftrightarrow y \in \{(z|x)(T)|z \in a\}
\]
が成り立つ.
ここで$z$が$a$及び$T$の中に自由変数として現れず, $x$が$a$の中に自由変数として現れないことから, 
代入法則 \ref{substosettrans}により$\{(z|x)(T)|z \in a\}$は$\{T|x \in a\}$と一致するから, 
上記の記号列は
\[
\tag{9}
  \exists z(z \in a \wedge y = (z|x)(T)) \leftrightarrow y \in \{T|x \in a\}
\]
と一致する.
故にこれが定理となる.
そこで(7), (8), (9)から, 推論法則 \ref{dedeqtrans}によって
\[
\tag{10}
  y \in {\rm pr}_{2}\langle (T)_{x \in a} \rangle \leftrightarrow y \in \{T|x \in a\}
\]
が成り立つことがわかる.
さて上述のように$y$は$(T)_{x \in a}$の中に自由変数として現れないから, 
変数法則 \ref{valprset}により, $y$は${\rm pr}_{2}\langle (T)_{x \in a} \rangle$の中に自由変数として現れない.
また$y$が$a$及び$T$の中に自由変数として現れないことから, 変数法則 \ref{valoset}により, 
$y$は$\{T|x \in a\}$の中にも自由変数として現れない.
また$y$は定数でない.
これらのことと, (10)が成り立つことから, 定理 \ref{sthmset=}により
\[
  {\rm pr}_{2}\langle (T)_{x \in a} \rangle = \{T|x \in a\}
\]
が成り立つ.
\halmos




\mathstrut
\begin{thm}
\label{sthmfuncfamily}%定理
$a$と$T$を集合とし, $x$を$a$の中に自由変数として現れない文字とする.
このとき$(T)_{x \in a}$は$a$から$\{T|x \in a\}$への函数である.
故に, $(T)_{x \in a}$は$a$における函数であり, また$(T)_{x \in a}$は函数である.
\end{thm}


\noindent{\bf 証明}
~$z$と$y$を, 互いに異なり, 共に$a$及び$T$の中に自由変数として現れない, 定数でない文字とする.
このとき変数法則 \ref{valfamily}により, $z$と$y$は共に$(T)_{x \in a}$の中に自由変数として現れない.
また$x$が$a$の中に自由変数として現れないことから, 定理 \ref{sthmpairinfamily}と推論法則 \ref{dedequiv}により
\[
  (z, y) \in (T)_{x \in a} \to z \in a \wedge y = (z|x)(T)
\]
が成り立つ.
故に推論法則 \ref{dedprewedge}により
\[
\tag{1}
  (z, y) \in (T)_{x \in a} \to y = (z|x)(T)
\]
が成り立つ.
いま$y$は$z$と異なり, $T$の中に自由変数として現れないから, 
変数法則 \ref{valsubst}により, $y$は$(z|x)(T)$の中に自由変数として現れない.
また$y$は定数でない.
これらのことと, (1)が成り立つことから, 推論法則 \ref{ded!thmconst}により
\[
  !y((z, y) \in (T)_{x \in a})
\]
が成り立つ.
いま$z$は定数でないので, これから推論法則 \ref{dedltthmquan}により
\[
\tag{2}
  \forall z(!y((z, y) \in (T)_{x \in a}))
\]
が成り立つ.
また$x$が$a$の中に自由変数として現れないことから, 
定理 \ref{sthmgraphfamily}より$(T)_{x \in a}$はグラフである.
そこでこのことと, (2)が成り立つことから, 推論法則 \ref{dedwedge}により
\[
  {\rm Graph}((T)_{x \in a}) \wedge \forall z(!y((z, y) \in (T)_{x \in a}))
\]
が成り立つ.
いま$z$と$y$は互いに異なり, 上述のように共に$(T)_{x \in a}$の中に自由変数として現れないから, 
定義によれば, 上記の記号列は
\[
\tag{3}
  {\rm Func}((T)_{x \in a})
\]
と一致する.
故にこれが定理となる.
即ち, $(T)_{x \in a}$は函数である.
また定理 \ref{sthmprsetfamily}より${\rm pr}_{1}\langle (T)_{x \in a} \rangle = a$が成り立つから, 
これと(3)から, 推論法則 \ref{dedwedge}により
\[
  {\rm Func}((T)_{x \in a}) \wedge {\rm pr}_{1}\langle (T)_{x \in a} \rangle = a, 
\]
即ち
\[
\tag{4}
  {\rm Func}((T)_{x \in a}; a)
\]
が成り立つ.
即ち, $(T)_{x \in a}$は$a$における函数である.
また定理 \ref{sthmprsetfamily}より${\rm pr}_{2}\langle (T)_{x \in a} \rangle = \{T|x \in a\}$が成り立つから, 
定理 \ref{sthm=tsubset}より${\rm pr}_{2}\langle (T)_{x \in a} \rangle \subset \{T|x \in a\}$が成り立つ.
そこでこれと(4)から, 推論法則 \ref{dedwedge}により
\[
  {\rm Func}((T)_{x \in a}; a) \wedge {\rm pr}_{2}\langle (T)_{x \in a} \rangle \subset \{T|x \in a\}, 
\]
即ち
\[
  {\rm Func}((T)_{x \in a}; a; \{T|x \in a\})
\]
が成り立つ.
即ち, $(T)_{x \in a}$は$a$から$\{T|x \in a\}$への函数である.
\halmos




\mathstrut
\begin{thm}
\label{sthmfuncvalfamily}%定理
$a$, $T$, $t$を集合とし, $x$を$a$の中に自由変数として現れない文字とする.
このとき
\[
  t \in a \to (T)_{x \in a}(t) = (t|x)(T)
\]
が成り立つ.
またこのことから, 次の($*$)が成り立つ: 

($*$) ~~$t \in a$が成り立つならば, $(T)_{x \in a}(t) = (t|x)(T)$が成り立つ.
\end{thm}


\noindent{\bf 証明}
~$x$が$a$の中に自由変数として現れないことから, 
定理 \ref{sthmpairinfamily2}と推論法則 \ref{dedequiv}により
\[
\tag{1}
  t \in a \to (t, (t|x)(T)) \in (T)_{x \in a}
\]
が成り立つ.
また$x$が$a$の中に自由変数として現れないことから, 
定理 \ref{sthmfuncfamily}より$(T)_{x \in a}$は函数である.
故に定理 \ref{sthmfuncvalbasispractical}により
\[
\tag{2}
  (t, (t|x)(T)) \in (T)_{x \in a} \to (t|x)(T) = (T)_{x \in a}(t)
\]
が成り立つ.
またThm \ref{x=yty=x}より
\[
\tag{3}
  (t|x)(T) = (T)_{x \in a}(t) \to (T)_{x \in a}(t) = (t|x)(T)
\]
が成り立つ.
そこで(1), (2), (3)から, 推論法則 \ref{dedmmp}によって
\[
  t \in a \to (T)_{x \in a}(t) = (t|x)(T)
\]
が成り立つことがわかる.
($*$)が成り立つことは, これと推論法則 \ref{dedmp}によって明らかである.
\halmos




\mathstrut
\begin{thm}
\label{sthmfamilysubset}%定理
\mbox{}

1)
$a$, $b$, $T$を集合とし, $x$を$a$及び$b$の中に自由変数として現れない文字とする.
このとき
\[
  a \subset b \to (T)_{x \in a} \subset (T)_{x \in b}
\]
が成り立つ.
またこのことから, 次の($*$)が成り立つ: 

($*$) ~~$a \subset b$が成り立つならば, $(T)_{x \in a} \subset (T)_{x \in b}$が成り立つ.

2)
$a$, $b$, $T$は1)と同じとする.
また$U$を集合とし, $x$, $y$を, それぞれ$a$, $b$の中に自由変数として現れない文字とする.
このとき
\[
  (T)_{x \in a} \subset (U)_{y \in b} \to a \subset b
\]
が成り立つ.
またこのことから, 次の($**$)が成り立つ: 

($**$) ~~$(T)_{x \in a} \subset (U)_{y \in b}$が成り立つならば, $a \subset b$が成り立つ.

3)
$a$, $b$, $T$, $x$は1)と同じとするとき, 
\[
  a \subset b \leftrightarrow (T)_{x \in a} \subset (T)_{x \in b}
\]
が成り立つ.
\end{thm}


\noindent{\bf 証明}
~1)
$z$と$y$を, 互いに異なり, 共に$a$, $b$, $T$のいずれの記号列の中にも自由変数として現れない, 
定数でない文字とする.

定理 \ref{sthmpairinfamily}と推論法則 \ref{dedequiv}
\[
  (z, y) \in (T)_{x \in a} \to z \in a \wedge y = (z|x)(T)
\]

推論法則 \ref{dedaddw}
\[
\tag{1}
  a \subset b \wedge (z, y) \in (T)_{x \in a} \to a \subset b \wedge (z \in a \wedge y = (z|x)(T))
\]

Thm \ref{aw1bwc1t1awb1wc}
\[
\tag{2}
  a \subset b \wedge (z \in a \wedge y = (z|x)(T)) \to (a \subset b \wedge z \in a) \wedge y = (z|x)(T)
\]

定理 \ref{sthmsubsetbasis}
\[
  a \subset b \to (z \in a \to z \in b)
\]

推論法則 \ref{dedtwch}
\[
  a \subset b \wedge z \in a \to z \in b
\]

推論法則 \ref{dedaddw}
\[
\tag{3}
  (a \subset b \wedge z \in a) \wedge y = (z|x)(T) \to z \in b \wedge y = (z|x)(T)
\]

定理 \ref{sthmpairinfamily}と推論法則 \ref{dedequiv}
\[
\tag{4}
  z \in b \wedge y = (z|x)(T) \to (z, y) \in (T)_{x \in b}
\]

(1)---(4)推論法則 \ref{dedmmp}
\[
  a \subset b \wedge (z, y) \in (T)_{x \in a} \to (z, y) \in (T)_{x \in b}
\]

推論法則 \ref{dedtwch}
\[
\tag{5}
  a \subset b \to ((z, y) \in (T)_{x \in a} \to (z, y) \in (T)_{x \in b})
\]

推論法則 \ref{dedalltquansepfreeconst}
\[
\tag{6}
  a \subset b \to \forall z(\forall y((z, y) \in (T)_{x \in a} \to (z, y) \in (T)_{x \in b}))
\]

定理 \ref{sthmgraphpairsubset}
\[
\tag{7}
  \forall z(\forall y((z, y) \in (T)_{x \in a} \to (z, y) \in (T)_{x \in b})) \to (T)_{x \in a} \subset (T)_{x \in b}
\]

(6), (7)推論法則 \ref{dedmmp}
\[
\tag{8}
  a \subset b \to (T)_{x \in a} \subset (T)_{x \in b}
\]





\noindent
2)

\[
\tag{9}
  (T)_{x \in a} \subset (U)_{y \in b} 
  \to {\rm pr}_{1}\langle (T)_{x \in a} \rangle \subset {\rm pr}_{1}\langle (U)_{y \in b} \rangle
\]

\[
  {\rm pr}_{1}\langle (T)_{x \in a} \rangle = a, ~~
  {\rm pr}_{1}\langle (U)_{y \in b} \rangle = b
\]

\begin{align*}
  {\rm pr}_{1}\langle (T)_{x \in a} \rangle \subset {\rm pr}_{1}\langle (U)_{y \in b} \rangle 
  &\leftrightarrow a \subset {\rm pr}_{1}\langle (U)_{y \in b} \rangle, \\
  \mbox{} \\
  a \subset {\rm pr}_{1}\langle (U)_{y \in b} \rangle 
  &\leftrightarrow a \subset b
\end{align*}

\begin{align*}
  \tag{10}
  {\rm pr}_{1}\langle (T)_{x \in a} \rangle \subset {\rm pr}_{1}\langle (U)_{y \in b} \rangle 
  &\to a \subset {\rm pr}_{1}\langle (U)_{y \in b} \rangle, \\
  \mbox{} \\
  \tag{11}
  a \subset {\rm pr}_{1}\langle (U)_{y \in b} \rangle 
  &\to a \subset b
\end{align*}

\[
  (T)_{x \in a} \subset (U)_{y \in b} \to a \subset b
\]



\noindent
3)

\[
\tag{12}
  (T)_{x \in a} \subset (T)_{x \in b} \to a \subset b
\]

\[
  a \subset b \leftrightarrow (T)_{x \in a} \subset (T)_{x \in b}
\]





\mathstrut
\begin{thm}
\label{sthmfamily=}%定理
\mbox{}

1)
$a$, $b$, $T$を集合とし, $x$を$a$及び$b$の中に自由変数として現れない文字とする.
このとき
\[
  a = b \to (T)_{x \in a} = (T)_{x \in b}
\]
が成り立つ.
またこのことから, 次の($*$)が成り立つ: 

($*$) ~~$a = b$が成り立つならば, $(T)_{x \in a} = (T)_{x \in b}$が成り立つ.

2)
$a$, $b$, $T$は1)と同じとする.
また$U$を集合とし, $x$, $y$を, それぞれ$a$, $b$の中に自由変数として現れない文字とする.
このとき
\[
  (T)_{x \in a} = (U)_{y \in b} \to a = b
\]
が成り立つ.
またこのことから, 次の($**$)が成り立つ: 

($**$) ~~$(T)_{x \in a} = (U)_{y \in b}$が成り立つならば, $a = b$が成り立つ.

3)
$a$, $b$, $T$, $x$は1)と同じとするとき, 
\[
  a = b \leftrightarrow (T)_{x \in a} = (T)_{x \in b}
\]
が成り立つ.
\end{thm}


\noindent{\bf 証明}
~1)



\noindent
2)




\noindent
3)







\mathstrut
\begin{thm}
\label{sthmfamily=2}%定理

\[
  (T)_{x \in a} = (U)_{x \in a} \leftrightarrow \forall x(x \in a \to T = U)
\]

\[
  (T)_{x \in a} = (U)_{x \in b} \leftrightarrow a = b \wedge \forall x(x \in a \to T = U)
\]


\end{thm}


\noindent{\bf 証明}
~










\mathstrut
\begin{thm}
\label{sthmfuncvalfamily2}%定理%改築
$a$と$f$を集合とし, $x$をこれらの中に自由変数として現れない文字とする.
このとき
\[
  {\rm Func}(f; a) \leftrightarrow f = (f(x))_{x \in a}
\]
が成り立つ.
またこのことから, 次の($*$)が成り立つ: 

($*$) ~~$f$が$a$における函数ならば, $f = (f(x))_{x \in a}$が成り立つ.
        逆に$f = (f(x))_{x \in a}$が成り立つならば, $f$は$a$における函数である.
\end{thm}


\noindent{\bf 証明}
~















\mathstrut
\begin{thm}
\label{sthmuopairfamily}%定理
$a$, $b$, $T$を集合とし, $x$を$a$と$b$の中に自由変数として現れない文字とする.
このとき
\[
  (T)_{x \in \{a, b\}} = \{(a, (a|x)(T)), (b, (b|x)(T))\}
\]
が成り立つ.
特に, 
\[
  (T)_{x \in \{a\}} = \{(a, (a|x)(T))\}
\]
が成り立つ.
\end{thm}


\noindent{\bf 証明}
~定理 \ref{sthmuopairoset}より
\[
  \{(x, T)|x \in \{a, b\}\} = \{(a|x)((x, T)), (b|x)((x, T))\}, 
\]
即ち
\[
  (T)_{x \in \{a, b\}} = \{(a|x)((x, T)), (b|x)((x, T))\}
\]
が成り立つ.
ここで代入法則 \ref{substpair}により, $(a|x)((x, T))$, $(b|x)((x, T))$はそれぞれ
$(a, (a|x)(T))$, $(b, (b|x)(T))$と一致するから, 上記の記号列は
\[
  (T)_{x \in \{a, b\}} = \{(a, (a|x)(T)), (b, (b|x)(T))\}
\]
と一致する.
故にこれが定理となる.
特に$b$を$a$に置き換えて考えれば, 
\[
  (T)_{x \in \{a, a\}} = \{(a, (a|x)(T)), (a, (a|x)(T))\}, 
\]
即ち
\[
  (T)_{x \in \{a\}} = \{(a, (a|x)(T))\}
\]
も成り立つことがわかる.
\halmos




\mathstrut
\begin{thm}
\label{sthmcupfamily}%定理
$a$, $b$, $T$を集合とし, $x$を$a$と$b$の中に自由変数として現れない文字とする.
このとき
\[
  (T)_{x \in a} \cup (T)_{x \in b} = (T)_{x \in a \cup b}
\]
が成り立つ.
\end{thm}


\noindent{\bf 証明}
~定理 \ref{sthmosetcup}より
\[
  \{(x, T)|x \in a \cup b\} = \{(x, T)|x \in a\} \cup \{(x, T)|x \in b\}, 
\]
即ち
\[
  (T)_{x \in a \cup b} = (T)_{x \in a} \cup (T)_{x \in b}
\]
が成り立つ.
故に推論法則 \ref{ded=ch}により
\[
  (T)_{x \in a} \cup (T)_{x \in b} = (T)_{x \in a \cup b}
\]
が成り立つ.
\halmos




\mathstrut
\begin{thm}
\label{sthmcapfamily}%定理
$a$, $b$, $T$を集合とし, $x$を$a$と$b$の中に自由変数として現れない文字とする.
このとき
\[
  (T)_{x \in a} \cap (T)_{x \in b} = (T)_{x \in a \cap b}
\]
が成り立つ.
\end{thm}


\noindent{\bf 証明}
~$z$と$y$を, 互いに異なり, 共に$a$, $b$, $T$のいずれの記号列の中にも自由変数として現れない, 
定数でない文字とする.
このとき変数法則 \ref{valcap}, \ref{valfamily}により, 
$z$と$y$は共に$(T)_{x \in a} \cap (T)_{x \in b}$及び$(T)_{x \in a \cap b}$の中に自由変数として現れない.
また定理 \ref{sthmcapelement}より
\[
\tag{1}
  (z, y) \in (T)_{x \in a} \cap (T)_{x \in b} \leftrightarrow (z, y) \in (T)_{x \in a} \wedge (z, y) \in (T)_{x \in b}
\]
が成り立つ.
また$x$が$a$及び$b$の中に自由変数として現れないことから, 定理 \ref{sthmpairinfamily}より
\[
  (z, y) \in (T)_{x \in a} \leftrightarrow z \in a \wedge y = (z|x)(T), ~~
  (z, y) \in (T)_{x \in b} \leftrightarrow z \in b \wedge y = (z|x)(T)
\]
が共に成り立つ.
故に推論法則 \ref{dedaddeqw}により
\[
\tag{2}
  (z, y) \in (T)_{x \in a} \wedge (z, y) \in (T)_{x \in b} \leftrightarrow (z \in a \wedge y = (z|x)(T)) \wedge (z \in b \wedge y = (z|x)(T))
\]
が成り立つ.
またThm \ref{aw1bwc1l1awb1w1awc1}と推論法則 \ref{dedeqch}により
\[
\tag{3}
  (z \in a \wedge y = (z|x)(T)) \wedge (z \in b \wedge y = (z|x)(T)) \leftrightarrow (z \in a \wedge z \in b) \wedge y = (z|x)(T)
\]
が成り立つ.
また定理 \ref{sthmcapelement}と推論法則 \ref{dedeqch}により
\[
  z \in a \wedge z \in b \leftrightarrow z \in a \cap b
\]
が成り立つから, 推論法則 \ref{dedaddeqw}により
\[
\tag{4}
  (z \in a \wedge z \in b) \wedge y = (z|x)(T) \leftrightarrow z \in a \cap b \wedge y = (z|x)(T)
\]
が成り立つ.
また$x$が$a$及び$b$の中に自由変数として現れないことから, 
変数法則 \ref{valcap}により, $x$は$a \cap b$の中に自由変数として現れないから, 
定理 \ref{sthmpairinfamily}と推論法則 \ref{dedeqch}により
\[
\tag{5}
  z \in a \cap b \wedge y = (z|x)(T) \leftrightarrow (z, y) \in (T)_{x \in a \cap b}
\]
が成り立つ.
そこで(1)---(5)から, 推論法則 \ref{dedeqtrans}によって
\[
\tag{6}
  (z, y) \in (T)_{x \in a} \cap (T)_{x \in b} \leftrightarrow (z, y) \in (T)_{x \in a \cap b}
\]
が成り立つことがわかる.
さていま定理 \ref{sthmgraphfamily}より, 
$(T)_{x \in a}$, $(T)_{x \in b}$, $(T)_{x \in a \cap b}$はいずれもグラフである.
故に定理 \ref{sthmcapgraph}により, $(T)_{x \in a} \cap (T)_{x \in b}$もグラフである.
また$z$と$y$は互いに異なり, 共に定数でなく, 上述のように共に
$(T)_{x \in a} \cap (T)_{x \in b}$及び$(T)_{x \in a \cap b}$の中に自由変数として現れない.
これらのことと, (6)が成り立つことから, 定理 \ref{sthmgraphpair=}により
\[
  (T)_{x \in a} \cap (T)_{x \in b} = (T)_{x \in a \cap b}
\]
が成り立つ.
\halmos




\mathstrut
\begin{thm}
\label{sthm-family}%定理
$a$, $b$, $T$を集合とし, $x$を$a$と$b$の中に自由変数として現れない文字とする.
このとき
\[
  (T)_{x \in a} - (T)_{x \in b} = (T)_{x \in a - b}
\]
が成り立つ.
\end{thm}


\noindent{\bf 証明}
~












\mathstrut
\begin{thm}
\label{sthmemptyfamily}%定理
$a$と$T$を集合とし, $x$を$a$の中に自由変数として現れない文字とする.
このとき
\[
  (T)_{x \in a} = \phi \leftrightarrow a = \phi
\]
が成り立つ.
またこのことから, 次の($*$), ($**$)が成り立つ: 

($*$) ~~$(T)_{x \in a}$が空ならば, $a$は空である.
        逆に$a$が空ならば, $(T)_{x \in a}$は空である.

($**$) ~~$(T)_{x \in \phi}$は空である.
\end{thm}


\noindent{\bf 証明}
~




























\newpage
\setcounter{defi}{0}
\section{単射, 全射, 全単射}%%%%%%%%%%%%%%%%%%%%%%%%%%%%%%%%%%%%%%%%%%%%%%%%%%%%%%%%%%%%%%%%%%%%%%%%%%%%%%%










\newpage
\setcounter{defi}{0}
\section{一般の和集合と共通部分}%%%%%%%%%%%%%%%%%%%%%%%%%%%%%%%%%%%%%%%%%%%%%%%%%%%%%%%%%%%%%%%%%%%%%%%%%%%




\mathstrut
\begin{thm}
\label{sthmbigcup}%定理
$a$を集合とする..
また$x$と$y$を, 互いに異なり, 共に$a$の中に自由変数として現れない文字とする.
このとき, 関係式$\exists y(y \in a \wedge x \in y)$は$x$について集合を作り得る.
\end{thm}


\noindent{\bf 証明}
~$u$と$v$を, 共に$x$とも$y$とも異なり, $a$の中に自由変数として現れない文字とする.
また$z$を, $x$, $y$, $u$のいずれとも異なる, 定数でない文字とする.
このときThm \ref{allx1rtr1}より, 
\[
  \forall x(x \in z \to x \in z)
\]
が成り立つ.
いま$u$は$x$とも$z$とも異なるから, この記号列は
\[
  \forall x((z|u)(x \in z \to x \in u))
\]
と一致する.
また$x$が$z$とも$u$とも異なることから, 代入法則 \ref{substquan}により, 
この記号列は
\[
  (z|u)(\forall x(x \in z \to x \in u))
\]
と一致する.
よってこれが定理となる.
そこで推論法則 \ref{deds4}により, 
\[
  \exists u(\forall x(x \in z \to x \in u))
\]
が成り立ち, 
$z$が定数でないことから, この定理に推論法則 \ref{dedltthmquan}を適用して, 
\[
  \forall z(\exists u(\forall x(x \in z \to x \in u)))
\]
が成り立つ.
ここで$y$が$x$, $z$, $u$のいずれとも異なることから, 変数法則 \ref{valquan}によって
$y$が$\exists u(\forall x(x \in z \to x \in u))$の中に自由変数として現れないことがわかる.
そこで代入法則 \ref{substquantrans}により, 上記の記号列は
\[
  \forall y((y|z)(\exists u(\forall x(x \in z \to x \in u))))
\]
と一致する.
また$u$と$x$が共に$z$とも$y$とも異なることから, 代入法則 \ref{substquan}により, 
この記号列は
\[
  \forall y(\exists u(\forall x((y|z)(x \in z \to x \in u))))
\]
と一致し, これは
\[
\tag{1}
  \forall y(\exists u(\forall x(x \in y \to x \in u)))
\]
と一致する.
よってこれが定理となる.
いま$x$と$y$は互いに異なる文字である.
また$u$と$v$は共に$x$とも$y$とも異なる文字で, 
従って関係式$x \in y$の中に自由変数として現れない.
そこでschema S7の適用により, 
\[
\tag{2}
  \forall y(\exists u(\forall x(x \in y \to x \in u))) \to \forall v({\rm Set}_{x}(\exists y(y \in v \wedge x \in y)))
\]
が成り立つ.
そこで(1), (2)から, 推論法則 \ref{dedmp}によって
\[
  \forall v({\rm Set}_{x}(\exists y(y \in v \wedge x \in y)))
\]
が成り立つ.
そこで推論法則 \ref{dedfromallthm}により, 
\[
  (a|v)({\rm Set}_{x}(\exists y(y \in v \wedge x \in y)))
\]
が成り立つ.
ここで$x$が$v$と異なり, $a$の中に自由変数として現れないことから, 
代入法則 \ref{substsm}により, 上記の記号列は
\[
  {\rm Set}_{x}((a|v)(\exists y(y \in v \wedge x \in y)))
\]
と一致する.
また$y$も$v$と異なり, $a$の中に自由変数として現れないから, 
代入法則 \ref{substquan}により, この記号列は
\[
  {\rm Set}_{x}(\exists y((a|v)(y \in v \wedge x \in y)))
\]
と一致する.
また$v$が$x$とも$y$とも異なることと代入法則 \ref{substwedge}により, 
この記号列は
\[
  {\rm Set}_{x}(\exists y(y \in a \wedge x \in y))
\]
と一致する.
よってこれが定理となる.
言い換えれば, 関係式$\exists y(y \in a \wedge x \in y)$は$x$について集合を作り得る.
\halmos




\mathstrut
\begin{defo}
\label{bigcup}%変形
$a$を記号列とする.
また$x$と$y$を, 互いに異なり, 共に$a$の中に自由変数として現れない文字とする.
同様に, $z$と$w$を, 互いに異なり, 共に$a$の中に自由変数として現れない文字とする.
このとき
\[
  \{x|\exists y(y \in a \wedge x \in y)\} \equiv \{z|\exists w(w \in a \wedge z \in w)\}
\]
が成り立つ.
\end{defo}


\noindent{\bf 証明}
~$u$と$v$を, 互いに異なり, 共に$x$, $y$, $z$, $w$のいずれとも異なり, 
$a$の中に自由変数として現れない文字とする.
このとき変数法則 \ref{valfund}, \ref{valwedge}, \ref{valquan}からわかるように, 
$u$は$\exists y(y \in a \wedge x \in y)$の中に自由変数として現れないから, 
代入法則 \ref{substisettrans}により, 
\[
\tag{1}
  \{x|\exists y(y \in a \wedge x \in y)\} \equiv \{u|(u|x)(\exists y(y \in a \wedge x \in y))\}
\]
が成り立つ.
また$y$が$x$とも$u$とも異なることから, 代入法則 \ref{substquan}により, 
\[
\tag{2}
  (u|x)(\exists y(y \in a \wedge x \in y)) \equiv \exists y((u|x)(y \in a \wedge x \in y))
\]
が成り立つ.
また$x$が$y$と異なり, $a$の中に自由変数として現れないことから, 
代入法則 \ref{substfree}, \ref{substfund}, \ref{substwedge}により, 
\[
\tag{3}
  (u|x)(y \in a \wedge x \in y) \equiv y \in a \wedge u \in y
\]
が成り立つ.
そこで(1), (2), (3)から, 
\[
\tag{4}
  \{x|\exists y(y \in a \wedge x \in y)\} \equiv \{u|\exists y(y \in a \wedge u \in y)\}
\]
が成り立つことがわかる.
また$v$が$y$とも$u$とも異なり, $a$の中に自由変数として現れないことから, 
変数法則 \ref{valfund}, \ref{valwedge}によって$v$が$y \in a \wedge u \in y$の中に
自由変数として現れないことがわかるから, 
代入法則 \ref{substquantrans}により, 
\[
\tag{5}
  \exists y(y \in a \wedge u \in y) \equiv \exists v((v|y)(y \in a \wedge u \in y))
\]
が成り立つ.
また$y$が$u$と異なり, $a$の中に自由変数として現れないことから, 
代入法則 \ref{substfree}, \ref{substfund}, \ref{substwedge}により, 
\[
\tag{6}
  (v|y)(y \in a \wedge u \in y) \equiv v \in a \wedge u \in v
\]
が成り立つ.
そこで(5), (6)から, 
\[
\tag{7}
  \{u|\exists y(y \in a \wedge u \in y)\} \equiv \{u|\exists v(v \in a \wedge u \in v)\}
\]
が成り立つことがわかる.
そこで(4), (7)によって
\[
  \{x|\exists y(y \in a \wedge x \in y)\} \equiv \{u|\exists v(v \in a \wedge u \in v)\}
\]
が成り立つ.
またここまでの議論と全く同様にして, 
\[
  \{z|\exists w(w \in a \wedge z \in w)\} \equiv \{u|\exists v(v \in a \wedge u \in v)\}
\]
も成り立つ.
故に$\{x|\exists y(y \in a \wedge x \in y)\}$と$\{z|\exists w(w \in a \wedge z \in w)\}$という
二つの記号列は一致する.
\halmos




\mathstrut
\begin{defi}
\label{defbigcup}%定義
$a$を記号列とする.
また$x$と$y$を, 互いに異なり, 共に$a$の中に自由変数として現れない文字とする.
同様に, $z$と$w$を, 互いに異なり, 共に$a$の中に自由変数として現れない文字とする.
このとき上記の変形法則 \ref{bigcup}によれば, 
$\{x|\exists y(y \in a \wedge x \in y)\}$と$\{z|\exists w(w \in a \wedge z \in w)\}$という
二つの記号列は一致する.
$a$に対して定まるこの記号列を, $\bigcup (a)$と書き表す(括弧は適宜省略する).
\end{defi}




\mathstrut
\begin{valu}
\label{valbigcup}%変数
$a$を記号列とし, $x$を文字とする.
$x$が$a$の中に自由変数として現れなければ, 
$x$は$\bigcup a$の中に自由変数として現れない.
\end{valu}


\noindent{\bf 証明}
~このとき, $y$を$x$と異なり, $a$の中に自由変数として現れない文字とすれば, 
定義から$\bigcup a$は$\{x|\exists y(y \in a \wedge x \in y)\}$と同じである.
変数法則 \ref{valiset}により, $x$はこの中に自由変数として現れない.
\halmos




\mathstrut
\begin{subs}
\label{substbigcup}%代入
$a$と$b$を記号列とし, $x$を文字とするとき, 
\[
  (b|x)(\bigcup a) \equiv \bigcup (b|x)(a)
\]
が成り立つ.
\end{subs}


\noindent{\bf 証明}
~$y$と$z$を, 互いに異なり, 共に$x$と異なり, $a$及び$b$の中に自由変数として現れない文字とする.
このとき定義から$\bigcup a$は$\{y|\exists z(z \in a \wedge y \in z)\}$と同じだから, 
\[
\tag{1}
  (b|x)(\bigcup a) \equiv (b|x)(\{y|\exists z(z \in a \wedge y \in z)\})
\]
である.
また$y$が$x$と異なり, $b$の中に自由変数として現れないことから, 
代入法則 \ref{substiset}により, 
\[
\tag{2}
  (b|x)(\{y|\exists z(z \in a \wedge y \in z)\}) \equiv \{y|(b|x)(\exists z(z \in a \wedge y \in z))\}
\]
が成り立つ.
また$z$も$x$と異なり, $b$の中に自由変数として現れないから, 
代入法則 \ref{substquan}により, 
\[
\tag{3}
  (b|x)(\exists z(z \in a \wedge y \in z)) \equiv \exists z((b|x)(z \in a \wedge y \in z))
\]
が成り立つ.
また$x$が$y$とも$z$とも異なることと代入法則 \ref{substfund}, \ref{substwedge}により, 
\[
\tag{4}
  (b|x)(z \in a \wedge y \in z) \equiv z \in (b|x)(a) \wedge y \in z
\]
が成り立つ.
そこで(1)---(4)から, $(b|x)(\bigcup a)$が$\{y|\exists z(z \in (b|x)(a) \wedge y \in z)\}$と
一致することがわかる.
いま$y$と$z$は共に$a$及び$b$の中に自由変数として現れないから, 
変数法則 \ref{valsubst}により, これらは共に$(b|x)(a)$の中にも自由変数として現れない.
また$y$と$z$は異なる文字である.
そこで定義から, 記号列$\{y|\exists z(z \in (b|x)(a) \wedge y \in z)\}$は
$\bigcup (b|x)(a)$と同じである.
故に$(b|x)(\bigcup a)$と$\bigcup (b|x)(a)$は一致する.
\halmos




\mathstrut
\begin{form}
\label{formbigcup}%構成
$a$が集合ならば, $\bigcup a$は集合である.
\end{form}


\noindent{\bf 証明}
~$x$と$y$を, 互いに異なり, 共に$a$の中に自由変数として現れない文字とすれば, 
定義から$\bigcup a$は$\{x|\exists y(y \in a \wedge x \in y)\}$と同じである.
$a$が集合ならば, 構成法則 \ref{formfund}, \ref{formwedge}, \ref{formquan}, \ref{formiset}によって
直ちにわかるように, これは集合となる.
\halmos




\mathstrut
\begin{thm}
\label{sthmbigcupelement}%定理
$a$と$b$を集合とし, $y$をこれらの中に自由変数として現れない文字とするとき, 
\[
  b \in \bigcup a \leftrightarrow \exists y(y \in a \wedge b \in y)
\]
が成り立つ.
\end{thm}


\noindent{\bf 証明}
~$x$を$y$と異なり, $a$の中に自由変数として現れない文字とするとき, 
定義から$\bigcup a$は$\{x|\exists y(y \in a \wedge x \in y)\}$と同じである.
また$x$と$y$に対する仮定から, 定理 \ref{sthmbigcup}により
$\exists y(y \in a \wedge x \in y)$は$x$について集合を作り得る.
よって定理 \ref{sthmisetbasis}により, 
\[
  b \in \bigcup a \leftrightarrow (b|x)(\exists y(y \in a \wedge x \in y))
\]
が成り立つ.
ここで$y$が$x$と異なり, $b$の中に自由変数として現れないことから, 
代入法則 \ref{substquan}により, この記号列は
\[
  b \in \bigcup a \leftrightarrow \exists y((b|x)(y \in a \wedge x \in y))
\]
と一致する.
また$x$が$y$と異なり, $a$の中に自由変数として現れないことから, 
代入法則 \ref{substfree}, \ref{substfund}, \ref{substwedge}により, この記号列は
\[
  b \in \bigcup a \leftrightarrow \exists y(y \in a \wedge b \in y)
\]
と一致する.
よってこれが定理となる.
\halmos




\mathstrut
\begin{thm}
\label{sthmbigcup=acupb}%定理
$a$と$b$を集合とするとき, 
\[
  \bigcup \{a, b\} = a \cup b
\]
が成り立つ.
\end{thm}


\noindent{\bf 証明}
~






\mathstrut
\begin{thm}
\label{sthmbigcap}%定理
$a$を集合とする..
また$x$と$y$を, 互いに異なり, 共に$a$の中に自由変数として現れない文字とする.
このとき
\[
  a \neq \phi \to {\rm Set}_{x}(\forall y(y \in a \to x \in y))
\]
が成り立つ.
このことから, 次の($*$)が成り立つ: 

($*$) ~~$a \neq \phi$が成り立つとき, 関係式$\forall y(y \in a \to x \in y)$は$x$について集合を作り得る.
\end{thm}


\noindent{\bf 証明}
~Thm \ref{thmeaquandm}と推論法則 \ref{dedequiv}により, 
\[
  \neg \exists x(x \in a) \to \forall x(x \notin a)
\]
が成り立つ.
また$x$が$a$の中に自由変数として現れないことから, 
定理 \ref{sthmnotinempty}と推論法則 \ref{dedequiv}により
\[
  \forall x(x \notin a) \to a = \phi
\]
が成り立つ.
よってこれらから, 推論法則 \ref{dedmmp}により
\[
  \neg \exists x(x \in a) \to a = \phi
\]
が成り立ち, これから推論法則 \ref{dedcp}によって
\[
  a \neq \phi \to \exists x(x \in a)
\]
が成り立つ.
ここで$\tau_{x}(x \in a)$を$T$と書けば, $T$は集合であり, 
変数法則 \ref{valtau}によってわかるように, $x$はこの中に自由変数として現れない.
また定義から, 上記の記号列は
\[
  a \neq \phi \to (T|x)(x \in a)
\]
と同じであり, $x$が$a$の中に自由変数として現れないことから, 
代入法則 \ref{substfree}, \ref{substfund}により, この記号列は
\[
\tag{1}
  a \neq \phi \to T \in a
\]
と一致する.
よってこれが定理となる.
さていま
\[
  U \equiv \{x \in T|\forall y(y \in a \to x \in y)\}
\]
と置くと, $U$は集合であり, 変数法則 \ref{valquan}, \ref{valsset}からわかるように, 
$y$は$U$の中に自由変数として現れない.
またいま
\[
  V \equiv \tau_{x}(\neg (x \in U \leftrightarrow \forall y(y \in a \to x \in y)))
\]
と置けば, $V$も集合である.
そして$y$は$x$と異なり, 上述のように$U$の中に自由変数として現れず, 
変数法則 \ref{valquan}により$\forall y(y \in a \to x \in y)$の中にも自由変数として現れないから, 
変数法則 \ref{valfund}, \ref{valtau}, \ref{valequiv}により, 
$y$が$V$の中に自由変数として現れないことがわかる.
さて上述のように$x$は$T$の中に自由変数として現れないから, 
$U$の定義から, 定理 \ref{sthmssetbasis}より
\[
  V \in U \leftrightarrow V \in T \wedge (V|x)(\forall y(y \in a \to x \in y))
\]
が成り立つ.
いま$y$は$x$と異なり, 上述のように$V$の中に自由変数として現れないから, 
代入法則 \ref{substquan}により, 上記の記号列は
\[
  V \in U \leftrightarrow V \in T \wedge \forall y((V|x)(y \in a \to x \in y))
\]
と一致する.
また$x$が$y$と異なり, $a$の中に自由変数として現れないことから, 
代入法則 \ref{substfree}, \ref{substfund}により, この記号列は
\[
  V \in U \leftrightarrow V \in T \wedge \forall y(y \in a \to V \in y)
\]
と一致する.
よってこれが定理となる.









\mathstrut
\begin{defo}
\label{bigcap}%変形
$a$を記号列とする.
また$x$と$y$を, 互いに異なり, 共に$a$の中に自由変数として現れない文字とする.
同様に, $z$と$w$を, 互いに異なり, 共に$a$の中に自由変数として現れない文字とする.
このとき
\[
  \{x|\forall y(y \in a \to x \in y)\} \equiv \{z|\forall w(w \in a \to z \in w)\}
\]
が成り立つ.
\end{defo}


\noindent{\bf 証明}
~$u$と$v$を, 互いに異なり, 共に$x$, $y$, $z$, $w$のいずれとも異なり, 
$a$の中に自由変数として現れない文字とする.
このとき変数法則 \ref{valfund}, \ref{valquan}からわかるように, 
$u$は$\forall y(y \in a \to x \in y)$の中に自由変数として現れないから, 
代入法則 \ref{substisettrans}により, 
\[
\tag{1}
  \{x|\forall y(y \in a \to x \in y)\} \equiv \{u|(u|x)(\forall y(y \in a \to x \in y))\}
\]
が成り立つ.
また$y$が$x$とも$u$とも異なることから, 代入法則 \ref{substquan}により, 
\[
\tag{2}
  (u|x)(\forall y(y \in a \to x \in y)) \equiv \forall y((u|x)(y \in a \to x \in y))
\]
が成り立つ.
また$x$が$y$と異なり, $a$の中に自由変数として現れないことから, 
代入法則 \ref{substfree}, \ref{substfund}により, 
\[
\tag{3}
  (u|x)(y \in a \to x \in y) \equiv y \in a \to u \in y
\]
が成り立つ.
そこで(1), (2), (3)から, 
\[
\tag{4}
  \{x|\forall y(y \in a \to x \in y)\} \equiv \{u|\forall y(y \in a \to u \in y)\}
\]
が成り立つことがわかる.
また$v$が$y$とも$u$とも異なり, $a$の中に自由変数として現れないことから, 
変数法則 \ref{valfund}によって$v$が$y \in a \to u \in y$の中に
自由変数として現れないことがわかるから, 
代入法則 \ref{substquantrans}により, 
\[
\tag{5}
  \forall y(y \in a \to u \in y) \equiv \forall v((v|y)(y \in a \to u \in y))
\]
が成り立つ.
また$y$が$u$と異なり, $a$の中に自由変数として現れないことから, 
代入法則 \ref{substfree}, \ref{substfund}により, 
\[
\tag{6}
  (v|y)(y \in a \to u \in y) \equiv v \in a \to u \in v
\]
が成り立つ.
そこで(5), (6)から, 
\[
\tag{7}
  \{u|\forall y(y \in a \to u \in y)\} \equiv \{u|\forall v(v \in a \to u \in v)\}
\]
が成り立つことがわかる.
そこで(4), (7)によって
\[
  \{x|\forall y(y \in a \to x \in y)\} \equiv \{u|\forall v(v \in a \to u \in v)\}
\]
が成り立つ.
またここまでの議論と全く同様にして, 
\[
  \{z|\forall w(w \in a \to z \in w)\} \equiv \{u|\forall v(v \in a \to u \in v)\}
\]
も成り立つ.
故に$\{x|\forall y(y \in a \to x \in y)\}$と$\{z|\forall w(w \in a \to z \in w)\}$という
二つの記号列は一致する.
\halmos




\mathstrut
\begin{defi}
\label{defbigcap}%定義
$a$を記号列とする.
また$x$と$y$を, 互いに異なり, 共に$a$の中に自由変数として現れない文字とする.
同様に, $z$と$w$を, 互いに異なり, 共に$a$の中に自由変数として現れない文字とする.
このとき上記の変形法則 \ref{bigcap}によれば, 
$\{x|\forall y(y \in a \to x \in y)\}$と$\{z|\forall w(w \in a \to z \in w)\}$という
二つの記号列は一致する.
$a$に対して定まるこの記号列を, $\bigcap (a)$と書き表す(括弧は適宜省略する).
\end{defi}




\mathstrut
\begin{valu}
\label{valbigcap}%変数
$a$を記号列とし, $x$を文字とする.
$x$が$a$の中に自由変数として現れなければ, 
$x$は$\bigcap a$の中に自由変数として現れない.
\end{valu}


\noindent{\bf 証明}
~このとき, $y$を$x$と異なり, $a$の中に自由変数として現れない文字とすれば, 
定義から$\bigcap a$は$\{x|\forall y(y \in a \to x \in y)\}$と同じである.
変数法則 \ref{valiset}により, $x$はこの中に自由変数として現れない.
\halmos




\mathstrut
\begin{subs}
\label{substbigcap}%代入
$a$と$b$を記号列とし, $x$を文字とするとき, 
\[
  (b|x)(\bigcap a) \equiv \bigcap (b|x)(a)
\]
が成り立つ.
\end{subs}


\noindent{\bf 証明}
~$y$と$z$を, 互いに異なり, 共に$x$と異なり, $a$及び$b$の中に自由変数として現れない文字とする.
このとき定義から$\bigcap a$は$\{y|\forall z(z \in a \to y \in z)\}$と同じだから, 
\[
\tag{1}
  (b|x)(\bigcap a) \equiv (b|x)(\{y|\forall z(z \in a \to y \in z)\})
\]
である.
また$y$が$x$と異なり, $b$の中に自由変数として現れないことから, 
代入法則 \ref{substiset}により, 
\[
\tag{2}
  (b|x)(\{y|\forall z(z \in a \to y \in z)\}) \equiv \{y|(b|x)(\forall z(z \in a \to y \in z))\}
\]
が成り立つ.
また$z$も$x$と異なり, $b$の中に自由変数として現れないから, 
代入法則 \ref{substquan}により, 
\[
\tag{3}
  (b|x)(\forall z(z \in a \to y \in z)) \equiv \forall z((b|x)(z \in a \to y \in z))
\]
が成り立つ.
また$x$が$y$とも$z$とも異なることと代入法則 \ref{substfund}により, 
\[
\tag{4}
  (b|x)(z \in a \to y \in z) \equiv z \in (b|x)(a) \to y \in z
\]
が成り立つ.
そこで(1)---(4)から, $(b|x)(\bigcap a)$が$\{y|\forall z(z \in (b|x)(a) \to y \in z)\}$と
一致することがわかる.
いま$y$と$z$は共に$a$及び$b$の中に自由変数として現れないから, 
変数法則 \ref{valsubst}により, これらは共に$(b|x)(a)$の中にも自由変数として現れない.
また$y$と$z$は異なる文字である.
そこで定義から, 記号列$\{y|\forall z(z \in (b|x)(a) \to y \in z)\}$は
$\bigcap (b|x)(a)$と同じである.
故に$(b|x)(\bigcap a)$と$\bigcap (b|x)(a)$は一致する.
\halmos




\mathstrut
\begin{form}
\label{formbigcap}%構成
$a$が集合ならば, $\bigcap a$は集合である.
\end{form}


\noindent{\bf 証明}
~$x$と$y$を, 互いに異なり, 共に$a$の中に自由変数として現れない文字とすれば, 
定義から$\bigcap a$は$\{x|\forall y(y \in a \to x \in y)\}$と同じである.
$a$が集合ならば, 構成法則 \ref{formfund}, \ref{formquan}, \ref{formiset}によって
直ちにわかるように, これは集合となる.
\halmos




\mathstrut
\begin{thm}
\label{sthmbigcapelement}%定理
$a$と$b$を集合とし, $y$をこれらの中に自由変数として現れない文字とする.
また$a \neq \phi$が成り立つとする.
このとき
\[
  b \in \bigcap a \leftrightarrow \forall y(y \in a \to b \in y)
\]
が成り立つ.
\end{thm}


\noindent{\bf 証明}
~







\mathstrut
\begin{thm}
\label{sthmbigcap=acapb}%定理
$a$と$b$を集合とするとき, 
\[
  \bigcap \{a, b\} = a \cap b
\]
が成り立つ.
\end{thm}


\noindent{\bf 証明}
~



























































\end{document}